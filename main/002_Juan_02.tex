%% -*- coding: utf-8 -*-
%% Time-stamp: <Chen Wang: 2024-04-02 11:42:41>

% {\noindent \zhu \zihao{5} \fzbyks } -> 注 (△ ○)
% {\noindent \shu \zihao{5} \fzkt } -> 疏

\part{虞書}

\chapter{卷二}

\section{堯典第一(堯典上)}

 {\noindent\zhu\zihao{6}\fzbyks \CJKunderwave{釋文}:“凡十六篇,十一篇亡,五篇見存。” \par}

 {\noindent\shu\zihao{5}\fzkt “古文尚書堯典第一”。正義曰:檢古本並石經,直言“\CJKunderwave{堯典}第一”,無“古文尚書”。以\CJKunderline{孔君}從隸古,仍號“古文”,故後人因而題於此,以別伏生所出、大小夏侯及歐陽所傳為今文故也。“堯典第一”,篇之名,當與眾篇相次。“第”訓為次也,於次第之內而處一,故曰“堯典第一”。以此第一者,以五帝之末接三王之初,典策既備,因機成務,交代揖讓,以垂無為,故為第一也。然\CJKunderwave{書}者理由舜史,勒成一家,可以為法,上取堯事,下終禪\CJKunderline{禹},以至舜終,皆為舜史所錄。其堯、舜之典,多陳行事之狀,其言寡矣,\CJKunderwave{禹貢}即全非君言,準之後代,不應入\CJKunderwave{書},此其一體之異,以此\CJKunderline{禹}之身事於禪後,無入\CJKunderwave{夏書}之理。自\CJKunderwave{甘誓}已下,皆多言辭,則古史所書於是乎始。知\CJKunderwave{五子之歌}亦非上言,典書草創,以義而錄,但致言有本,名隨其事。檢其此體,為例有十。一曰典,二曰謨,三曰貢,四曰歌,五曰誓,六曰誥,七曰訓,八曰命,九曰徵,十曰範。\CJKunderwave{堯典}、\CJKunderwave{舜典}二篇,典也。\CJKunderwave{大禹謨}、\CJKunderwave{皋陶謨}二篇,謨也。\CJKunderwave{禹貢}一篇,貢也。\CJKunderwave{五子之歌}一篇,歌也。\CJKunderwave{甘誓}、\CJKunderwave{泰誓}三篇、\CJKunderwave{湯誓}、\CJKunderwave{牧誓}、\CJKunderwave{費誓}、\CJKunderwave{泰誓}八篇,誓也。\CJKunderwave{仲虺之誥}、\CJKunderwave{湯誥}、\CJKunderwave{大誥}、\CJKunderwave{康誥}、\CJKunderwave{酒誥}、\CJKunderwave{召誥}、\CJKunderwave{洛誥}、\CJKunderwave{康王之誥}八篇,誥也。\CJKunderwave{伊訓}一篇,訓也。\CJKunderwave{說命}三篇,\CJKunderwave{微子之命}、\CJKunderwave{蔡仲之命}、\CJKunderwave{顧命}、\CJKunderwave{畢命}、\CJKunderwave{冏命}、\CJKunderwave{文侯之命}九篇,命也。\CJKunderwave{胤徵}一篇,徵也。\CJKunderwave{洪範}一篇,範也。此各隨事而言。\CJKunderwave{益稷}亦謨也,因其人稱言以別之。其\CJKunderwave{太甲}、\CJKunderwave{咸有一德},\CJKunderline{伊尹}訓道王,亦訓之類。\CJKunderwave{盤庚}亦誥也,故王肅云:“不言誥,何也?取其徒而立功,非但錄其誥。”\CJKunderwave{高宗肜日}與訓序連文,亦訓辭可知也。\CJKunderwave{西伯戡黎}云,“\CJKunderline{祖伊}恐,奔告於受”,亦誥也。\CJKunderwave{武成}云,“識其政事”,亦誥也。\CJKunderwave{旅獒}戒王,亦訓也。\CJKunderwave{金縢}自為一體,祝亦誥辭也。\CJKunderwave{梓材},\CJKunderwave{酒誥}分出,亦誥也。\CJKunderwave{多士}以王命誥,自然誥也。\CJKunderwave{無逸}戒王,亦訓也。\CJKunderwave{君奭}\CJKunderline{周公}誥\CJKunderline{召公},亦誥也。\CJKunderwave{多方}、\CJKunderwave{周官}上誥於下,亦誥也。\CJKunderwave{君陳}、\CJKunderwave{君牙}與\CJKunderwave{畢命}之類,亦命也。\CJKunderwave{呂刑}陳刑告王,亦誥也。\CJKunderwave{書}篇之名,因事而立,既無體例,隨便為文。其百篇次第,於序孔、鄭不同。孔以\CJKunderwave{湯誓}在\CJKunderwave{夏社}前,於百篇為第二十六;鄭以為在\CJKunderwave{臣扈}後,第二十九。孔以\CJKunderwave{咸有一德}次\CJKunderwave{太甲}後,第四十;鄭以為在\CJKunderwave{湯誥}後,第三十二。孔以\CJKunderwave{蔡仲之命}次\CJKunderwave{君奭}後,第八十三;鄭以為在\CJKunderwave{費誓}前,第九十六。孔以\CJKunderwave{周官}在\CJKunderwave{立政}後,第八十八;鄭以為在\CJKunderwave{立政}前,第八十六。孔以\CJKunderwave{費誓}在\CJKunderwave{文侯之命}後,第九十九;鄭以為在\CJKunderwave{呂刑}前,第九十七。不同者孔依壁內篇次及序為文,鄭依賈氏所奏\CJKunderwave{別錄}為次,孔未入學官。以此不同,考論次第,孔義是也。 \par}

 {\noindent\shu\zihao{5}\fzkt 正義曰:\CJKunderwave{堯典}雖曰唐事,本以虞史所錄,末言舜登庸由堯,故追堯作典,非唐史所錄,故謂之\CJKunderwave{虞書}也。\CJKunderline{鄭玄}雲“舜之美事,在於堯時”是也。案馬融、\CJKunderline{鄭玄}、王肅、\CJKunderwave{別錄}題皆曰\CJKunderwave{虞夏書},以虞、夏同科,雖虞事亦連夏。此直言\CJKunderwave{虞書},本無\CJKunderwave{夏書}之題也。案鄭序以為\CJKunderwave{虞夏書}二十篇,\CJKunderwave{商書}四十篇,\CJKunderwave{周書}四十篇,\CJKunderwave{贊}雲“三科之條,五家之教”,是虞、夏同科也。其孔於\CJKunderwave{禹貢}注云“\CJKunderline{禹}之王以是功,故為\CJKunderwave{夏書}之首”,則虞、夏別題也,以上為\CJKunderwave{虞書}則十六篇。又\CJKunderwave{帝告}、\CJKunderwave{釐沃}、\CJKunderwave{湯徵}、\CJKunderwave{汝鳩}、\CJKunderwave{汝方}於\CJKunderline{鄭玄}為\CJKunderwave{商書},而孔並於\CJKunderwave{胤徵}之下,或以為夏事,猶\CJKunderwave{西伯戡黎},則\CJKunderwave{夏書}九篇,\CJKunderwave{商書}三十五篇,此與鄭異也。或孔因\CJKunderwave{帝告}以下五篇亡,並注於\CJKunderwave{夏書}不廢,猶\CJKunderwave{商書}乎?別文所引皆雲“\CJKunderwave{虞書}曰”、“\CJKunderwave{夏書}曰”,無並言\CJKunderwave{虞夏書}者。又伏生雖有一\CJKunderwave{虞夏傳},以外亦有\CJKunderwave{虞傳}、\CJKunderwave{夏傳},此其所以宜別也,此孔依虞、夏各別而存之。莊八年\CJKunderwave{左傳}引“\CJKunderwave{夏書}曰‘\CJKunderline{皋陶}邁種德’”,僖二十四年\CJKunderwave{左傳}引夏書曰“地平天成”,二十七年引\CJKunderwave{夏書}“賦納以言”,襄二十六年引\CJKunderwave{夏書}曰“與其殺不辜,寧失不經”,皆在\CJKunderwave{大禹謨}、\CJKunderwave{皋陶謨}。當雲\CJKunderwave{虞書}而云\CJKunderwave{夏書}者,以事關\CJKunderline{禹},故引為\CJKunderwave{夏書}。若\CJKunderwave{洪範}以為\CJKunderwave{周書},以\CJKunderline{箕子}至周,商人所陳而傳引之,即曰\CJKunderwave{商書}也。案壁內所得,孔為傳者凡五十八篇,為四十六卷。三十三篇與鄭注同,二十五篇增多鄭注也。其二十五篇者,\CJKunderwave{大禹謨}一,\CJKunderwave{五子之歌}二,\CJKunderwave{胤徵}三,\CJKunderwave{仲虺之誥}四,\CJKunderwave{湯誥}五,\CJKunderwave{伊訓}六,\CJKunderwave{太甲}三篇九,\CJKunderwave{咸有一德}十,\CJKunderwave{說命}三篇十三,\CJKunderwave{泰誓}三篇十六,\CJKunderwave{武成}十七,\CJKunderwave{旅獒}十八,\CJKunderwave{微子之命}十九,\CJKunderwave{蔡仲之命}二十,\CJKunderwave{周官}二十一,\CJKunderwave{君陳}二十二,\CJKunderwave{畢命}二十三,\CJKunderwave{君牙}二十四,\CJKunderwave{冏命}二十五。但\CJKunderline{孔君}所傳,值巫蠱不行以終。前漢諸儒知孔本有五十八篇,不見孔傳,遂有張霸之徒於鄭注之外偽造\CJKunderwave{尚書}凡二十四篇,以足鄭注三十四篇為五十八篇。其數雖與孔同,其篇有異。孔則於伏生所傳二十九篇內無古文\CJKunderwave{泰誓},除\CJKunderwave{序}尚二十八篇,分出\CJKunderwave{舜典}、\CJKunderwave{益稷}、\CJKunderwave{盤庚}二篇、\CJKunderwave{康王之誥}為三十三,增二十五篇為五十八篇。\CJKunderline{鄭玄}則於伏生二十九篇之內分出\CJKunderwave{盤庚}二篇,\CJKunderwave{康王之誥}、又\CJKunderwave{泰誓}三篇,為三十四篇更增益偽書二十四篇為五十八。所增益二十四篇者,則鄭注\CJKunderwave{書序},\CJKunderwave{舜典}一,\CJKunderwave{汩作}二,\CJKunderwave{九共}九篇十一,\CJKunderwave{大禹謨}十二,\CJKunderwave{益稷}十三,\CJKunderwave{五子之歌}十四,\CJKunderwave{胤徵}十五,\CJKunderwave{湯誥}十六,\CJKunderwave{咸有一德}十七,\CJKunderwave{典寶}十八,\CJKunderwave{伊訓}十九,\CJKunderwave{肆命}二十,\CJKunderwave{原命}二十一,\CJKunderwave{武成}二十二,\CJKunderwave{旅獒}二十三,\CJKunderwave{冏命}二十四。以此二十四為十六卷,以\CJKunderwave{九共}九篇共卷,除八篇,故為十六。故\CJKunderwave{藝文志}、劉向\CJKunderwave{別錄}雲“五十八篇”。\CJKunderwave{藝文志}又云:“孔安國者,\CJKunderline{孔子}後也。悉得其書,以古文又多十六篇。”篇即卷也。即是偽書二十四篇也。劉向作\CJKunderwave{別錄},班固作\CJKunderwave{藝文志}並雲此言,不見孔傳也。劉歆作\CJKunderwave{三統曆},論\CJKunderline{武王}伐紂,引今文\CJKunderwave{泰誓}雲“丙午逮師”,又引\CJKunderwave{武成}“越若來三月五日甲子,咸劉商王受”,並不與孔同,亦不見孔傳也。後漢初賈逵\CJKunderwave{奏尚書疏}雲“流為烏”,是與孔亦異也。馬融\CJKunderwave{書序}云:“經傳所引\CJKunderwave{泰誓},\CJKunderwave{泰誓}並無此文。”又云:“逸十六篇,絕無師說。”是融亦不見也。服虔、杜預注\CJKunderwave{左傳}“亂其紀綱”,並雲\CJKunderline{夏桀}時,服虔、杜預皆不見也。\CJKunderline{鄭玄}亦不見之,故注\CJKunderwave{書序}、\CJKunderwave{舜典}雲“入麓伐木”,注\CJKunderwave{五子之歌}雲“避亂於洛汭”,注\CJKunderwave{胤徵}雲“胤徵,臣名”,又注\CJKunderwave{禹貢}引\CJKunderwave{胤徵}雲“厥匪玄黃,昭我周王”,又注\CJKunderwave{咸有一德}雲“\CJKunderline{伊陟}臣扈曰”,又注\CJKunderwave{典寶}引\CJKunderwave{伊訓}雲“載孚在毫”,又曰“徵是三朡”,又注\CJKunderwave{旅獒}雲“獒讀曰豪,謂是遒豪之長”,又古文有\CJKunderwave{仲虺之誥}、\CJKunderwave{太甲}、\CJKunderwave{說命}等見在而云亡,其\CJKunderwave{汩典}、\CJKunderwave{典寶}之等一十三篇見亡而云已,逸是不見古文也。案伏生所傳三十四篇者謂之今文,則夏侯勝、夏侯建、歐陽和伯等三家所傳及後漢末蔡邕所勒石經是也。孔所傳者,膠東庸生、劉歆、賈逵、馬融等所傳是也。\CJKunderline{鄭玄}\CJKunderwave{書贊}云:“我先師棘子下生安國,亦好此學,衛、賈、馬二三君子之業,則雅才好博,既宣之矣。”又云:“歐陽氏失其本義,今疾此蔽冒,猶復疑惑未悛。”是鄭意師祖孔學,傳授膠東庸生、劉歆、賈逵、馬融等學,而賤夏侯、歐陽等;何意鄭注\CJKunderwave{尚書},亡逸並與孔異,篇數並與三家同?又劉歆、賈逵、馬融之等並傳孔學,雲十六篇逸,與安國不同者,良由孔注之後,其書散逸,傳注不行。以庸生、賈、馬之等惟傳孔學經文三十三篇,故鄭與三家同,以為古文。而鄭承其後,所注皆同賈逵、馬融之學,題曰\CJKunderwave{古文尚書},篇與夏侯等同,而經字多異。夏侯等書“宅嵎夷”為“宅嵎鐵”,“昧谷”,曰“柳谷”,“心腹腎腸”曰“憂腎陽”,“劓刵劅剠”雲“臏宮劓割頭庶剠”,是鄭注不同也。三家之學傳孔業者,\CJKunderwave{漢書·儒林傳}云,安國傳都尉朝子俊,俊傳膠東庸生,生傳清河胡常,常傳徐敖,敖傳王璜及塗惲,惲傳河南桑欽。至後漢初衛、賈、馬亦傳孔學,故\CJKunderwave{書贊}云:“自世祖興後漢,衛、賈、馬二三君子之業是也,所得傳者三十三篇古經,亦無其五十八篇,及傳說絕無傳者。”至晉世王肅注\CJKunderwave{書},始似竊見孔傳,故注“亂其紀綱”為夏太康時。又\CJKunderwave{晉書·皇甫謐傳}云:“姑子外弟梁柳邊得\CJKunderwave{古文尚書},故作\CJKunderwave{帝王世紀},往往載孔傳五十八篇之書。”\CJKunderwave{晉書}又云:“晉太保公鄭衝以古文授扶風蘇愉,愉字休預。預授天水梁柳,字洪季,即謐之外弟也。季授城陽臧曹,字彥始。始授郡守子汝南梅賾,字仲真,又為豫章內史,遂於前晉奏上其書而施行焉。”時已亡失\CJKunderwave{舜典}一篇,晉末範寧為解時已不得焉。至齊蕭鸞建武四年姚方興於大航頭得而獻之,議者以為孔安國之所注也。值方興有罪,事亦隨寢。至隋開皇二年購慕遺典,乃得其篇焉。然孔注之後歷及後漢之末,無人傳說。至晉之初猶得存者,雖不列學官,散在民間,事雖久遠,故得猶存。 \par}

\CJKunderline{孔氏}傳\footnote{傳即注也,以傳述為義,舊說漢已前稱傳。}

{\noindent\shu\zihao{5}\fzkt 正義曰:以注者多門,故云其氏以別眾家。或當時自題\CJKunderline{孔氏},亦可以後人辨之。 \par}

\textcolor{red}{昔在}\CJKunderline{帝堯},聰明文思,光宅天下。\footnote{\textcolor{red}{言聖德之遠著}。昔,古也。堯,唐帝名。馬融云:“諡也,翼善傳聖曰堯。”宅,拥有。聰,千公反。思,息嗣反,又如字,下同。著,張慮反。}將遜於位,讓於\CJKunderline{虞舜},\footnote{遜,遁也。\textcolor{red}{老使}攝,遂\textcolor{red}{禪之}。遁本作遯,徒遜反,退也,避也。“遂禪”音時戰反,讓也,授也。}作\CJKunderwave{\textcolor{red}{堯典}}。

{\noindent\zhuan\zihao{6}\fzbyks 傳“言聖德之遠著”。正義曰:“聖德”解“聰明文思”,“遠著”解“光宅天下”。 \par}

{\noindent\zhuan\zihao{6}\fzbyks 傳“老使”至“禪之”。正義曰:“老使攝”者解“將遜於位”,雲“遂禪之”者,解“讓於\CJKunderline{虞舜}”也。以己年老,故遜之。使攝之,後功成而禪。禪即讓也。言“攝”者,“納於大麓”是也。禪者,“汝陟帝位”是也。雖舜受而攝之,而堯以為禪。或雲“汝陟帝位”為攝,因即直言為讓,故云“遂”也。\CJKunderline{鄭玄}云:“堯尊如故,舜攝其事”是也。 \par}

{\noindent\shu\zihao{5}\fzkt “昔在”至“堯典”。正義曰:此序\CJKunderline{鄭玄}、馬融、王肅並雲\CJKunderline{孔子}所作,孔義或然。\CJKunderwave{詩}、\CJKunderwave{書}理不應異,夫子為\CJKunderwave{書}作序,不作\CJKunderwave{詩}序者,此自或作或否,無義例也。鄭知\CJKunderline{孔子}作者,依緯文而知也。安國既以同序為卷,檢此百篇,凡有六十三序,序其九十六篇。\CJKunderwave{明居}、\CJKunderwave{咸有一德}、\CJKunderwave{立政}、\CJKunderwave{無逸}不序所由,直雲“\CJKunderline{咎單}作\CJKunderwave{明居}”、“\CJKunderline{伊尹}作\CJKunderwave{咸有一德}”、“\CJKunderline{周公}作\CJKunderwave{立政}”、“\CJKunderline{周公}作\CJKunderwave{無逸}”。六十三序者,若\CJKunderwave{汩作}、\CJKunderwave{九共}九篇,\CJKunderwave{槁飫},十一篇共序;其\CJKunderwave{咸乂}四篇同序;其\CJKunderwave{大禹謨}、\CJKunderwave{皋陶謨}、\CJKunderwave{益稷}、\CJKunderwave{夏社}、\CJKunderwave{疑至}、\CJKunderwave{臣扈}、\CJKunderwave{伊訓}、\CJKunderwave{肆命}、\CJKunderwave{徂後}、\CJKunderwave{太甲}三篇、\CJKunderwave{盤庚}三篇、\CJKunderwave{說命}三篇、\CJKunderwave{泰誓}三篇、\CJKunderwave{康誥}、\CJKunderwave{酒誥}、\CJKunderwave{梓材},二十四篇,皆三篇同序;其\CJKunderwave{帝告}、\CJKunderwave{釐沃}、\CJKunderwave{汝鳩}、\CJKunderwave{汝方}、\CJKunderwave{伊陟}、\CJKunderwave{原命}、\CJKunderwave{高宗肜日}、\CJKunderwave{高宗之訓}八篇皆共卷,類同,故同序。同序而別篇者三十三篇,通\CJKunderwave{明居}、\CJKunderwave{無逸}等四篇為三十七篇,加六十三即百篇也。序者以序別行辭為形勢。言昔日在於帝號堯之時也,此堯身智無不知聰也,神無不見明也。以此聰明之神智足可以經緯天地,即“文”也;又神智之運,深敏於機謀,即“思”也。“聰明文思”即其聖性行之於外,無不備知,故此德充滿居止於天下而遠著。德既如此,政化有成,天道衝盈,功成者退,以此故將遜遁避於帝位,以禪其有聖德之\CJKunderline{虞舜}。史序其事,而作\CJKunderwave{堯典}之篇。言“昔在”者,\CJKunderline{鄭玄}云:“\CJKunderwave{書}以堯為始,獨雲“昔在”,使若無先之典然也。”\CJKunderwave{詩}云:“自古在昔。”言“在昔”者,自下本上之辭。言“昔在”者,從上自下為稱,故曰“使若無先之”者。據代有先之,而書無所先,故云“昔”也。言“帝”者,天之一名,所以名“帝”。帝者,諦也。言天蕩然無心,忘於物我,言公平通遠,舉事審諦,故謂之“帝”也。五帝道同於此,亦能審諦,故取其名。若然,聖人皆能同天,故曰“大人”。大人者與天地合其德,即三王亦大人。不得稱帝者,以三王雖實聖人,內德同天,而外隨時運,不得盡其聖,用逐跡為名,故謂之為王。\CJKunderwave{禮運}曰,“大道之行,天下為公”,即帝也。“大道既隱,各親其親”,即王也。則聖德無大於天,三皇優於帝,豈過乎天哉!然則三皇亦不能過天,但遂同天之名,以為優劣。五帝有為而同天,三皇無為而同天,立名以為優劣耳。但有為無為亦逐多少以為分,三王亦順帝之則而不盡,故不得名帝。然天之與帝,義為一也。人主可得稱帝,不可得稱天者,以天隨體而立名,人主不可同天之體也。無由稱天者,以天德立號,王者可以同其德焉,所以可稱於帝。故繼天則謂之“天子”,其號謂之“帝”,不得雲“帝子”也。言“堯”者,孔無明解。案下傳云:“虞,氏。舜,名。”然堯、舜相配為義,既舜為名,則堯亦名也。以此而言,\CJKunderline{禹}、湯亦名。於下都無所解,而放勳、\CJKunderline{重華}、文命注隨其事而解其文以為義,不為堯、舜及\CJKunderline{禹}之名。據此,似堯、舜及\CJKunderline{禹}與湯相類,名則俱名,不應殊異。案鄭以下亦云:“虞,氏。舜,名。”與孔傳不殊。及鄭注\CJKunderwave{中侯},云:“\CJKunderline{重華},舜名。”則舜不得有二名。鄭注\CJKunderwave{禮記}云:“舜之言充。”是以舜為號諡之名,則下注雲“舜,名,亦號諡之名”也。推此則\CJKunderline{孔君}亦然。何以知之?既湯類堯、舜當為名,而孔注\CJKunderwave{論語}“曰予小子履”云,“履是殷湯名堯舜,是湯名履,而湯非名也。又此不雲堯、舜是名,則堯及舜、\CJKunderline{禹}非名,於是明矣。既非名,而放勳、\CJKunderline{重華}、文命蓋以為三王之名,同於\CJKunderline{鄭玄}矣。鄭知名者,以\CJKunderwave{帝系}雲“\CJKunderline{禹}名文命”,以上類之亦名。若然,名本題情記意,必有義者,蓋運命相符,名與運接,所以異於凡平。或說以其有義,皆以為字。古代尚質,若名之不顯,何以著字?必不獲已,以為非名非字可也。譙周以堯為號,皇甫謐以放勳、\CJKunderline{重華}、文命為名。案\CJKunderwave{諡法}“翼善傳聖曰堯,仁義盛明曰舜”,是堯、舜諡也。故馬融亦云諡也。又曰“淵源流通曰\CJKunderline{禹},雲行雨施曰湯”,則\CJKunderline{禹}、湯亦是諡法。而馬融云:“\CJKunderline{禹}湯不在\CJKunderwave{諡法}。”故疑之。將由\CJKunderwave{諡法}或本不同,故有致異。亦可本無\CJKunderline{禹}、湯為諡,後來所加,故或本曰“除虐去殘曰湯”,是以異也。\CJKunderwave{檀弓}曰:“死諡,周道也。”\CJKunderwave{周書}諡法\CJKunderline{周公}所作,而得有堯、舜、\CJKunderline{禹}、湯者,以周法死後乃追,故謂之為諡。諡者,累也,累其行而號也。隨其行以名之,則死諡猶生號。因上世之生號陳之為死諡,明上代生死同稱。上世質,非至善至惡無號,故與周異。以此堯、舜或雲號,或雲諡也。若然,湯名履而王侯,\CJKunderwave{世本}“湯名天乙”者,安國意蓋以湯受命之王,依殷法以乙日生,名天乙。至將為王,又改名為履,故二名也。亦可。安國不信\CJKunderwave{世本},無天乙之名。皇甫謐巧欲傅會,雲“以乙日生,故名履,字天乙”。又云\CJKunderline{祖乙},亦云“乙日生,複名乙”,引\CJKunderwave{易緯}“\CJKunderline{孔子}所謂天之錫命,故可同名”。既以天乙為字,何雲同名乎?斯又妄矣,號之曰堯者,\CJKunderwave{釋名}以為“其尊高堯堯然,物莫之先,故謂之堯也”。\CJKunderwave{諡法}云:“翼善傳聖曰堯。”堯者以天下之生善,因善欲禪之,故二八顯升,所謂為翼。能傳位於聖人,天下為公,此所以出眾而高也。言“聰明者,據人近驗,則聽遠為聰,見微為明,若離婁之視明也,師曠之聽聰也;以耳目之聞見,喻聖人之智慧,兼知天下之事,故在於聞見而已,故以“聰明”言之。智之所用,用於天地,經緯天地謂之“文”,故以聰明之用為文。須當其理,故又云“思”而會理也。經雲“欽明”,此為聰明者,彼方陳行事,故美其敬,此序其聖性,故稱其“聰”,隨事而變文。下\CJKunderwave{舜典}直雲“堯聞之聰明”,不雲“文思”者,此將言堯用,故云“文思”,彼要雲舜德,故直雲“聰明”,亦自此而可知也。言“光宅”者,經傳云:“光,充也。”不訓“宅”者,可知也。不於此訓“光”者,從經為正也。下“將遜於位”傳雲“遜,遁”者,以經無“遜”字,故在序訓之。 \par}

\textcolor{red}{堯典}\footnote{言堯可為百代常行之道。}

{\noindent\shu\zihao{5}\fzkt “堯典”。正義曰:序已雲“作堯典”而重言此者,此是經之篇目,不可因序有名,略其舊題,故諸篇皆重言本目而就目解之。稱“典”者,以道可百代常行。若堯、舜禪讓聖賢,\CJKunderline{禹}、湯傳授子孫,即是堯、舜之道不可常行,但惟德是與,非賢不授。授賢之事,道可常行,但后王德劣不能及古耳。然經之與典俱訓為常,名典不名經者,以經是總名,包殷、周以上,皆可為後代常法,故以經為名。典者,經中之別,特指堯、舜之德,於常行之內道最為優,故名典不名經也。其太宰六典及司寇三典者,自由當代常行,與此別矣。 \par}

\textcolor{red}{曰若}稽古,\CJKunderline{帝堯}\footnote{\textcolor{red}{若,順}。稽,考也。能順考古道而行之者\textcolor{red}{\CJKunderline{帝堯}}。}曰放勳,欽明文思安安\footnote{\textcolor{red}{勳,功}。欽,敬也。言堯放上世之功,化而以敬明文思之四德,安天下之當\textcolor{red}{安者}。放,方往反,注同。徐云:“鄭、王如字。”勳,許雲反,功也。馬云:“放勳,堯名。”皇甫謐同。一云:“放勳,堯字。”“欽明文思”,馬云:“威儀表備之欽,照臨四方謂之明,經緯天地謂之文,道德純備謂之思。”},允恭克讓,光被四表,格於\textcolor{red}{上下}。\footnote{\textcolor{red}{允,信}。克,能。光,充。格,至也。既有四德,又信恭能讓,故其名聞充溢四外,至於\textcolor{red}{天地}。被,皮寄反,徐扶義反。聞音問,本亦作問。溢音逸。}

{\noindent\zhuan\zihao{6}\fzbyks 傳“若順”至“\CJKunderline{帝堯}”。正義曰:“若、順”,\CJKunderwave{釋言}文。\CJKunderwave{詩}稱“考卜惟王”,\CJKunderwave{洪範}考卜之事謂之“稽疑”,是“稽”為考,經傳常訓也。\CJKunderwave{爾雅}一訓一也,孔所以約文,故數字俱訓,其末以一“也”結之。又已經訓者,後傳多不重訓。顯見可知,則徑言其義,皆務在省文故也。言“順考古道”者,古人之道非無得失,施之當時又有可否,考其事之是非,知其宜於今世,乃順而行之。言其行可否,順是不順非也。考“古”者自己之前,無遠近之限,但事有可取,皆考而順之。今古既異時,政必殊古,事雖不得盡行,又不可頓除古法,故\CJKunderwave{說命}曰:“事不師古,以克永世,匪說攸聞。”是後世為治當師古法,雖則聖人,必須順古。若空欲追遠,不知考擇,居今行古,更致禍災。若宋襄慕義,師敗身傷,徐偃行仁,國亡家滅,斯乃不考之失。故美其能順考也。\CJKunderline{鄭玄}信緯,訓“稽”為同,訓“古”為天,言“能順天而行之,與之同功”。\CJKunderwave{論語}稱惟堯則天,\CJKunderwave{詩}美\CJKunderline{文王}“順帝之則”,然則聖人之道莫不同天合德,豈待同天之語,然後得同之哉?\CJKunderwave{書}為世教,當因之人事,以人系天,於義無取,且“古”之為天,經無此訓。高貴鄉公皆以鄭為長,非篤論也。 \par}

{\noindent\zhuan\zihao{6}\fzbyks 傳“勳功”至“安者”。正義曰:“勳、功”、“欽、敬”,\CJKunderwave{釋詁}文。此經述上稽古之事,放效上世之功,即是考於古道也。經言“放勳”,放其功而已。傳兼言“化”者,據其勳業謂之功,指其教人則為化,功之與化所從言之異耳。\CJKunderline{鄭玄}云:“敬事節用謂之欽,照臨四方謂之明,經緯天地謂之文,慮深通敏謂之思。”孔無明說,當與之同。四者皆在身之德,故謂之“四德”。凡是臣人王者皆須安之,故廣言“安天下之當安者”。所安者則下文“九族、百姓、萬邦”是也。其“敬明文思”為此次者,顧氏云:“隨便而言,無義例也。”知者此先“聰”後“明”,\CJKunderwave{舜典}雲“明四目,達四聰”,先明後“聰”,故知無例也。今考\CJKunderwave{舜典}云:“浚哲文明”,又先“文”後“明”,與此不類,知顧氏為得也。 \par}

{\noindent\zhuan\zihao{6}\fzbyks 傳“允信”至“天地”。正義曰:“允、信”、“格、至”,\CJKunderwave{釋詁}文。“克,能”、“光,充”,\CJKunderwave{釋言}文。在身為德,施之曰行。\CJKunderline{鄭玄}云:“不懈於位曰恭,推賢尚善曰讓。”恭讓是施行之名。上言堯德,此言堯行,故傳以文次言之。言堯既有敬明文思之四德,又信實、恭勤、善能、推讓,下人愛其恭讓,傳其德音,故其名遠聞,旁行則充溢四方,上下則至於天地。持身能恭,與人能讓,自己及物,故先恭後讓。恭言信,讓言克,交互其文耳。皆言信實能為也。傳以“溢”解“被”,言其饒多盈溢,故被及之也。表裡內外相對之言,故以表為外,向不向上至有所限,旁行四方無復限極,故四表言“被”,上下言“至”。“四外”者,以其無限,自內言之,言其至於遠處,正謂四方之外畔者,當如\CJKunderwave{爾雅}所謂“四海”、“四荒”之地也。先“四表”後“上下”者,人之聲名,宜先及於人,後被四表,是人先知之,故先言至人。後言至於上下,言至於天地,喻其聲聞遠耳。\CJKunderwave{禮運}稱聖人為政,能使“天降膏露,地出醴泉”,是名聞遠達,使天地效靈,是亦格於上下之事。 \par}

{\noindent\shu\zihao{5}\fzkt “曰若”至“上下”。正義曰:史將述堯之美,故為題目之辭曰,能順考校古道而行之者,是\CJKunderline{帝堯}也。又申其順考古道之事曰,此\CJKunderline{帝堯}能放效上世之功而施其教化,心意恆敬,智慧甚明,發舉則有文謀,思慮則能通敏,以此四德安天下之當安者。在於己身則有此四德,其於外接物又能信實、恭勤、善能、謙讓。恭則人不敢侮,讓則人莫與爭,由此為下所服,名譽著聞,聖德美名充滿被溢於四方之外,又至於上天下地。言其日月所照,霜露所墜,莫不聞其聲名,被其恩澤。此即稽古之事也。 \par}


\textcolor{red}{克明}俊德,以親九族\footnote{\textcolor{red}{能明}俊德之士任用之,以睦高祖玄孫\textcolor{red}{之親}。九族,上自高祖,下至玄孫,凡九族。馬、鄭同。}。九族既睦,平章百姓\footnote{\textcolor{red}{既,已}也。百姓,百官。言化九族而平和\textcolor{red}{章明}。}。百姓昭明,協和萬邦。黎民於變\textcolor{red}{時雍}\footnote{\textcolor{red}{昭亦}明也。協,合。黎,眾。時,是。雍,和也。言天下眾民皆變化化上,是以風俗\textcolor{red}{大和}。黎,力兮反。}。

{\noindent\zhuan\zihao{6}\fzbyks 傳“能明”至“之親”。正義曰:\CJKunderline{鄭玄}云:“‘俊德’,賢才兼人者。”然則“俊德”謂有德。人能明俊德之士者,謂命為大官,賜之厚祿,用其才智,使之高顯也。以其有德,故任用之。以此賢臣之化,親睦高祖玄孫之親。上至高祖,下及玄孫,是為九族。同出高曾,皆當親之,故言之“親”也。\CJKunderwave{禮記·喪服小記}云:“親親以三為五,以五為九。”又\CJKunderwave{異義}、夏侯、歐陽等以為九族者,父族四、母族三、妻族二,皆據“異姓有服”。\CJKunderline{鄭玄}駁云:“異姓之服不過緦麻,言不廢昏。又\CJKunderwave{昏禮}請期雲‘惟是三族之不虞’,恐其廢昏,明非外族也。”是鄭與孔同。“九族”謂帝之九族,“百姓”謂百官族姓,“萬邦”謂天下眾民,自內及外,從高至卑,以為遠近之次也。知“九族”非民之九族者,以先親九族,次及百姓,百姓是群臣弟子,不宜越百姓而先下民。若是民之九族,則“九族既睦”,民已和矣,下句不當復言“協和萬邦”,以此知帝之九族也。堯不自親九族,而待臣使之親者,此言用臣法耳,豈有聖人在上,疏其骨肉者乎?若以堯自能親,不待臣化,則化萬邦百姓,堯豈不能化之,而待臣化之也?且言“親九族”者,非徒使帝親之,亦使臣親之,帝亦令其自相親愛,故須臣子之化也。 \par}

{\noindent\zhuan\zihao{6}\fzbyks 傳“既,已”至“章明”。正義曰:“既”、“已”義同,故訓“既”為已,經傳之言。“百姓”或指天下百姓,此下句乃有“黎民”,故知“百姓”即百官也。百官謂之百姓者,隱八年\CJKunderwave{左傳}云:“天子建德,因生以賜姓。”謂建立有德以為公卿,因其所生之地而賜之以為其姓,令其收斂族親,自為宗主。明王者任賢不任親,故以“百姓”言之。\CJKunderwave{周官}篇云:“唐、虞稽古,建官惟百。”\CJKunderwave{大禹謨}云:“率百官若帝之初。”是唐、虞之世經文皆稱“百官”。而\CJKunderwave{禮記·明堂位}雲“有虞氏之官五十”,後世所記不合經也。“平章”與“百姓”其文非九族之事,傳以此經之事文勢相因,先化九族,乃化百官,故云“化九族而平和章明”。謂九族與百官皆須導之以德義,平理之使之協和,教之以禮法,章顯之使之明著。 \par}

{\noindent\zhuan\zihao{6}\fzbyks 傳“昭亦”至“大和”。正義曰:\CJKunderwave{釋詁}以“昭”為光,光、明義同,經已有“明”,故云“昭亦明也”。\CJKunderwave{釋詁}以“協”為和,和、合義同,故訓“協”為合也。“黎,眾”、“時,是”,\CJKunderwave{釋詁}文。“雍,和”,\CJKunderwave{釋訓}文。堯民之變,明其變惡從善,人之所和,惟風俗耳。故知謂“天下眾人皆變化化上,是以風俗大和”,人俗大和,即是太平之事也。此上經三事相類,古史交互立文。以“親”言“既睦”,“平章”言“昭明”,“協和”言“時雍”。“睦”即“親”也,“章”即“明”也,“雍”即“和”也,各自變文以類相對。平九族使之親,平百姓使之明,正謂使從順禮義,恩情和合,故於萬邦變言“協和”,明“以親九族”、“平章百姓”亦是協和之也。但九族宜相親睦,百姓宜明禮義,萬邦宜盡和協,各因所宜為文,其實相通也。民言“於變”,謂從上化,則“九族既睦”、“百姓昭明”亦是變上,故得睦得明也。 \par}

{\noindent\shu\zihao{5}\fzkt “克明”至“時雍”。正義曰:言堯能名聞廣遠,由其委任賢哲,故復陳之。言堯之為君也,能尊明俊德之士,使之助己施化。以此賢臣之化,先令親其九族之親。九族蒙化已親睦矣,又使之和協顯明於百官之族姓。百姓蒙化皆有禮儀,昭然而明顯矣,又使之合會調和天下之萬國。其萬國之眾人於是變化從上,是以風俗大和,能使九族敦睦,百姓顯明,萬邦和睦,是“安天下之當安”者也。 \par}

\textcolor{red}{乃命}\CJKunderline{羲}、\CJKunderline{和},欽若昊天,曆象日月星辰,敬授人時\footnote{\textcolor{red}{重黎}之後羲氏、和氏世掌天地四時之官,故堯命之,使敬順昊天。昊天言元氣廣大。星,四方中星。辰,日月所會。曆象其分節。敬記天時以授人也。此舉其目,下別\textcolor{red}{序之}。羲和,馬云:“羲氏掌天官,和氏掌地官,四子掌四時。”昊,胡老反。重,直龍反,少昊之後。黎,高陽之後。“日月所會”謂日月交會於十二次也。寅曰析木,卯曰大火,辰曰壽星,巳曰鶉尾,午曰鶉火,未曰鶉首,申曰實沈,酉曰大梁,戌曰降婁,亥曰娵訾,子曰玄枵,醜曰星紀。}。

{\noindent\zhuan\zihao{6}\fzbyks 傳“重黎”至“序之”。正義曰:\CJKunderwave{楚語}云:“少昊氏之衰,九黎亂德,人神雜擾,不可方物。顓頊受之,乃命南正重司天以屬神,火正黎司地以屬民,使復舊常,無相侵瀆。其後三苗復九黎之惡,堯覆育重黎之後,不忘舊者,使復典之。以至於夏商。”據此文則自堯及商無他姓也。堯育重黎之後,是此羲和可知。是羲和為重黎之後,世掌天地之官文所出也。\CJKunderwave{呂刑}先“重”後“黎”,此文先“羲”後“和”,楊子\CJKunderwave{法言}云:“羲近重,和近黎。”是“羲”承“重”而“和”承“黎”矣。\CJKunderwave{呂刑}稱“乃命重黎”與此“命羲和”為一事也。故\CJKunderwave{呂刑}傳云:“重即羲也,黎即和也。”羲和雖別為氏族而出自重黎,故\CJKunderwave{呂刑}以“重黎”言之。\CJKunderwave{鄭語}云:“為高辛氏火正。”則高辛亦命重黎。故\CJKunderline{鄭玄}於此注云:“高辛氏世,命重為南正司天,黎為火正司地。”據“世掌”之文,用\CJKunderwave{楚語}為說也。\CJKunderwave{楚世家}云:“重黎為帝嚳火正,能光融天下,啻嚳命曰祝融。\CJKunderline{共工氏}作亂,帝嚳使重黎誅之而不盡。帝乃以庚寅日誅重黎,而以其弟吳回為重黎,復居火正,為祝融。”案昭二十九年\CJKunderwave{左傳}稱少昊氏有子曰重,顓頊氏有子曰黎。則重黎二人,各出一帝。而\CJKunderwave{史記}並以重黎為楚國之祖,吳回為重黎,以重黎為官號,此乃\CJKunderwave{史記}之謬。故束晳譏馬遷並兩人以為一,謂此是也。\CJKunderwave{左傳}稱重為句芒,黎為祝融,不言何帝使為此官。但黎是顓頊之子,其為祝融,必在顓頊之世。重雖少昊之胤,而與黎同命,明使重為句芒亦是顓頊時也。祝融火官可得稱為火正,句芒木官不應號為南正,且木不主天,火不主地,而\CJKunderwave{外傳}稱顓頊命南正司天,火正司地者,蓋使木官兼掌天,火官兼掌地。南為陽位,故掌天謂之南正。黎稱本官,故掌地猶為火正。鄭答趙商云:“先師以來,皆雲火掌為地,當雲黎為北正。”孔無明說,未必然也。昭十七年\CJKunderwave{左傳}郯子稱少昊氏以鳥名官,自顓頊已來乃命以民事。句芒、祝融皆以人事名官,明此當顓頊之時也。傳言少昊氏有四叔,當為後代子孫,非親子也。何則?傳稱\CJKunderline{共工氏}有子曰句龍,\CJKunderline{共工氏}在顓頊之前多歷年代,豈復\CJKunderline{共工氏}親子至顓頊時乎?明知少昊四叔亦非親子,高辛所命重黎或是重黎子孫,未必一人能歷二代。又高辛前命後誅,當是異人。何有罪而誅,不容列在祀典。明是重黎之後,世以重黎為號,所誅重黎是有功重黎之子孫也。\CJKunderwave{呂刑}說羲和之事,猶尚謂之重黎,況彼尚近重黎,何故不得稱之?以此知異世重黎號同人別。顓頊命重司天,黎司地,羲氏掌天,和氏掌地,其實重、黎、羲、和通掌之也。此雲“乃命羲和,欽若昊天”,是羲和二氏共掌天地之事。以乾坤相配,天地相成,運立施化者天,資生成物者地,天之功成其見在地,故下言“日中,星鳥”之類是天事也,“平秩東作”之類是地事也,各分掌其時,非別職矣。案\CJKunderwave{楚語}云,重司天以屬神,黎言地以屬人。天地既別,人神又殊,而云通掌之者,外傳之文說\CJKunderwave{呂刑}之義,以為少昊之衰,天地相通,人神雜擾,顓頊乃命重黎分而異之,以解絕地天通之言,故云各有所掌。天地相通,人神雜擾,見其能離絕天地,變異人神耳,非即別掌之。下文別序所掌,則羲主春夏,和主秋冬,俱掌天時,明其共職。彼又言:“至於夏商,世掌天地。”\CJKunderwave{胤徵}云:“羲和湎淫,廢時亂日。”不知日食,羲和同罪,明其世掌天地共職。可知顓頊命掌天地,惟重黎二人,堯命羲和則仲叔四人者,以羲和二氏賢者既多,且後代稍文,故分掌其職事,四人各職一時,兼職方岳,以有四嶽,故用四人。顓頊之命重黎,惟司天地,主嶽以否不可得知。設令亦主方岳,蓋重黎二人分主東西也。馬融、\CJKunderline{鄭玄}皆以此“命羲和”者,命為天地之官。下雲“分命”,申命為四時之職。天地之與四時於周則冢宰司徒之屬,六卿是也。孔言“此舉其目,下別序之”,則惟命四人,無六官也。下傳雲四嶽即羲和四子,\CJKunderwave{舜典}傳稱\CJKunderline{禹}、益六人新命有職,與四嶽十二牧凡為二十二人。然新命之六人,\CJKunderline{禹}命為百揆,契作司徒,\CJKunderline{\CJKunderline{伯夷}}為秩宗,\CJKunderline{皋陶}為士,垂作\CJKunderline{共工},亦\CJKunderline{禹}、契之輩即是卿官。卿官之外別有四嶽,四嶽非卿官也。孔意以羲和非是卿官,別掌天地,但天地行於四時,四時位在四方,平秩四時之人因主方岳之事,猶自別有卿官分掌諸職。\CJKunderwave{左傳}稱少昊氏以鳥名官,五鳩氏即周世之卿官也。五鳩之外別有鳳鳥氏,歷正也,班在五鳩之上。是上代以來皆重歷數,故知堯於卿官之外別命羲和掌天地也。於時羲和似尊於諸卿,後世以來稍益卑賤。\CJKunderwave{周禮}“太史掌正歲年以序事”,即古羲和之任也。桓十七年\CJKunderwave{左傳}雲“日官居卿以底日”,猶尚尊其所掌。周之卿官明是堯時重之,故特言“乃命羲和”。此“乃命羲和”重述“克明俊德”之事,得致雍和所由。已上論堯聖性,此說堯之任賢,據堯身而言用臣,故云“乃命”,非“時雍”之後方始命之。“使敬順昊天”,昊天者混元之氣,昊然廣大,故謂之“昊天”也。\CJKunderwave{釋天}云:“春為蒼天,夏為昊天,秋為旻天,冬為上天。”\CJKunderwave{毛詩}傳云:“尊而君之則稱皇天,元氣廣大則稱昊天,仁覆閔下則稱旻天,自上降監則稱上天,據遠視之蒼蒼然則稱蒼天。”\CJKunderwave{爾雅}四時異名,\CJKunderwave{詩}傳即隨事立稱。\CJKunderline{鄭玄}讀\CJKunderwave{爾雅}雲“春為昊天,夏為蒼天”,故駁\CJKunderwave{異義}云:“春氣博施,故以廣大言之。夏氣高明,故以遠言之。秋氣或生或殺,故以閔下言之。冬氣閉藏而清察,故以監下言之。皇天者尊而號之也。”六籍之中,諸稱天者以情所求言之耳,非必於其時稱之。然此言堯敬大四天,故以“廣大”言之。“星,四方中星”者,二十八宿,布在四方,隨天轉運,更互在南方,每月各有中者。\CJKunderwave{月令}每月昏旦,惟舉一星之中,若使每日視之,即諸宿每日昏旦莫不常中,中則人皆見之,故以中星表宿,“四方中星”總謂二十八宿也。或以\CJKunderwave{書傳}雲“主春者張,昏中,可以種穀。主夏者火,昏中,可以種黍。主秋者虛,昏中,可以種麥。主冬者昴,昏中,可以收斂。皆雲上告天子,下賦臣人。天子南面而視四方星之中,知人緩急,故曰敬授人時”,謂此“四方中星”如\CJKunderwave{書}傳之說。孔於虛昴諸星本無取中之事,用\CJKunderwave{書傳}為孔說非其旨矣。“辰,日月所會”者,昭七年\CJKunderwave{左傳}士文伯對晉侯之辭也。日行遲,月行疾,每月之朔月行及日而與之會,其必在宿。分二十八宿,是日月所會之處。辰,時也,集會有時,故謂之辰。“日月所會”與“四方中星”俱是二十八宿。舉其人目所見,以星言之。論其日月所會,以辰言之,其實一物,故星、辰共文。\CJKunderwave{益稷}稱古人之象,日月星辰共為一象,由其實同故也。日月與星,天之三光。四時變化,以此為政。故命羲和,令以算術推步,累歷其所行,法象其所在,具有分數節候,參差不等,敬記此天時以為歷而授人。此言星辰共為一物。\CJKunderwave{周禮·大宗伯}云:“實柴祀日月星辰。”\CJKunderline{鄭玄}雲“星謂五緯,辰謂日月所會十二次”者,以星、辰為二者。五緯與二十八宿俱是天星,天之神祇,禮無不祭,故\CJKunderline{鄭玄}隨事而分之。以此“敬授人時”無取五緯之義,故\CJKunderline{鄭玄}於此注亦以星、辰為一,觀文為說也。然則五星與日月皆別行,不與二十八宿同為不動也。 \par}

分命\CJKunderline{羲仲},宅嵎夷,曰暘谷\footnote{\textcolor{red}{宅,居}也。東表之地稱嵎夷。暘,明也。日出於谷而天下明,故稱暘谷。暘谷、嵎夷一也。\CJKunderline{羲仲}居治東方\textcolor{red}{之官}。嵎音隅,馬曰:“嵎,海嵎也。”夷,萊夷也。\CJKunderwave{尚書、·考靈耀}及\CJKunderwave{史記}作“禺銕”。暘音陽。谷,工木反,又音欲,下同。馬云:“暘谷,海嵎夷之地名。”“日出於谷”本或作“日出於陽穀”,“陽”衍字。}。寅賓出日,平秩東作\footnote{\textcolor{red}{寅,敬}。賓,導。秩,序也。歲起於東而始就耕,謂之東作。東方之官敬導出日,平均次序東作之事,以\textcolor{red}{務農}也。寅,徐以真反,又音夷,下同。賓如字,徐音殯,馬云:“從也。”出,尺遂反,又如字,注同。平如字,馬作蘋,普庚反,云:“使也。”下皆放此。秩如字。}。日中,星鳥,以殷仲春\footnote{\textcolor{red}{日中}謂春分之日。鳥,南方朱鳥七宿。殷,正也。春分之昏,鳥星畢見,以正仲春之氣節,轉以推季孟則\textcolor{red}{可知}。中,貞仲反,又如字。殷,於勤反,馬、鄭云:“中也。”宿音秀,下同。見,賢遍反,下同。}。厥民析,鳥獸孳尾\footnote{\textcolor{red}{冬寒}無事,併入室處。春事既起,丁壯就功。厥,其也。言其民老壯分析。乳化曰孳,交接\textcolor{red}{曰尾}。析,星曆反。孳音字。乳,儒付反。\CJKunderwave{說文}云:“人及鳥生子曰乳,獸曰產。”}。

{\noindent\zhuan\zihao{6}\fzbyks 傳“宅居”至“之官”。正義曰:“宅,居”,\CJKunderwave{釋言}文。\CJKunderwave{禹貢}青州云:“嵎夷既略。”青州在東界外之畔為表,故云“東表之地稱嵎夷”也。陰陽相對,陰暗而陽明也,故以“暘”為明。谷無陰陽之異,以日出於谷而天下皆明,故謂日出之處為“暘谷”。冬南夏北不常厥處,但日由空道,似行自谷,故以“谷”言之,非實有深谷而日從谷之出也。據日所出謂之“暘谷”,指其地名即稱“嵎夷”,故云“暘谷、嵎夷一也”。又解“居”者,居其官不居其地,故云“\CJKunderline{羲仲}居治東方之官”。此言“分命”者,上雲“乃命羲和”,總舉其目,就“乃命”之內分其職掌,使羲主春夏,和主秋冬,分一歲而別掌之,故言“分命”。就羲和之內又重分之,故於夏變言“申命”。既命仲而覆命叔,是其重命之也。所命無伯、季者,蓋時無伯、季,或有而不賢,則\CJKunderwave{外傳}稱“堯育重黎之後,不忘舊者,使復典之”,明仲叔能守舊業,故命之也。此羲和掌序天地,兼知人事,因主四時而分主四方,故舉東表之地,以明所舉之域。地東舉嵎夷之名,明分三方皆宜有地名,此為其始,故特詳舉其文。\CJKunderline{羲仲}居治東方之官,居在帝都而遙統領之。王肅雲“皆居京師而統之,亦有時述職”,是其事也。以春位在東,因治於東方,其實本主四方春政,故於\CJKunderline{和仲}之下云:“此居治西方之官,掌秋天之政。”明此,掌春天之政,孔以經事詳,故就下文而互發之。 \par}

{\noindent\zhuan\zihao{6}\fzbyks 傳“寅敬”至“務農”。正義曰:“寅,敬”,\CJKunderwave{釋詁}文。賓者主行導引,故“賓”為導也。\CJKunderwave{釋詁}以“秩”為常,常即次第有序,故“秩”為序也。一歲之事,在東則耕作,在南則化育,在西則成熟,在北則改易,故以方名配歲事為文,言順天時氣以勸課人務也。春則生物,秋則成物。日之出也,物始生長,人當順其生長,致力耕耘。日之入也,物皆成熟,人當順其成熟,致力收斂。東方之官當恭敬導引日出,平秩東作之事,使人耕耘。西方之官當恭敬從送日入,平秩西成之事,使人收斂。日之出入,自是其常,但由日出入,故物有生成。雖氣能生物,而非人不就。勤於耕稼,是導引之。勤於收藏,是從送之。冬夏之文無此類者,南北二方非日所出入,“平秩南訛”亦是導日之事,“平在朔易”亦是送日之事。依此春秋而共為賓餞,故冬夏二時無此一句。勸課下民,皆使致力,是敬導之。平均次序,即是授人,田裡各有疆埸,是平均之也。耕種收斂使不失其次序,王者以農為重,經主於農事。“寅賓出日”為“平秩”設文,故並解之也。言“敬導出日”者,正謂平秩次序東作之事以務農也。鄭以“作”為生,計秋言西成,春宜言東生。但四時之功皆須作力,不可不言力作,直說生成,明此以歲事初起,時言“東作”,以見四時亦當力作,故孔以耕作解之。\CJKunderline{鄭玄}云:“寅賓出日,謂春分朝日。”又以“寅餞納日,謂秋分夕日”也。 \par}

{\noindent\zhuan\zihao{6}\fzbyks 傳“日中”至“可知”。正義曰:其仲春、仲秋、冬至、夏至,馬融云:“古制刻漏晝夜百刻。晝長六十刻,夜短四十刻。晝短四十刻,夜長六十刻。晝中五十刻,夜亦五十刻。”融之此言據日出見為說。天之晝夜以日出入為分,人之晝夜以昏明為限。日未出前二刻半為明,日入後二刻半為昏,損夜五刻以裨於晝,則晝多於夜,覆校五刻。古今歷術與太史所候皆云,夏至之晝六十五刻,夜三十五刻。冬至之晝四十五刻,夜五十五刻。春分秋分之晝五十五刻,夜四十五刻。此其不易之法也。然今太史細候之法,則校常法半刻也。從春分至於夏至,晝暫長,增九刻半。夏至至於秋分,所減亦如之。從秋分至於冬至,晝暫短,減十刻半。從冬至至於春分,其增亦如之。又於每氣之間增減刻數,有多有少,不可通而為率。漢初未能審知,率九日增減一刻,和帝時待詔霍融始請改之。鄭注\CJKunderwave{書緯·考靈曜}仍雲“九日增減一刻”,猶尚未覺悟也。鄭注此云:“日長者日見之漏五十五刻,日短者日見之漏四十五刻。”與歷不同。故王肅難云:“知日見之漏減晝漏五刻,不意馬融為傳已減之矣。因馬融所減而又減之,故日長為五十五刻,因以冬至反之,取其夏至夜刻,以為冬至晝短,此其所以誤耳。”“鳥,南方朱鳥七宿”者,在天成象,星作鳥形。\CJKunderwave{曲禮}說軍陳象天之行,“前朱雀,後玄武,左青龍,右白虎”。“雀”即鳥也。“武”謂龜甲捍禦,故變文“玄武”焉。是天星有龍虎鳥龜之形也。四方皆有七宿,各成一形。東方成龍形,西方成虎形,皆南首而北尾。南方成鳥形,北方成龜形,皆西首而東尾。以南方之宿象鳥,故言鳥謂朱鳥七宿也。此經舉宿,為文不類。春言“星鳥”,總舉七宿。夏言“星火”,獨指房、心。虛、昴惟舉一宿。文不同者,互相通也。\CJKunderwave{釋言}以“殷”為中,中、正義同,故“殷”為正也。此經冬夏言“正”,春秋言“殷”者,其義同。春分之昏,觀鳥星畢見,以正仲春之氣節,計仲春日在奎、婁而入於酉地,則初昏之時井、鬼在午,柳、星、張在巳,軫、翼在辰,是朱鳥七宿皆得見也。春有三月,此經直雲“仲春”,故傳辨之云,既正仲春,轉以推季孟之月,則事亦可知也。天道左旋,日體右行,故星見之方與四時相逆。春則南方見,夏則東方見,秋則北方見,冬則西方見,此則勢自當然。而\CJKunderwave{書緯}為文生說,言“春夏相與交,秋冬相與互,謂之母成子,子助母”,斯假妄之談耳。馬融、\CJKunderline{鄭玄}以為“星鳥、星火謂正在南方。春分之昏七星中,仲夏之昏心星中,秋分之昏虛星中,冬至之昏昴星中,皆舉正中之星,不為一方盡”,見此其與孔異也。至於舉仲月以統一時,亦與孔同。王肅亦以星鳥之屬為昏中之星,其要異者以所宅為孟月,日中、日永為仲月,星鳥、星火為季月,“以殷”、“以正”皆總三時之月,讀“仲”為中,言各正三月之中氣也。以馬融、\CJKunderline{鄭玄}之言,不合天象,星火之屬仲月未中,故為每時皆歷陳三月,言日以正仲春,以正春之三月中氣。若正春之三月中,當言“以正春中”,不應言“以正仲春”。王氏之說非文勢也。\CJKunderline{孔氏}直取“畢見”,稍為迂闊,比諸王、馬,於理最優。 \par}

{\noindent\zhuan\zihao{6}\fzbyks 傳“冬寒”至“曰尾”。正義曰:“厥,其”,\CJKunderwave{釋言}文。其人老弱在室,丁壯適野,是老壯分析也。孳、字,古今同耳。字訓愛也,產生為乳,胎孕為化,孕產必愛之,故乳化曰“孳”。鳥獸皆以尾交接,故交接曰“尾”。計當先尾後孳,隨便言之。 \par}

申命\CJKunderline{羲叔},宅南交\footnote{\textcolor{red}{申,重}也。南交言夏與春交。舉一隅以見之。此居治南方\textcolor{red}{之官}。重,直用反。}。平秩南訛,敬致\footnote{\textcolor{red}{訛,化}也。掌夏之官平敘南方化育之事,敬行其教,以致其功。四時同之,亦舉\textcolor{red}{一隅}。訛,五和反。}。日永,星火,以正仲夏\footnote{\textcolor{red}{永,長}也,謂夏至之日。火,蒼龍之中星,舉中則七星見\textcolor{red}{可知}。以正仲夏之氣節,季孟亦可知。}。厥民因,鳥獸希革\footnote{\textcolor{red}{因,謂}老弱因就在田之丁壯以助農也。夏時鳥獸毛羽希少改易。\textcolor{red}{革,改}也。}。

{\noindent\zhuan\zihao{6}\fzbyks 傳“申重”至“之官”。正義曰:“申、重”,\CJKunderwave{釋詁}文。此官既主四時,亦主方面,經言“南交”,謂南方與東方交,傳言“夏與春交”,見其時、方皆掌之。春盡之日與立夏之初,時相交也,東方之南,南方之東,位相交也,言\CJKunderline{羲叔}所掌與\CJKunderline{羲仲}相交際也。四時皆舉仲月之候,嫌其不統季孟,於此言“交”,明四時皆然,故傳言“舉一隅以見之”。春上無冬,不得見其交接,至是夏與春交,故此言之。 \par}

{\noindent\zhuan\zihao{6}\fzbyks 傳“訛化”至“一隅”。正義曰:“訛,化”,\CJKunderwave{釋言}文。禾苗秀穗,化成子實,亦胎生乳化之類,故“掌夏之官平序南方化育之事”,謂勸課民耘耨,使苗得秀實。“敬行其教,以致其功”,謂敬行平秩之教,以致化育之功。農功歲終乃畢,敬行四時皆同,於此言之,見四時皆然,故云“亦舉一隅”也。夏日農功尤急,故就此言之。 \par}

{\noindent\zhuan\zihao{6}\fzbyks 傳“永長”至“可知”。正義曰:“永,長”,\CJKunderwave{釋詁}文。夏至之日日最長,故知謂夏至之日。計七宿房在其中,但房、心連體,心統其名。\CJKunderwave{左傳}言“火中”、“火見”,\CJKunderwave{詩}稱“七月流火”,皆指房、心為火,故曰“火,蒼龍之中星”。特舉一星,與鳥不類,故云“舉中則七星見可知”。計仲夏日在東井而入於酉地,即初昏之時角、亢在午,氐、房、心在巳,尾、箕在辰,是東方七宿皆得見也。 \par}

{\noindent\zhuan\zihao{6}\fzbyks 傳“因謂”至“革改”。正義曰:春既分析在外,今日因往就之,故言“因,謂老弱因就在田之丁壯以務農”也。鳥獸冬毛最多,春猶未脫,故至夏始毛羽希少,改易往前。“革”謂變革,故為改也。傳之訓字,或先或後,無義例也。 \par}

分命\CJKunderline{和仲},宅西,曰昧谷\footnote{\textcolor{red}{昧,冥}也。日入於谷而天下冥,故曰昧谷。昧谷曰西,則嵎夷東可知。此居治西方之官,掌秋天\textcolor{red}{之政}也。昧,武內反。冥,莫定反。}。寅餞納日,平秩西成\footnote{\textcolor{red}{餞,送}也。日出言導,日入言送,因事之宜。秋,西方,萬物成。平序其政,助\textcolor{red}{成物}。餞,賤衍反。馬云:“滅也,滅猶沒也。”}。宵中,星虛,以殷仲秋\footnote{\textcolor{red}{宵,夜}也。春言日,秋言夜,互相備。虛,玄武之中星,亦言七星皆以秋分日見,以正\textcolor{red}{三秋}。}。厥民夷,鳥獸毛\xpinyin*{毨}\footnote{\textcolor{red}{夷,平}也,老壯在田與夏平也。毨,理也,毛更生\textcolor{red}{整理}。“毛毨”,下先典反。\CJKunderwave{說文}云:“仲秋鳥獸毛盛,可選取以為器用也。”}。

{\noindent\zhuan\zihao{6}\fzbyks 傳“昧冥”至“之政”。正義曰:\CJKunderwave{釋言}云:“晦,冥也。”冥是暗,故“昧”為冥也。“谷”者日所行之道,日入於谷而天下皆冥,故謂日入之處為“昧谷”,非實有谷而日入也。此經春秋相對,春不言“東”,但舉昧谷曰“西”,則嵎夷東可知。然則東言“嵎夷”,則西亦有地明矣,闕其文所以互見之。傳於春言“東方之官”,不言“掌春”,夏言“掌夏之官”,不言“南方”,此言“居治西方之官,掌秋天之政”,互文明四時皆同。 \par}

{\noindent\zhuan\zihao{6}\fzbyks 傳“餞送”至“成物”。正義曰:送行飲酒謂之餞,故“餞”為送也。導者引前之言,送者從後之稱,因其欲出,導而引之,因其欲入,從而送之,是其因事之宜而立此文也。秋位在西,於時萬物成熟,平序其秋天之政,未成則耘耨,既熟則收斂,助天成物,以此而從送入日也。納、入義同,故傳以入解“納”。 \par}

{\noindent\zhuan\zihao{6}\fzbyks 傳“宵夜”至“三秋”。正義曰:“宵,夜”,\CJKunderwave{釋言}文。舍人曰:“宵,陽氣消也”。三時皆言日,惟秋言夜,故傳辨之云:“春言日,秋言夜,互相備”也,互著明也。明日中宵亦中,宵中日亦中,因此而推之,足知日永則宵短,日短則宵長,皆以此而備知也。正於此時變文者,以春之與秋日夜皆等,春言“出日”即以“日”言之,秋雲“納日”即以“夜”言之,亦事之宜也。北方七宿則虛為中,故虛為玄武之中星。計仲秋日在角、亢而入於酉地,初昏之時鬥、牛在午,女、虛、危在巳,室、壁在辰,舉虛中星言之,亦言七星皆以秋分之日昏時並見,以正秋之三月。 \par}

{\noindent\zhuan\zihao{6}\fzbyks 傳“夷平”至“整理”。正義曰:\CJKunderwave{釋詁}云:“夷、平,易也。”俱訓為易,是“夷”得為平。秋禾未熟,農事猶煩,故“老壯在田與夏平”也。“毨”者,毛羽美悅之狀,故為理也。夏時毛羽希少,今則毛羽復生,夏改而少,秋更生多,故言“更生整理”。 \par}

申命\CJKunderline{和叔},宅朔方,曰幽都。平在朔易\footnote{\textcolor{red}{北稱}朔,亦稱方,言一方則三方見矣。北稱幽都,南稱明從可知也。都謂所聚也。易謂歲改易北方,平均在察其政,以順天常。上總言羲和敬順昊天,此分別仲叔,各有\textcolor{red}{所掌}。別音彼列反,下同。}。日短,星昴,以正仲冬\footnote{日短,冬至之日。昴,白虎之中星,亦以七星並見,以正冬之三節。}。厥民\xpinyin{隩}{yu4},鳥獸\xpinyin*{氄}毛\footnote{\textcolor{red}{隩,室}也。民改歲入此室處,以闢風寒。鳥獸皆生而毳細毛以自\textcolor{red}{溫焉}。隩,於六反,馬云:“暖也。”氄,如勇反,徐又音而充反,馬云:“溫柔貌。”闢音避。耎,如兗反,本或作濡,音儒。毳,尺銳反。}。

{\noindent\zhuan\zihao{6}\fzbyks 傳“北稱”至“所掌”。正義曰:\CJKunderwave{釋訓}云:“朔,北方也。”舍人曰:“朔,盡也。北方萬物盡,故言朔也。”李巡曰:“萬物盡於北方,蘇而復生,故言北方。”是“北稱朔”也。羲和主四方之官,四時皆應言“方”,於此言“方”者,即三方皆見矣。春為歲首,故舉地名;夏與春交,故言“南交”;秋言“西”以見嵎夷當為東,冬言“方”以見三時皆有方。古史要約,其文互相發見也。“幽”之與“明”文恆相對,北既稱“幽”,則南當稱“明”,從此可知,故於夏無文。經冬言“幽都”,夏當雲“明都”,傳不言“都”者,從可知也。鄭云:“夏不言‘曰明都’三字,摩滅也。”伏生所誦與壁中舊本並無此字,非摩滅也。王肅以“夏無‘明都’,避‘敬致’,然即‘幽’足見‘明’,闕文相避”,如肅之言,義可通矣。“都謂所聚”者,總言此方是萬物所聚之處,非指都邑聚居也。“易謂歲改易於北方”者,人則三時在野,冬入隩室,物則三時生長,冬入囷倉,是人之與物皆改易也。王肅云:“改易者,謹約蓋藏,循行積聚。”引\CJKunderwave{詩}“嗟我婦子,曰為改歲,入此室處”。王肅言人物皆易,孔意亦當然也。\CJKunderwave{釋詁}云:“在,察也。舍人曰:“在,見物之察。”是“在”為察義,故言“平均在察其政,以順天常”。以“在察”須與“平均”連言,不復訓“在”為察,故\CJKunderwave{舜典}之傳別更訓之。三時皆言“平秩”,此獨言“平在”者,以三時乃役力田野,當次序之,冬則物皆藏入,須省察之,故異其文。秋日物成就,故傳言“助成物”,冬日蓋藏,天之常道,故言“順天常”,因明“東作”、“南訛”亦是助生物,類常道也。上總言羲和敬順昊天,此分別仲叔各有所掌,明此四時之節,即順天之政,實恐人以“敬順昊天”直是曆象日月,嫌仲叔所掌非順天之事,故重明之。 \par}

{\noindent\zhuan\zihao{6}\fzbyks 傳“隩室”至“溫焉”。正義曰:\CJKunderwave{釋宮}云:“西南隅謂之隩。”孫炎云:“室中隱隩之處也。”隩是室內之名,故以“隩”為室也。物生皆盡,野功咸畢,是歲改矣。以天氣改歲,故入此室處,以避風寒。天氣既至,故鳥獸皆生耎毳細毛以自溫焉。經言“氄毛”,謂附肉細毛,故以“耎毳”解之。 \par}

帝曰:“諮!汝羲暨和。\xpinyin{期}{ji1}三百有六旬有六日,以閏月定四時,成歲\footnote{\textcolor{red}{諮,嗟}。暨,與也。匝四時曰期。一歲十二月,月三十日,正三百六十日;除小月六,為六日,是為一歲有餘十二日;未盈三歲足得一月,則置閏焉,以定四時之氣節,成一歲之\textcolor{red}{曆象}。暨,其器反。期,居其反,下同。旬,似遵反,十日為旬。匝,子合反。}。允釐百工,庶績咸熙。”\footnote{\textcolor{red}{允,信}。釐,治。工,官。績,功。咸,皆。熙,廣也,興也。言定四時成歲歷,以告時授事,則能信治百官,眾功皆廣,嘆\textcolor{red}{其善}。釐,力之反。熙,許其反,興也。}

{\noindent\zhuan\zihao{6}\fzbyks 傳“諮嗟”至“曆象”。正義曰:“諮,嗟”、“暨,與”,皆\CJKunderwave{釋詁}文也。“迎四時曰期”,“期”即“迎”也。故王肅云:“期,四時是也。”然古時真歷遭戰國及秦而亡,漢存六歷雖詳於五紀之論,皆秦漢之際假託為之,實不得正要有梗概之言。周天三百六十五度四分度之一。而日日行一度,則一期三百六十五日四分日之一。今\CJKunderwave{考靈曜}、\CJKunderwave{幹鑿度}諸緯皆然。此言三百六十六日者,王肅云:“四分日之一又入六日之內,舉全數以言之,故云三百六十六日也。”傳又解所以須置閏之意,皆據大率以言之,云:“一歲十二月,月三十日,正三百六十日也;除小月六,又為六日。”今經雲三百六十六日,故云“餘十二日”,不成期。以一月不整三十日,今一年餘十二日,故未至盈滿三歲足得一月,則置閏也。以時分於歲,故云“氣節”,謂二十四氣,時月之節。歲總於時,故云“曆象日月星辰,敬授人時”,以相配成也。六歷、諸緯與\CJKunderwave{周髀}皆云,日行一度,月行十三度十九分度之七,為每月二十九日過半。日之於法,分為日九百四十分日之四百九十九,即月有二十九日半強。為十二月,六大之外有日分三百四十八,是除小月無六日,又大歲三百六十六日,小歲三百五十五日,則一歲所餘無十二日。今言“十二日”者,皆以大率據整而計之,其實一歲所餘正十一日弱也。以為十九年七閏,十九年年十一日則二百九日,其七月四大三小猶二百七日,況無四大乎?為每年十一日弱分明矣。所以弱者,以四分日之一於九百四十分,則一分為二百三十五分,少於小月餘分三百四十八。以二百三十五減三百四十八,不盡一百一十三,是四分日之一餘矣。皆以五日為率,其小月雖為歲日殘分所減,猶餘一百一十三,則實餘尚無六日。就六日抽一日為九百四十分減其一百一十三分,不盡八百二十七分。以不抽者五日並三百六十日外之五日為十日,其餘九百四十分日之八百二十七,為每歲之實餘。今十九年年,十日得整日一百九十。又以十九乘八百二十七分,得一萬五千七百一十三。以日法九百四十除之,得十六日。以並一百九十日為二百六日,不盡六百七十三分為日餘。今為閏月得七,每月二十九日,七月為二百三日。又每四百九十九分以七乘之得三千四百九十三,以日法九百四十分除之得三日。以二百三日亦為二百六日,不盡亦六百七十三為日餘,亦相當矣。所以無閏時不定,歲不成者,若以閏無,三年差一月,則以正月為二月,每月皆差;九年差三月,即以春為夏;若十七年差六月,即四時相反;時何由定,歲何得成乎?故須置閏以定四時。故\CJKunderwave{左傳}雲“履端於始,序則不愆;舉正於中,民則不惑;歸餘於終,事則不悖”是也,先王以重閏焉。王肅云:“鬥之所建,是為中氣,日月所在。鬥指兩辰之間,無中氣,故以為閏也。” \par}

{\noindent\zhuan\zihao{6}\fzbyks 傳“允信”至“其善”。正義曰:\CJKunderwave{釋訓}云:“鬼之為言歸也。”\CJKunderwave{鄉飲酒義}云:“春之為言蠢也。”然則\CJKunderwave{釋訓}之例有以聲相近而訓其義者,“釐,治”,“工,官”,皆以聲近為訓,他皆放此類也。“績,功”、“咸、皆”,\CJKunderwave{釋詁}文。“熙、廣”,\CJKunderwave{周語}文。此經文義承“成歲”之下,傳以文勢次之,言定歷授事能使眾功皆廣。“嘆其善”謂帝嘆羲和之功也。 \par}

{\noindent\shu\zihao{5}\fzkt “乃命”至“咸熙”。正義曰:上言能明俊德,又述能明之事,堯之聖德美政如上所陳。但聖不必獨理,必須賢輔。堯以須臣之故,乃命有俊明之人羲氏、和氏敬順昊天之命,歷此法象。其日之甲乙,月之大小,昏明遞中之星,日月所會之辰,定其所行之數,以為一歲之歷。乃依此歷,敬授下人以天時之早晚。其總為一歲之歷,其分有四時之異,既舉總目,更別序之。堯於羲和之內,乃分別命其羲氏而字仲者,令居治東方嵎夷之地也。日所出處名曰暘明之谷,於此處所主之職,使\CJKunderline{羲仲}主治之。既主東方之事,而日出於東方,令此\CJKunderline{羲仲}恭敬導引將出之日,平均次序東方耕作之事,使彼下民務勤種植。於日晝夜中分,刻漏正等,天星朱鳥南方七宿合昏畢見,以此天之時候調正仲春之氣節。此時農事已起,不居室內,其時之民宜分析適野。老弱居室,丁壯就功。於時鳥獸皆孕胎卵,孳尾匹合。又就所分羲氏之內,重命其羲氏而字叔者,使之居治南方之職,又於天分南方與東交,立夏以至立秋時之事,皆主之。均平次序南方化育之事,敬行其教,以致其功,於日正長,晝漏最多,天星大火東方七宿合昏畢見,以此天時之候調正仲夏之氣節。於時苗稼已殖,農事尤煩,其時之民,老弱因一丁壯就在田野。於時鳥獸羽毛希少,變改寒時。又分命和氏而字仲者,居治西方日所入處,名曰昧冥之谷。於此處所主之職,使\CJKunderline{和仲}主治之。既主西方之事,而日入在於西方,令此\CJKunderline{和仲}恭敬從送既入之日,平均次序西方成物之事,使彼下民務勤收斂。於晝夜中分,漏刻正等,天星之虛北方七宿合昏畢見,以此天時之候調正仲秋之氣節。於時禾苗秀實,農事未閒,其時之民與夏齊平,盡在田野。於時鳥獸毛羽更生,已稍整治。又重命和氏而字叔者,令居治北方名曰幽都之地,於此處所主之職,使\CJKunderline{和叔}主治之。平均視察北方歲改之事。於日正短,晝漏最少,天星之昴西方七宿合昏畢見,以此天時之候調正仲冬之氣節。於時禾稼已入,農事閒暇,其時之人皆處深隩之室,鳥獸皆生耎毳細毛以自溫暖。此是羲和敬天授人之實事也。羲和所掌如是,故\CJKunderline{帝堯}乃述而嘆之曰:“諮嗟!汝\CJKunderline{羲仲}、\CJKunderline{羲叔}與\CJKunderline{和仲}、\CJKunderline{和叔}。一期之間三百有六旬有六日,分為十二月,則餘日不盡,令氣朔參差,若以閏月補闕,令氣朔得正定四時之氣節,成一歲之曆象,是汝之美可嘆也。又以此歲歷告時授事,信能和治百官,使之眾功皆廣也。”嘆美羲和能敬天之節,眾功皆廣,則是風俗大和。 \par}

\textcolor{red}{帝曰}:“疇諮若時?登庸。”\footnote{\textcolor{red}{疇,誰}。庸,用也。誰能咸熙庶績,順是事者,將登\textcolor{red}{用之}。疇,直由反。}

{\noindent\zhuan\zihao{6}\fzbyks 傳“疇誰”至“用之”。正義曰:“疇,誰”,\CJKunderwave{釋詁}文。“庸”聲近“用”,故為用也。馬融以“羲和為卿官堯之末年,皆以老死,庶績多闕。故求賢順四時之職,欲用以代羲和”。孔於下傳云:“四嶽,即上羲和之四子。”帝就羲和求賢,則所求者別代他官,不代羲氏、和氏。孔以羲和掌天地之官,正在敬順昊天,告時授事而已,其施政者乃是百官之事,非復羲和之職。但羲和告時授事,流行百官,使百官庶績咸熙,今雲“咸熙庶績,順是事者”,指謂求代百官之闕,非求代羲和也。此經文承“庶績”之下而言“順是事者”,故孔以文勢次之,此言“誰能咸熙庶績,順是事者,將登用之”,蓋求卿士用任也。計堯即位至洪水之時六十餘年,百官有闕,皆應求代。求得賢者,則史亦不錄。不當帝意,乃始錄之,為求舜張本。故惟帝求一人,\CJKunderline{放齊}以一人對之,非六十餘年止求一人也。堯以聖德在位,庶績咸熙,蓋應久矣。此繼“咸熙”之下,非知早晚求之,史自歷序其事,不必與治水同時也。計四嶽職掌天地,當是朝臣之首。下文求治水者,帝“諮四嶽”,此不言“諮四嶽”者,帝求賢者固當博訪朝臣,但史以有嶽對者言“諮!四嶽”,此不言“諮”者,但此無嶽對,故不言耳。 \par}

\CJKunderline{放齊}曰:“胤子\CJKunderline{朱}啟明。”帝曰:“吁!\xpinyin*{嚚}訟,可乎?”\footnote{\textcolor{red}{\CJKunderline{放齊}},臣名。胤,國。子,爵。朱,名。啟,開也。吁,疑怪之辭。言不忠信為嚚,又好爭訟,可乎!言\textcolor{red}{不可}。放,方往反,注同。胤,引信反,馬云:“嗣也。”吁,況於反,徐往付反,一音於。嚚,魚巾反。訟,才用反,馬本作庸。好,呼報反,下注同。爭,鬥也。}

{\noindent\zhuan\zihao{6}\fzbyks 傳“\CJKunderline{放齊}”至“不可”。正義曰:以\CJKunderline{放齊}舉人對帝,故知臣名,為名為字,不可得知。傳言“名”者,辯此是為臣之名號耳,未必是臣之名也。夏王\CJKunderline{仲康}之時,\CJKunderline{胤侯}命掌六師,\CJKunderwave{顧命}陳寶有胤之舞衣,故知古有胤國。“胤”既是國,自然“子”為爵,“朱”為名也。馬融、\CJKunderline{鄭玄}以為“帝之胤子曰朱也”。求官而薦太子,太子下愚以為啟明,揆之人情,必不然矣。“啟”之為開,書傳通訓,言此人心志開解而明達。“吁”者必有所嫌而為此聲,故以為“疑怪之辭”。僖二十四年\CJKunderwave{左傳}曰:“口不道忠信之言為嚚。”是“言不忠信為嚚”也。其人心既頑嚚,又好爭訟,此實不可,而帝雲“可乎”,故吁聲而反之。“可乎”,言不可也。唐堯聖明之主,應任賢哲,\CJKunderline{放齊}聖朝之臣,當非庸品,人有善惡,無容不知,稱“嚚訟”以為“啟明”,舉愚臣以對聖帝,何哉?將以知人不易,人不易知,密意深心,固難照察。胤子矯飾容貌,但以惑人,\CJKunderline{放齊}內少鑑明,未能圓備,謂其實可任用,故承意舉之。以\CJKunderline{帝堯}之聖,乃知其嚚訟之事,\CJKunderline{放齊}所不知也。\CJKunderline{驩兜}薦舉\CJKunderline{共工},以為比周之惡,謂之四凶,投之遠裔。\CJKunderline{放齊}舉胤子,不為兇人者,胤子雖有嚚訟之失,不至滔天之罪,\CJKunderline{放齊}謂之實賢,非是苟為阿比。\CJKunderline{驩兜}則志不在公,私相朋黨,\CJKunderline{共工}行背其言,心反於貌,其罪並深,俱被流放,其意異於\CJKunderline{放齊}舉胤子故也。 \par}

帝曰:“疇諮若予採?”\footnote{\textcolor{red}{採,事}也。復求誰能順我\textcolor{red}{事者}。予音餘,又羊汝反。採,七在反,馬云:“官也。”復,扶又反。}

{\noindent\zhuan\zihao{6}\fzbyks 傳“採事”至“事者”。正義曰:“採,事”,\CJKunderwave{釋詁}文。上已求順時,不得其人,故復求順我事者。順時順事其義一也。史以上承“庶績”之下,故言順時,謂順是庶績之事,此不可復同前文,故變言順我帝事,其意亦如前經,當求卿士之任也。順我事之下亦宜有“登用”之言,上文已具,故於此略之。 \par}

\xpinyin*{\CJKunderline{驩}}\CJKunderline{兜}曰:“都!\CJKunderline{共工}方鳩\xpinyin{僝}{zhuan4}功。”\footnote{\textcolor{red}{\CJKunderline{驩兜}},臣名。都,於,嘆美之辭。\CJKunderline{共工},官稱。鳩,聚,或曰通救。僝,見也。嘆\CJKunderline{共工}能方方聚見\textcolor{red}{其功}。驩,呼端反。兜,丁侯反。共音恭,注同。僝,仕簡反,徐音撰,馬云:“具也。”於音烏。稱,尺證反。}

{\noindent\zhuan\zihao{6}\fzbyks 傳“\CJKunderline{驩兜}”至“其功”。正義曰:\CJKunderline{驩兜}亦舉人對帝,故知臣名。“都,於”,\CJKunderwave{釋詁}文。“於”即“嗚”字,嘆之辭也。將言\CJKunderline{共工}之善,故先嘆美之。\CJKunderwave{舜典}命垂作\CJKunderline{共工},知\CJKunderline{共工}是官稱。鄭以為“其人名氏未聞。先祖居此官。故以官氏也”。計稱人對帝。不應舉先世官名.孔直雲“官稱”,則其人於時居此官也。時見居官,則是已被任用,復舉之者,帝求順事之人,欲置之上位,以為大臣,所欲尊於\CJKunderline{共工},故舉之也。“鳩,聚”,\CJKunderwave{釋詁}文。僝然,見之狀,故為見。“嘆\CJKunderline{共工}能方方聚見其功”,謂每於所在之方,皆能聚集善事,以見其功,言可用也。若能\CJKunderline{共工}實有見功,則是可任用之人,帝言其庸違滔天不可任者,\CJKunderline{共工}言是行非,貌恭心很,取人之功以為己功,其人非無見功,但功非己有。\CJKunderwave{左傳}說\CJKunderline{驩兜}雲“醜類惡物”,是與比周;“天下之人謂之渾敦”,言\CJKunderline{驩兜}以\CJKunderline{共工}比周,妄相薦舉,知所言見功非其實功也。 \par}

帝曰:“吁!靜言庸違,象恭滔天。”\footnote{\textcolor{red}{靜,謀}。滔,漫也。言\CJKunderline{共工}自為謀言,起用行事而違背之,貌象恭敬而心傲很,若漫天。言不\textcolor{red}{可用}。滔,吐刀反。漫,末旦反,下同,又末寒反。背音佩。傲,五報反,下同。很,恨懇反。}

{\noindent\zhuan\zihao{6}\fzbyks 傳“靜謀”至“可用”。正義曰:“靜,謀”,\CJKunderwave{釋詁}文。滔者,漫浸之名,浸必漫其上,故“滔”為漫也。\CJKunderline{共工}險偽之人,自為謀慮之言皆合於道,及起用行事而背違之,言其語是而行非也。貌象恭敬而心傲很,其侮上陵下,若水漫天,言貌恭而心很也。行與言違,貌恭心反,乃是大佞之人,不可任用也。明君聖主莫先於堯,求賢審官王政所急,乃有\CJKunderline{放齊}之不識是非,\CJKunderline{驩兜}之朋黨惡物,\CJKunderline{共工}之巧言令色,崇伯之敗善亂常,聖人之朝不才總萃,雖曰難之,何其甚也!此等諸人,才實中品,亦雖行有不善,未為大惡,故能仕於聖代,致位大官。以\CJKunderline{帝堯}之末,洪水為災,欲責非常之功,非復常人所及,自非聖舜登庸,\CJKunderline{大禹}致力,則滔天之害未或可平。以舜\CJKunderline{禹}之成功,見此徒之多罪。勳業既謝,愆釁自生,為聖所誅,其咎益大。且虞史欲盛彰舜德,歸過前人,\CJKunderwave{春秋}史克以宣公比堯,辭頗增甚,知此等並非下愚,未有大惡。其為不善,惟帝所知,將言求舜,以見帝之知人耳。 \par}

帝曰:“諮!四嶽,\footnote{四嶽,即上羲和之四子,分掌四嶽之諸侯,故稱焉。}\xpinyin{湯}{shang1}湯洪水方割,\footnote{湯湯,流貌。洪,大。割,害也。言大水方方為害。湯音傷。洪音戶工反。}蕩蕩懷山襄陵,浩浩滔天。\footnote{蕩蕩,言之奔突有所滌除。懷,包。襄,上也。包山上陵,浩浩盛大,若漫天。浩,胡老反。滌,大曆反。上,時掌反。}下民其諮,有能俾乂?”\footnote{俾,使。乂,治也。言民諮嗟憂愁,病水困苦,故問四嶽,有能治者將使之。俾,必爾反。}

{\noindent\zhuan\zihao{6}\fzbyks 傳“四嶽”至“稱焉”。正義曰:上列羲和所掌雲宅嵎夷、朔方,言四子居治四方,主於外事。嶽者,四方之大山。今王朝大臣皆號稱“四嶽”,是與羲和所掌其事為一,以此知“四嶽,即上羲和之四子”也。又解謂之嶽者,以其“分掌四嶽之諸侯,故稱焉”。\CJKunderwave{舜典}稱“巡守至於岱宗,肆覲東後”,\CJKunderwave{周官}說巡守之禮云,諸侯各朝於方岳之下,是四方諸侯分屬四嶽也。計堯在位六十餘年,乃命羲和蓋應早矣。若使成人見命,至此近將百歲,故馬、鄭以為羲和皆死。孔以為四嶽即是羲和至今仍得在者。以羲和世掌天地,自當父子相承,不必仲叔之身皆悉在也。\CJKunderwave{書傳}雖出自伏生,其常聞諸先達,虞傳雖說\CJKunderwave{舜典}之四嶽尚有羲伯、和伯,是仲叔子孫世掌嶽事也。 \par}

{\noindent\zhuan\zihao{6}\fzbyks 傳“湯湯”至“為害”。正義曰:湯湯,波動之狀,故為“流貌”。“洪,大”,\CJKunderwave{釋詁}文。刀害為割,故“割”為害也。“言大水方方為害”謂其遍害四方也。 \par}

{\noindent\zhuan\zihao{6}\fzbyks 傳“蕩蕩”至“漫天”。正義曰:蕩蕩,廣平之貌,“言水勢奔突有所滌除”,謂平地之水,除地上之物,為水漂流,無所復見,蕩然惟有水耳。懷,藏,包裹之義,故“懷”為包也。\CJKunderwave{釋言}以“襄”為駕,駕乘牛馬皆車在其上,故“襄”為上也。“包山”謂繞其傍,“上陵”謂乘其上,平地已皆蕩蕩,又復繞山上陵,故為盛大之勢,總言浩浩盛大若漫天然也。天者無上之物,漫者加陵之辭,甚其盛大,故云“若漫天”也。 \par}

{\noindent\zhuan\zihao{6}\fzbyks 傳“俾,使”、“乂,治也”。正義曰:“俾,使”、“乂,治”,\CJKunderwave{釋詁}文。 \par}

僉曰:“於,\CJKunderline{鯀}哉!”\footnote{僉,皆也。\CJKunderline{鯀},崇伯之名。朝臣舉之。僉,七廉反,又七劍反。於音烏。\CJKunderline{鯀},故本反,馬云:“\CJKunderline{禹}父也。朝,直遙反。}

{\noindent\zhuan\zihao{6}\fzbyks 傳“僉皆”至“舉之”。正義曰:“僉,皆”,\CJKunderwave{釋詁}文。\CJKunderwave{周語}雲“有崇伯\CJKunderline{鯀}”,即\CJKunderline{鯀}是崇君;伯,爵;故云“\CJKunderline{鯀},崇伯之名”。帝以嶽為朝臣之首,故特言四嶽,其實求能治者,普問朝臣,不言嶽對而云皆曰,乃眾人舉之,非獨四嶽,故言“朝臣舉之”。 \par}

帝曰:“吁,咈哉!方命圮族。”\footnote{凡言“吁”者皆非帝意。咈,戾。圮,毀。族,類也。言\CJKunderline{鯀}性很戾,好比方名,命而行事,輒毀敗善類。咈,扶弗反,忿戾也。方如字,馬云:“方,放也。”徐云:“鄭、王音放。”圮音皮美反。戾音力計反。}

{\noindent\zhuan\zihao{6}\fzbyks 傳“凡言”至“善類”。正義曰:自上以來三經求人,所舉者帝言其惡,而辭皆稱“吁”,故知凡言“吁”者皆非帝之所當意也。“咈”者相乖詭之意,故為戾也。“圮,毀”,\CJKunderwave{釋詁}文。\CJKunderwave{左氏}稱“非我族類,其心必異”,族、類義同,故“族”為類也。言\CJKunderline{鯀}性很戾,多乖異眾人,好此方直之名,內有奸回之志,命而行事輒毀敗善類。何則?心性很戾,違眾用己,知善不從,故云“毀敗善類”。\CJKunderwave{詩}稱“貪人敗類”,與此同。鄭、王以“方為放,謂放棄教命”。\CJKunderwave{易·坤卦}六二“直、方、大”,是直方之事為人之美名。此經雲“方”,故依經為說。 \par}

嶽曰:“異哉!試可乃已。”\footnote{異,已也。已,退也。言餘人盡已,唯\CJKunderline{鯀}可試,無成乃退。異,徐云:“鄭音異,孔、王音怡。”}

{\noindent\zhuan\zihao{6}\fzbyks 傳“異,已。已,退也”。正義曰:“異”聲近已,故為已也。已訓為止,是停住之意,故為退也。 \par}

帝曰:“往,欽哉!”\footnote{敕\CJKunderline{鯀}往治水,命使敬其事。堯知其性很戾圮族,未明其所能,而據總言可試,故遂用之。}九載,績用\textcolor{red}{弗成}。\footnote{載,年也。三考九年,功用不成,則放退之。}

{\noindent\zhuan\zihao{6}\fzbyks 傳“敕\CJKunderline{鯀}”至“用之”。正義曰:傳解\CJKunderline{鯀}非帝所意而命使之者,堯知其性很戾圮族,未明其所能。夫管氏之好奢尚僣,翼贊霸圖;陳平之盜嫂受金,弼諧帝業,然則人有性雖不善,才堪立功者。而眾皆據之言\CJKunderline{鯀}可試,冀或有益,故遂用之。孔之此說,據跡立言,必其盡理而論,未是聖人之實。何則?\CJKunderline{禹}稱“帝德廣運,乃聖乃神”,夫以聖神之資,聰明之鑑,既知\CJKunderline{鯀}性很戾,何故使之治水者?馬融云:“堯以大聖,知時運當然,人力所不能治,下民其諮,亦當憂勞。屈己之是,從人之非,遂用於\CJKunderline{鯀}。”李顒云:“堯雖獨明於上,眾多不達於下,故不得不副倒懸之望,以供一切之求耳。” \par}

{\noindent\zhuan\zihao{6}\fzbyks 傳“載年”至“退之”。正義曰:\CJKunderwave{釋天}云:“載,歲也。夏曰歲,商曰祀,周曰年,唐虞曰載。”李巡云:“各自紀事,示不相襲也。”孫炎曰:“歲,取歲星行一次也。祀,取四時祭祀一訖也。年,取禾穀一熟也。載,取萬物終而更始;是載者年之別名,故以載為年也。”\CJKunderwave{舜典}云:“三載考績,三考,黜陟幽明。”是“三考,九年”也。功用不成,水害不息,故放退之,謂退使不復治水。至明年得舜,乃殛之羽山。\CJKunderwave{周禮·太宰職}云:“歲終則令百官各正其治,而詔王廢置。三年則大計群吏之治而誅賞。”然則考課功績必在歲終,此言“功用不成”,是九年歲終三考也。下雲“朕在位七十載”,而求得\CJKunderline{虞舜}歷試三載,即數登用之年,至七十二年為三載,即知“七十載”者與此異年,此時堯在位六十九年。\CJKunderline{鯀}初治水之時,堯在位六十一年。若然,\CJKunderline{鯀}既無功,早應黜廢。而待九年無成始退之者,水為大災,天之常運;而百官不悟,謂\CJKunderline{鯀}能治水,及遣往治,非無小益,下人見其有益,謂\CJKunderline{鯀}實能治之。日復一日,以終三考,三考無成,眾人乃服,然後退之,故至九年。\CJKunderwave{祭法}云:“\CJKunderline{鯀}障洪水而殛死,\CJKunderline{禹}能修\CJKunderline{鯀}之功。”然則\CJKunderline{禹}之大功,顧亦因\CJKunderline{鯀},是治水有益之驗。但不能成功,故誅殛之耳。若然,災以運來,時不可距,假使興\CJKunderline{禹},未必能治。何以治水之功不成而便殛\CJKunderline{鯀}者?以\CJKunderline{鯀}性傲很,帝所素知,又治水無功,法須貶黜,先有很戾之惡,復加無功之罪,所以殛之羽山,以示其罪。若然,\CJKunderline{禹}既聖人,當知洪水時未可治,何以不諫父者?梁主以為“舜之怨慕,由己之私;\CJKunderline{鯀}之治水,乃為國事。上令必行,非\CJKunderline{禹}能止。時又年小,不可干政也”。 \par}

{\noindent\shu\zihao{5}\fzkt “帝曰疇諮若予”至“九載績用弗成”。正義曰:史又序堯事。堯任羲和,眾功已廣,及其末年,群官有闕,復求賢人,欲任用之。帝曰:“誰乎?諮嗟。”嗟人之難得也。“有人能順此咸熙庶績之事者,我將登而用之”。有臣\CJKunderline{放齊}者對帝:“有胤國子爵之君,其名曰朱,其人心志開達,性識明悟。”言此人可登用也。帝疑怪嘆之曰:“吁!此人既頑且嚚,又好爭訟,豈可用乎?”總言不可也。史又記堯復求人。帝曰:“誰乎?諮嗟。”嗟人之難得也。“今有人能順我事者否乎?”言有即欲用之也。有臣\CJKunderline{驩兜}者對帝曰:“嗚乎!”嘆有人之大賢也。“帝臣\CJKunderline{共工}之官者,此人於所在之方能立事業,聚見其功”。言此人可用也。帝亦疑怪之曰:“吁!此人自作謀計之言,及起用行事而背違之,貌象恭敬而心傲很,若漫天。”言此人不可用也。頻頻求人,無當帝意。於是洪水為災,求人治之。帝曰:“諮嗟!”嗟水災之大也,呼掌嶽之官而告以須人之意。“汝四嶽等,今湯湯流行之水,所在方方為害。又其勢奔突蕩蕩然,滌除在地之物,包裹高山,乘上丘陵,浩浩盛大,勢若漫天。在下之人其皆諮嗟,困病其水矣。有能治者將使治之”。群臣皆曰:“嗚呼!”嘆其有人之能。“惟\CJKunderline{鯀}堪能治之”。帝又疑怪之曰:“吁!其人心很戾哉!好此方直之名,命而行事,輒毀敗善類。”言其不可使也。朝臣已共薦舉,四嶽又復然之。嶽曰:“帝若謂\CJKunderline{鯀}為不可,餘人悉皆已哉。”言不及\CJKunderline{鯀}也。“惟\CJKunderline{鯀}一人試之可也。試若無功,乃黜退之”。言洪水必須速治,餘人不復及\CJKunderline{鯀},故勸帝用之。帝以群臣固請,不得已而用之。乃告敕\CJKunderline{鯀}曰:“汝往治水,當敬其事哉!”\CJKunderline{鯀}治水九載,已經三考而功用不成。言帝實知人,而朝無賢臣,致使水害未除,待舜乃治。此經三言求人,未必一時之事,但歷言朝臣不賢,為求舜張本故也。 \par}


\textcolor{red}{帝曰}:“諮!四嶽,朕在位七十載,\footnote{堯年十六以唐侯升為天子,在位七十年,則時年八十六,老將求代。朕,直錦反,馬云:“我也。”}汝能庸命,巽朕位?”\footnote{巽,順也。言四嶽能用帝命,故欲使順行帝位之事。巽音遜,馬云:“讓也。”}

{\noindent\zhuan\zihao{6}\fzbyks 傳“堯年”至“求代”。正義曰:遍檢今之書傳,無堯即位之年。\CJKunderline{孔氏}博考群書,作為此傳,言“堯年十六以唐侯升為天子”,必當有所案據,未知出何書。計十六為天子,其歲稱元年,在位七十載,應年八十五。孔雲“八十六”者,\CJKunderwave{史記}諸書皆言,堯帝嚳之子,帝摯之弟。嚳崩,摯立。摯崩,乃傳位於堯。然則堯以弟代兄,蓋逾年改元,據其改元年則七十載,數其立年故八十六。下句求人巽位是“老將求代”也。此經文承“績用不成”之下,計治水之事於時最急,不求治水之人而先求代己者,堯以身既年老,臣無可任治水之事,非己所能,故求人代己,令代者自治。是虞史盛美舜功,言堯不能治水,以大事付舜,美舜能消大災,成堯美也。 \par}

{\noindent\zhuan\zihao{6}\fzbyks 傳“巽順”至“之事”。正義曰:“巽,順”,\CJKunderwave{易·說卦}文。帝呼四嶽,言“汝能庸命”,四嶽自謙,言“己否德”,故知“汝”,四嶽。言四嶽能用帝命,故帝欲使之順行帝位之事,將使攝也。在位之臣,四嶽為長,故讓位於四嶽也。 \par}

嶽曰:“否德,忝帝位。”\footnote{否,不。忝,辱也。辭不堪。否,方久反,又音鄙。忝音他簟反。}

{\noindent\zhuan\zihao{6}\fzbyks 傳“否不”至“不堪”。正義曰:否,古今不字。“忝,辱”,\CJKunderwave{釋言}文。己身不德,恐辱帝位,自辭不堪。嶽為群臣之首,自度既不堪,意以為在位之臣皆亦不堪,由是自辭而已,不薦餘人。故帝使之明舉側陋之處。 \par}

曰:“明明揚側陋。”\footnote{堯知子不肖,有禪位之志,故明舉明人在側陋者。廣求賢也。肖音笑,\CJKunderwave{說文}云:“肖,骨肉相似也,不似其先,故曰不肖。”}

{\noindent\zhuan\zihao{6}\fzbyks 傳“堯知”至“求賢”。正義曰:此經“曰”上無“帝”,以可知而省文也。傳解四嶽既辭,而復言此者,堯知子不肖,不堪為主,有禪位與人之志,故令四嶽明舉明人令其在側陋者,欲使廣求賢也。鄭注\CJKunderwave{雜記}云:“肖,似也,言不如人也。”\CJKunderwave{史記·五帝本紀}云:“堯知子\CJKunderline{丹朱}之不肖,不足授天下,於是權授舜。授舜則天下得其利而\CJKunderline{丹朱}病,授\CJKunderline{丹朱}則天下病而\CJKunderline{丹朱}得其利。堯曰‘終不以天下之病而利一人’,而卒授舜以天下。”是堯知子不肖而禪舜之意也。\CJKunderwave{文王世子}論舉賢之法云:“或以事舉,或以言揚。”揚亦舉也,故以舉解“揚”。經之“揚”字在於二“明”之下,傳進“舉”字於兩“明”之中,經於“明”中宜有“揚”字,言明舉明人於側陋之處。“明”下有“揚”,故上闕“揚”文。傳進“舉”於“明”上,互文以足之也。“側陋”者,僻側淺陋之處。意言不問貴賤,有人則舉,是令朝臣廣求賢人也。堯知有舜而朝臣不舉,故令廣求賢以啟之。臣亦以堯知側陋有人,故不得不舉舜耳。此言堯知子不肖,有志禪位,然則自有賢子,必不禪人。授賢爰自上代,堯舜而已。非堯舜獨可,彼皆不然。將以子不肖,時無聖者,乃運值汙隆,非聖有優劣。而\CJKunderwave{緯侯}之書附會其事,乃云:“河洛之符,名字之錄。”何其妄且俗也! \par}

師錫帝曰:“有鰥在下,曰\CJKunderline{虞舜}。”\footnote{師,眾。錫,與也。無妻曰鰥。虞,氏。舜,名。在下民之中。眾臣知舜聖賢,恥己不若,故不舉。乃不獲已而言之。錫,星曆反。鰥,故頑反。\CJKunderline{虞舜},虞,氏;舜,名也。馬云:“舜,諡也。舜死後賢臣錄之,臣子為諱,故變名言諡。”}

{\noindent\zhuan\zihao{6}\fzbyks 傳“師眾”至“言之”。正義曰:“師,眾”、“錫,與”,\CJKunderwave{釋詁}文。“無妻曰鰥”,\CJKunderwave{釋名}云:“愁悒不寐,目恆鰥鰥然,故鰥字從魚,魚目恆不閉。”\CJKunderwave{王制}云:“老而無妻曰鰥。”舜於時年未三十而謂之“鰥”者,\CJKunderwave{書傳}稱\CJKunderline{孔子}對子張曰:“舜父頑,母嚚,無室家之端,故謂之鰥。”“鰥”者無妻之名,不拘老少。少者無妻可以更娶,老者即不復更娶,謂之天民之窮,故\CJKunderwave{禮}舉老者耳。\CJKunderwave{詩}云:“何草不玄,何人不鰥。”暫離室家尚謂之鰥,不獨老而無妻始稱鰥矣。\CJKunderwave{書傳}以舜年尚少為之說耳。“虞,氏。舜,名”者,舜之為虞,猶\CJKunderline{禹}之為夏,外傳稱\CJKunderline{禹}氏曰“有夏”,則此舜氏曰“有虞”。顓頊已來,地為國號,而舜有天下,號曰“有虞氏”,是地名也。王肅云:“虞,地名也。”皇甫謐云:“堯以二女妻舜,封之於虞,今河東太陽山西虞地是也。”然則舜居虞地,以虞為氏,堯封之虞為諸侯,及王天下,遂為天子之號,故從微至著,常稱虞氏。舜為生號之名,前已具釋。傳又解眾人以舜與帝,則眾人盡知有舜。但舜在下人之中,未有官位,眾臣德不及之,而位居其上,雖知舜實聖賢,而恥己不若,故不舉之。以帝令舉及側陋,意謂帝知有舜,乃不獲己而言之耳。知然者,正以初不薦舉,至此始言,明是恥己不若,故不早舉。舜實聖人,而連言“賢”者,對則事有優劣,散即語亦相通。舜謂\CJKunderline{禹}曰“惟汝賢”,是言聖德稱“賢”也。傳以“師”為眾臣,為朝臣之眾,或亦通及吏人。王肅云:“古者將舉大事,訊群吏,訊萬人。堯將讓位,諮四嶽,使問群臣。眾舉側陋,眾皆願與舜。堯計事之大者莫過禪讓,必應博詢吏人,非獨在位。”王氏之言得其實矣。鄭以“師為諸侯之師”,帝諮四嶽,遍訪群臣,安得諸侯之師獨對帝也。 \par}

帝曰:“俞,予聞,如何?”\footnote{俞,然也。然其所舉,言我亦聞之,其德行如何?俞,羊朱反。行,不孟反,下“其行”同。}

{\noindent\zhuan\zihao{6}\fzbyks 傳“俞然”至“如何”。正義曰:“俞,然”,\CJKunderwave{釋言}文。“然其所舉,言我亦聞也,其德行如何?”恐所聞不審,故詳問之。堯知有舜,不召取禪之而訪四嶽,令眾舉薦者,以舜在卑賤,未有名聞,率暴禪之,則下人不服。故\CJKunderline{鄭玄}\CJKunderwave{六藝論}云:“若堯知命在舜,舜知命在\CJKunderline{禹},猶求於群臣,舉於側陋,上下交讓,務在服人。\CJKunderline{孔子}曰:‘人可使由之,不可使知之。’此之謂也。”是解堯使人舉舜之意也。 \par}

嶽曰:“瞽子,父頑,母嚚,\CJKunderline{象}傲,\footnote{無目曰瞽。舜父有目,不能分別好惡,故時人謂之瞽,配字曰瞍。瞍無目之稱。心不則德義之經為頑。象,舜弟之字,傲慢不友。言並惡。瞽音古。傲,五報反。瞍,素後反。稱,尺證反,又如字。}克諧,以孝烝烝,乂不格奸。”\footnote{諧,和。烝,進也。言能以至孝和諧頑嚚昏傲,使進進以善自治,不至於奸惡。諧,戶皆反。烝,之承反。奸,古顏反。}

{\noindent\zhuan\zihao{6}\fzbyks 傳“無目”至“並惡”。正義曰:\CJKunderwave{周禮}樂官有瞽蒙之職,以其無目使悤了相之,是“無目曰瞽”。又解稱瞽之意,舜父有目但不能識別好惡,與無目者同,“故時人謂之瞽”。“配字曰瞍”,“瞍”亦無目之稱,故或謂之為“瞽瞍”。\CJKunderwave{詩}雲“矇瞍奏公”,是“瞍”為瞽類。\CJKunderwave{大禹謨}雲“祗載見瞽瞍”,是相配之文。\CJKunderwave{史紀}云:“舜父瞽瞍盲。”以為“瞽瞍”是名,身實無目也。孔不然者,以經說舜德行,美其能養惡人,父自名瞍,何須言之?若實無目,即是身有固疾,非善惡之事,輒言舜是盲人之子,意欲何所見乎?\CJKunderwave{論語}云:“未見顏色而言謂之瞽。”則言瞽者非謂無目。\CJKunderwave{史記}又說瞽瞍使舜上廩,從下縱火焚廩;使舜穿井,下土實井。若其身自能然,不得謂之無目,明以不識善惡故稱瞽耳。“心不則德義之經為頑”,僖二十四年\CJKunderwave{左傳}文。“象,舜弟之字”,以字表象是人之名號,其為名字未可詳也。\CJKunderwave{釋訓}云:“善兄弟為友。”\CJKunderwave{孟子}說象與父母共謀殺舜,是“傲慢不友”。言舜父母與弟並皆惡也。此經先指舜身,因言瞽子,又稱父頑者,欲極其惡,故文重也。 \par}

{\noindent\zhuan\zihao{6}\fzbyks 傳“諧和”至“奸惡”。正義曰:“諧,和”、“烝,進”,\CJKunderwave{釋詁}文。上歷言三惡,此美舜能養之,言舜能和之以至孝之行,和頑嚚昏傲,使皆進進於善道,以善自治,不至於奸惡。以下愚難變化,令慕善是舜之美行,故以此對堯。案\CJKunderwave{孟子}及\CJKunderwave{史紀}稱瞽瞍縱火焚廩,舜以兩笠自扞而下;以土實井,舜從旁空井出;象與父母其分財物。舜之大孝升聞天朝,堯妻之二女,三惡尚謀殺舜,為奸之大莫甚於此。而言“不至奸”者,此三人性實下愚,動罹刑網,非舜養之,久被刑戮,猶尚有心殺舜,餘事何所不為?舜以權謀自免厄難,使瞽無殺子之愆,象無害兄之罪,“不至於奸惡於此益驗。終令瞽亦允若,象封有鼻,是“不至於奸惡”也。 \par}

帝曰:“我其試哉!\footnote{○言欲試舜,觀其行跡。}女於時,觀厥刑于二女。”\footnote{女,妻。刑,法也。堯於是以二女妻舜,觀其法度接二女,以治家觀治國。女音而據反。妻音千計反。}釐降二女於\xpinyin*{媯}汭,嬪於虞。\footnote{降,下。嬪,婦也。舜為匹夫,能以義理下帝女之心於所居媯水之汭,使行婦道於虞氏。媯音居危反。汭音如銳反,水之內也。杜預注\CJKunderwave{左傳}云:“水之隈曲曰汭。”嬪音毗人反。}

{\noindent\zhuan\zihao{6}\fzbyks 傳“言欲”至“行跡”。正義曰:下言妻舜以女,觀其治家,是試舜觀其行跡也。馬、鄭、王本說“此經皆無‘帝曰’”,當時庸生之徒漏之也。\CJKunderline{鄭玄}云:“試以為臣之事。”王肅云:“試之以官。”鄭、王皆以\CJKunderwave{舜典}合於此篇,故指歷試之事充此“試哉”之言。孔據古今別卷,此言“試哉”正謂以女試之,既善於治家,別更試以難事,與此異也。 \par}

{\noindent\zhuan\zihao{6}\fzbyks 傳“女妻”至“治國”。正義曰:\CJKunderwave{左傳}稱“宋雍氏女於鄭莊公”,“晉伐驪戎,驪戎男女以驪姬”,以女妻人謂之女,故云“女,妻”也。“刑,法”,\CJKunderwave{釋詁}文。此已下皆史述堯事,非復堯語。言“女於時”,謂妻舜於是,故傳倒文以曉,民“堯於是以二女妻舜”。必妻之者,舜家有三惡,身為匹夫,忽納帝女,難以和協,觀其施法度於二女,以法治家觀治國。將使治國,故先使治家。敵夫曰妻,不得有二女,言“女於時”者,總言之耳。二女之中當有貴賤長幼,劉向\CJKunderwave{列女傳}云:“二女長曰娥皇,次曰女英。舜既升為天子,娥皇為後,女英為妃。”然則初適舜時,即娥皇為妻。鄭“不言妻者,不告其父,不序其正”。又注\CJKunderwave{禮記}云:“舜不告而娶,不立正妃。”此則鄭自所說,未有書傳云然。案\CJKunderwave{世本}“堯是黃帝玄孫,舜是黃帝八代之孫”,計堯女於舜之曾祖為四從姊妹,以之為妻,於義不可。\CJKunderwave{世本}之言未可據信,或者古道質故也。 \par}

{\noindent\zhuan\zihao{6}\fzbyks 傳“降下”至“虞氏”。正義曰:“降,下”,\CJKunderwave{釋詁}文。\CJKunderwave{周禮}九嬪之職“掌婦學之法”,嬪是婦之別名,故以“嬪”為婦。“釐降”,未能以義理下之,則女意初時不下,故傳解之,言舜為匹夫,帝女下嫁,以貴適賤,必自驕矜,故美舜能以義理下帝女尊亢之心於所居媯水之汭,使之服行婦道於虞氏。“虞”與“媯汭”為一地,見其心下,乃行婦道,故分為二文。言“匹夫”者,士大夫已上則有妾媵,庶人無妾媵,惟夫妻相匹,其名既定,雖單亦通,謂之匹夫匹婦。媯水在河東虞鄉縣歷山西,西流至蒲阪縣,南入於河,舜居其旁。周\CJKunderline{武王}賜陳胡公之姓為媯,為舜居媯水故也。舜仕堯朝,不家在於京師,而令二女歸虞者,蓋舜以大孝示法,使妻歸事於其親,以帝之賢女事頑嚚舅姑,美其能行婦道,故云“嬪於虞”。 \par}

帝曰:“\textcolor{red}{欽哉}!”\footnote{嘆舜能修己行敬以安人,則其所能者大矣。}

{\noindent\zhuan\zihao{6}\fzbyks 傳“嘆舜”至“大矣”。正義曰:二女行婦道,乃由舜之敬,故帝言“欽哉”。嘆能修己行敬以安民也。能修己及安人,則是所能者大,故嘆之。\CJKunderwave{論語}云:“修己以安百姓,堯舜其猶病諸。”傳意出於彼也。 \par}

{\noindent\shu\zihao{5}\fzkt “帝曰諮四”至“欽哉”。正義曰:帝以\CJKunderline{鯀}功不成,又已年老,求得授位明聖,代御天災,故諮嗟:“汝四嶽等,我在天子之位七十載矣。”言己年老,不堪在位。“汝等四嶽之內,有能用我之命,使之順我帝位之事。”言欲讓位與之也。四嶽對帝曰:“我等四嶽皆不有用命之德,若使順行帝事,即辱於帝位。”言己不堪也。帝又言曰:“汝當明白舉其明德之人於僻隱鄙陋之處,何必在位之臣乃舉之也。”於是朝廷眾臣乃與帝之明人曰:“有無妻之鰥夫,在下民之內,其名曰\CJKunderline{虞舜}。”言側陋之處有此賢人。帝曰:“然,我亦聞之,其德行如何?”四嶽又對帝曰:“其人愚瞽之子,其父頑,母嚚,其弟字象,性又傲慢。家有三惡,其人能諧和以至孝之行,使此頑嚚傲慢者皆進進於善以自治,不至於奸惡。”言能調和惡人,是為賢也。帝曰:“其行如此,當可任用,我其召而試之哉!欲配女與試之也。”即以女妻舜,於是欲觀其居家治否也。舜能以義理下二女之心於媯水之汭,使行婦道於虞氏。帝嘆曰:“此舜能敬其事哉!”嘆其善治家,知其可以治國,故下篇言其授以官位而歷試諸難。 \par}


%%% Local Variables:
%%% mode: latex
%%% TeX-engine: xetex
%%% TeX-master: "../Main"
%%% End:
