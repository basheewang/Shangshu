%% -*- coding: utf-8 -*-
%% Time-stamp: <Chen Wang: 2024-04-02 11:42:41>

% {\noindent \zhu \zihao{5} \fzbyks } -> 注 (△ ○)
% {\noindent \shu \zihao{5} \fzkt } -> 疏

\chapter{卷十四}


\section{康誥第十一}


\CJKunderline{成王}既伐\CJKunderline{管叔}、\CJKunderline{蔡叔},\footnote{滅三監。}以殷餘民封\CJKunderline{康叔},\footnote{以三監之民國\CJKunderline{康叔}為\CJKunderline{衛侯},\CJKunderline{周公}懲其數叛,故使賢母弟主之。○數叛,上所角反,下亦作畔。}作\CJKunderwave{康誥}、\CJKunderwave{酒誥}、\CJKunderwave{梓材}。

康誥\footnote{命\CJKunderline{康叔}之誥。康,圻內國名。叔,封字。○梓音子。圻,具依反。}


{\noindent\zhuan\zihao{6}\fzbyks 傳“以三”至“主之”。正義曰:此序亦與上相顧為首引。初言“三監叛”,又言“黜殷命”,此雲“既伐\CJKunderline{管叔}、\CJKunderline{蔡叔}”,言“以殷餘民”,圻內之餘民,故云“以三監之民國\CJKunderline{康叔}為\CJKunderline{衛侯}”。然古字“邦”、“封”同,故漢有上邦、下邦縣,“邦”字如“封”字,此亦云“邦\CJKunderline{康叔}”,若\CJKunderwave{分器}序雲“邦諸侯”,故云“國\CJKunderline{康叔}”。並以三監之地封之者,\CJKunderline{周公}懲其數叛,故使賢母弟主之。此始一叛而云“數叛”者,以六州之眾悉來歸周,殷之頑民叛逆天命,至今又叛,據周言之,故云“數叛”。故\CJKunderwave{多方}云:“爾乃不大宅天命,爾乃屑播天命。”以不從天命,故云“叛”也。古者大國不過百里,\CJKunderwave{周禮}上公五百里,侯四百里,孟軻有所不信。\CJKunderwave{費誓}注云,\CJKunderline{伯禽}率七百里之內附庸諸侯,則魯猶非七百里之封。而\CJKunderline{康叔}封千里者,\CJKunderline{康叔}時為方伯,殷之圻內諸侯並屬之,故得總言“三監”,且其實地不方平,計亦不能大於魯也。故\CJKunderwave{左傳}云:“宋、衛,吾匹也。”又曰:“寡君未嘗後衛君。”且言千里,亦大率言之耳。何者?邢在襄國,河內即東圻之限,故以賜諸侯西山,即有黎、潞、河、濟之西,以曹地約有千里也。以此鄭云:“初封於衛,至子孫而並邶、鄘也。”其\CJKunderwave{地理志}邶、鄘之民皆遷,分衛民於邶、鄘,故異國而同風,所以\CJKunderwave{詩}分為三。孔與同否未明也。既三年滅三監,七年始封\CJKunderline{康叔},則於其間更遣人鎮守,自不知名號耳。 \par}

{\noindent\zhuan\zihao{6}\fzbyks 傳“命\CJKunderline{康叔}”至“封字”。正義曰:以定四年\CJKunderwave{左傳}祝佗雲“命以\CJKunderwave{康誥}”,故以為“命\CJKunderline{康叔}之誥”。知“康,圻內國名”者,以管、蔡、郕、霍皆國名,則康亦國名而在圻內。馬、王亦然,惟\CJKunderline{鄭玄}以康為諡號,以\CJKunderwave{史記·世家}雲“生康伯”故也。則孔以康伯為號諡,而\CJKunderline{康叔}之“康”猶為國,而號諡不見耳。 \par}

{\noindent\shu\zihao{5}\fzkt “\CJKunderline{成王}”至“康誥”。正義曰:既伐叛人三監之\CJKunderline{管叔}、\CJKunderline{蔡叔}等,以殷餘民國\CJKunderline{康叔}為\CJKunderline{衛侯},\CJKunderline{周公}以王命戒之,作\CJKunderwave{康誥}、\CJKunderwave{酒誥}、\CJKunderwave{梓材}三篇之書也。其\CJKunderwave{酒誥}、\CJKunderwave{梓材}亦戒\CJKunderline{康叔},但因事而分之。然\CJKunderwave{康誥}戒以德刑,又以化紂嗜酒,故次以\CJKunderwave{酒誥},卒若梓人之治材為器為善政以結之。 \par}

惟三月哉生魄,\footnote{\CJKunderline{周公}攝政七年三月。始生魄,月十六日,明消而魄生。○魄,字又作𩲸,普白反,馬云:“𩲸,朏也,謂月三日始生兆朏,名曰魄。”}\CJKunderline{周公}初基,作新大邑於東國洛,四方民大和會。\footnote{初造基,建作王城大都邑於東國洛汭,居天下上中,四方之民大和悅而集會。○汭,如銳反。}


{\noindent\zhuan\zihao{6}\fzbyks 傳“\CJKunderline{周公}”至“魄生”。正義曰:知“\CJKunderline{周公}攝政七年之三月”者,以\CJKunderwave{洛誥}即七年反政而言新邑營及獻卜之事,與\CJKunderwave{召誥}參同,俱為七年,此亦言作新邑,又同\CJKunderwave{召誥},故知七年三月也。若然,\CJKunderwave{書傳}雲四年建\CJKunderline{衛侯}而封\CJKunderline{康叔},五年營成洛邑,六年制禮作樂。\CJKunderwave{明堂位}雲“昔者\CJKunderline{周公}朝諸侯於明堂之位”,即雲“頒度量,而天下大服”,又云“六年制禮作樂”,是六年已有明堂在洛邑而朝諸侯。言六年已作洛邑而有明堂者,\CJKunderwave{禮記}後儒所錄,\CJKunderwave{書傳}伏生所造,皆孔所不用。始生魄,月十六日戊午,社於新邑之明日。“魄”與“明”反,故云“明消而魄生”。 \par}

{\noindent\zhuan\zihao{6}\fzbyks 傳“初造”至“集會”。正義曰:所以初基東國洛者,以天下土中故也。其\CJKunderwave{召誥}與\CJKunderwave{大司徒}文之所出。\CJKunderwave{釋言}云:“集,會也。”以主治民,故民服悅而見大平也。“初基”者,謂初始營建基址,作此新邑,此史總序言之。鄭以為此時未作新邑,而以“基”為謀,大不辭矣。 \par}

侯、甸、男邦、採、衛,百工播民和,見士於周。\footnote{此五服諸侯,服五百里。侯服去王城千里,甸服千五百里,男服去王城二千里,採服二千五百里,衛服三千里,與\CJKunderwave{禹貢}異制。五服之百官,播率其民和悅,並見即事於周。}\CJKunderline{周公}咸勤,乃洪大誥治。\footnote{\CJKunderline{周公}皆勞勉五服之,人遂乃因大封命,大誥以治道。○陸云:“乃洪治同,一本作‘\CJKunderline{周公}乃洪大誥治’。”勞,力報反。}

{\noindent\zhuan\zihao{6}\fzbyks 傳“此五”至“於周”。正義曰:“男”下獨有“邦”,以五服男居其中,故舉中則五服皆有“邦”可知,言“邦”見其國君焉。以\CJKunderwave{大司馬職}\CJKunderwave{大行人},故知五服,“服五百里”。\CJKunderwave{禹貢}五服通王畿,此在畿外,去王城五百里,故每畿計之,至衛服三千里,言“與\CJKunderwave{禹貢}異制”也,通王畿與不通為異。以此計畿之均,故須土中。若然,黃帝與帝嚳居偃師,餘非土中者,自出當時之宜。實在土中,因得而美善之也。不見要服者,鄭云:“以遠於役事而恆闕焉。”君行必有臣從,即卿大夫及士,見亦主其勞,故云五服之內,百官播率其民和悅即事。以土功勞事,民之所苦也,而此和悅,見太平也。而\CJKunderwave{書傳}云:“示之以力役,其民猶至,況導之以禮樂乎?”是也。 \par}

{\noindent\zhuan\zihao{6}\fzbyks 傳“\CJKunderline{周公}”至“治道”。正義曰:太保以戊申至,七月庚戌,已雲“庶殷攻位於洛汭”,則庶殷先與之期於前至也。\CJKunderline{周公}以十三日乙卯“朝至於洛,則達觀於新邑營”,此日當勉其民,此因命而並言之。序雲“邦\CJKunderline{康叔}”,“洪”,大也,為大封命大誥\CJKunderline{康叔}以治道也。\CJKunderline{鄭玄}以“洪”為代,言\CJKunderline{周公}代\CJKunderline{成王}誥。何故代誥而反誥王,呼之曰“\CJKunderline{孟侯}”?為不辭矣。 \par}

{\noindent\shu\zihao{5}\fzkt “惟三”至“誥治”。正義曰:言惟以\CJKunderline{周公}攝政七年之三月,始明死而生魄,月十六日己未,於時\CJKunderline{周公}初造基趾,作新大邑於東國洛水之汭,四方之民大和悅而集會,言政治也。此所集之民,即侯、甸、男、採、衛五服。百官播率其民和悅,並見即事於周之東國,而\CJKunderline{周公}皆慰勞勸勉之。乃因大封命以\CJKunderline{康叔}為\CJKunderline{衛侯},大誥以治道。 \par}

王若曰:“\CJKunderline{孟侯},朕其弟,小子\CJKunderline{封}。\footnote{\CJKunderline{周公}稱\CJKunderline{成王}命,順\CJKunderline{康叔}之德,命為\CJKunderline{孟侯}。孟,長也。五侯之長,謂方伯,使\CJKunderline{康叔}為之。言王使我命其弟封。封,\CJKunderline{康叔}名。稱小子,明當受教訓。○長,丁丈反,下同。}惟乃丕顯考\CJKunderline{文王},克明德慎罰,\footnote{惟汝大明父\CJKunderline{文王},能顯用俊德,慎去刑罰,以為教首。○去,羌呂反,下“欲去”、“去疾”同。}不敢侮鰥寡,庸庸,祇祇,威威,顯民。\footnote{惠恤窮民,不慢鰥夫寡婦,用可用,敬可敬,刑可刑,明此道以示民。}用肇造我區夏,越我一二邦,\footnote{用此明德慎罰之道,始為政於我區域諸夏,故於我一二邦皆以修治。}以修我西土。惟時怙冒,聞於上帝,帝休。\footnote{我西土岐周,惟是怙恃\CJKunderline{文王}之道,故其政教冒被四表,上聞於天,天美其治。○怙音戶。冒,莫報反,覆也。聞如字,徐又音問。}天乃大命\CJKunderline{文王},殪戎殷,誕受厥命,\footnote{天美\CJKunderline{文王},乃大命之殺兵殷,大受其王命。謂三分天下有其二,以授\CJKunderline{武王}。○殪,於計反。}越厥邦厥民,惟時敘。\footnote{於其國,於其民,惟是次序,皆\CJKunderline{文王}教。}乃寡兄勖,肆汝小子\CJKunderline{封},在茲東土。”\footnote{汝寡有之兄\CJKunderline{武王},勉行\CJKunderline{文王}之道,故汝小子封得在此東土為諸侯。○勖,許玉反。}


{\noindent\zhuan\zihao{6}\fzbyks 傳“\CJKunderline{周公}”至“教訓”。正義曰:以“曰”者為命辭,故曰“\CJKunderline{周公}稱\CJKunderline{成王}命,順\CJKunderline{康叔}之德,命為\CJKunderline{孟侯}。孟,長也。五侯之長,謂方伯”。使\CJKunderline{康叔}為之長者,即州牧也。“五侯之長”,五等諸侯之長也。而\CJKunderwave{左傳}云:“五侯九伯,汝實徵之。”彼謂上公之伯,故徵九伯。而此“五侯”當州牧之“五侯”,與彼不同。\CJKunderwave{王制}有連、屬、率、伯也,孔以五侯亦方伯,則四方者皆可為方伯,而此方伯自是州牧也。\CJKunderline{康叔}以母弟令德受大國封命,固非卒及連、屬也。虞夏及周既有牧,又\CJKunderwave{離騷}雲“伯昌作牧”,殷亦有牧,伯四代皆通也,非如\CJKunderline{鄭玄}雲“殷之州長曰伯”。以“稱小子”為幼弱,故“明當受教訓”,故云“使我命其弟”,為親親而使我用戒故也。此指命\CJKunderline{康叔}為之,而鄭以總告諸侯,依\CJKunderwave{略說}以太子十八為\CJKunderline{孟侯}而呼\CJKunderline{成王}。既禮制無文,義理駢曲,豈\CJKunderline{周公}自許天子,以王為\CJKunderline{孟侯}?皆不可信也。 \par}

{\noindent\zhuan\zihao{6}\fzbyks 傳“惟汝”至“教首”。正義曰:以近而可法,不過子之法父,故舉\CJKunderline{文王}也。法者不過除惡行善,故云“明德慎罰”也。 \par}

{\noindent\zhuan\zihao{6}\fzbyks 傳“惠恤”至“示民”。正義曰:“用可用,敬可敬”,即“明德”也。“用可用”謂小德小官,“敬可敬”謂大德大官,“刑可刑”謂“慎罰”也。 \par}

{\noindent\zhuan\zihao{6}\fzbyks 傳“天美”至“\CJKunderline{武王}”。正義曰:“天美\CJKunderline{文王},乃天命之殺兵殷”者,“殪”,殺也,“戎”,兵也,用誅殺之道以兵患殷。\CJKunderline{文王}以伐殷事未卒而言“殺兵殷”者,謂三分有二,為滅殷之資也。 \par}

{\noindent\shu\zihao{5}\fzkt “王若”至“東土”。正義曰:言\CJKunderline{周公}稱\CJKunderline{成王}命,順\CJKunderline{康叔}之德而言曰:“命汝為\CJKunderline{孟侯}。王又使我教命其弟小子封。其所教命者,惟汝大明德之父\CJKunderline{文王},能顯用俊德,慎去刑罰,以為教首。故惠恤窮民,不侮慢鰥夫寡婦,況貴強乎?其明德,用可用,敬可敬,其慎罰,威可威者,顯此道以示民。用此道,故始為政於我區域諸夏,由是於我一二諸國漸以修治也。上政既修,我西土惟是怙恃\CJKunderline{文王}之道,故其政教冒被四表,聞於上天,天美其治道。以此上天乃大命\CJKunderline{文王}以誅殺之道,用兵除害於殷,大受其王命,三分天下而有其二也。其所受二分者,於其國,於其民,惟是皆有次序,以\CJKunderline{文王}之德故也。汝寡有之兄\CJKunderline{武王},勉行\CJKunderline{文王}之道,故受命克殷,今汝小子封故得在此東土為諸侯。是\CJKunderline{文王}之道,明德慎罰,既用受命,\CJKunderline{武王}無所復加,以為勉行,所以汝必法之。” \par}

王曰:“嗚呼!\CJKunderline{封},汝念哉!\footnote{念我所以告汝之言。}今民將在祇遹乃文考,紹聞衣德言。\footnote{今治民將在敬循汝文德之父,繼其所聞,服行其德言,以為政教。○遹音聿,又音述,馬紹:“述也。”衣如字,徐於既反。}往敷求於殷先哲王,用保乂民。\footnote{汝往之國,當布求殷先智王之道,用安治民。}汝丕遠惟商耇成人,宅心知訓。\footnote{汝當大遠求商家耇老成人之道,常以居心,則知訓民。○耇音狗。}別求聞由古先哲王,用康保民。\footnote{又當別求所聞父兄用古先智王之道,用其安者以安民。}弘於天,若德裕乃身,不廢在王命。”\footnote{大於天,為順德,則不見廢,常在王命。}


{\noindent\zhuan\zihao{6}\fzbyks 傳“今治”至“政教”。正義曰:“繼其所聞,服行其德言”者,謂\CJKunderline{文王}先有所聞善事,今令\CJKunderline{康叔}繼續其\CJKunderline{文王}所聞善事,被服而施行其德言,以為政教也。 \par}

{\noindent\zhuan\zihao{6}\fzbyks 傳“汝當”至“訓民”。正義曰:上雲“敷求殷先哲王”,謂求殷之賢君,此言“求商家耇老成人”,謂求殷之賢臣。“大遠”者,備遍求之。 \par}

{\noindent\zhuan\zihao{6}\fzbyks 傳“又當”至“安民”。正義曰:以父兄乃所居殷外,故云“別求”。上只言“遹乃文考”,並言“兄”者,以上雲“寡兄勖”,則以文武道同,言文可以兼武,故並言“父兄”也。“古先哲王”,鄭云:“虞夏也。”孔亦當然。以上代與今事遠,不可以同,故言“用其安者”。 \par}

{\noindent\zhuan\zihao{6}\fzbyks 傳“大於”至“王命”。正義曰:以天道人用而光大之,故因雲“大”也。其\CJKunderline{文王}及殷古先哲王,與天其道不異,以前後聖蹟雖殊,同天不二也。以\CJKunderline{康叔}亞聖大賢,治殷餘惡,故使之用天道為順德也。 \par}

{\noindent\shu\zihao{5}\fzkt “王曰嗚呼封汝”至“王命”。正義曰:既言\CJKunderline{文王}“明德慎罰”之訓,\CJKunderline{武王}尚行之,汝既得為君,方別陳明德之事,故稱王命而言曰:“嗚呼!封,汝常念我所以告汝之言哉!今治民所行,將在敬循汝文德之父,繼其所聞者,服行其德言,以為政教。汝往之國,當分佈求於殷先智王之道,用安治民。不但法其先君,汝又當須大遠求商家耇老成人之道,居之於心,即知訓民矣。其外又更當別求所聞父兄用古先智王之道,用其安者以安民。即古虞夏之道也。人事既然,又闡大於天之道而為順德,又加之寬容,則汝身不見廢,常在王命。” \par}

王曰:“嗚呼!小子\CJKunderline{封},\xpinyin*{恫瘝}乃身,敬哉!\footnote{恫,痛。瘝,病。治民務除惡政,當如痛病在汝身欲去之,敬行我言。○恫音通,又敕動反。敕,古頑反。}天畏棐忱,民情大可見,小人難保。\footnote{天德可畏,以其輔誠。人情大可見,以小人難安。○棐音匪,又芳鬼反。忱,巿林反。}往盡乃心,無康好逸豫,乃其乂民。\footnote{往當盡汝心為政,無自安好逸豫寬身,其乃治民。○盡,徐子忍反。好,呼報反。}我聞曰:‘怨不在大,亦不在小,惠不惠,懋不懋。’\footnote{不在大,起於小;不在小,小至於大。言怨不可為,故當使不順者順,不勉者勉。○懋音茂。}已!汝惟小子,乃服惟弘王,應保殷民,\footnote{已乎!汝惟小子,乃當服行德政,惟弘\CJKunderline{大王}道,上以應天,下以安我所受殷之民眾。○應,應對之應,注同,徐於甑反。}亦惟助王宅天命,作新民。”\footnote{弘王道,安殷民,亦所以惟助王者居順天命,為民日新之教。}


{\noindent\zhuan\zihao{6}\fzbyks 傳“恫痛”至“我言”。正義曰:“恫”聲類於痛,故“恫”為痛也。“瘝,病”,\CJKunderwave{釋詁}文。以痛病在汝身以述治民,故務除惡政如已病也。戒之而言“敬”,故知“敬行我言”也。\CJKunderline{鄭玄}云:“刑罰及已為痛病。”其義不及去惡若已病也。 \par}

{\noindent\zhuan\zihao{6}\fzbyks 傳“天德”至“難安”。正義曰:人情所以大可見者,以小人難安為可見,故須安之。 \par}

{\noindent\zhuan\zihao{6}\fzbyks 傳“不在”至“者勉”。正義曰:以致怨恐謂由大惡,故云“不在大,起於小”,言怨由小事起。“不在小”者,謂為怨不恆在小,言其初小,漸至於大怨,故使不順者順,不勉者勉,其怨自消也。 \par}

{\noindent\zhuan\zihao{6}\fzbyks 傳“弘王”至“之教”。正義曰:“亦所以惟助王”者,言非直\CJKunderline{康叔}身行有益,亦惟助王者居順天命,為民日新之教,謂漸致太平,政教日日益新也。 \par}

{\noindent\shu\zihao{5}\fzkt “王曰嗚呼小”至“新民”。正義曰:所明而云行天人之德者,其要在於治民,故言王曰:“嗚呼!小子封,治民為善而除惡政,當如痛病在汝身欲去之,敬行我言哉!所以去惡政者,以天德可畏者,以其輔誠故也。以民情大率可見,所以可見者,以小人難保也。安之既難,其往治之,當盡汝心為政,無自安好逸豫而寬縱,乃其可以治民。我聞名遺言曰,人之怨不在事大,或由小事而起。雖由小事而起,亦不恆在事小,因小至大。是為民所怨,事不可為。當使施順,令不順者順。勉力勸行,令不勉者勉。則其怨小大都消,令汝消怨者。已乎!汝惟小子,乃當服行政德,惟弘\CJKunderline{大王}道,上以應天,下以安我所受殷民。不但汝身所當行,此亦惟助王者居順天命,為民日新之教。” \par}

王曰:“嗚呼!\CJKunderline{封},敬明乃罰。\footnote{嘆而敕之,凡行刑罰,汝必敬明之。欲其重慎。}人有小罪,非眚,乃惟終,自作不典,式爾,\footnote{小罪非過失,乃惟終身行之,自為不常,用犯汝。○眚,所領反,本亦作省。}有厥罪小,乃不可不殺。乃有大罪,非終,乃惟眚災,適爾,既道極厥辜,時乃不可殺。”\footnote{汝盡聽訟之理以極其罪,是人所犯,亦不可殺,當以罰宥論之。○宥,於救反。}

{\noindent\shu\zihao{5}\fzkt “王曰嗚呼封敬”至“可殺”。正義曰:以上既言“明德”之理,故此又云“慎罰”之義,而王言曰:“嗚呼!封,又當敬明汝所行刑罰,須明其犯意。人有小罪,非過誤為之,乃惟終身自為不常之行,用犯汝,如此者,有其罪小,乃不可不殺,以故犯而不可赦。若人乃有大罪,非終行之,乃惟過誤為之,以此故,汝當盡斷獄之道以窮極其罪,是人所犯,乃不可以殺,當以罰宥論之,以誤故也。即原心定罪,斷獄之本,所以須敬明之也。” \par}

王曰:“嗚呼!\CJKunderline{封},有敘,時乃大明服,\footnote{嘆政教有次敘,是乃治理大明,則民服。}惟民其敕懋和。\footnote{民既服化,乃其自敕正勉為和。}若有疾,惟民其畢棄咎。\footnote{化惡為善,如欲去疾,治之以理,則惟民其盡棄惡修善。○咎,其九反。}若保赤子,惟民其康乂。\footnote{愛養紉宴安孩兒赤子,不失其欲,惟民其皆安治。○孩,亥才反。}非汝\CJKunderline{封}刑人殺人,\footnote{言得刑殺罪人。}無或刑人殺人。\footnote{無以得刑殺人,而有妄刑殺非辜者。}非汝封又\CJKunderline{曰}\xpinyin*{劓刵}人,\footnote{劓,截鼻。刵,截耳。刑之輕者,亦言所得行。○劓,魚器反。刵,如志反。}無或劓刵人。”\footnote{所以舉輕以戒,為人輕行之。}


{\noindent\zhuan\zihao{6}\fzbyks 傳“化惡”至“修善”。正義曰:人之有疾,治之以理則疾去。人之有惡,化之以道則惡除。 \par}

{\noindent\zhuan\zihao{6}\fzbyks 傳“愛養”至“安治”。正義曰:既去惡,乃須愛養之為善,人為上養,則化所行,故言其皆安治。子生赤色,故言“赤子”。 \par}

{\noindent\zhuan\zihao{6}\fzbyks 傳“劓截”至“得行”。正義曰:以國君故得專刑殺於國中,而不可濫其刑,即墨、劓、剕、宮也。“劓”在五刑為截鼻,而有“刵”者,\CJKunderwave{周官}五刑所無,而\CJKunderwave{呂刑}亦云“劓刵”,\CJKunderwave{易·噬嗑}上九雲“何校滅耳”。\CJKunderline{鄭玄}以臣從君坐之刑,孔意然否未明,要有刵而不在五刑之類。言“又曰”者,\CJKunderline{周公}述\CJKunderline{康叔},豈非“汝封”又自言曰得劓刵人?此“又曰”者,述\CJKunderline{康叔}之“又曰”。 \par}

{\noindent\shu\zihao{5}\fzkt “王曰嗚呼封有”至“刵人”。正義曰:以刑者政之助,不得已即用之;非情好殺害,故又本於政不可以濫刑,而王言曰:“嗚呼!封,欲正刑之本,要而汝政教有次序,是乃治理大明則民服。惟民既服從化,其自敕正勉力而平和。然政之化惡為善,若有病而欲去之,治之以理,則惟民其盡棄惡而修善。言愛養人陽母之安赤子,惟民為善,其皆安治。為政保民之如此,不可行以淫刑,豈非汝封得刑人殺人乎?言得刑殺不可以得故,而有濫刑人殺人無辜也。非汝封又曰劓刵人,無以得故,而有所濫劓刵人之無罪者也。” \par}

王曰:“外事,汝陳時臬,司師,茲殷罰有倫。”\footnote{言外土諸侯奉王事,汝當布陳是法,司牧其眾,及此殷家刑罰有倫理者兼用之。○臬,魚列反。}又曰:“要囚,服念五六日,至於旬時,丕蔽要囚。”\footnote{要囚,謂察其要辭以斷獄。既得其辭,服膺思念五六日,至於十日,至於三月,乃大斷之。言必反覆思念,重刑之至也。○要,於宵反。蔽,必世反。斷,丁亂反,下及篇末同。覆,芳服反。}


{\noindent\zhuan\zihao{6}\fzbyks 傳“言外”至“用之”。正義曰:外土以獄事上於州牧之官,為奉土事,汝當用刑書,為布陳是刑法,為司牧其眾,故受而聽之。既衛居殷墟,又周承於殷後,刑書相因,故兼用其有理者。謂當時刑書,或無正條,而殷有故事,可兼用,若今律無條,求故事之比也。“臬”為準限之義,故為法也。 \par}

{\noindent\zhuan\zihao{6}\fzbyks 傳“要囚”至“之至”。正義曰:言“要囚”,明取要辭於囚以思訖事定,故言“乃大斷之”。多至三月,故云“反覆思念,重刑之至”。顧氏云:“‘又曰’者,\CJKunderline{周公}重言之也。” \par}

{\noindent\shu\zihao{5}\fzkt “王曰外事”至“要囚”。正義曰:言不濫刑,不但國內,而王言曰:“若外土諸侯奉王事以至汝,汝當布陳是刑法以司牧其眾,及此殷家刑罰有倫理者兼用之。”\CJKunderline{周公}又重言曰:“既用刑法,要察囚情,得其要辭,以斷其獄。當須服膺思念之,五日六日,次至於十日,遠至於三月,一時乃大斷囚之要辭。”言必反覆重之如此,乃得無濫故耳。 \par}

王曰:“汝陳時臬事,罰蔽殷彝,\footnote{陳是法事,其刑罰斷獄,用殷家常法,謂典刑故事。○彝,以支反。}用其義刑義殺,勿庸以次汝封。\footnote{義,宜也。用舊法典刑,宜於時世者以刑殺,勿庸以就汝封之心所安。}乃汝盡遜曰時敘,惟曰未有遜事。\footnote{乃使汝所行盡順,曰是有次敘,惟當自謂未有順事,君子將興,自以為不足。}已!汝惟小子,未其有若汝\CJKunderline{封}之心,朕心朕德惟乃知。\footnote{已乎!他人未其有若汝封之心。言汝心最善,我心我德惟汝所知。欲其明\CJKunderline{成王}所以命已之款心。○款,苦管反。}


{\noindent\zhuan\zihao{6}\fzbyks 傳“陳是”至“故事”。正義曰:“陳是法事”,即上“汝陳時臬事”,“罰蔽殷彝”即上“殷罰有倫”。上據有初思念得失,此據臨時行事也。 \par}

{\noindent\zhuan\zihao{6}\fzbyks 傳“已乎”至“款心”。正義曰:此言“我”,我王也,以王命,故言王為“我”,以\CJKunderline{康叔}為“己”。若汝不善我王家心德,汝所不知,則我不順命汝款曲之心。只由汝最善,我王心德汝所遍知,故我王命汝以款曲之心。述\CJKunderline{康叔}為言,故云“己”,欲令\CJKunderline{康叔}明識此意也。 \par}

{\noindent\shu\zihao{5}\fzkt “王曰汝”至“乃知”。正義曰:此又申上既要囚思念,定其大斷若為,而王言曰:“汝當陳是刑書之法以行事,其刑法斷獄,用殷家所行常法故事,其陳法殷彝,皆用其合宜者以刑殺,勿用以就汝封意之所安而自行也,以用心不如依法故耳。言汝不但依法,乃使汝所行盡順,曰是有次敘,猶當自惟曰未有順事,其有餘若不足故耳。必期汝於大幸已乎!汝惟小子耳,而他人未其有若汝封之心。言汝心最善,汝心既善,我心我德惟汝所委知也。 \par}

凡民自得罪,寇攘奸宄,殺越人於貨,\footnote{凡民用得罪,為寇盜攘竊奸宄,殺人顛越人,於是以取貨利。○攘,如羊反。宄音軌。}\xpinyin*{暋}不畏死,罔弗憝。”\footnote{暋,強也。自強為惡而不畏死,人無不惡之者,言當消絕之。○暋音敏。憝,徒對反,徐徒猥反。強,其丈反。無不惡,烏路反,下“所大惡”、“疾惡”、“亦惡”並音同。}


{\noindent\zhuan\zihao{6}\fzbyks 傳“凡民”至“貨利”。正義曰:“目”,用也,言所用得罪者,由寇攘也。而為之於外內,既有劫竊,其劫竊皆有殺有傷,“越人”謂不死而傷,皆為之而取貨利故也。 \par}

{\noindent\zhuan\zihao{6}\fzbyks 傳“暋強”至“絕之”。正義曰:“暋,強也”,於\CJKunderwave{盤庚}已訓而此重詳之,以由此得罪,當須絕之。 \par}

{\noindent\shu\zihao{5}\fzkt “凡民”至“弗憝”。正義曰:言人所慎刑者,以凡民所用得罪者,寇盜攘竊於外奸內宄,而殺害及顛越於人以取貨利也。自強為之而不畏死,此為人無不惡之者,以此須刑絕之,故當慎刑罰耳。 \par}

王曰:“\CJKunderline{封},元惡大憝,矧惟不孝不友。\footnote{大惡之人猶為人所大惡,況不善父母,不友兄弟者乎?言人之罪惡,莫大於不孝不友。}子弗祇服厥父事,大傷厥考心。\footnote{為人子不能敬身服行父道,而怠忽其業,大傷其父心,是不孝。}於父不能字厥子,乃疾厥子。\footnote{於為人父不能字愛其子,乃疾惡其子,是不慈。}於弟弗念天顯,乃弗克恭厥兄。\footnote{於為人弟不念天之明道,乃不能恭事其兄,是不恭。}兄亦不念鞠子哀,大不友于弟。\footnote{為人兄亦不念稚子之可哀,大不篤友于弟,是不友。○鞠,居六反。}


{\noindent\zhuan\zihao{6}\fzbyks 傳“大惡”至“不友”。正義曰:言將有作奸宄大惡,猶為人所大惡,況不孝父母,不善兄弟者乎?\CJKunderwave{孝經}雲“五刑之屬三千,而罪莫大於不孝”是也。\CJKunderwave{釋親}云:“善父母為孝,善兄弟為友。”下文不言“母”,母同於父。父子尊卑而異等,故“孝”名上不通於下。其兄弟雖有長幼而同倫,故共“友”名也。 \par}

{\noindent\zhuan\zihao{6}\fzbyks 傳“為人”至“不孝”。正義曰:“考”亦通生死,即此文及\CJKunderwave{酒誥}是也。下\CJKunderwave{曲禮}雲“死曰考”,是對例耳。人予以述成父事為孝,怠忽其業,即“其肯曰,我有後,不棄基”,故為大傷父心,即是上不孝也。則子不述父事,當輕於盜殺,況以為甚者,此聖人緣心立法,人莫不緣身本於父母也,自親以及物,天然之理,故\CJKunderwave{孝經}曰:“不愛其親而愛他人者,謂之悖德。不敬其親而敬他人者,謂之悖禮。以順則逆,民無則焉。不在於善,而皆在於凶德”是也。以此言賊殺他人,罪小於骨肉相乖阻。但於他人言其極者,於親言其小者,小則有不和詈爭鬥訟相傷者也。於親小則傷心,大乃逆命,毆罵殺害,互相發起而可知也。 \par}

{\noindent\zhuan\zihao{6}\fzbyks 傳“於為”至“不慈”。正義曰:上文不言“不慈”,意以“不孝”為總焉。父當言“義”而云“不慈”者,以父母於子併為慈,因父有愛敬多少而分之。言父義母慈,而由慈以義,故雖義言“不慈”,且見父兼母耳。 \par}

{\noindent\zhuan\zihao{6}\fzbyks 傳“於為”至“不恭”。正義曰:善兄弟曰友,此言“不恭”者,“友”思念之辭,兄弟同倫,故俱言“友”;雖同倫而有長幼,其心友而貌恭,故因兄弟而分“友”文為二而言“恭”也。五教,即\CJKunderwave{左傳}文十八年史克言也。於此言“天之明道”者,父子天性,不嫌非天明,故於兄弟言之。因上先言“不孝”,先言子於父,故此“不友”先言弟於兄,若舉中以見上下,故此言天明,見五教皆是,即\CJKunderwave{孝經}雲“則天之明”,\CJKunderwave{左傳}雲“為父子兄弟姻媾以象天明”,是於天理常然,為天明白之道。 \par}

{\noindent\zhuan\zihao{6}\fzbyks 傳“為人”至“不友”。正義曰:言“亦”者,以兄弟同等而相亦,所謂\CJKunderwave{周書}雲“父子兄弟罪不相及”,即此文也。不孝罪子,非及於父之輩,理所當然。而\CJKunderwave{周官}鄰保以比伍相及,而趙商疑而發問,鄭答云:“\CJKunderwave{周禮}太平制,此為居殷亂而言,斯不然矣。\CJKunderwave{康誥}所云,以骨肉之親,得相容隱,故\CJKunderwave{左傳}云:‘父子兄弟罪不相及。’\CJKunderwave{周禮}所云,據疏人相督率之法,故相連獲罪。故今之律令,大功已上得相容隱,鄰保罪有相及是也。” \par}

惟弔,茲不於我政人得罪,\footnote{惟人至此不孝、不慈、弗友、不恭,不於我執政之人得罪乎?道教不至所致。○吊音的。}天惟與我民彝大湣亂。\footnote{天與我民五常,使父義、母慈、兄友、弟恭、子孝,而廢棄不行,是大滅亂天道。○泯,徐武軫反。}曰,乃其速由\CJKunderline{文王}作罰,刑茲無赦。\footnote{言當速用\CJKunderline{文王}所作違教之罰,刑此亂五常者,無得赦。}

{\noindent\shu\zihao{5}\fzkt “王曰封元”至“無赦”。正義曰:以是所用得其罪,不但寇盜,王命而言曰:“封,非於骨肉之人為大惡,猶尚為人所大惡之,況惟不孝父母,不友兄弟者乎?其罪莫大於不孝也。何者?為人之子不能敬身服行其父事,而怠忽其業,大傷其父心,是不孝也。於為人父不能字愛其子,乃疾惡其子,是不慈也。於為人弟不能念天之明道,故乃不能恭事其兄,是不恭也。為人兄亦不能念稚子之可哀哉,大不友愛於弟,是不友也。惟人所行以至此不孝不友者,豈不由我執政之人道教不至,以得此罪乎?既人罪由教而致,天惟與我民以五常之性,使有恭孝,廢棄不行,是大滅亂天道也。以由我滅亂,曰,乃其疾用\CJKunderline{文王}所作違教之罰,刑此亂五常者,不可赦放也。” \par}

不率大戛,矧惟外庶子、訓人,\footnote{戛,常也。凡民不循大常之教,猶刑之無赦,況在外掌眾子之官主訓民者而親犯乎?○戛,簡八反。}惟厥正人,越小臣諸節。\footnote{惟其正官之人,於小臣諸有符節之吏,及外庶子,其有不循大常者,則亦在無赦之科。}乃別播敷造。民大譽,弗念弗庸,瘝厥君,時乃引惡,惟朕憝。\footnote{汝今往之國,當分別播佈德教,以立民大善之譽。若不念我言、不用我法者,病其君道,是汝長惡,惟我亦惡汝。○別,彼列反,注同。汝長,丁丈反,下同。}


{\noindent\zhuan\zihao{6}\fzbyks 傳“戛常”至“犯乎”。正義曰:“戛”猶楷也,言為楷模之常,故“戛”為常也。述上凡民自得罪,故言“凡民不循大常之教”也。“猶刑之”即上雲“刑茲無赦”故也。亦愚以況智,故言“況在外掌眾子之官主訓民者而親犯乎”,即\CJKunderwave{周官}雲“諸子”,\CJKunderwave{文王世子}雲“庶子”也。以致教諸子,故為“訓人”。\CJKunderwave{周禮}諸子之官亦是王朝之臣,言“在外”者,對父子兄弟為外。惟舉庶子之官者,以其教訓公卿子弟,最為急故也。\CJKunderline{鄭玄}以“訓人”為師長,亦各一家之道也。 \par}

{\noindent\zhuan\zihao{6}\fzbyks 傳“惟其”至“之科”。正義曰:“正官之人”,若\CJKunderwave{周官}三百六十職正官之首。“於小臣諸有符節”者,謂正人之下,非長官之身,下至符吏。“諸有符節”,為教人之故,故言有符節者。非要行道之符節,若為官行文書而有符,今之印者也。以上況之,故言“不循大常,亦在無赦之科”矣。在軍者有旌節,亦得為有符節耳。 \par}

{\noindent\zhuan\zihao{6}\fzbyks 傳“汝今”至“惡汝”。正義曰:言“分別播佈德教”,謂分遣卿大夫為之教民使善。而已有善譽,是“立民以大善之譽”。 \par}

已!汝乃其速由茲義率殺。亦惟君惟長,\footnote{汝乃其速用此典刑宜於時世者,循理以刑殺,則亦惟君長之正道。}不能厥家人,越厥小臣、外正,惟威惟虐,大放王命,乃非德用乂。\footnote{為人君長而不能治其家人之道,則於其小臣外正官之吏,併為威虐,大放棄王命,乃由非德用治之故。}


{\noindent\zhuan\zihao{6}\fzbyks 傳“汝乃”至“正道”。正義曰:此用宜於時以刑殺上不循五常之道者。其“君長”,對則大夫為長,散則人君為長,君而居之,是君亦與長為一。\CJKunderwave{孝經}對例以長為大夫耳。 \par}

{\noindent\zhuan\zihao{6}\fzbyks 傳“為人”至“之故”。正義曰:以五常父母兄弟子即家人之道,\CJKunderwave{易}有家人卦,亦與此同也。不行五教為不能治家人之道,家人不治,則君不明。君既不明,則不察下故則,於其小臣外正官之吏併為威虐,大放棄王命,非德用治,是不明為非德也。 \par}

汝亦罔不克敬典,乃由裕民,惟\CJKunderline{文王}之敬忌。\footnote{常事人之所輕,故戒以無不能敬常。汝用寬民之道,當惟念\CJKunderline{文王}之所敬思而法之。}乃裕民,曰:‘我惟有及。’則予一人以懌。”\footnote{汝行寬民之政,曰:“我惟有及於古。”則我一人以此悅懌汝德。○懌音亦。}

{\noindent\zhuan\zihao{6}\fzbyks 傳“常事”至“法之”。正義曰:“常事”,常所行之事也。人見尋常不為異,故輕之,而以為戒。“\CJKunderline{文王}所敬忌”,即敬德忌刑。鄭云:“‘祇祇、威威’是也。” \par}

{\noindent\zhuan\zihao{6}\fzbyks 傳“汝行”至“汝德”。正義曰:寬則得眾,故五教在寬。上既言“乃由裕民”,此又疊之,汝行寬民之政,曰:“我惟有及於古。”即古賢諸侯。汝惡,我則惡之。汝善,我則愛之。以此,我一人悅懌汝德也。 \par}

{\noindent\shu\zihao{5}\fzkt “不率”至“以懌”。正義曰:言滅五常之害當除,凡民不循大道五常之教,猶刑之,況在外土掌庶子之官,主於訓民,惟其正官之人,及於小臣諸有符節者,併為教首,其心不循大常,豈可赦也?以人之須有五常,汝今往之國,乃當分別播佈德教,以立民大善之譽。若不念我言,不用我法,即病其為君之道,是汝長為惡矣,以此惟我亦惡汝也。已乎!既惡不可為,汝乃其疾用此典刑宜於時世者,循理以刑殺亂常者,則亦惟為人君,惟為人長之正道。既為人君長,不能治其五教,施於家人之道,則於其卑小臣外土正官之吏,惟為威暴,則為酷虐,大放棄王命矣。如是乃由汝非以道德用治之故。由此汝亦無得不能敬其常事,汝用寬民之道,當思惟念用\CJKunderline{文王}之所敬畏而法之。汝以此行寬民之政,曰:“我原惟有及於古。”則我一人天子以此悅懌汝德矣。汝惟宜勤之。 \par}

王曰:“\CJKunderline{封},爽惟民迪吉康,\footnote{明惟治民之道而善安之。}我時其惟殷先哲王德,用康乂民作求。\footnote{我是其惟殷先智王之德,用安治民,為求等。}矧今民罔迪不適,不迪則罔政在厥邦。”\footnote{治民乃欲求等殷先智王,況今民無道不之。言從教也。不以道訓之,則無善政在其國。}


{\noindent\zhuan\zihao{6}\fzbyks 傳“明惟”至“安之”。正義曰:以慎德刑為明治民之道,教之五常為善,富而不擾為安也。鄭以“迪”為下讀,各為一通也。 \par}

{\noindent\zhuan\zihao{6}\fzbyks 傳“治民”至“其國”。正義曰:以己喻\CJKunderline{康叔},言我未治之時,乃欲求等殷先智王以致太平者,況今民無道不之。言易從教。不以正道訓民,民不知道,故無善政在其國,為無吉康也。 \par}

{\noindent\shu\zihao{5}\fzkt “王曰封爽”至“厥邦”。正義曰:既言德刑事終而總言之,我所以令汝明德慎罰以施政者,王命所以言曰:“封,為人君,當明惟為治民之道而善安之,故我以是須汝善安民,故我其惟念殷先智聖王之德,用安治民,為求而等之。我於民未治之時,尚求等殷先智王,況今民無道不之而易化,汝若不以道訓之,則無善政在其國,所以須安民以德刑也。” \par}

王曰:“\CJKunderline{封},予惟不可不監,告汝德之說於罰之行。\footnote{我惟不可不監視古義,告汝施德之說於罰之所行。欲其勤德慎刑。○說如字,徐始銳反。}今惟民不靜,未戾厥心,迪屢未同,\footnote{假令今天下民不安,未定其心,於周教道屢數而未和同。設事之言。○令,力呈反。數,所角反。}爽惟天其罰殛我,我其不怨。\footnote{明惟天其以民不安罰誅我,我其不怨天。汝不治,我罰汝,汝亦不可怨我。○殛,紀力反。}惟厥罪無在大,亦無在多,矧曰其尚顯聞於天?”\footnote{民之不安,雖小邑少民,猶有罰誅,不在多大,況曰不慎罰,明聞於天者乎?言罪大。}


{\noindent\zhuan\zihao{6}\fzbyks 傳“我惟”至“慎刑”。正義曰:以敷求殷先哲王,及別求古先哲王,為己視古義也。德由說而罰須行,故德之言“說”而罰言“行”也。以事終而結上,故云德也。 \par}

{\noindent\zhuan\zihao{6}\fzbyks 傳“假令”至“之言”。正義曰:天下不安,為總說。所以不安,猶“未定其心,於周道屢數而未和同”也。時以大和會,故言“假令”,設不和同事言耳。 \par}

{\noindent\zhuan\zihao{6}\fzbyks 傳“明惟”至“怨我”。正義曰:顧氏云:“明惟天者,言天明察在上,見民不安,乃以刑罰誅戮於我。” \par}

{\noindent\zhuan\zihao{6}\fzbyks 傳“民之”至“罪大”。正義曰:此總德刑而直雲“不慎罰”者,政以德為主,不嫌不明,政失由於濫刑,故舉“罰”以言之。下言“無作怨”,以失罰為罪大。 \par}

{\noindent\shu\zihao{5}\fzkt “王曰封予”至“於天”。正義曰:以汝須善政在國,令我民安,當為政以慎德刑為教,故王又命之曰:“封,我惟不可不視古義,告汝施德之說於罰之所有。”欲其勤德慎刑也。“假令惟天下民不安,未定其心,於周教道屢數而未和同,明惟天其以民不安其罰誅我,我其不怨於天。則汝不治,是其罪,我罰汝,汝亦不可怨我。我以民之不安,惟其罰之,無在大邑,無在多民,以少猶誅罰,況曰為君不慎德刑,其上明聞於天。”是為罪大不可赦。 \par}

王曰:“嗚呼!\CJKunderline{封},敬哉!無作怨,勿用非謀、非彝,\footnote{言當修己以敬,無為可怨之事,勿用非善謀、非常法。}蔽時忱。丕則敏德,\footnote{斷行是誠道,大法敏德,信則人任焉,敏則有功。}用康乃心,顧乃德,遠乃猷,\footnote{用是誠道安汝心,顧省汝德,無令有非,遠汝謀,思為長久。}裕乃以。民寧,不汝瑕\xpinyin*{殄}。”\footnote{行寬政乃以民安,則我不汝罪過,不絕亡汝。}


{\noindent\zhuan\zihao{6}\fzbyks 傳“斷行”至“有功”。正義曰:以誠在於心,故決斷行之,亦心誠而行敏,為見事之速,事有善而須德法,故云“大法敏德”也。正以此二者,以“信則人任焉,敏則有功”故也。\CJKunderwave{論語}文。 \par}

{\noindent\zhuan\zihao{6}\fzbyks 傳“用是”至“長久”。正義曰:上文有“忱”有“敏”,此惟雲“用是誠道”,不雲“敏”者,“敏”在“誠”下,亦用之可知。 \par}

{\noindent\shu\zihao{5}\fzkt “王曰嗚呼”至“瑕殄”。正義曰:以罰不可失,故王命言曰:“嗚呼!封,當修己以敬哉!無為可怨之事,勿用非善謀、非常法,而以決斷行是誠信之道,大當法為機敏之德。用是信敏安汝心,顧省汝德,廣遠汝謀,能行寬政,乃以民安,則我不於汝罪過而絕亡汝。” \par}

王曰:“嗚呼!肆汝小子\CJKunderline{封},惟命不於常,\footnote{以民安則不絕亡汝,故當念天命之不於常,汝行善則得之,行惡則失之。}汝念哉!無我享殄,\footnote{無絕棄我言而不念。}明乃服命,\footnote{享有國土,當明汝所服行之命令,使可則。}高乃聽,用康乂民。”\footnote{高汝聽,聽先王道德之言,以安治民。}


{\noindent\zhuan\zihao{6}\fzbyks 傳“享有”至“可則”。正義曰:以“不瑕殄”,即享有國土也。“服行之命”,謂德刑也。 \par}

{\noindent\shu\zihao{5}\fzkt “王曰嗚呼肆”至“乂民”。正義曰:與上相首引。王命言曰:“嗚呼!以民安則不汝絕亡之故,汝小子封,當念天命之不於常也。惟行善則得之,行惡則失之。汝念此無常哉!無絕棄我言而不念。若享有國土,當明汝服行之教令,使可法。高大汝所聽,用先王道德之言以安治民也。” \par}

王若曰:“往哉!\CJKunderline{封},勿替敬典,\footnote{汝往之國,勿廢所宜敬之常法。}聽朕告汝,乃以殷民世享。”\footnote{順從我所告之言,即汝乃以殷民世世享國,福流後世。}

{\noindent\shu\zihao{5}\fzkt “王若”至“世享”。正義曰:以須高聽治民,故王命順其德而言曰:“汝往之國哉!封乎,勿廢所宜敬之常法,即聽用我誥是也。汝如此,則汝乃得以殷民世世享國。”而言不絕國祚,短長由德也。又言“王若曰”者,一篇終始言之,明於中亦有“若”也。 \par}

\section{酒誥第十二}


酒誥\footnote{\CJKunderline{康叔}監殷民。殷民化紂嗜酒,故以戒酒誥。○嗜,巿志反。}

{\noindent\shu\zihao{5}\fzkt 傳“\CJKunderline{康叔}”至“酒誥”。正義曰:以\CJKunderwave{梓材}雲“若茲監”,故云“\CJKunderline{康叔}監殷民”也。鄭以為“連屬之監,則為牧而言”,然\CJKunderline{康叔}時實為牧,而所戒為居殷墟,化紂餘民,不主於牧;下篇雲“監”,“監”亦指為君言之也,明“監”即國君監一國。故此言“監殷民”,不言“監一州”,若大宰之建牧立監也。 \par}

王若曰:“明大命於妹邦。\footnote{\CJKunderline{周公}以\CJKunderline{成王}命誥\CJKunderline{康叔},順其事而言之,欲令明施大教命於妹國。妹,地名,紂所都朝歌以北是。○王若,馬本作“\CJKunderline{成王}若曰”,注云:“言\CJKunderline{成王}者,未聞也。俗儒以為\CJKunderline{成王}骨節始成,故曰\CJKunderline{成王}。或曰以\CJKunderline{成王}為少成二聖之功,生號曰\CJKunderline{成王},沒因為諡。衛、賈以為戒成\CJKunderline{康叔}以慎酒,成就人之道也,故曰成。此三者吾無取焉。吾以為後錄\CJKunderwave{書}者加之,未敢專從,故曰未聞也。”妹邦,馬云:“妹邦,即牧養之地。”欲令,力呈反,下“始令”、“勿令”同。}乃穆考\CJKunderline{文王},肇國在西土。\footnote{父昭子穆,\CJKunderline{文王}弟稱穆,將言始國在西土。西土,岐周之政。○\CJKunderline{文王}弟稱穆,周自後稷而封,為始祖,后稷生不窋為昭,鞠陶為穆,\CJKunderline{公劉}為昭,慶節為穆,皇僕為昭,羌弗為穆,毀揄為昭,公非為穆,高圉為昭,亞圉為穆,諸盩為昭,\CJKunderline{大王}為穆,\CJKunderline{王季}為昭,\CJKunderline{文王}為穆。故\CJKunderwave{左傳}宮之奇云:“大伯、虞仲,\CJKunderline{大王}之昭也。虢仲、\CJKunderline{虢叔},\CJKunderline{王季}之穆也。”又富辰云,管蔡已下十六國,文之昭也。昭一音韶。窋音竹律反。揄音投。盩音張流反。大並音太。}


{\noindent\zhuan\zihao{6}\fzbyks 傳“\CJKunderline{周公}”至“北是”。正義曰:此為下之目,故言“明施大教命於妹國”。此“妹”與“沬”一也,故沬為地名,紂所都朝歌以北。但妹為朝歌之所居也,朝歌近妹邑之南,故云“以北是”。\CJKunderwave{詩}又云“沬之東矣”,“沬之鄉矣”,即東與北為鄉也。妹屬鄘,紂所都在妹,又在北與東,是地不方平,偏在鄘多故也。馬、鄭、王本以文涉三家而有“成”字,\CJKunderline{鄭玄}雲“\CJKunderline{成王}所言,成道之王”,三家雲“王年長骨節成立”,皆為妄也。 \par}

{\noindent\zhuan\zihao{6}\fzbyks 傳“父昭”至“之政”。正義曰:以“穆”連“考”,故以昭穆言之。\CJKunderline{文王}廟次為穆,以周自後稷以至\CJKunderline{文王}十五世。案\CJKunderwave{世本}云:“后稷生不窋為昭,不窋生鞠陶為穆,鞠陶生\CJKunderline{公劉}為昭,\CJKunderline{公劉}生慶節為穆,慶節生皇僕為昭,皇僕生羌弗為穆,羌弗生毀榆為昭,毀榆生公飛為穆,公飛生高圉為昭,高圉生亞圉為穆,亞圉生組紺為昭,組紺生\CJKunderline{大王}亶父為穆,亶父生季歷為昭,季歷生\CJKunderline{文王}為穆。”據世次偶為穆也。\CJKunderwave{左傳}曰“大伯、虞仲,\CJKunderline{大王}之昭”,言\CJKunderline{大王}為穆,而子為昭。又曰“虢仲、\CJKunderline{虢叔},\CJKunderline{王季}之穆”,亦\CJKunderline{王季}為昭而子為穆,與\CJKunderline{文王}同穆也。又管、蔡、郕、霍等十六國亦曰\CJKunderline{文王}之昭,則以\CJKunderline{文王}為穆,其子與\CJKunderline{武王}為昭。又曰“邗晉應韓,武之穆”,以繼\CJKunderline{武王}為昭也。“將言始國在西土。西土,岐周之政”者,據今本先故言“始”,為初始為政,然則居豐前,故云“西土”,欲將言道\CJKunderline{文王}誥毖庶邦以下之政,故先本之雲“肇國在西土”。 \par}

厥誥毖庶邦、庶士越少正、御事,朝夕曰:‘祀茲酒。’\footnote{文王其所告慎眾國眾士於少正官、御治事吏,朝夕敕之:“惟祭祀而用此酒,不常飲。”○毖音秘。少,詩照反。}惟天降命,肇我民,惟元祀。\footnote{惟天下教命,始令我民知作酒者,惟為祭祀。○為,於偽反,下同。}天降威,我民用大亂喪德,亦罔非酒惟行。\footnote{天下威罰,使民亂德,亦無非以酒為行者。言酒本為祭祀,亦為亂行。○惟行,下孟反,注及下注“之行”同。}越小大邦用喪,亦罔非酒惟辜。\footnote{於小大之國所用喪亡,亦無不以酒為罪也。}

{\noindent\zhuan\zihao{6}\fzbyks 傳“\CJKunderline{文王}”至“常飲”。正義曰:告敕使之敬慎,故曰“告慎其眾國”,即眾多國君。“眾士”,朝臣也。既總呼為“士”,則卿大夫俱在內。少正、御治事以其卑賤,更別目之。“朝夕敕之”,丁寧慎之至也。 \par}

{\noindent\zhuan\zihao{6}\fzbyks 傳“惟天”至“祭祀”。正義曰:\CJKunderwave{世本}云,儀狄造酒,夏禹之臣,又云杜康造酒,則人自意所為。言“天下教命”者,以天非人,不因人為者,亦天之所使,故凡造立皆雲本之天。“元祀”者,言酒惟用於大祭祀,見戒酒之深也。顧氏云:“元,大也。\CJKunderwave{洛誥}‘稱秩元祀’,孔以為‘舉秩大祀’。”大劉以“元”為始,誤也。 \par}

{\noindent\zhuan\zihao{6}\fzbyks 傳“天下”至“亂行”。正義曰:民自飲酒致亂,以被威罰,言“天下威”者,亦如上言天之下教命,令民作酒也。為亂而罪,天理當然,故曰“天討有罪,五刑五用哉”。俗本雲“不為亂行”,定本雲“亦為亂行”,俗本誤也。 \par}

{\noindent\zhuan\zihao{6}\fzbyks 傳“於小”至“為罪也”。正義曰:“小大之國”,謂諸侯之國有小大也。上言“民用大亂”,指其身為罪。此言“邦用喪”,言其邦國喪滅。上文總謂貴賤之人,此則專指諸侯之身故也。惟行用酒,惟罪身得罪,亦互相通也。 \par}

{\noindent\shu\zihao{5}\fzkt “王若”至“惟辜”。正義曰:\CJKunderline{周公}以王命誥\CJKunderline{康叔},順其事而言曰:“汝當明施大教命於妹國而戒之以酒。所以須戒酒者,以汝父於廟以穆考\CJKunderline{文王},始國在西土岐周為政也。其誥慎所職眾國眾士於少正官、御治事吏,朝夕敕之曰:‘惟祭祀而用此酒,不常為飲也。’所以不常為飲者,以惟天之下教命,始令我民知作酒者,惟為大祭祀,故以酒為祭,不主飲。故天下威罰於我民,用使之大為亂,以喪其德,亦無非以酒為行而用之。故於小大之國,用使之喪亡,亦無非以酒為罪,以此眾事少正,皆須戒酒也。是\CJKunderline{文王}以酒為重戒,汝不可不法也。” \par}

\CJKunderline{文王}誥教小子有正有事,無彝酒。\footnote{小子,民之子孫也。正官治事,謂下群吏。教之皆無常飲酒。}越庶國,飲惟祀,德將無醉。\footnote{於所治眾國,飲酒惟當因祭祀,以德自將,無令至醉。}惟曰我民迪,小子惟土物愛,厥心臧。\footnote{文王化我民,教道子孫,惟土地所生之物皆愛惜之,則其心善。}聰聽祖考之彝訓,越小大德,小子惟一。\footnote{言子孫皆聰聽父祖之常教,於小大之人皆念德,則子孫惟專一。}


{\noindent\zhuan\zihao{6}\fzbyks 傳“小子”至“飲酒”。正義曰:知“小子”謂民之子孫者,以下文二“我民迪小子”,又云“奔走事厥考厥長”,故知“小子”謂民之子孫也。知“有正有事”非士大夫,而云“正官治事,謂下群吏”者,以文與“小子”相連,故知是正官下治事之群吏。 \par}

{\noindent\zhuan\zihao{6}\fzbyks 傳“於所”至“至醉”。正義曰:以述上文內外雙舉,此為小子及民與士大夫可知。其外宜有國君,故下雲指戒\CJKunderline{康叔}為國之事,故總言“眾國”。惟於祭祀得飲酒,猶以德自將,無令至醉。\CJKunderwave{大傳}因此言“宗室將有事,族人皆入侍”,得有醉與不醉而出與不出之事。而以德自將,無令至醉,亦一隅之驗。\CJKunderline{文王}為諸侯而云“眾國”者,\CJKunderline{文王}為西伯,又三分有二諸侯,故得戒眾國也。 \par}

{\noindent\zhuan\zihao{6}\fzbyks 傳“\CJKunderline{文王}”至“心善”。正義曰:以“惟曰”為教辭,故言“\CJKunderline{文王}化我”。民愛惜土物而不損耗,則不嗜酒,故心善。 \par}

{\noindent\shu\zihao{5}\fzkt “\CJKunderline{文王}”至“惟一”。正義曰:前\CJKunderline{文王}戒酒,以為所供當重飲之,則有滅亡之害。此更戒之,令以德自將,不可常飲。故又云,\CJKunderline{文王}誥教其民之小子與正官之下有職事之人。謂群吏。汝等無得常飲酒也。於所治眾國之君臣民眾等,言飲酒惟當因祭祀,以德自將,無令至醉。又自申\CJKunderline{文王}之教小子者,不但身自教之,又化民使自教其子弟。惟教其民曰:“惟我民等,當教道子孫小子,令土地所生之物,皆愛惜之,則其心善矣。”以愛物,則不為酒而損耗故也。既父祖稟\CJKunderline{文王}之教以化其子孫,而子孫能聰審聽用祖考之常訓。言愛物以戒酒也。不但民之小子為然,其於小大德之士大夫等,亦皆能念行\CJKunderline{文王}之德以教其子孫,故子孫亦聰聽之。小子惟皆專一而戒其酒,其民及在位,不問貴賤,子孫皆化,則至成長為德可知也。 \par}

妹土嗣爾股肱純,其藝黍稷,奔走事厥考厥長。\footnote{今往當使妹土之人繼汝股肱之教,為純一之行,其當勤種黍稷,奔走事其父兄。○長,丁丈反,下注“長官諸侯”之長同。}肇牽車牛,遠服賈,用孝養厥父母。\footnote{農功既畢,始牽車牛,載其所有,求易所無,遠行賈賣,用其所得珍異孝養其父母。○賈音古。養,牛亮反。}厥父母慶,自洗腆,致用酒。\footnote{其父母善子之行,子乃自絜厚,致用酒養也。}庶士、有正越庶伯、君子,其爾典聽朕教。\footnote{眾伯君子、長官大夫、統庶士有正者,其汝常聽我教,勿違犯。}


{\noindent\zhuan\zihao{6}\fzbyks 傳“今往”至“父兄”。正義曰:以妹土為所封之都,故言“今往”。“繼汝股肱之教”者,君為元首,臣作股肱,君倡臣行,施由股肱,故言繼其教也。言“奔走”者,顧氏云:“勤種黍稷,奔馳趨走也。” \par}

{\noindent\zhuan\zihao{6}\fzbyks 傳“農功”至“父母”。正義曰:若當農功,則有所廢,故知既畢乃行,故云“始牽車牛”,即牽將大車,載有易無,遠求盈利,所得珍異而本不損,故可孝養其父母,亦愛土物之義也。 \par}

{\noindent\zhuan\zihao{6}\fzbyks 傳“其父”至“酒養”。正義曰:以人父母欲家生之富者,若非盈利,雖得其養,有喪家資,則父母所不善。今勤商得利,富而得養,所以善子之行也。 \par}

{\noindent\zhuan\zihao{6}\fzbyks 傳“眾伯”至“違犯”。正義曰:眾伯君子,統眾士有正者,經雲“庶士有正”者,戒其慎酒,從卑至尊,故先教子孫,乃及庶士眾百君子。 \par}

爾大克羞耇,惟君,爾乃飲食醉飽。\footnote{汝大能進老成人之道,則為君矣。如此汝乃飲食醉飽之道。先戒群吏以聽教,次戒\CJKunderline{康叔}以君義。}丕惟曰,爾克永觀省,作稽中德。\footnote{我大惟教汝曰,汝能長觀省古道,為考中正之德,則君道成矣。○省,悉井反。}爾尚克羞饋祀,爾乃自介用逸。\footnote{能考中德,則汝庶幾能進饋祀於祖考矣。能進饋祀,則汝乃能自大用逸之道。}茲乃允惟王正事之臣,\footnote{汝能以進老成人為醉飽,考中德為用逸,則此乃信任王者正事之大臣。○任音壬。}茲亦惟天若元德,永不忘在王家。”\footnote{言此非但正事之臣,亦惟天順其大德而佑之,長不見忘在王家。}

{\noindent\zhuan\zihao{6}\fzbyks 傳“汝大”至“君義”。正義曰:\CJKunderwave{釋詁}云:“羞,進也。”既以慎酒立教,是大能進行老成人之道,是惟可為人君矣。以人君若治不得,有所民事可憂,雖得酒食,不能醉飽。若能進德,民事可乎,故為飲食可醉飽之道。以群臣言,“聽教”即為臣義,不過慎酒進德,次戒\CJKunderline{康叔}以君義,亦有“聽教”,明為互矣。 \par}

{\noindent\zhuan\zihao{6}\fzbyks 傳“我大”至“成矣”。正義曰:以言“曰”,故以為教辭,即教以“大克羞者”。長省古道,是老成人之德,考其中正,是能大進行,可以惟為君,故云“則君道成矣”。 \par}

{\noindent\zhuan\zihao{6}\fzbyks 傳“能考”至“之道”。正義曰:以聖人為能饗帝,孝子為能饗親,考德為君,則人治之,已成民事,可以祭神,故考中德,能進饋祀於祖考。人愛神助,可以無為,故大用逸之道,即上雲“飲食醉飽之道”也。鄭以為助祭於君,亦非其義勢也。以下然並亦惟天據人事,是惟王正事大臣,本天理,故天順其大德,不見忘在於王家,反覆相成之勢也。 \par}

{\noindent\shu\zihao{5}\fzkt “妹土”至“王家”。正義曰:既上言\CJKunderline{文王}之教,今指戒\CJKunderline{康叔}之身,實如汝當法,\CJKunderline{文王}斷酒之法故今往當使妹土之人繼爾股肱之,教為純一之行。其當勤於耕種黍稷,奔馳趨走供事其父與兄。其農功既畢,始牽車牛遠行賈賣,用其所得珍異孝養其父母,以子如此,善子之行,子乃自洗潔,謹敬厚致用酒以養,此亦小子土物愛也。又謂汝眾士有正之人,及於眾伯君子長官大夫統眾士有正者,其汝亦常聽用我斷酒之教,勿違犯也。汝\CJKunderline{康叔}大能進行老成人之道,則惟可為君矣。如此汝乃為飲食醉飽之道。由須進行老成人,故我大惟教汝曰:“汝能長觀省古道,所為考行中正之德,即是進行老成人,惟堪為君。能考中德,用汝庶幾能進饋祀於祖考矣。以能進饋祀,人神所助,則汝乃能自大用逸之道。如此用逸,則乃信惟王正事之大臣。不但正事大臣,如此亦惟天順其大德而佑助之,長不見遺忘在王家矣。可不務乎?” \par}

王曰:“\CJKunderline{封},我西土棐徂邦君御事,小子尚克用\CJKunderline{文王}教,不腆於酒。\footnote{我\CJKunderline{文王}在西土,輔訓往日國君及御治事者、下民子孫,皆庶幾能用上教,不厚於酒。言不常飲。}故我至於今,克受殷之命。”\footnote{以不厚於酒,故我周家至於今能受殷王之命。}


{\noindent\zhuan\zihao{6}\fzbyks 傳“我文”至“常飲”。正義曰:“棐”,輔也。“徂”,往也。以事已過,故言“往日”。恐嗜酒不成其德,故以斷酒輔成之。其“御事”謂國君之下眾臣也。“不厚於酒”即“無彝酒”也,故云“不常飲”,總述上也。 \par}

{\noindent\shu\zihao{5}\fzkt “王曰封我西”至“之命”。正義曰:於此乃總言不可不用\CJKunderline{文王}慎酒之教,王命之曰:“封,我\CJKunderline{文王}本在西土,以道輔訓往日國君及治事之臣大夫士與其民之小子,其此等皆庶幾能用\CJKunderline{文王}教,而不厚於酒。故我周家至於今能受殷之王命。以此故,不可不用其教以斷酒。” \par}

王曰:“\CJKunderline{封},我聞惟曰,在昔殷先哲王,迪畏天顯小民,\footnote{聞之於古。殷先智王,謂湯蹈道畏天,明著小民。}經德秉哲,自\CJKunderline{成湯}咸至於\CJKunderline{帝乙},成王畏相。\footnote{能常德持智,從湯至\CJKunderline{帝乙}中間之王猶保成其王道,畏敬輔相之臣,不敢為非。○相,息亮反,下同。}惟御事厥棐有恭,不敢自暇自逸,\footnote{惟殷御治事之臣,其輔佐畏相之君,有恭敬之德,不敢自寬暇,自逸豫。○暇,遐嫁反。}矧曰其敢崇飲?\footnote{崇,聚也。自假自逸猶不敢,況敢聚會飲酒乎?明無也。}越在外服,侯、甸、男、衛邦伯,\footnote{於在外國侯服、甸服、男服、衛服國伯諸侯之長。言皆化湯畏相之德。}

{\noindent\zhuan\zihao{6}\fzbyks 傳“聞之”至“小民”。正義曰:言“聞之於古”,是事明眾見也。下言“自\CJKunderline{成湯}”,知此別道湯事也。王者上承天,下恤民,皆由蹈行於為,畏天之罰已故也。又以道教民,故明德著小民。 \par}

{\noindent\zhuan\zihao{6}\fzbyks 傳“能常”至“為非”。正義曰:德在於身,智在於心,故能常德持智,即上迪畏天,顯小民,為自湯後皆爾。 \par}

{\noindent\zhuan\zihao{6}\fzbyks 傳“惟殷”至“逸豫”。正義曰:此事當公卿,故下別雲“越在內服百僚庶尹”也。為君畏相,故輔之。若寬暇與逸豫,則不恭敬,故不敢為也。 \par}

{\noindent\zhuan\zihao{6}\fzbyks 傳“崇聚”至“明無”。正義曰:\CJKunderwave{釋詁}云:“崇,充也。”充實則集聚,故“崇”為聚也。飲必待暇逸,猶尚不敢暇逸,故言“況敢聚集飲酒乎?明無也”。 \par}

{\noindent\zhuan\zihao{6}\fzbyks 傳“於在”至“之德”。正義曰:以公卿與國為體,承君共事,故先言之。然後見廣,故自外及內,舉四者以總六服,又因“衛”為蕃衛,故不言“採”也。“國”謂國君,“伯”言長,連、屬、卒、牧皆是,見遍在外為君,故言“化湯畏相之德”。 \par}

越在內服,百僚、庶尹、惟亞、惟服、宗工,\footnote{於在內服治事百官眾正及次大夫服事尊官,亦不自逸。}越百姓里居,\footnote{於百官族姓及卿大夫致仕居田裡者。}罔敢湎於酒。不惟不敢,亦不暇。\footnote{自外服至里居,皆無敢沈湎於酒。非徒不敢,志在助君敬法,亦不暇飲酒。○湎,面善反。}惟助\CJKunderline{成王}德顯,越尹人祇闢。\footnote{所以不暇飲酒,惟助其君\CJKunderline{成王}道,明其德於正人之道,必正身敬法,其身正,不令而行。闢,扶亦反。}

{\noindent\zhuan\zihao{6}\fzbyks 傳“於在”至“自逸”。正義曰:畿外有服數,畿內無服數,故為“服治事”也。言“百官眾正”,為總之文。但百官眾正除六卿亦有大夫及士,士亦有官首而為政者。“惟亞”,傳雲“次大夫”者,謂雖為大夫不為官首者,亞次官首,故云“亞”。舉大夫尊者為言,其實士亦為亞次之官。必知“惟亞”兼士者,以此經文上下更無別見士之文,故知兼之。“惟服宗工”,總上“百僚庶尹”及“惟亞”,言服治職事尊官之故,亦不自逸。“惟亞”雖不為官首,亦助上服治政事,或可非官首者服事在上之尊官,亦不自逸。 \par}

{\noindent\zhuan\zihao{6}\fzbyks 傳“於百”至“裡者”。正義曰:每言“於”者,繼上君與御事為“於”。此不言“在”,從上“內服”故也。“百官族姓”謂其每官之族姓,而與“里居”為總,故云“卿大夫致仕居田裡者”也。 \par}

{\noindent\zhuan\zihao{6}\fzbyks 傳“自外”至“飲酒”。正義曰:自外服至里居,皆無敢沈湎,亦上御事,雲“亦不暇”,不暇則不逸可知,助君敬法,逆探下經也。 \par}

{\noindent\shu\zihao{5}\fzkt “王曰封我聞”至“祇闢”。正義曰:以周受於殷,\CJKunderline{文王}之前殷代也,今又衛居殷地,故舉殷代以酒興亡得失而為戒。王命之曰:“封,我聞於古,所聞惟曰,殷之先代智道之王\CJKunderline{成湯},於上蹈道以畏天威,於下明著加於小民,即能常德持智以為政教。自\CJKunderline{成湯}之後皆然,以至於\CJKunderline{帝乙},猶保成其王道,畏敬輔相之臣。其君既然,惟殷御治事之臣,其輔相於君,有恭敬之德,不敢自寬暇,自逸豫,況曰其敢聚會群飲酒乎?於是在外之服侯、甸、男、衛、國君之長,於是在內之服治事百官眾正惟次大夫惟服事尊官,於百官族姓及致仕在田裡而居者,皆無敢沈湎於酒。不惟不敢,亦自不暇飲。所以不暇者,惟以助其君成其王道,令德顯明;又於正人之道,必正身敬法,正身以化下,不令而行,故不暇飲。是亦可以為法也。” \par}

我聞亦惟曰,在今後嗣王酣身,\footnote{嗣王,紂也。酣樂其身,不憂政事。○酣,戶甘反。樂音洛。}厥命罔顯於民,祇保越怨不易。\footnote{言紂暴虐,施其政令於民,無顯明之德,所敬所安,皆在於怨,不可變易。○易如字,馬以豉反。}誕惟厥縱淫泆於非彝,用燕喪威儀,民罔不\xpinyin*{衋}傷心。\footnote{紂大惟其縱淫泆於非常,用燕安喪其威儀,民無不衋然痛傷其心。○縱,子用反,注同。泆音溢,又作逸,亦作佚。衋,許力反。}惟荒腆於酒,不惟自息乃逸,\footnote{言紂大厚於酒,晝夜不念自息,乃過差。○差,初佳反,又初賣反。}厥心疾很,不克畏死。\footnote{紂疾很其心,不能畏死。言無忌憚。○很,胡懇反。}辜在商邑,越殷國滅無罹。\footnote{紂聚罪人在都邑而任之,於殷國滅亡無憂懼。}弗惟德馨香,祀登聞於天,誕惟民怨。\footnote{紂不念發聞其德,使祀見享,升聞於天,大行淫虐,惟為民所怨咎。}庶群自酒,腥聞在上,故天降喪於殷,罔愛於殷,惟逸。\footnote{紂眾群臣用酒沈荒,腥穢聞在上天,故天下喪亡於殷,無愛於殷,惟以紂奢逸故。○聞音問。}天非虐,惟民自速辜。”\footnote{言凡為天所亡,天非虐民,惟民行惡自召罪。}


{\noindent\zhuan\zihao{6}\fzbyks 傳“言紂”至“變易”。正義曰:“施其政令於民,無顯明之德”,言所施者皆是闇亂之政也。紂意謂之為善,所敬之所安之者,及其施行,皆是害民之事,為民所怨。紂之為惡,執心堅固,不可變易也。 \par}

{\noindent\zhuan\zihao{6}\fzbyks 傳“紂大”至“其心”。正義曰:“誕”訓為大,言紂大惟其縱淫泆於非常之事。 \par}

{\noindent\zhuan\zihao{6}\fzbyks 傳“紂眾”至“逸故”。正義曰:“紂眾群臣用酒沈荒”,“用”者解經之“自”。定本作“自”,俗本多誤為“嗜”。 \par}

{\noindent\zhuan\zihao{6}\fzbyks 傳“言凡”至“召罪”。正義曰:此言“惟人”,謂紂也。今變言“人”者,見雖非紂亦然。 \par}

{\noindent\shu\zihao{5}\fzkt “我聞”至“速辜”。正義曰:既言\CJKunderline{帝乙}以上慎酒以存,故又言紂嗜酒而滅:“我聞亦惟曰,殷之在今\CJKunderline{帝乙}後嗣之謂紂王,酣樂其身,不憂於政事,施其政令,無顯明之德於民,所敬所安,皆在於怨,不可變易。大惟其縱淫泆於非常,用燕安之故,喪其威儀,民見之無不衋然痛傷其心也。皆由惟大愛厚於酒,晝夜不念自止息,乃過逸。其內心疾害很戾,不能畏死。聚罪人在商邑而任之,於殷國滅亡無憂懼也。不念發聞其德令之馨香,使祀見享,升聞於天,大惟行其淫虐,為民下所怨。紂眾群臣集聚用酒荒淫,腥穢聞在上天,故天下喪亡於殷,無愛念於殷,惟以紂奢逸故。非天虐殷以滅之,惟紂為人自召此罪故也。” \par}

王曰:“\CJKunderline{封},予不惟若茲多誥。\footnote{我不惟若此多誥汝,我親行之。}古人有言曰:‘人無於水監,當於民監。’\footnote{古賢聖有言,人無於水監,當於民監。視水見己形,視民行事見吉凶。○監,工陷反,下及注同。}今惟殷墜厥命,我其可不大監撫於時?\footnote{今惟殷紂無道,墜失天命,我其可不大視此為戒,撫安天下於是?}

{\noindent\shu\zihao{5}\fzkt “王曰封予”至“於時”。正義曰:既陳殷之戒酒與嗜酒以致興亡之異,故誥之,王命言曰:“封,我不惟若此徒多出言以誥汝而已,我自戒酒,己親行之,汝可法之也。所以親行者,古人有言曰:‘人無於水監,當於民監。’以水監但見己形,以民監知成敗故也。以須民監之故,今殷紂無道,墜失其天命,我其可不大視以為戒,撫安天下於今時也?” \par}

予惟曰,汝劼毖殷獻臣,\footnote{劼,固也。我惟告汝曰,汝當固慎殷之善臣信用之。○劼,若八反。}侯、甸、男、衛,矧太史友,內史友、\footnote{侯、甸、男、衛之國當慎接之,況太史、內史掌國典法所賓友乎?}越獻臣、百宗工,矧惟爾事服休、服採?\footnote{於善臣百尊官不可不慎,況汝身事服行美道,服事治民乎?}



{\noindent\zhuan\zihao{6}\fzbyks 傳“劼固”至“用之”。正義曰:“劼,固”,\CJKunderwave{釋詁}文。將欲斷酒為重,故節文以相況。“毖”訓為慎,言誠堅固謹慎,皆敬而釋任之,其文通於下,皆固慎。 \par}

{\noindent\zhuan\zihao{6}\fzbyks 傳“侯甸”至“賓友乎”。正義曰:太史掌國六典,依\CJKunderwave{周禮},治典、教典、禮典、政典、刑典、事典也。內史掌八柄之法者,爵、祿、廢、置、殺、生、與、奪。此“太史”、“內史”即\CJKunderline{康叔}之國大夫,知者,以下“圻父”、“農父”、“宏父”是諸侯之三卿,明“太史”、“內史”非王朝之官。“所賓友”者,敬也。 \par}

{\noindent\zhuan\zihao{6}\fzbyks 傳“於善”至“民乎”。正義曰:“於善臣”即上經“殷獻臣”也。“百尊官”即上“侯甸男衛”、“太史”、“內史”也。“服行美道,服事治民”即上汝之身事。知“服事”是治民者,民惟邦本,諸侯治民為事故也。\CJKunderline{鄭玄}以“服休”為燕息之近臣,“服採”為朝祭之近臣,非孔意也。 \par}

矧惟若疇圻父,薄違農父?\footnote{圻父,司馬。農父,司徒。身事且宜敬慎,況所順疇諮之司馬乎?況能迫回萬民之司徒乎?言任大。○圻,臣依反。父音甫。薄,蒲各反,徐又扶各反。違如字,徐音回,馬云:“違行也。”}若保宏父,定闢,矧汝剛制於酒?\footnote{宏,大也。宏父,司空。當順安之。司馬、司徒、司空,列國諸侯三卿,慎擇其人而任之,則君道定,況汝剛斷於酒乎?○闢,必亦反。斷,丁亂反。}

{\noindent\zhuan\zihao{6}\fzbyks 傳“圻父”至“任大”。正義曰:司馬主圻封,故云“圻父”。“父”者,尊之辭。以司徒教民五土之藝,故言“農父”也。以司馬征伐在乎閫外所專,故隨順而疇諮之,言君所順疇也。迫近迴繞於萬民,言近民事也。二者皆任大。 \par}

{\noindent\zhuan\zihao{6}\fzbyks 傳“宏大”至“酒乎”。正義曰:“宏,大”,\CJKunderwave{釋詁}文。以司空亦君所順所安和之,故言“當順安之”。諸侯之三卿,以上有司馬、司徒,故知“宏父”是司空。言大父者,以營造為廣大國家之父。因節文而分之,乃總之言“司馬、司徒、司空”。列國三卿,令慎擇其人而任之,則君道定,況剛斷於酒乎?為甚之義也。其“定闢”總上自“劼毖殷獻”已下,獨言三卿者,因文相況而接之,其實總上也。三卿不次者,以司馬征伐為重;次以政教安萬民,司徒為重;司空直指營造,故在下也。司徒言於萬民為迫回者,事務為主故也。司徒不言“若”者,互相明,皆為治民,而君所順也。 \par}

{\noindent\shu\zihao{5}\fzkt “予惟”至“於酒”。正義曰:殷之存亡既可以為監若是,故我惟告汝曰:“汝當堅固愛慎殷之善臣及侯、甸、男、衛之君,則在外尚然,況已下太史所賓友,內史所賓友,於善臣百尊官而不固慎乎?此之卑官猶尚固慎,況惟汝之身事所服行美道,服行美事治民,而可不固慎乎?於己身事猶當固慎,況惟所敬順疇諮之圻父,能迫回萬民之農父,所順所安之宏父?此等大臣能得固慎,則可定其為君之道,固慎大臣,雖非急要,尚能使君道得定,況汝又能剛斷於酒乎?善所莫大,不可加也。” \par}

厥或誥曰:‘群飲。’汝勿佚。\footnote{具有誥汝曰:“民群聚飲酒。”不用上命,則汝收捕之,勿令失也。}盡執拘以歸於周,予其殺。\footnote{盡執拘群飲酒者以歸於京師,我其擇罪重者而殺之。○盡,子忍反。}又惟殷之迪諸臣惟工,乃湎於酒,勿庸殺之,\footnote{又惟殷家蹈惡俗諸臣,惟眾官化紂日久,乃沈湎於酒,勿用法殺之。○惡,烏各反。}姑惟教之,有斯明享。\footnote{以其漸染惡俗,故必三申法令,且惟教之,則汝有此明訓以享國。○三,息暫反,又如字。}乃不用我教辭,惟我一人弗恤,弗蠲乃事,時同於殺。”\footnote{汝若忽怠不用我教辭,惟我一人不憂汝,乃不潔汝政事,是汝同於見殺之罪。}


{\noindent\zhuan\zihao{6}\fzbyks 傳“盡執”至“殺之”。正義曰:言“周”,故為“京師”。但飲有稀數,罪有大小,不可一皆盡殺,故知“擇罪重者殺之”。 \par}

{\noindent\zhuan\zihao{6}\fzbyks 傳“又惟”至“殺之”。正義曰:言“諸臣”,謂尊者,及其下列職眾官,不可用法殺之,明法有張弛。此由殷之諸臣,漸染紂之惡俗日久,故不可即殺。其衛國之民,先非紂之舊臣,乃群聚飲酒,恐增長昏亂,故擇罪重者殺之。據意不同,故殺否有異。 \par}

{\noindent\zhuan\zihao{6}\fzbyks 傳“以其”至“享國”。正義曰:禮成於三,故必三申法令。“有此明訓”,總上之辭,故得享國。 \par}

{\noindent\zhuan\zihao{6}\fzbyks 傳“汝若”至“之罪”。正義曰:汝不用我教辭,則不足憂念,故“惟我一人不憂汝”。“不絜汝之政事”,事惟穢惡,不復教之使絜靜也。 \par}

{\noindent\shu\zihao{5}\fzkt “厥或”至“於殺”。正義曰:以為政莫重於斷酒,故其有人誥汝曰:“民今飲酒,相與群聚。”是不用上命,則汝收捕之,勿令失矣。盡執拘以歸於周之京師,我其擇罪重而殺之也。又惟殷之蹈惡俗諸臣,惟其眾官化紂日久,乃沈湎於酒,勿用法殺之。以漸染惡俗,故三申法令,且惟教之,則汝有此明訓,可以享國。汝若不用我教辭,惟我一人天子不憂汝,不潔汝政事,是汝同於見殺之罪,不可不慎。 \par}

王曰:“\CJKunderline{封},汝典聽朕毖,\footnote{汝當常聽念我所慎而篤行之。}勿辯乃司民湎於酒。”\footnote{辯,使也。勿使汝主民之吏湎於酒。言當正身以帥民。}

{\noindent\shu\zihao{5}\fzkt “王曰封”至“於酒”。正義曰:以戒酒事終,故結之。王命言曰:“封,汝當常聽念我所使汝慎者,篤而行之。勿使汝主民之吏若宰人者沈湎於酒,當正身以帥民。” \par}

\section{梓材第十三}


梓材\footnote{告\CJKunderline{康叔}以為政之道,亦如梓人治材。○梓音子,本亦作杼,馬云:“古作梓字。治木器曰梓,治土器曰陶,治金器曰冶。”}

{\noindent\shu\zihao{5}\fzkt 傳“告康”至“治材”。正義曰:此取下言“若作梓材,既勤樸斫”,故云“為政之道,如梓人治材”。此古“杍”字,今文作“梓”。“梓”,木名,木之善者,治之宜精,因以為木之工匠之名。下有“稽田”、“作室”,乃言“梓材”,三種獨用“梓材”者,雖三者同喻,田在於外,室總於家,猶非指事之器,故取“梓材”以為功也。因戒德刑與酒事終,言治人似治器而結之故也。 \par}

王曰:“\CJKunderline{封},以厥庶民暨厥臣達大家,\footnote{言當用其眾人之賢者與其小臣之良者,以通達卿大夫及都家之政於國。○暨,其器反。}以厥臣達王惟邦君。\footnote{汝當信用其臣以通王教於民。言通民事於國,通王教於民,惟乃國君之道。}汝若恆越曰:‘我有師師、\footnote{汝惟君道使順常,於是曰:“我有典常之師可師法。”}司徒、司馬、司空、尹旅曰:‘予罔厲殺人。’\footnote{言國之三卿、正官眾大夫皆順典常,而曰:“我無厲虐殺人之事。”如此則善矣。}


{\noindent\zhuan\zihao{6}\fzbyks 傳“言當”至“於國”。正義曰:“以”,用也。“暨”,與也。言“用”,通“厥臣”可用,明此皆賢與良也。“厥臣”文在“大家”之上,故知“小臣”也。言用之者,既用其言以為政,又用其人以為輔,本之得大家所用統之,即君所遣也。以大夫稱家,對士庶有家而非大,故云“大家”,卿大夫在朝者。“都家”亦卿大夫所得邑也,又公邑而大夫所治亦是也。用此以行政令,上達於國,使人君知之也。即是庶人升為士,又用庶人進在官者,小臣亦得進等而用之。\CJKunderwave{周禮}有都家之官,鄭云:“都謂王子弟所封及公卿所食邑,家謂大夫所食采地。”傳以“大家”言之,總包大臣,故言“卿大夫及都家之政”。卿大夫之政謂在朝所掌者,都家之政謂采邑所有政事,二者並當通達之於國,故連言之。 \par}

{\noindent\zhuan\zihao{6}\fzbyks 傳“汝當”至“之道”。正義曰:言汝當信用臣,即信用卿大夫及都家,自然大家也。傳用小臣與庶人,故得“通王教於民”也。人君上承於王,下治民事,故交通其政,“惟乃國君之道”而已。鄭以“於邑言達大家,於國言達王與邦君,王為二王之後”,即亂名實也。 \par}

{\noindent\zhuan\zihao{6}\fzbyks 傳“汝惟”至“師法”。正義曰:即上民事王教通於國人,是順常也,故總上“惟邦君”,言“汝惟君道使順常”也。“典常可師”即順常也。 \par}

{\noindent\zhuan\zihao{6}\fzbyks 傳“言國”至“善矣”。正義曰:此連上蒙“若恆”之文,故云“國之三卿、正官眾大夫皆順典常”也。不言“士”,從可知也。此曰“予罔厲殺人”,所謂令\CJKunderline{康叔}之語,但在臣下,宜為此也。以上令下行,行之在臣,故云“我無厲虐殺人之事”,互明君及臣皆師法而無虐。 \par}

亦厥君先敬勞,肆徂厥敬勞。\footnote{亦其為君之道,當先敬勞民,故汝往治民,必敬勞來之。○勞,力報反,下同。來,力代反。}肆往,奸宄、殺人、歷人,宥。\footnote{以民當敬勞之故,汝往之國,又當詳察奸宄之人及殺人賊,所過歷之人,有所寬宥,亦所以敬勞之。○宄音軌。}肆亦見厥君事,戕敗人,宥。\footnote{聽訟折獄,當務從寬恕,故往治民,亦當見其為君之事,察民以過誤殘敗人者,當寬宥之。○見如字,徐賢遍反。戕敗,徐在羊反,又七良反,馬云:“殘也。”折,之舌反。}

{\noindent\zhuan\zihao{6}\fzbyks 傳“亦其”至“來之”。正義曰:“亦其為君之道”者,為邦君之道,非直順常,亦須敬勞,故往必敬勞,即\CJKunderwave{論語}雲“先之,勞之”是也。 \par}

{\noindent\zhuan\zihao{6}\fzbyks 傳“以民”至“勞之”。正義曰:上文無罪敬勞,此惟就有罪者原情免宥,亦敬勞也。其實“奸宄”不殺人者,“殺人”亦是奸宄,但重言而別其文。奸宄及殺人,二者並是賊害,自當合罪,不可寬宥。其所過歷之人,情所不知,故詳察寬宥,以為敬勞之。 \par}

{\noindent\zhuan\zihao{6}\fzbyks 傳“聽訟”至“宥之”。正義曰:以君者立於無過之地,使物不失其所,故宥罪原情,當見其為君之事,與上“厥君”始終相承。於“奸”上言“肆往”,此亦以罪事往可知也。言“宥”,明情亦可原,故知“過誤殘敗人”也。 \par}

{\noindent\shu\zihao{5}\fzkt “王曰”至“人宥”。正義曰:王曰:“封,汝為政,當用其眾人之賢者與其小臣之良者,以通達卿大夫及都家等大家之政於國,然後汝當信用其臣以通達王教於民,惟乃可為國君之道。汝為君道,故當使上下順常,於是曰:‘我有典常之師可師法。’是君之順典常也。其下司徒、司馬、司空國之三卿,及正官眾大夫亦皆順典常,而曰:‘我無虐厲殺人之事。’是使臣之順常也。如此君臣皆能順常,則為善矣。為君之道,非但順常,亦須敬勞之。故云亦其為君之道,當先敬心以愛勞民。故汝往治民,必敬勞之。又以民須敬勞之故,汝往之國,詳察其奸宄及殺人之人,二者所過歷之人,原情不知,有所寬宥。以斷獄務從寬,故汝往治亦當見其為君之事,而民有過誤殘敗人者,當寬宥之,此亦為敬勞之也。” \par}

王啟監,厥亂為民。\footnote{言王者開置監官,其治為民,不可不勉。○監,工暫反,劉工銜反,下同。為,於偽反,注同。治,直吏反。}曰:‘無胥戕,無胥虐,至於敬寡,至於屬婦,合由以容。’\footnote{當教民無得相殘傷,相虐殺,至於敬養寡弱,至於存恤妾婦,和合其教,用大道以容之,無令見冤枉。○屬婦,上音蜀,妾之事妻也。令,力呈反,篇末同。冤,紆元反,一本作以冤。}王其效邦君,越御事,厥命曷以?\footnote{王者其效實國君,及於御治事者,知其教命所施何用,不可不勤。}‘引養引恬’,自古王若茲監,罔攸闢。”\footnote{能長養民,長安民,用古王道如此,監無所復罪,當務之。○恬,田廉反。闢,扶亦反。}

{\noindent\zhuan\zihao{6}\fzbyks 傳“當教”至“冤枉”。正義曰:以言“曰”,故知“當教民”也。“殘”謂不死,“虐”,甚則殺,故二文也。經言“屬婦”,傳言“妾婦”者,以妾屬於人,故名“屬婦”。此經“屬婦”與“寡弱”為例,則非關嫡婦也。何者?妻子是家中之貴者,不至冤枉故也。 \par}

{\noindent\zhuan\zihao{6}\fzbyks 傳“王者”至“不勤”。正義曰:以君臣共國事,故並效御治事,而知其所施,則下不得為非,即是王使存省侯伯監治是也,故不可不勤。 \par}

{\noindent\shu\zihao{5}\fzkt “王啟”至“攸闢”。正義曰:\CJKunderline{周公}云:“所以敬勞者,以王者開置監官,其治主為於民故也。以此當教民曰:‘無得相殘傷,無得相虐殺,而為重害也。何但不可為重害,民之相於,當至於敬養寡弱,至於存恤屬婦,合和其教,用大道以相容,無使至冤枉。’所以如此者,以王者其當效實國君,及於御治事者,惟須知其教命所施何用,知其善惡,故不可不勤也。所效實若能長養民,長安民,用古昔明王之,道而治之如此為監,無所復罪,汝當務之。” \par}

“惟曰,若稽田,既勤敷菑,惟其陳修,為厥疆畎。\footnote{言為君監民,惟若農夫之考田,已勞力布發之,惟其陳列修治,為其疆畔畎壟,然後功成。以喻教化。○菑,側其反。畎,工犬反。}若作室家,既勤垣墉,惟其塗塈茨。\footnote{如人為室,家已勤立垣牆,惟其當塗既茨蓋之。○垣音袁。墉音庸,馬云:“卑曰垣,高曰庸。”塈,徐許既反,\CJKunderwave{說文}云:“仰塗也。”\CJKunderwave{廣雅}云:“塗也。”馬云:“堊色。”一音故愛反。茨,徐在私反。}若作梓材,既勤樸斲,惟其塗丹\xpinyin*{雘}。\footnote{為政之術,如梓人治材為器,已勞力樸治斫削,惟其當塗以漆丹以朱而後成。以言教化亦須禮義然後治”。○樸,普角反,馬云:“未成器也。”斫,丁角反。雘,枉略反,徐烏郭反。馬云:“善丹也。”\CJKunderwave{說文}云:“讀與霍同也。”又一郭反,\CJKunderwave{字林}音同。}


{\noindent\zhuan\zihao{6}\fzbyks 傳“為政”至“後治”。正義曰:此三者事別而喻同也。先遠而類疏者,乃漸漸以事近而切者次之。皆言既勤於初,乃言修治於未,明為政孜孜,因前基而修,使善垣墉故也。皆詳而復言之,室器皆雲其事終,而考田止言疆畎,不雲刈獲者,田以一種,但陳修終至收成,故開其初,與下二文互也。二文皆言“斁”,即古“塗”字,明其終而塗飾之。其室言“塗暨”,“暨”亦塗也,總是以物塗之。“茨”謂蓋覆也。器言“塗丹雘”,“塗”、“丹”皆飾物之名,謂塗雘以朱雘。“雘”是彩色之名,有青色者,有硃色者,故\CJKunderline{鄭玄}引\CJKunderwave{山海經}云:“青丘之山,多有青雘。”此經知是“朱”者,與“丹”連文故也。 \par}

{\noindent\shu\zihao{5}\fzkt “惟曰”至“丹雘”。正義曰:既言王者所以效實國君為政之事,故此言國君為政之喻惟為監之事。曰:“若農人之考田也,已勞力遍佈菑而耕發其田,又須為其陳列修治,為疆畔畎壟,以至收穫然後功成。又若人為室家,已勤力立其垣墉,又當惟其塗而暨飾茨蓋之,功乃成也。又若梓人治材為器,已勞力樸治斫削其材,惟其當塗而丹漆以朱雘乃後成。以喻人君為政之道,亦勞心施政,除民之疾,又當惟其飾以禮義,使之行善然後治。” \par}

今王惟曰,先王既勤用明德,懷為夾,\footnote{言文武已勤用明德,懷遠為近,汝治國當法之。○夾音協,近也。}庶邦享作,兄弟方來,,亦既用明德。\footnote{眾國朝享於王,又親仁善鄰為兄弟之國,方方皆來賓服,亦已奉用先王之明德。○朝,直遙反。}後式典集,庶邦丕享。\footnote{君天下能用常法,則和集眾國,大來朝享。}皇天既付中國民,越厥疆土,於先王肆。\footnote{大天已付周家治中國民矣,能遠拓其界壤,則於先王之道遂大。○付如字,馬本作附。拓音託。}王惟德用,和懌先後迷民,用懌先王受命。\footnote{今王惟用德,和悅先後天下迷愚之民。先後謂教訓,所以悅先王受命之義。○懌音亦,字又作斁,下同。先,悉薦反,注同。}已若茲監,惟曰欲至於萬年惟王,\footnote{為監所行已如此所陳法,則我周家惟欲使至於萬年承奉王室。○監,古陷反。為,於威反。}子子孫孫永保民。”\footnote{又欲令其子孫累世長居國以安民。}

{\noindent\zhuan\zihao{6}\fzbyks 傳“言文”至“法之”。正義曰:言“先王”,知謂文武也。“夾”者,是人左右而夾之,故言近也。 \par}

{\noindent\zhuan\zihao{6}\fzbyks 傳“眾國”至“明德”。正義曰:“享”施於王,而“兄弟”為相於之辭,明彼此皆和協。“親仁善鄰”,\CJKunderwave{左傳}文。以先王用明德,於下之所行,今亦奉用,為亦先王耳。 \par}

{\noindent\zhuan\zihao{6}\fzbyks 傳“大天”至“遂大”。正義曰:“肆”,遂也,申遂故為大。“越”,遠也,使天下賓服,故遠柘界壤以益先王,故為“遂大”也。 \par}

{\noindent\zhuan\zihao{6}\fzbyks 傳“今王”至“之義”。正義曰:言“用德”,亦是明德也。“先後”若\CJKunderwave{詩}雲“予曰有先後”,謂於民心先未悟,而啟之已悟,於後化成之,故謂“教訓”也。先王本欲子孫成其事,今化天下使善,是“悅先王受命”。其和悅先王即遠拓疆土,悅其受命即“遂大”也。 \par}

{\noindent\shu\zihao{5}\fzkt “今王”至“保民”。正義曰:此戒\CJKunderline{康叔}已滿三篇,其事將終,須有總結,因其政術言法於明王,上下相承,資以成治,故稱今者王命惟告汝曰:“先王文武在於前世,以自勤用明德,招懷遠人,使來以為親近也。以明德懷柔之故,眾國朝享於王,又相親善為兄弟之國,萬方皆來賓服,亦已化上奉用先王之明德矣。是先王有明德,下亦行明德,以從之而可法也。先王既然,凡為君以君天下者,亦如先王用常法,則和集眾國,使之大來朝享,亦須同先王用明德也。君天下者當如此,今大天已付周家治九州之中國民矣。周家之王,若能為政用明德以懷萬國,遠拓其疆界土壤,則先王之道遂更光大。以此今王須大先王之政,惟明德之大道而用之,以此和悅而先後其天下迷愚之民,使之政治用此,所以悅先王受命使之遂大之義故也。是明德不可不務,故我周王今亦行之。汝為人臣,可以不法乎?當法王家勤用明德治國也。汝若能法我王家而用明德,是為善不可加。”因嘆云:“已乎!如此為監,則我周家惟曰,欲汝至於萬年,惟以承奉王室,今其子子孫孫累世長居國以安民。” \par}

%%% Local Variables:
%%% mode: latex
%%% TeX-engine: xetex
%%% TeX-master: "../Main"
%%% End:
