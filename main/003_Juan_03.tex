%% -*- coding: utf-8 -*-
%% Time-stamp: <Chen Wang: 2024-04-02 11:42:41>

% {\noindent \zhu \zihao{5} \fzbyks } -> 注 (△ ○)
% {\noindent \shu \zihao{5} \fzkt } -> 疏

\chapter{卷三}


\section{舜典第二(堯典下)}


 {\noindent\zihao{6}\fzbyks \CJKunderwave{釋文}:“王氏注,相承云:‘\CJKunderline{梅頤}上\CJKunderline{孔氏}傳\CJKunderwave{古文尚書},雲\CJKunderwave{舜典}一篇。’時以\CJKunderline{王肅}注頗類\CJKunderline{孔氏},故取王注從‘謹徽五典’以下為\CJKunderwave{舜典},以續孔傳徐仙民亦音此本,今依舊音之。” \par}

\textcolor{red}{\CJKunderline{虞舜}}側微\footnote{為庶人,故微賤。},\CJKunderline{堯}聞之聰明,將使嗣位,歷試諸難\footnote{嗣,繼也。試以治民之難事。難,乃丹反。},作\CJKunderwave{\textcolor{red}{舜典}}。

{\noindent\zhuan\zihao{6}\fzbyks 傳“為庶人,故微賤”。正義曰:此雲“側微”即\CJKunderwave{堯典}“側陋”也。不在朝廷謂之“側”,其人貧賤謂之“微”,居處褊隘故言“陋”,此指解“微”,故云“為庶人,故微賤”也。\CJKunderwave{帝系}云:“\CJKunderline{顓頊}生窮蟬,窮蟬生敬康,敬康生句芒,句芒生蟜牛,蟜牛生瞽瞍,瞽瞍生舜。”昭八年\CJKunderwave{左傳}云:“自幕至于瞽瞍無違命。”似其繼世相傳,當有國土。孔言“為庶人”者,\CJKunderwave{堯典}雲“有鰥在下”,此雲“\CJKunderline{虞舜}側微”,必是為庶人矣,蓋至瞽瞍始失國也。 \par}

{\noindent\zhuan\zihao{6}\fzbyks 傳“嗣繼”至“難事”。正義曰:“嗣,繼”,\CJKunderwave{釋詁}文。經所云“慎徽五典”、“納于百揆”、“賓于四門”皆是試以治民之難事也。 \par}

舜典\footnote{“典”之義與堯同。}

{\noindent\shu\zihao{5}\fzkt “\CJKunderline{虞舜}”至“舜典”。正義曰:\CJKunderline{虞舜}所居側陋,身又微賤,堯聞之有聰明聖德,將使之繼己帝位,歷試于諸所難為之事,史述其事,故作\CJKunderwave{舜典} \par}

\textcolor{red}{曰若}稽古,\CJKunderline{帝舜}\footnote{亦言其順考古道而行之。},曰\CJKunderline{重華},協于帝\footnote{華謂文德,言其光文重合于堯,俱聖明。“曰若稽古,\CJKunderline{帝舜},曰\CJKunderline{重華},協于帝”,此十二字是姚方興所上,\CJKunderline{孔氏}傳本無。阮孝緒\CJKunderwave{七錄}亦云然。方興本或此下更有“濬哲文明,溫恭允塞,玄德升聞,乃命以位”。此二十八字異,聊出之,于王注無施也。}。濬哲文明,溫恭允塞\footnote{濬,深。哲,智也。舜有深智文明溫恭之德,信允塞上下。},玄德升聞,乃命\textcolor{red}{以位}\footnote{玄謂幽潛,潛行道德,升聞天朝,遂見徵用。}。

{\noindent\zhuan\zihao{6}\fzbyks 傳“濬深”至“上下”。正義曰:“濬,深”、“哲,智”皆\CJKunderwave{釋言}文。舍人曰:“濬,下之深也。哲,大智也。”舜有深智,言其智之深,所知不淺近也。經緯天地曰“文”,照臨四方曰“明”。\CJKunderwave{詩}云:“溫溫恭人。”言其色溫而貌恭也。舜既有深遠之智,又有文明溫恭之德,信能充實上下也。\CJKunderwave{詩}毛傳訓“塞”為實,言能充滿天地之間,\CJKunderwave{堯典}所謂“格于上下”是也。不言四表者,以四表外無限極,非可實滿,故不言之。堯舜道同,德亦如一,史官錯互為文,故與上篇相類,是其所“合于堯”也。 \par}

{\noindent\zhuan\zihao{6}\fzbyks 傳“玄謂”至“徵用”。正義曰:老子云:“玄之又玄,眾妙之門。”則“玄”者微妙之名,故云“玄謂幽潛”也。舜在畎畝之間,潛行道德,顯彰于外,升聞天朝。“天朝”者,天子之朝也。從下而上謂之為“升”。天子聞之,故遂見徵用。 \par}

{\noindent\shu\zihao{5}\fzkt “曰若”至“以位”。正義曰:昔東晉之初,豫章內史梅賾上\CJKunderline{孔氏}傳,猶闕\CJKunderwave{舜典}。自此“乃命以位”已上二十八字,世所不傳。多用王、範之注補之,而皆以“慎徽”已下為\CJKunderwave{舜典}之初。至齊蕭鸞建武四年,吳興姚方興于大航頭得\CJKunderline{孔氏}傳古文\CJKunderwave{舜典},亦類太康中書,乃表上之。事未施行,方興以罪致戮。至隋開皇初購求遺典,始得之。史將錄舜之美,故為題目之辭曰,能順而考案古道而行之者,是為\CJKunderline{帝舜}也。又申其順考古道之事曰,此舜能繼堯,重其文德之光華,用此德合于\CJKunderline{帝堯},與堯俱聖明也。此舜性有深沈智慧,文章明鑑,溫和之色,恭遜之容,由名聞遠達,信能充實上下,潛行道德,升聞天朝,堯乃徵用,命之以位而試之也。 \par}

\textcolor{red}{慎徽}五典,五典克從。\footnote{徽,美也。五典,五常之教,父義、母慈、兄友、弟恭、子孝。舜慎美篤行斯道,舉八元使布之于四方,五教能從,無違命。徽,許韋反,王雲美,馬云:“善也。”從,才容反。八元,\CJKunderwave{左傳}“\CJKunderline{高辛}氏有才子八人,伯奮、仲堪、叔獻、季仲、伯虎、仲熊、叔豹”。}納于百揆,百揆時敘。\footnote{揆,度也。度百事,總百官,納舜于此官。舜舉八凱,使揆度百事,百事時敘,無廢事業。揆音葵癸反。凱音開在反。\CJKunderwave{左傳}“高陽氏有才子八人,蒼舒、隤敱、檮戭、大臨、尨降、庭堅、仲容、叔達,齊聖廣淵,明允篤誠,天下之民謂之八凱”。}

{\noindent\zhuan\zihao{6}\fzbyks 傳“徽美”至“違命”。正義曰:\CJKunderwave{釋詁}雲“徽,善也”,善亦美也。此\CJKunderwave{五典}與下文“五品”、“五教”其事一也。一家之內品有五,謂父母兄弟子也。教此五者各以一事,教父以義,教母以慈,教兄以友,教弟以恭,教子以孝,是為五教也。五者皆可常行,謂之\CJKunderwave{五典},是五者同為一事,所從言之異耳。文十八年\CJKunderwave{左傳}曰:“昔\CJKunderline{高辛}氏有才子八人,伯奮、仲堪、叔獻、季仲、伯虎、仲熊、叔豹、季貍,忠肅恭懿,宣慈惠和,天下之民謂之八元。舜臣堯,舉八元,使布五教于四方,父義、母慈、兄友、弟恭、子孝。”以此知\CJKunderwave{五典}是五常之教,謂此父義之等五事也。\CJKunderwave{皋陶謨}云:“天敘有典,敕我五典五惇哉!”惇,厚也,行此五典須厚行之,篤亦厚也。言舜謹慎美善,篤行斯道,舉八元使布之于四方,命教天下之民。以此五教能使天下皆順從之,無違逆舜之命也。\CJKunderwave{左傳}又云:“故\CJKunderwave{虞書}數舜之功,曰‘慎徽五典,五典克從’,無違教也。”父母于子並宜為慈,今分之者,以父主教訓,母主撫養;撫養在于恩愛,故以慈為名;教訓愛而加嚴,故以義為稱。義者宜也,理也,教之以義,方使得事理之宜,故為義也。\CJKunderwave{釋訓}雲“善兄弟為友”,則兄弟之恩俱名為友,今雲“兄友、弟恭”者,以其同志曰友,友是相愛之名;但兄弟相愛,乃有長幼,故分其弟使之為恭,恭敬于兄,而兄友愛之。 \par}

{\noindent\zhuan\zihao{6}\fzbyks 傳“揆度”至“事業”。正義曰:“揆,度”,\CJKunderwave{釋言}文。“百揆”者,言百事皆度之。國事散在諸官,故度百事為“總百官”也。\CJKunderwave{周官}云:“\CJKunderline{唐}、\CJKunderline{虞}稽古,建官惟百,內有百揆四岳。”則“百揆”為官名,故云“納舜于此官”也。文十八年\CJKunderwave{左傳}云:“昔高陽氏有才子八人,蒼舒、隤敱、檮戭、大臨、尨降、庭堅、仲容、叔達,齊聖廣淵,明允篤誠,天下之民謂之八凱。舜臣堯,舉八凱,使主后土,以揆百事,莫不時敘,地平天成。”又云:“\CJKunderwave{虞書}數舜之功,曰‘納于百揆,百揆時敘’,無廢事業也。”是言百官于是得其次敘,皆無廢事業。舜既臣堯,乃舉元、凱,主后土,布五教,同時為之。史官立文,自以人事外內為次,故孔先言“八元”。若\CJKunderwave{左傳}據所出代之先後,故先舉“八凱”。堯既得舜,庶事委之。舜既臣堯,任無不統。非“五典克從”之後方始“納于百揆”,“百揆時敘”之後方始“賓于四門”,“四門穆穆”謂流四凶,流放四凶最在于前矣。\CJKunderwave{洪範}云:“\CJKunderline{鯀}則殛死,\CJKunderline{禹}乃嗣興。”是先誅\CJKunderline{鯀}而後用\CJKunderline{禹}明。此言三事皆同時為之,但言“百揆時敘”,故言“納于百揆”,其實“納于百揆”初得即然,由舜既居百揆,故得舉用二八,若偏居一職,不得分使元、凱。 \par}

賓于四門,四門穆穆。\footnote{穆穆,美也。四門,四方之門。舜流四凶族,四方諸侯來朝者,舜賓迎之,皆有美德,無兇人。朝,直遙反。}納于大麓,烈風雷雨弗迷。\footnote{麓,錄也。納舜使大錄萬機之政,陰陽和,風雨時,各以其節,不有迷錯愆伏。明舜之德合于天。麓音鹿,王云:“錄也。”馬、鄭云:“山足也。”愆音起虔反。}

{\noindent\zhuan\zihao{6}\fzbyks 傳“穆穆美”至“兇人”。正義曰:“穆穆,美也”,\CJKunderwave{釋詁}文。“四門,四方之門”,謂四方諸侯來朝者從四門而入。文十八年\CJKunderwave{左傳}歷言四凶之行乃云:“舜臣堯,流四凶族,渾敦、窮奇、檮杌、饕餮投諸四裔,以御螭魅。”又曰:“\CJKunderwave{虞書}數舜之功,曰‘賓于四門,四門穆穆’,無兇人也。”是言“皆有美德,無兇人”也。案驗四凶之族,皆是王朝之臣,舜流王朝之臣,而言諸侯無兇人者,以外見內,諸侯無兇人,則于朝必無矣。\CJKunderline{鄭玄}以“賓”為擯,謂“舜為上擯以迎諸侯”。今孔不為擯者,則謂舜既錄攝事,無不統以諸侯為賓,舜主其禮,迎而待之,非謂身為擯也。 \par}

{\noindent\zhuan\zihao{6}\fzbyks 傳“麓錄”至“于天”。正義曰:“麓”聲近錄,故為錄也。\CJKunderwave{皋陶謨}云:“一日二日萬幾。”言天下之事,事之微者有萬,喻其多無數也。納舜使大錄萬機之政,還是納于百揆,揆度百事,大錄萬機,總是一事,不為異也。但此言“德合于天”,故以“大錄”言耳。\CJKunderwave{論語}稱\CJKunderline{孔子}曰:“迅雷風烈必變。”\CJKunderwave{書傳}稱:“越常之使久矣,天之無烈風淫雨。”則烈風是猛疾之風,非善風也。經言“烈風雷雨弗迷”,言舜居大錄之時,陰陽和,風雨時,無此猛烈之風,又雷雨各以其節,不有迷錯愆伏也。“迷錯”者,應有而無,應無而有也。昭四年\CJKunderwave{左傳}雲“冬無愆陽,夏無伏陰”。無愆伏者,無冬溫夏寒也。舜錄大政,天時如此,明舜之德合于天也。此文與上三事亦同時也。上為變人,此為動天,故最後言之,以為功成之驗。\CJKunderline{王肅}云:“堯得舜任之,事無不統,自‘慎徽五典’以下是也。”其言合孔意。 \par}

帝曰:“格,汝\CJKunderline{舜}。詢事考言,乃言\xpinyin*{厎}可績,三載。汝陟帝位。”\footnote{格,來。詢,謀。乃,汝。厎,致。陟,升也。堯呼舜曰:“來,汝所謀事,我考汝言。汝言致,可以立功,三年矣。三載考績,故命使升帝位。”將禪之。詢音荀。厎音之履反,王云:“致也。”馬云:“定也。”}\CJKunderline{舜}讓于德,\textcolor{red}{弗嗣}。\footnote{辭讓于德不堪,不能嗣成帝位。}

{\noindent\zhuan\zihao{6}\fzbyks 傳“格來”至“禪之”。正義曰:“格,來”,\CJKunderwave{釋言}文。“詢,謀”、“陟,升”\CJKunderwave{釋詁}文。“厎”聲近致,故為致也。經傳言“汝”多呼為“乃”,知“乃”、“汝”義同。凡事之始必先謀之,後為之,堯呼舜曰:“來,汝舜。”呼使前而與之言也。“汝所謀事,我考汝言。汝所為之事,皆副汝所謀,致可以立功,于今三年矣”。從徵得至此為三年也。君之馭臣,必三年考績。考既有功,故使升帝位,將禪之也。\CJKunderline{鯀}三考乃退,此一考使升者,\CJKunderline{鯀}待三考,冀其有成,無成功乃黜,為緩刑之義;舜既有成,更無所待,故一考即升之。且大聖之事,不可以常法論也。若然,\CJKunderwave{禹貢}“兗州作十有三載乃同”,是\CJKunderline{禹}治兗州之水,乃積十有三年。此始三年已言“地平天成”者,\CJKunderwave{祭法}云:“\CJKunderline{鯀}障洪水而殛死,\CJKunderline{禹}能修\CJKunderline{鯀}之功。”先儒馬融等皆以為“\CJKunderline{鯀}既九年,又加此三年為十二年,惟兗州未得盡平,至明年乃畢”。八州已平,一州未畢,足以為成功也。 \par}

{\noindent\shu\zihao{5}\fzkt “慎徽”至“弗嗣”。正義曰:此承“乃命以位”之下,言命之以位,試之以事也。堯使舜慎美篤行五常之教,而五常之教皆能順從而行之,無違命也。又納于百官之事,命揆度行之,而百事所揆度者于是皆得次序,無廢事也。又命使賓迎諸侯于四門,而來入者穆穆然皆有美德,無兇人也。又納于大官,總錄萬機之政,而陰陽和,風雨時,烈風雷雨不有迷惑錯謬。明舜之德合于天,天人和協,其功成矣。\CJKunderline{帝堯}乃謂之曰:“來,汝舜,有所謀之事,我考驗汝舜之所言。汝言致可以立功,于今三年,汝功已成,汝可升處帝位。”告以此言,欲禪之也。舜辭讓于德,言已德不堪嗣成帝也。 \par}

\textcolor{red}{正月}上日,受終于文祖。\footnote{上日,朔日也。終謂堯終帝位之事。文祖者堯文德之祖廟。正音政,又音徵。王云:“文祖,廟名。”馬云:“文祖,天也。天為文萬物之祖,故曰文祖。”}在\xpinyin*{璿}璣玉衡,以齊七政。\footnote{在,察也。璿,美玉。璣、衡,王者正天文之器,可運轉者。七政,日月五星各異政。舜察天文,齊七政,以審己當天心與否。璿音旋。}

{\noindent\zhuan\zihao{6}\fzbyks 傳“上日”至“祖廟”。正義曰:月之始日謂之朔日,每月皆有朔日,此是正月之朔,故云“上日”,言一歲日之上也。下雲“元日”亦然。\CJKunderline{鄭玄}以為“帝王易代,莫不改正。堯正建醜,舜正建子。此時未改堯正,故云‘正月上日’。即位,乃改堯正,故云‘月正元日’”,故以異文。先儒\CJKunderline{王肅}等以為“惟殷周改正,易民視聽。自夏已上,皆以建寅為正。此篇二文不同,史異辭耳”。孔意亦然。下雲“歲二月”,傳雲“既班瑞之明月”,以此為建寅之月也。“受終”者,堯為天子,于此事終而授與舜,故知“終謂堯終帝位之事”,“終”言堯終舜始也。禮有大事,行之于廟,況此是事之大者,知“文祖者,堯文德之祖廟”也。且下云:“歸,格于藝祖。”“藝”、“文”義同。知“文祖”是廟者,\CJKunderwave{咸有一德}云:“七世之廟,可以觀德。”則天子七廟,其來自遠。堯之文祖,蓋是堯始祖之廟,不知為誰也。\CJKunderwave{帝系}及\CJKunderwave{世本}皆雲“\CJKunderline{黃帝}生玄囂,玄囂生僑極,僑極生帝嚳,帝嚳生堯”。即如彼言,\CJKunderline{黃帝}為堯之高祖,\CJKunderline{黃帝}以上不知復祭何人,充此七數,況彼二書未必可信,堯之文祖不可強言。 \par}

{\noindent\zhuan\zihao{6}\fzbyks 傳“在察”至“與否”。正義曰:“在,察”,\CJKunderwave{釋詁}文。\CJKunderwave{說文}云:“璿,美玉也。”玉是大名,璿是玉之別稱。璣衡俱以玉飾,但史之立文,不可以玉璣、玉衡一指玉體,一指玉名,猶\CJKunderwave{左傳}雲“瓊弁玉纓”。所以變其文,傳以總言玉名,故云“美玉”,其實玉衡亦美玉也。\CJKunderwave{易·賁卦}彖云:“觀乎天文以察時變。”日月星宿運行于天,是為天之文也。璣衡者,璣為轉運,衡為橫簫,運璣使動,于下以衡望之,是“王者正天文之器”。漢世以來,謂之渾天儀者是也。馬融云:“渾天儀可旋轉,故曰璣。衡,其橫簫,所以視星宿也。以璿為璣,以玉為衡,蓋貴天象也。”蔡邕雲“玉衡長八尺,孔徑一寸,下端望之以視星辰。蓋懸璣以象天而衡望之,轉璣窺衡以知星宿”,是其說也。“七政”,其政有七,于璣衡察之,必在天者,知“七政”謂日月與五星也。木曰歲星,火曰熒惑星,土曰鎮星,金曰太白星,水曰辰星。\CJKunderwave{易·繫辭}云:“天垂象,見吉凶,聖人象之。”此日月五星有吉凶之象,因其變動為佔,七者各自異政,故為七政。得失由政,故稱政也。舜既受終,乃察璣衡,是“舜察天文,齊七政,以審已之受禪當天心與否”也。馬融云:“日月星皆以璿璣玉衡度知其盈縮、進退、失政所在。聖人謙讓,猶不自安,視璿璣玉衡以驗齊日月五星行度,知其政是與否,重審已之事也。”上天之體,不可得知,測天之事,見于經者唯有此“璿璣玉衡”一事而已。蔡邕\CJKunderwave{天文志}云:“言天體者有三家,一曰周髀,二曰宣夜,三曰渾天。宣夜絕無師說。周髀術數具在,考驗天象,多所違失,故史官不用。惟渾天者近得其情,今史所用候臺銅儀,則其法也。”虞喜云:“宣,明也。夜,幽也。幽明之數,其術兼之,故曰宣夜。”但絕無師說,不知其狀如何。周髀之術以為天似覆盆,蓋以斗極為中,中高而四邊下,日月旁行繞之。日近而見之為晝,日遠而不見為夜。渾天者以為地在其中,天周其外,日月初登于天,後入于地。晝則日在地上,夜則日入地下。王蕃\CJKunderwave{渾天說}曰:“天之形狀似鳥卵,天包地外,猶卵之裡黃,圓如彈丸,故曰渾天。”言其形體渾渾然也。其術以為天半覆地上,半在地下。其天居地上,見有一百八十二度半強,地下亦然。北極出地上三十六度,南極入地下亦三十六度,而嵩高正當天之中極,南五十五度當嵩高之上。又其南十二度為夏至之日道,又其南二十四度為春秋分之日道,又其南二十四度為冬至之日道,南下去地三十一度而已。是夏至日北去極六十七度,春秋分去極九十一度,冬至去極一百一十五度,此其大率也。其南北極持其兩端,其天與日月星宿斜而回轉,此必古有其法,遭秦而滅。楊子\CJKunderwave{法言}云:“或問渾天,曰,落下閎營之,鮮于妄人度之,耿中丞象之,幾乎,幾乎,莫之能違也!”是楊雄之意,以唬骸半天而問之也。閎與妄人,武帝時人。宣帝時司農中丞耿壽昌始鑄銅為之象,史官施用焉。後漢張衡作\CJKunderwave{靈憲}以說其狀。蔡邕、\CJKunderline{鄭玄}、陸績、吳時王藩、晉世姜岌、張衡、葛洪皆論渾天之義,並以渾說為長。江南宋元嘉年皮延宗又作是\CJKunderwave{渾天論},太史丞錢樂鑄銅作渾天儀,傳于齊梁,周平、江陵遷其器于長安,今在太史書矣。衡長八尺,璣徑八尺,圓周二丈五尺強,轉而望之,有其法也。 \par}

肆類于上帝,\footnote{堯不聽舜讓,使之攝位。舜察天文,考齊七政而當天心,故行其事。肆,遂也。類謂攝位事類。遂以攝告天及五帝。王云:“上帝,天也。”馬云:“上帝,太一神,在紫微宮,天之最尊者。”}禋于六宗,\footnote{精意以享謂之禋。宗,尊也。所尊祭者,其祀有六,謂四時也、寒暑也、日也、月也、星也、水旱也。祭亦以攝告。禋音因,王云:“絜祀也。”馬云:“精意以享也。”六宗,王云:“四時、寒暑、日、月、星、水旱也。”馬云:“天地四時也。”}望于山川,徧于群神。\footnote{九州名山大川、五嶽四瀆之屬,皆一時望祭之。群神謂丘陵、墳衍、古之聖賢,皆祭之。墳,扶雲反。衍音演。}

{\noindent\zhuan\zihao{6}\fzbyks 傳“堯不”至“五帝”。正義曰:傳以既受終事,又察璣衡,方始祭于群神,是舜察天文,考齊七政,知己攝位而當于天心,故行其天子之事也。\CJKunderwave{祭法}云:“有而下者祭百神”遍祭群神是天子事也。“肆”是縱緩之言,此因前事而行後事,故以“肆”為遂也。“類”謂攝位事類,既知攝當天心,遂以攝位事類告天帝也。此“類”與下“禋”、“望”相次,當為祭名。\CJKunderwave{詩}雲“是類是禡”,\CJKunderwave{周禮·肆師}雲“類造上帝”,\CJKunderwave{王制}雲“天子將出類乎上帝”,所言“類”者皆是祭天之事,言以事類而祭也。\CJKunderwave{周禮·小宗伯}云:“天地之大災,類社稷,則為位。”是類之為祭,所及者廣。而傳雲“類謂攝位事類”者,以攝位而告祭,故類為祭名。\CJKunderwave{周禮·司服}云:“王祀昊天上帝,則服大裘而冕,祀五帝亦如之。”是昊天外更有五帝,上帝可以兼之,故以“告天及五帝”也。\CJKunderline{鄭玄}篤信讖緯,以為“昊天上帝謂天皇大帝,北辰之星也。五帝謂靈威仰等,太微宮中有五帝座星是也”。如鄭之言,天神有六也。\CJKunderwave{家語}云:“季康子問五帝之名,\CJKunderline{孔子}曰:‘天有五行:金木水火土。分時化育,以成萬物,其神謂之五帝。’”\CJKunderline{王肅}云:“五行之神,助天理物者也。”孔意亦當然矣。此經惟有祭天,不言祭地及社稷,必皆祭之,但史略文耳。 \par}

{\noindent\zhuan\zihao{6}\fzbyks 傳“精意”至“攝告”。正義曰:\CJKunderwave{國語}云:“精意以享禋也。”\CJKunderwave{釋詁}云:“禋,祭也。”孫炎曰:“禋,絜敬之祭也。”\CJKunderwave{周禮·大宗伯}云:“以禋祀祀昊天上帝,以實柴祀日月星辰,以槱燎祀司中、司命、風師、雨師。”鄭云:“禋之言禋,周人尚臭,煙氣之臭聞者也。”鄭以“禋祀”之文在“燎”、“柴”之上,故以“禋”為此解耳。而\CJKunderwave{洛誥}雲“秬鬯二卣,曰‘明禋’”,又曰“禋于\CJKunderline{文王}、\CJKunderline{武王}”,又曰“王賓殺禋咸格”。經傳之文,此類多矣,非燔柴祭之也,知“禋”是精誠絜敬之名耳。“宗”之為尊,常訓也。名曰“六宗”,明是所尊祭者有六,但不知六者為何神耳。\CJKunderwave{祭法}云:“埋少牢于太昭,祭時。相近于坎壇,祭寒暑。王宮,祭日。夜明,祭月。幽禜,祭星。雩禜,祭水旱也。”據此言六宗,彼祭六神,故傳以彼六神謂此六宗。必謂彼之所祭是此六宗者,彼文上有祭天、祭地,下有山谷、丘陵,此“六宗”之文在上帝之下,山川之上,二者次第相類,故知是此六宗。\CJKunderline{王肅}亦引彼文乃云:“禋于六宗,此之謂矣。”\CJKunderline{鄭玄}注彼云:“四時謂陰陽之神也。”然則陰陽寒暑水旱各自有神,此言“禋于六宗”,則六宗常禮也。禮無此文,不知以何時祀之。鄭以彼皆為祈禱之祭,則不可用\CJKunderline{鄭玄}注以解此傳也。漢世以來,說六宗者多矣。歐陽及大小夏侯說\CJKunderwave{尚書}皆云:“所祭者六,上不謂天,下不謂地,旁不謂四方,在六者之間,助陰陽變化,實一而名六宗矣。”孔光、劉歆以“六宗謂乾坤六子:水火雷風山澤也”。賈逵以為:“六宗者,天宗三,日月星也;地宗三,河海岱也。”馬融云:“萬物非天不覆,非地不載,非春不生,非夏不長,非秋不收,非冬不藏,此其謂六也。”\CJKunderline{鄭玄}以六宗言“禋”,與祭天同名,則六者皆是天之神祗,謂“星、辰、司中、司命、風師、雨師。星謂五緯星,辰謂日月所會十二次也。司中、司命文昌第五第四星也。風師,箕也。雨師,畢也”。晉初幽州秀才張髦上表云:“臣謂禋于六宗,祀祖考所尊者六,三昭三穆是也。”司馬彪又上表云,歷難諸家及自言己意“天宗者,日月星辰寒暑之屬也;地宗,社稷五祀之屬也;四方之宗,四時五帝之屬”。惟\CJKunderline{王肅}據\CJKunderwave{家語}六宗與孔同。各言其志,未知孰是。司馬彪\CJKunderwave{續漢書}云:“安帝元初六年,立六宗祠于洛陽城西北亥地,祀比大社,魏亦因之。晉初荀顗定新祀,以六宗之神諸說不同廢之。摯虞駁之,謂:‘宜依舊,近代以來皆不立六宗之祠也。’” \par}

{\noindent\zhuan\zihao{6}\fzbyks 傳“九州”至“祭之”。正義曰:“望于山川”,大總之語,故知九州之內所有名山大川、五嶽四瀆之屬皆一時望祭之也。\CJKunderwave{王制}云:“名山大川不以封。”山川大,乃有名,是“名”、“大”互言之耳。\CJKunderwave{釋山}云:“泰山為東嶽,華山為西嶽,霍山為南嶽,恆山為北嶽,嵩高山為中嶽。”\CJKunderwave{白虎通}云:“嶽者何?捔也,捔考功德也。”應劭\CJKunderwave{風俗通}云:“嶽者,捔考功德黜陟也。”然則四萬方有一大山,天子巡守至其下,捔考諸侯功德而黜陟之,故謂之“嶽”。\CJKunderwave{釋水}云:“江河淮濟為四瀆。四瀆者,發源注海者也。”\CJKunderwave{釋名}云:“瀆,獨也,各獨出其水而入海也。”嶽是名山,瀆是大川,故先言名山大川,又舉嶽瀆以見之。嶽瀆之外猶有名山大川,故言“之屬”以包之。\CJKunderwave{周禮·大司樂}云:“四鎮五嶽崩,令去樂。”鄭云:“四鎮,山之重大者,謂揚州之會稽山,青州之沂山,幽州醫無閭山,冀州之霍山。”是五嶽之外名山也。\CJKunderwave{周禮·職方氏}每州雲“其川”、“其浸”,若雍州雲“其川涇、汭,其浸渭、洛”,如此之類,是四瀆之外大川也。言“遍于群神”,則神無不遍,故“群神謂丘陵、墳衍、古之聖賢,皆祭之”。\CJKunderwave{周禮·大司樂}云:“凡六樂者,一變而致川澤之示,再變而致山林之示,三變而致丘陵之示,四變而致墳衍之示。”\CJKunderline{鄭玄}\CJKunderwave{大司徒}注云:“積石曰山,竹木曰林,注瀆曰川,水鍾曰澤,土高曰丘,大阜曰陵,水崖曰墳,下平曰衍。”此傳舉“丘陵墳衍”則林澤亦包之矣。“古之聖賢”謂\CJKunderwave{祭法}所云“在祀典”者,\CJKunderline{黃帝}、\CJKunderline{顓頊}、句龍之類皆祭之也。 \par}

輯五瑞,既月乃日,覲四岳群牧,班瑞于\textcolor{red}{群后}。\footnote{輯,斂。既,盡。覲,見。班,還。後,君也。舜斂公侯伯子男之瑞圭璧,盡以正月中,乃日日見四岳及九州牧監,還五瑞于諸侯,與之正始。輯,徐音集,王雲合,馬云:“斂也。”瑞,垂偽反,信也。牧,牧養之牧,徐音目。}

{\noindent\zhuan\zihao{6}\fzbyks 傳“輯斂”至“正始”。正義曰:“覲,見”,“後,君”,\CJKunderwave{釋詁}文。\CJKunderwave{釋言}云:“輯,合也。”“輯”是合聚之義,故為斂也。日月食盡謂之既,是“既”為盡也。\CJKunderwave{釋言}云:“班,賦也。”孫炎曰:“謂布與也。”“輯”是斂聚,“班”為散佈,故為還也。下雲“班瑞于群后”,則知“輯”者從群后而斂之,故云“舜斂公侯伯子男之瑞圭璧”也。\CJKunderwave{周禮·典瑞}云:“公執桓圭,侯執信圭,伯執躬圭,子執谷璧,男執蒲璧。”是圭璧為五等之瑞。諸侯執之以為王者瑞信,故稱“瑞”也。舜以朔日受終于文祖,又遍祭群神及斂五瑞,則入月以多日矣。“盡以正月中”謂從斂瑞以後至月末也。“乃日日見四岳及九州牧監”,舜初攝位,當發號出令,日日見之,與之言也。州牧各監一州諸侯,故言“監”也。更復還五瑞于諸侯者,此瑞本受于堯,斂而又還之,若言舜新付之,改為舜臣,與之正新君之始也。 \par}

{\noindent\shu\zihao{5}\fzkt “正月”至“群后”。正義曰:舜既讓而不許,乃以堯禪之。明年正月上日,受堯終帝位之事于堯文祖之廟。雖受堯命,猶不自安。又以璿為璣、以玉為衡者,是為王者正天文之器也,乃復察此璿璣玉衡,以齊整天之日月五星七曜之政。觀其齊與不齊,齊則受之是也,不齊則受之非也。見七政皆齊,知己受為是,遂行為帝之事,而以告攝事類祭于上帝,祭昊天及五帝也。又禋祭于六宗等尊卑之神,望祭于名山大川、五嶽四瀆,而又遍祭于山川、丘陵、墳衍、古之聖賢之群神,以告已之受禪也。告祭既畢,乃斂公侯伯子男五等之瑞玉。其圭與璧悉斂取之盡,以正月之中,乃日月見四岳及群牧,既而更班所斂五瑞于五等之群后,而與之更始,見己受堯之禪,行天子之事也。 \par}

\textcolor{red}{歲二月},東巡守,至于岱宗,柴,\footnote{諸侯為天子守土,故稱守,巡行之。既班瑞之明月,乃順春東巡。岱宗,泰山,為四岳所宗。燔柴祭天告至。○巡,似遵反,徐養純反。守,時救反,本或作狩。岱音代,泰山也。柴,士皆反。\CJKunderwave{爾雅}:“祭天曰燔柴。”馬曰:“祭時積柴,加牲其上而燔之。”行,下孟反。燔,扶袁反,又扶雲反。}望秩于山川,\footnote{東嶽諸侯竟內名山大川如其秩次望祭之。謂五嶽牲禮視三公,四瀆視諸侯,其餘視伯子男。○瀆,徒木反。}肆覲東后。\footnote{遂見東方之國君。}

{\noindent\zhuan\zihao{6}\fzbyks 傳“諸侯”至“告至”。正義曰:王者所為巡守者,以諸侯自專一國,威福在己,恐其擁遏上命,澤不下流,故時自巡行,問民疾苦。\CJKunderwave{孟子}稱晏子對齊景公云:“天子適諸侯曰巡守。巡守者,巡所守也。”是言天子巡守主謂巡行諸侯,故言諸侯為天子守土,故稱守,而往巡行之。定四年\CJKunderwave{左傳}祝鮀言衛國“取相土之東都,以會王之東搜”,“搜”是獵之名也。王者因巡諸侯,或亦獵以教戰,其守皆作“狩”。\CJKunderwave{白虎通}云:“王者所以巡狩者何?巡者循也,狩者收也,為天子循收養人。”彼因名以附說,不如晏子之言得其本也。正月班瑞,二月即行,故云“既班瑞之明月,乃順春東巡”。春位在東,故“順春”也。\CJKunderwave{爾雅}:“泰山為東嶽。”此巡守至于岱,岱之與泰,其山有二名也。\CJKunderwave{風俗通}云:“泰山,山之尊者,一曰岱宗。岱,始也。宗,長也。萬物之始,陰陽交代,故為五嶽之長。”是解岱即泰山,為四岳之宗,稱岱宗也。\CJKunderwave{郊特牲}云:“天子適四方,先柴。”是燔柴為祭天告至也。 \par}

{\noindent\zhuan\zihao{6}\fzbyks 傳“東嶽”至“子男”。正義曰:四時各至其方岳,望祭其方岳山川,故云“東嶽諸侯境內名山大川如其秩次望祭之”也。言秩次而祭,知遍于群神,故云“五嶽牲禮視三公,四瀆視諸侯,其餘視伯子男”也。其尊卑所視,\CJKunderwave{王制}及\CJKunderwave{書傳}之文,“牲禮”二字孔增之也。諸侯五等,三公為上等,諸侯為中等,伯子男為下等,則所言諸侯,惟謂侯爵者耳。其言所視,蓋視其祭祀。祭五嶽如祭三公之禮,祭四瀆如祭諸侯之禮,祭山川如祭伯子男之禮。公侯伯子男尊卑既有等級,其祭禮必不同,但古典亡滅,不可復知。\CJKunderline{鄭玄}注\CJKunderwave{書傳}云:“所視者,謂其牲幣粢盛籩豆爵獻之數。”案五等諸侯適天子皆膳用太牢,禮諸侯祭皆用太牢,無上下之別。又\CJKunderwave{大行人}云,上公九獻,侯伯七獻,子男五獻。\CJKunderwave{掌客}上公饔餼九牢、飧五牢,侯伯饔餼七牢、飧四牢,子男饔餼五牢、飧三牢。又上公豆四十,侯伯三十二,子男二十四。並\CJKunderline{伯與}侯同。又鄭注\CJKunderwave{禮器}“四望”、“五獻”據此諸文。與孔傳\CJKunderwave{王制}不同者,\CJKunderwave{掌客}、\CJKunderwave{行人}自是周法,孔與\CJKunderwave{王制}先代之禮。必知然者,以\CJKunderwave{周禮}侯與伯同,\CJKunderwave{公羊}及\CJKunderwave{左氏傳}皆以公為上,伯子男為下,是其異也。 \par}

協時月正日,同律度量衡。\footnote{合四時之氣節,月之大小,日之甲乙,使齊一也。律法制及尺丈、斛鬥、斤兩,皆均同。○同律,王云:“同,齊也。律,六律也。”馬云:“律,法也。”鄭云:“陰呂陽律也。”度如字,丈尺也。量,力尚反,鬥斛也。衡,稱也。}修五禮、五玉、\footnote{修吉、兇、賓、軍、嘉之禮。五等諸侯執其玉。}三帛、二生、一死贄,\footnote{三帛,諸侯世子執纁,公之孤執玄,附庸之君執黃。二生,卿執羔,大夫執雁。一死,士執雉。玉、帛、生、死,所以為贄以見之。贄音至,本又作摯。纁,許雲反。}如五器,卒乃復。\footnote{卒,終。復,還也。器謂圭璧。如五器,禮終則還之。三帛、生、死則否。復,扶又反,下同。還音旋。}

{\noindent\zhuan\zihao{6}\fzbyks 傳“合四”至“均同”。正義曰:上篇已訓“協”為合,故注即以合言之也,他皆仿此。\CJKunderwave{周禮·太史}云:“正歲年,頒告朔于邦國。”則節氣晦朔皆天子頒之。猶恐諸侯國異或不齊同,故因巡守而合和之。節是月初,氣是月半也。\CJKunderwave{世本}云:“容成作歷。大撓作甲子。”二人皆\CJKunderline{黃帝}之臣,蓋自\CJKunderline{黃帝}已來始用甲子紀日,每六十日而甲子一周。\CJKunderwave{史記}稱紂為長夜之飲,忘其日辰。恐諸侯或有此之類,故須合日之甲乙也。時也,月也,日也,三者皆當勘檢諸國使齊一也。“律”者候氣之管,而度量衡三者法制皆出于律,故云“律法制”也。度有丈尺,量有斛鬥,衡有斤兩,皆取法于律,故孔解“律”為法制,即雲“及尺丈、斛鬥、斤兩皆均同之”。\CJKunderwave{漢書日·律歷志}云,度量衡出于黃鐘之律也。度者,分、寸、尺、丈、引,所以度長短也。本起于黃鐘之管長。以子谷秬黍中者,以一黍之廣度之,千二百黍為一分。十分為寸,十寸為尺,十尺為丈,十丈為引,而五度審矣。量謂龠、合、升、鬥、斛,所以量多少也。本起于黃鐘之龠,以子谷秬黍中者千有二百實為一龠。十龠為合,十合為升,十升為鬥,十鬥為斛,而五量嘉矣。權者,銖、兩、斤、鈞、石,所以稱物知輕重也。本起于黃鐘之龠,一龠容千二百黍,重十二銖,兩之為兩,十六兩為斤,三十斤為鈞,四鈞為石,而五權謹矣。權、衡一物。衡,平也;權,重也;稱上謂之衡,稱錘謂之權,所從言之異耳。如彼\CJKunderwave{志}文,是度量衡本起于律也。時月言“協”,日言“正”,度量衡言“同”者,以時月須與他月和合,故言“協”;日有正與不正,故言“正”;度量衡俱是明之,所用恐不齊同,故言“同”;因事宜而變名耳。 \par}

{\noindent\zhuan\zihao{6}\fzbyks 傳“修吉”至“其玉”。正義曰:\CJKunderwave{周禮·大宗伯}云:“以吉禮事邦國之鬼神示,以凶禮哀邦國之憂,以賓禮親邦國,以軍禮同邦國,以嘉禮親萬民之昏姻。”知“五禮”謂此也。帝王之名既異,古今之禮或殊,而以周之五禮為此“五禮”者,以帝王相承,事有損益,後代之禮亦當是前代禮也。且歷驗此經,亦有五事:此篇“類于上帝”,吉也;“如喪考妣”,兇也;“群后四朝”,賓也;\CJKunderwave{大禹謨}雲“汝徂徵”,軍也;\CJKunderwave{堯典}雲“女于時”,嘉也。五禮之事,並見于經,知與後世不異也。此雲“五玉”,即上文“五瑞”,故知“五等諸侯執其玉”也。\CJKunderline{鄭玄}云:“執之曰瑞,陳列曰玉。” \par}

{\noindent\zhuan\zihao{6}\fzbyks 傳“諸侯”至“執黃”。正義曰:\CJKunderwave{周禮·典命}云:“凡諸侯之適子,誓于天子,攝其君,則下其君之禮一等。未誓,則以皮帛繼子男之下。公之孤四命,以皮帛視小國之君。”是諸侯世子、公之孤執帛也。附庸雖則無文,而為南面之君,是一國之主,春秋時附庸之君適魯皆稱“來朝”,未有爵命,不得執玉,則亦繼小國之君同執帛也。經言“三帛”,必有三色,所云纁、玄、黃者,孔時或有所據,末知出何書也。\CJKunderline{王肅}云:“三帛,纁、玄、黃也。附庸與諸侯之適子、公之孤執皮帛,其執之色未詳。聞或曰孤執玄,諸侯之適子執纁,附庸執黃。”\CJKunderline{王肅}之注\CJKunderwave{尚書},其言多同孔傳。\CJKunderwave{周禮}孤與世子皆執皮帛,\CJKunderline{鄭玄}云:“皮帛者,束帛而表,之以皮為之飾。皮,虎豹皮也。”此三帛不言皮,蓋于時未以皮為飾。 \par}

{\noindent\zhuan\zihao{6}\fzbyks 傳“卿執”至“執雉”。正義曰:此皆\CJKunderwave{大宗伯}文也。\CJKunderline{鄭玄}曰:“羔,小羊,取其群而不失其類也。雁,取其候時而行也。雉,取其守介,死不失節也。\CJKunderwave{曲禮}雲‘飾羔雁者以繢’,謂衣之以布而又畫之。雉執之無飾。\CJKunderwave{士相見之禮},卿大夫飾贄以布。不言繢,此諸侯之臣與天子之臣異也。”鄭之此言,論周之禮耳,虞時每事猶質,羔雁不必有飾。 \par}

{\noindent\zhuan\zihao{6}\fzbyks 傳“玉帛”至“見之”。正義曰:\CJKunderwave{曲禮}云:“贄,諸侯圭,卿羔,大夫雁,士雉。”雉不可生,知“一死”是雉,“二生”是羔、雁也。\CJKunderline{鄭玄}云:“贄之言至,所執以自至也。”自“五玉”以下,蒙上“修”文者,執之使有常也。若不言“贄”,則不知所用,故言“贄”以結上,又見玉、帛、生、死皆所以為贄,以見君與自相見,其贄同也。 \par}

{\noindent\zhuan\zihao{6}\fzbyks 傳“卒終”至“則否”。正義曰:“卒,終”,\CJKunderwave{釋詁}文。\CJKunderwave{釋言}云:“還、復,返也。”是還、復同義,故為還也。“五器”文在“贄”下,則是贄內之物。\CJKunderwave{周禮·大宗伯}云:“以玉作六器。”知“器謂圭璧”,即五玉是也。如,若也。言諸侯贄之內,若是五器,禮終乃還之,如三帛、生、死則不還也。\CJKunderwave{聘義}云:“以圭璋聘,重禮也。已聘而還圭璋,此輕財而重禮之義也。”\CJKunderwave{聘義}主于說聘,其朝禮亦然。\CJKunderwave{周禮·司儀}云:“諸公相見為賓,還圭,如將幣之儀。”是圭璧皆還之也。\CJKunderwave{士相見禮}言大夫以下見國君之禮云:“若他邦之人,則使擯者還其贄。”己臣皆不還其贄,是“三帛、生、死則否”。 \par}

五月南巡守,至于南嶽,如岱禮。\footnote{南嶽,衡山。自東嶽南巡,五月至。}八月西巡守,至于西嶽,如初。\footnote{西嶽,華山。初謂岱宗。華,戶化反。華山在弘農。}十有一月朔巡守,至于北嶽,如西禮。\footnote{北嶽,恆山。有,如字,徐于救反。如西禮,方興本同,馬本作“如初”。}歸,格于藝祖,用特。\footnote{巡守四岳然後歸,告至文祖之廟。藝,文也。言祖則考著。特,一牛。藝,魚世反,馬、王云:“禰也。”}五載一巡守,群后四朝。\footnote{各會朝于方岳之下,凡四處,故曰“四朝”。將說“敷奏”之事,故申言之。堯舜同道,舜攝則然,堯又可知。四朝,馬、王皆云:“四面朝于方岳之下。”鄭云:“四朝,四季朝京師也。”朝音直遙反,注同。}敷奏以言,明試以功,車服\textcolor{red}{以庸}。\footnote{敷,陳。奏,進也。諸侯四朝,各使陳進治禮之言。明試其言,以要其功,功成則賜車服以表顯其能用。敷音孚。}

{\noindent\zhuan\zihao{6}\fzbyks 傳“南嶽”至“月至”。正義曰:\CJKunderwave{釋山}云:“河南華,河東岱,河北恆,江南衡。”李巡云:“華,西嶽華山也。岱,東嶽泰山也。恆,北嶽恆山也。衡,南嶽衡山也。”郭璞云:“恆山一名常山,避漢文帝諱。”\CJKunderwave{釋山}又云:“泰山為東嶽,華山為西嶽,霍山為南嶽,恆山為北嶽。”岱之與泰,衡之與霍,皆一山而有兩名也。\CJKunderline{張揖}云:“天柱謂之霍山。”\CJKunderwave{漢書·地理志}云,天柱在廬江灊縣,則霍山在江北。而與江南衡為一者,郭璞\CJKunderwave{爾雅}注云:“霍山今在廬江灊縣,潛水出焉,別名天柱山。漢武帝以衡山遼曠,故移其神于此。今其彼土俗人皆呼之為南嶽。南嶽本自以兩山為名,非從近來也。而學者多以霍山不得為南嶽,又云漢武帝來始乃名之。即如此言,謂武帝在\CJKunderwave{爾雅}前乎?斯不然矣。”是解衡、霍二名之由也。書傳多雲“五嶽”,以嵩高為中嶽,此雲“四岳”者,明巡守至于四岳故也。\CJKunderwave{風俗通}云:“泰山,山之尊者,一曰岱宗。岱,始也。宗,長也。萬物之始,陰陽交代,故為五嶽之長。王者受命恆封禪之。衡山一名霍山,言萬物霍然大也。華,變也,萬物變由西方也。恆,常也,萬物伏北方有常也。”二月至于岱宗,不指“嶽”名者,巡守之始,故詳其文,三時言嶽名,明岱亦是嶽,因事宜而互相見也。四巡之後乃雲“歸,格”,則是一出而周四岳。故知自東嶽而即南行,以五月至也。王者順天道以行人事,故四時之月各當其時之中,故以仲月至其嶽。上雲“歲二月,東巡守”,以二月始發者,此四時巡守之月,皆以至嶽為文,東巡以二月至,非發時也,但舜以正月有事,二月即發行耳。\CJKunderline{鄭玄}以為“每嶽禮畢而歸,仲月乃復更去”。若如鄭言,當于東巡之下即言“歸,格”,後以“如初”包之,何當北巡之後始言歸乎?且若來而復去,計程不得周遍,此事不必然也。其經南雲“如岱禮”,西雲“如初”,北雲“如西禮”者,見四時之禮皆同,互文以明耳。不巡中嶽者,蓋近京師,有事必聞,不慮枉滯,且諸侯分配四方,無屬中嶽,故不須巡之也。“朔巡守”。正義曰:\CJKunderwave{釋訓}云:“朔,北方也。”故\CJKunderwave{堯典}及此與\CJKunderwave{禹貢}皆以“朔”言北,史變文耳。 \par}

{\noindent\zhuan\zihao{6}\fzbyks 傳“巡守”至“一牛”。正義曰:此承四巡之下,是巡守既遍,然後歸也。以上受終在文祖之廟,知此以“告至文祖之廟”。才、藝、文、德其義相通,故“藝”為文也。“文祖”、“藝祖”史變文耳。\CJKunderwave{王制}說巡守之禮云:“歸,格于祖禰,用特。”此不言“禰”,故傳推之。“言祖則考著”,考近于祖,舉尊以及卑也。“特”者獨也,故為“一牛”。此惟言“文祖”,故云“一牛”。遍告諸廟,廟用一牛,故鄭注:“彼雲祖下及禰皆一牛也。”此時舜始攝位,未自立廟,故知告堯之文祖也。 \par}

{\noindent\zhuan\zihao{6}\fzbyks 傳“各會”至“可知”。正義曰:此總說巡守之事,而言“群后四朝”,是言四方諸侯各自會朝于方岳之下。凡四處別朝,故云“四朝”。上文“肆覲東后”是為一朝,四岳禮同,四朝見矣。計此不宜須重言之。為將說“敷奏”之事,“敷奏”因朝而為,故申言之。申,重也。此是巡守大法,文在舜攝位之時,嫌堯本不然,故云“堯舜同道,舜攝則然,堯又可知”也。堯法已然,舜無增改,而言此以美舜者,道同于堯,足以為美,故史錄之。 \par}

{\noindent\zhuan\zihao{6}\fzbyks 傳“敷陳”至“能用”。正義曰:“敷”者佈散之言,與陳設義同,故為陳也。“奏”是進上之語,故為進也。諸侯四處來朝,每朝之處,舜各使陳進其治理之言,令自說巳之治政。既得其言,乃依其言明試之,以要其功。以如其言,即功實成,則賜之車服,以表顯其人有才能可用也。人以車服為榮,故天子之賞諸侯,皆以車服賜之。\CJKunderwave{覲禮}雲“天子賜侯氏以車服”是也。 \par}

{\noindent\shu\zihao{5}\fzkt “歲二月”至“以庸”。正義曰:舜既班瑞群后,即以其歲二月東行巡省守土之諸侯,至于岱宗之嶽,燔柴告至,又望而以秩次祭于其方岳山川。柴望既畢,遂以禮見東方諸侯諸國之君,于此諸國協其四時氣節、月之大小,正其日之甲乙,使之齊一。均同其國之法制、度之丈尺、量之斛鬥、衡之斤兩,皆使齊同,無輕重大小。又修五禮:吉、兇、賓、軍、嘉之禮。修五玉:公侯伯子男所執之圭璧也。又修三帛:諸侯世子、公之孤、附庸之君所執玄、纁、黃之帛也。又修二生:卿所執羔、大夫所執雁也。又修一死:士所執雉也。自“五玉”至于“一死”,皆蒙上“修”文,總言所用。玉、帛、生、死皆為贄以見天子也。其贄之內,如五玉之器,禮終乃復還之。其帛與生、死則不還也。東嶽禮畢,即向衡山,五月南巡守,至于南嶽之下,柴望以下一如岱宗之禮。南嶽禮畢,即向華山,八月西巡守,至于西嶽之下,其禮如初時,如岱宗所行。西嶽禮畢,即向恆山。朔,北也。十有一月北巡守,至于北嶽之下,一如西嶽之禮。巡守既周,乃歸京師。藝,文也。至于文祖之廟,用特牛之牲設祭以告巡守歸至也。從是以後每五載一巡守,其巡守之年,諸侯群后四方各朝天子于方岳之下。其朝之時,各使自陳進其所以治化之言。天子明試其言,以考其功,功成有驗,則賜之車服,以表顯其有功,能用事。 \par}

\textcolor{red}{肇十}有二州,\footnote{肇,始也。\CJKunderline{禹}治水之後,舜分冀州為幽州、幷州,分青州為營州,始置十二州。肇音兆。十有二州,謂冀、兗、青、徐、荊、楊、豫、梁、雍、並、幽、營也。}封十有二山,濬川。\footnote{封,大也。每州之名山殊大之,以為其州之鎮。有流川則深之,使通利。濬,荀俊反。}

{\noindent\zhuan\zihao{6}\fzbyks 傳“肇始”至“二州”。正義曰:“肇,始”,\CJKunderwave{釋詁}文。\CJKunderwave{禹貢}治水之時,猶為九州,今始為十二州,知“\CJKunderline{禹}治水之後”也。\CJKunderline{禹}之治水,通\CJKunderline{鯀}九載,為作十有三載,則舜攝位元年,九州始畢。當是二年之後,以境界太遠,始別置之。知“分冀州為幽州、幷州”者,以王者廢置,理必相沿。\CJKunderwave{周禮·職方氏}九州之名有幽、並,無徐、梁。周立州名必因于古,知舜時當有幽、並。\CJKunderwave{職方}幽、並山川于\CJKunderwave{禹貢}皆冀州之域,知分冀州之域為之也。\CJKunderwave{爾雅·釋地}九州之名于\CJKunderwave{禹貢}無樑、青,而有幽、營,雲“燕曰幽州,齊曰營州”。孫炎以\CJKunderwave{爾雅}之文與\CJKunderwave{職方}、\CJKunderwave{禹貢}並皆不同,疑是殷制。則營州亦有所因,知舜時亦有營州。齊即青州之地,知分青州為之。于此居攝之時,始置十有二州,蓋終舜之世常然。宣三年\CJKunderwave{左傳}云:“昔夏之方有德也,貢金九牧。”則\CJKunderline{禹}登王位,還置九州,其名蓋如\CJKunderwave{禹貢},其境界不可知也。 \par}

{\noindent\zhuan\zihao{6}\fzbyks 傳“封大”至“通利”。正義曰:\CJKunderwave{釋詁}云:“冢,大也。”舍人曰:“冢,封之大也。”定四年\CJKunderwave{左傳}雲“封豕長蛇”,相對是“封”為大也。\CJKunderwave{周禮·職方氏}每州皆雲“其山鎮曰某山”,揚州會稽,荊州衡山,豫州華山,雍州吳山,冀州霍山,幷州恆山,幽州醫無閭,青州沂山,兗州岱山,是周時九州之內最大之山。舜時十有二山,事亦然也。州內雖有多山,取其最高大者,以為其州之鎮,特舉其名,是殊大之也。其有川,無大無小,皆當深之,故云“濬川”,有“流川則深之,使通利也。”\CJKunderwave{職方氏}每州皆雲其川、其浸,亦舉其州內大川,但令小大俱通,不復舉其大者,故直雲濬之而已。 \par}

象以典刑,\footnote{象,法也。法用常刑,用不越法。}流宥五刑,\footnote{宥,寬也。以流放之法寬五刑。宥音又,馬云:“宥,二宥也。”}鞭作官刑,\footnote{鞭以作為治官事之刑。}朴作教刑,\footnote{朴,\xpinyin*{榎}楚也。不勤道業則撻之。朴,普卜反,徐敷卜反。榎,皆雅反。}金作贖刑。\footnote{金,黃金。誤而入刑,出金以贖罪。○贖,石欲反,徐音樹。}

{\noindent\zhuan\zihao{6}\fzbyks 傳“象法”至“越法”。正義曰:\CJKunderwave{易·繫辭}云:“象也者,象此者也。”又曰:“天垂象,聖人則之。”是“象”為仿法,故為法也。五刑雖有常法,所犯未必當條,皆須原其本情,然後斷決。或情有差降,俱被重科;或意有不同,失出失入,皆是違其常法。故令依法用其常刑,用之使不越法也。 \par}

{\noindent\zhuan\zihao{6}\fzbyks 傳“宥寬”至“五刑”。正義曰:“宥,寬”,\CJKunderwave{周語}文,“流”謂徙之遠方;“放”,使生活;以流放之法寬縱五刑也。此惟解以流寬之刑,而不解宥寬之意。\CJKunderline{鄭玄}云:“其輕者或流放之,四罪是也。”\CJKunderline{王肅}云:“謂君不忍刑殺,宥之以遠方。”然則知此是據狀合刑,而情差可恕,全赦則太輕,致刑即太重,不忍依例刑殺,故完全其體,宥之遠方。應刑不刑,是寬縱之也。上言“典刑”,此言“五刑”者,其法是常,其數則五,“象以典刑”謂其刑之也,“流宥五刑”謂其遠縱之也。“流”言“五刑”,則“典刑”亦五,其文互以相見。\CJKunderline{王肅}云:“言宥五刑,則正五刑見矣。”是言二文相通之意也。“典刑”是其身,“流宥”離其鄉,流放致罪為輕,比鞭為重,故次“典刑”之下。先言“流宥”,鞭撲雖輕,猶虧其體,比于出金贖罪又為輕,且\CJKunderwave{呂刑}五罰雖主贖五刑,其鞭撲之罪亦容輸贖,故後言之。此正刑五與流宥鞭撲俱有常法,“典”字可以統之,故發首言“典刑”也。 \par}

{\noindent\zhuan\zihao{6}\fzbyks 傳“以鞭”至“之刑”。正義曰:此有鞭刑,則用鞭久矣。\CJKunderwave{周禮·滌狼氏}:“誓大夫曰,敢不關,鞭五百。”\CJKunderwave{左傳}有鞭徒人費、圉人犖是也,子玉使鞭七人,\CJKunderline{衛侯}鞭師曹三百,日來亦皆施用。大隨造律,方使廢之。“治官事之刑”者,言若于官事不治則鞭之,蓋量狀加之,未必有定數也。 \par}

{\noindent\zhuan\zihao{6}\fzbyks 傳“撲榎”至“撻之”。正義曰:\CJKunderwave{學記}云:“榎楚二物,以收其威。”\CJKunderline{鄭玄}云:“榎,槄也。楚,荊也。二物可以撲撻犯禮者。”知“撲”是榎楚也。既言“以收其威”,知“不勤道業則撻之”。\CJKunderwave{益稷}云:“撻以記之。”又\CJKunderwave{大射}、\CJKunderwave{鄉射}皆雲司馬搢撲。則撲亦官刑,惟言“作教刑”者,官刑鞭撲俱用,教刑惟撲而已,故屬撲于教。其實官刑亦當用撲,蓋重者鞭之,輕者撻之。 \par}

{\noindent\zhuan\zihao{6}\fzbyks 傳“金黃”至“贖罪”。正義曰:此以“金”為黃金。\CJKunderwave{呂刑}“其罰百鍰”,傳為“黃鐵”。俱是贖罪而金鐵不同者,古之金銀銅鐵總號為金,別之四名耳。\CJKunderwave{釋器}云:“黃金謂之蕩,白金謂之銀。”是黃金白銀俱名金也。\CJKunderwave{周禮·考工記}攻金之工築氏為削,冶氏為殺矢,鳧氏為鍾,\hanaa{㮚}氏為量,段氏為鎛,桃氏為劍,其所為者有銅,有鐵,是銅鐵俱名為金,則鐵名亦包銅矣。此傳“黃金”、\CJKunderwave{呂刑}“黃鐵”,皆是今之銅也。古之贖罪者皆用銅,漢始改用黃金,但少其斤兩,令與銅相敵。故\CJKunderline{鄭玄}駁\CJKunderwave{異義}言:“贖死罪千鍰,鍰六兩大半兩,為四百一十六斤十兩六半兩銅,與今贖死罪金三斤為價相依附。”是古贖罪皆用銅也。實謂銅而謂之金、鐵,知傳之所言謂銅為金、鐵耳。漢及後魏贖罪皆用黃金,後魏以金難得,合金一兩收絹十匹。今律乃復依古,死罪贖銅一百二十斤,于古稱為三百六十斤。孔以鍰為六兩,計千鍰為三百七十五斤,今贖輕于古也。誤而入罪,出金以贖,即律“過失殺傷人,各依其狀以贖論”是也。\CJKunderwave{呂刑}所言“疑赦”乃罰者,即今律“疑罪各從其實,以贖論”是也。疑謂虛實之證等,是非之理均,或事涉疑似,旁無證見,或雖有證見,事非疑似,如此之類,言皆為疑罪。疑而罰贖,\CJKunderwave{呂刑}已明言。誤而輸贖,于文不顯,故此傳指言誤而入罪以解此“贖”。鞭撲加于人身,可雲“撲作教刑”,金非加人之物,而言“金作贖刑”,出金之與受撲俱是人之所患,故得指其所出,以為刑名。 \par}

眚災肆赦,怙終賊刑。\footnote{眚,過。災,害。肆,緩。賊,殺也。過而有害,當緩赦之。怙奸自終,當刑殺之。眚,所景反。怙音戶。}欽哉,欽哉,惟刑之恤哉!\footnote{舜陳典刑之義,敕天下使敬之,憂欲得中。恤,峻律反,憂也。}

{\noindent\zhuan\zihao{6}\fzbyks 傳“眚過”至“殺之”。正義曰:\CJKunderwave{春秋}言“肆眚”者,皆謂緩縱過失之人,是“肆”爰緩也,“眚”爰過也。\CJKunderwave{公羊傳}雲“害物曰災”,是為害也。宣二年\CJKunderwave{左傳}晉侯殺趙盾,“使鉏麑賊之”,是“賊”為殺也。此經二句承上“典刑”之下,總言用刑之要。過而有害,雖據狀合罪,而原心非故,如此者當緩赦之。小則恕之,大則宥之,上言“流宥”、“贖刑”是也。怙恃奸詐,欺罔時人,以此自終,無心改悔,如此者當刑殺之。小者刑之,大者殺之,上言“典刑”及“鞭”、“撲”皆是也。經言“賊刑”,傳雲“刑殺”,不順經文者,隨便言之。 \par}

{\noindent\zhuan\zihao{6}\fzbyks 傳“舜陳”至“得中”。正義曰:此經二句,舜之言也。不言“舜曰”,以可知而略之。舜既制此典刑,又陳典刑之義,以敕天下百官,使敬之哉,敬之哉,惟刑之憂哉。憂念此刑,恐有濫失,欲使得中也。 \par}

流\CJKunderline{共工}于幽洲,\footnote{象恭滔天,足以惑世,故流放之。幽洲,北裔。水中可居者曰州。共音恭。\CJKunderwave{左傳}:“少皞氏有不才子,毀信廢忠,崇飾惡言,靖僣庸回,服讒搜慝,以誣盛德,天下之民謂之窮奇。”杜預云:“即\CJKunderline{共工}。”裔,以制反。}放\CJKunderline{驩兜}于崇山,\footnote{黨于\CJKunderline{共工},罪惡同。崇山,南裔。驩,呼端反。兜,丁侯反。\CJKunderwave{左傳}:“帝鴻氏有不才子,掩義隱賊,好行兇德,醜類惡物,頑嚚不友,是與比周,天下之民謂之渾敦。”杜預云:“即\CJKunderline{驩兜}也。帝鴻,\CJKunderline{黃帝}也。”}竄三苗于三危,\footnote{三苗,國名。縉雲氏之後,為諸侯,號饕餮。三危,西裔。竄,七亂反。三苗,馬、王云:“國名也。縉雲氏之後,為諸侯,蓋饕餮也。”\CJKunderwave{左傳}:“縉雲氏有不才子,貪于飲食,冒于貨賄,侵欲崇侈,不可盈厭,聚斂積實,不知紀極,不念孤寡,不恤窮匱,天下之民以比三兇,謂之饕餮。”杜預云:“縉云,\CJKunderline{黃帝}時官名,非帝子孫,故以比三兇也。貪財曰饕,貪食曰餮。”縉音晉。饕,土刀反。餮,他節反。}殛\CJKunderline{鯀}于羽山,\footnote{方命圮族,績用不成,殛竄放流,皆誅也。異其文,述作之體。羽山,東裔,在海中。殛,紀力反。\CJKunderline{鯀},故本反。\CJKunderwave{左傳}:“\CJKunderline{顓頊}氏有不才子,不可教訓,不知話言,告之則頑,舍之則嚚,傲狼明德,以亂天常,天下之民謂之檮杌。”杜預云:“即\CJKunderline{鯀}也。檮杌,兇頑無儔匹之貌。”}四罪而天下\textcolor{red}{咸服}。\footnote{皆服舜用刑當其罪,故作者先敘典刑而連引四罪,明皆徵用所行,于此總見之。}

{\noindent\zhuan\zihao{6}\fzbyks 傳“象恭”至“曰洲”。正義曰:\CJKunderwave{堯典}言\CJKunderline{共工}之行云:“靜言庸違,象恭滔天。”言貌象恭敬,傲狠漫天,足以疑惑世人,故流放也。\CJKunderwave{左傳}說此事言“投諸四裔”。\CJKunderwave{釋地}雲“燕曰幽州”,知“北裔”也。“水中可居者曰洲”,\CJKunderwave{釋水}文。李巡曰:“四方有水,中央高,獨可居,故曰洲。”天地之勢,四邊有水,鄒衍書說“九州之外有瀛海環之”,是九州居水內,故以州為名,共在一洲之上,分之為九耳。州取水內為名,故引\CJKunderwave{爾雅}解“州”也。“投之四裔”,“裔”訓遠也,當在九州之外,而言“于幽州”者,在州境之北邊也。\CJKunderwave{禹貢}羽山在徐州,三危在雍州,故知北裔在幽州。下三者所居皆言山名,此\CJKunderline{共工}所處不近大山,故舉州言之。此流四凶在治水前,于時未作十有二州,則無幽州之名,而云“幽州”者,史據後定言之。 \par}

{\noindent\zhuan\zihao{6}\fzbyks 傳“黨于”至“南裔”。正義曰:\CJKunderline{共工}象恭滔天而\CJKunderline{驩兜}薦之,是“黨于\CJKunderline{共工},罪惡同”,故放之也。\CJKunderwave{左傳}說此事云:“流四凶族,投諸四裔。”則四方方各有一人,幽州在北裔,雍州三危在西裔,徐州羽山在東裔,三方既明,知崇山在南裔也。\CJKunderwave{禹貢}無崇山,不知其處,蓋在衡嶺之南也。 \par}

{\noindent\zhuan\zihao{6}\fzbyks 傳“三苗”至“西裔”。正義曰:昭元年\CJKunderwave{左傳}說自古諸侯不用王命者,“虞有三苗,夏有觀扈”,知三苗是國,其國以三苗為名,非三國也。杜預言“三苗地闕,不知其處”。三兇皆是王臣,則三苗亦應是諸夏之國入仕王朝者也。文十八年\CJKunderwave{左傳}言:“縉雲氏有不才子,貪于飲食,冒于貨賄,侵欲崇侈,不可盈厭,聚斂積實,不知紀極,不分孤寡,不恤窮匱天下之民以比三兇,謂之饕餮。”即此三苗是也。知其然者,以\CJKunderwave{左傳}說此事言:“舜臣堯,流四凶族渾敦、窮奇、檮杌、饕餮,投諸四裔,以御螭魅。”謂此\CJKunderline{驩兜}、\CJKunderline{共工}、三苗與\CJKunderline{鯀}也。雖知彼言四凶,此等四人,但名不同,莫知孰是,惟當驗其行跡,以別其人。\CJKunderwave{左傳}說窮奇之行雲“靖譖庸回”,\CJKunderwave{堯典}言\CJKunderline{共工}之行雲“靜言庸違”,其事既同,知窮奇是\CJKunderline{共工}也。\CJKunderwave{左傳}說渾敦之行雲“醜類惡物,是與比周”,\CJKunderwave{堯典}言\CJKunderline{驩兜}薦舉\CJKunderline{共工},與惡比周,知渾敦是\CJKunderline{驩兜}也。\CJKunderwave{左傳}說檮杌之行言“不可教訓,不知話言,傲狠明德,以亂天常”,\CJKunderwave{堯典}言\CJKunderline{鯀}之行雲“咈哉,方命圮族”,其事既同,知檮杌是\CJKunderline{鯀}也。惟三苗之行\CJKunderwave{堯典}無文,\CJKunderline{鄭玄}具引\CJKunderwave{左傳}之文乃云:“命\CJKunderline{驩兜}舉\CJKunderline{共工},則\CJKunderline{驩兜}為渾敦也,\CJKunderline{共工}為窮奇也,\CJKunderline{鯀}為檮杌也,而三苗為饕餮亦可知。”是先儒以書傳相考,知三苗是饕餮也。\CJKunderwave{禹貢}雍州言“三危既宅,三苗丕敘”,知三危是西裔也。 \par}

{\noindent\zhuan\zihao{6}\fzbyks 傳“方命”至“海中”。正義曰:“方命圮族”,是其本性;“績用不成”,試而無功;二者俱是其罪,故並言之。\CJKunderwave{釋言}云:“殛,誅也。”傳稱流四凶族者,皆是流而謂之“殛竄放流,皆誅”者,流者移其居處,若水流然,罪之正名,故先言也。放者使之自活,竄者投棄之名,殛者誅責之稱,俱是流徙,異其文,述作之體也。四者之次,蓋以罪重者先。\CJKunderline{共工}滔天,為罪之最大。\CJKunderline{驩兜}與之同惡,故以次之。\CJKunderwave{祭法}以\CJKunderline{鯀}障洪水,故列諸祀典,功雖不就,為罪最輕,故後言之。\CJKunderwave{禹貢}徐州雲“蒙羽其藝”,是羽山為東裔也。\CJKunderwave{漢書·地理志}羽山在東海郡祝其縣西南,海水漸及,故言“在海中”也。 \par}

{\noindent\zhuan\zihao{6}\fzbyks 傳“皆服”至“見之”。正義曰:此四罪者徵用之初即流之也,舜以微賤超升上宰,初來之時,天下未服,既行四罪,故天下皆服舜用刑得當其罪也。自“象以典刑”以下,徵用而即行之,于此居攝之後,追論成功之狀。故作者先敘典刑,言舜重刑之事,而連引四罪,述其刑當之驗,明此諸事皆是徵用之時所行,于此總見之也。知此等諸事皆“徵用所行”者,\CJKunderwave{洪範}雲“\CJKunderline{鯀}則殛死,\CJKunderline{禹}乃嗣興”,僖三十三年\CJKunderwave{左傳}雲“舜之罪也殛\CJKunderline{鯀},其舉也興\CJKunderline{禹}”,襄二十一年\CJKunderwave{左傳}雲“\CJKunderline{鯀}殛而\CJKunderline{禹}興”,此三者皆言殛\CJKunderline{鯀}而後用\CJKunderline{禹},為治水是徵用時事,四罪在治水之前,明是“徵用所行”也。又下雲\CJKunderline{禹}讓稷、契、\CJKunderline{皋陶},帝因追美三人之功,所言稷播百穀、契敷五教、\CJKunderline{皋陶}作士皆是徵用時事,\CJKunderline{皋陶}所行“五刑有服”、“五流有宅”即是“象以典刑”、“流宥五刑”,此為徵用時事足可明矣。而\CJKunderline{鄭玄}以為“\CJKunderline{禹}治水事畢,乃流四凶”。故\CJKunderline{王肅}難鄭言:“若待\CJKunderline{禹}治水功成,而後以\CJKunderline{鯀}為無功殛之,是為舜用人子之功,而流放其父,則\CJKunderline{禹}之勤勞適足使父致殛,為舜失五典克從之義,\CJKunderline{禹}陷三千莫大之罪,進退無據,亦甚迂哉!” \par}

{\noindent\shu\zihao{5}\fzkt “肇十”至“咸服”。正義曰:史言舜既攝位,出行巡守,復分置州域,重慎刑罰。于\CJKunderline{禹}治水後,始分置十有二州,每州以一大山為鎮,殊大者十有二山。深其州內之川,使水通利。又留意于民,詳其罪罰,依法用其常刑,使罪各當,刑不越法。用流放之法寬宥五刑。五刑雖有犯者,或以恩減降,不使身服其罪,所以流放宥之。五刑之外,更有鞭作治官事之刑;有撲作師儒教訓之刑;其有意善功惡,則令出金贖罪之刑;若過誤為害,原情非故者,則緩縱而赦放之;若怙恃奸詐,終行不改者,則賊殺而刑罪之。舜慎刑如此,又設言以誡百官曰:“敬之哉!敬之哉!推此刑罰之事最須憂念之哉!”令勤念刑罰,不使枉濫也。又言舜非于攝位之後,方始重慎刑罪,初于登用之日即用刑當其罪,流徙\CJKunderline{共工}于北裔之幽州,放逐\CJKunderline{驩兜}于南裔之崇山,竄三苗于西裔之三危,誅殛伯\CJKunderline{鯀}于東裔之羽山。行此四罪,各得其實,而天下皆服從之。 \par}

\textcolor{red}{二十}有八載,帝乃殂落。\footnote{殂落,死也。堯年十六即位,七十載求禪,試舜三載,自正月上日至崩二十八載,堯死壽一百一十七歲。殂,才枯反。}百姓如喪考妣,\footnote{考妣,父母。言百官感德思慕。喪如字,又息浪反。妣,必履反,父曰考,母曰妣。}三載,四海遏密\textcolor{red}{八音}。\footnote{遏,絕。密,靜也。八音,金、石、絲、竹、匏、土、革、木。四夷絕音三年,則華夏可知。言盛德恩化所及者遠。遏,安葛反,或音謁。八音謂金,鍾也;石,磬也;絲,琴瑟也;竹,篪笛也;匏,笙也;土,壎也;革,鼓也;木,柷敔也。匏,白交反。}

{\noindent\zhuan\zihao{6}\fzbyks 傳“殂落”至“七歲”。正義曰:“殂落,死也”,\CJKunderwave{釋詁}文。李巡曰:“殂落,堯死之稱。”郭璞曰:“古死尊卑同稱。故\CJKunderwave{書}堯曰‘殂落’,舜曰‘陟方乃死’。”謂之“殂落”者,蓋“殂”為往也,言人命盡而往;“落”者若草木葉落也。堯以十六即位,明年乃為元年。七十載求禪,求禪之時八十六也。試舜三年,自正月上日至崩二十八載,總計其數,凡壽一百一十七歲。案\CJKunderwave{堯典}求禪之年即得舜而試之,求禪試舜共在一年也。更得二年,即為歷試三年,故下傳雲“歷試二年”。與攝位二十八年,合得為“三十在位”。故\CJKunderline{王肅}云:“徵用三載,其一在徵用之年,其餘二載,與攝位二十八年凡三十歲也。”故孔傳云:“歷試二年。”明其一年在徵用之限。以此計之,淮有一百一十六歲,不得有七,蓋誤為七也。 \par}

{\noindent\zhuan\zihao{6}\fzbyks 傳“考妣”至“思慕”。正義曰:\CJKunderwave{曲禮}云:“生曰父母,死曰考妣。”\CJKunderline{鄭玄}云:“考,成也,言其德行之成也。妣之言媲也,媲于考也。”\CJKunderwave{喪服}為父為君,同服斬衰。\CJKunderwave{檀弓}說事君之禮云:“服勤至死,方喪三年。”\CJKunderline{鄭玄}云:“方喪資于事父,凡此以義為制。”義重則恩輕,其情異于父。“如喪考妣”言百官感德,情同父母,思慕深也。諸經傳言“百姓”或為百官,或為萬民,知此“百姓”是百官者,以\CJKunderwave{喪服}庶民為天子齊衰三月,畿外之民無服,不得如考妣,故知百官也。 \par}

{\noindent\zhuan\zihao{6}\fzbyks 傳“遏絕”至“者遠”。正義曰:“密,靜”,\CJKunderwave{釋詁}文。“遏”,止絕之義,故為絕也。\CJKunderwave{周禮·太師}云:“播之以八音:金、石、土、革、絲、木、匏、竹。”鄭云:“金,鍾鎛也;石,磬也;土,壎也;革,鼓鼗也;絲,琴瑟也;木,柷敔也;匏,笙也;竹,管簫也。”傳言“八音”與彼次不同者,隨便言耳。\CJKunderwave{釋地}云:“九夷八狄七戎六蠻謂之四海。”夷狄尚絕音三年,則華夏內國可知也。\CJKunderwave{喪服}諸侯之大夫為天子正服穗衰,既葬除之。今能使四夷三載絕音,言堯有盛德,恩化所及遠也。 \par}

{\noindent\shu\zihao{5}\fzkt “二十”至“八音”。正義曰:舜受終之後,攝天子之事二十有八載,\CJKunderline{帝堯}乃死。百官感德思慕,如喪考妣。三載之內,四海之人,蠻夷戎狄皆絕靜八音而不復作樂。是堯盛德恩化所及者遠也。 \par}

\textcolor{red}{月正}元日,\CJKunderline{舜}格于文祖,\footnote{月正,正月。元日,上日也。舜服堯喪三年畢,將即政,故復至文祖廟告。復,扶又反。}詢于四岳,闢四門,\footnote{詢,謀也。謀政治于四岳,開闢四方之門未開者,廣致眾賢。闢,婢亦反,徐甫亦反。}明四目,達四聰。\footnote{廣視聽于四方,使天下無壅塞。}

{\noindent\zhuan\zihao{6}\fzbyks 傳“月正”至“廟告”。正義曰:“正”訓長也,“月正”言月之最長,正月長于諸月,“月正”還是正月也。上日,日之最上;元日,日之最長,“元日”還是上日。\CJKunderline{王肅}云:“月正元日,猶言正月上日。”變文耳。\CJKunderwave{禮}雲“令月吉日”,又變文言“吉月令辰”,此之類也。知“舜服堯喪三年畢,將即政”者,以堯存且攝其位,堯崩謙而不居。\CJKunderwave{孟子}云:“堯崩,三年喪畢,舜避\CJKunderline{丹朱}于南河之南。天下諸侯朝覲者不之堯子而之舜,獄訟者不之堯子而之舜,謳歌者不之堯子而謳歌舜。曰‘天也。’然後之中國踐天子位。”既言然矣,此文又承“三載”之下,故知舜服堯喪三年畢,將欲即政。“復至文祖廟告”,前以攝位告,今以即政告也。此猶是堯之文祖,自此以後舜當自立文祖之廟,堯之文祖當遷于\CJKunderline{丹朱}之國也。 \par}

{\noindent\zhuan\zihao{6}\fzbyks 傳“詢謀”至“眾賢”。正義曰:“詢,謀”,\CJKunderwave{釋詁}文。“闢”訓開,開四方之門,謂開仕路,引賢人也。\CJKunderwave{論語}云:“從我于陳蔡者,皆不及門也。”門者行之所由,故以門言仕路。以堯舜之聖,求賢久矣,今更言開門,是開“其未開”者,謂多設取士之科,以此廣致眾賢也。 \par}

{\noindent\zhuan\zihao{6}\fzbyks 傳“廣視”至“壅塞”。正義曰:“聰”謂耳聞之也。既雲“明四目”不雲“聰四耳”者,目視苦其不明,耳聰貴其及遠,“明”謂所見博,“達”謂聽至遠,二者互以相見。故傳總申其意“廣視聽于四方,使天下無壅塞”。天子之聞見在下,必由近臣四岳親近之官,故與謀此事也。 \par}

諮十有二牧,曰:“食哉,惟時\footnote{諮亦謀也。所重在于民食,惟當敬授民時。}柔遠能邇,惇德允元,\footnote{柔,安。邇,近。敦,厚也。元,善之長。言當安遠,乃能安近。厚行德信,使足長善。惇音敦。長,張丈反,下同。}而難任人,蠻夷\textcolor{red}{率服}。”\footnote{任,佞。難,拒也。佞人斥遠之,則忠信昭于四夷,皆相率而來服。難音乃旦反。任音壬,又音而鴆反。}

{\noindent\zhuan\zihao{6}\fzbyks 傳“諮亦”至“民時”。正義曰:“諮,謀”,\CJKunderwave{釋詁}文。以上“帝曰,諮”,上連“帝曰”,故為諮嗟,此則上有“詢于四岳”,言“諮十有二牧”,故為謀也。立君所以牧民,民生在于粒食,是君之所重。\CJKunderwave{論語}雲“所重民食”,謂年穀也。種殖收斂,及時乃獲,故“惟當敬授民時”。 \par}

{\noindent\zhuan\zihao{6}\fzbyks 傳“柔安”至“長善”。正義曰:“柔,安”、“邇,近”、“惇,厚”,皆\CJKunderwave{釋詁}文。“元,善之長”,\CJKunderwave{易}文言也。安近不能安遠,遠人或來擾亂,雖欲安近,近亦不安。人君為政,若其不能安近,但戒使之柔遠,故能安近。言當安彼遠人,乃能安近。欲令遠言皆安也。\CJKunderline{王肅}云:“能安遠者先能安近。”知不然者,以牧在遠方,故據遠言之。“惇德”者,令人君厚行德也。“允元”者,信使足為長善也。言人君厚行德之與信,使足為善長,民必效之為善而行也。 \par}

{\noindent\zhuan\zihao{6}\fzbyks 傳“任佞”至“來服”。正義曰:“任,佞”,\CJKunderwave{釋詁}文。孫炎云:“似可任之佞也。”\CJKunderwave{論語}說為邦之法雲“遠佞人”,“佞人殆”,故以難距佞人為“斥遠之”,令不幹朝政。朝無佞人,則“忠信昭于四夷,皆相率而來服”也。舉蠻夷而戎狄亦見矣。 \par}

{\noindent\shu\zihao{5}\fzkt “月正”至“率服”。正義曰:自此已下言舜真為天子,命百官授職之事。舜既除堯喪,以明年之月正元日,舜至于文祖之廟,告己將即正位為天子也。告廟既訖,乃謀政治于四岳之官。所謀開四方之門,大為仕路致眾賢也。明四方之目,使為己遠視四方也。達四方之聰,使為己遠聽聞四方也。恐遠方有所擁塞,令為己悉聞見之。既謀于四岳,又別敕州牧,諮十有二牧曰:“人君最所重者,在于民之食哉!惟當敬授民之天時,無失其農要。為政務在安民,當安彼遠人,則能安近人耳。遠人不安,則近亦不安。”欲令遠近皆安之也。“又當厚行德信,而使足為善長”。欲令諸侯皆厚行其德,為民之師長。“而難拒佞人,斥遠之,使不幹朝政,如是則誠信昭于四夷,自然蠻夷皆相率而來服也”。 \par}

\textcolor{red}{\CJKunderline{舜}曰}:“諮!四岳,有能奮庸熙帝之載,\footnote{奮,起。庸,功。載,事也。訪群臣有能起發其功,廣堯之事者。言“舜曰”以別堯。奮,弗運反。}使宅百揆,亮采惠疇?”\footnote{亮,信。惠,順也。求其人使居百揆之官,信立其功,順其事者誰乎?}僉曰:“\CJKunderline{伯禹}作司空。”\footnote{四岳同辭而對,\CJKunderline{禹}代\CJKunderline{鯀}為宗伯,入為天子司空。治洪水有成功,言可用之。}

{\noindent\zhuan\zihao{6}\fzbyks 傳“奮起”至“別堯”。正義曰:“奮”是起動之意,故為起也。\CJKunderwave{釋詁}云:“庸,勞也。”勞亦功也。\CJKunderline{鄭玄}云:“載,行也。”\CJKunderline{王肅}云:“載,成也。”孔以“載”為事也,各自以意訓耳。舜受堯禪,當繼行其道。行之在于任臣,百揆臣之最貴,求能起發其功,廣大\CJKunderline{帝堯}之事者,欲任之。舜既即位,可以稱帝,而言“舜曰”者,承堯事下,言“舜曰”以別堯,于此一別,以下稱帝也。 \par}

{\noindent\zhuan\zihao{6}\fzbyks 傳“亮信”至“誰乎”。正義曰:“亮,信”,\CJKunderwave{釋詁}文。“惠,順”,\CJKunderwave{釋言}文。上雲舜“納于百揆”,“百揆”是官名,故求其人,使居百揆之官。居官則當信立其功,能順其事者誰乎?此官任重,當統群職繼堯之功,故歷言所順而後始問誰乎,異于餘官先言“疇”也。 \par}

{\noindent\zhuan\zihao{6}\fzbyks 傳“四岳”至“用之”。正義曰:“僉”訓為皆,故云“四岳皆同辭而對”也。\CJKunderwave{國語}云:“有崇伯\CJKunderline{鯀},堯殛之于羽山。”賈逵云:“崇,國名。伯,爵也。”\CJKunderline{禹}代\CJKunderline{鯀}為崇伯,入為天子司空,以其伯爵,故稱“\CJKunderline{伯禹}”。言人之賢而舉其為官,知\CJKunderline{禹}“治洪水自成功,言可用”也。 \par}

帝曰:“俞,諮!\CJKunderline{禹},汝平水土,惟時懋哉!”\footnote{然其所舉,稱\CJKunderline{禹}前功以命之。懋,勉也。惟居是百揆,勉行之。俞,以朱反。懋音茂,王云:“勉也。”馬云:“美也。”}\CJKunderline{禹}拜稽首,讓于\CJKunderline{稷}、\CJKunderline{契}暨\CJKunderline{皋陶}。\footnote{居稷官者棄也。契、\CJKunderline{皋陶},二臣名。稽首,首至地。稽音啟。稽首,首至地,臣事君之禮。契,息列反。陶音遙。}帝曰:“俞,汝\textcolor{red}{往哉}!”\footnote{然其所推之賢,不許其讓,敕使往宅百揆。}

{\noindent\zhuan\zihao{6}\fzbyks 傳“然其”至“行之”。正義曰:\CJKunderline{禹}平水土,往前之事,嫌其今覆命之令平水土,故云“稱\CJKunderline{禹}前功以命之”。“懋,勉”,\CJKunderwave{釋詁}文。 \par}

{\noindent\zhuan\zihao{6}\fzbyks 傳“居稷”至“首至地”。正義曰:下文帝述三人,遂變“稷”為“棄”,故解之“居稷官者棄也”。獨稱“官”者,出自\CJKunderline{禹}意耳,不必著義。鄭云:“時天下賴后稷之功,故以官名通稱。”或當然也。經因“稷、契”名單,共文言“暨\CJKunderline{皋陶}”,為文勢耳。三人為此次者,蓋以官尊卑為先後也。\CJKunderwave{周禮·太祝}:“辨九拜,一曰稽首。”稽首為敬之極,故為“首至地”。稽首是拜內之別名,\CJKunderline{禹}拜乃稽首,故云“拜稽首”也。 \par}

{\noindent\shu\zihao{5}\fzkt “舜曰”至“往哉”。正義曰:舜本以百揆攝位,今既即政,故求置其官。曰:“諮嗟!四岳等,汝于群臣之內,有能起發其功,廣大\CJKunderline{帝堯}之事者,我欲使之居百揆之官。在官而信立其功,于事能順者,其是唯乎?”四岳皆曰:“\CJKunderline{伯禹}作司空,有成功,惟此人可用。”帝曰:“然。”然其所舉得人也。乃諮嗟敕\CJKunderline{禹}:“汝本平水土,實有成功,惟當居是百揆而勉力行哉!”\CJKunderline{禹}拜稽首,讓于稷、契與\CJKunderline{皋陶}。帝曰:“然。”然其所讓實賢也。“汝但往居此職”。不許其讓也。 \par}

\textcolor{red}{帝曰}:“\CJKunderline{棄},黎民阻飢,汝后稷,播時\textcolor{red}{百穀}。”\footnote{阻,難。播,布也。眾人之難在于飢,汝后稷,布種是百穀以濟之。美其前功以勉之。阻,莊呂反,王云:“難也。”播,波左反。}

{\noindent\zhuan\zihao{6}\fzbyks 傳“阻難”至“勉之”。正義曰:“阻,難”,\CJKunderwave{釋詁}文。“播”是分散之義,故為布也。\CJKunderline{王肅}云:“播,敷也。”堯遭洪水,民不粒食,故眾民之難在于飢也。“稷”是五穀之長,立官主此稷事。“後”訓君也。帝言:“汝君此稷官,布種是百穀以濟救之。”追美其功以勸勉之。上文“讓于稷、契”,\CJKunderwave{益稷}雲“暨稷”\CJKunderwave{呂刑}雲“稷降播種”,\CJKunderwave{國語}雲“稷為天官”。單名為“稷”,尊而君之稱為“后稷”,故\CJKunderwave{詩傳}、\CJKunderwave{孝經}皆以“后稷”為言,非官稱“後”也。 \par}

{\noindent\shu\zihao{5}\fzkt “帝曰棄”至“百穀”。正義曰:帝因\CJKunderline{禹}讓三人而官不轉,各述其功以勸之。帝呼稷曰:“棄,往者洪水之時,眾民之難難在于飢,汝君為此稷之官,教民布種是百穀以濟活之。”言我佑汝功,當勉之。 \par}

\textcolor{red}{帝曰}:“\CJKunderline{契},百姓不親,五品不遜,\footnote{五品謂五常。遜,順也。}汝作司徒,敬敷五教,\textcolor{red}{在寬}。”\footnote{布五常之教務在寬,所以得人心,亦美其前功。}

{\noindent\zhuan\zihao{6}\fzbyks 傳“五品”至“順也”。正義曰:“品”謂品秩,一家之內尊卑之差,即父母兄弟子是也。教之義慈友恭孝,此事可常行,乃為“五常”耳。傳上雲“五典克從”,即此五品能順。上傳以解\CJKunderwave{五典}為五常,又解此以同之,故云“五品謂五常”。其實五常據教為言,不據品也。“遜,順”,常訓也。不順謂不義、不慈、不友、不恭、不孝也。 \par}

{\noindent\zhuan\zihao{6}\fzbyks 傳“布五”至“前功”。正義曰:文十八年\CJKunderwave{左傳}雲“布五教于四方,父義、母慈、兄友、弟恭、子孝”,是“布五常之教”也。\CJKunderwave{論語}雲“寬則得眾”,故“務在寬,所以得民心”也。治不遜之罪,宜峻法以繩之,而貴其務在寬者,此五品不遜,直是禮教不行,風俗未淳耳,未有殺害之罪,故教之務在于寬。若其不孝不恭,其人至于逆亂而後治之,于事不得寬也。 \par}

{\noindent\shu\zihao{5}\fzkt “帝曰契”至“在寬”。正義曰:帝又呼契曰:“往者天下百姓不相親睦,家內尊卑五品不能和順。汝作司徒之官,謹敬布其五常之教,務在于寬,故使五典克從。是汝之功,宜當勉之。” \par}

\textcolor{red}{帝曰}:“\CJKunderline{皋陶},蠻夷猾夏,寇賊姦宄,\footnote{猾,亂也。夏,華夏。群行攻劫曰寇,殺人曰賊。在外曰奸,在內曰宄。言無教之致。猾,戶八反。寇,苦豆反。宄音軌。}汝作士。五刑有服,\footnote{士,理官也。五刑,墨、劓、剕、宮、大辟。服,從也。言得輕重之中正。劓,魚器反,截鼻也。剕,扶味反,刖足也。大辟,婢亦反,死刑也。}五服三就。\footnote{既從五刑,謂服罪也。行刑當就三處,大罪于原野,大夫于朝,士于市。處,昌慮反。朝,直遙反。}五流有宅,五宅三居。\footnote{謂不忍加刑,則流放之,若四凶者。五刑之流,各有所居。五居之差,有三等之居,大罪四裔,次九州之外,次千里之外。}惟明\textcolor{red}{克允}。”\footnote{言\CJKunderline{皋陶}能明信五刑,施之遠近,蠻夷猾夏,使咸信服,無敢犯者。因\CJKunderline{禹}讓三臣,故歷述之。}

{\noindent\zhuan\zihao{6}\fzbyks 傳“猾亂”至“之致”。正義曰:“猾”者狡猾相亂,故“猾”為亂也。“夏”訓大也,中國有文章光華,禮義之大。定十年\CJKunderwave{左傳}雲“裔不謀夏,夷不亂華”,是中國為華夏也。“寇”者眾聚為之,賊者殺害之稱,故“群行攻劫曰寇,殺人曰賊”。成十七年\CJKunderwave{左傳}云:“亂在外為奸,在內為宄。”是“在外曰奸,在內曰宄”也。“寇賊姦宄”皆是作亂害物之名也。“蠻夷猾夏”,興兵犯邊,害大,故先言之。“寇賊姦宄”,皆國內之害,小,故後言之。\CJKunderwave{管子}曰:“倉廩實知禮節,衣食足知榮辱,讓生于有餘,爭生于不足。”往者洪水為災,下民飢困,內有寇賊為害,外則四夷犯邊,皆言無教之致也。唐堯之聖,協和萬邦,不應末年,頓至于此,蓋少有其事,辭頗增甚,歸功于人,作與奪之勢耳。 \par}

{\noindent\zhuan\zihao{6}\fzbyks 傳“士理”至“中正”。正義曰:“士”即\CJKunderwave{周禮}司寇之屬,有士師、卿士等,皆以“士”為官名。\CJKunderline{鄭玄}云:“士,察也,主察獄訟之事。”\CJKunderwave{月令}云:“命大理。”昭十四年\CJKunderwave{左傳}云:“叔魚攝理。”是謂獄官為理官也。準\CJKunderwave{呂刑}文,知五刑謂墨、劓、剕、宮、大辟也。人心服罪是順從之義,故為從也。所以服者,言得輕重之中正也。\CJKunderwave{呂刑}雲“咸庶中正”是也。 \par}

{\noindent\zhuan\zihao{6}\fzbyks 傳“既從”至“于市”。正義曰:經言“五服”,謂\CJKunderline{皋陶}所斷五刑皆服其罪,傳既訓“服”為從,故云“既從五刑謂服罪也”。“行刑當就三處”,惟謂大辟罪耳。\CJKunderwave{魯語}云:“刑五而已,無有隱者。大刑用甲兵,次刑斧鉞,中刑刀鋸,其次鑽筰,薄刑鞭撲,以威民。故大者陳之原野,小者致之市朝,五刑三次,是無隱也。”孔用彼為說,故以“三就”為原野與朝、市也。\CJKunderwave{國語}賈逵注云:“用兵甲者,諸侯逆命,征討之刑也。大夫已上于朝,士已下于市。”傳雖不言“已上、已下”,為義亦當然也。\CJKunderwave{國語}雲五刑者,謂甲兵也,斧鉞也,刀鋸也,鑽筰也,鞭撲也,與\CJKunderwave{呂刑}之五刑異也。所言“三次”即此“三就”是也。惟死罪當分就處所,其墨、劓、剕、官無常處可就也。馬、鄭、王三家皆以“三就為原野也、市朝也、甸師氏也”。案刑于甸師氏者,王之同族,刑于隱者,不與國人,慮兄弟耳,非所刑之正處。此言正刑,不當數甸師也。又市、朝異所,不得合以為一,且皆\CJKunderwave{國語}之文,其義不可通也。 \par}

{\noindent\zhuan\zihao{6}\fzbyks 傳“謂不”至“之外”。正義曰:此“五流有宅”即“流宥五刑”也。當在五刑而流放之,故知謂“不忍加刑,則流放之,若四凶”也。\CJKunderline{鄭玄}云:“舜不刑此四人者,以為堯臣,不忍刑之。”\CJKunderline{王肅}云:“謂在八議之闢,君不忍殺,宥之以遠。”八議者,\CJKunderwave{周禮·小司寇}所云議親、議故、議賢、議能、議貴、議賓、議勤是也。以君恩不忍殺,罪重不可全赦,故流之也。“五刑之流,各有所居”,謂徙置有處也。“五居之差,有三等之居”,量其罪狀為遠近之差也。四裔最遠,在四海之表,故“大罪四裔”,謂不犯死罪也。故\CJKunderwave{周禮·調人職}雲“父之仇,闢諸海外”,即與“四裔”為一也。“次九州之外”,即\CJKunderwave{王制}雲入學不率教者,“屏之遠方,西方曰僰,東方曰寄”,注云“逼寄于夷狄也”,與此“九州之外”同也。“次千里之外”者,即\CJKunderwave{調人職}雲“兄弟之仇,闢諸千里之外”也。\CJKunderwave{立政}雲“中國之外”,不同者,言“中國”者,據罪人所居之國定千里也。據其遠近,其實一也。\CJKunderwave{周禮}與\CJKunderwave{王制}既有三處之別,故約以為言。\CJKunderline{鄭玄}云:“三處者,自九州之外至于四海,三分其地,遠近若周之夷、鎮、蕃也。”然罪有輕重不同,豈五百里之校乎?不可從也。 \par}

{\noindent\zhuan\zihao{6}\fzbyks 傳“言皋”至“述之”。正義曰:“惟明”謂\CJKunderline{皋陶}之明,“克允”謂受罪者信服。故\CJKunderline{王肅}云:“惟明其罪,能使之信服。”是信施于彼也。但彼人信服,由\CJKunderline{皋陶}有信,故傳言:“\CJKunderline{皋陶}能明信五刑,施之遠近蠻夷,使咸信服。”主言信者,見其\CJKunderline{皋陶}有信,故彼信之也。 \par}

{\noindent\shu\zihao{5}\fzkt “帝曰\CJKunderline{皋陶}”至“克允”。正義曰:帝呼\CJKunderline{皋陶}曰:“往者蠻夷戎狄猾亂華夏,又有強寇劫賊外奸內宄者,為害甚大。汝作士官治之,皆能審得其情,致之五刑之罪,受罪者皆有服從之心。”言輕重得中,悉無怨恨也。“五刑有服從者,于三處就而殺之。其有不忍刑其身者,則斷為五刑而流放之。五刑之流,各有所居處。五刑所居,于三處居之。所以輕重罪得其宜,受罪無怨者,惟汝識見之明,能使之信服,故奸邪之人無敢更犯。是汝之功,宜當勉之”。因\CJKunderline{禹}之讓,以次誡之。 \par}

帝曰:“疇若予工?”僉曰:“\CJKunderline{垂}哉!”\footnote{問:“誰能順我百工事者?”朝臣舉垂。垂,臣名。垂如字,徐音睡。}帝曰:“俞,諮!\CJKunderline{垂},汝共工。”\footnote{共謂供其職事。共音恭。}

{\noindent\zhuan\zihao{6}\fzbyks 傳“問誰”至“臣名”。正義曰:\CJKunderwave{考工記}云:“國有六職,百工與居一焉。”“工”即百工,故云“問誰能順我百工事者”。直言“帝曰”,無所偏諮,故知“僉曰”是朝臣共舉垂也。 \par}

{\noindent\zhuan\zihao{6}\fzbyks 傳“共謂供其職事”。正義曰:\CJKunderwave{堯典}傳云:“共工,官稱。”即彼以“共工”二字為官名。上雲“疇若予工”,單舉“工”名,今命此人云“汝作共工”,明是帝謂此人堪供此職,非是呼此官名為“共工”也。其官或以“共工”為名,要帝意言“共”謂供此職也。 \par}

\CJKunderline{垂}拜稽首,讓于\CJKunderline{殳}\xpinyin*{\CJKunderline{斨}}暨\CJKunderline{伯與}\footnote{\CJKunderline{殳斨}、\CJKunderline{伯與}二臣名。\xpinyin*{斨},七良反。與音餘。}。帝曰:“俞,往哉!汝諧。”\footnote{汝能諧和此官。}帝曰:“疇若予上下草木鳥獸?”僉曰:“\CJKunderline{益}哉!”\footnote{上謂山,下謂澤,順謂施其政教,取之有時,用之有節。言伯益能之。益,\CJKunderline{皋陶}子也。}帝曰:“俞,諮!\CJKunderline{益},汝作朕虞。”\footnote{虞,掌山澤之官。}

{\noindent\zhuan\zihao{6}\fzbyks 傳“上謂”至“能之”。正義曰:言“上下草木鳥獸”,則上之與下各有草木鳥獸,即\CJKunderwave{周禮}山虞、澤虞之官各掌其教,知“上謂山,下謂澤”也。順其草木鳥獸之宜,明是“施其政教,取之有時,用之有節”也。馬、鄭、王本皆為“\CJKunderline{禹}曰:‘益哉!’”是字相近而彼誤耳。 \par}

{\noindent\zhuan\zihao{6}\fzbyks “作朕虞”。正義曰:此官以“虞”為名,帝言作我虞耳,“朕”非官名也。\CJKunderline{鄭玄}云:“言朕虞,重鳥獸草木。”\CJKunderwave{漢書}王莽自稱為予,立予虞之官。則莽謂此官名為“朕虞”,其義必不然也。 \par}

\CJKunderline{益}拜稽首,讓于\CJKunderline{朱虎}、\CJKunderline{熊羆}。帝曰:“俞,往哉!汝諧。”\footnote{\CJKunderline{朱虎}、\CJKunderline{熊羆},二臣名。垂、益所讓四人皆在元凱之中。羆,彼皮反。}帝曰:“諮!四岳,有能典朕三禮?”僉曰:“\CJKunderline{伯夷}。”\footnote{三禮,天地人之禮。\CJKunderline{伯夷},臣名,姜姓。}

{\noindent\zhuan\zihao{6}\fzbyks 傳“\CJKunderline{朱虎}”至“之中”。正義曰:知“垂、益所讓四人皆在元凱之中”者,以文十八年\CJKunderwave{左傳}八元之內有伯虎、仲熊,即此“\CJKunderline{朱虎}、\CJKunderline{熊羆}”是也。虎、熊在元凱之內,明\CJKunderline{殳斨}、\CJKunderline{伯與}亦在其內,但不知彼誰當之耳。益是\CJKunderline{皋陶}之子,\CJKunderline{皋陶}即庭堅也。益在八凱之內,垂則不可知也。傳不在\CJKunderline{伯夷}、夔龍之下為此言者,以\CJKunderline{伯夷}姜姓,不在元凱之內,夔龍亦不可知,惟言此四人耳。傳雖言\CJKunderline{殳斨}、\CJKunderline{伯與},亦難知也。 \par}

{\noindent\zhuan\zihao{6}\fzbyks 傳“三禮”至“姜姓”。正義曰:此時“秩宗”,即\CJKunderwave{周禮}之宗伯也,其職雲“掌天神、人鬼、地祇之禮”,雖三者併為吉禮,要言三禮者是天地人之事,故知三禮是“天地人之禮”。上文舜之巡守言“修五禮”,此雲“典朕三禮”,各有其事,則五禮皆據其所施于三處。五禮所施于天地人耳。言“三”足以包五,故舉“三”以言之。\CJKunderwave{鄭語}云:“姜,\CJKunderline{伯夷}之後也。\CJKunderline{伯夷}能禮于神以佐堯。”是\CJKunderline{伯夷}為姜姓也。此經不言“疇”者,訪其有能,是問誰可知,上文已具,此略之也。 \par}

帝曰:“俞,諮!\CJKunderline{伯},汝作秩宗。\footnote{秩序宗尊也,主郊廟之官。}夙夜惟寅,直哉惟清。”\footnote{夙,早也。言早夜敬思其職,典禮施政教,使正直而清明”寅如字,徐音夷。}

{\noindent\zhuan\zihao{6}\fzbyks 傳“秩序”至“之官”。正義曰:\CJKunderwave{堯典}傳已訓“秩”為序,此複訓者,此為官名,須辨官名之義,故詳之也。“宗”之為尊,常訓也。主郊廟之官,掌序鬼神尊卑,故以“秩宗”為名。“郊”謂祭天南郊,祭地北郊;“廟”謂祭先祖,即\CJKunderwave{周禮}所謂“天神、人鬼、地祇之禮”是也。 \par}

{\noindent\zhuan\zihao{6}\fzbyks 傳“夙,早也。言早夜敬思其職,典禮施政教,使正直而清明”。正義曰:“夙,早”,\CJKunderwave{釋詁}文。“早夜敬服其職”,謂侵早已起,深夜乃臥,謹敬其職事也。典禮之官施行教化,使正直而清明。正直,不枉曲也。清明,不闇昧也。 \par}

\CJKunderline{伯}拜稽首,讓于\CJKunderline{夔}、\CJKunderline{龍}。\footnote{夔、龍,二臣名。夔音求龜反。}帝曰:“俞,往,欽哉!”\footnote{然其賢,不許讓。}\textcolor{red}{帝曰}:“\CJKunderline{夔},命汝典樂,教胄子,\footnote{胄,長也,謂元子以下至卿大夫子弟。以歌詩蹈之舞之,教長國子中、和、祇、庸、孝、友。胄,直又反,王云:“胄子,國子也。”馬云:“胄,長也,教長天下之子弟。”}直而溫,寬而栗,\footnote{教之正直而溫和,寬弘而能莊栗。莊栗,戰栗也。}剛而無虐,簡而無傲。\footnote{剛失入虐,簡失入傲,教之以防其失。}

{\noindent\zhuan\zihao{6}\fzbyks 傳“胄長”至“孝友”。正義曰:\CJKunderwave{說文}云:“胄,胤也。”\CJKunderwave{釋詁}云:“胤,繼也。”繼父世者惟長子耳,故以“胄”為長也。“謂元子已下至卿大夫子弟”者,\CJKunderwave{王制}云:“樂正崇四術,立四教。王太子、王子、群后之太子、卿大夫元士之適子皆造焉。”是“下至卿大夫”也。不言“元士”,士卑,故略之。彼鄭注云:“王子,王之庶子也。”此傳兼言“弟”者,蓋指太子之弟耳。或孔意公卿大夫之弟亦教之,國子以適為主,故言“胄子”也。命典樂之官,使教胄子。下句又言詩歌之事,是令夔以歌詩蹈之舞之,教此適長國子也。\CJKunderwave{周禮·大司樂}云:“以樂德教國子中、和、祇、庸、孝、友。”鄭云:“中猶忠也。和,剛柔適也。祇,敬也。庸,有常也。善父母曰孝。善兄弟曰友。”是言樂官用樂教之,使成此六德也。\CJKunderwave{樂記}又云:“樂在宗廟之中,君臣上下同聽之,則莫不和敬。在族黨鄉里之中,長幼同聽之,則莫不和順。在閨門之內,父子兄弟同聽之,則莫不和親。”是樂之感人,能成忠、和、祇、庸、孝、友之六德也。 \par}

{\noindent\zhuan\zihao{6}\fzbyks 傳“教之”至“莊栗”。正義曰:此“直而溫”與下三句皆使夔教胄子,令性行當然,故傳發首言“教之”也。正直者失于太嚴,故令“正直而溫和”;寬弘者失于緩慢,故令“寬弘而莊栗”;謂矜莊嚴栗。栗者,謹敬也。 \par}

{\noindent\zhuan\zihao{6}\fzbyks 傳“剛失”至“其失”。正義曰:剛彊之失入于苛虐,故令人剛而無虐。簡易之失入于傲慢,故令簡而無傲。剛、簡是其本性,教之使無虐、傲,是言教之以防其失也。由此而言之,上二句亦直、寬是其本性,直失于不溫,寬失于不栗,故教之使溫、栗也。直、寬、剛、簡即\CJKunderline{皋陶}所謀之九德也。九德而獨舉此四事者,人之大體,故特言之。 \par}

詩言志,歌永言,\footnote{謂詩言志以導之,歌詠其義以長其言。永徐音詠,又如字。}聲依永,律和聲。\footnote{聲謂五聲:宮、商、角、徵、羽。律謂六律、六呂,十二月之音氣。言當依聲律以和樂。}八音克諧,無相奪倫,神人以和。”\footnote{倫,理也。八音能諧,理不錯奪,則神人咸和。命夔使勉之。}

{\noindent\zhuan\zihao{6}\fzbyks 傳“謂詩”至“其言”。正義曰:作詩者自言己志,則詩是言志之書,習之可以生長志意,故教其詩言志以導胄子之志,使開悟也。作詩者直言不足以申意,故長歌之,教令歌詠其詩之義以長其言,謂聲長續之。定本經作“永”字,明訓“永”為長也。 \par}

{\noindent\zhuan\zihao{6}\fzbyks 傳“聲謂”至“和樂”。正義曰:\CJKunderwave{周禮·太師}云:“文之以五聲:宮、商、角、徵、羽。”言五聲之清濁有五品,分之為五聲也。又“太師掌六律、六呂以合陰陽之聲。陽聲黃鐘、太簇、姑洗、蕤賓、夷則、無射。陰聲大呂、應鐘、南呂、林鐘、仲呂、夾鍾”。是六律、六呂之名也。\CJKunderwave{漢書·律歷志}云:“律有十二,陽六為律,陰六為呂。”是陰律名同,亦名呂也。\CJKunderline{鄭玄}云:“律述氣也,同助陰宣氣,與之同也。”又云:“呂,旅也,言旅助陽宣氣也。”\CJKunderwave{志}又云:“律\CJKunderline{黃帝}之所作也,\CJKunderline{黃帝}使伶倫氏自大夏之西、崑崙之陰,取竹于嶰谷之中各生、其竅厚薄均者,斷兩節之間吹之,以為黃鐘之宮。制十二籥,以聽鳳皇之鳴,其雄聲為六,雌鳴亦六,以比黃鐘之宮,是為律之本。”言律之所作如此。聖人之作律也,既以出音,又以候氣,布十二律于十二月之位,氣至則律應,是六律、六呂述十二月之音氣也。“聲依永”者,謂五聲依附長言而為之,其聲未和,乃用此律呂調和其五聲,使應于節奏也。 \par}

{\noindent\zhuan\zihao{6}\fzbyks 傳“倫理”至“勉之”。正義曰:“倫”之為理,常訓也。八音能諧,相應和也。各自守分,不相奪道理,是言理不錯亂相奪也。如此則神人咸和矣。帝言此者,命夔使勉之也。\CJKunderwave{大司樂}云:“大合樂以致鬼神示,以和邦國,以諧萬民,以安賓客,以說遠人。”是神人和也。 \par}

\CJKunderline{夔}曰:“于!予擊石拊石,百獸\textcolor{red}{率舞}。”\footnote{石,磬也。磬,音之清者。拊亦擊也。舉清者和則其餘皆從矣。樂感百獸,使相率而舞,則神人和可知。于如字,或音烏而絕句者,非。拊音撫,徐音府。}

{\noindent\zhuan\zihao{6}\fzbyks 傳“石磬”至“可知”。正義曰:樂器惟磬以石為之,故云:“石,磬也。”八音之音石磬最清,故知磬是音之聲清者。磬必擊以鳴之,故云拊亦擊之。重其文者,擊有大小,“擊”是大擊,“拊”是小擊。音聲濁者粗,清者精,精則難和,舉清者和,則其餘皆從矣。\CJKunderwave{商頌}云:“依我磬聲。”是言磬聲清,諸音來依之。“百獸率舞”即\CJKunderwave{大司樂}雲“以作動物”、\CJKunderwave{益稷}雲“鳥獸蹌蹌”是也。人神易感,鳥獸難感,百獸相率而舞,則神人和可知也。夔言此者,以帝戒之雲“神人以和”,欲使勉力感神人也。乃答帝雲“百獸率舞”,則神人以和,言帝德及鳥獸也。 \par}

{\noindent\shu\zihao{5}\fzkt “帝曰夔”至“率舞”。正義曰:帝因\CJKunderline{伯夷}所讓,隨才而任用之。帝呼\CJKunderline{夔}曰:“我念命女典掌樂事,當以詩樂教訓世適長子,使此長子正直而溫和,寬弘而莊栗,剛毅而不苛虐,簡易而不傲慢。教之詩樂,所以然者,詩言人之志意,歌詠其義以長其言。樂聲依此長歌為節,律呂和此長歌為聲。八音皆能和諧,無令相奪道理,如此則神人以此和矣。”夔答舜曰:“嗚呼!我擊其石磬,拊其石磬,諸音莫不和諧,百獸相率而舞。”樂之所感如此,是人神既已和矣。 \par}

\textcolor{red}{帝曰}:“\CJKunderline{龍},朕\xpinyin*{堲}讒說\xpinyin*{殄}行,震驚朕師。\footnote{堲,疾。殄,絕。震,動也。言我疾讒說絕君子之行而動驚我眾,欲遏絕之。堲,徐在力反。讒,\CJKunderwave{切韻}仕咸反。說如字,注同,徐失銳反。殄,\CJKunderwave{切韻}徒典反。行,下孟反,注同。}命汝作納言,夙夜出納朕命,\textcolor{red}{惟允}。”\footnote{納言,喉舌之官。聽下言納于上,受上言宣于下,必以信。喉音侯。}

{\noindent\zhuan\zihao{6}\fzbyks 傳“堲疾”至“絕之”。正義曰:“堲”聲近疾,故為疾也。“殄,絕”、“震,動”皆\CJKunderwave{釋詁}文。讒人以善為惡,以惡為善,故言“我疾讒說絕君子之行”。眾人畏其讒口,故為讒也,“動驚我眾”,欲遏止之。 \par}

{\noindent\zhuan\zihao{6}\fzbyks 傳“納言”至“以信”。正義曰:\CJKunderwave{詩}美仲山甫為王之喉舌。喉舌者,宣出王命,如王咽喉口舌,故納言為“喉舌之官”也。此官主聽下言納于上,故以“納言”為名。亦主受上言宣于下,故言出朕命。“納言”不納于下,“朕命”有出無入,官名“納言”,雲“出納朕命”,互相見也。“必以信”者,不妄傳下言,不妄宣帝命,出納皆以信也。 \par}

{\noindent\shu\zihao{5}\fzkt “帝曰龍”至“惟允”。正義曰:帝呼龍曰:“龍,我憎疾人為讒佞之說,絕君子之行,而動驚我眾人,欲遏之。故命汝作納言之官。從早至夜出納我之教命,惟以誠信。”每事皆信則讒言自絕,命龍使勉之。 \par}

\textcolor{red}{帝曰}:“諮!汝二十有二人,\footnote{禹、垂、益、\CJKunderline{伯夷}、夔、龍六人新命有職,四岳、十二牧凡二十二人,特敕命之。}欽哉!惟時亮\textcolor{red}{天功}。”\footnote{各敬其職,惟是乃能信立天下之功。}

{\noindent\zhuan\zihao{6}\fzbyks 傳“\CJKunderline{禹}垂”至“命之”。正義曰:傳以此文總結上事,據上文“詢于四岳”,“諮,十有二牧”,及新命六官等,適滿二十二人,謂此也。其稷、契、\CJKunderline{皋陶}、\CJKunderline{殳斨}、\CJKunderline{伯與}、\CJKunderline{朱虎}、\CJKunderline{熊羆}七人仍舊,故不須敕命之。嶽、牧亦應是舊而敕命之者,嶽牧外內之官,常所諮詢,故亦敕之。\CJKunderline{鄭玄}云:“自‘諮,十有二牧’至‘帝曰龍’,皆月正元日格于文祖所敕命也。”案經“格于文祖”之後方始詢于四岳,諮十二州牧,未必一日之內即得行此諸事,傳既不說,或歷日命授,乃總敕之,未必即是元日之事也。鄭以為“二十二人數\CJKunderline{殳斨}、\CJKunderline{伯與}、\CJKunderline{朱虎}、\CJKunderline{熊羆},不數四岳”。彼四人者直被讓而已,不言居官,何故敕使敬之也?嶽、牧俱是帝所諮詢,何以敕牧不敕嶽也?必非經旨,故孔說不然。 \par}

{\noindent\shu\zihao{5}\fzkt “帝曰諮”至“天功”。正義曰:帝既命用眾官,乃總戒敕之曰:“諮嗟!汝新命六人,及四岳、十二牧凡二十有二人,汝各當敬其職事哉!惟是汝等敬事,則信實能立天下之功。天下之功,\CJKunderline{成王}在于汝,可得不敬之哉!” \par}

\textcolor{red}{三載}考績,三考黜陟,幽明\footnote{三年有成,故以考功。九歲則能否幽明有別,黜退其幽者,升進其明者。黜,醜律反。}庶績咸熙。分北\textcolor{red}{三苗}。\footnote{考績法明,眾功皆廣。三苗幽暗,君臣善否,分北流之,不令相從。善惡明。北如字,又音佩。令,力呈反。}

{\noindent\zhuan\zihao{6}\fzbyks 傳“三年”至“明者”。正義曰:三年一閏,天道成,人亦可以成功,故以三年考校其功之成否也。九年三考,則人之能否可知,幽明有別。“黜退其幽者”,或奪其官爵,或徙之遠方。“升進其明者”,或益其土地,或進其爵位也。 \par}

{\noindent\zhuan\zihao{6}\fzbyks 傳“考績”至“惡明”。正義曰:考績法明,人皆自勵,故得“眾功皆廣”也。“分北三苗”即是黜幽之事,故于“考績”之下言其流之。“分”謂別之。雲“北”者,言相背,必善惡不同。故知三苗幽暗,宜黜其君臣,乃有善否,分背流之,不令相從。俱徙之則善從惡,俱不徙則惡從善,言善惡不使相從,言舜之黜陟善惡明也。\CJKunderline{鄭玄}以為“流四凶者,卿為伯子,大夫為男,降其位耳,猶為國君”,故以“三苗為西裔諸侯,猶為惡,乃復分北流之”,謂分北西裔之三苗也。孔傳“竄三苗”為誅也,其身無復官爵,必非黜陟之限,其所“分北”,非彼“竄”者。\CJKunderline{王肅}云:“三苗之民有赦宥者,復不從化,不令相從,分北流之。”\CJKunderline{王肅}意彼赦宥者復繼為國君,至不復從化,故分北流之。\CJKunderline{禹}繼\CJKunderline{鯀}為崇伯,三苗未必絕後,傳意或如肅言。 \par}

{\noindent\shu\zihao{5}\fzkt “三載”至“三苗”。正義曰:自此以下史述舜事,非帝語也。言帝命群官之後,經三載乃考其功績,經三考則九載。“黜陟幽明”,明者升之,暗者退之。群官懼黜思升,各敬其事,故得“眾功皆廣”。前流四凶時,三苗之君竄之西裔,更紹其嗣,不滅其國。舜即政之後,三苗復不從化,是暗當黜之。其君臣有善有惡,舜復分北流其三苗。北,背也。善留惡去,使分背也。 \par}

\CJKunderline{舜}生三十,徵庸\footnote{言其始見試用。}三十,在位\footnote{歷試二年,攝位二十八年。}五十載,陟方乃死。\footnote{方,道也。舜即位五十年,升道南方巡守,死于蒼梧之野而葬焉。陟方,猶巡狩,天子外出巡視。三十徵庸,三十在位,服喪三年,其一在三十之數,為天子五十年,凡壽百一十二歲。}

{\noindent\zhuan\zihao{6}\fzbyks 傳“歷試”至“八年”。正義曰:上云:“乃言厎可績,三載”,則歷試當三年。雲“二年”者,其一即是徵用之年,已在上句三十之數,故惟有二年耳。受終居攝,尚在臣位,故歷試併為三十。“在位”謂在臣位也。 \par}

{\noindent\zhuan\zihao{6}\fzbyks 傳“方道”至“十二歲”。正義曰:\CJKunderwave{論語}云:“可謂仁之方也已。”孔注亦以“方”為道,常訓也。“舜即位五十年”,從格于文祖之後數之。“升道”謂乘道而行也。天子之行必是巡其所守之國,故通以“巡守”為名,未必以仲夏之月巡守南嶽也。\CJKunderwave{檀弓}雲“舜葬蒼梧之野”,是舜死蒼梧之野因而葬焉。孔以“月正元日”在“三載”、“遏密”之下,又\CJKunderwave{孟子}云,舜服堯三年喪畢,避堯之子,故“服喪三年”。三年之喪,二十五月而畢,其一年即在三十在位之數,惟有二年。是舜年六十二,為天子五十年,是舜“凡壽百一十二歲”也。\CJKunderwave{大禹謨}雲“帝曰‘朕宅帝位三十有三載’”,乃求禪\CJKunderline{禹}。\CJKunderwave{孟子}云:“舜薦\CJKunderline{禹}于天子,十有七年。”是在位五十年,其文明矣。\CJKunderline{鄭玄}讀此經云:“‘舜生三十’,謂生三十年也。‘登庸二十’,謂歷試二十年。‘在位五十載,陟方乃死’,謂攝位至死為五十年。舜年一百歲也。”\CJKunderwave{史記}云:“舜年三十堯舉用之,年五十攝行天子事,年五十八堯崩,年六十一而踐天子位,三十九年崩。”皆謬耳。 \par}

\textcolor{red}{帝釐}下土,方設居方,\footnote{言舜理四方諸侯,各設其官居其方。釐,力之反,馬云:“賜也,理也。”下土,絕句;一讀至“方”字絕句。}別生分類。\footnote{生,姓也。別其姓族,分其類,使相從。別,彼列反。分,方雲反,徐扶問反。}作\CJKunderwave{汩作}、\footnote{汩,治。作,興也。言其治民之功興,故為\CJKunderwave{汩作}之篇。亡。汩音骨。}\CJKunderwave{九共}九篇、\CJKunderwave{\textcolor{red}{槁飫}}。\footnote{槁,勞也。飫,賜也。凡十一篇,皆亡。共音恭,王己勇反,法也,馬同。槁,苦報反。飫,于據反。\CJKunderwave{槁飫}亦\CJKunderwave{書}篇名也。\CJKunderwave{汩作}等十一篇同此序。其文皆亡,而序與百篇之序同編,故存。今馬、鄭之徒百篇之序總為一卷,孔以各冠其篇首,而亡篇之序即隨其次篇居見存者之間。眾家經文並盡此,唯王注本下更有“\CJKunderwave{汩作}、\CJKunderwave{九共}故逸。故亦作古”。}

{\noindent\zhuan\zihao{6}\fzbyks 傳“言舜”至“其方”。正義曰:在\CJKunderwave{虞書},知“帝”是舜也。“下土”對天子之辭,故云“理四方諸侯,各為其官居其方”。不知若為設之。凡此三篇之序,亦既不見其經,暗射無以可中。\CJKunderline{孔氏}為傳,復順其文為其傳耳,是非不可知也。他皆仿此。 \par}

{\noindent\zhuan\zihao{6}\fzbyks 傳“汩治”至“篇亡”。正義曰:“汩”之為治,無正訓也。“作”是起義,故為興也。“言其治民之功興”,以意言之耳。 \par}

{\noindent\zhuan\zihao{6}\fzbyks 傳“槁,勞。飫,賜也”。正義曰:\CJKunderwave{左傳}言“槁師”者,以師枯槁,用酒食勞之,是“槁”得為勞也。襄二十六年\CJKunderwave{左傳}云:“將賞,為之加膳,加膳則飫賜。”是“飫”得為賜也。亦不知勞賜之何所謂也。 \par}

{\noindent\shu\zihao{5}\fzkt “帝釐”至“槁飫”。正義曰:此序也,孔以\CJKunderwave{書序}序所以為作者之意,宜相附近,故引之各冠其篇首。其經亡者,以序附于本篇次而為之傳,故此序在此也。\CJKunderline{帝舜}治理下土諸侯之事,為各于其方置設其官,居其所在之方而統治之。又為民別其姓族之生,分別異類,各使相從作\CJKunderwave{汩作}篇,又作\CJKunderwave{九共}九篇,又作\CJKunderwave{槁飫}之篇,凡十一篇,皆亡。 \par}

%%% Local Variables:
%%% mode: latex
%%% TeX-engine: xetex
%%% TeX-master: "../Main"
%%% End:
