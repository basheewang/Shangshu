%% -*- coding: utf-8 -*-
%% Time-stamp: <Chen Wang: 2024-04-02 11:42:41>

% {\noindent \zhu \zihao{5} \fzbyks } -> 注 (△ ○)
% {\noindent \shu \zihao{5} \fzkt } -> 疏

\chapter{卷十八}


\section{周官第二十二【偽】}


成王既黜殷命,滅淮夷,\footnote{黜殷在周公東征時,滅淮夷在成王即政後,事相因,故連言之。}還歸在豐,作\CJKunderwave{周官}。\footnote{成王雖作洛邑,猶還西周。}

周官\footnote{言周家設官分職用人之法。}


{\noindent\zhuan\zihao{6}\fzbyks 傳“黜殷”至“言之”。正義曰:據\CJKunderwave{金縢}之經、\CJKunderwave{大誥}之序,知“黜殷命”在周公攝政三年東征之時也。據\CJKunderwave{成王政}之序、\CJKunderwave{費誓}之經,知“滅淮夷”在成王即政之後也。淮夷於攝政之時與武庚同叛,成王既滅淮夷,天下始定。淮夷本因武庚而叛,黜殷命與滅淮夷其事相因,故雖則異年而連言之,以見天下既定,乃作\CJKunderwave{周官}故也。下經言“四徵弗庭”是黜滅之事也,“罔不承德”是安寧之狀也,序顧經文,故追言“黜殷命”,以接“滅淮夷”,見征伐乃安定之意也。 \par}

{\noindent\zhuan\zihao{6}\fzbyks 傳“成王”至“西周”。正義曰:以\CJKunderwave{洛誥}之文言“王在新邑”,今復雲“在豐”,故解之也。\CJKunderwave{史記·周本紀}云:“太史公曰:“學者皆稱周伐紂,居洛邑,綜其實不然。武王營之,成王使召公卜,居九鼎焉,而周復都豐、鎬。”是言成王雖作洛邑,猶還西周之事也。\CJKunderwave{多方}云:“王來自奄,至於宗周。”宗周即鎬京也,於彼不解,至此始為傳者,宗周雖是鎬京,文無“豐鎬”之字,故就此解之。武王既以遷鎬京,今王覆在豐者,豐、鎬相近,舊都不毀,豐有文王之廟,故事就豐宣之故也。 \par}

{\noindent\zhuan\zihao{6}\fzbyks 傳“言周”至“之法”。正義曰:\CJKunderwave{周禮}每官言人之員數及職所掌,立其定法,授與成王。成王即政之初,即有淮夷叛逆,未暇得以立官之意號令群臣。今既滅淮夷,天下清泰,故以周家設官分職用人之法以誥群臣,使知立官之大旨也。“設官分職”,\CJKunderwave{周禮}序官之文,言設置群官,分其職掌。經言立三公六卿,是“設官”也。各言所掌,是“分職”也。各舉其官之所掌,示以才堪乃得居之,是說“用人之法”。 \par}

{\noindent\shu\zihao{5}\fzkt “成王”至“周官”。正義曰:成王於周公攝政之時既黜殷命,及其即位之後滅淮夷,於是天下大定。自滅淮夷,還歸在豐,號令群臣,言周家設官分職用人之法。史敘其事,作\CJKunderwave{周官}。 \par}

惟周王撫萬邦,巡侯甸,\footnote{即政撫萬國,巡行天下侯服、甸服。}四徵弗庭,綏厥兆民。\footnote{四面征討諸侯之不直者,所以安其兆民。十億曰兆,言多。}六服群辟,罔不承德。歸於宗周,董正治官。\footnote{六服諸侯,奉承周德。言協服。還歸於豐,督正治理職司之百官。○闢,必亦反。治,直吏反,下至“冢宰”經注同。}


{\noindent\zhuan\zihao{6}\fzbyks 傳“即政”至“甸服”。正義曰:檢\CJKunderwave{成王政}之序與\CJKunderwave{費誓}之經,知成王即政之年,奄與淮夷又叛,叛即往伐,今始還歸。\CJKunderwave{多方}云:“五月丁亥,王來自奄,至於宗周。”與此滅淮夷而還歸在豐為一事也。年初始叛,五月即歸,其間未得巡守於四方也。而此言“撫萬國,巡行天下”,其實止得撫巡向淮夷之道所過諸侯爾,未是用四仲之月大巡守也。以撫諸侯巡守是天子之大事,因即大言之爾。周之法制無萬國也,惟伐淮夷非四徵也,言“萬國”、“四徵”亦是大言之爾。六服而惟言“侯甸”者,二服去圻最近,舉近以言之,言王巡省遍六服也。 \par}

{\noindent\zhuan\zihao{6}\fzbyks 傳“四面”至“言多”。正義曰:“四徵”,從京師而四面徵也。\CJKunderwave{釋詁}云:“庭,直也。綏,安也。”諸侯不直,謂叛逆王命,侵削下民,故“四面征討諸侯之不直者,所以安其兆民”。\CJKunderwave{楚語}云:“十日、百姓、千品、萬官、億醜、兆民。”每數相十,知“十億曰兆”。稱“兆”,言其多也。 \par}

{\noindent\zhuan\zihao{6}\fzbyks 傳“六服”至“百官”。正義曰:\CJKunderwave{周禮}“九服”,此惟言“六”者,夷、鎮、蕃三服在九州之外夷狄之地,王者之於夷狄,羈縻而已,不可同於華夏,故惟舉六服諸侯。奉承周德,言協服也。序雲“還歸在豐”,知宗周即豐也。周為天下所宗,王都所在皆得稱之,故豐、鎬與洛邑皆名“宗周”。\CJKunderwave{釋詁}云:“董、督,正也。”是“董”得為督,督正治理職司之百官。下戒敕是“董正”也。 \par}

{\noindent\shu\zihao{5}\fzkt “惟周”至“治官”。正義曰:惟周之王者,佈政教,撫安萬國,巡行天下侯服、甸服,四面征討諸侯之不直者,所以安其海內兆民。六服之內崑崙諸侯之君,無有不奉承周王之德者。自滅淮夷而歸於宗周豐邑,乃督正治理職司之百官。敘王發言之端也。 \par}

王曰:“若昔大猷,制治於未亂,保邦於未危。”\footnote{言當順古大道,制治安國,必於未亂未危之前,思患預防之。}

{\noindent\shu\zihao{5}\fzkt “王曰”至“未危”。正義曰:“治”謂政教,“邦”謂國家。治有失則亂,家不安則危。恐其亂則預為之制,慮其危則謀之使安,制其治於未亂之始,安其國於未危之前。張官設府,使分職明察,任賢委能,令事務順理,如是則政治而國安矣。標此二句於前,以示立官之意。必於未亂未危之前為之者,思患而預防之。“思患而預防之”,\CJKunderwave{易·既濟卦}象辭也。 \par}

曰:“唐虞稽古,建官惟百。內有百揆四嶽,外有州牧侯伯。\footnote{道堯舜考古以建百官,內置百揆四嶽,象天之有五行,外置州牧十二及五國之長,上下相維,外內咸治。言有法。○長,丁丈反,下“官長”、“助長”並同。}庶政惟和,萬國咸寧。\footnote{官職有序,故眾政惟和,萬國皆安,所以為正治。}夏商官倍,亦克用乂。\footnote{禹湯建官二百,亦能用治。言不及唐虞之清要。}明王立政,不惟其官,惟其人。\footnote{言聖帝明王立政修教,不惟多其官,惟在得其人。}


{\noindent\zhuan\zihao{6}\fzbyks 傳“道堯”至“有法”。正義曰:百人無主,不散則亂,有父則有君也。君不獨治,必須輔佐,有君則有臣也。\CJKunderwave{易·序卦}云:“有父子然後有君臣。”則君臣之興,次父子之後,人民之始,則當有之,未知其所由來也。雖遠舉唐虞,復考古也。\CJKunderwave{說命}曰:“明王奉若天道,建邦設都。”則王者立官,皆象天為之,故“內置百揆四嶽,象天之有五行”也。五行佐天,群臣佐主,以此為象天爾。不必其數有五乃象五行,故以“百揆四嶽”為五行之象。\CJKunderwave{左傳}少昊立五鳩氏,顓頊已來立五行之官,其數亦有五,故置於五行矣。\CJKunderwave{舜典}雲“肇十有二州”,此說虞事,知“置州牧十二”也。“侯伯”謂諸侯之長,\CJKunderwave{益稷}篇\CJKunderline{禹}言治水時事雲“外薄四海,咸建五長”,知“侯伯”是五國之長也。成王說此事者,言堯舜所制,上下相維,內外咸治,言有法也。此言“建官惟百”,“夏商官倍”,則唐虞一百,夏商二百。\CJKunderwave{禮記·明堂位}雲“有虞氏官五十,夏后氏官百”者,\CJKunderwave{禮記}是後世之言,不與經典合也。 \par}

{\noindent\shu\zihao{5}\fzkt “曰唐虞”至“其人”。正義曰:既言須立官之意,乃追述前代之法。止而復言,故更加一“曰”。唐堯\CJKunderline{虞舜}考行古道,立官惟數止一百也。“內有百揆四嶽”者,“百揆”,揆度百事,為群官之首,立一人也;“四嶽”,內典四時之政,外主太嶽之事,立四人也。“外有州牧侯伯”,牧,一州之長;侯伯,五國之長,各監其所部之國。外內置官,各有所掌,眾政惟以協和,萬邦所以皆安也。夏禹商湯立官倍多於唐虞,雖不及唐虞之清簡,亦能用以為治。明王立其政教,不惟多其官,惟在得其人。言自古制法皆明開官司,求賢以處之也。 \par}

今予小子,祗勤於德,夙夜不逮。\footnote{今我小子敬勤於德,雖夙夜匪懈,不能及古人。言自有極。○逮音代,一音大計反。懈,隹賣反。}仰惟前代時若,訓迪厥官。\footnote{言仰惟先代之法是順,訓蹈其所建官而則之,不敢自同堯舜之官,準擬夏殷而蹈之。}立太師、太傅、太保,茲惟三公。論道經邦,燮理陰陽。\footnote{師,天子所師法;傅,傅相天子;保,保安天子於德義者,此惟三公之任。佐王論道,以經緯國事,和理陰陽。言有德乃堪之。○燮,素協反。相,息亮反。}官不必備,惟其人。\footnote{三公之官不必備員,惟其人有德乃處之。○處,昌呂反。}少師、少傅、少保,曰三孤。\footnote{此三官名曰三孤。孤,特也。言卑於公,尊於卿,特置此三者。○少,詩照反,下同。}貳公弘化,寅亮天地,弼予一人。\footnote{副貳三公,弘大道化,敬信天地之教,以輔我一人之治。}冢宰掌邦治,統百官,均四海。\footnote{\CJKunderwave{天官}卿稱太宰,主國政治,統理百官,均平四海之內邦國。言任大。}司徒掌邦教,敷五典,擾兆民。\footnote{\CJKunderwave{地官}卿,司徒主國教化,布五常之教,以安和天下眾民,使小大皆協睦。○擾,而小反,徐音饒。}


{\noindent\zhuan\zihao{6}\fzbyks 傳“師天”至“堪之”。正義曰:三公俱是教道天子,輔相天子,緣其事而為之名。三公皆當運致天子,使歸於德義。傳於“保”下言“保安天子於德義”,總上三者,言皆然也。\CJKunderwave{禮記·文王世子}云:“師也”者,教之以事而喻諸德者也。保也者,慎其身以輔翼之而歸諸道者也。”道、德別掌者,內得於心,出行於道,道德不甚相遠,因其並釋“師”、“保”,故分配之爾。於公雲“燮理陰陽”,於孤雲“寅亮天地”,“和理”、“敬信”義亦同爾。以孤副貳三公,故其事所掌不異。 \par}

{\noindent\zhuan\zihao{6}\fzbyks 傳“天官”至“任大”。正義曰:此經言六卿所掌之事,撮引\CJKunderwave{周禮}為之總目,或據\CJKunderwave{禮}文,或取\CJKunderwave{禮}意,雖言有小異,義皆不殊。\CJKunderwave{周禮}云:“乃立天官冢宰,使帥其屬而掌邦治。治官之屬,太宰卿一人。”馬融云:“冢,大也。宰,治也。大治者,兼萬事之名也。”\CJKunderline{鄭玄}云:“變冢言大,進退異名也。百官總焉則謂之冢,列職於王則稱大。冢者,大之上也。山頂曰冢。”是解“冢”、“大”異名之意。\CJKunderwave{太宰職}云:“三曰禮典,以統百官。”馬融云:“統,本也。百官是宗伯之事也。”此“統百官”在“冢宰”之下,當以冢尊,故命統治百官為冢宰之事,治官、禮官俱得統之也。\CJKunderwave{禮}雲“以佐王均邦國”,此言“均四海”,故傳辨之“均平四海之內邦國”,與孔意不異。 \par}

{\noindent\zhuan\zihao{6}\fzbyks 傳“地官”至“協睦”。正義曰:\CJKunderwave{周禮}云:“乃立地官司徒,使帥其屬而掌邦教,以佐王安擾邦國。”\CJKunderwave{太宰職}云:“二曰教典,以擾萬民。”\CJKunderline{鄭玄}云:“擾亦安也,言饒衍之。”傳亦以“擾”為安。\CJKunderwave{五典}即五教也,布五常之教,以安和天下之人民,使小大協睦也。\CJKunderwave{舜典}云:“契為司徒,敬敷五教。”\CJKunderwave{周禮}:“司徒掌十有二教。一曰以祀禮教敬,則民不苟。二曰以陽禮教讓,則民不爭。三曰以陰禮教親,則民不怨。四曰以樂禮教和,則民不乖。五曰以儀辨等,則民不越。六曰以俗教安,則民不偷。七曰以刑教中,則民不暴。八曰以誓教恤,則民不怠。九曰以度教節,則民知足。十曰以世事教能,則民不失職。十有一曰以賢制爵,則民慎德。十有二曰以庸制祿,則民興功。”\CJKunderline{鄭玄}云:“有虞氏五而周十有二焉,然則十有二細分五教。為之五教,可以常行,謂之五典。五典謂父義、母慈、兄友、弟恭、子孝也。” \par}

宗伯掌邦禮,治神人,和上下。\footnote{\CJKunderwave{春官}卿,宗廟官長,主國禮,治天地神祗人鬼之事,及國之吉、兇、賓、軍、嘉五禮,以和上下尊卑等列。}司馬掌邦政,統六師,平邦國。\footnote{\CJKunderwave{夏官}卿,主戎馬之事,掌國征伐,統正六軍,平治王邦四方國之亂者。}司寇掌邦禁,詰奸慝,刑暴亂。\footnote{\CJKunderwave{秋官}卿,主寇賊法禁,治奸惡,刑強暴作亂者。夏司馬討惡助長物,秋司寇刑奸順時殺。}司空掌邦土,居四民,時地利。\footnote{\CJKunderwave{冬官}卿,主國空土以居民,士農工商四人。使順天時,分地利,授之土。能吐生百穀,故曰土。}六卿分職,各率其屬,以倡九牧,阜成兆民。\footnote{六卿各率其屬官大夫士,治其所分之職,以倡道九州牧伯為政,大成兆民之性命,皆能其官,則政治。○倡,尺亮反,下同。阜音負。治,直吏反。}

{\noindent\zhuan\zihao{6}\fzbyks 傳“春官”至“等列”。正義曰:\CJKunderwave{周禮}云:“乃立春官宗伯,使帥其屬而掌邦禮,以佐王和邦國宗廟也。”“伯”,長也。宗廟官之長,故名其官為“宗伯”。其職云:“掌建邦之天神、人鬼、地祗之禮。”又主吉、兇、賓、軍、嘉之五禮。吉禮之別十有二,凶禮之別有五,賓禮之別有八,軍禮之別有五,嘉禮之別有六,總有三十六禮,皆在宗伯職掌之文,文煩不可具載。\CJKunderwave{太宰職}云:“三曰禮典,以和邦國,以諧萬民。”其職又有“以玉作六瑞,以等邦國。以禽作六贄,以等諸臣”,是“以和上下尊卑等列”也。 \par}

{\noindent\zhuan\zihao{6}\fzbyks 傳“夏官”至“亂者”。正義曰:\CJKunderwave{周禮}云:“乃立夏官司馬,使帥其屬而掌邦政,以佐王平邦國。”其職主戎馬之事,有掌征伐,統正六軍,平治王邦四方國之亂者。天子六軍,軍師之通名也。案其職“掌九伐之法,馮弱犯寡則眚之,賊賢害民則伐之,暴內陵外則壇之,野荒民散則削之,負固不服則侵之,賊殺其親則正之,放弒其君則殘之,犯令陵政則杜之,外內亂、鳥獸行則滅之”。 \par}

{\noindent\zhuan\zihao{6}\fzbyks 傳“秋官”至“時殺”。正義曰:\CJKunderwave{周禮}云:“乃立秋官司寇,使帥其屬而掌邦禁,以佐王刑邦國。”其職云:“刑邦國,詰四方。”馬融云:“詰猶窮也,窮四方之奸也。”孔以“詰”為治,是主寇賊法禁,治奸慝之人,刑殺其強暴作亂者。夏官主征伐,秋官主刑殺,征伐亦殺人而官屬異時者,夏司馬討惡助夏時之長物,秋司寇刑奸順秋時之殺物也。\CJKunderwave{周禮}雲“掌邦刑”,此雲“掌邦禁”者,避下“刑暴亂”之文,故云“掌邦禁”。 \par}

{\noindent\zhuan\zihao{6}\fzbyks 傳“冬官”至“曰土”。正義曰:\CJKunderwave{周禮·冬官}亡。\CJKunderwave{小宰職}云:“六曰冬官,掌邦事。“又云:“六曰事職,以富國,以養萬民。”馬融云:“事職掌百工、器用、耒耜、弓車之屬。”與此主土居民全不相當。\CJKunderwave{冬官}既亡,不知其本。\CJKunderwave{禮記·王制}記司空之事云:“量地以制邑,度地以居民。”足明\CJKunderwave{冬官}本有主土居民之事也。\CJKunderwave{齊語}云:“管仲製法,令士農工商四民不雜。”即此“居民使順天時,分地利,授之土”也。土則地利為之名,以其吐生百穀,故曰“土”也。\CJKunderwave{周禮}雲“事”,此雲“土”者,為下有“居四民”,故云“土”以居民,為急故也。 \par}

{\noindent\shu\zihao{5}\fzkt “今予”至“厥官”。正義曰:王言:“今我小子,敬勤於德,雖早夜不懈怠,猶不能及於唐虞。仰惟先代夏商之法是順,順蹈其前代建官而法則之。”言不敢同堯舜之官,準擬行夏殷之官爾。“若”與“訓”俱訓為順也。 \par}

六年,五服一朝。\footnote{五服,侯、甸、男、採、衛。六年一朝會京師。}又六年,王乃時巡,考制度於四嶽。\footnote{周制十二年一巡守,春東、夏南、秋西、冬北,故曰時巡。考正制度禮法於四嶽之下,如虞帝巡守然。}諸侯各朝於方岳,大明黜陟。”\footnote{覲四方諸侯,各朝於方岳之下,大明考績黜陟之法。}


{\noindent\zhuan\zihao{6}\fzbyks 傳“周制”至“守然”。正義曰:\CJKunderwave{周禮·大行人}雲“十有二歲王巡守殷國”,是“周制十二年一巡守”也。如\CJKunderwave{舜典}所云春東、夏南、秋西、冬北以四時巡行,故曰“時巡”。“考正制度禮法於四嶽之下,如虞帝巡守然”,據\CJKunderwave{舜典},“同律度量衡”已下皆是也。 \par}

{\noindent\shu\zihao{5}\fzkt “六年”至“黜陟”。正義曰:此篇說六卿職掌,皆與\CJKunderwave{周禮}符同,則“六年,五服一朝”亦應是\CJKunderwave{周禮}之法,而\CJKunderwave{周禮}無此法也。\CJKunderwave{周禮·大行人}云:“侯服歲一見,其貢祀物。甸服二歲一見,其貢嬪物。男服三歲一見,其貢器物。採服四歲一見,其貢服物。衛服五歲一見,其貢材物。要服六歲一見,其貢貨物。”先儒說\CJKunderwave{周禮}者,皆雲“見”謂來朝也。必如所言,則周之諸侯各以服數來朝,無六年一朝之事。昭十三年\CJKunderwave{左傳}叔向云:“明王之制,使諸侯歲聘以志業,間朝以講禮,再朝而會以示威,再會而盟以顯昭明。自古已來,未之或失也。存亡之道,恆由是興。”說\CJKunderwave{左傳}者以為三年一朝、六年一會、十二年而盟,事與\CJKunderwave{周禮}不同。謂之前代明王之法,先儒未嘗措意,不知異之所由。計彼六年一會,與此“六年,五服一朝”事相當也。再會而盟,與此十二年“王乃時巡”,諸侯各朝於方岳亦相當也。叔向盛陳此法,以懼齊人使盟,若周無此,禮叔向妄說,齊人當以辭拒之,何所畏懼而敬以從命乎?且雲“自古以來,未之或失”,則當時猶尚行之,不得為前代之法,脅當時之人明矣。明周有此法,\CJKunderwave{禮}文不具爾。\CJKunderwave{大行人}所云,見者皆言貢物,或可因貢而見,何必見者皆是君自朝乎?遣使貢物亦應可矣。\CJKunderwave{大宗伯}云:“時見曰會,殷見曰同。”“時見”、“殷見”不雲年限,“時見曰會”何必不是“再朝而會”乎?“殷見曰同”何必不是“再會而盟”乎?周公制禮若無此法,豈成王謬言,叔向妄說也?計六年大集,應六服俱來,而此文惟言“五服”。孔以五服為侯、甸、男、採、衛,蓋以要服路遠,外逼四夷,不必常能及期,故寬言之而不數也。 \par}

王曰:“嗚呼!凡我有官君子,欽乃攸司,慎乃出令,令出惟行,弗惟反。\footnote{有官君子,大夫以上。嘆而戒之,使敬汝所司,慎汝出令,從政之本。令出必惟行之,不惟反改。若二三其令,亂之道。}以公滅私,民其允懷。\footnote{從政以公平滅私情,則民其信歸之。}學古入官,議事以制,政乃不迷。\footnote{言當先學古訓,然後入官治政。凡制事必以古義議度終始,政乃不迷錯。○度,待洛反。}其爾典常作之師,無以利口亂厥官。\footnote{其汝為政當以舊典常故事為師法,無以利口辯佞亂其官。}


{\noindent\zhuan\zihao{6}\fzbyks 傳“有官”至“之道”。正義曰:教之出令,使之號令在下,則是尊官,故知“有官君子”是“大夫已上”也,下雲“三事暨大夫”是也。安危在於出令,故慎汝出令是從政之本也。令既出口,必須行之,令而不行,是去而更反,故謂之“反”也。“不惟反者”,令其必行之,勿使反也。若前令不行而倒反,別出後令以改前令,二三其政,則在下不知所從,是亂之道也。 \par}

{\noindent\zhuan\zihao{6}\fzbyks 傳“言當”至“迷錯”。正義曰:襄三十一年\CJKunderwave{左傳}子產云:“我聞學而後入政,未聞以政學者也。”言將欲入政,先學古之訓典,觀古之成敗,擇善而從之,然後可以入官治政矣。凡欲制斷當今之事,必以古之義理議論量度其終始,合於古義,然後行之。則其為之政教,乃不迷錯也。 \par}

{\noindent\shu\zihao{5}\fzkt “王曰”至“厥官”。正義曰:王言而嘆曰:“嗚呼!凡我有官君子。”謂大夫已上有職事者。“汝等皆敬汝所主之職事,慎汝所出之號令。令出於口,惟即行之,不惟反之而不用,是去而復反也。為政之法,以公平之心滅己之私慾,則見下民其信汝而歸汝矣。學古之典訓,然後入官治政。論議時事,必以古之制度,如此則政教乃不迷錯矣。其汝為政,當以舊典常故事作師法,無以利口辯佞亂其官”。教之以居官為政之法也。 \par}

蓄疑敗謀,怠忽荒政,不學牆面,蒞事惟煩。\footnote{積疑不決,必敗其謀。怠惰忽略,必亂其政。人而不學,其猶正牆面而立,臨政事必煩。○蓄,敕六反。蒞音利,又音類。}戒爾卿士,功崇惟志,業廣惟勤,惟克果斷,乃罔後艱。\footnote{此戒凡有官位,但言卿士,舉其掌事者。功高由志,業廣由勤,惟能果斷行事,乃無後難。言多疑必致患。○斷,丁亂反,下注同。}

{\noindent\shu\zihao{5}\fzkt “蓄疑”至“後艱”。正義曰:又戒群臣,使彊於割斷,勤於職事。蓄積疑惑,不能彊斷,則必敗其謀慮。怠惰忽略,不能恪勤,則荒廢政事。人而不學,如面向牆,無所睹見,以此臨事,則惟煩亂,不能治理。戒汝卿之有事者,功之高者惟志意強正,業之大者惟勤力在公,惟能果敢決斷,乃無有後日艱難。言多疑必將致後患矣,申說“蓄疑敗謀”也。 \par}

位不期驕,祿不期侈。\footnote{貴不與驕期而驕自至,富不與侈期而侈自來,驕侈以行己,所以速亡。}恭儉惟德,無載爾偽。\footnote{言當恭儉,惟以立德,無行奸偽。}作德,心逸日休。作偽,心勞日拙。\footnote{為德,直道而行,於心逸豫而名日美。為偽,飾巧百端,於心勞苦而事日拙,不可為。}居寵思危,罔不惟畏,弗畏入畏。\footnote{言雖居貴寵,當思危,惟無所不畏。若乃不畏,則入可畏之刑。}推賢讓能,庶官乃和,不和政厖。\footnote{賢能相讓,俊乂在官,所以和諧。厖,亂也。○厖,武江反。}舉能其官,惟爾之能。稱匪其人,惟爾不任。”\footnote{所舉能修其官,惟亦汝之功能。舉非其人,亦惟汝之不勝其任。○勝音升。}王曰:“嗚呼!三事暨大夫,敬爾有官,亂爾有政,\footnote{嘆而敕之,公卿已下,各敬居汝所有之官,治汝所有之職。}以佑乃闢。永康兆民,萬邦惟無斁。”\footnote{言當敬治官政,以助汝君,長安天下兆民,則天下萬國惟乃無厭我周德。○斁音亦。長,直良反。厭,於豔反。}

{\noindent\shu\zihao{5}\fzkt 傳“為德”至“可為”。正義曰:為德者自得於己,直道而行,無所經營,於心逸豫,功成則譽顯,而名益美也。為偽者行違其方,枉道求進,思念欺巧,於心勞苦,詐窮則道屈,而事日益拙也。以此故,偽不可為。申說“無載爾偽”也。 \par}

成王既伐東夷,肅慎來賀。\footnote{海東諸夷駒麗、扶余、馯貌之屬,武王克商,皆通道焉。成王即政而叛,王伐而服之,故肅慎氏來賀。○肅慎,馬本作息慎,云:“北夷也。”駒,俱付反,又如字。麗,力支反。馯,戶旦反,\CJKunderwave{地理志}音寒。貊,孟白反,\CJKunderwave{說文}作“貉,北方豸種。\CJKunderline{孔子}曰,貉之言貊。貊,惡也。”}王俾榮伯作\CJKunderwave{賄肅慎之命}。\footnote{榮,國名。同姓諸侯,為卿大夫。王使之為命書,以幣賄賜肅慎之夷。亡。○俾,必爾反,馬本作辯。}


{\noindent\zhuan\zihao{6}\fzbyks 傳“海東”至“來賀”。正義曰:成王伐淮夷,滅徐奄,指言其國之名。此傳言“東夷”,非徒淮水之上夷也,故以為“海東諸夷駒麗、扶余、馯貊之屬”,此皆於\CJKunderline{孔君}之時有此名也。\CJKunderwave{周禮·職方氏}四夷之名、八蠻、九貉,\CJKunderline{鄭玄}雲“北方曰貉”,又云“東北夷也”。\CJKunderwave{漢書}有高駒麗、扶余、韓,無此馯,馯即彼韓也,音同而字異爾。\CJKunderwave{多方}云:“王來自奄。”奄在後滅,言滅奄即來,必非滅奄之後更伐東夷。夷在海東路遠,又不得先伐遠夷,後來滅奄,此雲“成王既伐東夷”,不知何時伐之。\CJKunderwave{魯語}云:“武王克商,遂通道於九夷、八蠻,於是肅慎氏來賀,貢楛矢。”則武王之時,東夷服也。成王即政,奄與淮夷近者尚叛,明知遠夷亦叛。蓋成王親伐淮夷而滅之,又使偏師伐東夷而服之。君統臣功,故言王伐,不是成王親自伐也。肅慎之於中國,又遠於所伐諸夷,見諸夷既服,故懼而來賀也。 \par}

{\noindent\zhuan\zihao{6}\fzbyks 傳“榮國”至“夷亡”。正義曰:\CJKunderwave{晉語}云:“文王諏於蔡、原,訪於辛、尹,重之以周、召、畢、榮。”於文王之時,名次畢公之下,則是大臣也。未知此時榮伯是彼榮公以否,或是其子孫也。“同姓諸侯”,相傳為然,注\CJKunderwave{國語}者亦云榮、周同姓。不知時為何官,故並雲“卿大夫”。王使榮伯,明使中所作。史錄其篇,名為\CJKunderwave{賄肅慎之命},明是王使之為命書,以幣賜肅慎氏之夷也。 \par}

{\noindent\shu\zihao{5}\fzkt “成王”至“之命”。正義曰:成王即政之初,東夷背叛。成王既伐而服之,東北遠夷其國有名肅慎氏者,以王戰勝,遠來朝賀。王賜以財賄,使榮國之伯為策書,以命肅慎之夷,嘉其慶賀,慰其勞苦之意。史敘其事,作\CJKunderwave{賄肅慎之命}名篇也。 \par}

周公在豐,\footnote{致政老歸。}將沒,欲葬成周。\footnote{己所營作,示終始念之。}公薨,成王葬於畢,\footnote{不敢臣周公,故使近文武之墓。○近,附近之近。}告周公,作\CJKunderwave{亳姑}。\footnote{周公徙奄君於亳姑,因告柩以葬畢之義,並及奄君已定亳姑,言所遷之功成。亡。○柩,其久反。}


{\noindent\zhuan\zihao{6}\fzbyks 傳“致政老歸”。正義曰:周公既還政成王,成王又留為太師,今言“周公在豐”,則是去離王朝,又致太師之政,告老歸於豐,如伊尹之告歸也。成王封伯禽於魯,以為周公後。公老不歸魯而在豐者,文十三年\CJKunderwave{公羊傳}云:“周公曷為不之魯?欲天下之一乎周也。”何休云:“周公聖人,德至重,功至大,東征則西國怨,西征則東國怨。嫌之魯,恐天下回心趣向之。故封伯禽,命使遙供養,死則奔喪為主,所以一天下之心於周室。”是言周公不歸魯之意也。歸豐者,蓋以先王之都,欲近其宗廟故也。 \par}

{\noindent\zhuan\zihao{6}\fzbyks 傳“周公”至“成亡”。正義曰:序說葬周公之事,其篇乃名\CJKunderwave{亳姑},篇名與序不相允會。其篇既亡,不知所道,故傳原其意而為之說。上篇將遷亳姑,序言“成王既踐奄,將遷其君於亳姑”者,是周公之意。今告周公之柩以葬畢之義,乃用“亳姑”為篇名,必是告葬之時,並言及奄君已定於亳姑,言周公所遷之功成,故以名篇也。 \par}

{\noindent\shu\zihao{5}\fzkt “周公”至“亳姑”。正義曰:周公既致政於王,歸在豐邑。將沒,遺言欲得葬於成周。以成周是己所營,示己終始念之,故欲葬焉。及公薨,成王葬於畢,以文武之墓在畢,示己不敢臣周公,使近文武之墓。王以葬畢之義告周公之柩,又周公徙奄君於亳姑,因言亳姑功成。史敘其事,作\CJKunderwave{亳姑}之篇。案\CJKunderwave{帝王世紀}云:“文武葬於畢。”畢在杜南,\CJKunderwave{晉書·地道記}亦云畢在杜南,與畢陌別,俱在長安西北。 \par}

\section{君陳第二十三【偽】}


周公既沒,命君陳分正東郊成周,\footnote{成王重周公所營,故命君陳分居,正東郊成周之邑里官司。}作\CJKunderwave{君陳}。\footnote{作書命之。}

君陳\footnote{臣名也,因以名篇。○鄭注\CJKunderwave{禮記}云:“周公之子。”}

王若曰:“君陳,惟爾令德孝恭。\footnote{言其有令德,善事父母,行己以恭。}惟孝,友于兄弟,克施有政。\footnote{言善父母者必友于兄弟,能施有政令。}命汝尹茲東郊,敬哉!\footnote{正此東郊,監殷頑民,教訓之。○監,工銜反。}昔周公師保萬民,民懷其德。往慎乃司,茲率厥常。\footnote{言周公師安天下之民,民歸其德。今往承其業,當慎汝所主,此循其常法而教訓之。}懋昭周公之訓,惟民其乂。\footnote{勉明周公之教,惟民其治。○懋音茂。治,直吏反,下注“政治”同。}


{\noindent\zhuan\zihao{6}\fzbyks 傳“成王”至“官司”。正義曰:“成周”,周之下都。監成周者正是一邑宰爾,而特命君陳大其事者,成王重周公所營,猶恐殷民有不服之者,故命君陳分居,正東郊成周之邑里官司也。以\CJKunderwave{畢命}之序言“分居”,知此“分”亦為分居,分別殷民善惡所居,即\CJKunderwave{畢命}所云“旌別淑慝,表厥宅裡”是也。言“東郊者”,\CJKunderline{鄭玄}云:“天子之國五十里為近郊,今河南洛陽相去則然。”是言成周之邑為周之東郊也。 \par}

{\noindent\zhuan\zihao{6}\fzbyks 傳“臣名”至“名篇”。正義曰:孔直雲“臣名”,則非周公子也。\CJKunderline{鄭玄}注\CJKunderwave{中庸}雲“君陳,蓋周公子”者,以經雲“周公既沒,命君陳”,猶若蔡叔既沒,命蔡仲故也。孔未必然矣。 \par}

{\noindent\zhuan\zihao{6}\fzbyks 傳“言其”至“以恭”。正義曰:“令德”,在身之大名。“孝”是事親之稱,“恭”是身之所行,言其善事父母,行己以恭也。\CJKunderwave{釋訓}云:“善父母為孝,善兄弟為友。” \par}

{\noindent\zhuan\zihao{6}\fzbyks 傳“言善”至“政令”。正義曰:父母尊之極,兄弟親之甚,緣其施孝於極尊,乃能施友于甚親。言善事父母者必友于兄弟,推此親親之心,以至於疏遠,每事以仁恕行之,故能施有政令也。 \par}

{\noindent\shu\zihao{5}\fzkt “周公”至“君陳”。正義曰:周公遷殷頑民於成周,頑民既遷,周公親自監之。周公既沒,成王命其臣名君陳代周公監之,分別居處,正此東郊成周之邑,以策書命之。史錄其事,作策書,為\CJKunderwave{君陳}篇名。 \par}

“我聞曰:‘至治馨香,感於神明。黍稷非馨,明德惟馨。’\footnote{所聞之古聖賢之言,政治之至者,芬芳馨氣動於神明。所謂芬芳非黍稷之氣,乃明德之馨。勵之以德。}爾尚式時周公之猷訓,惟日孜孜,無敢逸豫。\footnote{汝庶幾用是周公之道教殷民,惟當日孜孜勤行之,無敢自寬暇逸豫。○孜音茲。}

{\noindent\shu\zihao{5}\fzkt “我聞”至“逸豫”。正義曰:我聞人之言曰:“有至美治之善者,乃有馨香之氣,感動於神明。所言馨香感神者,黍稷飲食之氣非馨香也,明德之所遠及乃惟為馨香爾。”勉勵君陳使為德也。欲必為明德,惟法周公,汝當庶幾用是周公之道。惟當每日孜孜勤法行之,無敢自寬暇逸豫。教使勤於事也。 \par}

凡人未見聖,若不克見。既見聖,亦不克由聖,\footnote{此言凡人有初無終,未見聖道如不能得見。已見聖道,亦不能用之,所以無成。}爾其戒哉!爾惟風,下民惟草。\footnote{汝戒,勿為凡人之行。民從上教而變,猶草應風而偃,不可不慎。}圖厥政,莫或不艱,有廢有興。出入自爾師虞,庶言同則繹。\footnote{謀其政,無有不先慮其難,有所廢,有所起。出納之事,當用汝總言度之。眾言同,則陳而布之。禁其專。○繹音亦。度,待洛反。}爾有嘉謀嘉猷,則入告爾後於內,爾乃順之於外,\footnote{汝有善謀善道,則入告汝君於內,汝乃順行之於外。}曰:‘斯謀斯猷,惟我後之德。’\footnote{此善謀此善道,惟我君之德。善則稱君,人臣之義。}嗚呼!臣人咸若時,惟良顯哉!”\footnote{嘆而美之曰:“臣於人者皆順此道,是惟良臣,則君顯明於世。”}王曰:“君陳,爾惟弘周公丕訓,無依勢作威,無倚法以削,\footnote{汝為政當闡大周公之大訓,無乘勢位作威人上,無倚法制以行刻削之政。}寬而有制,從容以和。\footnote{寬不失制,動不失和,德教之治。○從,七容反。}殷民在闢,予曰闢,爾惟勿闢。予曰宥,爾惟勿宥。惟厥中。\footnote{殷人有罪在刑法者,我曰:“刑之。”汝勿刑。我曰:“赦宥。”汝勿宥。惟其當以中正平理斷之。○闢,扶亦反,下同。中如字,或丁仲反。斷,丁亂反。}有弗若於汝政,弗化於汝訓,闢以止闢,乃闢。\footnote{有不順於汝政,不變於汝教,刑之而懲止犯刑者,乃刑之。}狃於奸宄,敗常亂俗,三細不宥。\footnote{習於奸宄兇惡,毀敗五常之道,以亂風俗之教,罪雖小,三犯不赦,所以絕惡源。○狃,女九反。}


{\noindent\zhuan\zihao{6}\fzbyks 傳“汝為”至“之政”。正義曰:君陳之智,必不及周公,而令闡大周公訓者,遵行其法,使廣被於民,即是闡揚而大之。非遣君陳為法,使大於周公法也。凡在人上,位貴於人,勢足可畏者,多乘是形勢以作威刑于人,倚附公法以行刻削之政,故禁之也。 \par}

{\noindent\zhuan\zihao{6}\fzbyks 傳“寬不”至“之治”。正義曰:“寬不失制”,則經“寬而有制”。“動不失和”,則經“從容以和”。言“動”,謂“從容”也。 \par}

{\noindent\zhuan\zihao{6}\fzbyks 傳“習於”至“惡源”。正義曰:\CJKunderwave{釋言}云:“狃,復也。”孫炎曰:“狃忕,前復為也。”古言“狃忕”是貫習之義,故以“習”解“狃”。“習於奸宄兇惡”,言為之不知止也。“敗常亂俗”,有大有小,罪雖小者,三犯不赦,恐其滋大,所以絕惡源也。此謂所犯小事,言“三”者,再猶可赦爾。 \par}

{\noindent\shu\zihao{5}\fzkt “王曰”至“不宥”。正義曰:王呼之曰:“君陳,汝今為政,當弘大周公之大訓。周公既有大訓,汝當遵而行之,使其法更寬大。汝奉周公之訓,無得依恃形勢以作威於人,無得倚附法制以行刻削百姓。必當寬容而有法制,使疏而不漏。從容以和協於物,莫為褊急。此成周殷民,有犯事在於刑法未斷決者,我告汝曰:‘刑罰之。’汝惟勿得刑罰之。我告汝曰:‘赦宥之。’汝惟勿得赦宥之。惟其以中正平法斷決之,不得從上意也。其有不順於汝之政令,不化於汝之訓教,其罪既大,當行刑中。刑罰一人可以止息後犯者,故云犯刑者乃刑之。如其罪或輕細,罰不當理,雖刑勿息,故不可輒刑。若有人習於奸宄兇惡,敗五常之道,亂風俗之教,三犯其事者,事雖細小,勿得宥之。以其知而故犯,當殺之以絕惡源也。” \par}

爾無忿疾於頑,無求備於一夫。\footnote{人有頑嚚不喻,汝當訓之,無忿怒疾之。使人當器之,無責備於一夫。}必有忍,其乃有濟。有容,德乃大。\footnote{為人君長,必有所含忍,其乃有所成。有所包容,德乃為大。欲其忍恥藏垢。○長,誅丈反。垢,工口反。}簡厥修,亦簡其或不修。\footnote{簡別其德行修者,亦別其有不修者,善以勸能,惡以沮否。○別,彼列反。沮,在汝反。否,方九反,又音鄙。}進厥良,以率其或不良。\footnote{進顯其賢良者,以率勉其有不良者,使為善。}

{\noindent\shu\zihao{5}\fzkt “爾無”至“不良”。正義曰:民者,冥也,當以漸教之。故戒君陳:“民有不知道者,汝無忿怒疾惡。頑嚚之民,當以漸教訓之。無求備於一人,當取其所能。在為人君,必有所含忍,其事乃有所成。有所寬容,其德乃能大。”欲其寬大不褊隘也。“汝之為政,須知民之善惡,簡別其德行修者,亦簡別其有不修德行者。進顯其賢良,以率勵其不良者”。欲令其化惡,使為善也。 \par}

惟民生厚,因物有遷。\footnote{言人自然之性敦厚,因所見所習之物有遷變之道,故必慎所以示之。}違上所命,從厥攸好。\footnote{人之於上,不從其令,從其所好,故人主不可不慎所好。○好,呼報反。}爾克敬典在德,時乃罔不變。允升於大猷。\footnote{汝治人能敬常在道德,是乃無不變化,其政教則信升於大道。}惟予一人膺受多福,\footnote{汝能升大道,則惟我一人亦當受其多福,無兇危。}其爾之休,終有辭於永世。”\footnote{非但我受多福而已,其汝之美名,亦終見稱誦於長世。言沒而不朽。長如字。朽,許久反。}

{\noindent\shu\zihao{5}\fzkt “惟民”至“永世”。正義曰:惟民初生,自然之性皆敦厚矣。因見所習之物,本性乃有遷變,為惡皆由習效使然。人之情性,好違上所命,命之不必從也,從其君所好。君之所好,民必從之,在上者不可不慎所好也。汝之治民能敬,當從終常在於道德教之。汝以道德教之,是民乃無不變化。民皆變從汝化,則信升於大道矣。汝能如此,惟我一人亦當受其多福,無兇危矣。其汝之美名,亦終有稱誦之美辭於長世矣。 \par}

\section{顧命第二十四(顧命上)}


成王將崩,命召公、畢公。\footnote{二公為二伯,中分天下而治之。}率諸侯相康王,作\CJKunderwave{顧命}。\footnote{臨終之命曰顧命。○相,息亮反。顧,工戶反,馬云:“成王將崩,顧命康王,命召公、畢公率諸侯輔相之。”}


{\noindent\zhuan\zihao{6}\fzbyks 傳“二公”至“治之”。正義曰:\CJKunderwave{禮記·曲禮下}文云,“九州之長曰牧”,“五官之長曰伯,是職方”。\CJKunderline{鄭玄}云:“職,主也。謂為三公者,是伯分主東西者也。”\CJKunderwave{周禮·大宗伯}云:“八命作牧,九命作伯。”鄭云,謂“上公有功德者,加命為二伯”。此\CJKunderwave{禮}文皆伯尊於牧,牧主一州,明伯是中分天下者也。\CJKunderwave{禮}言“職方”,是各主一方也。此二伯即以三公為之。隱五年\CJKunderwave{公羊傳}云:“諸公者何?天子三公。天子三公者何?天子之相也。天子之相何以三?自陝而東者周公主之,自陝而西者召公主之,一相處乎內。”是言三公為二伯也。\CJKunderwave{公羊傳}漢世之書,陝縣者漢之弘農郡所治,其地居二京之中,故以為二伯分掌之界,周公所分亦當然也。\CJKunderwave{公羊傳}所言周、召分主,謂成王即位之初,此時周公已薨,故畢公代之。\CJKunderwave{周官}篇三公之次太師、太傅、太保,太保最在下。此篇以召公為先者,三公命數尊卑同也,王就其中委任賢者,任之重者則在前耳。 \par}

{\noindent\zhuan\zihao{6}\fzbyks 傳“臨終”至“顧命”。正義曰:\CJKunderwave{說文}云:“顧,還視也。”\CJKunderline{鄭玄}云:“回首曰顧,顧是將去之意。”此言“臨終之命曰顧命”,言臨將死去,回顧而為語也。 \par}

{\noindent\shu\zihao{5}\fzkt “成王”至“顧命”。正義曰:成王病困將崩,召集群臣以言,命太保召公、太師畢公,使率領天下諸侯輔相康王。史敘其事。作\CJKunderwave{顧命}。 \par}

顧命\footnote{實命群臣,敘以要言。}惟四月哉生魄,王不懌。\footnote{成王崩年之四月,始生魄,月十六日,王有疾,故不悅懌。○懌音亦。馬本作“不釋”,云:“不釋,疾不解也。”}甲子,王乃洮頮水。相被冕服,憑玉幾。\footnote{王大發大命,臨群臣,必齋戒沐浴。今疾病,故但洮盥頮面。扶相者,被以冠冕,加朝服,憑玉幾以出命。○洮,他刀反,徐音逃,馬云:“洮,洮發也。”頮音悔,\CJKunderwave{說文}作沬,云:“古文作頮。”馬云:“頮,頮面也。”被,皮義反。徐,扶偽反,注同。憑,皮冰反,下同。\CJKunderwave{說文}作憑,云:“依幾也。”\CJKunderwave{字林}同,父冰反。齊,側皆反。盥音管,又音灌。朝,直遙反。}



{\noindent\zhuan\zihao{6}\fzbyks 傳“實命”至“要言”。正義曰:王之所命,實普命群臣,序以要約為言,直雲“命召公、畢公”。傳不於上“召公、畢公”之下而解,於“顧命”之下言之者,以上欲指明二公中分天下之事,非是總語,故“命”不得言之。“顧命”是總命群臣,非但召、畢而已,故於此解也。 \par}

{\noindent\zhuan\zihao{6}\fzbyks 傳“成王”至“悅謂”。正義曰:成王崩年,經典不載,\CJKunderwave{漢書·律曆志}云,成王即位“三十年四月庚戌朔,十五日甲子哉生魄”,即引此\CJKunderwave{顧命}之文。以為成王即位三十年而崩,此是劉歆說也。孔以甲子為十六日,則不得與歆同矣。\CJKunderline{鄭玄}云:“此成王二十八年。”傳惟言“成王崩年”,未知成王即位幾年崩也。\CJKunderwave{志}又云:“死魄,朔也。生魄,望也。”明死魄生從望為始,故始生魄為月十六日,即是望之日也。\CJKunderwave{釋詁}云:“懌,樂也。”有疾,故不悅懌。下雲“病日臻,既彌留”,則成王遇病已多日矣。於“哉生魄”下始言“王不懌”者,甲子是發命之日,為“洮頮”張本耳。 \par}

{\noindent\zhuan\zihao{6}\fzbyks 傳“王大發”至“出命”。正義曰:凡有敬事,皆當潔清。王將發大命,臨群臣,必齋戒沐浴。今以病疾之故,不能沐浴,故但洮頮而已。\CJKunderwave{禮}洗手謂之“盥”,洗面謂之“靧”。\CJKunderwave{內則}云,子事父母“面垢,燂潘請靧”。“頮”是洗面,知“洮”為盥手。言“水”謂洮盥俱用水。扶相王者,以冕服加王。\CJKunderline{鄭玄}云:“相者,正王服位之臣,謂太僕。”或當然也。“被以冠冕”,以冕服被王首也。“加朝服”,以服加王身也,謂以袞冕朝諸侯之服加王身也。鄭以為玄冕,知不然者,以顧命群臣,大發大命,以文武之業傳社稷之重,不應惟服玄冕而已。\CJKunderwave{覲禮}王服袞冕而有玉幾。此既“憑玉幾”,明服袞冕也。\CJKunderwave{周禮·司几筵}云,凡大朝覲,王位設黼扆,扆前南向設左右玉幾。是王見群臣當憑玉幾以出命。 \par}

乃同召太保奭、芮伯、彤伯、畢公、衛侯、毛公、\footnote{同召六卿,下至御治事。太、保、畢毛稱公,則三公矣。此先後六卿次第,冢宰第一,召公領之。司徒第二,芮伯為之。宗伯第三,彤伯為之。司馬第四,畢公領之。司寇第五,衛侯為之。司空第六,毛公領之。召、芮、彤、畢、衛、毛皆國名,入為天子公卿。○奭音釋。芮,如銳反。彤,徒冬反。}師氏、虎臣、百尹、御事。\footnote{師氏,大夫官。虎臣,虎賁氏。百尹,百官之長。及諸御治事者。}

{\noindent\zhuan\zihao{6}\fzbyks 傳“同召”至“公卿”。正義曰:下及御事,蒙此同召之,文故云同召六卿,下及御事也。以王病甚,故同時俱召之。“太保”是三公官名,畢、毛又亦稱“公”,知此三人是三公也。三人是三公,而與侯伯相次,知六者是六卿。衛侯為司寇而位第五,知此先後是六卿次第也。以三公尊,故特言“公”,其餘三卿舉其本爵,見其以國君入為卿也。天子三公皆以卿為之,不復別置其人。高官兼攝下司者,漢世以來謂之為“領”,故言“召公領之”,“毛公領之”。定四年\CJKunderwave{左傳}雲“康叔為司寇”,知此六人依\CJKunderwave{周禮}次第為六卿也。王肅云:“彤,姒姓之國,其餘五國姬姓。畢、毛,文王庶子。衛侯,康叔所封,武王母弟。”依\CJKunderwave{世本}、\CJKunderwave{史記}為說也。 \par}

{\noindent\zhuan\zihao{6}\fzbyks 傳“師氏”至“事者”。正義曰:\CJKunderwave{周禮}師氏,中大夫,掌以美詔王,居虎門之左,司王朝得失之事,帥其屬守王之門。重其所掌,故於“虎臣”並於“百尹”之上,特言之。“尹”訓正也,故“百尹”為百官之長。“諸御治事”謂諸掌事者,蓋大夫、士皆被召也。王肅云:“治事,蓋群士也。” \par}

{\noindent\shu\zihao{5}\fzkt “顧命”至“御事”。正義曰:發首至“百尹、御事”,敘王以病召臣,為發言之端。自“王曰”至“冒貢於非幾”是顧命之辭也。“茲既受命”至“立於側階”,言命后王崩,欲宣王命,布陳儀衛之事也。自“王麻冕”已下,敘康王受命之事。 \par}

王曰:“嗚呼!疾大漸,惟幾。\footnote{自嘆其疾大進篤,惟危殆。○幾音機,徐音畿,下同。}病日臻,既彌留,棒恐不獲誓言嗣,茲予審訓命汝。\footnote{病日至,言困甚。已久留,言無瘳。恐不得結信出言嗣續我志,以此故,我詳審教命汝。○瘳,敕留反。}昔君文王、武王宣重光,奠麗陳教則肄。\footnote{言昔先君文武,布其重光,累聖之德,定天命,施陳教,則勤勞。○重光,馬云:“日月星也。太極上元十一月朔旦冬至,日月如疊璧,五星如連珠,故曰重光。”重,直龍反。麗,力馳反。肄,徐以至反,又以制反。}肄不違,用克達殷,集大命。\footnote{文武定命陳教,雖勞而不違道,故能通殷為周,成其大命。}在後之侗,敬迓天威,嗣守文武大訓,無敢昏逾。\footnote{在文武后之侗稚,成王自斥。敬迎天之威命,言奉順繼守文武大教。無敢昏亂逾越,言戰慄畏懼。○侗徐音同,又敕動反,馬本作詷,云:“共也。”斥,昌亦反。}今天降疾殆,弗興弗悟。爾尚明時朕言,\footnote{今天下疾我身,甚危殆,不起不悟。言必死。汝當庶幾明是我言,勿忽略。}用敬保元子釗,弘濟於艱難,\footnote{用奉我言,敬安太子釗。釗,康王名。大渡於艱難,勤德政。○釗,姜遼反,又音昭,徐之餚反。}柔遠能邇,安勸小大庶邦。\footnote{言當和遠,又能和近,安小大眾國,勸使為善。}思夫人自亂於威儀,爾無以釗冒貢於非幾。”\footnote{群臣皆宜思夫人,夫人自治正於威儀。有威可畏,有儀可象,然後足以率人。汝無以釗冒進於非危之事。}


{\noindent\zhuan\zihao{6}\fzbyks 傳“病日”至“命汝”。正義曰:“病日至”者,言日日益至,遍於身體,困甚也。“已久留”者,言病來多日,無瘳愈也。“恐死不得結信出言嗣續我志”,志欲有言,若不能言,則不得續志。以此,及今能言,故我詳審出言教命汝。言己詳審,欲其敬聽之。 \par}

{\noindent\zhuan\zihao{6}\fzbyks 傳“今天”至“忽略”。正義曰:孔讀“殆”上屬為句,“今天下疾我身,甚危殆”也。“不起”言身不能起,“不悟”言心不能覺悟。病者形弱神亂,“不起不悟”,言必死也。 \par}

{\noindent\shu\zihao{5}\fzkt “王曰”至“非幾”。正義曰:王召群臣既集,乃言而嘆曰,嗚呼!我疾大進益重,惟危殆矣。病日日益至,言病困已甚。病既久留於我身,恐一旦暴死,不得結誓出言語以繼續我志,以此故,我今詳審教訓命誥汝等。昔先君文王、武王,布其重光,累聖之德,安定天命,施陳教誨,則勤勞矣。文武定命陳教,雖勞而不違於道,用能通殷為周,成其大命,代殷為主。至文武后之侗稚,成王自謂己也。言己常敬迎天之威命,終當奉順天道,繼守文武大教,無敢昏逾越。言常戰慄畏懼,恐墜文武之業。今天降疾於我身,甚危殆矣。不能更起,不復覺悟。言己必死。汝等庶幾明是我言,勿忽略之。用我之語,敬安太子釗,大渡於艱難。言當安和遠人,又須能和近人,當為善政,遠近俱安之。又當安勸小大眾國,於彼小大眾國皆安之勸之。安之使國得安存,勸之使相勸為善。汝群臣等思夫人,夫人眾國各自治正於威儀。有威有儀,然後可以率人。無威無儀,則民不從命。戒使慎威儀也。汝無以釗冒進於非事危事。欲令戒其不為惡也。 \par}

茲既受命還,\footnote{此群臣已受顧命,各還本位。}出綴衣於庭。越翼日乙丑,王崩。\footnote{綴衣,幄帳。群臣既退,徹出幄帳於庭。王寢於北墉下,東首,反初生。於其明日,王崩。○出如字,徐尺遂反。綴,丁衛反,下同。王崩,馬本作“成王崩”,注:“安民立政曰成。”幄,於角反,下同。墉音容,本亦作牖。首,手又反。}太保命仲桓、南宮毛,\footnote{冢宰攝政,故命二臣。桓、毛,名。}俾爰齊侯呂伋,以二干戈、虎賁百人,逆子釗於南門之外。\footnote{臣子皆侍左右,將正太子之尊,故出於路寢門外。使桓、毛二臣各執干戈,於齊侯呂伋索虎賁百人,更新逆門外,所以殊之。伋為天子虎賁氏。○俾,必爾反。伋,居及反,齊侯名,太公子。}


{\noindent\zhuan\zihao{6}\fzbyks 傳“此群”至“本位”。正義曰:\CJKunderwave{周禮}:“射人掌國之三公、孤、卿大夫之位。三公北面,孤東面,卿大夫西面。”\CJKunderline{鄭玄}云:“不言士者,此與諸侯之賓射,士不與也。凡朝、燕及射,臣見於君之禮同。”鄭知然者,以\CJKunderwave{周禮}司士掌治朝之位,與\CJKunderwave{射人}同,是天子之澄芻與射禮位同。案\CJKunderwave{燕禮}小臣納卿大夫,卿大夫皆北面,公命爾卿東方西面,爾大夫少進,皆北面。\CJKunderwave{大射禮}其位亦然。是諸侯燕位與射位同,故云“朝、燕及射,臣見於君之禮同”。但天子臣多,故三公北面,孤東面,卿大夫西面。諸侯臣少,故卿西面,大夫北面,其士與天子同,皆門內西方東面。其入門當立定位如是,及王呼與言,必各自前進。已受顧命,退還本位者,謂還本治事之位,故孔下傳云:“朝臣就次,謂退王庭而還治事之處。” \par}

{\noindent\zhuan\zihao{6}\fzbyks 傳“綴衣”至“王崩”。正義曰:“綴衣”者,連綴衣物,出之於庭,則是從內而出。下雲“狄設黼扆綴衣”,則綴衣是黼扆之類。黼扆是王坐立之處,知綴衣是施張於王坐之上,故以為“幄帳”也。\CJKunderwave{周禮}:“幕人掌帷幕、幄帟、綬之事。”\CJKunderline{鄭玄}云:“在旁曰帷,在上曰幕。帷幕皆以布為之,四合象宮室曰幄,王所居之帳也。帟,王在幕居幄中,坐上承塵也。幄帟皆以繒為之。”然則幄帳是黼扆之上所張之物。此言“出綴衣於庭”,則亦並出黼扆,故下句雲象王平生之時,更復設之。王發顧命在此黼扆幄帳之坐,命訖,乃復反於寢處。以王病重,不復能臨此坐,故徹出幄帳於庭,將欲為死備也。傳更解徹去幄帳之意,以王病困,寢不在此。\CJKunderwave{喪大記}云:“疾病,君、大夫徹懸,士去琴瑟,寢東首於北墉下,廢床。”\CJKunderline{鄭玄}云:“廢,去也。人始生在地,去床,庶其生氣反。也。”\CJKunderwave{記}言君、大夫、士,則尊卑皆然,故知此時王亦“寢於北墉下,東首,反初生”也。 \par}

{\noindent\zhuan\zihao{6}\fzbyks 傳“臣子”至“賁氏”。正義曰:天子初崩,太子必在其側。解其迎於門外之意,於時臣子皆侍左右,將正太子之尊,故使太子出於路寢門外,更迎入,所以殊之也。經言“以二干戈”,文在“齊侯呂伋”下,似就齊侯取干戈。傳言“使桓、毛二臣各執干戈,於齊侯呂伋索虎賁”,則是執干戈就齊侯。傳似反於經者,於時新遭大禍,內外嚴戒,桓、毛二人必是武臣宿衛,先執干戈,太保就命,使之就干戈以往,傳達其意,故移“干戈”之文於“齊侯”之上,傳言是實也。經言“於齊侯呂伋”,下言“以二干戈、虎賁百人”者,指說迎太子之時有此備衛耳,非言二人干戈亦是齊侯授也。\CJKunderwave{周禮}虎賁氏下大夫,其屬有虎士八百人。知伋為天子虎賁氏,故就伋取虎賁也。 \par}

延入翼室,恤宅宗。\footnote{明室,路寢。延之使居憂,為天下宗主。}丁卯,命作冊度。\footnote{三日,命史為冊書法度,傳顧命於康王。○度,舊音杜洛反,恐誤,注云“作冊書法度”,音宜如字。傳,直專反。}

{\noindent\zhuan\zihao{6}\fzbyks 傳“明室”至“宗主”。正義曰:\CJKunderwave{釋言}云:“翼,明也。”\CJKunderwave{喪大記}云:“君夫人卒於路寢。”以諸侯薨於路寢,知天子亦崩於路寢。今延太子入室,必延入喪所,知“翼室”是明室,謂路寢也。路寢之大者,故以“明”言之。延之使憂居喪主,為天下宗主也。 \par}

{\noindent\zhuan\zihao{6}\fzbyks 傳“三日”至“康王”。正義曰:\CJKunderwave{周禮}內史掌策命,故命內史為策書也。經不言“命史”,史是常職,不假言之。王之將崩,雖口有遺命,未作策書,故以此日作之。既作策書,因作受策法度。下雲“曰皇后憑玉幾”,宣成王言,是策書也。將受命時,升階即位,及傳命已後,康王答命,受同祭饗,皆是法度。 \par}

{\noindent\shu\zihao{5}\fzkt “茲既”至“冊度”。正義曰:此群臣既受王命,還複本位,出連綴之衣,王所坐幄帳,置之於庭。於其明日乙丑,王崩矣。太保召公命仲桓、南宮毛,使此二人於齊侯呂伋之所,以二干戈,桓毛各執其一,又取虎賁之士百人,迎太子釗於南門之外。逆此太子,使入於路寢明室。令太子在室當喪憂居,為天下宗主,正其將王之位以系群臣之心也。 \par}

越七日癸酉,伯相命士須材。\footnote{邦伯為相,則召公。於丁卯七日癸酉,召公命士致材木,須待以供喪用。○相,息亮反。供音恭。}狄設黼扆、綴衣。\footnote{狄,下士。扆,屏風,畫為斧文,置戶牖間。復設幄帳,象平生所為。○黼音甫,徐音補。扆,於豈反。屏,步經反。畫,胡卦反。牖音酉。復,扶又反。}


{\noindent\zhuan\zihao{6}\fzbyks 傳“邦伯”至“喪用”。正義曰:成王既崩,事皆聽於冢宰,自非召公無由發命,知“伯相”即召公也。王肅云:“召公為二伯,相王室,故曰伯相。”上言“太保命仲桓”,此改言“伯相”者,於此所命士多,非是國相不得大命諸侯,故改言“伯相”,以見政皆在焉。“於丁卯七日癸酉”,則王乙丑崩,於今已九日矣。於九日始傳顧命,不知其所由也。\CJKunderline{鄭玄}云:“癸酉蓋大斂之明日也。”鄭大夫已上殯斂,皆以死之來日數,天子七日而殯,於死日為八,故以癸酉為殯之明日。孔不為傳,不必如鄭說也。“須”訓待也。今所命者皆為喪事,知“命士須材”者,“召公命士致材木,須待以供喪用”,謂槨與明器,是喪之雜用也。案\CJKunderwave{士喪禮}將葬,筮宅之後,始作槨及明器。此既殯即須材木者,以天子禮大,當須預營之。故\CJKunderwave{禮記}云:“虞人致百祀之木,可為棺槨者斬之。”是與士禮不同。顧氏亦云:“命士供葬槨之材。” \par}

{\noindent\zhuan\zihao{6}\fzbyks 傳“狄下”至“所為”。正義曰:\CJKunderwave{禮記·祭統}云:“狄者樂吏之賤者也。”是賤官有名為狄者,故以狄為下士。\CJKunderwave{喪大記}復魄之禮雲“狄人設階”,是喪事使狄與此同也。\CJKunderwave{釋宮}云:“牖戶之間謂之扆。”李巡曰:“謂牖之東戶之西為扆。”郭璞曰:“窗東戶西也。\CJKunderwave{禮}雲‘斧,扆者’,以其所在處名之。”郭璞又云:“\CJKunderwave{禮}有斧扆,形如屏風,畫為斧文,置於扆地,因名為扆。”是先儒相傳,黼扆者,屏風,畫為斧文,在於戶牖之間。\CJKunderwave{考工記}云:“畫繢之事,白與黑謂之黼。”是用白黑畫屏風置之於扆地,故名此物為“黼扆”。上文言“出綴衣於庭”,此復設黼扆帷幄帳者,象王平生時所為也。經於四坐之上言“設黼扆、綴衣”,則四坐皆設之。此經所云“狄設”,亦是伯相命狄使設之。不言“命”者,上雲“命士”,此蒙“命”文。設四坐及陳寶玉、兵器與輅車,各有所司,皆是相命,不言所命之人,從上省文也。 \par}

{\noindent\shu\zihao{5}\fzkt “越七日”至“癸酉”。正義曰:自此以下至“立於側階”,惟“命士須材”是擬供喪用,其餘皆是將欲傳命佈設之事。四坐王之所處者,器物國之所寶者,車輅王之所乘者,陳之所以華國,且以示重顧命。其執兵器立於門內堂階者,所以備不虞,亦為國家之威儀也。 \par}

牖間南向,敷重篾席,黼純,華玉仍幾。\footnote{蔑,桃枝竹。白黑雜繒緣之。華,彩色。華玉以飾憑几。仍,因也。因生時,幾不改作。此見群臣、覲諸侯之坐。○向,許亮反。篾,眠結反,馬云:“纖蒻。”純,之允反,又之閏反,下同。緣,悅絹反,本或作純。}西序東向,敷重厎席,綴純,文貝仍幾。\footnote{東西廂謂之序。厎,蒻蘋。綴,雜彩。有文之貝飾幾。此旦夕聽事之坐。○厎,之履反,馬云:“青蒲也。”蒻音弱。蘋音平。}東序西向,敷重豐席,畫純,雕玉仍幾。\footnote{豐,莞。彩色為畫。雕,刻鏤。此養國老饗群臣之坐。○豐,芳弓反。莞音官,又音關。鏤,來豆反。}


{\noindent\zhuan\zihao{6}\fzbyks 傳“篾桃”至“之坐”。正義曰:此篾席與\CJKunderwave{周禮}“次席”一也。鄭注彼云:“次席,桃枝席,有次列成文。”\CJKunderline{鄭玄}不見孔傳,亦言是桃枝席,則此席用桃枝之竹,必相傳有舊說也。鄭注此下則云:“篾,析竹之次青者。”王肅云:“篾席,纖蒻蘋席。”並不知其所據也。\CJKunderwave{考工記}云:“白與黑謂之黼。”\CJKunderwave{釋器}云:“緣謂之純。”知黼純是白黑雜繒緣之,蓋以白繒黑繒錯雜彩以緣之。\CJKunderline{鄭玄}注\CJKunderwave{周禮}云:“斧謂之黼,其繡白黑採也,以絳帛為質。”其意以白黑之線縫剌為黼又以緣席,其事或當然也。“華”是彩之別名,故以為“彩色,用華玉以飾憑几”也。\CJKunderline{鄭玄}云:“華玉,五色玉也。”“仍,因也”,\CJKunderwave{釋詁}文。\CJKunderwave{周禮}云:“凡吉事變幾,凶事仍幾。”禮之於幾有變有仍,故特言“仍幾”,以見“因生時,幾不改作”也。“此見群臣、覲諸侯之坐”,\CJKunderwave{周禮}之文知之。又\CJKunderwave{覲禮},天子待諸侯,“設斧扆於戶牖之間,左右幾,天子袞冕負斧扆”。彼在朝,此在寢為異,其牖間之坐則同。 \par}

{\noindent\zhuan\zihao{6}\fzbyks 傳“東西”至“之坐”。正義曰:“東西廂謂之序”,\CJKunderwave{釋宮}文。孫炎曰:“堂東西牆,所以別序內外也。”\CJKunderwave{禮}注謂蒲席為蒻蘋,孔以“厎席”為蒻蘋,當謂蒲為蒲蒻之席也。史游\CJKunderwave{急就篇}雲“蒲蒻藺席”,“蒲蒻”謂此也。王肅云:“厎席,青蒲席也。”\CJKunderline{鄭玄}云:“厎,致也。篾纖致席也。”鄭謂此“厎席”亦竹蓆也。凡此重席,非有明文可據,各自以意說耳。“綴”者,連綴諸色。席必以彩為緣,故以“綴”為雜彩也。“貝”者水蟲,取其甲以飾器物。\CJKunderwave{釋魚}於貝之下云:“餘蚳,黃白文。餘泉,白黃文。”李巡曰:“貝甲以黃為質,白為文彩,名為餘蚳。貝甲以白為質,黃為文彩,名為餘泉。”“有文之貝飾幾”,謂用此餘蚳、餘泉之貝飾幾也。“此旦夕聽事之坐”,鄭、王亦以為然。牖間是“見群臣、覲諸侯之坐”,見於\CJKunderwave{周禮}。其東序西向,“養國老饗群臣之坐”者,案\CJKunderwave{燕禮}雲“坐於阼階上,西向”,則養國老及饗與\CJKunderwave{燕禮}同。其西序之坐在燕饗坐前,以其旦夕聽事,重於燕飲,故西序為“旦夕聽事之坐”。夾室之坐在燕饗坐後,又夾室是隱映之處,又親屬輕於燕饗,故夾室為“親屬私宴之坐”。案朝士職掌治朝之位,王南面,此“西序東向”者,以此諸坐並陳,避牖間南鄉覲諸侯之坐故也。王肅說四坐,皆與孔同。 \par}

{\noindent\zhuan\zihao{6}\fzbyks 傳“豐莞”至“之坐”。正義曰:\CJKunderwave{釋草}云:“莞,苻籬。”郭璞曰:“今之西方人呼蒲為莞,用之為席也。”又云“𦸣,鼠莞”。樊光曰:“\CJKunderwave{詩}云:‘下莞上簟。’”郭璞曰:“似莞而纖細,今蜀中所出莞席是也。”王肅亦云:“豐席,莞。”\CJKunderline{鄭玄}云:“豐席,刮凍竹蓆。”\CJKunderwave{考工記}云:“畫繢之事,雜五色。”是彩色為畫,蓋以五彩色畫帛以為緣。\CJKunderline{鄭玄}云:“似雲氣,畫之為緣。”\CJKunderwave{釋器}云:“玉謂之雕,金謂之鏤,木謂之刻。”是“雕”為刻鏤之類,故以“刻鏤”解“雕”,蓋雜以金玉,刻鏤為飾也。 \par}

西夾南向,敷重筍席,玄紛純,漆仍幾。\footnote{西廂夾室之前。筍,蒻竹。玄紛,黑綬。此親屬私宴之坐,故席幾質飾。○夾,工洽反,徐音頰,注同。筍,息允反,馬云:“箁,箬也。”徐云:“竹子。竹為席。於貧反。”紛,孚雲反。漆音七,徐七利反。綬音受。}越玉五重,陳寶,\footnote{於東西序坐北,列玉五重,又陳先王所寶之器物。○越玉,馬云:“越地所獻玉也。”重,直容反。}赤刀、大訓、弘璧、琬琰,在西序。\footnote{寶刀,赤刀削。大訓,\CJKunderwave{虞書}典謨。大璧、琬琰之珪為二重。○琬,紆晚反。琰,以冉反。削音笑。}


{\noindent\zhuan\zihao{6}\fzbyks 傳“西廂”至“質飾”。正義曰:下傳雲“西房,西夾坐東”,“東房,東廂夾室”,然則“房”與“夾室”實同而異名。天子之室有左右房,房即室也。以其夾中央之大室,故謂之“夾室”。此坐在西廂夾室之前,故系“夾室”言之。\CJKunderwave{釋草}云:“筍,竹萌。”孫炎曰:“竹初萌生謂之筍。”是“筍”為蒻竹,取筍竹之皮以為席也。“紛”則組之小別。\CJKunderline{鄭玄}\CJKunderwave{周禮}注云:“紛如綬,有文而狹者也。”然則紛、綬一物,小大異名,故傳以“玄紛”為黑綬。鄭於此注云:“以玄組為之緣。”\CJKunderwave{周禮·大宗伯}云:“以飲食之禮,親宗族兄弟。”\CJKunderline{鄭玄}云:“親者,使之相親。人君有食宗族飲酒之禮,所以親之也。\CJKunderwave{文王世子}云:‘族食,世降一等。’”是天子有與親屬私宴之事。以骨肉情親,不事華麗,故席幾質飾也。 \par}

{\noindent\zhuan\zihao{6}\fzbyks 傳“於東”至“器物”。正義曰:此經為下總目,下復分別言之越訓於也。“越”訓於也。“於”者,於其處所。上雲“西序東向”、“東序西向”,則序旁已有王之坐矣。下句陳玉復雲“在西序”、“在東序”者,明於東西序坐北也。“序”者牆之別名,其牆南北長,坐北猶有序牆,故言“在西序”、“在東序”也。西序二重,東序三重,二序共為列玉五重。又陳先王所寶之器物,河圖、大訓、貝、鼓、戈、弓皆是先王之寶器也。 \par}

{\noindent\zhuan\zihao{6}\fzbyks 傳“寶刀”至“二重”。正義曰:上言“陳寶”,非寶則不得陳之,故知“赤刀”為寶刀也。謂之“赤刀”者,其刀必有赤處。刀一名削,故名赤刃削也。\CJKunderwave{禮記·少儀}記執物授人之儀云:“刀授穎,削授拊。”\CJKunderline{鄭玄}云:“避用時也。穎,鐶也。拊謂把也。”然則刀施鐶,削用把。削似小於刀,相對為異,散文則通,故傳以“赤刀”為“赤刀削”。\CJKunderwave{吳錄}稱吳人嚴白虎聚眾反,遣弟興詣孫策,策引白削斫席,興體動,曰:“我見刃為然。”然赤刃為赤削,白刃為白削,是削為刀之別名明矣。\CJKunderwave{周禮·考工記}云:“築氏為削,合六而成規。”鄭注云:“曲刃刀也。”又云:“赤刀者,武王誅紂時刀,赤為飾,周正色。”不知其言何所出也。“大訓,\CJKunderwave{虞書}典謨”王肅亦以為然,鄭雲“大訓謂禮法,先王德教”,皆是以意言耳。“弘”訓大也。“大璧、琬琰之圭為二重”,則琬琰共為一重。\CJKunderwave{周禮·典瑞}雲“琬圭以治德,琰圭以易行”,則琬琰別玉而共為重者,蓋以其玉形質同,故不別為重也。\CJKunderwave{考工記}琬圭、琰圭皆九寸。\CJKunderline{鄭玄}云:“大璧、琬、琰皆度尺二寸者。”孔既不分為二重,亦不知何所據也。 \par}

大玉、夷玉、天球、河圖,在東序。\footnote{三玉為三重。夷,常也。球,雍州所貢。河圖,八卦。伏犧王天下,龍馬出河,遂則其文以畫八卦,謂之河圖,及典謨皆歷代傳寶之。○夷玉,馬云:“東夷之美玉。”\CJKunderwave{說文}夷玉即珣玗琪。球音求,馬云:“玉磬。”雍,於用反,本亦作邕。}胤之舞衣、大貝、鼖鼓,在西房。\footnote{胤國所為舞者之衣,皆中法。大貝,如車渠。鼖鼓長八尺,商周傳寶之。西房,西夾坐東。○鼖,扶雲反,注同。中,丁仲反。車,尺遮反。車渠,車輈也。}兌之戈、和之弓、垂之竹矢,在東房。\footnote{兌、和,古之巧人。垂,舜\CJKunderline{共工}。所為皆中法,故亦傳寶之。東房,東廂夾室。○兌,徒外反。共音恭。}


{\noindent\zhuan\zihao{6}\fzbyks 傳“三玉”至“寶之”。正義曰:“三玉為三重”,與上共為五重也。“夷,常”,\CJKunderwave{釋詁}文。\CJKunderwave{禹貢}雍州所貢球、琳、琅玕,知球是雍州所貢也。常玉、天球傳不解“常”、“天”之義,未審孔意如何。王肅云:“夷玉,東夷之美玉。天球,玉磬也。”亦不解稱天之意。\CJKunderline{鄭玄}云:“大玉,華山之球也。夷玉,東北之珣玕琪也。天球,雍州所貢之玉,色如天者。皆璞,未見琢治,故不以禮器名之。”\CJKunderwave{釋地}云:“東方之美者,有醫無閭之珣玕琪焉。”東方實有此玉。鄭以夷玉為彼玉,未知經意為然否。“河圖,八卦。是伏羲氏王天下,龍馬出河,遂則其文以畫八卦,謂之河圖”,當孔之時,必有書為此說也。\CJKunderwave{漢書·五行志}:“劉歆以為伏犧氏繼天而王,受河圖,則而畫之,八卦是也。”劉歆亦如孔說,是必有書明矣。\CJKunderwave{易·繫辭}云:“古者包犧氏之王天下也,仰則觀象於天,俯則觀法於地,觀鳥獸之文與地之宜,近取諸身,遠取諸物,於是始作八卦。”都不言法河圖也。而此傳言“河圖”者,蓋\CJKunderwave{易}理寬弘,無所不法,直如\CJKunderwave{繫辭}之言,所法已自多矣,亦何妨更法河圖也。且\CJKunderwave{繫辭}又云:“河出圖,洛出書,聖人則之。”若八卦不則河圖,餘復何所則也?王肅亦云:“河圖,八卦也。”璧,玉人之所貴,是為可寶之物。八卦、典謨非金玉之類,嫌其非寶,故云“河圖及典謨皆歷代傳寶之”。此西序、東序各陳四物,皆是臨時處置,未必別有他義。下二房各有二物,亦應無別意也。 \par}

{\noindent\zhuan\zihao{6}\fzbyks 傳“胤國”至“坐東”。正義曰:以夏有胤侯,知“胤”是國名也。胤是前代之國,舞衣至今猶在,明其所為中法,故常寶之。亦不知舞者之衣是何衣也。大貝必大於餘貝。伏生\CJKunderwave{書傳}云:“散宜生之江淮,取大貝,如大車之渠。”是言大小如車渠也。\CJKunderwave{考工記}謂車罔為渠。大小如車罔,其貝形曲如車罔,故比之也。\CJKunderwave{考工記}云:“鼓長八尺,謂之鼖鼓。”\CJKunderwave{釋樂}云:“大鼓謂之鼖。”此鼓必有所異。周興至此未久,當是先代之器,故云“商周傳寶之”。西序即是西夾,西夾之前已有南向坐矣,西序亦陳之寶,近在此坐之西,知此“在西房”者,在西夾室東也。 \par}

{\noindent\zhuan\zihao{6}\fzbyks 傳“兌和”至“夾室”。正義曰:戈、弓、竹矢,巧人所作。垂是巧人,知兌、和亦古之巧人也。垂,舜\CJKunderline{共工},\CJKunderwave{舜典}文。若不中法,即不足可寶,知“所為皆中法,故亦傳寶之”。垂是舜之\CJKunderline{共工},竹矢蓋舜時之物。其兌、和之所作,則不知寶來幾何世也,故皆言“傳寶之”耳。東夾室無坐,故直言“東廂夾室”,陳於夾室之前也。案鄭注\CJKunderwave{周禮},宗廟、路寢制如明堂。明堂則五室,此路寢得有東房、西房者,\CJKunderwave{鄭志}張逸以此問,鄭答云:“成王崩在鎬京。鎬京宮室因文武,更不改作,故同諸侯之制,有左右旁也。”孔無明說,或與鄭異,路寢之制不必同明堂也。 \par}

大輅在賓階面,綴輅在阼階面,\footnote{大輅,玉。綴輅,金。面,前。皆南向。○阼,才故反。向,許亮反。}先輅在左塾之前,次輅在右塾之前。\footnote{先輅,象。次輅,木。金、玉、象皆以飾車,木則無飾,皆在路寢門內,左右塾前北面。凡所陳列,皆象成王生時華國之事,所以重顧命。○塾音孰,一音育。重,直用反。}

{\noindent\zhuan\zihao{6}\fzbyks 傳“大輅”至“南向”。正義曰:\CJKunderwave{周禮}巾車“掌王之五輅”,玉輅、金輅、象輅、革輅、木輅,是為五輅也。此經所陳四輅必是\CJKunderwave{周禮}五輅之四。“大輅”,輅之最大,故知大輅玉輅也。“綴輅”,系綴於下,必是玉輅之次,故為金輅也。“面前”者,據人在堂上,面向南方,知面前皆南向,謂轅向南也。地道尊右,故玉輅在西,金輅在東。 \par}

{\noindent\zhuan\zihao{6}\fzbyks 傳“先輅”至“顧命”。正義曰:此經四輅兩兩相配,上言“大輅”、“綴輅”,此言“先輅”、“次輅”,二者各自以前後為文。五輅金即次象,故言“先輅,象”。其木輅在象輅之下,故云“次輅,木”也。又解四輅之名,“金、玉、象皆以飾車”,三者以飾為之名;“木則無飾”,故指木為名耳。\CJKunderline{鄭玄}\CJKunderwave{周禮}注云“革輅,挽之以革而漆之”,“木輅,不挽以革,漆之而已”,以直漆其木,故以“木”為名。木輅之上猶有革輅,不以“次輅”為革輅者,禮五輅而此四輅,於五之內必將少一,蓋以革輅是兵戎之用,於此不必陳之,故不雲革輅而以木輅為次。馬融、王肅皆云:“不陳戎輅者,兵事非常,故不陳之。”孔意或當然也。\CJKunderline{鄭玄}以“綴、次是從後之言,二者皆為副貳之車。先輅是象輅也,綴輅是玉輅之貳,次輅是象輅之貳,不陳金輅、革輅、木輅者,主於朝祀而已”。未知孔、鄭誰得經旨。成王殯在路寢,下雲“二人執惠,立於畢門之內”,畢門是路寢之門,知此陳設車輅“皆在路寢門內”也。\CJKunderwave{釋宮}云:“門側之堂謂之塾。”孫炎曰:“夾門堂也。”塾前陳車,必以轅向堂,故知“左右塾前皆北面”也。左塾者謂門內之西,右塾者門內之東,故以北面言之為左右。所陳坐位、器物皆以西為上,由王殯在西序故也。其執兵宿衛之人則先東而後西者,以王在東宿,衛敬新王故也。顧氏云:“先輅在左塾之前,在寢門內之西,北面對玉輅。次輅在右塾之前,在寢門內之東,對金輅也。”凡所陳列自“狄設黼扆”已下至此,皆象成王生時華國之事,所以重顧命也。\CJKunderline{鄭玄}亦云:“陳寶者,方有大事以華國也。”\CJKunderwave{周禮·典路}云:“若有大祭祀,則出路。大喪大賓客亦如之。”是大喪出輅為常禮也。 \par}

{\noindent\shu\zihao{5}\fzkt “牖間”至“漆仍幾”。正義曰:“牖”謂窗也,“間”者窗東戶西戶牖之間也。\CJKunderwave{周禮·司几筵}云:“凡大朝覲、大饗射,凡封國命諸侯,王位設黼扆。扆前南向,設莞筵紛純,加繅席畫純,加次席黼純,左右玉幾。”彼所設者即此坐也。又云:“戶牖之間謂之扆。”彼言“扆前”,此言“牖間”,即一坐也。彼言“次席黼純”,此言“篾席黼純”,亦一物也。\CJKunderwave{周禮}天子之席三重,諸侯之席再重,則此四坐所言敷重席者,其席皆敷三重。舉其上席而言“重”,知其下更有席也。此牖間之坐即是\CJKunderwave{周禮}扆前之坐,篾席之下二重,其次是繅席畫純,其下是莞筵紛純也。此一坐有\CJKunderwave{周禮}可據,知其下二席必然。下文三坐,\CJKunderwave{禮}無其事,以扆前一坐敷三種之席,知下三坐必非一重之席敷三重,但不知其下二重是何席耳。\CJKunderwave{周禮}天子左右幾,諸侯惟右幾,此言“仍幾”,則四坐皆左右幾也。\CJKunderline{鄭玄}云:“左右有幾,優至尊也”。 \par}

二人雀弁,執惠,立於畢門之內。\footnote{士衛殯與在廟同,故雀韋弁。惠,三隅矛。路寢門,一名畢門。○弁,皮彥反,徐扶變反。}四人綦弁,執戈上刃,夾兩階戺。\footnote{綦,文鹿子皮弁。亦士。堂廉曰戺,士所立處。○綦音其,馬本作騏,云:“青黑色。”夾,徐工洽反。戺音俟,徐音士。廉,力佔反,稜也。}


{\noindent\zhuan\zihao{6}\fzbyks 傳“士衛”至“畢門”。正義曰:士入廟助祭,乃服雀弁,於此服雀弁者,士衛王殯,與在廟同,故爵韋弁也。\CJKunderline{鄭玄}云:“赤黑曰雀,言如雀頭色也,雀弁制如冕,黑色,但無藻耳。”然則雀弁所用當與冕同。阮諶\CJKunderwave{三禮圖}云:“雀弁以三十升布為之。”此傳言“雀韋弁”者,蓋以\CJKunderwave{周禮·司服}雲“凡兵事,韋弁服”,此人執兵,宜以韋為之,異於祭服,故言“雀韋弁”。下雲“綦弁”,孔言鹿子皮為弁,然則下言冕執兵者,不可以韋為冕,未知孔意如何。天子五門,皋、庫、雉、應、路也。下雲“王出在應門之內”,出畢門始至應門之內,知畢門即是路寢之門,一名畢門也。此經所陳七種之兵,惟戈經傳多言之,\CJKunderwave{考工記}有其形制,其餘皆無文。傳惟言“惠,三隅矛”,銳亦矛也,“戣、瞿皆戟屬”,不知何所據也。“劉,鉞屬”者,以“劉”與“鉞”相對,故言“屬”以似之,而別又不知何以為異。古今兵器名異體殊,此等形制,皆不可得而知也。\CJKunderline{鄭玄}云:“惠狀蓋斜刃,宜芟刈。戈即今之句孑戟。劉蓋今鑱斧。鉞,大斧。戣、瞿蓋今三鋒矛。銳,矛屬。凡此七兵,或施矜,或著柄。\CJKunderwave{周禮}戈長六尺六寸,其餘未聞長短之數。”王肅惟云:“皆兵器之名也。” \par}

{\noindent\zhuan\zihao{6}\fzbyks 傳“綦文”至“立處”。正義曰:\CJKunderline{鄭玄}雲“青黑曰綦”,王肅雲“綦,赤黑色”,孔以為“綦,文攏蘗子皮弁”,各以意言,無正文也。大夫則服冕,此服弁,知“亦士”也。“堂廉曰戺”,相傳為然。“廉”者,稜也。所立在堂下,近於堂稜。 \par}

一人冕,執劉,立於東堂。一人冕,執鉞,立於西堂。\footnote{冕,皆大夫也。劉,鉞屬。立於東西廂之前堂。}一人冕,執戣,立於東垂。一人冕,執瞿,立於西垂。\footnote{戣、瞿皆戟屬。立於東西下之階上。○戣音逵。瞿,其俱反,徐音懼。}一人冕,執銳,立於側階。\footnote{銳,矛屬也。側階,北下立階上。○銳,以稅反。}

{\noindent\zhuan\zihao{6}\fzbyks 傳“冕皆”至“前堂”。正義曰:\CJKunderwave{周禮·司服}云:“大夫之服,自玄冕而下。”知服冕者皆大夫也。\CJKunderline{鄭玄}云:“序內半以前曰堂。”謂序內簷下,自室壁至於堂廉,中半以前總名為“堂”。此立於東堂、西堂者,當在東西廂近階而立,以備升階之人也。 \par}

{\noindent\zhuan\zihao{6}\fzbyks 傳“戣瞿”至“階上”。正義曰:\CJKunderwave{釋詁}云:“疆、界、邊、衛、圉、垂也。”則“垂”是遠外之名。此經所言“冕”則在堂上,“弁”則在堂下,此二人服冕,知在堂上也。堂上而言“東垂”、“西垂”,知在堂上之遠地。堂之遠地,當於序外東廂西廂,必有階上堂,知此立於東西堂之階上也。 \par}

{\noindent\zhuan\zihao{6}\fzbyks 傳“銳矛”至“階上”。正義曰:鄭、王皆以側階為東下階也。然立於東垂者已在東下階上,何由此人復共並立?故傳以為“北下階上”,謂堂北階,北階則惟堂北一階而已。“側”猶特也。 \par}

{\noindent\shu\zihao{5}\fzkt “二人”至“側階”。正義曰:\CJKunderwave{禮}大夫服冕,士服弁也。此所執者凡有七兵,立於畢門之內,及夾兩階立堂下者,服爵弁、綦弁者,皆士也。以其去殯遠,故使士為之。其在堂上服冕者,皆大夫也。以其去殯近,皆使大夫為之。先門,次階,次堂,從外向內而敘之也。次東西垂,次側階,又從近向遠而敘之也。在門者兩,守門兩廂各一人,故“二人”。在階者,兩廂各二人,故“四人”。\CJKunderwave{禮記·明堂位}:“三公在中階之前。”\CJKunderwave{考工記}:“夏后氏世室,九階。”\CJKunderline{鄭玄}云:“南面三,三面各二。”\CJKunderline{鄭玄}又云,宗廟及路寢制如明堂,則路寢南面亦當有三階矣。此惟四人夾兩階,不守中階者,路寢制如明堂,惟\CJKunderline{鄭玄}之說耳,路寢三階不書,亦未有明文,縱有中階,中階無人升降,不須以兵衛之。 \par}

王麻冕黼裳,由賓階隮。\footnote{王及群臣皆吉服,用西階升,不敢當主。}卿士、邦君麻冕蟻裳,入即位。\footnote{公卿大夫及諸侯皆同服,亦廟中之禮。蟻,裳名,色玄。○蟻,魚綺反。}太保、太史、太宗皆麻冕彤裳。\footnote{執事各異裳。彤,纁也。太宗,上宗,即宗伯也。}


{\noindent\zhuan\zihao{6}\fzbyks 傳“王及”至“當主”。正義曰:\CJKunderwave{禮}績麻三十升以為冕,故稱“麻冕”。傳嫌麻非吉服,故言“王及群臣皆吉服”也。“王麻冕”者,蓋袞冕也。\CJKunderwave{周禮·司服}:“享先王則袞冕。”此禮授王冊命,進酒祭王,且袞是王之上服,於此正王之尊,明其服必袞冕也。其卿士、邦君當各以命服,服即助祭之冕矣。“袞”,\CJKunderline{鄭玄}\CJKunderwave{周禮注}云:“袞之衣五章,裳四章。”則袞衣之裳,非獨有黼。言“黼裳”者,以裳之章色,黼黻有文,故特取為文。\CJKunderwave{詩·采菽}之篇言王賜諸侯云:“玄袞及黼。”以黼有文,故特言之。\CJKunderline{鄭玄}於此注云:“黼裳者,冕服有文者也。”是言貴文故稱之。\CJKunderwave{禮}“君升阼階”,此用西階升者,以未受顧命,不敢當主也。 \par}

{\noindent\zhuan\zihao{6}\fzbyks 傳“公卿”至“色玄”。正義曰:“卿士”,卿之有事者,公則卿兼之。此行大禮,大夫亦與焉。略舉“卿士”為文,公與大夫必在,故傳言“公卿大夫及諸侯皆同服”,言同服吉服,此“亦廟中之禮”也。言其如助祭,各服其冕服也。\CJKunderwave{禮}無“蟻裳”,今雲“蟻”者,裳之名也。“蟻”者,蚍蜉蟲也,此蟲色黑,知蟻裳色玄,以色玄如蟻,故以蟻名之。\CJKunderwave{禮}祭服皆玄衣纁裳,此獨雲玄裳者,卿士、邦君於此無事,不可全與祭同,改其裳以示變於常也。太保、太史有所主者,則純如祭服,暫從吉也。“入即位”者,\CJKunderline{鄭玄}云:“卿西面,諸侯北面。”\CJKunderline{鄭玄}惟據經“卿士、邦君”言之,其公亦北面,孤東面也。 \par}

{\noindent\zhuan\zihao{6}\fzbyks 傳“執事”至“宗伯”。正義曰:此三官者皆執事,俱“彤裳”,而言“各異裳”者,各自異於卿士、邦君也。“彤”,赤也。\CJKunderwave{禮}祭服纁裳。纁是赤色之淺者,故以“彤”為纁,言是常祭服也。“太宗”與下文“上宗”一人,即宗伯之卿也。 \par}

太保承介圭,上宗奉同、瑁,由阼階隮。\footnote{大圭尺二寸,天子守之,故奉以奠康王所位。同,爵名。瑁,所以冒諸侯圭,以齊瑞信,方四寸,邪刻之。用阼階升,由便不嫌。○冒,莫報反。}太史秉書,由賓階隮,御王冊命。\footnote{太史持冊書顧命進康王,故同階。}

{\noindent\zhuan\zihao{6}\fzbyks 傳“大圭”至“不嫌”。正義曰:\CJKunderwave{考工記·玉人}云:“鎮圭尺有二寸,天子守之。”鎮圭,圭之大者。“介”訓大也,故知是彼鎮圭。天子之所守,故奉之以奠康王所位,以明正位為天子也。\CJKunderwave{禮}又有“大圭長三尺”,知“介圭”非彼三尺圭者,\CJKunderwave{典瑞}云:“王搢大圭,執鎮圭以朝日。”\CJKunderwave{玉人}云:“大圭長三尺,天子服之。”彼搢於紳帶是天子之笏,不是天子所守,故知非幣妖尺之大圭也。“上宗奉同、瑁”,則下文雲天子“受同、瑁”。大保必奠於位,其奉介圭,下文不言“受介圭”者,以同、瑁並在手中,故不得執之,太保必奠於其位,但文不見耳。\CJKunderwave{禮}於奠爵無名“同”者,但下文祭酢皆用同奉酒,知“同”是酒爵之名也。\CJKunderwave{玉人}云:“天子執冒四寸以朝諸侯。”\CJKunderline{鄭玄}注云:“名玉曰冒者,言德能覆蓋天下也。四寸者,方。以尊接卑,以小為貴。”\CJKunderwave{禮}天子所以執瑁者,諸侯即位,天子賜之以命圭,圭頭邪銳,其瑁當下邪刻之,其刻闊狹長短如圭頭。諸侯來朝,執圭以授天子,天子以冒之刻處冒彼圭頭,若大小相當,則是本所賜;其或不同,則圭是偽作,知諸侯信與不信。故天子執瑁,所以冒諸侯之圭以齊瑞信,猶今之合符。然經傳惟言圭之長短,不言闊狹。瑁方四寸,容彼圭頭,則圭頭之闊無四寸也。天子以一瑁冒天下之圭,則公侯伯之圭闊狹等也。此瑁惟冒圭耳,不得冒璧。璧亦稱瑞,不知所以齊信,未得而聞之也。“阼階”者,東階也。謂之“阼”者,\CJKunderline{鄭玄}\CJKunderwave{士冠禮}注云“阼猶酢也。東階所以答酢賓客”,是其義也。\CJKunderwave{禮}凶事設洗於西階西南,吉事設洗於東階東南。此太保、上宗皆行吉事,盥洗在東,故用阼階升,由便,以卑不嫌為主人也。\CJKunderline{鄭玄}云:“上宗猶太宗,變其文者,宗伯之長,大宗伯一人,與小宗伯二人,凡三人,使其上二人也,一人奉同,一人奉瑁。”傳無明解,當同於鄭也。 \par}

{\noindent\zhuan\zihao{6}\fzbyks 傳“太史”至“同階”。正義曰:訓“御”為進。太史持策書顧命欲以進王,故與王同升西階。\CJKunderline{鄭玄}云:“御猶向也。王此時正立賓階上少東,太史東面於殯西南而讀策書,以命王嗣位之事。”孔雖以“御”為進,其意當如鄭言。不言王面北,可知也。篇以“顧命”為名,指上文為言。顧命策書,稟王之意為言,亦是顧命之事,故傳言“策書顧命”。 \par}

{\noindent\shu\zihao{5}\fzkt “王麻”至“冊命”。正義曰:此將傳顧命,佈設位次,即上所作法度也。凡諸行禮,皆賤者先至,此必卿、下士、邦君即位既定,然後王始升階。但以君臣之序,先言王服,因服之下即言升階,從省文。卿士、邦君無所執事,故直言“即位”而已。太保、太史、太宗皆執事之人,故別言衣服。各自所職,不得即言升階,故別言所執,各從升階為文次也。卿士王臣,故先於邦君。太史乃是大宗之屬,而先於太宗者,太史之職掌冊書,此禮主以為冊命,太史所掌事重,故先言之。 \par}

曰:“皇后憑玉幾,道揚末命,命汝嗣訓,\footnote{冊命之辭。大君,成王。言憑玉幾所道,稱揚終命,所以感動康王。命汝繼嗣其道,言任重,因以託戒。○憑,皮冰反。}臨君周邦,率循大卞,\footnote{用是道臨君周國,率群臣循大法。卞,皮彥反,徐扶變反。}燮和天下,用答揚文武之光訓。”\footnote{言用和道和天下,用對揚聖祖文武之大教。敘成王意。}王再拜,興,答曰:“眇眇予末小子,其能而亂四方,以敬忌天威?”\footnote{言微微我淺末小子,其能如父祖治四方,以敬忌天威德乎?謙辭,託不能。○眇,彌小反。}


{\noindent\zhuan\zihao{6}\fzbyks 傳“冊命”至“託戒”。正義曰:言“憑玉幾所道”,以示不憑玉幾則不能言,所以感動康王,令其哀而聽之,不敢忽也。以“訓”為道,命汝繼嗣其道,繼父道為天下之主。言所任者重,因以託戒也。 \par}

{\noindent\zhuan\zihao{6}\fzbyks 傳“用是”至“大法”。正義曰:“卞”之為法,無正訓也。告以為法之道,令率群臣循之,明所循者法也,故以“大卞”為大法。王肅亦同也。 \par}

{\noindent\shu\zihao{5}\fzkt “曰皇”至“光訓”。正義曰:此即丁卯命作之冊書也。誥康王曰:“大君成王病困之時,憑玉幾所道,稱揚將終之教命。命汝繼嗣其道,代為民主。用是道以臨君周邦,率群臣循大法,用和道和天下,用對揚聖祖文武之大教。”敘成王之意,言成王命汝如此也。 \par}

乃受同、瑁,王三宿,三祭,三吒。\footnote{王受瑁為主,受同以祭。禮成於三,故酌者實三爵於王,王三進爵,三祭酒,三奠爵,告已受群臣所傳顧命。○吒,陟嫁反,字亦作宅,又音妒,徐又音託,又豬夜反。\CJKunderwave{說文}作詫,丁故反,奠爵也。馬作詫,與\CJKunderwave{說文}音義同。}上宗曰:“饗!”\footnote{祭必受福,贊王曰:“饗福酒。”}太保受同,降,\footnote{受王所饗同,下堂反於篚。}盥以異同,秉璋以酢。\footnote{太保以盥手洗異同,實酒秉璋以酢祭。半圭曰璋,臣所奉。王已祭,太保又祭。報祭曰酢。○酢,才各反。}


{\noindent\zhuan\zihao{6}\fzbyks 傳“王受”至“顧命”。正義曰:天子執瑁,故“受瑁為主”。“同”是酒器,故“受同以祭”。\CJKunderline{鄭玄}云:“王既對神,則一手受同,一手受瑁。”然既受之後,王受同而祭,則瑁以授人。“禮成於三,酌者實三爵於王”,當是實三爵而續送,三祭各用一同,非一同而三反也。\CJKunderwave{釋詁}云:“肅,進也。”“宿”即肅也,故以宿爵而續送。祭各用一同為一進,“三宿”謂三進爵,從立處而三進至神所也。三祭酒,三酹酒於神坐也。每一酹酒,則一奠爵,三奠爵於地也。為此祭者,告神言已已受群臣所傳顧命,白神使知也。經典無此“吒”字,“吒”為奠爵,傳記無文。正以既祭必當奠爵,既言“三祭”,知“三吒”為三奠爵也。王肅亦以“吒”為奠爵。\CJKunderline{鄭玄}云:“徐行前曰肅,卻行曰吒。王徐行前三祭,又三卻複本位。”與孔異也。 \par}

{\noindent\zhuan\zihao{6}\fzbyks 傳“祭必”至“福酒”。正義曰:\CJKunderwave{禮}於祭末必飲神之酒,受神之福,其大祭則有受嘏之福。\CJKunderwave{禮·特牲}、\CJKunderwave{少牢}主人受嘏福,是受神之福也。其告祭小祀,則不得備儀,直飲酒而已。此非大祭,故於王三奠爵訖,上宗以同酌酒進王,贊王曰:“饗福酒也。”王取同嚌之,乃以同授太保也。 \par}

{\noindent\zhuan\zihao{6}\fzbyks 傳“受王”至“於篚”。正義曰:上宗贊王以饗福酒也,即雲“太保受同”,明是“受王所饗同”也。祭祀飲酒之禮,爵未用皆實於篚,既飲皆反於篚,知此“下堂反於篚”也。 \par}

{\noindent\zhuan\zihao{6}\fzbyks 傳“太保”至“曰酢”。正義曰:祭祀以變為敬,不可即用王同,故太保以盥手更洗異同,實酒於同中,乃秉璋以酢祭。於上祭後更復報祭,猶如正祭大禮之亞獻也。\CJKunderwave{周禮·典瑞}云:“四圭有邸,以祀天。兩圭有邸,以祀地。圭璧以祀日月,璋邸射以祀山川。”從上而下,遞減其半,知“半圭曰璋”。\CJKunderwave{祭統}云:“君執圭瓚,大宗執璋瓚。”謂亞獻用璋瓚,此非正祭,亦是亞獻之類,故亦執璋。若助祭,公侯伯子男自得執圭璧也。秉璋以酢,是報祭之事。王已祭,太保又報祭也。“酢”訓報也,故“報祭曰酢”。飲酒之禮稱“獻酢”者,亦是報之義也。 \par}

授宗人同,拜,王答拜。\footnote{宗人,小宗伯,佐大宗伯。太宗供王,宗人供太保。拜白已傳顧命,故授宗人同。拜,王答拜,尊所受命。}太保受同,祭,嚌,\footnote{太宗既拜而祭,既祭,受福。嚌至齒,則王亦至齒。王言饗,太保言嚌,互相備。○嚌,才細反。互音戶。}宅,授宗人同,拜,王答拜。\footnote{太保居其所,授宗人同,拜白成王以事畢。王答拜,敬所白。○宅如字,馬同,徐殆故反。}太保降,收。\footnote{太保下堂,則王下可知。有司於此盡收徹。○徹,醜列反,又徐直列反。}

{\noindent\zhuan\zihao{6}\fzbyks 傳“宗人”至“受命”。正義曰:“上宗”為太宗伯,知“宗人”為小宗伯也。太保所以拜者,白成王言已已傳顧命訖也。將欲拜,故先授宗人同。拜者,自為拜神,不拜康王,但白神言已傳顧命之事,先告王已受顧命。王答拜者,尊所受之命,亦告神使知,故答拜也。王既祭,則奠同於地。太保不敢奠於地,故以同授宗人,然後拜也。太保既酢祭而拜,則王之奠爵,每奠必拜。於王祭不言“拜”者,祭酒必拜,乃是常禮。於王不言“拜”,於太保言“拜”者,足以見王拜也。 \par}

{\noindent\zhuan\zihao{6}\fzbyks 傳“太保”至“相備”。正義曰:“太保受同”者,謂太保既拜之後,於宗人邊受前所授之同,而進以祭神。既祭神之後,遂更受福酒,嚌以至齒。禮之通例,啐入口,是嚌至於齒,示飲而實不飲也。太保報王之祭,事與王祭禮同,而史錄其事,二文不等,故傳辨其意,於太保言“嚌至齒”,則王饗福酒,亦嚌至齒也;於王言“上宗曰饗”,則太保亦應有宗人曰饗,二文不同,互見以相備。 \par}

{\noindent\zhuan\zihao{6}\fzbyks 傳“太保”至“所白”。正義曰:“宅”訓居也。太保居其所,於受福酒之處,足不移,為將拜,故授宗人同。祭祀既畢而更拜者,白成王以事畢也。既拜白成王以傳顧命事畢,則王受顧命亦畢,王答拜,敬所白也。 \par}

{\noindent\shu\zihao{5}\fzkt “乃受”至“降收”。正義曰:王受冊命之時,立於西階上少東,北面。太史於柩西南,東面讀策書。讀冊既訖,王再拜。上宗於王西南,北面奉同、瑁以授王。王一手受同,一手受瑁。王又以瑁授宗人。王乃執同,就樽於兩楹之間,酌酒,乃於殯東西面立,三進於神坐前,祭神如前祭。凡前祭酒酹地而奠爵訖,復位。再拜,王又於樽所別以同酌酒,祭神如前。復三祭,故云“三宿、三祭、三吒”。然後酌福酒以授王,上宗贊王曰:“饗福酒。”王再拜,受酒,跪而祭。先嚌至齒,興,再拜。太保受同,降自東階,反於篚。又盥以異同,執璋升自東階,適樽所酌酒,至殯東西面報祭之。欲祭之時,授宗人同,拜白王柩云:“已傳顧命訖。”王則答拜。拜柩,尊所受命。太保乃於宗人處受同,祭柩如王禮,但一祭而已。祭訖,乃受福。祝酌同以授太保,宗人贊太保曰:“饗福酒。”太保再拜,受同,亦祭,先而嚌至齒,與,再拜訖,於所居位授宗人同。太保更拜白柩以事畢。王又答拜,拜柩,敬所白。王與太保降階而下堂,有司於是收徹器物。 \par}

諸侯出廟門俟。\footnote{言諸侯,則卿士已下亦可知。殯之所處,故曰廟。待王后命。○處,昌呂反。}

{\noindent\shu\zihao{5}\fzkt “諸侯出廟門俟”。正義曰:“廟門”謂路寢門也。出門待王后命,即作後篇。後篇云,二伯率諸侯入應門,則諸侯之出應門之外,非出廟門而已。以其在廟行事,事畢出於廟門,不言出廟門即止也。 \par}

%%% Local Variables:
%%% mode: latex
%%% TeX-engine: xetex
%%% TeX-master: "../Main"
%%% End:
