%% -*- coding: utf-8 -*-
%% Time-stamp: <Chen Wang: 2024-04-02 11:42:41>

% {\noindent \zhu \zihao{5} \fzbyks } -> 注 (△ ○)
% {\noindent \shu \zihao{5} \fzkt } -> 疏

\chapter{卷十}


\section{說命上第十二【偽】}


高宗夢得說,\footnote{盤庚弟小乙子,名武丁,德高可尊,故號高宗。夢得賢相,其名曰說。說,本又作兌,音悅,注及下篇同。相,息亮反。下同。}使百工營求諸野,得諸傅巖,\footnote{使百官以所夢之形象經求之於野,得之於傅巖之谿。}作\CJKunderwave{說命}三篇。\footnote{命說為相,使攝政。}


{\noindent\zhuan\zihao{6}\fzbyks 傳“\CJKunderline{盤庚}”至“曰說”。正義曰:\CJKunderwave{世本}云:“\CJKunderline{盤庚}崩,弟小辛立。崩,弟小乙立。崩,子武丁立。”是武丁為“\CJKunderline{盤庚}弟小乙子”也。\CJKunderwave{喪服四制}云:“高宗者,武丁。武丁者,殷之賢王也。當此之時,殷衰而復興,禮廢而復起,中而高之,故謂之高宗。”是“德高可尊,故號高宗”也。經雲“爰立作相”,王呼之曰“說”,知其“名曰說”。 \par}

{\noindent\zhuan\zihao{6}\fzbyks 傳“使百”至“之谿”。正義曰:以“工”為官,見其求者眾多,故舉“百官”言之。使百官以所夢之形象經營求於外野。皇甫謐雲“使百工寫其形象”,則謂“工”為工巧之人,與孔異也。\CJKunderwave{釋水}云:“水注川曰谿。”李巡曰:“水出於山,入於川曰谿。”然則“谿”是水流之處,“巖”是山崖之名。序稱“得諸傅巖”,傳雲“得之於傅巖之谿”,以“巖”是總名,故序言之耳。 \par}

{\noindent\zhuan\zihao{6}\fzbyks 傳“命說”至“攝政”。正義曰:經稱“爰立作相”,是命為相也。“惟說命總百官”,是“使攝位”也。 \par}

{\noindent\shu\zihao{5}\fzkt “高宗”至“三篇”。正義曰:殷之賢王有高宗者,夢得賢相,其名曰“說”。群臣之內既無其人,使百官以所夢之形象經營求之於野外,得之於傅氏之巖,遂命以為相。史敘其事,作\CJKunderwave{說命}三篇。 \par}

說命\footnote{始求得而命之。}

{\noindent\shu\zihao{5}\fzkt “說命”。正義曰:此三篇上篇言夢說,始求得而命之;中篇說既總百官,戒王為政;下篇王欲師說而學,說報王為學之有益,王又厲說以\CJKunderline{伊尹}之功。相對以成章,史分序以為三篇也。 \par}

王宅憂,亮陰三祀。\footnote{陰,默也。居憂,信默三年不言。亮,本又作諒,如字,又力章反。}


{\noindent\zhuan\zihao{6}\fzbyks 傳“陰默”至“不言”。正義曰:“陰”者,幽暗之義,“默”亦暗義,故為默也。\CJKunderwave{易}稱“君子之道,或默或語”,則“默”者,不言之謂也。\CJKunderwave{無逸}傳雲“乃有信默,三年不言”,有此“信默”,則“信”謂信任冢宰也。 \par}

{\noindent\shu\zihao{5}\fzkt “王宅憂,亮陰三祀”。正義曰:言王居父憂,信任冢宰,默而不言已三年矣。三年不言,自是常事,史錄此句於首者,謂既免喪事,可以言而猶不言,故述此以發端也。 \par}

既免喪,其惟弗言,\footnote{除喪,猶不言政。}群臣咸諫於王曰:“嗚呼!知之曰明哲,明晢實作則。\footnote{知事則為明智,明智則能製作法則。哲,本又作喆。}天子惟君萬邦,百官承式,\footnote{天下待令,百官仰法。}王言惟作命,不言臣下罔攸稟令。”\footnote{稟,受。令亦命也。}王庸作書以誥曰:“以臺正於四方,臺恐德弗類,茲故弗言。\footnote{用臣下怪之,故作誥。類,善也。我正四方,恐德不善,此故不言。誥,故報反。臺音怡。}恭默思道,夢帝賚予良弼,其代予言。”\footnote{夢天與我輔弼良佐,將代我言政教。賚,力代反,徐音來。}乃審厥象,俾以形旁求於天下。\footnote{審所夢之人,刻其形象,以四方旁求之於民間。俾,必爾反。}說築傅巖之野,惟肖。\footnote{傅氏之巖,在虞虢之界,通道所經,有澗水壞道,常使胥靡刑人築護此道。說賢而隱,代胥靡築之以供食。肖,似。似所夢之形。肖音笑。虢,寡白反。壞音怪。供音恭。}

{\noindent\zhuan\zihao{6}\fzbyks 傳“傅氏”至“之形”。正義曰:傳以“傅”為氏,此巖以“傅”為名,明巖傍有姓傅之民,故云“傅氏之巖”也。\CJKunderwave{屍子}云:“傅巖在北海之洲。”傳言“虞虢之界”,孔必有所案據而言之也。\CJKunderwave{史記·殷本紀}云:“是時說為胥靡,築於傅險。”晉灼\CJKunderwave{漢書音義}云:“胥,相也。靡,隨也。古者相隨坐輕刑之名。”言於時築傅險,則以杵築地,傅說賢人,必身不犯罪,言其說為胥靡,當是時代胥靡也。傳云:“通道所經,有澗水壞道,常使胥靡刑人築護此道。說賢而隱,代胥靡築之以供食。”或亦有成文也。\CJKunderwave{殷本紀}又云,武丁得說,“舉以為相,遂以傅險姓之,號曰傅說”。鄭云:“得諸傅巖,高宗因以傅命說為氏。”案序直言“夢得說”,不言“傅”,或如馬鄭之言。如高宗始命為傅氏,不知舊何氏也。皇甫謐云:“高宗夢天賜賢人,胥靡之衣蒙之而來,且云:‘我徒也,姓傅名說,天下得我者豈徒也哉!’武丁悟而推之曰:‘傅者,相也。說者,歡悅也。天下當有傅我而說民者哉!’明以夢視百官,百官皆非也。乃使百工寫其形象,求諸天下,果見築者胥靡衣褐帶索,執役於虞虢之間、傅巖之野,名說。以其得之傅巖,謂之傅說。”案謐言初夢即雲“姓傅名說”,又言“得之傅巖,謂之傅說”,其言自不相副,謐惟見此書傅,會為近世之語,其言非實事也。 \par}

爰立作相,王置諸其左右。\footnote{於是禮命立以為相,使在左右。}命之曰:“朝夕納誨,以輔臺德。\footnote{言當納諫誨直辭,以輔我德。朝,張遙反。}若金,用汝作礪。\footnote{鐵須礪以成利器。礪,力世反。}若濟巨川,用汝作舟楫。\footnote{渡大水待舟楫。楫音接,徐音集。}若歲大旱,用汝作霖雨。\footnote{霖,三日雨。霖以救旱。}

{\noindent\zhuan\zihao{6}\fzbyks 傳“霖,三日雨”。正義曰:隱九年\CJKunderwave{左傳}云:“凡雨自三日已往為霖。” \par}

啟乃心,沃朕心。若藥弗瞑眩,厥疾弗瘳。\footnote{開汝心,以沃我心。如服藥必瞑眩極,其病乃除。欲其出切言以自警。}


{\noindent\zhuan\zihao{6}\fzbyks 傳“開汝”至“自警”。正義曰:“瞑眩”者,令人憤悶之意也。\CJKunderwave{方言}云:“凡飲藥而毒東齊海岱間或謂之瞑,或謂之眩。”郭璞云:“瞑眩亦通語也。”然則藥之攻病,先使人瞑眩憤亂,病乃得瘳。傳言“瞑眩極”者,言悶極藥乃行也。\CJKunderwave{楚語}稱衛武公作懿以自警,“懿”即\CJKunderwave{大雅·抑}詩也。切言出於傅說,據王以為自警也。 \par}

{\noindent\shu\zihao{5}\fzkt “啟乃”至“弗瘳”。正義曰:當開汝心所有,以灌沃我心。欲令以彼所見,教己未知故也。其沃我心,須切至,若服藥不使人瞑眩憤亂,則其疾不得瘳愈。言藥毒乃得除病,言切乃得去惑也。 \par}

若跣弗視地,厥足用傷。\footnote{跣必視地,足乃無害。言欲使為已視聽。跣,先典反,徐七顯反。為,於偽反。}惟暨乃僚,罔不同心,以匡乃闢。\footnote{與汝並官,皆當倡率,無不同心以匡正汝君。闢,必亦反。}俾率先王,迪我高後,以康兆民。\footnote{言匡正汝君,使循先王之道,蹈\CJKunderline{成湯}之蹤,以安天下。}嗚呼!欽予時命,其惟有終。”\footnote{敬我是命,修其職,使有終。}說復於王曰:“惟木從繩則正,後從諫則聖。\footnote{言木以繩直,君以諫明。}後克聖,臣不命其承,\footnote{君能受諫,則臣不待命,其承意而諫之。}疇敢不祗若王之休命?”\footnote{言王如此,誰敢不敬順王之美命而諫者乎?}

\section{說命中第十三【偽】}

惟說命總百官,\footnote{在冢宰之任。總音揔。}

{\noindent\shu\zihao{5}\fzkt “惟說命總百官”。正義曰:惟此傅說,受王命總百官之職,謂在“冢宰之任”也。說以官高任重,乃進言於王,故史特標此句為發言之端也。 \par}

乃進於王曰:“嗚呼!明王奉若天道,建邦設都,\footnote{天有日月北斗五星二十八宿,皆有尊卑相正之法,言明王奉順此道,以立國設都。宿音秀。}

{\noindent\zhuan\zihao{6}\fzbyks 傳“天有”至“設都”。正義曰:\CJKunderwave{晉語}云:“大者天地,其次君臣。”\CJKunderwave{易·繫辭}云:“天垂象,見吉凶,聖人象之。”皆言人君法天以設官,順天以致治也。天有日月照臨晝夜,猶王官之伯率領諸侯也。北斗環繞北極,猶卿士之周衛天子也。五星行於列宿,猶州牧之省察諸侯也。二十八宿佈於四方,猶諸侯為天子守土也。天象皆有尊卑相正之法,言明王奉順天道以立國設都也。“立國”謂立王國及邦國,“設都”謂設帝都及諸侯國都,總言建國立家之事。 \par}

樹后王君公,承以大夫師長,\footnote{言立君臣上下,將陳為治之本,故先舉其始。王,於方反。長,丁丈反。治,直吏反,下同。}

{\noindent\shu\zihao{5}\fzkt “樹後”至“師長”。正義曰:此又總言設官分職之事也。“樹”,立也。“后王”謂天子也。“君公”謂諸侯也。“承”者奉上之名。“后王君公”,人主也。“大夫師長”,人臣也。臣當奉行君命,故以“承”言之。\CJKunderwave{周禮}立官多以“師”為名,“師”者眾所法,亦是長之義也。大夫已下,分職不同,每官各有其長,故以“師長”言之。三公則“君公”之內包之,卿則“大夫”之文兼之,“師長”之言亦通有士。將陳為治之本,故先舉其始,略言設官,故辭不詳備。為治之本,“惟天聰明”已下皆是也。 \par}

不惟逸豫,惟以亂民。\footnote{不使有位者逸豫民上,言立之主使治民。豫,羊慮反。}惟天聰明,惟聖時憲,惟臣欽若,惟民從乂。\footnote{憲,法也。言聖王法天以立教,臣敬順而奉之,民以從上為治。從,才容反。}

{\noindent\zhuan\zihao{6}\fzbyks 傳“憲法”至“為治”。正義曰:“憲,法”,\CJKunderwave{釋詁}文。人之聞見於耳目,天無形體,假人事以言也。“聰”謂無所不聞,“明”謂無所不見。惟聖人於是法天,言法天以立教,於下無不聞見,除其所惡,納之於善。雖復運有推移,道有升降,其所施為未嘗不法天也。“臣敬順而奉之”,“奉”即上文“承”也,奉承君命而布之於民。“民以從上為治”,不從上命則亂,故“從乂”也。 \par}

惟口起羞,惟甲冑起戎,\footnote{甲,鎧。胄,兜鍪也。言不可輕教令,易用兵。胄,直又反。鎧,苦代反。兜,丁侯反。鍪,莫侯反。易,以豉反。}惟衣裳在笥,惟干戈省厥躬。\footnote{言服不可加非其人,兵不可任非其才。○笥,息嗣反。省,息井反,一本作眚。}


{\noindent\zhuan\zihao{6}\fzbyks 傳“甲鎧”至“用兵”。正義曰:經傳之文無“鎧”與“兜鍪”,蓋秦漢已來始有此名,傳以今曉古也。古之甲冑皆用犀兕,未有用鐵者,而“鍪”、“鎧”之字皆從金,蓋後世始用鐵耳。口之出言為教令,甲冑興師乃用之,言不可輕教令,易用兵也。“易”亦輕也。安危在出令,令之不善,則人違背之,是“起羞”也。靜亂在用兵,伐之無罪,則人叛違之,是“起戎”也。 \par}

{\noindent\zhuan\zihao{6}\fzbyks 傳“言服”至“其才”。正義曰:“非其人”、“非其才”,義同而互文也。\CJKunderwave{周禮·大宗伯}:“以九儀之命正邦國之位,一命受職,再命受服,三命受位,四命受器,五命賜則,六命賜官,七命賜國,八命作牧,九命作伯。”鄭云:“一命始見,命為正吏。受職,治職事也。列國之士一命,王之下士亦一命。再命受服,受玄冕之服。列國之大夫再命,王之中士亦再命。”然則“再命”已上始受衣服,未賜之時在官之篋笥也。甲冑干戈俱是軍器,上言不可輕用兵,此言不可妄委人,雖文重而意異也。 \par}

{\noindent\shu\zihao{5}\fzkt “惟口”至“厥躬”。正義曰:言王者法天施化,其舉止不可不慎。惟口出令不善,以起羞辱;惟甲冑伐非其罪,以起戎兵;言不可輕教令,易用兵也。惟衣裳在篋笥,不可加非其人,觀其能足稱職,然後賜之。惟干戈在府庫,不可任非其才,省其身堪將帥,然後授之。上二句事相類,下二句文不同者,衣裳言在篋笥,干戈不言所在,干戈雲“省厥躬”,衣裳不言視其人,令其互相足也。 \par}

王惟戒茲,允茲克明,乃罔不休。\footnote{言王戒慎此四“惟”之事,信能明,政乃無不美。}惟治亂在庶官。\footnote{言所官得人則治,失人則亂。}官不及私暱,惟其能。\footnote{不加私暱,惟能是官。}\footnote{暱,女乙反。}爵罔及惡德,惟其賢。\footnote{言非賢不爵。}

{\noindent\shu\zihao{5}\fzkt “官不”至“其賢”。正義曰:\CJKunderwave{王制}云:“論定然後官之,任官然後爵之。”鄭云:“官之,使之試守也。爵之,命之也。”然則治其事謂之“官”,受其位謂之“爵”,“官”、“爵”一也,所從言之異耳。“賢”謂德行,“能”謂才用;治事必用能,故“官”雲“惟其能”;受位宜得賢,故“爵”雲“惟其賢”。\CJKunderwave{詩序}云:“任賢使能。”\CJKunderwave{周禮·鄉大夫}:“三年則大比,考其德行道藝,而與賢者能者。”鄭云:“賢者,有德行者。能者,有道藝者。”是“賢”、“能”為異耳。“私暱”謂知其不可而用之,“惡德”謂不知其非而任之,戒王使審求人,絕私好也。 \par}

慮善以動,動惟厥時。\footnote{非善非時不可動。}有其善,喪厥善。矜其能,喪厥功。\footnote{雖天子亦必讓以得之。喪,息浪反。}

{\noindent\shu\zihao{5}\fzkt “有其”至“厥功”。正義曰:人生尚謙讓而憎自取,自有其善,則人不以為善,故實善而喪其善。自誇其能,則人不以為能,故實能而喪其能。由其自取,故人不與之。“有其善”則伐善也。舜美\CJKunderline{禹}云:“汝惟不矜,天下莫與汝爭能。汝惟不伐,天下莫與汝爭功。”是言推而不有,故名反歸之也。 \par}

惟事事乃其有備,有備無患。\footnote{事事,非一事。}無啟寵納侮,\footnote{開寵非其人,則納侮之道。}

{\noindent\shu\zihao{5}\fzkt “無啟寵納侮”。正義曰:君子位高益恭,小人得寵則慢。若寵小人,則必恃寵慢主,無得開小人以寵,自納此輕侮也。“開”謂君出恩以寵臣,“納”謂臣入慢以輕王,據君而言“開”、“納”,以出、入為文也。 \par}

無恥過作非。\footnote{恥過誤而文之,遂成大非。}

{\noindent\zhuan\zihao{6}\fzbyks 傳“恥過”至“大非”。正義曰:\CJKunderline{仲虺}之美\CJKunderline{成湯}云:“改過不吝。”明小人有過,皆惜而不改。\CJKunderwave{論語}云:“小人之過也必文。”恥有過誤而更以言辭文飾之,望人不覺,其非彌甚,故“遂成大非”也。 \par}

惟厥攸居,政事惟醇。\footnote{其所居行,皆如所言,則王之政事醇粹。醇音純。粹,雖遂反。}黷於祭祀,時謂弗欽。禮煩則亂,事神則難。”\footnote{祭不欲數,數則黷,黷則不敬。事神禮煩,則亂而難行。高宗之祀特豐數近廟,故說因以戒之。黷,徒木反。數,色角反。}

{\noindent\zhuan\zihao{6}\fzbyks 傳“祭不”至“戒之”。正義曰:“祭不欲數,數則黷,黷則不敬”,\CJKunderwave{禮記·祭義}文也。此一經皆言祭祀之事,“禮煩”亦謂祭祀之煩,故傳總云:“事神禮煩,則亂而難行。”孔以\CJKunderwave{高宗肜日}祖已訓諸王“祀無豐於暱”,謂傅說此言為彼事而發,故云高宗之祀特豐數於近廟,故說因而戒之。 \par}

王曰:“旨哉!說,乃言惟服。\footnote{旨,美也。美其所言皆可服行。}乃不良於言,予罔聞於行。”\footnote{汝若不善於所言,則我無聞於所行之事。}說拜稽首,曰:“非知之艱,行之惟艱。\footnote{言知之易,行之難。以勉高宗。}王忱不艱,允協於先王成德,\footnote{王心誠不以行之為難,則信合於先王成德。忱,市林反。}惟說不言有厥咎。”\footnote{王能行善,而說不言,則有其咎罪。}

\section{說命下第十四【偽】}

王曰:“來,汝說。臺小子舊學於甘盤,\footnote{學先王之道。甘盤,殷賢臣有道德者。臺音怡。}

{\noindent\shu\zihao{5}\fzkt “王曰”至“甘盤”。正義曰:“舊學於甘盤”謂為王子時也。\CJKunderwave{君奭}篇周公仰陳殷之賢臣云:“在武丁,時則有若甘盤。”然則甘盤於高宗之時有大功也。上篇高宗免喪不言,即求傅說,似得說時無賢臣矣。蓋甘盤於小乙之世以為大臣,小乙將崩,受遺輔政,高宗之初得有大功。及高宗免喪,甘盤已死,故\CJKunderwave{君奭}傳曰:“高宗即位,甘盤佐之,後有傅說。”是言傅說之前有甘盤也。但下句言“既乃遯於荒野”,是學訖乃遁,非即位之初從甘盤學也。 \par}

既乃遁於荒野,入宅於河。\footnote{既學而中廢業,遁居田野。河,洲也。其父欲使高宗知民之艱苦,故使居民間。遁,徒頓反。}

{\noindent\zhuan\zihao{6}\fzbyks 傳“既學”至“民間”。正義曰:“河”是水名,水不可居,而云“入宅於河”,知在河之洲也。\CJKunderwave{釋水}云:“水中可居者曰洲。”初遁田野,後入河洲,言其徙居無常也。\CJKunderwave{無逸}云:“其在高宗,時舊勞於外,爰暨小人。”言“其父欲使高宗知民之艱苦,故使居民間”也。於時蓋未為太子,殷道雖質,不可既為太子,更得與民雜居。 \par}

自河徂亳,暨厥終罔顯。\footnote{自河往居亳,與今其終,故遂無顯明之德。}爾惟訓於朕志,\footnote{言汝當教訓於我,使我志通達。}若作酒醴,爾惟曲糵;\footnote{酒醴須曲糵以成,亦言我須汝以成。曲,起六反。糵,魚列反。}若作和羹,爾惟鹽梅。\footnote{鹽,咸。梅,醋。羹須咸醋以和之。羹音庚,一音衡。鹽,餘廉反。梅亦作楳。醋,七故反。和如字,又胡臥反。}爾交修予,罔予棄,予惟克邁乃訓。”\footnote{交,非一之義。邁,行也。言我能行汝教。}

{\noindent\zhuan\zihao{6}\fzbyks 傳“交非”至“汝教”。正義曰:“爾交修予”,令其交更修治己也。故以“交”為“非一之義”,言交互教之,非一事之義。“邁,行”,\CJKunderwave{釋詁}文。 \par}

說曰:“王,人求多聞,時惟建事,學於古訓,乃有獲。\footnote{王者求多聞以立事,學於古訓,乃有所得。}事不師古,以克永世,匪說攸聞。\footnote{事不法古訓而以能長世,非說所聞。言無是道。}惟學遜志,務時敏,厥修乃來。\footnote{學以順志,務是敏疾,其德之修乃來。}

{\noindent\shu\zihao{5}\fzkt “惟學”至“乃來”。正義曰:人志本欲求善,欲學順人本志,學能務是敏疾,則其德之修乃自來。言務之既疾,則德自來歸己也。 \par}

允懷於茲,道積於厥躬。\footnote{信懷此學志,則道積於其身。}惟斆學半,念終始典於學,厥德修罔覺。\footnote{斆,教也。教然後知所困,是學之半。終始常念學,則其德之修,無能自覺。斆,戶孝反。}

{\noindent\shu\zihao{5}\fzkt “惟斆”至“罔覺”。正義曰:教人然後知困,知困必將自強,惟教人乃是學之半,言其功半於學也。於學之法,念終念始,常在於學,則其德之修漸漸進益,無能自學其進。言日有所益,不能自知也。 \par}

監於先王成憲,其永無愆。\footnote{愆,過也。視先王成法,其長無過,其惟學乎!愆,起虔反。}惟說式克欽承,旁招俊乂,列於庶位。”\footnote{言王能志學,說亦用能敬承王志,廣招俊乂,使列眾官。俊,本又作畯。}王曰:“嗚呼!說,四海之內,咸仰朕德,時乃風。\footnote{風,教也。使天下皆仰我德,是汝教。仰如字,徐五亮反。}股肱惟人,良臣惟聖。\footnote{手足具,乃成人。有良臣,乃成聖。}昔先正保衡,作我先王,\footnote{保衡,\CJKunderline{伊尹}也。作,起。正,長也。言先世長官之臣。長,丁丈反,下同。}

{\noindent\zhuan\zihao{6}\fzbyks 傳“保衡”至“之臣”。正義曰:保衡、阿衡俱\CJKunderline{伊尹}也。\CJKunderwave{君奭}傳曰:“\CJKunderline{伊尹}為保衡,言天下所取安所取平也。”鄭箋云:“阿,倚。衡,平也。\CJKunderline{伊尹}湯所依倚而取平也,故以為官名。”又云:“\CJKunderline{太甲}時曰保衡。”鄭不見古文\CJKunderwave{太甲}雲“不惠於阿衡”,故此為解,孔所不用。計此阿衡、保衡非常人之官名,蓋當時特以此名號\CJKunderline{伊尹}也。“作”訓為起,言起而助湯也。“正,長”,\CJKunderwave{釋詁}文。 \par}

乃曰:‘予弗克俾厥後惟堯舜,其心愧恥,若撻於市。”\footnote{言\CJKunderline{伊尹}不能使其君如堯舜,則恥之,若見撻於市,故成其能。俾,必爾反。撻,他達反。}一夫不獲,則曰時予之辜。\footnote{\CJKunderline{伊尹}見一夫不得其所,則以為己罪。}佑我烈祖,格於皇天。\footnote{言以此道左右\CJKunderline{成湯},功至大天,無能及者。}爾尚明保予,罔俾阿衡,專美有商。\footnote{汝庶幾明安我事,則與\CJKunderline{伊尹}同美。阿,烏何反。}惟後非賢不乂,惟賢非後不食。\footnote{言君須賢治,賢須君食。治,直吏反。}其爾克紹乃闢於先王,永綏民。”\footnote{能繼汝君於先王,長安民,則汝亦有保衡之功。闢,必亦反。}說拜稽首,曰:“敢對揚天子之休命。”\footnote{對,答也。答受美命而稱揚之。}

\section{高宗肜日第十五}


高宗祭\CJKunderline{成湯},有飛雉升鼎耳而雊,\footnote{耳不聰之異。雊,鳴。雊,工豆反。}祖已訓諸王,\footnote{賢臣也,以訓道諫王。己音紀。}作\CJKunderwave{高宗肜日}、\CJKunderwave{高宗之訓}。\footnote{所以訓也,亡。肜音融。}


{\noindent\zhuan\zihao{6}\fzbyks 傳“耳不”至“雊鳴”。正義曰:經言“肜日,有雊雉”,不知祭何廟,鳴何處,故序言“祭\CJKunderline{成湯}”、“升鼎耳”以足之。禘祫與四時之祭,祭之明日皆為肜祭,不知此肜是何祭之肜也。\CJKunderwave{洪範}“五事”有貌、言、視、聽、思,若貌不恭、言不從、視不明、聽不聰、思不睿,各有妖異興焉。雉乃野鳥,不應入室,今乃入宗廟之內,升鼎耳而鳴,孔以雉鳴在鼎耳,故以為“耳不聰之異”也。\CJKunderwave{洪範·五行傳}云:“視之不明,時則有羽蟲之孽。聽之不聰,時則有介蟲之孽。”言之不從,時則有毛蟲之孽。貌之不恭,時則有鱗蟲之孽。思之不睿,時則有倮蟲之孽。”先儒多以此為羽蟲之孽,非為“耳不聰”也。\CJKunderwave{漢書·五行志}:“劉歆以為鼎三足,三公象也,而以耳行。野鳥居鼎耳,是小人將居公位,敗宗廟之祀也。”鄭云:“鼎,三公象也,又用耳行,雉升鼎耳而鳴,象視不明,天意若雲當任三公之謀以為政。”劉、鄭雖小異,其為羽蟲之孽則同,與孔意異。\CJKunderwave{詩}云:“雉之朝雊,尚求其雌。”\CJKunderwave{說文}云:“雊,雄雉鳴也。雷始動,雉乃鳴而雊其頸。” \par}

{\noindent\zhuan\zihao{6}\fzbyks 傳“所以訓也,亡”。正義曰:名\CJKunderwave{高宗之訓},所以訓高宗也。此二篇俱是祖己之言,並是訓王之事,經雲“乃訓於王”,此篇亦是訓也。但所訓事異,分為二篇,標此為發言之端,故以“肜日”為名。下篇總諫王之事,故名之“訓”,終始互相明也。\CJKunderwave{肆命}、\CJKunderwave{徂後},孔歷其名於\CJKunderwave{伊訓}之下,別為之傳。此\CJKunderwave{高宗之訓}因序為傳,不重出名者,此以訓王事同,因解文便作傳,不為例也。 \par}

{\noindent\shu\zihao{5}\fzkt “高宗”至“之訓”。正義曰:高宗祭其太祖\CJKunderline{成湯}於肜祭之日,有飛雉來升祭之鼎耳而雊鳴,其臣祖已以為王有失德而致此祥,遂以道義訓王,勸王改修德政。史敘其事,作\CJKunderwave{高宗肜日}、\CJKunderwave{高宗之訓}二篇。 \par}

高宗肜日\footnote{祭之明日又祭。殷曰肜,周曰繹。○繹音亦,字書作釋。\CJKunderwave{爾雅}云:“又祭也。周曰繹,商曰肜,夏曰復胙。”}

{\noindent\zhuan\zihao{6}\fzbyks 傳“祭之”至“曰繹”。正義曰:\CJKunderwave{釋天}云:“繹,又祭也。周曰繹,商曰肜。”孫炎曰:“祭之明日尋繹復祭也。”“肜”者,相尋不絕之意。\CJKunderwave{春秋}宣八年六月“辛巳,有事於太廟。壬午,猶繹”。\CJKunderwave{穀梁傳}曰:“繹者,祭之旦日之享賓也。”是肜者,“祭之明日又祭”也。\CJKunderwave{爾雅}因繹祭而本之上世,故先周後商,此以上代先後,故與\CJKunderwave{爾雅}倒也。\CJKunderwave{釋天}又云“夏曰復胙”,郭璞雲“未見所出”,或無此一句。孔傳不言“夏曰復胙”,於義非所須,或本無此事也。\CJKunderwave{儀禮·有司徹}上大夫曰“儐屍”,與正祭同日。鄭康成注\CJKunderwave{詩·鳧鷖}云,祭天地社稷山川,五祀皆有繹祭。 \par}

高宗肜日,越有雊雉。\footnote{於肜日有雉異。}祖己曰:“惟先格王,正厥事。”\footnote{言至道之王遭變異,正其事而異自消。}


{\noindent\zhuan\zihao{6}\fzbyks 傳“言至”至“自消”。正義曰:“格”訓至也。“至道之王”謂用心至極,行合於道。遭遇變異,改修德教,正其事而異自消。\CJKunderline{大戊}拱木,武丁雊雉,皆感變而懼,殷道復興,是異自消之驗也。至道之王,當無災異,而云遭變消災者,天或有譴告,使之至道,未必為道不至而致此異。且匆尋戒之辭,不可執文以害意也。此經直雲“祖己曰”,不知與誰語,鄭雲“謂其黨”,王肅雲“言於王”。下句始言“乃訓於王”,此句未是告王之辭,私言告人,鄭說是也。 \par}

{\noindent\shu\zihao{5}\fzkt “高宗”至“厥事”。正義曰:高宗既祭\CJKunderline{成湯},肜祭之日,於是有雊鳴之雉在於鼎耳,此乃怪異之事。賢臣祖已見其事而私自言曰:“惟先世至道之王遭遇變異,則正其事而異自消也。”既作此言,乃進言訓王。史錄其事,以為訓王之端也。 \par}

乃訓於王。曰:“惟天監下民,典厥義。\footnote{祖己既言,遂以道訓諫王,言天視下民,以義為常。}降年有永有不永,非天夭民,民中絕命。\footnote{言天之下年與民,有義者長,無義者不長,非天欲夭民,民自不修義以致絕命。中,丁仲反,又如字。}民有不若德,不聽罪。天既孚命正厥德。\footnote{不順德,言無義。不服罪,不改修。天已信命正其德,謂有永有不永。}


{\noindent\zhuan\zihao{6}\fzbyks 傳“言天”至“絕命”。正義曰:經惟言“有永有不永”,安和由義者?以上句雲“惟天監下民,典厥義”,天既以義為常,知命之長短莫不由義,故云“天之下年與民,有義者長。無義者不長”也。民有五常之性,謂仁義禮智信也,此獨以“義”為言者,五常指體則別,理亦相通;“義”者,宜也,得其事宜;五常之名,皆以適宜為用,故稱“義”可以總之也。民有貴賤貧富愚智好醜,不同多矣,獨以夭壽為言者,\CJKunderline{鄭玄}云:“年命者,蠢愚之人尤愒焉,故引以諫王也。”愒,貪也。\CJKunderwave{洪範}“五福”以壽為首,“六極”以短折為先,是年壽者最是人之所貪,故祖己引此以諫王也。 \par}

{\noindent\zhuan\zihao{6}\fzbyks 傳“不順”至“不永”。正義曰:傳亦顧上經,故“不順德,言無義”也。“聽”謂聽從,故以“不聽”為“不服罪”。言既為罪,過而不肯改修也。“天已信命正其德”,言天自信命,賞有義,罰無義,此事必信也。天自正其德,福善禍淫,其德必不差也。謂民有永有不永,天隨其善惡而報之。勸王改過修德以求永也。 \par}

{\noindent\shu\zihao{5}\fzkt “乃訓”至“厥德”。正義曰:祖己既私言其事,乃以道訓諫於王曰:“惟天視此下民,常用其義。”言以義視下,觀其為義以否。“其下年與民,有長者,有不長者”。言與為義者長,不義者短。“短命者非是天欲夭民,民自不修義,使中道絕其性命。但人有為行不順德義,有過不服聽罪,過而不改,乃致天罰,非天欲夭之也。天既信行賞罰之命,正其馭民之德,欲使有義者長,不義者短,王安得不行義事,求長命也?” \par}

乃曰:‘其如臺。’\footnote{祖己恐王未受其言,故乃復曰,天道其如其所言。臺音怡。復,扶又反。}嗚呼!王司敬民,罔非天胤,典祀無豐於暱。”\footnote{胤,嗣。暱,近也。嘆以感王入其言,王者主民,當敬民事。民事無非天所嗣常也,祭祀有常,不當特豐於近廟。欲王因異服罪改修之。豐,芳弓反。暱,女乙反。\CJKunderwave{屍子}云:“不避遠暱。”暱,近也。又乃禮反。馬云:“暱,考也,謂禰廟也。”}


{\noindent\zhuan\zihao{6}\fzbyks 傳“胤嗣”至“改修之”。正義曰:\CJKunderwave{釋詁}云:“胤、嗣,繼也。”俱訓為繼,是“胤”德為嗣,嗣亦繼之義也。\CJKunderwave{釋詁}云:“即,尼也。”孫炎曰:“即猶今也,尼者近也。”郭璞引\CJKunderwave{屍子}曰“悅尼而來遠”,是“尼”為近也。“尼”與“暱”音義同。烝民不能自治,自立君以主之,是“王者主民”也。既與民為主,當敬慎民事。民事無大小,無非天所嗣常也,言天意欲令繼嗣行之,所以為常道也。“祭祀有常”,謂犧牲粢盛尊彝俎豆之數禮有常法。“不當特豐於近廟”,謂犧牲禮物多也。祖己知高宗豐於近廟,欲王因此雊雉之異,服罪改修以從禮耳,其異不必由豐近而致之也。王肅亦云:“高宗豐於禰,故有雊雉升遠祖\CJKunderline{成湯}廟鼎之異。” \par}

{\noindent\shu\zihao{5}\fzkt “嗚呼”至“於暱”。正義曰:祖己恐其言不入王意,又嘆而戒之:“嗚呼!王者主民,當謹敬民事。民事無非天所繼嗣以為常道者也。天以其事為常,王當繼天行之。祀禮亦有常,無得豐厚於近廟,若特豐於近廟,是失於常道。”高宗豐於近廟,欲王服罪改修也。 \par}

\section{西伯戡黎第十六}


殷始咎周,\footnote{咎,惡。○咎,其九反,馬云:“咎周者,為周所咎。}周人乘黎。\footnote{乘,勝也。所以見惡。黎,力兮反,國名,\CJKunderwave{尚書大傳}作耆。}祖伊恐,\footnote{祖己後賢臣。}告於受,\footnote{受,紂也,音相亂。帝乙之子,嗣立,暴虐無道。○受如字,傳云:“受,紂也。音相亂。”馬云:“受讀曰紂。或曰受婦人之言,故號曰受也。”}作\CJKunderwave{西伯戡黎}。\footnote{戡亦勝也。○伯亦作柏。戡音堪,\CJKunderwave{說文}作𢦟,雲“殺也”。以此戡訓刺,音竹甚反。勝,詩證反。}

{\noindent\zhuan\zihao{6}\fzbyks 傳“咎惡”,又云“乘勝”至“見惡”。正義曰:\CJKunderwave{易·繫辭}雲“無咎者善補過也”,則“咎”是過之別名,以彼過而憎惡之,故“咎”為惡也。以其勝黎,所以見惡,釋其見惡之由,是周人勝黎之後始惡之。\CJKunderwave{詩毛傳}云:“乘,陵也。”乘駕是加陵之意,故“乘”為勝也。\CJKunderline{鄭玄}云:“紂聞文王斷虞芮之訟,又三伐皆勝,而始畏惡之。”所言據\CJKunderwave{書傳}為說,伏生\CJKunderwave{書傳}雲“文王受命,一年斷虞芮之質,二年伐邘,三年伐密須,四年伐犬夷,五年伐耆,六年伐崇,七年而崩”。耆即黎也。乘黎之前始言惡周,故鄭以伐邘、伐密須、伐犬夷三伐皆勝,始畏惡之。\CJKunderwave{武成}篇文王“誕膺天命”,九年乃崩,則伐國之年不得如\CJKunderwave{書傳}所說,未必見三伐皆勝始畏之。 \par}

{\noindent\zhuan\zihao{6}\fzbyks 傳“祖己後賢臣”。正義曰:此無所出,正以同為祖氏,知是其後,明能先覺,故知賢臣。 \par}

{\noindent\zhuan\zihao{6}\fzbyks 傳“受紂”至“無道”。正義曰:經雲“奔告於王”,王無諡號,故序言“受”以明之。此及\CJKunderwave{泰誓}、\CJKunderwave{武成}皆呼此君為“受”,自外書傳皆呼為“紂”。“受”即“紂”也,音相亂,故字改易耳。\CJKunderwave{殷本紀}云:“帝乙崩,子辛立,是為帝辛,天下謂之紂。”\CJKunderline{鄭玄}云:“紂,帝乙之少子,名辛。帝乙愛而欲立焉,號曰受德,時人傳聲轉作紂也。”史掌書,知其本,故曰“受”,與孔大同。\CJKunderwave{諡法}云:“殘義損善曰紂。”殷時未有諡法,後人見其惡,為作惡義耳。 \par}

{\noindent\zhuan\zihao{6}\fzbyks 傳“戡亦勝也”。正義曰:“戡,勝”,\CJKunderwave{釋詁}文。孫炎曰:“戡,強之勝也。” \par}

{\noindent\shu\zihao{5}\fzkt “殷始”至“戡黎”。正義曰:文王功業稍高,王兆漸著,殷之朝廷之臣始畏惡周家。所以畏惡之者,以周人伐而勝黎邑故也。殷臣祖伊見周克黎國之易,恐其終必伐殷,奔走告受,言殷將滅。史敘其事,作\CJKunderwave{西伯戡黎}。 \par}

西伯戡黎

西伯既戡黎,\footnote{近王圻之諸侯,在上黨東北。近,附近之近。圻,巨依反。}


{\noindent\zhuan\zihao{6}\fzbyks 傳“近王”至“東北”。正義曰:黎國,漢之上黨郡壺關所治黎亭是也。紂都朝歌,王圻千里,黎在朝歌之西,故為“近王圻之諸侯”也。鄭云:“入紂圻內。”文王猶尚事紂,不可伐其圻內。所言“圻內”,亦無文也。 \par}

{\noindent\shu\zihao{5}\fzkt “西伯戡黎”。正義曰:\CJKunderline{鄭玄}云:“西伯,周文王也。時國於岐,封為雍州伯也。國在西,故曰西伯。”王肅云:“王者中分天下,為二公總治之,謂之二伯,得專行征伐,文王為西伯。黎侯無道,文王伐而勝之。”兩說不同,孔無明解。下傳雲“文王率諸侯以事紂”,非獨率一州之諸侯也。\CJKunderwave{論語}稱“三分天下有其二,以服事殷”,謂文王也。終乃三分有二,豈獨一州牧乎?且言“西伯”對東為名,不得以國在西而稱“西伯”也,蓋同王肅之說。 \par}

祖伊恐,奔告於王。曰:“天子,天既訖我殷命,\footnote{文王率諸侯以事紂,內秉王心,紂不能制,今又克有黎國,迫近王圻,故知天已畢訖殷之王命。言將化為周。王心,於況反,下注“宜王者”同。}

{\noindent\zhuan\zihao{6}\fzbyks 傳“文王”至“為周”。正義曰:襄四年\CJKunderwave{左傳}云:“文王率殷之叛國以事紂。”是率諸侯共事紂也。貌雖事紂,內秉王心,佈德行威,有將王之意。而紂不能制,日益強大,今復克有黎國,迫近王圻,似有天助之力,故云“天已畢訖殷之王命”,言殷祚至此而畢,將欲化為周也。 \par}

格人元龜,罔敢知吉。\footnote{至人以人事觀殷,大龜以神靈考之,皆無知吉。}

{\noindent\zhuan\zihao{6}\fzbyks 傳“至人”至“知吉”。正義曰:“格”訓為至,“至人”謂至道之人,有所識解者也。至人以人事觀殷,大龜有神靈逆知來物,故“大龜以神靈考之”。二者皆無知殷有吉者,言必兇也。祖伊未必問至人,親灼龜,但假之以為言耳。 \par}

非先王不相我後人,惟王淫戲用自絕。\footnote{非先祖不助子孫,以王淫過戲怠,用自絕於先王。相,息亮反。}故天棄我,不有康食。不虞天性,不迪率典。\footnote{以紂自絕於先王,故天亦棄之,宗廟不有安食於天下。而王不度知天性命所在,而所行不蹈循常法。言多罪。度,待洛反。}

{\noindent\zhuan\zihao{6}\fzbyks 傳“以紂”至“多罪”。正義曰:\CJKunderwave{禮記}稱“萬物本於天,人本於祖”,則天與先王俱是人君之本。紂既自絕於先王,亦自絕於天。上經言紂自絕先王,此言天棄紂,互明紂自絕,然後天與先王棄絕之。故傳申通其意,“以紂自絕先王,故天亦棄之”。“亦”者,亦先王,言先王與天俱棄之也。\CJKunderwave{孝經}言天子得萬國之歡心,以事其先王,然後祭則鬼享之。今紂既自絕於先王,先王不有安食於天下,言紂雖以天子之尊事宗廟,宗廟之神不得安食也。而王不度知天命所在,不知已之性命當盡也,而所行不蹈循常法,動悉違法,言多罪。 \par}

今我民罔弗欲喪,曰:‘天曷不降威?大命不摯?’今王其如臺。”\footnote{摯,至也。民無不欲王之亡,言:“天何不下罪誅之?有大命宜王者,何以不至?”王之兇害,其如我所言。摯音至,本又作𡠗。}

{\noindent\zhuan\zihao{6}\fzbyks 傳“摯至也”至“所言”。正義曰:摯”、“至”同音,故“摯”為至也。“言天何不下罪誅之”,恨其久行虐政,欲得早殺之也。“有大命宜王者,何以不至”,向望大聖之君,欲令早伐紂也。“王之兇禍,其如我之所言”,以王不信,故審告之也。 \par}

王曰:“嗚呼!我生不有命在天?”\footnote{言我生有壽命在天,民之所言,豈能害我。遂惡之辭。}祖伊反曰:“嗚呼!乃罪多參在上,乃能責命於天?”\footnote{反,報紂也,言汝罪惡眾多,參列於上天,天誅罰汝,汝能責命於天,拒天誅乎?參,七南反,馬云:“參字累在上。”}殷之即喪,指乃功,不無戮於爾邦。”\footnote{言殷之就亡,指汝功事所致,汝不得無死戮於殷國,必將滅亡,立可待。}

\section{微子第十七}


殷既錯天命,\footnote{錯,亂也。錯,七各反,馬云:“廢也。”}微子作誥父師、少師。\footnote{告二師而去紂。少,詩照反。}


{\noindent\zhuan\zihao{6}\fzbyks 傳“錯,亂也”。正義曰:交錯是渾亂之義,故為亂也。不指言紂惡而言“錯亂天命”者,天生烝民,立君以牧之,為君而無君道,是錯亂天命,為惡之大,故舉此以見惡之極耳。 \par}

{\noindent\shu\zihao{5}\fzkt “殷既”至“少師”。正義曰:殷紂既暴虐無道,錯亂天命,其兄微子知紂必亡,以作言誥告父師箕子、少師比干。史敘其事而作此篇也。名曰\CJKunderwave{微子}而不言“作\CJKunderwave{微子}”者,已言“微子作誥”,以可知而省文也。 \par}

微子\footnote{微,圻內國名。子,爵。為紂卿士,去無道。}

{\noindent\zhuan\zihao{6}\fzbyks 傳“微圻”至“無道”。正義曰:微國在圻內,先儒相傳為然。\CJKunderline{鄭玄}以為微與箕俱在圻內,孔雖不言箕,亦當在圻內也。王肅云:“微,國名。子,爵。入為王卿士。”肅意蓋以微為圻外,故言“入”也。微子名啟,\CJKunderwave{世家}作開,避漢景帝諱也。啟與其弟仲衍,皆是紂之同母庶兄,\CJKunderwave{史記}稱“微仲衍”。衍亦稱“微”者,微子封微,以微為氏,故弟亦稱微,猶如春秋之世虞公之弟稱虞叔,祭公之弟稱祭叔。微子若非大臣,則無假憂紂,亦不必須去,以此知其為卿士也。傳雲“去無道”者,以“去”見其為卿士也。 \par}

微子若曰:“父師、少師,\footnote{父師,太師,三公,箕子也。少師,孤卿,比干。微子以紂距諫,知其必亡,順其事而言之。}殷其弗或亂正四方。\footnote{或,有也。言殷其不有治正四方之事,將必亡。治,直吏反。}我祖底遂陳於上,\footnote{言湯致遂其功,陳列於上世。}我用沈酗於酒,用亂敗厥德於下。\footnote{我,紂也。沈湎酗醟,敗亂湯德於後世。沈,徐直金反。酗,況具反,以酒為兇曰酗,\CJKunderwave{說文}作䣱,云:“酒醟。”湎,面善反。醟音詠,\CJKunderwave{說文}於命反,䣱酒也。}


{\noindent\zhuan\zihao{6}\fzbyks 傳“父師”至“言之”。正義曰:以\CJKunderwave{畢命}之篇王呼畢公為“父師”,畢公時為太師也。\CJKunderwave{周官}云:“太師、太傅、太保,茲惟三公。少師、少傅、少保曰三孤。”\CJKunderwave{家語}云:“比干官則少師。”少師是比干,知太師是箕子也。遍檢書傳,不見箕子之名,惟司馬彪注\CJKunderwave{莊子}云:“箕子名胥餘。”不知出何書也。\CJKunderwave{周官}以少師為孤,此傳言“孤卿”者,孤亦卿也,\CJKunderwave{考工記}曰“外有九室,九卿朝焉”,是三孤六卿共為九卿也。比干不言封爵,或本無爵,或有而不言也。\CJKunderwave{家語}云:“比干是紂之親,則諸父。”知比干是紂之諸父耳。箕子則無文。\CJKunderwave{宋世家}云:“箕子者,紂親戚也。”止言親戚,不知為父為兄也。\CJKunderline{鄭玄}、王肅皆以箕子為紂之諸父,服虔、杜預以為紂之庶兄,既無正文,各以意言之耳。微子以紂距諫,知其必亡,心欲去之,故順其去事而言,呼二師以告之。 \par}

{\noindent\zhuan\zihao{6}\fzbyks 傳“或有”至“必亡”。正義曰:“或”者不定之辭,其事欲當然,則是有此事,故以“或”為有也。\CJKunderline{鄭玄}\CJKunderwave{論語注}亦云:“或之言有也,不有言無也。”天子,天下之主,所以治正四方,“言殷其不有治正四方之事”,言將必亡。 \par}

{\noindent\zhuan\zihao{6}\fzbyks 傳“我紂”至“後世”。正義曰:嗜酒亂德,是紂之行,故知“我”,我紂也。人以酒亂,若沈於水,故以耽酒為“沈”也。湎然是齊同之意,\CJKunderwave{詩}云:“天不湎爾以酒。”鄭云:“天不同汝顏色以酒。”是“湎”謂酒變面色,湎然齊同,無復平時之容也。\CJKunderwave{說文}云:“酗,醟也。”然則“酗”、“醟”一物,謂飲酒醉而發怒。經言亂敗其德,必有所屬,上言“我祖”指謂\CJKunderline{成湯},知言“敗亂湯德於後世”也。上謂前世,故下為後世也。 \par}

殷罔不小大,好草竊奸宄。\footnote{草野竊盜,又為奸宄於內外。好,呼報反。宄音軌。}卿士師師非度,凡有辜罪,乃罔恆獲。\footnote{六卿典士相師效,為非法度,皆有辜罪,無秉常得中者。度如字。}小民方興,相為敵讎。\footnote{卿士既亂,而小人各起一方,共為敵讎。言不和同。讎,常周反。}今殷其淪喪,若涉大水,其無津涯。\footnote{淪,沒也。言殷將沒亡,如涉大水,無涯際,無所依就。淪音倫,徐力允反。喪,息浪反。涯,五皆反,又宜佳反。}殷遂喪,越至於今。”\footnote{言遂喪亡於是,至於今,到不待久。}

{\noindent\zhuan\zihao{6}\fzbyks 傳“六卿”至“中者”。正義曰:“士”訓事也,故“卿士”為“六卿典事”。“師師”言相師效為非法度之事也。止言“卿士”,以貴者尚爾,見賤者皆然。故王肅云:“卿士以下,轉相師效為非法度之事也。”鄭云:“凡猶皆也。”傳意亦然,以“凡”為皆,言卿士以下在朝之臣,其所舉動皆有辜罪,無人能秉常行得中正者。 \par}

{\noindent\shu\zihao{5}\fzkt “微子”至“於今”。正義曰:微子將欲去殷,順其去事而言,曰“父師”、“少師”,呼二師與之言也。今殷國其將不復有治正四方之事,言其必滅亡也。昔我祖\CJKunderline{成湯},致行其道,遂其功業,陳列於上世矣。今我紂惟用沈湎酗E1於酒,用是亂敗其祖之德於下。由紂亂敗之故,今日殷人無不小大皆好草竊奸宄。雖在朝卿士,相師師為非法度之事。朝廷之臣皆有辜罪,乃無有一人能秉常得中者。在外小人,方方各起,相與共為敵讎。荒亂如此,今殷其沒,亡若涉大水,其無津濟涯岸。殷遂喪亡,言不復久也。“此喪亡於是,至於今,到必不得更久也”。 \par}

曰:“父師、少師,我其發出狂,吾家耄遜於荒。\footnote{我念殷亡,發疾生狂,在家耄亂,故欲遯出於荒野。言愁悶。出,尺遂反。耄,字又作旄,莫報反,注同。遯,徒困反,徐徒頓反,一音都困反。}今爾無指,告予顛隮,若之何其?”\footnote{汝無指意告我殷邦顛隕隮墜,如之何其救之?隮,子細反,\CJKunderwave{玉篇}子兮反,\CJKunderwave{切韻}祖稽反。隕,于敏反。}


{\noindent\zhuan\zihao{6}\fzbyks 傳“我念”至“愁悶”。正義曰:狂生於心而出於外,故傳以“出狂”為“生狂”。應璩詩云“積念發狂痴”,此其事也。在家思念之深,精神益以耄亂。\CJKunderline{鄭玄}云:“耄,昏亂也。”在家不堪耄亂,故欲遯出於荒野,言愁悶之至。\CJKunderwave{詩}云:“駕言出遊,以寫我憂。”亦此意也。 \par}

{\noindent\zhuan\zihao{6}\fzbyks 傳“汝無”至“救之”。正義曰:“無指意告我者”,謂無指殷亡之事告我,言殷將隕墜,欲留我救之。“顛”謂從上而隕,“隮”謂墜於溝壑,皆滅亡之意也。昭十三年\CJKunderwave{左傳}曰:“小人老而無子,知隮於溝壑矣。”王肅云:“隮,隮溝壑。”言此“隮”之義如\CJKunderwave{左傳}也。 \par}

{\noindent\shu\zihao{5}\fzkt “曰父師”至“何其”。正義曰:微子既言紂亂,乃問身之所宜,止而復言,故別加一“曰父師少師”,更呼而告之也。“我念殷亡之故,其心發疾生狂,吾在家心內耄亂,欲遜遯出於荒野。今汝父師少師無指滅亡之意告我云,殷邦其隕墜,則當如之何其救之乎?”恐其留己共救之也。 \par}

父師若曰:“王子,\footnote{比干不見,明心同,省文。微子帝乙元子,故曰王子。見,賢遍反。省,所景反。}天毒降災荒殷邦,方興沈酗於酒,\footnote{天生紂為亂,是天毒下災,四方化紂沈湎,不可如何。}乃罔畏畏,咈其耇長舊有位人。\footnote{言起沈湎,上不畏天災,下不畏賢人。違戾耇老之長致仕之賢,不用其教,法紂故。咈,扶勿反。耇,工口反。長,丁丈反,注同。}


{\noindent\zhuan\zihao{6}\fzbyks 傳“比干”至“王子”。正義曰:諮二人而一人答,“明心同,省文”也。鄭云:“少師不答,志在必死。”然則箕子本意豈必求生乎?身若求生,何以不去?既“不顧行遁”,明期於必死,但紂自不殺之耳。若比干意異,箕子則別有答,安得默而不言?孔解“心同”是也。“微子帝乙元子”,\CJKunderwave{微子之命}有其文也。父師言微子為“王子”,則父師非王子矣,鄭、王等以為紂之諸父當是實也。 \par}

{\noindent\zhuan\zihao{6}\fzbyks 傳“天生”至“如何”。正義曰:“荒殷邦”者,乃是紂也,而云“天毒降災”,故言“天生紂為亂”,本之於天,天毒下災也。以微子云“若之何”,此答彼意,故言“四方化紂沈湎,不可如何”。 \par}

{\noindent\zhuan\zihao{6}\fzbyks 傳“言起”至“紂故”。正義曰:文在“方興沈酗”之下,則此無所畏畏者,謂當時四方之民也。民所當畏,惟畏天與人耳,故知二畏者,上不畏天,下不畏賢人。違戾耇長與舊有位人,即是不畏賢人,故不用其教,紂無所畏,此民無所畏,謂法紂故也。 \par}

今殷民乃攘竊神祇之犧牷牲用,以容將食,無災。\footnote{自來而取曰攘。色純曰犧。體完曰牷。牛羊豕曰牲。器實曰用。盜天地宗廟牲用,相容行食之,無災罪之者。言政亂。攘,如羊反,因來而取曰攘。竊,馬云:“往盜曰竊。”神祇,天曰神,地曰祇。犧,許宜反。牷音全。}降監殷民,用乂讎斂,召敵讎不怠。\footnote{下視殷民,所用治者,皆重賦傷民、斂聚怨讎之道,而又亟行暴虐,自召敵讎不解怠。讎如字,下同。徐云:“鄭音疇。”馬本作稠,云:“數也。”豔,力檢反;馬、鄭力豔反,謂賦斂也;徐云:“鄭力劍反。”治,直吏反。亟,欺忌反,數也;又紀力反;本又作極,如字,至也。解,佳賣反。}罪合於一,多瘠罔詔。\footnote{言殷民上下有罪,皆合於一法紂,故使民多瘠病,而無詔救之者。瘠,在益反。}商今其有災,我興受其敗。\footnote{災滅在近,我起受其敗,言宗室大臣義不忍去。}


{\noindent\zhuan\zihao{6}\fzbyks 傳“自來”至“政亂”。正義曰:“攘”、“竊”同文,則“攘”是竊類。\CJKunderwave{釋詁}云:“攘,因也。”是因其自來而取之名“攘”也。\CJKunderwave{說文}云:“犧,宗廟牲也。”\CJKunderwave{曲禮}云:“天子以犧牛。”天子祭牲必用純色,故知“色純曰犧”也。\CJKunderwave{周禮}:“牧人掌牧六牲,以供祭祀之牲牷。”以“牷”為言,必是體全具也,故“體完曰牷”。經傳多言“三牲”,知“牲”是牛羊豕也。以“犧”、“牷”、“牲”三者既為俎實,則“用”者簠簋之實,謂黍稷稻粱,故云“器實曰用”,謂粢盛也。\CJKunderwave{禮}“天曰神,地曰祗”,舉天地則人鬼在其間矣,故總雲“盜天地宗廟牲用”也。訓“將”為行,“相容行食之”謂所司相通容,使盜者得行盜而食之。大祭祀之物,物之重者,盜而無罪,言政亂甚也。漢魏以來著律皆云:“敢盜郊祀宗廟之物,無多少皆死。”為特重故也。 \par}

{\noindent\zhuan\zihao{6}\fzbyks 傳“下視”至“懈怠”。正義曰:箕子身為三公,下觀世俗,故云“下視殷民”。“所用治者”謂卿士已下是治民之官也。以紂暴虐,務稱上旨,“皆重賦傷民”。民既傷矣,則以上為讎,\CJKunderwave{泰誓}所謂“虐我則讎”是也。重斂民財,乃是“聚斂怨讎之道”。既為重斂,而又亟行暴虐。亟,急也。急行暴虐,欲以威民,乃是“自召敵讎”。勤行虐政,是“不懈怠”也。 \par}

商其淪喪,我罔為臣僕。詔王子出迪。\footnote{商其沒亡,我二人無所為臣僕,欲以死諫紂。我教王子出,合於道。臣僕,一本無臣字。}我舊雲刻子,王子弗出,我乃顛隮。\footnote{刻,病也。我久知子賢,言於帝乙。病立子,帝乙不肯。病子不得立,則宜為殷後者子。今若不出逃難,我殷家宗廟乃隕墜無主。舊云,馬云:“言也。”刻音克,馬云:“侵刻也。”難,乃旦反。}自靖。人自獻於先王,\footnote{各自謀行其志,人人自獻達於先王,以不失道。靖,馬本作清,謂潔也。}我不顧行遁。”\footnote{言將與紂俱死,所執各異,皆歸於仁,明君子之道,出處語默非一途。顧音故,徐音鼓。}

{\noindent\zhuan\zihao{6}\fzbyks 傳“商其”至“於道”。正義曰:“有災”與“淪喪”一事,而重出文者,上言“商今其有災,我興受其敗”,逆言災雖未至,至則己必受禍;此言“商其淪喪,我罔為臣僕”,豫言殷滅之後,言己不事異姓,辭有二意,故重出其文。我無所為臣僕,言不能與人為臣僕,必欲以死諫紂。但箕子之諫,值紂怒不甚,故得不死耳。“我教王子出,合於道”,保全身命,終為殷後,使宗廟有主,享祀不絕,是合其道也。 \par}

{\noindent\zhuan\zihao{6}\fzbyks 傳“刻病”至“無主”。正義曰:“刻”者,傷害之義,故為病也。\CJKunderwave{呂氏春秋·仲冬紀}云:“紂之母生微子啟與仲衍,其時猶尚為妾,改而為妻後生紂。紂之父欲立微子啟為太子,太史據法而爭,曰:‘有妻之子,不可立妾之子。’故立紂為後。”於時箕子蓋謂請立啟而帝乙不聽,今追恨其事,我久知子賢,言於帝乙,欲立子為太子,而帝乙不肯,我病子不得立,則宜為殷後。 \par}

{\noindent\zhuan\zihao{6}\fzbyks 傳“言將”至“一途”。正義曰:不肯遁以求生,“言將與紂俱死”也。或去或留,所執各異,皆歸於仁。\CJKunderline{孔子}稱“殷有三仁焉”,是“皆歸於仁”也。\CJKunderwave{易·繫辭}曰:“君子之道,或出或處,或默或語。”是“非一途”也。何晏云:“仁者愛人,三人行異而同稱仁者,以其俱在憂亂寧民。” \par}

{\noindent\shu\zihao{5}\fzkt “父師”至“行遁”。正義曰:父師亦順其事而報微子曰:“王子,今天酷毒下災,生此昏虐之君,以荒亂殷之邦國。紂既沈湎,四方化之,皆起而沈湎酗醟於酒,不可如何。小人皆自放恣,乃無所。上不畏天災,下不畏賢人,違戾其耇老之長與舊有爵位致仕之賢人。今殷民乃攘竊祭祀神祗之犧牷牲用,以相通容,行取食之,無災罪之者。”盜天地大祀之物用而不得罪,言政亂甚也。“我又下視殷民,所用為治者,皆讎怨斂聚之道”也。言重賦傷民,民以在上為讎,重賦乃是斂讎也。“既為重賦,又急行暴虐,此所以益招民怨,是乃自召敵讎不懈怠也。上下各有罪,合於一紂之身”。言紂化之使然也。“故使民多瘠病,而無詔救之者。商今其有滅亡之災,我起而受其敗。商其沒亡喪滅,我無所為人臣僕”。言不可別事他人,必欲諫取死也。“我教王子出奔於外,是道也。我久雲子賢,言於帝乙,欲立子,不肯。我乃病傷子不得立為王,則宜終為殷後。若王子不出,則我殷家宗廟乃隕墜無主”。既勸之出,即與之別云:“各自謀行其志,人人各自獻達於先王,我不顧念行遁之事。”明期與紂俱死。 \par}

%%% Local Variables:
%%% mode: latex
%%% TeX-engine: xetex
%%% TeX-master: "../Main"
%%% End:
