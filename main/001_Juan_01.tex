%% -*- coding: utf-8 -*-
%% Time-stamp: <Chen Wang: 2024-07-23 00:25:13>

% {\noindent \zhu \zihao{5} \fzbyks } -> 注(△ ○)
% {\noindent \shu \zihao{5} \fzkt } -> 疏

\part{序}


\chapter{卷一}

\section{尚書序}

 {\noindent \zhu\zihao{6}\fzbyks \CJKunderwave{釋文}:“此\CJKunderline{孔氏}所作,述\CJKunderwave{尚書}起之時代,並敘為注之由。故相承講之,今依舊為音。”\par}

 {\noindent \shu\zihao{5}\fzkt 正義曰:道本衝寂,非有名言。既形以道生,物由名舉,則凡諸經史,因物立名。物有本形,形從事著,聖賢闡教,事顯於言,言愜群心,書而示法,既書有法,因號曰“書”。後人見其久遠,自於上世,“尚”者,上也。言此上代以來之書,故曰“尚書”。且言者意之聲,書者言之記,是故存言以聲意,立書以記言。故易曰:“書不盡言,言不盡意。”是言者意之筌蹄,書言相生者也。書者,舒也。\CJKunderwave{書緯·璿璣鈐}云:“書者,如也。”則書者,寫其言,如其意,情得展舒也。又\CJKunderline{劉熙}\CJKunderwave{釋名}云:“書者,庶也,以記庶物。又為著。”言事得彰著。五經六籍皆是筆書,此獨稱“書”者,以彼五經者非是君口出言,即書為法,所書之事,各有云為,遂以所為別立其稱。稱以事立,故不名“書”。至於此書者,本書君事,事雖有別,正是君言,言而見書,因而立號,以此之故,名異諸部。但諸部之書,隨事立名,名以事舉,要名立之後,亦是筆書,故百氏六經總曰“書”也。\CJKunderwave{論讖}所謂“題意別名,各自載耳”。昭二年\CJKunderwave{左傳}曰,晉\CJKunderline{韓起}適魯,“觀書於太史氏,見易象與\CJKunderwave{魯春秋}”。此總名“書”也。“序”者,言序述\CJKunderwave{尚書}起、存亡注說之由,序為\CJKunderwave{尚書}而作,故曰“尚書序”。\CJKunderwave{周頌}曰:“繼序思不忘。”\CJKunderwave{毛傳}云:“序者,緒也。”則緒述其事,使理相胤續,若繭之抽緒。但易有\CJKunderwave{序卦},\CJKunderline{子夏}作\CJKunderwave{詩序},\CJKunderline{孔子}亦作\CJKunderwave{尚書序},故\CJKunderline{孔君}因此作序名也。\CJKunderline{鄭玄}謂之“贊”者,以序不分散,避其序名,故謂之“贊”。贊者,明也,佐也。佐成序義,明以註解故也。安國以\CJKunderline{孔子}之序分附篇端,故已之總述亦謂之“序”。事不煩重,義無所嫌故也。\par}

\textcolor{red}{古者}伏犧氏之王天下也,始畫八卦,造書契,以代結繩之政,由是文籍\textcolor{red}{生焉}。\footnote{伏犧氏,伏古作虙,犧本又作羲,亦作戲,辭皮反。\CJKunderwave{說文}云,賈侍中說此犧非古字。張揖\CJKunderwave{字詁}云:“羲古字,戲今字。”一號包羲氏。三皇之最先,風姓,母曰華胥,以木德王,即太皞也。王,於況反。畫,乎麥反。卦,俱賣反。契,苦計反。書者,文字;契者,刻木而書其側:故曰“書契”也。一雲以書契約其事也。\CJKunderline{鄭玄}云:“以書書木邊,言其事,刻其木,謂之書契也。”結繩,\CJKunderwave{易·繫辭}上:“上古結繩以治,後世聖人易之以書契。”文,文字也。籍,籍書。}

{\noindent\shu\zihao{5}\fzkt “古者”至“生焉”。正義曰:“代結繩”者,言前世之政用結繩,今有書契以代之。則伏犧時始有文字以書事,故曰“由是文籍生焉”。自今本昔曰“古”。古者以聖德伏物教人取犧牲,故曰“伏犧”。字或作宓犧,音亦同。\CJKunderwave{律曆志}曰:結作網罟,以取犧牲,故曰“伏犧”。或曰“包犧”,言取犧而包之。顧氏讀包為庖,取其犧牲以供庖廚。顧氏又引\CJKunderwave{帝王世紀}云:“伏犧母曰華胥,有巨人跡,出於雷澤,華胥以足履之,有娠,生伏犧於成紀,蛇身人首。”\CJKunderwave{月令}云:“其帝太昊。”\CJKunderwave{繫辭}云:“古者包犧氏之王天下也。”是直變“包”言“伏”耳。則伏犧是皇,言“王天下”者,以皇與帝、王據跡為優劣,通亦為王。故\CJKunderwave{禮運}雲“昔者先王”,亦謂上代為王。但自下言之,則以上身為王,據王身於下,謂之“王天下”也。知伏犧“始畫八卦”者,以\CJKunderwave{繫辭}雲“包犧氏之王天下也”,後乃云:“始畫八卦以通神明之德,以類萬物之情”,故知之也。知時“造書契以代結繩之政”者,亦以\CJKunderwave{繫辭}雲“上古結繩而治,後世聖人易之以書契,蓋取諸夬”,是造書契可以代結繩也。彼直言“後世聖人”,知是伏犧者,以理比況而知。何則?八卦畫萬物之象,文字書百事之名,故\CJKunderwave{繫辭}曰:“仰則觀象於天,俯則觀法於地,觀鳥獸之文與地之宜,近取諸身,遠取諸物,始畫八卦。”是萬象見於卦。然畫亦書也,與卦相類,故知書契亦伏犧時也。由此孔意正欲須言伏犧時有書契,本不取於八卦。今雲“八卦”者,明書、卦相類,據\CJKunderwave{繫辭}有畫八卦之成文而言,明伏犧造書契也。言“結繩”者,當如鄭注云:“為約,事大大其繩,事小小其繩。王肅亦曰“結繩,識其政事”是也。言“書契”者,鄭云:“書之於木,刻其側為契,各持其一,後以相考合,若結繩之為治。”孔無明說,義或當然。\CJKunderwave{說文}云:“文者,物象之本也。”“籍”者,借也,藉此簡書以記錄政事,故曰“籍”。“蓋取諸夬”,“夬”者,決也,言文籍所以決斷,宣揚王政,是以夬。\CJKunderwave{繇}曰:“揚於王庭。”\CJKunderwave{繫辭}雲“包犧氏之王天下”,又云“作結繩而為罔罟,蓋取諸離”。彼謂結罔罟之繩,與結為政之繩異也。若然,\CJKunderwave{尚書緯}及\CJKunderwave{孝經讖}皆雲三皇無文字,又班固、馬融、\CJKunderline{鄭玄}、王肅諸儒皆以為文籍初自五帝,亦云三皇未有文字,與此說不同。何也?又蒼頡造書出於\CJKunderwave{世本},蒼頡豈伏犧時乎?且\CJKunderwave{繫辭}雲黃帝、堯、舜為九事之目,未乃云:“上古結繩以治,後世聖人易之以書契。”是後世聖人即黃帝、堯、舜,何得為伏犧哉?孔何所據而更與\CJKunderwave{繫辭}相反,如此不同者?\CJKunderwave{藝文志}曰:“仲尼沒而微言絕,七十子喪而大義乖。”況遭秦焚書之後,群言競出,其緯文鄙近,不出聖人,前賢共疑,有所不取。通人考正,偽起哀、平,則\CJKunderline{孔君}之時,未有此\CJKunderwave{緯},何可引以為難乎?其馬、鄭諸儒,以據文立說,見後世聖人在九事之科,便謂書起五帝,自所見有異,亦不可難孔也。而\CJKunderwave{繫辭}雲後世聖人在九事之下者,有以而然。案彼文先歷說伏犧、神農“蓋取”,下乃云:“黃帝、堯、舜垂衣裳而天下治,蓋取諸乾坤。”是黃帝、堯、舜之事也。又舟楫取渙,服牛取隨,重門取豫,臼杵取小過,弧矢取睽,此五者時無所繫,在黃帝、堯、舜時以否皆可以通也。至於宮室、葬與書契,皆先言“上古”、“古者”,乃言“後世聖人易之”,則別起事之端,不指黃帝、堯、舜時。以此葬事雲“古者”不雲“上古”,而云“易之以棺槨”。棺槨自殷湯而然,非是彼時之驗,則上古結繩何廢伏犧前也?其蒼頡,則說者不同,故\CJKunderwave{世本}云:“蒼頡作書。”司馬遷、班固、韋誕、宋忠、傅玄皆云:“蒼頡,黃帝之史官也。”崔瑗,曹植,蔡邕,索靖皆直云:“古之王也。”徐整云:“在神農、黃帝之間。”譙周云:“在炎帝之世。”衛氏云:“當在庖犧、蒼帝之世。”慎到云:“在庖犧之前。”張揖云:“蒼頡為帝王,生於禪通之紀。”\CJKunderwave{廣雅}曰:自開闢至獲麟二百七十六萬歲,分為十紀。則大率一紀二十七萬六千年。十紀者,九頭一也,五龍二也,攝提三也,合雒四也,連通五也,序命六也,循飛七也,因提八也,禪通九也,流訖十也。如揖此言,則蒼頡在獲麟前二十七萬六千餘年。是說蒼頡其年代莫能有定,亦不可以難孔也。然紀自燧人而下,揖以為自開闢而設,又伏犧前六紀後三紀,亦為據張揖、慎到、徐整等說,亦不可以年斷。其涗訖之紀,似自黃帝為始耳。又依\CJKunderwave{易緯·通卦驗},燧人在伏犧前,“表計置其刻曰:蒼牙通靈,昌之成,孔演命,明道經”。\CJKunderline{鄭玄}注云:“刻謂刻石而記識之。”據此,伏犧前己有文字矣。又\CJKunderwave{陰陽書}稱天老對黃帝云:“鳳皇之象,首戴德,揹負仁,頸荷義,膺抱信,足履政,尾系武。”又\CJKunderwave{山海經}云:“鳳皇首文曰德,背文曰義,翼文曰順,膺文曰仁,腹文曰信。”又\CJKunderwave{易·繫辭}云:“河出圖,洛出書,聖人則之。是文字與天地並興焉。又\CJKunderwave{韓詩外傳}稱古封太山、禪梁甫者萬餘人,仲尼觀焉不能盡識。又\CJKunderwave{管子}書稱管仲對齊桓公曰:“古之封太山者七十二家,夷吾所識十二而已。”首有“無懷氏封太山,禪云云”。其登封者皆刻石紀號,但遠者字有雕毀,故不可識,則夷吾所不識者六十家,又在無懷氏前,\CJKunderline{孔子}睹而不識,又多於夷吾,是文字在伏犧之前己自久遠,何怪伏犧而有書契乎?如此者,蓋文字在三皇之前未用之教世,至伏犧乃用造書契以代結繩之政,是教世之用,猶燧人有火,中古用以燔黍捭豚,後聖乃修其利相似。文字理本有之,用否隨世而漸也。若然,惟\CJKunderwave{繫辭}至神農始有噬嗑與益,則伏犧時其卦未重,當無雜卦,而得有取諸夬者,此自\CJKunderline{鄭玄}等說耳。案\CJKunderwave{說卦}曰:“昔者聖人幽贊於神明而生蓍。”\CJKunderwave{繫辭}曰:“天生神物,聖人則之。”則伏犧用蓍而筮矣。故鄭注\CJKunderwave{說卦}亦曰:“昔者聖人,謂伏犧文王也。”\CJKunderwave{繫辭}又曰:“十有八變而成卦。”是言爻皆三歸奇為三變,十八變則六爻明矣。則筮皆六爻,伏犧有筮,則有六爻,何為不重而怪有夬卦乎? \par}

\textcolor{red}{伏犧}、神農、黃帝之書,謂之\CJKunderwave{三墳},言大道也。少昊、顓頊、高辛、唐、虞之書,謂之\CJKunderwave{五典},言\textcolor{red}{常道}也。\footnote{少,詩照反。昊,胡老反。少昊。金天氏,名摯,字青陽,一曰玄器,已姓。黃帝之子,母曰女節。以金德王,五帝之最先。顓音專。頊,許玉反。顓頊,高陽氏,姬姓。黃帝之孫,昌意之子,母曰景僕,謂之女樞。以水德王,五帝之二也。高辛,帝嚳也,姬姓。嚳,口毒反。母曰不見。以木德王,五帝之三也。唐,\CJKunderline{帝堯}也,姓伊耆氏。堯初為唐侯,後為天子,都陶,故號陶唐氏。帝嚳之子,帝摯之弟,母曰慶都。以火德王,五帝之四也。虞,\CJKunderline{帝舜}也,姓姚氏,國號有虞。顓頊六世孫,瞽瞍之子,母曰握登。以土德王,五帝之五也。先儒解三皇五帝與\CJKunderline{孔子}同,並見發題。}

{\noindent\shu\zihao{5}\fzkt “伏犧”至“常道”也。正義曰:墳”,大也。以所論三皇之事,其道至大,故曰“言大道也”。以“典”者,常也,言五帝之道,可以百代常行,故曰“言常道也”。此三皇五帝,或舉德號,或舉地名,或直指其人,言及便稱,不為義例。顧氏引\CJKunderwave{帝王世紀}云:農母曰女登,有神龍首感女登而生炎帝,人身牛首。黃帝母曰附寶,見大電光繞北斗樞星,附寶感而懷孕,二十四月而生黃帝,日角龍顏。少昊金天氏母曰女節,有星如虹下流,意感而生少昊。顓頊母曰景僕,昌意正妃,謂之女樞,有星貫月如虹,感女樞於幽房之宮而生顓頊。堯母曰慶都,觀河遇赤龍,晻然陰風,感而有孕,十四月而生堯。又云舜母曰握登,見大虹感而生舜。此言“謂之三墳”,“謂之五典”者,因\CJKunderwave{左傳}有“三墳五典”之文,故指而謂之。然五帝之書皆謂之典,則\CJKunderwave{虞書·皋陶謨}、\CJKunderwave{益稷}之屬,亦應稱典。所以別立名者,若主論帝德,則以典為名,其臣下所為,隨義立稱。其\CJKunderwave{三墳}直雲“言大道也”,\CJKunderwave{五典}直雲“言常道也”,不訓“墳”、“典”之名者,以墳大典常,常訓可知,故略之也。“常道”所以與“大道”為異者,以帝者公平天下,其道可以常行,故以“典”言之。而皇優於帝,其道不但可常行而已,又更大於常,故言“墳”也。此為對例耳。雖少有優劣,皆是大道,並可常行。故\CJKunderwave{禮運}云:“以大道之行”為五帝時也。然帝號同天,名所莫加,優而稱“皇”者,以“皇”是美大之名,言大於帝也。故後代措廟立主,尊之曰“皇”,生者莫敢稱焉。而士庶祖父稱曰“皇”者,以取美名,可以通稱故也。案\CJKunderwave{左傳}上有“三墳五典”,不言墳是三皇之書,典是五帝之書。孔知然者,案今\CJKunderwave{堯典}、\CJKunderwave{舜典}是二帝二典,推此二典而上,則五帝當五典,是五典為五帝之書。今三墳之書在五典之上,數與三皇相當,墳又大名,與皇義相類,故云三皇之書為三墳。\CJKunderline{孔君}必知三皇有書者,案\CJKunderwave{周禮·小史職}“掌三皇五帝之書”,是其明文也。\CJKunderline{鄭玄}亦云其書即三墳五典。但\CJKunderline{鄭玄}以三皇無文,或據後錄定。\CJKunderline{孔君}以為書者記當時之事,不可以在後追錄,若當時無書,後代何以得知其道也?此亦\CJKunderline{孔君}所據三皇有文字之驗耳。\CJKunderline{鄭玄}注\CJKunderwave{中候},依\CJKunderwave{運鬥樞}以伏犧、女媧、神農為三皇,又云五帝坐,帝鴻、金天、高陽、高辛、唐、虞氏。知不爾者,\CJKunderline{孔君}既不依緯,不可以緯難之。又易興作之條,不見有女媧,何以輒數?又\CJKunderline{鄭玄}云:“女媧修伏犧之道,無改作則。”已上修舊者眾,豈皆為皇乎?既不數女媧,不可不取黃帝以充三皇耳。又\CJKunderline{鄭玄}數五帝,何以六人?或為之說云,德協五帝,座不限多少,故六人亦名五帝。若六帝何有五座?而皇指大帝,所謂“耀魄寶”,止一而已,本自無三皇,何雲三皇?豈可三皇數人,五帝數座,二文舛互,自相乖阻也。其諸儒說三皇,或數燧人,或數祝融以配犧、農者,其五帝皆自軒轅,不數少昊,斯亦非矣。何燧人說者以為伏犧之前,據易曰“帝出於震”,震,東方,其帝太昊。又云“古者包犧氏之王天下也”,言古者製作莫先於伏犧,何以燧人側在前乎?又祝融及顓頊以下火官之號,金天已上百官之號,以其徵五經,無雲祝融為皇者,縱有,不過如\CJKunderline{共工氏}。\CJKunderline{共工}有水瑞,乃與犧、農、軒、摯相類,尚雲霸其九州,祝融本無此瑞,何可數之乎?\CJKunderwave{左傳}曰:“少昊之立,鳳鳥適至。”於\CJKunderwave{月令}又在秋享食,所謂白帝之室者也,何為獨非帝乎?故\CJKunderline{孔君}以黃帝上數為皇,少昊為五帝之首耳。若然,案今\CJKunderwave{世本·帝系}及\CJKunderwave{大戴禮·五帝德}並\CJKunderwave{家語}宰我問、太史公\CJKunderwave{五帝本紀}皆以黃帝為五帝,此乃史籍明文,而\CJKunderline{孔君}不從之者。孟軻曰:“信\CJKunderwave{書}不如其無\CJKunderwave{書},吾於\CJKunderwave{武成}取二三策而已。”言書以漸染之濫也。孟軻已然,況後之說者乎?又\CJKunderwave{帝系}、\CJKunderwave{本紀}、\CJKunderwave{家語}、\CJKunderwave{五帝德}皆云:“少昊即黃帝子青陽是也,顓頊黃帝孫、昌意子,帝嚳高辛氏為黃帝曾孫、玄囂孫、僑極子,堯為帝嚳子,舜為顓頊七世孫。此等之書,說五帝而以黃帝為首者,原由\CJKunderwave{世本}。經於暴秦,為儒者所亂。\CJKunderwave{家語}則王肅多私定,\CJKunderwave{大戴禮}、\CJKunderwave{本紀}出於\CJKunderwave{世本},以此而同。蓋以少昊而下皆出黃帝,故不得不先說黃帝,因此謬為五帝耳。亦由\CJKunderwave{繫辭}以黃帝與堯、舜同事,故儒者共數之焉。\CJKunderline{孔君}今者意以\CJKunderwave{月令}春曰太昊,夏曰炎帝,中央曰黃帝,依次以為三皇。又依\CJKunderwave{繫辭},先包犧氏王。沒,神農氏作。又沒,黃帝氏作。亦文相次。皆著作見於易,此三皇之明文也。\CJKunderwave{月令}秋曰少昊,冬曰顓頊,自此為五帝。然黃帝是皇,今言“帝”不雲“皇”者,以皇亦帝也,別其美名耳。太昊為皇,\CJKunderwave{月令}亦曰“其帝太昊”,易曰“帝出於震”是也。又軒轅之稱黃帝,猶神農之雲炎帝,神農於\CJKunderwave{月令}為炎帝,不怪炎帝為皇,何怪軒轅稱帝?而梁主云:“書起軒轅,同以燧人為皇,其五帝自黃帝至堯而止。知帝不可以過五,故曰舜非三王,亦非五帝,與三王為四代而已。”其言與詩之為體,不雅則風,除皇已下不王則帝,何有非王非帝,以為何人乎?典,謨皆雲“帝曰”,非帝如何! \par}

\textcolor{red}{至於}夏、商、周之書,雖設教不倫,雅誥奧義,其歸\textcolor{red}{一\xpinyin*{揆}}。\footnote{夏,禹天下號也。以金德王,三王之最先。商,湯天下號,亦號殷,以水德王,三王之二也。周,文王、武王有天下號也。以木德王,三王之三也。誥,故報反,告也,示也。奧,烏報反,深也。揆,葵癸反,度也,道理也。}

{\noindent\shu\zihao{5}\fzkt “至於”至“一揆”。正義曰:既皇書稱“墳”,帝書稱“典”,除皇與帝墳、典之外,以次累陳,故言“至於”。夏、商、周三代之書,雖復當時所設之教,與皇及帝墳、典之等不相倫類,要其言皆是雅正辭誥,有深奧之義,其所歸趣與墳、典一揆。明雖事異墳、典而理趣終同,故所以同入\CJKunderwave{尚書},共為世教也。\CJKunderline{孔君}之意,以墳、典亦是\CJKunderwave{尚書},故此因墳、典而及三代。下雲“討論墳、典,斷自唐、虞以下”,是墳、典亦是\CJKunderwave{尚書}之內,而小史偏掌之者,以其遠代故也。此既言墳、典,不依外文連類,解八索、九丘,而言三代之書廁於其間者,孔意以墳、典是\CJKunderwave{尚書},丘、索是\CJKunderwave{尚書}外物,欲先說\CJKunderwave{尚書}事訖,然後及其外物,故先言之也。夏、商、周之書,皆訓誥誓命之事,言“設教”者,以此訓誥誓命即為教而設,故云“設教”也。言“不倫”者,倫,類也。三代戰爭不與皇帝等類,若然,五帝稱“典”,三王劣而不倫,不得稱“典”。則三代非典,不可常行,何以垂法乎?然三王世澆,不如上代,故隨事立名,雖篇不目典,理實是典,故曰“雅誥奧義,其歸一揆”,即為典之謂也。然三王之書,惟無典謨,以外訓、誥、誓、命、歌、貢、徵、範,類猶有八,獨言“誥”者,以別而言之。其類有八,文從要約,一“誥”兼焉。何者?以此八事皆有言以誥示,故總謂之“誥”。又言“奧義”者,指其言謂之誥,論其理謂之義,故以義配焉。言“其歸一揆”,見三代自歸於一,亦與墳、典為一揆者,況喻之義。假譬人射,莫不皆發,志揆度於的,猶如聖人立教,亦同揆度於至理,故云“一揆”。 \par}

是故歷代寶之,以為大訓。

{\noindent\shu\zihao{5}\fzkt 正義曰:\CJKunderwave{顧命}云:“越玉五重,陳寶。”即以赤刀、大訓在西序,是“寶之,以為大訓”之文。彼注以典謨為之,與此相當。要六藝皆是,此直為\CJKunderwave{書}者,指而言之,故彼注亦然也。彼直周時寶之,此知歷代者,以墳、典久遠,周尚寶之,前代可知,故言“歷代”耳。 \par}

\textcolor{red}{八卦}之說,謂之\CJKunderwave{八索},求其義也。九州之志,謂之\CJKunderwave{九丘}。言九州所有,土地所生,風氣所宜,皆聚\textcolor{red}{此書}也。\footnote{丘,聚也。八索,所白反,下同,求也。徐音素,本或作素。}

{\noindent\shu\zihao{5}\fzkt “八卦”至“此書”也。正義曰:以墳、典因外文而知其丘、索與墳、典文連,故連而說之,故總引傳文以充足己意,且為於下見與墳、典俱被黜削,故說而以為首引。言為論八卦事義之說者,其書謂之\CJKunderwave{八索}。其論九州之事所有志記者,其書謂之九丘,所以名“丘”者,以丘,聚也,言於九州當有土地所生之物,風氣所宜之事,莫不皆聚見於此書,故謂之\CJKunderwave{九丘}焉。然八卦言之“說”,九州言之“志”,不同者,以八卦交互相說其理,九州當州有所志識,以此而不同。此“索”謂求索,亦為搜索,以易八卦為主,故易曰:“八卦成列,象在其中矣。因而重之,爻在其中矣。”又曰:“八卦相蕩。”是六十四卦,三百八十四爻,皆出於八卦。就八卦而求其理,則萬有一千五百二十策,天下之事得,故謂之“索”,非一索再索而已。此“索”於\CJKunderwave{左傳}亦或謂之“索”,說有不同,皆後人失其真理,妄穿鑿耳。其\CJKunderwave{九丘}取名於聚,義多如山丘,故為聚。\CJKunderwave{左傳}或謂之“九區”,得為說當九州之區域,義亦通也。又言“九州所有”,此一句與下為總,即“土地所生,風氣所宜”是所有也。言“土地所生”,即其動物、植物,大率土之所生不出此二者。又云“風氣所宜”者,亦與土地所生大同。何者?以九州各有土地,有生與不生,由風氣所宜與不宜。此亦\CJKunderwave{職方}、\CJKunderwave{禹貢}之類。別而言之,“土地所生”若\CJKunderwave{禹貢}之“厥貢”、“厥篚”也,“風氣所宜”若\CJKunderwave{職方}其畜宜若干、其民若干男、若干女是也。上“墳”、“典”及“索”不別訓之,以可知,故略之。“丘”訓既難,又須別言“九州所宜”已下,故先訓之,於下結義,故云“皆聚此書”也。 \par}

\CJKunderwave{\textcolor{red}{春秋}左氏傳}曰,楚左史倚相“能讀三墳、五典、八索、九丘”,即謂上世帝王\textcolor{red}{遺書}也。\footnote{左史,史官左右。倚,於綺反,劉琴綺反。相,息亮反。倚相,楚靈王時史官也。}

{\noindent\shu\zihao{5}\fzkt “春秋”至“遺書”也。正義曰:以上因有外文言墳、典、丘、索而謂之,故引成文以證結之。此昭十二年\CJKunderwave{左傳}楚靈王見倚相趨過,告右尹子革以此辭。知“倚相”是其名字,蓋為太史,而主記左動之事,謂之“左史”。不然,或楚俗與諸國不同,官多以左右為名,或別有此左史乎?彼子革答王云:“倚相,臣問\CJKunderwave{祈招}之詩而不知,若問遠焉,其焉能知之?”彼以為倚相不能讀之。此雲“能”者,以此據\CJKunderwave{左傳}成文,因王言而引之。假不能讀,事亦無妨,況子革欲開諫王之路,倚相未必不能讀也。言此墳、典、丘,索即此書是謂上世帝王遺餘之書也。以楚王論時已在三王之末,故云“遺書”。其丘、索知是前事,亦不知在何代,故直總言“帝王”耳。 \par}

\textcolor{red}{先君}\CJKunderline{孔子},生於周末,睹史籍之煩文,懼覽之者不一,遂乃定\CJKunderwave{禮}、\CJKunderwave{樂},明舊章,刪\CJKunderwave{詩}為三百篇,約史記而修\CJKunderwave{春秋},贊\CJKunderwave{易}道以黜\CJKunderwave{八索},述\CJKunderwave{職方}以除\CJKunderwave{\textcolor{red}{九丘}}。

{\noindent\shu\zihao{5}\fzkt “先君”至“九丘”。正義曰:既結申帝王遺書,欲言\CJKunderline{孔子}就而刊定。\CJKunderwave{孔子世家}云,安國是\CJKunderline{孔子}十一世孫,而上尊先祖,故曰“先君”。\CJKunderwave{穀梁}以為魯襄公二十一年冬十一月庚子\CJKunderline{孔子}生,\CJKunderwave{左傳}哀公十六年夏四月已醜\CJKunderline{孔子}卒,計以周靈王時生,敬王時卒,故為“周末”。上雲“文籍”,下雲“滅先代典籍”,此言“史籍”。“籍”者,古書之大名。由文而有籍。謂之“文籍”;因史所書謂之“史籍”;可以為常,故曰“典籍”,義亦相通也。但上因書契而言“文”,下傷秦滅道以稱“典”,於此言“史”者,不但義通上下,又以此“史籍”不必是先王正史,是後代好事者作,以此懼其不一,故曰:“蓋有不知而作之者,我無是也。”先言“定\CJKunderwave{禮}、\CJKunderwave{樂}”者,欲明\CJKunderline{孔子}欲反於聖道以歸於一,故先言其舊行可從者。修而不改曰“定”,就而減削曰“刪”,準依其事曰“約”,因而佐成曰“贊”,顯而明之曰“述”,各從義理而言。獨\CJKunderwave{禮}、\CJKunderwave{樂}不改者,以\CJKunderwave{禮}、\CJKunderwave{樂}聖人制作,已無貴位,故因而定之。又云“明舊章”者,即\CJKunderwave{禮}、\CJKunderwave{樂}、\CJKunderwave{詩}、易、\CJKunderwave{春秋}是也。以“易道”、“職方”與“黜八索”、“除九丘”相對,其約史記以刪\CJKunderwave{詩}、\CJKunderwave{書}為偶,其定\CJKunderwave{禮}、\CJKunderwave{樂}文孤,故以“明舊章”配之,作文之體也。易亦是聖人所作,不言“定”者,以易非如\CJKunderwave{禮}、\CJKunderwave{樂},人之行事,不須雲“定”。又因而為作\CJKunderwave{十翼},故云“贊”耳。易文在下者,亦為“黜八索”與“除九丘”相近故也。為文之便,不為義例。\CJKunderline{孔子}之修六藝,年月孔無明說。\CJKunderwave{論語}曰:“吾自衛反魯,然後樂正,\CJKunderwave{雅}、\CJKunderwave{頌}各得其所。”則\CJKunderline{孔子}以魯哀公十一年反魯為大夫,十二年\CJKunderwave{孟子}卒,\CJKunderline{孔子}吊,則致仕時年七十以後。“修”,述也。\CJKunderwave{詩}有序三百一十一篇,全者三百五篇,雲“三百”者,亦舉全數計。\CJKunderwave{職方}在\CJKunderwave{周禮·夏官},亦武帝時出於山岩屋壁,即藏秘府,世人莫見。以\CJKunderline{孔君}為武帝博士,於秘府而見為。知必“黜八索”、“除九丘”者,以三墳、五典本有八,今序只有二典而已,其三典、三墳今乃寂寞,明其除去,既墳、典書內之正尚有去者,況書外乎?故知丘、索亦黜除也。“黜”與“除”其義一也,黜退不用而除去之。必雲“贊易道以黜”者,以不有所興,孰有所廢故也。\CJKunderwave{職方}即\CJKunderwave{周禮}也,上已雲“定\CJKunderwave{禮}、\CJKunderwave{樂}”,即\CJKunderwave{職方}在其內。別雲述之,以為“除九丘”,舉其類者以言之。則雲“述”者,以定而不改即是遵述,更有書以述之。 \par}

\textcolor{red}{討論}\CJKunderwave{墳}、\CJKunderwave{典},斷自唐虞以下,訖於周。芟夷煩亂,翦截浮辭,舉其宏綱,撮其機要,足以垂世立教,典、謨、訓、誥、誓、命之文凡\textcolor{red}{百篇}。\footnote{斷,丁亂反。訖,居乙反,又許乙反。芟,色鹹反。翦,諮淺反。撮,七活反。機,本又作幾。典凡十五篇,正典二,攝十三,十一篇亡。謨,莫胡反。凡三篇,正二,攝一。訓凡十六篇,正二篇亡,攝十四,三篇亡。誥凡三十八篇,正八,攝三十,十八篇亡。誓,市制反。凡十篇,正八,攝二,十篇亡。命凡十八篇,正十二,三篇亡,攝六,四篇亡。}

{\noindent\shu\zihao{5}\fzkt “討論”至“百篇”。正義曰:言\CJKunderline{孔子}既懼覽之者不一,不但刪\CJKunderwave{詩}、約史、定\CJKunderwave{禮}、贊易,有所黜除而已,又討整論理此三墳、五典並三代之書也。\CJKunderwave{論語}曰:“世叔討論之。”鄭以“討論”為整理,\CJKunderline{孔君}既取彼文,義亦當然。以書是亂物,故就而整理之。若然,墳、典周公制禮,使小史掌之;而\CJKunderline{孔子}除之者,蓋隨世不同亦可,\CJKunderline{孔子}之時,墳、典已雜亂,故因去之。\CJKunderwave{左傳}曰“芟夷蘊崇之”,又曰“俘翦惟命”,\CJKunderwave{詩}曰“海外有截”,此\CJKunderline{孔君}所取之文也。“芟夷”者,據全代、全篇似草隨次皆芟,使平夷。若自帝嚳己上三典、三墳是芟夷之文,自夏至周雖有所留,全篇去之而多者,即“芟夷”也。“翦截”者,就代就篇辭有浮者翦截而去之,去而少者為“翦截”也。“舉其宏綱”即上“芟夷煩亂”也,“撮其機要”即上“翦截浮辭”也。且“宏綱”雲“舉”,是據篇、代大者言之;“機要”雲“撮”,為就篇、代之內而撮出之耳。“宏”,大也;“綱”者,網之索,舉大綱則眾目隨之。“機”者,機關,撮取其機關之要者,“斷自唐虞以下”者,孔無明說。\CJKunderwave{書緯}以為帝嚳以上,樸略難傳,唐虞已來,煥炳可法。又禪讓之首,至周五代一意故耳,孔義或然。“典”即\CJKunderwave{堯典}、\CJKunderwave{舜典},“謨”即\CJKunderwave{大禹謨}、\CJKunderwave{皋陶謨},“訓”即\CJKunderwave{伊訓}、\CJKunderwave{高宗之訓},“誥”即\CJKunderwave{湯誥}、\CJKunderwave{大誥},“誓”即\CJKunderwave{甘誓}、\CJKunderwave{湯誓},“命”即\CJKunderwave{畢命}、\CJKunderwave{顧命}之等是也。說者以\CJKunderwave{書}體例有十,此六者之外尚有徵、貢、歌、範四者,並之則十矣。若\CJKunderwave{益稷}、\CJKunderwave{盤庚},單言附於十事之例。今孔不言者,不但舉其機約,亦自徵、貢、歌、範非君出言之名,六者可以兼之。此雲“凡百篇”,據序而數故耳。或雲百二篇者,誤有所由。以前漢之時,有東萊張霸偽造\CJKunderwave{尚書}百兩篇,而為緯者附之。因此鄭云:“異者其在大司徒、大僕正乎?此事為不經也。”鄭作\CJKunderwave{書論},依\CJKunderwave{尚書緯}云:“\CJKunderline{孔子}求書,得黃帝玄孫帝魁之書,迄於秦穆公,凡三千二百四十篇。斷遠取近,定可以為世法者百二十篇,以百二篇為\CJKunderwave{尚書},十八篇為\CJKunderwave{中候}。”以為去三千一百二十篇,以上取黃帝玄孫,以為不可依用。今所考核\CJKunderwave{尚書},首自舜之末年以禪於禹,上錄舜之得用之事,由堯以為\CJKunderwave{堯典},下取舜禪之後,以為舜讓得人,故史體例別,而不必君言。若\CJKunderwave{禹貢}全非君言,而禹身事受禪之後無入\CJKunderwave{夏書}之言。是舜史自錄成一法,後代因之耳。 \par}

\textcolor{red}{所以}恢弘至道,示人主以軌範也。帝王之制,坦然明白,可舉而行,三千之徒並受\textcolor{red}{其義}。\footnote{恢,苦回反,大也。坦,土管反。}

{\noindent\shu\zihao{5}\fzkt “所以”至“其義”。正義曰:此論\CJKunderline{孔子}正理群經已畢,總而結之,故為此言。\CJKunderwave{家語}及\CJKunderwave{史記}皆雲“\CJKunderline{孔子}弟子三千人”,故云“三千之徒”也。 \par}

\textcolor{red}{及秦}始皇滅先代典籍,焚書坑儒,天下學士,逃難解散,我先人用藏其家書於\textcolor{red}{屋壁}。\footnote{始皇名政,二十六年初並六國,自號始皇帝。焚\CJKunderwave{書}、\CJKunderwave{詩}在始皇之三十四年,坑儒在三十五年。坑,苦庚反。難,乃旦反。解音蟹。}

{\noindent\shu\zihao{5}\fzkt “及秦”至“屋壁”。正義曰:言\CJKunderline{孔子}既定此書後,雖曰明白,反遭秦始皇滅除之。依\CJKunderwave{秦本紀}云,秦王正二十六年平定天下,尊為皇帝,不復立諡,以為初並天下,故號始皇。為滅先代典籍,故云“坑儒焚書”。以即位三十四年,因置酒於咸陽宮,丞相李斯奏請“天下敢有藏\CJKunderwave{詩}、\CJKunderwave{書}、百家語者,悉詣守、尉親燒之。有敢偶語\CJKunderwave{詩}、\CJKunderwave{書}者棄市。令下三十日不燒,黥為城旦”。制曰:“可。”是“焚書”也。三十五年,始皇以方士盧生求仙藥不得,以為誹謗,諸生連相告引,四百六十餘人皆坑之咸陽。是“坑儒”也。又衛宏\CJKunderwave{古文奇字序}云:“秦改古文以為篆隸,國人多誹謗。秦患天下不從,而召諸生,至者皆拜為郎,凡七百人。又密令冬月種瓜於驪山硎谷之中溫處,瓜實,乃使人上書曰:‘瓜冬有實。’有詔天下博士諸生說之,人人各異,則皆使往視之。而為伏機,諸生方相論難,因發機,從上填之以土,皆終命也。”“我先人用藏其家書於屋壁”者,\CJKunderwave{史記·孔子世家}云,\CJKunderline{孔子}生鯉,字伯魚。魚生伋,字子思。思生白,字子上。上生求,字子家。家生箕,字子京。京生穿,字子高。高生慎,慎為魏相。慎生鮒,鮒為陳涉博士。鮒弟子襄,為惠帝博士,長沙太守。襄生中。中生武。武生延陵及安國,為武帝博士,臨淮太守。\CJKunderwave{家語序}云:“子襄以秦法峻急,壁中藏其家書。”是安國祖藏之。 \par}

\textcolor{red}{漢室}龍興,開設學校,旁求儒雅,以闡大猷。濟南伏生,年過九十,失其本經,口以傳授。裁二十餘篇。以其上古之書,謂之\CJKunderwave{尚書}。百篇之義,世莫\textcolor{red}{得聞}。\footnote{校,戶教反。\CJKunderwave{詩}箋云:“鄭國謂學為校。”闡,尺善反,大也,明也。濟,子禮反,郡名也。伏生,名勝。過,古臥反,後同。傳,直專反,下“傳之”同。“二十餘篇”即馬、鄭所注二十九篇也。}

{\noindent\shu\zihao{5}\fzkt “漢室”至“得聞”。正義曰:將言所藏之書得之所由,故本之也。言“龍興”者,以易龍能變化,故比之聖人。九五“飛龍在天”,猶聖人在天子之位,故謂之“龍興”也。言“學校”者,校,學之一名也。故\CJKunderwave{鄭詩序}雲“子衿,刺學校廢”,\CJKunderwave{左傳}雲“然明請毀鄉校”是也。\CJKunderwave{漢書}云:“惠帝除挾書之律,立學興教,招聘名士。文景以後儒者更眾,至武帝尤甚。”故云“旁求儒雅”。\CJKunderwave{詩·小雅}曰:“匪先民是程,匪大猷是經。”彼注云:“猷,道也。”大道即先王六籍是也。伏生名勝,為秦二世博士,\CJKunderwave{儒林傳}云:“孝文帝時,求能治\CJKunderwave{尚書}者,天下無有,聞伏生治之,欲召。時伏生年已九十有餘,老不能行,於是詔太常,使掌故臣晁錯往受之。得二十九篇,即以教於齊魯之間。”是“年過九十”也。案\CJKunderwave{史記}:“秦時焚書,伏生壁藏之。其後兵火起,流。漢定天下,伏生求其書,亡數十篇,獨得二十九篇,以教於齊魯之閒。”則伏生壁內得二十九篇。而云“失其本經,口以傳授”者,蓋伏生初實壁內得之以教齊魯,傳教既久,誦文則熟,至其末年,因其習誦,或亦目暗,至年九十晁錯往受之時,不執經而口授之故也。又言“裁二十餘篇”者,意在傷亡,為少之文勢。何者?以數法隨所近而言之,若欲多之,當雲得三十篇,今“裁二十餘篇”,言“裁”亦意以為少之辭。又二十九篇自是計卷,若計篇則三十四,去\CJKunderwave{泰誓}猶有三十一。案\CJKunderwave{史記}及\CJKunderwave{儒林傳}皆雲“伏生獨得二十九篇,以教齊魯”,則今之\CJKunderwave{泰誓},非初伏生所得。案馬融雲“\CJKunderwave{泰誓}後得”,\CJKunderline{鄭玄}\CJKunderwave{書論}亦云“民間得\CJKunderwave{泰誓}”。\CJKunderwave{別錄}曰:“武帝末,民有得\CJKunderwave{泰誓}書於壁內者,獻之。與博士使讀說之,數月皆起,傳以教人。”則\CJKunderwave{泰誓}非伏生所傳。而言二十九篇者,以司馬遷在武帝之世見\CJKunderwave{泰誓}出而得行,入於伏生所傳內,故為史總之,並雲伏生所出,不復曲別分析。雲民間所得,其實得時不與伏生所傳同也。但伏生雖無此一篇,而\CJKunderwave{書}傳有八百諸侯俱至孟津,白魚入舟之事,與\CJKunderwave{泰誓}事同,不知為伏生先為此說?不知為是\CJKunderwave{泰誓}出後。後人加增此語?案王充\CJKunderwave{論衡}及\CJKunderwave{後漢史}獻帝建安十四年黃門侍郎房宏等說云,宣帝本始元年,河內女子有壞老子屋,得古文\CJKunderwave{泰誓}三篇。\CJKunderwave{論衡}又云:“以掘地所得者。”今\CJKunderwave{史}、\CJKunderwave{漢}書皆雲伏生傳二十九篇,則司馬遷時已得\CJKunderwave{泰誓},以並歸於伏生,不得雲宣帝時始出也。則雲宣帝時女子所得,亦不可信。或者爾時重得之,故於後亦據而言之。\CJKunderwave{史記}雲伏生得二十九篇,\CJKunderwave{武帝記}載今文\CJKunderwave{泰誓}末篇,由此劉向之作\CJKunderwave{別錄},班固為\CJKunderwave{儒林傳},不分明,因同於\CJKunderwave{史記}。而劉向雲武帝末得之\CJKunderwave{泰誓},理當是一。而古今文不同者,即馬融所云:“吾見書傳多矣,凡諸所引,今之\CJKunderwave{泰誓}皆無此言,而古文皆有。”則古文為真,亦復何疑?但於先有張霸之徒偽造\CJKunderwave{泰誓},以藏壁中,故後得而惑世也。亦可今之\CJKunderwave{泰誓}百篇之外,若\CJKunderwave{周書}之例,以於時實有觀兵之誓,但不錄入\CJKunderwave{尚書}。故古文\CJKunderwave{泰誓}曰“皇天震怒,命我文考,肅將天威,大勳未集。肆予小子發,以爾友邦冢君,觀政於商”是也。又云:“以其上古之書,謂之\CJKunderwave{尚書}”者,此文繼在“伏生”之下,則言“以其上古之書,謂之\CJKunderwave{尚書}”,此伏生意也。若以伏生指解\CJKunderwave{尚書}之名,名已先有,有則當雲名之\CJKunderwave{尚書}。既言“以其上古之書”,今先雲“以其”,則伏生意之所加,則知“尚”字乃伏生所加也。以“尚”解上,則“尚”訓為上。上者,下所慕尚,故義得為通也。\CJKunderline{孔君}既陳伏生此義,於下更無是非,明即用伏生之說,故書此而論之。馬融雖不見\CJKunderline{孔君}此說,理自然同,故曰“上古有虞氏之書,故曰\CJKunderwave{尚書}”是也。王肅曰:“上所言,史所書,故曰\CJKunderwave{尚書}。”鄭氏云:“尚者上也,尊而重之,若天書然,故曰\CJKunderwave{尚書}。”二家以“尚”與“書”相將,則上名不正出於伏生?\CJKunderline{鄭玄}依\CJKunderwave{書緯},以“尚”字是\CJKunderline{孔子}所加,故\CJKunderwave{書贊}曰:“\CJKunderline{孔子}乃尊而命之曰\CJKunderwave{尚書}。”\CJKunderwave{璿璣鈐}云:“因而謂之\CJKunderwave{書},加尚以尊之。”又曰:“\CJKunderwave{書}務以天言之。”\CJKunderline{鄭玄}溺於\CJKunderwave{書緯}之說,何有人言而須系之於天乎?且\CJKunderline{孔君}親見伏生,不容不悉,自雲伏生“以其上古之書,謂之\CJKunderwave{尚書}”,何雲\CJKunderline{孔子}加也?王肅雲“上所言,史所書”,則“尚”字與“書”俱有,無先後。既直雲“尚“,何以明上之所言?書者,以筆畫記之辭,群書皆是,何知\CJKunderwave{書}要責史所為也?此其不若前儒之說密耳。雲“上古”者亦無指定之目,自伏生言之,則於漢世,仰遵前代,自周已上皆是。馬融云:“有虞氏為\CJKunderwave{書}之初耳。”若易歷三世,則伏犧為上古,文王為中古,\CJKunderline{孔子}為下古。\CJKunderwave{禮運}\CJKunderline{鄭玄}以先王食腥與易“上古結繩”同時,為上古,神農為中古,五帝為下古。其不相對,則無例耳。且太之與上為義不異,\CJKunderwave{禮}以唐虞為太古,以下有三代,冠而推之為然,是為不定則。但今世已上,仰之已古,便為上古耳。以\CJKunderwave{書}是本名,“尚”是伏生所加,故諸引\CJKunderwave{書}直雲“\CJKunderwave{書}曰”,若有配代而言,則曰“\CJKunderwave{夏書}”無言“\CJKunderwave{尚書}”者。 \par}

\textcolor{red}{至魯}共王好治宮室,壞\CJKunderline{孔子}舊宅,以廣其居,於壁中得先人所藏古文虞、夏、商、周之書及傳\CJKunderwave{論語}、\CJKunderwave{孝經},皆科斗文字。王又升\CJKunderline{孔子}堂,聞金石絲竹之音,乃不\textcolor{red}{壞宅}。\footnote{共音恭,亦作龔,又作恭。共王,漢景帝之子,名餘。好,呼報反,下“好古”同。壞音怪,下同。\CJKunderwave{字林}作𣀤,雲公壞反,毀也。傳謂\CJKunderwave{春秋}也,一雲周易十翼,非經謂之傳。論如字,又音倫。科,苦禾反。科斗,蟲名,蝦蟆子。書形似之。}

{\noindent\shu\zihao{5}\fzkt “至魯”至“壞宅”。正義曰:欲雲得百篇之由,故序其事。漢景帝之子名餘,封於魯,為王,死諡曰共。存日以居於魯,近\CJKunderline{孔子}宅。好治宮室,故欲裒益,乃壞\CJKunderline{孔子}舊宅,以增廣其居。於所壞壁內得安國先人所藏古文虞夏商周之書及傳\CJKunderwave{論語}、\CJKunderwave{孝經},皆是科斗文字。王雖得此書,猶壞不止。又升\CJKunderline{孔子}廟堂,聞金鐘石磬絲琴竹管之音,以懼其神異乃止,不復敢壞宅也。上言藏家書於屋壁,此亦屋壁內得書也,亦得及傳\CJKunderwave{論語}、\CJKunderwave{孝經}等。不從約雲“得\CJKunderwave{尚書}”,而煩文言“虞夏商周之書”者,以壁內所得,上有題目“虞夏商周書”,其序直雲“書序”,皆無“尚”字,故其目錄亦然,故不雲“尚書”而言“虞夏商周之書”。安國亦以此知“尚”字是伏生所加。惟此壁內所無,則書本無“尚”字明矣。凡書,非經則謂之傳。言“及傳\CJKunderwave{論語}、\CJKunderwave{孝經}”,正謂\CJKunderwave{論語}、\CJKunderwave{孝經}是傳也。漢武帝謂東方朔云:“傳曰:‘時然後言,人不厭其言。’”又漢東平王劉雲與其太師策書云:“傳曰:‘陳力就列,不能者止。’”又成帝賜翟方進策書云:“傳曰:‘高而不危,所以長守貴也。’”是漢世通謂\CJKunderwave{論語}、\CJKunderwave{孝經}為傳也。以\CJKunderwave{論語}、\CJKunderwave{孝經}非先王之書,是\CJKunderline{孔子}所傳說,故謂之傳,所以異於先王之書也。上已雲“壞\CJKunderline{孔子}舊宅”,又云“乃不壞宅”者,初王意欲壞之,已壞其屋壁,聞八音之聲,乃止,餘者不壞,明知已壞者亦不敢居,故云“乃不壞宅”耳。 \par}

\textcolor{red}{悉以}書還\CJKunderline{孔氏}。科斗書廢已久,時人無能知者,以所聞伏生之書考論文義,定其可知者,為隸古定,更以竹簡寫之,增多伏生二十五篇。伏生又以\CJKunderwave{舜典}合於\CJKunderwave{堯典},\CJKunderwave{益稷}合於\CJKunderwave{皋陶謨},\CJKunderwave{盤庚}三篇合為一,\CJKunderwave{康王之誥}合於\CJKunderwave{顧命},復出此篇,並序,凡五十九篇,為四十六卷。其餘錯亂摩滅,弗可復知,悉上送官,藏之書府,以待\textcolor{red}{能者}。\footnote{隸音麗,謂用隸書寫古文。二十五篇謂\CJKunderwave{虞書·大禹謨},\CJKunderwave{夏書·五子之歌}、\CJKunderwave{胤徵},\CJKunderwave{商書·仲虺之誥}、\CJKunderwave{湯誥}、\CJKunderwave{伊訓}、\CJKunderwave{太甲}三篇、\CJKunderwave{鹹有一德}、\CJKunderwave{說命}三篇,\CJKunderwave{周書·泰誓}三篇、\CJKunderwave{武成}、\CJKunderwave{旅獒}、\CJKunderwave{微子之命}、\CJKunderwave{蔡仲之命}、\CJKunderwave{周官}、\CJKunderwave{君陳}、\CJKunderwave{畢命}、\CJKunderwave{君牙}、\CJKunderwave{冏命}。合舊音合,又如字,下同。皋音高,本又作咎。陶音遙,本又作繇。盤,步幹反,本又作般。復,扶又反,下同。五十九篇即今所行五十八篇,其一是百篇之序。謂\CJKunderwave{虞書·汩作}、\CJKunderwave{九共}九篇、\CJKunderwave{膏飫},\CJKunderwave{夏書·帝告}、\CJKunderwave{釐沃}、\CJKunderwave{湯徵}、\CJKunderwave{汝鳩}、\CJKunderwave{汝方},\CJKunderwave{商書·夏社}、\CJKunderwave{疑至}、\CJKunderwave{臣扈}、\CJKunderwave{典寶}、\CJKunderwave{明居}、\CJKunderwave{肆命}、\CJKunderwave{徂後}、\CJKunderwave{沃丁}、\CJKunderwave{鹹乂}四篇、\CJKunderwave{伊陟}、\CJKunderwave{原命}、\CJKunderwave{仲丁}、\CJKunderwave{河亶甲}、\CJKunderwave{祖乙}、\CJKunderwave{高宗之訓},\CJKunderwave{周書·分器}、\CJKunderwave{旅巢命}、\CJKunderwave{歸禾}、\CJKunderwave{嘉禾}、\CJKunderwave{成王政}、\CJKunderwave{將蒲姑}、\CJKunderwave{賄肅慎之命}、\CJKunderwave{亳姑},凡四十二篇,亡。上,時掌反。}

{\noindent\shu\zihao{5}\fzkt “悉以”至“能者”。正義曰:既雲王不壞宅,以懼神靈,因還其書。已前所得,言“悉以書還\CJKunderline{孔氏}”,則上“傳\CJKunderwave{論語}、\CJKunderwave{孝經}”等皆還之,故言“悉”也。科斗書,古文也,所謂蒼頡本體,周所用之,以今所不識,是古人所為,故名“古文”。形多頭粗尾細,狀腹團圓,似水蟲之科斗,故曰“科斗”也。以古文經秦不用,故云廢已久矣,時人無能知識者。\CJKunderline{孔君}以人無能知識之故,己欲傳之,故以所聞伏生之書,比校起發,考論古文之義。考文而云“義”者,以上下事義推考其文,故云“義”也。“定其可知”者,就古文內定可知識者為隸古定。不言就伏生之書而云以其所聞者,明用伏生書外亦考之。故云“可知者”,謂並伏生書外有可知,不徒伏生書內而已。言“隸古”者,正謂就古文體而從隸定之。存古為可慕,以隸為可識,故曰“隸古”,以雖隸而猶古。由此故謂\CJKunderline{孔君}所傳為古文也。古文者,蒼頡舊體,周世所用之文字。案班固\CJKunderwave{漢志}及許氏\CJKunderwave{說文},書本有六體:一曰指事,上下;二曰象形,日月;三曰形聲,江河;四曰會意,武信;五曰轉註,考老;六曰假借,令長。此造字之本也。自蒼頡以至今,字體雖變,此本皆同,古今不易也。自蒼頡以至周宣,皆蒼頡之體,未聞其異。宣王紀其史籀始有大篆十五篇,號曰篆籀,惟篆與蒼頡二體而已。衛恆曰:“蒼頡造書,觀於鳥跡,因而遂滋,則謂之字。字有六義,其文至於三代不改。及秦用篆書,焚燒先代典籍,古文絕矣。”許慎\CJKunderwave{說文}言自秦有八體:一曰大篆,二曰小篆,三曰刻符,四曰蟲書,五曰摹印,六曰署書,七曰殳書,八曰隸書。亡新居攝,以應制作,改定古文,使甄豐校定,時有六書:一曰古文,\CJKunderline{孔子}壁內書也;二曰奇字,即古字有異者;三曰篆書,即小篆,下杜人程邈所作也;四曰佐書,秦隸書也;五曰繆篆,所以摹印也;六曰鳥蟲書,所以書幡信也。由此而論,即秦罷古文而有八體,非古文矣。以至亡新六書並八體,亦用書之六體以造其字。其亡新六書於秦八體,用其小篆、蟲書、摹印、隸書,去其大篆、刻符、殳書、署書,而加以古文與奇字,其刻符及署書蓋同摹印,殳書同於繆篆,大篆正古文之別,以慕古故乃用古文與奇字而不用大篆也。是\CJKunderline{孔子}壁內古文即蒼頡之體,故\CJKunderline{鄭玄}云:“書初出屋壁,皆周時象形文字,今所謂科斗書。”以形言之為科斗,指體即周之古文。\CJKunderline{鄭玄}知者,若於周時秦世所有,至漢猶當識之,不得雲無能知者。又亡新古文亦云即\CJKunderline{孔氏}壁內古文,是其證也。或以古文即大篆,非也。何者?八體六書自大篆,與古文不同;又秦有大篆,若大篆是古文,不得雲“古文遂絕”,以此知大篆非古文也。六書古文與蟲書本別,則蟲書非科斗書也。\CJKunderline{鄭玄}雲周之象形文字者,總指六書象科斗之形,不謂六書之內“一曰象形”也。又云“更以竹簡寫之”,明留其壁內之本也。顧氏云:“策長二尺四寸,簡長一尺二寸。”“曾多伏生二十五篇”者,以壁內古文篇題殊別,故知“以\CJKunderwave{舜典}合於\CJKunderwave{堯典},\CJKunderwave{益稷}合於\CJKunderwave{皋陶謨}”。伏生之本亦壁內古文而合之者,蓋以老而口授之時,因誦而連之,故殊耳。其\CJKunderwave{盤庚}本當同卷,故有並也。\CJKunderwave{康王之誥}以一時之事,連誦而同卷,當以“王出在應門之內”為篇首,及以“王若曰,庶邦”亦誤矣。以伏生本二十八篇,\CJKunderwave{盤庚}出二篇,加\CJKunderwave{舜典}、\CJKunderwave{益稷}、\CJKunderwave{康王之誥}凡五篇為三十三篇,加所增二十五篇為五十八,加序一篇為五十九篇,雲“復出此篇,並序,凡五十九篇”。此雲“為四十六卷”者,謂除序也。下雲“定五十八篇,既畢”,不更雲卷數,明四十六卷故爾。又伏生二十九卷而序在外,故知然矣。此雲“四十六卷”者,不見安國明說,蓋以同序者同卷,異序者異卷,故五十八篇為四十六卷。何者?五十八篇內有\CJKunderwave{太甲}、\CJKunderwave{盤庚}、\CJKunderwave{說命}、\CJKunderwave{泰誓}皆三篇共卷,減其八,又\CJKunderwave{大禹謨}、\CJKunderwave{皋陶謨}、\CJKunderwave{益稷}又三篇同序共卷,其\CJKunderwave{康誥}、\CJKunderwave{酒誥}、\CJKunderwave{梓材}亦三篇同序共卷,則又減四,通前十二,以五十八減十二,非四十六卷而何?其\CJKunderwave{康王之誥}乃與\CJKunderwave{顧命}別卷,以別序故也。“其餘錯亂摩滅”,五十八篇外四十二篇也,以不可復知,亦上送官。其可知者己用竹簡寫得其本,亦俱送入府,故在秘府得有古文也。以後生可畏,或賢聖間出,故須藏之以待能整理讀之者。 \par}

\textcolor{red}{承詔}為五十九篇作傳,於是遂研精覃思,博考經籍,採\xpinyin*{摭}群言,以立訓傳。約文申義,敷暢厥旨,庶幾有補於\textcolor{red}{將來}。\footnote{為,於偽反。覃,徒南反,深也。思,息嗣反。採,本又作採。摭,之石反,一音之若反。敷,芳夫反。暢,醜亮反。}

{\noindent\shu\zihao{5}\fzkt “承詔”至“將來”。正義曰:安國時為武帝博士,\CJKunderline{孔君}考正古文之日,帝之所知,亦既定訖,當以聞於帝,帝令註解,故云“承詔為五十九篇作傳”。以注者多言曰“傳”,“傳”者,傳通故也。以“傳”名出自丘明。賓牟賈對\CJKunderline{孔子}曰“史失其傳”。又\CJKunderwave{喪服}儒者皆雲\CJKunderline{子夏}作傳,是“傳”名久矣。但大率秦漢之際,多名為“傳”;於後儒者以其傳多,或有改之別雲“註解”者;仍有同者,以當時之意耳。說者為例院“前漢稱傳,於後皆稱注”,誤矣。何者?馬融、王肅亦稱註名為“傳”,傳何有例乎?以聖道弘深,當須詳悉,於是研核精審,覃靜思慮以求其理,冀免乖違,既顧察經文,又取證於外,故須廣博推考群經六籍,又捃拾採摭群書之言,以此文證造立訓解,為之作傳。明不率爾。雖復廣證,亦不煩多,為傳直約省文,令得申盡其義。明文要義通,不假煩多也。以此得申,故能遍佈通暢\CJKunderwave{書}之旨意,是辭達而已,不求於煩。既義暢而文要,則觀者曉悟,故云庶幾有所補益於將來,讀之者得悟而有益也。敷,布也。厥,其也。庶,幸也。幾,冀也。\CJKunderwave{爾雅}有訓。既雲“經籍”,又稱“群言”者,“經籍”,五經是也;“群言”,子史是也。以\CJKunderwave{書}與經籍理相因通,故云“博考”;子史時有所須,故云“採摭”耳。案\CJKunderline{孔君}此傳辭旨不多,是“約文”也。要文無不解,是“申義”也。其義既申,故云敷暢其義之旨趣耳。考其此注,不但言少,\CJKunderwave{書}之為言多須詁訓,而\CJKunderline{孔君}為例,一訓之後,重訓者少,此亦約文也。 \par}

\CJKunderwave{\textcolor{red}{書序}},序所以為作者之意。昭然義見,宜相附近,故引之各冠其篇首,定五十八篇。既畢,會國有巫蠱事,經籍道息,用不復以聞,傳之子孫,以貽後代。若好古博雅君子,與我同志,亦所不\textcolor{red}{隱也}。\footnote{為,於偽反,又如字。見,賢遍反。冠,工亂反。巫蠱,漢武帝末徵和中,江充造蠱敗戾太子,故經籍道息焉。巫音無。蠱音古。貽,以之反。}

{\noindent\shu\zihao{5}\fzkt “書序”至“隱也”。正義曰:\CJKunderline{孔君}既言己立傳之意,又當斟酌所宜。而\CJKunderwave{書序}雖名為序,不是總陳書意泛論,乃篇篇各序作意,但作序者不敢廁於正經,故謙而聚於下。而注述者不可代作者之謙,須從利益而欲分之,從便雲序,序所以當篇為作此書之意,則是當篇作意觀序而昭然,意義顯見。既義見由序,此序宜各與其本篇相從附近,不宜聚於一處。故每篇引而分之,各冠加於篇首,令意昭見。序既分散,損其一篇,故定五十八篇。然此本承詔而作,作畢當以上奏聞知,但會值國家有巫蠱之事,好愛經籍之道滅息,假奏亦不能行用,為此之故,不復以此傳奏聞。亦以既傳成不得聞上,惟自傳於己之子孫,以遺與後世之人使行之。亦不敢望後世必行,故云若後世有好愛古道、廣博學問、志懷雅正如此之君子,冀能與我同於慕古之志,以行我道。我道得此人流行,亦所以傳不隱蔽。是弘道由人也。言“巫蠱”者,\CJKunderwave{王制}曰:“執左道以亂政者殺。”\CJKunderline{鄭玄}注云:“左道謂巫蠱之屬。”以非正道,故謂之左道。以蠱皆巫之所行,故云巫蠱。蠱者總名。\CJKunderwave{左傳}云:“惑蠱其君。”則蠱者怪惑之名。指體則藥毒害人者是,若行符厭俗之為魅,令人蠱惑夭年傷性皆是也。依\CJKunderwave{漢書},此時武帝末年,上已年老,淫惑鬼神,崇信巫術。由此奸人江充因而行詐,先於太子宮埋桐人,告上云:“太子宮有蠱氣。”上信之,使江充治之,於太子宮果得桐人。太子知己不為此,以江充故為陷己,因而殺之。而帝不知太子實心,謂江充言為實,即詔丞相劉屈犛發三輔兵討之。太子看長安因與鬥,不勝而出走,奔湖關自殺。此即巫蠱事也。言“不隱”者,不謂恐隱藏己道,以己道人所不知,懼其幽隱,人能行之使顯,為不隱蔽耳。易曰:“謙謙君子。”仁者好謙,而\CJKunderline{孔君}自作揄揚。雲君子知己者亦意在教世,欲令人睹此言,知己傳是深遠,因而有所曉寤,令之有益,故不可以苟謙也。亦猶\CJKunderline{孔子}曰:“何有於我哉”。 \par}

%%% Local Variables:
%%% mode: latex
%%% TeX-engine: xetex
%%% TeX-master: "../Main"
%%% End:
