%% -*- coding: utf-8 -*-
%% Time-stamp: <Chen Wang: 2024-04-02 11:42:41>

% {\noindent \zhu \zihao{5} \fzbyks } -> 注 (△ ○)
% {\noindent \shu \zihao{5} \fzkt } -> 疏

\chapter{卷七}

\section{甘誓第二}

\CJKunderline{啟}與有扈戰於甘之野,作\CJKunderwave{甘誓}。\footnote{夏啟嗣\CJKunderline{禹}位,伐有扈之罪。○啟,\CJKunderline{禹}子,嗣\CJKunderline{禹}為天子也。扈音戶。有扈,國名,與夏同姓。馬云:“姒姓之國,為無道者。”案京兆鄠縣即有扈之國也。甘,有扈郊地名,馬云:“南郊地也。”甘,水名,今在鄠縣西。誓,馬云:“軍旅曰誓,會同曰誥。”}

{\noindent\zhuan\zihao{6}\fzbyks 傳“夏啟”至“之罪”。正義曰:\CJKunderwave{孟子}稱,\CJKunderline{禹}薦益於天,七年,\CJKunderline{禹}崩之後,益避啟於箕山之陰,天下諸侯不歸益而歸啟,曰:“吾君之子也。”啟遂即天子位。\CJKunderwave{史記·夏本紀}稱,啟立,有扈氏不服,故伐之。蓋由自堯舜受禪相承,啟獨見繼父,以此不服,故云“夏啟嗣\CJKunderline{禹}立,伐有扈之罪”,言繼立者,見其由嗣立,故不服也。 \par}

{\noindent\shu\zihao{5}\fzkt “啟與”至“甘誓”。正義曰:夏王啟之時,諸侯有扈氏叛,王命率眾親征之。有扈氏發兵拒啟,啟與戰於甘地之野。將戰,集將士而誓戒之。史敘其事,作\CJKunderwave{甘誓}。 \par}

甘誓\footnote{甘,有扈郊地名。將戰先誓。}

{\noindent\zhuan\zihao{6}\fzbyks 傳“甘有”至“先誓”。正義曰:\CJKunderwave{地理志}扶風鄠縣,古扈國,夏啟所伐者也。“鄠”、“扈”音同,未知何故改也。啟伐有扈,必將至其國,乃出兵與啟戰,故以“甘”為有扈之郊地名。馬融云:“甘,有扈南郊地名。”計啟西行伐之,當在東郊。融則扶風人,或當知其處也。“將戰先誓”,誓是臨戰時也。\CJKunderwave{甘誓}、\CJKunderwave{牧誓}、\CJKunderwave{費誓}皆取誓地為名,\CJKunderwave{湯誓}舉其王號,\CJKunderwave{泰誓}不言“武誓”者,皆史官不同,故立名有異耳。\CJKunderwave{泰誓}未戰而誓,故別為之名。\CJKunderwave{秦誓}自悔而誓,非為戰誓,自約其心,故舉其國名。 \par}

{\noindent\shu\zihao{5}\fzkt “甘誓”。正義曰:發首二句敘其誓之由,其“王曰”已下皆是誓之辭也。\CJKunderwave{曲禮}云:“約信曰誓。”將與敵戰,恐其損敗,與將士設約,示賞罰之信也。將戰而誓,是誓之大者。\CJKunderwave{禮}將祭而號令齊百官,亦謂之誓。\CJKunderwave{周禮·大宰}云:“祀五帝則掌百官之誓戒。”\CJKunderline{鄭玄}云:“誓戒,要之以刑,重失禮也。”\CJKunderwave{明堂位}所謂“各揚其職,百官廢職,服大刑”,是誓辭之略也。彼亦是約信,但小於戰之誓。馬融云:“軍旅曰誓,會同曰誥。”“誥”、“誓”俱是號令之辭,意小異耳。 \par}

大戰於甘,乃召六卿。\footnote{天子六軍,其將皆命卿。將,子匠反。}王曰:“嗟!六事之人,\footnote{各有軍事,故曰六事。}予誓告汝:有扈氏威侮五行,怠棄三正,\footnote{五行之德,王者相承所取法。有扈與夏同姓,恃親而不恭,是則威虐侮慢五行,怠惰棄廢天地人之正道。言亂常。侮,亡甫反。正如字,徐音徵,馬云:“建子、建醜、建寅,三正也。”惰,徒臥反。}天用剿絕其命,\footnote{用其失道故。剿,截也。截絕,謂滅之。剿,子六反,\CJKunderwave{玉篇}子小反,馬本作巢,與\CJKunderwave{玉篇}、\CJKunderwave{切韻}同。}今予惟恭行天之罰。\footnote{恭,奉也,言欲截絕之。罰音伐。}


{\noindent\zhuan\zihao{6}\fzbyks 傳“天子”至“命卿”。正義曰:將戰而召六卿,明是卿為軍將。“天子六軍,其將皆命卿”,\CJKunderwave{周禮·夏官序}文也。\CJKunderline{鄭玄}云:“夏亦然,則三王同也。”經言“大戰”者,\CJKunderline{鄭玄}云:“天子之兵,故曰大。”孔無明說,蓋以六軍並行,威震多大,故稱“大戰”。 \par}

{\noindent\zhuan\zihao{6}\fzbyks 傳“各有”至“六事”。正義曰:卿為軍將,故云“乃召六卿”,及其誓之,非六卿而已。\CJKunderline{鄭玄}云:“變六卿言六事之人者,言軍吏下及士卒也。”下文戒左右與御,是遍敕在軍之士,步卒亦在其間。六卿之身及所部之人,各有軍事,故“六事之人”為總呼之辭。 \par}

{\noindent\zhuan\zihao{6}\fzbyks 傳“五行”至“亂帝”。正義曰:“五行”,水、火、金、木、土也。分行四時,各有其德。\CJKunderwave{月令}孟春三日,太史謁於天子,曰:“某日立春,盛德在木。”夏雲“盛德在火”,秋雲“盛德在金”,冬雲“盛德在水”。此五行之德,王者雖易姓,相承其所取法同也。言王者共所取法,而有扈氏獨侮慢之,所以為大罪也。且五行在人為仁、義、禮、智、信,威侮五行,亦為侮慢此五常而不行也。有扈與夏同姓,恃親而不恭天子,廢君臣之義,失相親之恩,五常之道盡矣,是“威侮五行”也。無所畏忌,作威虐而侮慢之,故云“威虐侮慢”。\CJKunderwave{易·說卦}云:“立天之道曰陰與陽,立地之道曰柔與剛,立人之道曰仁與義。”物之為大,無大於此者,\CJKunderwave{周易}謂之“三才”。人生天地之間,莫不法天地而行事,以此知“怠惰棄廢天地人之正道”。棄廢此道,“言亂常也”。孔、馬、鄭、王與皇甫謐等皆言有扈與夏同姓,並依\CJKunderwave{世本}之文。\CJKunderwave{楚語}云,昭王使觀射父傅太子,射父辭之曰:“堯有\CJKunderline{丹朱},舜有商均,夏有觀扈,周有管蔡。”是其“恃親而不恭”也。\CJKunderwave{周語}云,帝嘉\CJKunderline{禹}德,賜姓曰姒,\CJKunderline{禹}始得姓。有扈與夏同姓,則為啟之兄弟。知此者,蓋\CJKunderline{禹}未賜姓之前,以姒為姓,故\CJKunderline{禹}之親屬舊已姓姒,帝嘉其德,又以姒姓顯揚之。猶若\CJKunderline{伯夷}\CJKunderwave{國語}稱賜姓曰姜,然\CJKunderline{伯夷}是炎帝之後,未賜姓之前先為姜姓,與此同也。故有扈以為夏之同姓。 \par}

{\noindent\zhuan\zihao{6}\fzbyks 傳“用其”至“滅之”。正義曰:天子用兵,稱“恭行天罰”,諸侯討有罪,稱“肅將王誅”,皆示有所稟承,不敢專也。有扈既有大罪,宜其絕滅,故原天之意,言天用其失道之故,欲截絕其命,謂滅之也。“剿”是斬斷之義,故為截也。 \par}

左不攻於左,汝不恭命。\footnote{左,車左,左方主射。絕之也,治其職。}右不攻於右,汝不恭命。\footnote{右,車右,勇力之士,執戈矛以退敵。}御非其馬之正,汝不恭命。\footnote{御以正馬為政。三者有失,皆不奉我命。御,魚慮反。}


{\noindent\zhuan\zihao{6}\fzbyks 傳“左車”至“其職”。正義曰:歷言“左”、“右”及“御”,此三人人在一車之上也,故“左”為車左,則“右”為車右明矣宣十二年\CJKunderwave{左傳}云:“楚許伯御樂伯,攝叔為右,以致晉師。樂伯曰:‘吾聞致師者,左射以菆。’攝叔曰:‘吾聞致師者,右入壘,折馘執俘而還。’”是左方主射,右主擊刺,而御居中也。御言“正馬”,而左右不言所職者,以戰主殺敵,左右用兵是戰之常事,故略而不言;御惟主馬,故特言之,互相明也。此謂凡常兵車,甲士三人,所主皆如此耳。若將之兵車,則御者在左,勇力之士在右,將居鼓下,在中央,主擊鼓,與軍人為節度。成二年\CJKunderwave{左傳}說晉伐齊云:“晉解張御郤克,\CJKunderline{鄭玄}緩為右。卻克傷於矢,未絕鼓音,曰:‘餘病矣。’張侯曰:‘自始合,而矢貫餘手及肘,餘折以御,左輪朱殷,豈敢言病?’”郤克傷於矢而鼓音未絕,張侯為御而血染左輪,是御在左而將居中也。“攻”之為治,常訓也。“治其職”者,左當射人,右當擊刺,是其所掌職事也。 \par}

{\noindent\zhuan\zihao{6}\fzbyks 傳“御以”至“我命”。正義曰:“御以正馬為政”,言御之政事,事在正馬,故馬不正則罪之。\CJKunderwave{詩}云:“兩驂如手。”傳云:“進止如御者之手。”是為馬之正也。左、右與御三者有失,言“皆不奉我命”,以御在後,故總解之。 \par}

用命,賞於祖。\footnote{天子親征,必載遷廟之祖主行,有功則賞祖主前,示不專。}弗用命,戮於社,\footnote{天子親征,又載社主,謂之社事,不用命奔北者,則戮之於社主前。社主陰,陰主殺,親祖嚴社之義。戮音六。北如字,又音佩,軍走曰北。}予則孥戮汝。”\footnote{孥,子也。非但止汝身,辱及汝子。言恥累也。孥音奴,累,劣偽反。}

{\noindent\zhuan\zihao{6}\fzbyks 傳“天子”至“不專”。正義曰:\CJKunderwave{曾子問}云:“\CJKunderline{孔子}曰:‘天子巡守,以遷廟之主行,載於齊車,言必有尊也。’”巡守尚然,征伐必也。故云“天子親征,必載遷廟之祖主行,有功則賞祖主前,示不專”也。\CJKunderwave{周禮·大司馬}云:“若師不功,則厭而奉主車。”\CJKunderline{鄭玄}云:“厭,伏冠也。奉,猶送也。”送主歸於廟與社,亦是征伐載主之事也。 \par}

{\noindent\zhuan\zihao{6}\fzbyks 傳“天子”至“之義”。正義曰:定四年\CJKunderwave{左傳}云:“君以軍行,祓社釁鼓,祝奉以從。”是天子親征,又載社主行也。\CJKunderwave{郊特牲}云:“惟為社事,單出裡。”故以“社事”言之。“不用命奔北者,則戮之於社主之前”,“奔北”謂背陳走也。所以刑賞異處者,社主陰,陰主殺,則祖主陽,陽主生。\CJKunderwave{禮}左宗廟,右社稷,是祖陽而社陰。就祖賞,就社殺,親祖嚴社之義也。大功大罪則在軍賞罰,其遍敘諸勳,乃至太祖賞耳。 \par}

{\noindent\zhuan\zihao{6}\fzbyks 傳“孥子”至“恥累也”。正義曰:\CJKunderwave{詩}雲“樂爾妻孥”,對“妻”別文,是“孥”為子也。非但止辱汝身,並及汝子亦殺,言以恥惡累之。\CJKunderwave{湯誓}云:“予則孥戮汝。”傳曰:“古之用刑,父子兄弟罪不相及,今雲‘孥戮汝’,權以脅之,使勿犯。”此亦然也。 \par}

{\noindent\shu\zihao{5}\fzkt “大戰”至“戮汝”。正義曰:史官自先敘其事,啟與有扈大戰於甘之野,將欲交戰,乃召六卿,令與眾士俱集。王乃言曰:“嗟!”重其事,故嗟嘆而呼之:“汝六卿者,各有軍事之人。我設要誓之言以敕告汝:今有扈氏威虐侮慢五行之盛德,怠惰棄廢三才之正道,上天用失道之故,今欲截絕其命。天既如此,故我今惟奉行天之威罰,不敢違天也。我既奉天,汝當奉我。汝諸士眾在車左者,不治理於車左之事,是汝不奉我命。在車右者,不治理於車右之事,是汝不奉我命。御車者非其馬之正,令馬進退違戾,是汝不奉我命。汝等若用我命,我則賞之於祖主之前。若不用我命,則戮之於社主之前。所戮者非但止汝身而已,我則並殺汝子以戮辱汝。汝等不可不用我命以求殺敵。”戒之使齊力戰也。 \par}

\section{五子之歌第三【偽】}


\CJKunderline{太康}失邦,\footnote{啟子也。盤於遊田,不恤民事,為羿所逐,不得反國。}昆弟五人須於洛汭,作\CJKunderwave{五子之歌}。\footnote{\CJKunderline{太康}五弟與其母待\CJKunderline{太康}於洛水之北,怨其不反,故作歌。五子名字,書傳無聞,\CJKunderline{仲康}蓋其一也。須,馬云:“止也。”汭,如銳反,本又作內,音同。}


{\noindent\zhuan\zihao{6}\fzbyks 傳“\CJKunderline{太康}”至“作歌”。正義曰:“昆弟五人”,自有長幼,故稱“昆弟”,嫌是\CJKunderline{太康}之昆,故云“\CJKunderline{太康}之五弟”。 \par}

{\noindent\shu\zihao{5}\fzkt “\CJKunderline{太康}”至“之歌”。正義曰:\CJKunderline{啟}子\CJKunderline{太康},以遊畋棄民,為羿所逐,失其邦國。其未失國之前,畋於洛水之表,\CJKunderline{太康}之弟,更有昆弟五人,從\CJKunderline{太康}畋獵,與其母待\CJKunderline{太康}於洛水之北。\CJKunderline{太康}為羿所距,不得反國,其弟五人,即啟之五子,並怨\CJKunderline{太康},各自作歌。史敘其事,作\CJKunderwave{五子之歌}。 \par}

五子之歌\footnote{啟之五子,因以名篇。}


{\noindent\zhuan\zihao{6}\fzbyks 傳“啟之”至“名篇”。正義曰:直言“五子”,不知謂誰,故言“啟之五子”。\CJKunderline{太康}之弟,敘怨作歌,不言“五弟”而言“五子”者,以其迷祖之訓,故系父以言之。 \par}

{\noindent\shu\zihao{5}\fzkt “五子之歌”。正義曰:史述作歌之由,先敘失國之事,“其一曰”以下乃是歌辭。此五子作歌五章,每章各是一人之作,辭相連接,自為終始。初言“皇祖有訓”,未必則指怨\CJKunderline{太康}。必是五子之歌相顧,從輕至甚。“其一”、“其二”蓋是昆弟之次,或是作歌之次,不可知也。 \par}

\CJKunderline{太康}尸位以逸豫,\footnote{屍,主也。主以尊位,為逸豫不勤。逸,本又作佾。豫,本又作忬,音同。}滅厥德,黎民咸貳。\footnote{君喪其德,則眾民皆二心矣。黎,力兮反。喪,息浪反。}乃盤遊無度,\footnote{盤樂遊逸無法度。盤,步幹反,本或作槃。度如字。樂如字。}畋於有洛之表,十旬弗反。\footnote{洛水之表,水之南。十日曰旬。田獵過百日不還。畋音田。}有窮\CJKunderline{后羿},因民弗忍,距於河。\footnote{有窮,國名。羿,諸侯名。距\CJKunderline{太康}於河,不得入國,遂廢之。羿,五計反,徐胡細反。距音巨。}厥弟五人御其母以從,\footnote{御,待也,言從畋。從如字,或作才用反,非。}徯於洛之汭。五子咸怨,\footnote{待\CJKunderline{太康},怨其久畋失國。徯,胡啟反。}述\CJKunderline{大禹}之戒以作歌。\footnote{述,循也。歌以敘怨。}

{\noindent\zhuan\zihao{6}\fzbyks 傳“屍,主也”。正義曰:\CJKunderwave{釋詁}文。 \par}

{\noindent\zhuan\zihao{6}\fzbyks 傳“有窮”至“廢之”。正義曰:襄四年\CJKunderwave{左傳}曰:“夏之方衰也,\CJKunderline{后羿}自鉏遷於窮石。”然則羿居窮石,故曰“有窮,國名”。“窮”是諸侯之國,“羿”是其君之名也。\CJKunderwave{說文}云:“羿,帝嚳射官也。”賈逵云:“羿之先祖,世為先王射官,故帝賜羿弓矢使司射。”\CJKunderwave{淮南子}云:“堯時十日並生,堯使羿射九日而落之。”\CJKunderwave{楚辭·天問}云:“羿焉彃日烏解羽?”\CJKunderwave{歸藏易}亦云:“羿彃十日。”\CJKunderwave{說文}云:“彃者,射也。”此三者言雖不經以取信,要言帝嚳時有羿,堯時亦有羿,則羿是善射之號,非復人之名字。信如彼言,則不知羿名為何也。夏都河北,洛在河南,距\CJKunderline{太康}於河北,不得入國,遂廢\CJKunderline{太康}耳。羿猶立\CJKunderline{仲康},不自立也。 \par}

{\noindent\zhuan\zihao{6}\fzbyks 傳“述循”至“敘怨”。正義曰:“述,循”,\CJKunderwave{釋詁}文。循其所戒,用作歌以敘怨也。其一曰“皇祖有訓”,其二曰“訓有之”,是“述\CJKunderline{大禹}之戒”也。其三恨亡國都,其四恨絕宗祀,其五言追悔無及,直是指怨\CJKunderline{太康},非為述祖戒也。本述戒作歌,因即言及時事,故言祖戒以總之。 \par}

{\noindent\shu\zihao{5}\fzkt “\CJKunderline{太康}”至“作歌”。正義曰:天子之在天位,職當牧養兆民。\CJKunderline{太康}主以尊位,用為逸豫,滅其人君之德,眾人皆有二心。\CJKunderline{太康}乃復愛樂遊,逸無有法度,畋獵於洛水之表,一出十旬不反。有窮國君其名曰羿,因民不能堪忍\CJKunderline{太康}之惡,率眾距之於河,不得反國。\CJKunderline{太康}初去之時,其弟五人侍其母以從\CJKunderline{太康}。\CJKunderline{太康}畋於洛南,五弟待於洛北,\CJKunderline{太康}久而不反,致使羿距於河。五子皆怨\CJKunderline{太康},追述\CJKunderline{大禹}之戒以作歌,而各敘已怨之志也。其弟待母以從\CJKunderline{太康},\CJKunderline{太康}初去即然。待於洛水之北,以冀\CJKunderline{太康}速反。羿既距之,五子乃怨。史述\CJKunderline{太康}之惡既盡,然後言其作歌,故令“羿距”之文乃在“母從”之上,作文之勢當然也。 \par}

其一曰:“皇祖有訓,民可近,不可下,\footnote{皇,君也。君祖\CJKunderline{禹}有訓戒。近謂親之。下謂失分。近,附近之近。分,扶問反。}民惟邦本,本固邦寧。\footnote{言人君當固民以安國。}予視天下,愚夫愚婦,一能勝予,\footnote{言能畏敬小民,所以得眾心。}一人三失,怨豈在明?不見是圖。\footnote{三失,過非一也。不見是謀,備其微。三如字,又息暫反。見,賢遍反。}予臨兆民,懍乎若朽索之馭六馬,\footnote{十萬曰億,十億曰兆,言多。懍,危貌。朽,腐也。腐索馭六馬,言危懼甚。懍,力甚反。朽,許久反。馭音御。腐,扶甫反。}為人上者,奈何不敬?”\footnote{能敬則不驕,在上不驕,則高而不危。}


{\noindent\zhuan\zihao{6}\fzbyks 傳“皇君”至“失分”。正義曰:“皇,君”,\CJKunderwave{釋詁}文。述\CJKunderline{禹}之戒,知“君祖”是\CJKunderline{禹},\CJKunderline{禹}有訓也。“民可近”者,據“君”為文。“近”謂親近之也,“下”謂卑下輕忽之,失本分也。奪其農時,勞以橫役,是失分也,故下雲“予視天下,愚夫愚婦,一能勝予”,是畏敬下民也。 \par}

{\noindent\zhuan\zihao{6}\fzbyks 傳“言能”至“眾心”。正義曰:我視愚夫愚婦,當能勝我身,是畏敬小民也。由能畏敬小民,故以小民從命,是“得眾心”也。 \par}

{\noindent\zhuan\zihao{6}\fzbyks 傳“三失”至“其微”。正義曰:顧氏云:“怨豈在明?未必皆在明著之時,必於未形之日思善道以自防衛之。”是備慎其微也。 \par}

{\noindent\zhuan\zihao{6}\fzbyks 傳“十萬”至“懼甚”。正義曰:古數十萬曰億,十億曰兆,言多也。懍懍,心懼之意,故為危貌。“朽,腐”,常訓也。腐索馭六馬,索絕馬驚,馬驚則逸,言危懼甚也。經傳之文,惟此言“六馬”,漢世此經不傳,餘書多言駕四者,\CJKunderwave{春秋·公羊}說天子駕六,\CJKunderwave{毛詩}說天子至大夫皆駕四,許慎案\CJKunderwave{王度記}雲天子駕六,\CJKunderline{鄭玄}以\CJKunderwave{周禮}校人養馬,“乘馬一師四圉”,四馬曰乘,\CJKunderwave{康王之誥}雲“皆布乘黃朱”,以為天子駕四。漢世天子駕六,非常法也。然則此言馬多懼深,故舉六以言之。 \par}

{\noindent\shu\zihao{5}\fzkt “其一”至“不敬”。正義曰:我君祖\CJKunderline{大禹}有訓戒之事,言民可親近,不可卑賤輕下。令其失分,則人懷怨,則事上之心不固矣。民惟邦國之本,本固則邦寧。言在上不可使人怨也。我視天下之民,愚夫愚婦,一能過勝我,安得不敬畏之也?所以畏其怨者,一人之身,三度有失;凡所過失,為人所怨,豈在明著?大過皆由小事而起。言小事不防,易致大過,故於不見細微之時,當於是豫圖謀之,使人不怨也。我臨兆民之上,常畏人怨,懍懍乎危懼,若腐索之馭六馬。索絕則馬逸,言危懼之甚。人之可畏如是,為民上者奈何不敬慎乎?怨\CJKunderline{太康}之不恤下民也。 \par}

其二曰:“訓有之,內作色荒,外作禽荒。\footnote{作,為也。迷亂曰荒。色,女色。禽,鳥獸。}

{\noindent\zhuan\zihao{6}\fzbyks 傳“作為”至“鳥獸”。正義曰:“作,為”,\CJKunderwave{釋言}文。昭元年\CJKunderwave{左傳}晉平公近女色過度,惑以喪志。\CJKunderwave{老子}云:“馳騁田獵,令人心發狂。”好色好田則精神迷亂,故“迷亂曰荒”。女有美色,男子悅之,經傳通謂女人為“色”。獵則鳥獸並取,故以“禽”為鳥獸也。 \par}

甘酒嗜音,峻宇雕牆。\footnote{甘嗜無厭足。峻,高大。雕,飾畫。甘,一音戶甘反。嗜,市志反。峻,思俊反。牆,慈羊反。厭,於鹽反,又於豔反。有一於此,未或不亡。”此六者,棄德之君必有其一。}有一必亡,況兼有乎!

其三曰:“惟彼\CJKunderline{陶唐},有此冀方。\footnote{\CJKunderline{陶唐},\CJKunderline{帝堯}氏,都冀州,統天下四方。}

{\noindent\zhuan\zihao{6}\fzbyks 傳“\CJKunderline{陶唐}”至“四方”。正義曰:\CJKunderwave{世本}云:“\CJKunderline{帝堯}為\CJKunderline{陶唐氏}。”韋昭云:“陶、唐皆國名,猶湯稱殷商也。”案書傳皆言堯以唐侯升為天子,不言封於\CJKunderline{陶唐},“\CJKunderline{陶唐}”二字或共為地名,未必如昭言也。以天子王有天下,非獨冀州一方,故以“冀方”為“都冀州,統天下四方”。堯都平陽,舜都蒲阪,\CJKunderline{禹}都安邑,相去不盈二百,皆在冀州,自堯以來其都不出此地,故舉\CJKunderline{陶唐}以言之。 \par}

今失厥道,亂其紀綱,乃厎滅亡。”\footnote{言失堯之道,亂其法制,自致滅亡。厎,之履反。}

其四曰:“明明我祖,萬邦之君。有典有則,貽厥子孫。\footnote{君萬國為天子。典謂經籍。則,法。貽,遺也。言仁及後世。貽,以之反。遺,唯季反。}關石和鈞,王府則有。荒墜厥緒,覆宗絕祀。”\footnote{金鐵曰石,供民器用,通之使和平,則官民足。言古制存,而\CJKunderline{太康}失其業,以取亡。覆,芳服反。供音恭。}


{\noindent\zhuan\zihao{6}\fzbyks 傳“君萬”至“後世”。正義曰:“萬邦之君”,謂君統萬國為天子也。“典”謂先王之典,可憑據而行之,故為經籍。“則,法”,\CJKunderwave{釋詁}文。“典”謂先王舊典,“法”謂當時所制,其事不為大異,重言以備文耳。“貽,遺”,\CJKunderwave{釋言}文。以典法遺子孫,言仁恩及後世。 \par}

{\noindent\zhuan\zihao{6}\fzbyks 傳“金鐵”至“取亡”。正義曰:“關”者,通也。名“石”而可通者,惟衡量之器耳。\CJKunderwave{律曆志}云:“二十四銖為兩,十六兩為斤,三十斤為鈞,四鈞為石。”是石為稱之最重,以石而稱則為重物,故“金鐵曰石”。言絲綿止於斤兩,金鐵乃至於石,舉“石”而言之,則所稱之物皆通之也。傳取金鐵重物以解言“石”之意,非謂所關通者惟金鐵耳。米粟則鬥斛以量之,布帛則丈尺以度之,惟言關通權衡,則度量之物,懋遷有無,亦關通矣,舉一以言之耳。衡石所稱之物,以供民之器用,其土或有或無,通使和平也。\CJKunderwave{論語}云:“百姓足,君孰與不足?”民既足用,則官亦富饒,故“通之使和平,則官民皆足”。有典有法可依而行,官民足可坐而守,言古制存,而\CJKunderline{太康}失其業,所以亡也。訓“緒”為業,費氏、顧氏等意云,通金鐵於人,官不禁障,民得取之以供器用。器既具,所以上下充足。以金鐵皆從石而生,則金鐵亦石之類也。故\CJKunderwave{漢書·五行志}雲石為怪異,入金不從革之條。費、顧之義,亦得通也。 \par}

{\noindent\shu\zihao{5}\fzkt “其四”至“絕祀”。正義曰:有明明之德,我祖\CJKunderline{大禹}也。以有明德為萬邦之君,謂為天子也。有治國之典,有為君之法,遺其後世之子孫,使法則之。又關通衡石之用,使之和平。人既足用,王之府藏則皆有矣。典存國富,宜以為政,今\CJKunderline{太康}荒廢墜失其業,覆滅宗族,斷絕祭祀。言\CJKunderline{太康}棄典法,所以滅宗祀也。 \par}

其五曰:“嗚呼!曷歸?予懷之悲。\footnote{曷,何也。言思而悲。曷,戶割反。}萬姓仇予,予將疇依?\footnote{仇,怨也。言當依誰以復國乎?}鬱陶乎予心,顏厚有忸怩。\footnote{鬱陶,言哀思也。顏厚,色愧。忸怩,心慚,慚愧於仁人賢士。鬱音蔚。陶音桃。鬱陶,憂思也。忸,女六反。怩,女姬反,徐乃私反。思,息嗣反。}弗慎厥德,雖悔可追?”\footnote{言人君行己不慎其德,以速滅敗,雖欲改悔,其可追及乎?言無益。雖如字,或作雎。}


{\noindent\zhuan\zihao{6}\fzbyks 傳“仇怨”至“國乎”。正義曰:桓二年\CJKunderwave{左傳}雲“怨耦曰仇”,故為怨也。羿距於河,不得復反,乃思\CJKunderline{太康},欲歸依之,言當依誰以復國乎? \par}

{\noindent\zhuan\zihao{6}\fzbyks 傳“鬱陶”至“賢士”。正義曰:\CJKunderwave{孟子}稱舜弟象見舜云:“思君正鬱陶。”“鬱陶”,精神憤結積聚之意,故為哀思也。\CJKunderwave{詩}云:“顏之厚矣。”羞愧之情見於面貌,似如面皮厚然,故以“顏厚”為色愧忸怩,羞不能言,心慚之狀。小人不足以知得失,故“慚愧於仁人賢士”。 \par}

{\noindent\shu\zihao{5}\fzkt “其五”至“可追”。正義曰:嗚呼!\CJKunderline{太康}已覆滅矣,我將何所依歸?我以此故,思之而悲。\CJKunderline{太康}為惡,毒遍天下,萬姓皆共仇我,我將誰依就乎?鬱陶而哀思乎,我之心也!我以此故,外貌顏厚而內情忸怩羞慚。由\CJKunderline{太康}不慎其德,以致此見距,雖欲改悔,其可追及之乎?事已往矣,不可如何。從首漸怨,至此為深,皆是羿距時事也。 \par}

\section{胤徵第四【偽】}


\CJKunderline{羲}、\CJKunderline{和}湎淫,廢時亂日,\footnote{羲氏、和氏,世掌天地四時之官,自\CJKunderline{唐}、\CJKunderline{虞}至三代,世職不絕。承\CJKunderline{太康}之後,沈湎於酒,過差非度,廢天時,亂甲乙。湎,徐音緬,面善反。差,初賣反,又初佳反。}胤往徵之,作\CJKunderwave{胤徵}。\footnote{胤國之君受王命往徵之。胤,國名。}

胤徵\footnote{奉辭罰罪曰徵。}


{\noindent\zhuan\zihao{6}\fzbyks 傳“羲氏”至“甲乙”。正義曰:“羲氏、和氏世掌天地四時之官”,\CJKunderwave{堯典}所言是其事也。羲和是重黎之後,\CJKunderwave{楚語}稱堯育重黎之後,使典天地,以至於夏商,是“自\CJKunderline{唐}、\CJKunderline{虞}至三代,世職不絕”,故此時羲和仍掌時日。以\CJKunderline{太康}逸豫,臣亦縱弛。此承\CJKunderline{太康}之後,於今仍亦懈惰,沈湎於酒,過差非度,廢天時,亂甲乙,是其罪也。經雲“酒荒於厥邑”,惟言荒酒,不言好色,故訓“淫”為過,言耽酒為過差也。聖人作歷數以紀天時,不存歷數,是“廢天時”也。日以甲乙為紀,不知日食,是“亂甲乙”也。 \par}

{\noindent\zhuan\zihao{6}\fzbyks 傳“奉辭罰罪”。正義曰:奉責讓之辭,伐不恭之罪,名之曰“徵”。徵者,正也。伐之以正其罪。 \par}

{\noindent\shu\zihao{5}\fzkt “羲和”至“胤徵”。正義曰:羲氏、和氏,世掌天地四時之官,今乃沈湎於酒,過差非度,廢天時,亂甲乙,不以所掌為意,胤國之侯受王命往徵之。史敘其事,作\CJKunderwave{胤徵}。 \par}

惟\CJKunderline{仲康}肇位四海,\footnote{羿廢\CJKunderline{太康},而立其弟\CJKunderline{仲康}為天子。○肇音兆。}\CJKunderline{胤侯}命掌六師。\footnote{\CJKunderline{仲康}命\CJKunderline{胤侯}掌王六師,為大司馬。}\CJKunderline{羲}、\CJKunderline{和}廢厥職,酒荒於厥邑,\footnote{舍其職官,還其私邑,以酒迷亂,不修其業。舍音舍。}\CJKunderline{胤後}承王命徂徵。\footnote{徂,往也,就其私邑往討之。}


{\noindent\zhuan\zihao{6}\fzbyks 傳“羿廢”至“天子”。正義曰:以羿距\CJKunderline{太康}於河,於時必廢之也。\CJKunderwave{夏本紀}云:“\CJKunderline{太康}崩,弟\CJKunderline{仲康}立。”襄四年\CJKunderwave{左傳}云:“羿因夏民以伐夏政。”則羿於其後篡天子之位,\CJKunderline{仲康}不能殺羿,必是羿握其權,知\CJKunderline{仲康}之立,是羿立之矣,故云“羿廢\CJKunderline{太康},而立其弟\CJKunderline{仲康}為天子”。計\CJKunderwave{五子之歌},\CJKunderline{仲康}當是其一,\CJKunderline{仲康}必賢於\CJKunderline{太康},但形勢既衰,政由羿耳。羿在夏世為一代大賊,\CJKunderwave{左傳}稱羿既篡位,寒浞殺之。羿滅夏後相,相子少康始滅浞復夏政。計羿、浞相承,向有百載,為夏亂甚矣。而\CJKunderwave{夏本紀}云:“\CJKunderline{太康}崩,其弟\CJKunderline{仲康}立。\CJKunderline{仲康}崩,子相立。相崩,子少康立。”都不言羿、浞之事,是馬遷之說疏矣。 \par}

{\noindent\shu\zihao{5}\fzkt “惟\CJKunderline{仲康}”至“徂徵”。正義曰:惟\CJKunderline{仲康}始即王位,臨四海,胤國之侯受王命為大司馬,掌六師。於是有羲氏、和氏廢其所掌之職,縱酒荒迷,亂於私邑,胤國之君承王命往徵之。 \par}

告於眾曰:“嗟予有眾,\footnote{誓敕之。}聖有謨訓,明徵定保。\footnote{徵,證。保,安也。聖人所謀之教訓,為世明證,所以定國安家。}先王克謹天戒,臣人克有常憲,\footnote{言君能慎戒,臣能奉有常法。}百官修輔,厥後惟明明。\footnote{修職輔君,君臣俱明。}每歲孟春,遒人以木鐸徇於路,\footnote{遒人,宣令之官。木鐸,金鈴木舌,所以振文教。遒,在由反。鐸,待洛反。鈴音令。}官師相規,工執藝事以諫。\footnote{官眾,從官。更相規闕,百工各執其所治技藝以諫,諫失常。藝,本又作埶。更音庚。技,其綺反。}其或不恭,邦有常刑。\footnote{言百官廢職,服大刑。}

{\noindent\zhuan\zihao{6}\fzbyks 傳“徵證”至“安家”。正義曰:成八年\CJKunderwave{左傳}稱晉殺趙括,欒、郤為徵。“徵”是證驗之義,故為證也。能自保守是安定之義,故為安也。聖人將為教訓,必謀而後行,故言“所謀之教訓”。聖人之言,必有其驗,故為“世之明證”。用聖人之謨訓,必有成功,故“所以定國安家”。 \par}

{\noindent\zhuan\zihao{6}\fzbyks 傳“言君”至“常法”。正義曰:王者代天理官,故稱“天戒”。臣人奉主法令,故言“常憲”。君當奉天,臣當奉君,言君能戒慎天戒也,臣能奉有常法,奉行君法也。此謂大臣,下雲“百官修輔”,謂眾臣。 \par}

{\noindent\zhuan\zihao{6}\fzbyks 傳“遒人”至“文教”。正義曰:以執木鐸徇於路,是宣令之事,故言“宣令之官”。\CJKunderwave{周禮}無此官,惟\CJKunderwave{小宰}云:“正歲,帥理官之屬而觀治象之法,徇以木鐸曰:‘不用法者,國有常刑。’”宣令之事,略與此同。此似別置其官,非如周之小宰。名曰“遒人”,不知其意,蓋訓“遒”為聚,聚人而令之,故以為名也。\CJKunderwave{禮}有“金鐸”、“木鐸”,“鐸”是鈴也,其體以金為之,明舌有金木之異,知木鐸是木舌也。\CJKunderwave{周禮}教鼓人“以金鐸通鼓”,大司馬“教振旅,兩司馬執鐸”,\CJKunderwave{明堂位}雲“振木鐸於朝”,是武事振金鐸,文事振木鐸。今雲“木鐸”,故云“所以振文教”也。 \par}

{\noindent\zhuan\zihao{6}\fzbyks 傳“官眾”至“失常”。正義曰:“相規”,相平等之辭,故“官眾”謂“眾官”,“相規”謂“更相規闕”。平等有闕,已尚相規,見上之過,諫之必矣。“百工各執其所治技藝以諫”,謂被遣作器,工有奢儉,若\CJKunderwave{月令}雲“無作淫巧,以蕩上心”,見其淫巧不正,當執之以諫,諫失常也。百工之賤,猶令進諫,則百工以上,不得不諫矣。 \par}

{\noindent\zhuan\zihao{6}\fzbyks 傳“言百”至“大刑”。正義曰:“百官廢職,服大刑”,\CJKunderwave{明堂位}文也。顧氏云:“百官群臣其有廢職懈怠不恭謹者,國家當有常刑。” \par}

{\noindent\shu\zihao{5}\fzkt “告於”至“常刑”。正義曰:\CJKunderline{胤侯}將徵羲和,告於所部之眾曰:“嗟乎!我所有之眾人,聖人有謨之訓,所以為世之明證,可以定國安家。其所謀者,言先王能謹慎敬畏天戒,臣人者能奉先王常法,百官修常職輔其君,君臣相與如是,則君臣俱明,惟為明君明臣。”言君當謹慎以畏天,臣當守職以輔君也。“先王恐其不然,大開諫爭之路。每歲孟春,遒人之官以木鐸徇於道路,以號令臣下,使在官之眾更相規闕;百工雖賤,令執其藝能之事以諫上之失常。其有違諫不恭謹者,國家則有常刑”。 \par}

“惟時羲和,顛覆厥德,\footnote{顛覆言反倒。將陳羲和所犯,故先舉孟春之令,犯令之誅。覆,芳服反。倒,丁老反。}沈亂於酒,畔宮離次,\footnote{沈謂醉冥。失次位也。離如字,又力智反。冥,莫定反,又亡丁反。}俶擾天紀,遐棄厥司。\footnote{俶,始。擾,亂。遐,遠也。紀謂時日。司,所主也。俶,本又作{\hanaa 㑐},亦作叔,同尺六反。擾,而小反。}乃季秋月朔,辰弗集於房,\footnote{辰,日月所會。房,所舍之次。集,合也。不合即日食可知。}

{\noindent\zhuan\zihao{6}\fzbyks 傳“顛覆”至“之誅”。正義曰:“顛覆言反倒”,謂人反倒也。人當豎立,今乃反倒,猶臣當事君,今乃廢職,似人之反倒然。言臣以事君為德,故言“顛覆厥德”。\CJKunderline{胤侯}將陳羲和之罪,故先舉孟春之令,犯令之誅,舉輕以見重,小事犯令猶有常刑,況叛官離次為大罪乎! \par}

{\noindent\zhuan\zihao{6}\fzbyks 傳“沈謂”至“次位”。正義曰:沒水謂之沈,大醉冥然,無所復知,猶瀋水然,故謂醉為“沈”。 \par}

{\noindent\zhuan\zihao{6}\fzbyks 傳“俶始”至“所主”。正義曰:“俶,始”、“遐,遠”皆\CJKunderwave{釋詁}文。“擾”謂煩亂,故為亂也。\CJKunderwave{洪範}五紀,“五曰歷數”,歷數所以紀天時。此言“天紀”,謂時日。此時日之事是羲和所司,言棄其所主。 \par}

{\noindent\zhuan\zihao{6}\fzbyks 傳“辰日”至“可知”。正義曰:昭七年\CJKunderwave{左傳}曰,晉侯問於士文伯曰:“何謂辰?”對曰:“日月之會是謂辰。”是“辰”為日月之會。日月俱右行於天,日行遲,月行疾,日每日行一度,月日行十三度十九分度之七,計二十九日過半,月已行天一周,又逐及日而與日聚會,謂此聚會為“辰”。一歲十二會,故為十二辰,即子、醜、寅、卯之屬是也。“房”謂室之房也,故為“所舍之次”。計九月之朔,日月當會於大火之次。\CJKunderwave{釋言}云:“集,會也。”會即是合,故為“合”也。日月當聚會共舍,今言日月不合於舍,則是日食可知也。日食者,月掩之也,月體掩日,日被月映,即不成共處,故以不集言日食也。或以為“房”謂房星,九月日月會於大火之次。房、心共為大火,言辰在房星,事有似矣。知不然者,以集是止舍之處,言其不集於舍,故得以表日食;若言不集於房星,似太遲太疾,惟可見歷錯,不得以表日食也。且日之所在,星宿不見,正可推算以知之,非能舉目見之。君子慎疑,寧當以日在之宿為文,以此知其必非房星也。 \par}

瞽奏鼓、嗇夫馳,庶人走。\footnote{凡日食,天子伐鼓於社,責上公。瞽,樂官,樂官進鼓則伐之。嗇夫,主幣之官,馳取幣禮天神。眾人走,供救日食之百役也。嗇音色。馳,車馬曰馳。走,步曰走。供音恭。}羲和屍厥官,罔聞知,\footnote{主其官而無聞知於日食之變異,所以罪重。}昏迷於天象,以干先王之誅。\footnote{暗錯天象,言昏亂之甚。干,犯也。}政典曰:‘先時者殺無赦,\footnote{政典,夏後為政之典籍。若\CJKunderwave{周官}六卿之治典。先時,謂曆象之法,四時節氣,弦望晦朔。先天時則罪死無赦。先,悉薦反,又如字,注“先時”、“先天”同。赦亦作亦攵。治,直吏反。}不及時者殺無赦。’\footnote{不及謂曆象後天時。雖治其官,苟有先後之差,則無赦,況廢官乎!後,胡豆反。}

{\noindent\zhuan\zihao{6}\fzbyks 傳“凡日”至“百役”。正義曰:文十五年\CJKunderwave{左傳}云:“日有食之,天子不舉,伐鼓於社。諸侯用幣於社,伐鼓於朝。”杜預以為“伐鼓於社”,“責群陰”也。此傳言“責上公”者,\CJKunderwave{郊特牲}云:“社祭土而主陰氣也,君南向北墉下,答陰之義也。”是言社主陰也。日食陰侵陽,故杜預以為“責群陰”也。昭二十九年\CJKunderwave{左傳}云:“封為上公,祀為貴神。社稷五祀,是尊是奉。”是社祭句龍為上公之神也。日食臣侵君之象,故傳以為“責上公”,亦當群陰上公並責之也。\CJKunderwave{周禮}瞽蒙之官掌作樂,瞽為樂官。樂官用無目之人,以其無目,於音聲審也。\CJKunderwave{詩}雲“奏鼓簡簡”,謂伐鼓為奏鼓,知“樂官進鼓則伐之”。\CJKunderwave{周禮·太僕}:“軍旅、田役,贊王鼓。救日月,亦如之。”\CJKunderline{鄭玄}云:“王通鼓,佐聲其餘面。”則救日之時,王或親鼓。莊二十五年\CJKunderwave{穀梁傳}曰:“天子救日,置五麾,陳五兵、五鼓。”陳既多,皆樂人伐之。\CJKunderwave{周禮}無嗇夫之官,\CJKunderwave{禮}云:“嗇夫承命,告於天子。”\CJKunderline{鄭玄}云:“嗇夫蓋司空之屬也。”嗇夫主幣,\CJKunderwave{禮}無其文,此雲“嗇夫馳”,必馳走有所取也。\CJKunderwave{左傳}云:“諸侯用幣。”則天子亦當有用幣之處,嗇夫必是主幣之官,馳取幣也。社神尊於諸侯,故諸侯用幣於社以請救。天子伐鼓於社,必不用,知嗇夫“馳取幣禮天神”。“庶人走”,蓋是庶人在官者,謂諸侯胥徒也。其走必有事,知為“供救日食之百役”也。\CJKunderwave{曾子問}云:“諸侯從天子救日食,各以方色與其兵。”\CJKunderwave{周禮·庭氏}云:“救日之弓矢。”是救日必有多役,庶人走供之。鄭注\CJKunderwave{庭氏}云,以救日為太陽之弓,救月為太陰之弓,救日以枉矢,救月以恆矢。其鼓則蓋用祭天之雷鼓也。昭十七年“夏六月甲戌朔,日有食之”。\CJKunderwave{左傳}云:“季平子曰:‘惟正月朔,慝未作,日有食之,於是乎有伐鼓用幣,禮也,其餘則否。’太史曰:‘在此月也,當夏四月,是謂孟夏。’”如彼傳文,惟夏四月有伐鼓用幣禮,余月則不然,此以九月日食亦奏鼓用幣者,顧氏云:“夏禮異於周禮也。” \par}

{\noindent\zhuan\zihao{6}\fzbyks 傳“政典”至“無赦”。正義曰:\CJKunderline{胤侯},夏之卿士,引政典而不言古典,則當時之書,知是“夏後為政之典籍”也。\CJKunderwave{周禮}:“太宰掌建邦之六典,以佐王治邦國。一曰治典,二曰教典,三曰禮典,四曰政典,五曰刑典,六曰事典。”若周官六卿之治典,謂此也。“先時”、“不及”者,謂此曆象之法,四時節氣,弦望晦朔,不得先天時,不得後天時。四時時各九十日有餘,分為八節,節各四十五日有餘也。節氣者,周天三百六十五日四分日之一,四時分之,均分為十二月,則月各得三十日十六分日之七,以初為節氣,半為中氣,故一歲有二十四氣也。計十二月,每月二十九日強半也。以月初為朔,月盡為晦,當月之中,日月相望,故以月半為望。望去晦、朔,皆不滿十五日也。又半此望去晦、朔之數,名之曰弦。弦者,言其月光正半如弓弦也。晦者,月盡無月,言其暗也。朔者,蘇也,言月死而更蘇也。先天時者,所名之日,在天時之先。假令天之正時,當以甲子為朔,今歷乃以癸亥為朔,是造歷先天時也;若以乙丑為朔,是造歷後天時也。後即是“不及時”也。其氣、望等皆亦如此。 \par}

{\noindent\shu\zihao{5}\fzkt “惟時”至“無赦”。正義曰:言不諫尚有刑,廢職懈怠,是為大罪。惟是羲和,顛倒其奉上之德,而沉沒昏亂於酒,違叛其所掌之官,離其所居位次,始亂天之紀綱,遠棄所主之事。乃季秋九月之朔,日月當合於辰。其日之辰,日月不合於舍;不得合辰,謂日被月食,日有食之。\CJKunderwave{禮}有救日之法,於是瞽人樂官進鼓而擊之,嗇夫馳騁而取幣以禮天神,庶人奔走供救日食之百役。此為災異之大,群官促遽若此,羲和主其官而不聞知日食,是大罪也。此羲和昏暗迷錯於天象,以犯先王之誅,此罪不可赦也。故先王為政之典曰:“主歷之官,為歷之法,節氣先天時者殺無赦,不及時者殺無赦。”失前失後尚猶合殺,況乎不知日食;其罪不可赦也,況彼罪之大。言已所以徵也。 \par}

“今予以爾有眾,奉將天罰。\footnote{將,行也。奉王命行王誅,謂殺湎淫之身,立其賢子弟。}爾眾士同力王室,尚弼予欽承天子威命。\footnote{以天子威命督其士眾,使用命。}火炎崐岡,玉石俱焚。\footnote{山脊曰岡。崐山出玉,言火逸而害玉。崐音昆。}天吏逸德,烈於猛火。\footnote{逸,過也。天王之吏為過惡之德,其傷害天下,甚於火之害玉。猛火烈矣,又烈於火。}殲厥渠魁,脅從罔治。\footnote{殲,滅。渠,大。魁,帥也。指謂羲和罪人之身,其脅從距王師者,皆無治。殲,子廉反。魁,苦回反。脅,虛業反。帥,色類反。}舊染汙俗,咸與惟新。\footnote{言其餘人久染汙俗,本無噁心,皆與更新,一無所問。汙,烏故反,汙辱之汙;又音烏,涴泥著物也,一音烏臥反。}嗚呼!威克厥愛,允濟。\footnote{嘆能以威勝所愛,則必有成功。}愛克厥威,允罔功。\footnote{以愛勝威,無以濟眾,信無功。}其爾眾士,懋戒哉!”\footnote{言當勉以用命,戒以闢戮。懋音茂。闢音避。}


{\noindent\zhuan\zihao{6}\fzbyks 傳“將行”至“子弟”。正義曰:“將”之為行,常訓也。天欲加罪,王者順天之罰,則王誅也。“奉王命行王誅,謂殺淫湎之身”,羲和之罪,不及其嗣,故知殺其身,立其賢子弟。\CJKunderwave{楚語}云,重黎之後,世掌天地四時之官,至於夏商。則此不滅其族,故傳言此也。 \par}

{\noindent\zhuan\zihao{6}\fzbyks 傳“山脊”至“害玉”。正義曰:\CJKunderwave{釋山}云:“山脊,岡。”孫炎曰:“長山之脊也。”以崐山出玉,言火逸害玉,喻誅惡害善也。 \par}

{\noindent\zhuan\zihao{6}\fzbyks 傳“逸過”至“於火”。正義曰:“逸”即佚也,佚是淫縱之名,故為過也。“天王之吏”,言位貴而威高,乘貴勢而逞毒心,或睚眥而害良善,故為“過惡之德,其傷害天下,甚於火之害玉”。猛火為烈甚矣,又復烈之於火,言其害之深也。 \par}

{\noindent\zhuan\zihao{6}\fzbyks 傳“殲滅”至“無治”。正義曰:“殲,盡也”,\CJKunderwave{釋詁}文。舍人曰:“殲,眾之盡也。”眾皆死盡為殲也。“渠,大”、“魁,帥”,無正訓。以上“殲厥渠魁”謂滅其元首,故以“渠”為大,“魁”為帥。史傳因此謂賊之首領為“渠帥”,本原出於此。 \par}

{\noindent\shu\zihao{5}\fzkt “今予”至“戒哉”。正義曰:“羲和所犯如上,故今我用汝所有之眾,奉王命,行天罰。汝等眾士,當同心盡力於王室,庶幾輔我敬承天子之命,使我伐必克之。”又恐兵威所及,濫殺無辜,故假喻以戒之:“火炎崐山之岡,玉石俱被焚燒。天王之吏為過惡之德,則酷烈甚於猛火。宜誅惡存善,不得濫殺。滅其為惡大帥,罪止羲和之身,其被迫脅而從距王師者,皆無治責其罪。久染汙穢之俗,本無噁心,皆與惟德更新,一無所問。”又言將軍之法,必有殺戮。“嗚呼!”重其事,故嘆而言之。“將軍威嚴能勝其愛心,有罪者雖愛必誅,信有成功。若愛心勝其威嚴,親愛者有罪不殺,信無功矣”。言我雖愛汝,有罪必殺。其汝眾士宜勉力以戒慎哉!勿違我命以取殺也。 \par}

自\CJKunderline{契}至於\CJKunderline{成湯},八遷。\footnote{十四世,凡八徙國都。契,息列反,殷之始祖。八遷之書,史唯見四。}湯始居亳,從先王居,\footnote{契父帝嚳都亳,湯自商丘遷焉,故曰“從先王居”。亳,旁各反,徐扶各反。嚳,苦毒反。}作\CJKunderwave{帝告}、\CJKunderwave{釐沃}。\footnote{告來居,治沃土,二篇皆亡。告,工毒反。釐,力之反。沃,徐烏酷反。此五亡篇,舊解是\CJKunderwave{夏書},馬、鄭之徒以為\CJKunderwave{商書},兩義並通。}


{\noindent\zhuan\zihao{6}\fzbyks 傳“十四”至“國都”。正義曰:\CJKunderwave{周語}曰:“玄王勤商,十四世而興。”“玄王”謂契也,勤殖功業十四世,至湯而興,為天子也。\CJKunderwave{殷本紀}云,契生昭明。“昭明卒,子相土立。相土卒,子昌若立。昌若卒,子曹圉立。曹圉卒,子冥立。冥卒,子振立。振卒,子微立。微卒,子報丁立。報丁卒,子報乙立。報乙卒,子報丙立。報丙卒,子主壬立。主壬卒,子主癸立。主癸卒,子天乙立。天乙是為\CJKunderline{成湯}”是也。契至\CJKunderline{成湯}十四世,凡八遷國都者,\CJKunderwave{商頌}雲“帝立子生商”,是契居商也。\CJKunderwave{世本}云:“昭明居砥石。”\CJKunderwave{左傳}稱相土居商丘,及今湯居亳,事見經傳者有此四遷,其餘四遷未詳聞也。\CJKunderline{鄭玄}云:“契本封商,國在太華之陽。”皇甫謐云:“今上洛商是也。”襄九年\CJKunderwave{左傳}云:“\CJKunderline{陶唐氏}之火正閼伯居商丘,相土因之。”杜預云:“今梁國睢陽宋都是也。”其“砥石”先儒無言,不知所在。自契至湯,諸侯之國而得數遷都者,蓋以時王命之使遷。至湯乃以商為天下號,則都雖數遷,商名不改。今湯遷亳,乃作此篇,若是諸侯遷都,則不得史錄其事以為\CJKunderwave{商書}之首。文在“湯徵諸侯”、“\CJKunderline{伊尹}去亳”之上,是湯將欲為王時事。史以商有天下,乃追錄初興,並\CJKunderwave{湯徵}與\CJKunderwave{汝鳩}、\CJKunderwave{汝方},皆是伐桀前事,後追錄之也。 \par}

{\noindent\zhuan\zihao{6}\fzbyks 傳“契父”至“王居”。正義曰:“先王”,天子也。自契已下,皆是諸侯,且文稱契至湯,今雲“從先王居”者,必從契之先世天子所居也。\CJKunderwave{世本}、\CJKunderwave{本紀}皆雲契是帝嚳子,知先王是契父帝嚳。帝嚳本居亳,今湯往從之。嚳實帝也,言“先王”者,對文論優劣,則有皇與帝及王之別,散文則雖皇與帝皆得言王也。故\CJKunderwave{禮運}云:“昔者先王未有宮室。”乃謂上皇為王,是其類也。孔言“湯自商丘遷焉”,以相土之居商丘,其文見於\CJKunderwave{左傳},因之言自商丘徙耳。此言不必然也,何則?相土,契之孫也,自契至湯凡八遷,若相土至湯都遂不改,豈契至相土三世而七遷也?相土至湯必更遷都,但不知湯從何地而遷亳耳,必不從商丘遷也。\CJKunderline{鄭玄}云:“亳,今河南偃師縣有湯亭。”\CJKunderwave{漢書音義}臣瓚者云:“湯居亳,今濟陰亳縣是也,今亳有湯冢。己氏有\CJKunderline{伊尹}冢。”杜預云:“梁國蒙縣北有亳城,城中有\CJKunderline{成湯}冢,其西又有\CJKunderline{伊尹}冢。”皇甫謐云:“\CJKunderwave{孟子}稱湯居亳,與葛為鄰,\CJKunderline{葛伯}不祀,湯使亳眾為之耕。葛即今梁國寧陵之葛鄉也。若湯居偃師,去寧陵八百餘里,豈當使民為之耕乎?亳今梁國谷熟縣是也。”諸說不同,未知孰是。 \par}

{\noindent\zhuan\zihao{6}\fzbyks 傳“告來”至“皆亡”。正義曰:經文既亡,其義難明,孔以意言耳。所言“帝告”,不知告誰,序言“從先王居”,或當告帝嚳也。 \par}

{\noindent\shu\zihao{5}\fzkt “自契”至“釐沃”。正義曰:自此已下皆\CJKunderwave{商書}也。序本別卷,與經不連,孔以經序宜相附近,引之各冠其篇首。此篇經亡序存,文無所託,不可以無經之序為卷之首,本書在此,故附此卷之末。契是商之始祖,故遠本之。自契至於\CJKunderline{成湯},凡八遷都。至湯始往居亳,從其先王帝嚳舊居。當時湯有言告,史序其事,作\CJKunderwave{帝告}、\CJKunderwave{釐沃}二篇。 \par}

湯徵諸侯,\footnote{為夏方伯,得專征伐。}\CJKunderline{葛伯}不祀,湯始徵之,\footnote{葛,國。伯,爵也。廢其土地山川及宗廟神祗,皆不祀,湯始伐之。伐始於葛。祗,巨支反。}

{\noindent\shu\zihao{5}\fzkt 傳“葛國”至“於葛”。正義曰:序言“湯徵諸侯”,知其人是葛國之君,伯爵。直雲“不祀”,文無指斥。\CJKunderwave{王制}云:“山川神祗有不舉者為不敬,不敬者君削以地。宗廟有不順者為不孝,不孝者君黜以爵。”是言不祀必廢其土地山川之神祗及宗廟,皆不祀,故湯始徵之。湯伐諸侯,伐始於葛,\CJKunderwave{仲虺之誥}雲“初征自葛”是也。\CJKunderwave{孟子}云:“湯居亳,與葛為鄰。\CJKunderline{葛伯}不祀,湯使人問之曰:‘何為不祀?’曰:‘無以供犧牲也。’湯使遺之牛羊。\CJKunderline{葛伯}食之,又不祀。湯又使人問之曰:‘何為不祀?’曰:‘無以供粢盛也。’湯使亳往為之耕,老弱饋食。\CJKunderline{葛伯}率其人,要其酒食黍稻者,劫而奪之,不授者殺之。有童子以黍肉餉,殺而奪之。\CJKunderwave{書}曰:‘\CJKunderline{葛伯}仇餉。’此之謂也。”是說伐始於葛之事也。 \par}

作\CJKunderwave{湯徵}。\footnote{述始徵之義也,亡。}\CJKunderline{伊尹}去亳適夏,\footnote{\CJKunderline{伊尹},字剩{\hanaa 䂍},湯進於桀。}

{\noindent\zhuan\zihao{6}\fzbyks 傳“\CJKunderline{伊尹}”至“於桀”。正義曰:伊,氏;尹,字,故云“字氏”,倒文以曉人也。\CJKunderline{伊尹}不得叛湯,知湯貢之於桀。必貢之者,湯欲以誠輔桀,冀其用賢以治;不可匡輔,乃始伐之,此時未有伐桀之意,故貢\CJKunderline{伊尹}使輔之。\CJKunderwave{孫武兵書·反間篇}曰:“商之興也,\CJKunderline{伊尹}在夏。周之興也,呂牙在殷。”言使之為反間也,與此說殊。 \par}

既醜有夏,復歸於亳。\footnote{醜惡其政。不能用賢,故退還。復,扶又反。}入自北門,乃遇\CJKunderline{汝鳩}、\CJKunderline{汝方},\footnote{鳩、方,二人湯之賢臣。不期而會曰遇。}作\CJKunderwave{汝鳩}、\CJKunderwave{汝方}。\footnote{言所以醜夏而還之意,二篇皆亡。}

{\noindent\zhuan\zihao{6}\fzbyks 傳“鳩方”至“曰遇”。正義曰:\CJKunderline{伊尹}與之言,知是賢臣也。“不期而會曰遇”,隱八年\CJKunderwave{穀梁傳}文也。 \par}

%%% Local Variables:
%%% mode: latex
%%% TeX-engine: xetex
%%% TeX-master: "../Main"
%%% End:
