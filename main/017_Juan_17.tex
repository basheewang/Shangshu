%% Time-stamp: <Chen Wang: 2024-04-02 11:42:41>
%% -*- coding: utf-8 -*-

% {\noindent \zhu \zihao{5} \fzbyks } -> 注 (△ ○)
% {\noindent \shu \zihao{5} \fzkt } -> 疏

\chapter{卷十七}


\section{蔡仲之命第十九【偽】}


\CJKunderline{蔡叔}既沒,\footnote{以罪放而卒。}王命蔡仲,踐諸侯位,\footnote{\CJKunderline{成王}也。父卒命子,罪不相及。}作\CJKunderwave{蔡仲之命}。\footnote{冊書命之。}

蔡仲之命\footnote{蔡,國名。仲,字。因以名篇。}


{\noindent\zhuan\zihao{6}\fzbyks 傳“\CJKunderline{成王}”至“相及”。正義曰:編書以世先後為次,此篇在\CJKunderline{成王}書內,知“王命蔡仲”是\CJKunderline{成王}命之也。\CJKunderline{蔡叔}之沒,不知何年,其命蔡仲,未必初卒即命,以其繼父命子,故系之\CJKunderline{蔡叔}之後也。\CJKunderline{蔡叔}有罪而命蔡仲者,“父卒命子,罪不相及”也。昭二十年\CJKunderwave{左傳}曰:“父子兄弟,罪不相及。”其言“罪不相及”,謂蔡仲不坐父爾。若父有大罪,罪當絕滅,正可別封他國,不得仍取蔡名,以\CJKunderline{蔡叔}為始祖也。\CJKunderline{蔡叔}身尚不死,明其罪輕。不立\CJKunderline{管叔}之後者,蓋罪重無子,或有而不賢故也。 \par}

{\noindent\shu\zihao{5}\fzkt “\CJKunderline{蔡叔}”至“之命”。正義曰:\CJKunderline{蔡叔}與\CJKunderline{管叔}流言於國,謗毀\CJKunderline{周公},\CJKunderline{周公}囚之郭鄰,至死不赦。\CJKunderline{蔡叔}既沒,\CJKunderline{成王}命\CJKunderline{蔡叔}之子蔡仲踐諸侯之位,封為國君,以策書命之。史敘其事,故作\CJKunderwave{蔡仲之命}。 \par}

惟\CJKunderline{周公}位冢宰,正百工,\footnote{百官總己以聽冢宰,謂\CJKunderline{武王}崩時。}群叔流言,乃致闢\CJKunderline{管叔}於商;囚\CJKunderline{蔡叔}於郭鄰,以車七乘;\footnote{致法謂誅殺。囚謂制其出入。郭鄰,中國之外地名。從車七乘,言少。管、蔡,國名。○闢,婢亦反,徐扶亦反。乘,繩證反。從,才用反。}降霍叔於庶人,三年不齒。\footnote{罪輕,故退為眾人,三年之後乃齒錄,封為霍侯,子孫為晉所滅。}蔡仲克庸祗德,\CJKunderline{周公}以為卿士。\footnote{蔡仲能用敬德,稱其賢也。明王之法,誅父用子,言至公。\CJKunderline{周公},圻內諸侯,二卿治事。○圻,巨依反,下同。}叔卒,乃命諸王邦之蔡。\footnote{叔之所封,圻內之蔡。仲之所封,淮汝之間。圻內之蔡名已滅,故取其名以名新國,欲其戒之。}


{\noindent\zhuan\zihao{6}\fzbyks 傳“致法”至“國名”。正義曰:\CJKunderwave{周禮}有掌囚之官,鄭云,囚,拘也,主拘繫當刑殺者。拘繫之是為制其出入,不得輒行。“郭鄰,中國之外地名”,蓋相傳為然,不知在何方。\CJKunderwave{舜典}雲“流宥五刑”,謂流之遠地,任其自生,此則徙之郭鄰,而又囚之。\CJKunderwave{管蔡世家}雲“封叔鮮于管,封叔度於蔡”,是管、蔡為國名。杜預云:“管在滎陽京縣東北。” \par}

{\noindent\zhuan\zihao{6}\fzbyks 傳“罪輕”至“所滅”。正義曰:言“群叔流言”,則霍叔亦流言也。而知其罪輕者,以其不死不遷,有降黜而已,明其罪輕也。霍叔不監殷民,\CJKunderline{周公}惟伐管蔡,不言伐霍叔,於時霍叔蓋在京邑,聞管蔡之語,流傳其言,謂其實然,不與朝廷同心,故退之。\CJKunderwave{世家}雲“\CJKunderline{武王}已克商平天下,封功臣昆弟,封叔處於霍”,則\CJKunderline{武王}已封之矣。後黜為庶人,奪其爵祿,三年之後乃更爵祿,蓋復其舊封,封為霍侯。\CJKunderwave{春秋}閔元年晉侯滅霍,既子孫得為國君,為晉所滅,知三年之後復得封也。\CJKunderwave{世家}惟雲封霍,不雲其爵,傳言“霍侯”,或當有所據而知之。 \par}

{\noindent\zhuan\zihao{6}\fzbyks 傳“蔡仲”至“治事”。正義曰:\CJKunderwave{周禮·冢宰}:“以八則治都鄙。”馬融云:“距王城四百里至五百里謂之都鄙。鄙,邊邑也,以封王之子弟在畿內者。”\CJKunderwave{冢宰}又云:“乃施則于都鄙而建其長,立其兩。”馬、鄭皆雲“立卿兩人”,是畿內諸侯立二卿。定四年\CJKunderwave{左傳}說此事雲“\CJKunderline{周公}舉之,以為己卿士”,是為\CJKunderline{周公}圻內之卿士也。\CJKunderwave{世家}雲“\CJKunderline{周公}舉胡以為魯卿士,魯國治。於是\CJKunderline{周公}言於\CJKunderline{成王},復封之於蔡。”案\CJKunderwave{魯世家}云,\CJKunderline{成王}封\CJKunderline{周公}於魯,\CJKunderline{周公}不就封,留佐\CJKunderline{成王}。則\CJKunderline{周公}身不就封,安得使胡為卿士?馬遷說之謬爾。 \par}

{\noindent\zhuan\zihao{6}\fzbyks 傳“叔之”至“戒之”。正義曰:“仲之所封,淮汝之間”,\CJKunderwave{左傳}有文。“叔之所封,圻內之蔡”,其事不知所出也。\CJKunderwave{世家}云:“\CJKunderline{蔡叔}居上蔡。”宋仲子云:“胡徙居新蔡。”杜預云:“\CJKunderline{武王}封叔度於汝南上蔡,至平侯徙新蔡,昭侯徙居九江下蔡。”檢其地,上蔡、新蔡皆屬汝南郡,去京師太遠,叔若封於上蔡,不得在圻內也。孔言叔封圻內,或當有以知之。但圻內蔡地,不知所在爾。 \par}

{\noindent\shu\zihao{5}\fzkt “惟周”至“之蔡”。正義曰:惟\CJKunderline{周公}於\CJKunderline{武王}崩後,其位為冢宰之卿,正百官之治,攝王政,治天下。於時管、蔡、霍等群叔流言於國,謗毀\CJKunderline{周公}。\CJKunderline{周公}乃以王命致法,殺\CJKunderline{管叔}於商,就殷都殺之。囚\CJKunderline{蔡叔},遷之於郭鄰之地,惟與之從車七乘。降黜霍叔於庶人,若今除名為民,三年之內不得與兄弟年齒相次。\CJKunderline{蔡叔}之子蔡仲能用敬德,\CJKunderline{周公}為畿內諸侯,得立二卿,以蔡仲為己之卿士。\CJKunderline{周公}善其為人,及\CJKunderline{蔡叔}既卒,乃將蔡仲命之於王,國之於蔡為諸侯也。 \par}

王若曰:“小子胡,\footnote{言小子,明當受教訓。胡,仲名。順其事而告之。}惟爾率德改行,克慎厥猷,\footnote{言汝循祖之德,改父之行,能慎其道。嘆其賢。}肆予命爾侯於東土。往即乃封,敬哉!\footnote{以汝率德改行之故,故我命汝為諸侯於東土。往就汝所封之國,當修已以敬哉!}爾尚蓋前人之愆,惟忠惟孝,\footnote{汝當庶幾修德,尚蓋前人之過。子能蓋父,所以為惟忠惟孝。}爾乃邁跡自身,克勤無怠,以垂憲乃後。\footnote{汝乃行善跡用汝身,使可蹤跡而法循之,能勤無懈怠,以垂法子孫,世世稱頌,乃當我意。}率乃祖\CJKunderline{文王}之彝訓,無若爾考之違王命。\footnote{言當循文武之常教,以父違命為世戒。}皇天無親,惟德是輔。民心無常,惟惠之懷。\footnote{天之於人,無有親疏,惟有德者則輔佑之。民之於上,無有常主,惟愛己者則歸之。}為善不同,同歸於治。為惡不同,同歸於亂。\footnote{言人為善為惡,各有百端,未必正同。而治亂所歸不殊,宜慎其微。○治,直吏反。}爾其戒哉!慎厥初,惟厥終,終以不困。不惟厥終,終以困窮。\footnote{汝其戒治亂之機哉!作事云為,必慎其初,念其終,則終用不困窮。}懋乃攸績,睦乃四鄰,以蕃王室,以和兄弟。\footnote{勉汝所立之功,親汝四鄰之國,以蕃屏王室,以和協同姓之邦,諸侯之道。○懋音茂。蕃,方元反,注同。}康濟小民,率自中,無作聰明亂舊章。\footnote{汝為政,當安小民之居,成小民之業,循用大中之道,無敢為小聰明,作異辯,以變亂舊典文章。}詳乃視聽,罔以側言改厥度,則予一人汝嘉。”\footnote{詳審汝視聽,非禮義勿視聽,無以邪巧之言易其常度,必斷之以義,則我一人善汝矣。○度如字,注同。斷,丁亂反。}王曰:“嗚呼!小子胡,汝往哉!無荒棄朕命。”\footnote{嘆而敕之,欲其念戒:“小子胡,汝往之國哉!無廢棄我命。”欲其終身奉行,後世遵則。}


{\noindent\zhuan\zihao{6}\fzbyks 傳“汝當”至“惟孝”。正義曰:忠施於君,孝施於父,子能蓋父,惟得為孝,而亦得為忠者,父以不忠獲罪,若能改父之行,蓋父之愆,是為忠臣也。 \par}

{\noindent\shu\zihao{5}\fzkt “侯於東土”。正義曰:此使之為諸侯於東土爾,不知何爵也。\CJKunderwave{世家}云:“蔡仲卒,子蔡伯荒立。卒,子宮侯立。”自此已下遂皆稱侯,則蔡仲初封即為侯也。“蔡伯荒”者,自稱其字,“伯”非爵也。 \par}

\CJKunderline{成王}東伐淮夷,遂踐奄,\footnote{\CJKunderline{成王}即政,淮夷奄國又叛,王親征之,遂滅奄而徙之,以其數反覆。○踐,似淺反,馬同,\CJKunderwave{大傳}云:“藉也。”數,色角反。覆,芳服反。}作\CJKunderwave{\CJKunderline{成王}政}。\footnote{為平淮夷徙奄之政令。亡。○政如字,馬本作徵,雲正。}


{\noindent\zhuan\zihao{6}\fzbyks 傳“\CJKunderline{成王}”至“反覆”。正義曰:\CJKunderwave{洛誥}之篇言\CJKunderline{周公}歸政\CJKunderline{成王},\CJKunderwave{多士}已下皆是\CJKunderline{成王}即政初事。編篇以先後為次,此篇在\CJKunderline{成王}書內,知是“\CJKunderline{成王}即政”,淮夷奄國又叛,王親征之”。又案\CJKunderwave{洛誥}\CJKunderline{成王}即政,始封\CJKunderline{伯禽}。\CJKunderline{伯禽}既為魯侯,乃居曲阜。\CJKunderwave{費誓}稱“魯侯\CJKunderline{伯禽}宅曲阜”,“淮夷、徐戎並興”,魯侯徵之,作\CJKunderwave{費誓}。彼言淮夷並興,即此“伐淮夷”。王伐淮夷,魯伐徐戎,是同時伐,明是\CJKunderline{成王}即政之年復重叛也。\CJKunderline{鄭玄}謂此伐淮夷與踐奄是攝政三年伐管蔡時事,其編篇於此,即雲未聞。\CJKunderwave{費誓}之篇言淮夷之叛,則是重叛明矣。\CJKunderwave{多方}之篇責殷臣云:“我惟時其戰要囚之,至於再,至於三。”若\CJKunderline{武王}伐紂之後,惟攝政三年之一叛,正可至於再爾,安得至於三乎?故知是\CJKunderline{成王}即政又叛也。\CJKunderline{鄭玄}讀“踐”為翦,翦滅也。孔不破字,蓋以踐其國即是踐滅之事,故孔以“踐”為滅也。下篇序云:“\CJKunderline{成王}既踐奄,將遷其君”,是滅其奄而徙之,以其數反覆故也。 \par}

{\noindent\shu\zihao{5}\fzkt “\CJKunderline{成王}東”至“王政”。正義曰:\CJKunderline{周公}攝政之初,奄與淮夷從管蔡作亂,\CJKunderline{周公}徵而定之。\CJKunderline{成王}即政之初,淮夷與奄又叛,\CJKunderline{成王}親往徵之。\CJKunderline{成王}東伐淮夷,遂踐滅奄國。以其數叛,徙奄民。作誥命之辭,言平淮夷徙奄之政令。史敘其事,作\CJKunderwave{\CJKunderline{成王}政}之篇。“成”訓平也,言平此叛逆之民,以為王者政令,故以“\CJKunderline{成王}政為”篇名。 \par}

\CJKunderline{成王}既踐奄,將遷其君於蒲姑,\footnote{已滅奄,而徙其君及人臣之惡者於蒲姑。蒲姑,齊地,近中國,教化之。○蒲如字,徐又扶各反,馬本作薄。近,附近之近。}\CJKunderline{周公}告\CJKunderline{召公},作\CJKunderwave{將蒲姑}。\footnote{言將徙奄新立之君於蒲姑,告\CJKunderline{召公}使此冊書告令之。亡。}


{\noindent\zhuan\zihao{6}\fzbyks 傳“已滅”至“化之”。正義曰:昭二十年\CJKunderwave{左傳}晏子云,古人居此地者,有蒲姑氏。杜預云:“樂安博昌縣北有蒲姑城。”是蒲姑為齊地也。\CJKunderline{周公}遷殷頑民於成周,近京師,教化之,知今遷奄君臣於蒲姑,為“近中國,教化之”。必如此言,則奄去中國遠於蒲姑。杜預云:“奄闕,不知所在。”鄭雲“奄蓋在淮夷之地”,亦未能詳。\CJKunderline{成王}先伐淮夷,遂滅奄,奄似遠於淮夷也。 \par}

{\noindent\zhuan\zihao{6}\fzbyks 傳“言將”至“之亡”。正義曰:\CJKunderwave{禮}天子不滅國,諸侯有罪,則殺其君而擇立次賢者,故知所徙者言“將徙奄新立之君於蒲姑”也。上言\CJKunderline{周公}告\CJKunderline{召公},其篇既亡,不知告以何事。孔以意卜之“告\CJKunderline{召公}使為此策書告令之”,不能知其必然否也。 \par}

{\noindent\shu\zihao{5}\fzkt “\CJKunderline{成王}既”至“作蒲姑”。正義曰:\CJKunderline{成王}既踐滅奄國,將遷其君於蒲姑之地,\CJKunderline{周公}告\CJKunderline{召公},使作冊書,言將遷奄君於蒲姑之地。史敘其事,作\CJKunderwave{將蒲姑}之篇。 \par}

\section{多方第二十}


\CJKunderline{成王}歸自奄,\footnote{伐奄歸。}在宗周,誥庶邦,\footnote{誥以禍福。}作\CJKunderwave{多方}。

多方\footnote{眾方天下諸侯。}

惟五月丁亥,王來自奄,至於宗周。\footnote{\CJKunderline{周公}歸政之明年,淮夷奄又叛。魯徵淮夷,作\CJKunderwave{費誓}。王親征奄,滅其國,五月還至鎬京。○費音秘。}

{\noindent\zhuan\zihao{6}\fzbyks 傳“眾方天下諸侯”。正義曰:自\CJKunderline{武王}伐紂,及\CJKunderline{成王}即政,新封建者甚少。天下諸侯多是殷之舊國,其心未服周家,由是奄君重叛。今因滅奄新歸,故告天下諸侯以興亡之戒,欲令其無二心也。語雖普告天下,意在殷之舊國。篇末亦告殷之多士,獨言“諸侯”者,舉其尊者,以其篇王告殷之諸侯故也。 \par}

{\noindent\zhuan\zihao{6}\fzbyks 傳“\CJKunderline{周公}”至“鎬京”。正義曰:以\CJKunderwave{洛誥}言歸政之事,\CJKunderwave{多士}之篇次之,\CJKunderwave{多士}是歸政明年之事,故知此篇亦歸政明年之事。事猶不明,故取\CJKunderwave{費誓}為證。以\CJKunderwave{成以政}之序言“\CJKunderline{成王}東伐淮夷”,\CJKunderwave{費誓}之篇言“淮夷、徐戎並興”,俱言“淮夷”,明是一事,故言“魯徵淮夷,作\CJKunderwave{費誓},王親征奄,滅其國”,以明二者為一時之事也。上序言“\CJKunderline{成王}伐淮夷”,而此傳言“魯徵淮夷”者,當時淮夷徐戎並起為亂,魯與二國相近,發意欲並徵二國,故以二國誓眾,但\CJKunderline{成王}恐魯不能獨平二國,故復親往徵之,所以\CJKunderwave{\CJKunderline{成王}政}之序與\CJKunderwave{費誓}之經並言“淮夷”,為此故也。傳言“五月還至鎬京”,明此“宗周”即鎬京也。\CJKunderwave{禮記·祭統}衛孔悝之鼎銘雲“即宮於宗周”,彼“宗周”謂洛邑也。是洛邑亦名宗周,知此是鎬京者,\CJKunderline{成王}以\CJKunderline{周公}歸政之時,暫至洛邑,還歸處西都,鎬京是王常居,知“至於宗周”,至鎬京也。且此與\CJKunderwave{周官}同時事也,\CJKunderwave{周官}序雲“還歸在豐”,經雲“歸於宗周”,豐、鎬相近,即此“宗周”是鎬京也。 \par}

{\noindent\shu\zihao{5}\fzkt “\CJKunderline{成王}”至“多方”。正義曰:\CJKunderline{成王}歸自伐奄,在於宗周鎬京,諸侯以王徵還,皆來朝集,\CJKunderline{周公}稱王命,以禍福咸告天下諸侯國。史敘其事,作\CJKunderwave{多方}。 \par}

\CJKunderline{周公}曰:“王若曰,猷告爾四國多方。\footnote{\CJKunderline{周公}以王命順大道,告四方。稱\CJKunderline{周公},以別王自告。○別,彼列反。}惟爾殷侯尹民,我惟大降爾命,爾罔不知。\footnote{殷之諸侯王民者,我大降汝命,謂誅紂也。言天下無不知紂暴虐以取亡。}


{\noindent\zhuan\zihao{6}\fzbyks 傳“\CJKunderline{周公}”至“自告”。正義曰:\CJKunderline{成王}新始即政,\CJKunderline{周公}留而輔之。\CJKunderline{周公}以王命告令諸侯,所告實非王言,故加“\CJKunderline{周公}曰”於“王若曰”之上,以明\CJKunderline{周公}宣\CJKunderline{成王}之意也。“猷”,道也,\CJKunderline{周公}以王命順大道告四方也。既言“四國”,又言“多方”,見四方國多也。不直言“王曰”,稱“\CJKunderline{周公}”,以別王自告也。王肅云:“\CJKunderline{周公}攝政,稱\CJKunderline{成王}命以告。及還政,稱‘王曰’嫌自\CJKunderline{成王}辭,故加‘\CJKunderline{周公}’以明之。”然\CJKunderwave{多士}之篇“王若曰”之上不加“\CJKunderline{周公}曰”者,以彼上句雲“\CJKunderline{周公}初於新邑洛,用告”,知是\CJKunderline{周公}故也。 \par}

{\noindent\zhuan\zihao{6}\fzbyks 傳“殷之”至“取亡”。正義曰:諸侯為民之主,民所取正,故謂之“正民”。民以君為命,死生在君,天下之命,在於一人紂,言我大黜下汝之民命,正謂\CJKunderline{武王}誅紂也。言天下無不知紂以暴虐取亡,欲使思念之,令其心棄殷而慕周也。 \par}

{\noindent\shu\zihao{5}\fzkt “\CJKunderline{周公}”至“不知”。正義曰:\CJKunderline{周公}以\CJKunderline{成王}之意告眾方之諸侯曰:“我王順大道以告汝四方之國多方諸侯,惟爾殷之諸侯正民者,我\CJKunderline{武王}大下汝天下民命,誅殺虐紂。汝諸侯天下之民,無有不知紂以暴虐取亡。”欲令其思念之。 \par}

洪惟圖天之命,弗永寅念於祀,惟帝降格於夏。\footnote{大惟為王謀天之命,不長敬念於祭祀。謂\CJKunderline{夏桀}。惟天下至戒於夏以譴告之。謂災異。○譴,棄淺反。}有夏誕厥逸,不肯戚言於民,\footnote{有\CJKunderline{夏桀}不畏天戒而大其逸豫,不肯憂言於民。無憂民之言。}乃大淫昏,不克終日勸於帝之迪,\footnote{言桀乃大為過昏之行,不能終日勸於天之道。○迪,徒歷反,馬本作攸,云:“所也。”行,下孟反。}乃爾攸聞。\footnote{言桀之惡乃汝所聞。}


{\noindent\zhuan\zihao{6}\fzbyks 傳“大惟”至“災異”。正義曰:上天之命,去惡與善,凡為民主,皆當謀之。恐天捨己而去,常須敬念祭祀。天所譴告,謂下災異。天不言,故下災異以譴告,責人主,冀自修政也。 \par}

{\noindent\shu\zihao{5}\fzkt “洪惟”至“攸聞”。正義曰:以諸侯心未服周,故舉夏殷為戒。此章皆說桀亡湯興之事,言\CJKunderline{夏桀}大惟居天子之位,謀上天之命,而不能長敬念於祭祀,惟天下至戒於\CJKunderline{夏桀}。謂下災異譴告之,冀其見災而懼,改修政德。而有\CJKunderline{夏桀}不畏天命,乃大其逸豫,不肯憂言於民,惟乃自樂其身,無憂民之言。\CJKunderline{夏桀}乃復大為淫昏之行,不能終竟一日勉於天之道。言不能一日行天道也。桀之此惡,乃是汝之所聞。言不虛也。 \par}

厥圖帝之命,不克開於民之麗。\footnote{桀其謀天之命,不能開於民所施政教。麗,施也。言昏昧。○麗,力馳反。}乃大降罰,崇亂有夏,因甲於內亂。\footnote{桀乃大下罰於民,重亂有夏。言殘虐。外不憂民,內不勤德,因甲於二亂之內。言昏甚。○重,直用反,又直龍反。}不克靈承於旅,罔丕惟進之恭,洪舒於民。\footnote{言桀不能善奉於人眾,無大惟進恭德,而大舒惰於治民。}亦惟有夏之民叨懫,日欽劓割夏邑。\footnote{桀洪舒於民,故亦惟有夏之民貪叨忿懫而逆命,於是桀民尊敬其能劓割夏邑者。謂殘賊臣。○懫,敕二反。劓,魚器反。}


{\noindent\zhuan\zihao{6}\fzbyks 傳“桀乃”至“昏甚”。正義曰:\CJKunderwave{釋詁}云:“崇,重也。”桀既為惡政,無以悛改,乃復大下罪罰於民,重亂有夏之國。言其殘虐大也。“夾”聲近“甲”,古人“甲”與“夾”通用。夾於二事之內,而為亂行,故傳以二事充之。外不憂民,內不勤德,桀身夾於二亂之內,言其昏暗甚也。鄭、王皆以“甲”為狎,王云:“狎習災異於內外為禍亂。”鄭云:“習為鳥獸之行於內為淫亂。”與孔異也。 \par}

{\noindent\zhuan\zihao{6}\fzbyks 傳“言桀”至“治民”。正義曰:民當奉王,而責桀不能善奉於民眾者,君之奉民,謂設美政於民也。以善奉民,當敬以循之,不敢懈惰。桀乃無大惟進於恭德,而大舒緩懈惰於治民,令民益困,而政益亂也。 \par}

{\noindent\zhuan\zihao{6}\fzbyks 傳“桀洪”至“賊臣”。正義曰:\CJKunderwave{禮記}云:“言悖而出,亦悖而入。”桀既不憂於民,故民亦違逆桀命,為貪饕忿懫之行。文十八年\CJKunderwave{左傳}云:“縉雲氏有不才子,貪於飲食,冒於貨賄,天下之民謂之饕餮。”說者皆言貪財為饕,貪食為餮。“饕”即“叨”也,叨餮謂貪財貪食也。“忿懫”言忿怒違理也。民既如此,桀無如之何,惟日日尊敬其能劓割夏邑者,謂性能殘賊者,任用之。 \par}

{\noindent\shu\zihao{5}\fzkt “厥圖”至“夏邑”。正義曰:又言桀惡。桀其謀天之命,不能開發於民之所施政教。正謂不能開發善政,以施於民。桀乃大下罪罰於民,重亂有夏之國。外不憂民,內不勤德,因復甲於二者之內,為亂之行。桀不能以善道奉承於眾民,無大惟進之恭德,而大舒惰於民。言桀不能進行恭德,而舒惰於治民。桀既舒惰於民,故亦惟有夏之民貪饕忿懫而違逆桀命,於是桀日日尊敬殘賊之臣能劓割夏邑者,任用之,使威服下民也。 \par}

天惟時求民主,乃大降顯休命於\CJKunderline{成湯},\footnote{天惟是桀惡,故更求民主以代之,大下明美之命於\CJKunderline{成湯},使王天下。}刑殄有夏,惟天不畀純。\footnote{命湯刑絕有夏,惟天不與桀,亦已大。○畀,必二反。}乃惟以爾多方之義民,不克永於多享。\footnote{天所以不與桀,以其乃惟用汝多方之義民為臣,而不能長久多享國故。}惟夏之恭多士,大不克明保享於民,\footnote{惟桀之所謂恭人眾士,大不能明安享於民。言亂主所任,任同己者。}乃胥惟虐於民,至於百為,大不克開。\footnote{桀之眾士,乃相與惟暴虐於民,至於百端所為。言虐非一。大不能開民以善。言與桀合志。}


{\noindent\zhuan\zihao{6}\fzbyks 傳“惟桀”至“己者”。正義曰:惟桀之所謂恭人眾士,實非恭人。亂主所好,好用同己者,以其同己,謂之為恭人,實非善人,故不能明享於民。杜預訓“享”為受,受國者謂受而有之。此言不能安享於民,謂不能安存享受於民眾也。 \par}

{\noindent\shu\zihao{5}\fzkt “天惟”至“克開”。正義曰:天惟桀惡之故,更求民主以代。天乃大下明美之命於\CJKunderline{成湯},使之代桀王天下。乃命湯施刑罰絕有夏,惟天不與\CJKunderline{夏桀},亦已大矣。天所不與之者,乃惟此桀用汝多方之義民為臣,而不能長久於多享國故也。義民實賢人也,\CJKunderline{夏桀}不用。惟\CJKunderline{夏桀}之所謂恭人眾士者,大不能用明道安存享於眾民,乃相與惟行暴虐於民,至於百端所為。言虐無所不作。大不能開民以善,其臣與桀同惡,夏家所以滅亡也。 \par}

乃惟\CJKunderline{成湯},克以爾多方,簡代夏作民主。\footnote{乃惟\CJKunderline{成湯},能用汝眾方之賢,大代夏政,為天下民主。}慎厥麗乃勸,厥民刑用勸。\footnote{湯慎其施政於民,民乃勸善。其人雖刑,亦用勸善。言政刑清。}以至於\CJKunderline{帝乙},罔不明德慎罰,亦克用勸。\footnote{言自湯至於\CJKunderline{帝乙},皆能成其王道,長慎輔相,無不明有德,慎去刑罰,亦能用勸善。○相,息亮反。去,羌呂反。}要囚,殄戮多罪,亦克用勸。開釋無辜,亦克用勸。\footnote{\CJKunderline{帝乙}已上,要察囚情,絕戮眾罪,亦能用勸善。開放無罪之人,必無枉縱,亦能用勸善。○要,一遙反,又一妙反,注同。殄,亭遍反。上,時掌反。}今至於爾闢,弗克以爾多方,享天之命。\footnote{今至於汝君,謂紂,不能用汝眾方,享天之命,故誅滅之。○闢,必亦反。}


{\noindent\zhuan\zihao{6}\fzbyks 傳“乃惟”至“民主”。正義曰:“大代夏”者,言天位之重,湯能代之,謂之“大代夏”也。王肅云:“以大道代夏為民主。” \par}

{\noindent\zhuan\zihao{6}\fzbyks 傳“湯慎”至“刑清”。正義曰:“慎厥麗”者,總謂施政教爾。但下句言“刑用勸”,勸用刑則厥麗之言有賞,賞謂賞用勸也。但所施政教,其事既多,非徒刑賞而已。舉事得中,民皆勸也。政無失,刑無濫,民以是勸善。言政刑清。 \par}

{\noindent\zhuan\zihao{6}\fzbyks 傳“\CJKunderline{帝乙}”至“勸善”。正義曰:將欲斷罪,必受其要辭,察其虛實,故言“要囚”也。“殄戮多罪”,罪者不濫。開釋無罪者,不枉殺人,不縱有罪,亦是政刑清,故能用勸善也。 \par}

{\noindent\shu\zihao{5}\fzkt “乃惟”至“之命”。正義曰:桀殘虐於民,乃惟\CJKunderline{成湯},能用汝眾方之賢人,大代\CJKunderline{夏桀},作天下民主。慎其所施政教於民,民乃勸勉為善。其民雖被刑殺,亦用勸勉為善。非徒湯聖,後世亦賢。自湯至於\CJKunderline{帝乙},皆能成其王道,無不顯用有德,畏慎刑罰,亦能用勸勉為善。要察囚情,絕戮眾罪,亦能用勸勉為善。開放無罪,亦能用勸勉為善。今至於汝君紂,反先王之道,不能用汝多方之民,享有上天之命,由此故被誅滅。汝等宜當知之,不當更令如殷也。 \par}

“嗚呼!王若曰,誥告爾多方,非天庸釋有夏,\footnote{嘆而順其事以告汝眾方,非天用釋棄桀,桀縱惡自棄,故誅放。}非天庸釋有殷,乃惟爾闢,以爾多方,大淫圖天之命,屑有辭。\footnote{非天用棄有殷,乃惟汝君紂,用汝眾方大為過惡者,共謀天之命,惡事盡有辭說,布在天下,故見誅滅也。}

{\noindent\shu\zihao{5}\fzkt “嗚呼”至“有辭”。正義曰:\CJKunderline{周公}先自嘆,而複稱王命云:“王順其事而言曰,以言告人謂之誥,我告汝眾方諸侯,非天用廢有夏,\CJKunderline{夏桀}縱惡自棄也。非天用廢有殷,殷紂縱惡自棄也。”又指說紂惡:“乃惟汝君殷紂,用汝眾方之民大為過惡者,共此惡人,謀天之命。其惡事盡有辭說,布在天下,以此故見誅滅。” \par}

乃惟有夏圖厥政,不集於享,天降時喪,有邦間之。\footnote{更說桀也。言桀謀其政,不成於享,故天下是喪亡以禍之,使天下有國聖人代之。言有國,明皇天無親,佑有德。○間,間廁之間。}

{\noindent\shu\zihao{5}\fzkt “乃惟”至“間之”。正義曰:更說桀亡之由,乃惟有\CJKunderline{夏桀}謀其政,不能成於享國,所謀皆是惡事,故天下是喪亡以禍之,使有國聖人來代之。言皇天無親,惟佑有德,故以聖君代暗主也。湯是夏之諸侯,故云“有國”。 \par}

乃惟爾商后王,逸厥逸,\footnote{后王紂逸豫其過逸。言縱恣無度。}圖厥政,不蠲烝,天惟降時喪。\footnote{紂謀其政,不絜進於善,故天惟下是喪亡。謂誅滅。○蠲,吉玄反,馬云:“明也。”一音圭。烝,絕句,之承反,馬云:“升也。”}惟聖罔唸作狂,惟狂克唸作聖。\footnote{惟聖人無念於善,則為狂人。惟狂人能念於善,則為聖人。言桀紂非實狂愚,以不念善,故滅亡。}天惟五年,須暇之子孫,誕作民主,罔可念聽。\footnote{天以湯故,五年須暇湯之子孫,冀其改悔。而紂大為民主,肆行無道,事無可念,言無可聽。\CJKunderline{武王}服喪三年,還師二年。}


{\noindent\zhuan\zihao{6}\fzbyks 傳“惟聖”至“滅亡”。正義曰:“聖”者上智之名,“狂”者下愚之稱。\CJKunderline{孔子}曰:“惟上智與下愚不移。”是聖必不可為狂,狂必不能為聖,此事決矣。而此言“惟聖人無念於善,則為狂人。惟狂人能念於善,則為聖人”者,方言天須暇於紂,冀其改悔,說有此理爾,不言此事是實也。謂之為聖,寧肯無念於善?已名為狂,豈能念善?中人念與不念,其實少有所移,欲見念善有益,故舉狂聖極善惡者言之。 \par}

{\noindent\zhuan\zihao{6}\fzbyks 傳“天以”至“二年”。正義曰:湯是創業聖王,理當祚胤長遠。計紂未死五年之前,已合喪滅,但紂是湯之子孫,天以湯聖人之故,故五年須待閒暇湯之子孫,冀其改悔,能念善道。而紂大為民主,肆行無道。所為皆惡事,無可念者;言皆惡言,無可聽者;由是天始滅之。五年者,以\CJKunderline{武王}討紂,初立即應伐之,故從\CJKunderline{武王}初立之年,數至伐紂為五年。\CJKunderline{文王}受命九年而崩,其年\CJKunderline{武王}嗣立。服喪三年,未得征伐。十一年服闋,乃觀兵於孟津,十三年方始殺紂。從九年至十三年,是五年也。然服喪三年,還師二年,乃事理宜然,而云以湯故須暇之者,以殷紂惡盈,久合誅滅,逢\CJKunderline{文王}崩,未暇行師,兼之示弱,凡經五載,聖人因言之以為法教爾。其實非天不知紂狂,望其後改悔,亦非曲念湯德,延此歲年也。 \par}

{\noindent\shu\zihao{5}\fzkt “乃惟”至“念聽”。正義曰:更說紂亡之由。乃惟汝商之後王紂,逸豫其過,縱恣無度。紂謀其為政,不能絜進於善,惟行惡事,天惟下是喪亡以禍之。惟聖人無念於善,則為狂人。惟狂人能念於善,則為聖人。紂雖狂愚,冀其念善也。計紂為惡,早應誅滅,天惟以\CJKunderline{成湯}之故,故積五年須待閒暇湯之子孫。誕緩多年,冀其改悔。而紂大為民主,肆行無道,事無可念,言無可聽,由是天始改意,故誅滅之。 \par}

天惟求爾多方,大動以威,開厥顧天。\footnote{天惟求汝眾方之賢者,大動紂以威,開其能顧天可以代者。}惟爾多方,罔堪顧之。惟我周王,靈承於旅。\footnote{惟汝眾方之中,無堪顧天之道者。惟我周王,善奉於眾。言以仁政得人心。}克堪用德,惟典神天。\footnote{言周文武能堪用德,惟可以主神天之祀,任天王。○任音壬。}天惟式教我用休,簡畀殷命,尹爾多方。\footnote{天以我用德之故,惟用教我用美道代殷,天與我殷之王命,以正汝眾方之諸侯。}


{\noindent\zhuan\zihao{6}\fzbyks 傳“天惟”至“代者”。正義曰:“天惟求汝眾方之賢”,言欲選賢以為天子也。“大動紂以威”,謂誅殺紂也。天意復開其能顧天可以代者,欲使代之。“顧”謂回視,有聖德者,天回視之。\CJKunderwave{詩}所謂“乃眷西顧,此惟與宅”,與彼“顧”同,言天顧\CJKunderline{文王}而與之居,即此意也。但謂天顧人,人亦顧天,此雲“開厥顧天”,謂人顧天也。下雲“罔堪顧之”,謂天顧人也。言多方人皆無德,不堪使天顧之。傳以顧事通於彼,故皆以天言之。 \par}

{\noindent\zhuan\zihao{6}\fzbyks 傳“天以”至“諸侯”。正義曰:周以能行美道,乃得天顧,復言天用教我美道者,人之美惡,何事非天?由為美道,為天所顧,以美歸功於天,言教我用美道,故得當天意也。 \par}

{\noindent\shu\zihao{5}\fzkt “天惟”至“多方”。正義曰:天以紂惡之故,將選人代之。惟求賢人於汝眾方,大動紂以威。謂誅去紂也。開其有德能顧天之者,欲以伐紂,惟汝眾方之君,悉皆無德,無堪使天顧之。惟我周王善奉於眾,能以仁政得人心,文武能堪用德,惟可以主神天之祀,任作天子也。天惟以我用德之故,故教我使用美道,大與我殷王之命,命我代殷為王,正汝眾方諸侯。言天授我以此世也。 \par}

今我曷敢多誥,我惟大降爾四國民命。\footnote{今我何敢多誥汝而已,我惟大下汝四國民命。謂誅管、蔡、商、奄之君。}爾曷不忱裕之於爾多方?\footnote{汝何不以誠信行寬裕之道於汝眾方?欲其戒四國,崇和協。}爾曷不夾介乂我周王,享天之命?\footnote{夾,近也。汝何不近大見治於我周王,以享天之命,而為不安乎?○夾音協,注同。}今爾尚宅爾宅,畋爾田,爾曷不惠王熙天之命?\footnote{今汝殷之諸侯皆尚得居汝常居,臣民皆尚得畋汝故田,汝何不順從王政,廣天之命,而自懷疑乎?}爾乃迪屢不靜,爾心未愛。\footnote{汝所蹈行,數為不安,汝心未愛我周故。○數,色各反。}爾乃不大宅天命,爾乃屑播天命,\footnote{汝乃不大居安天命,是汝乃盡播棄天命。}爾乃自作不典圖忱於正。\footnote{汝未愛我周,播棄天命,是汝乃自為不常謀信於正道。}


{\noindent\zhuan\zihao{6}\fzbyks 傳“今我”至“之君”。正義曰:今我何敢多為言誥而已,實殺其君,非徒口告。管、蔡、商、奄,皆為叛逆受誅,故今因奄重叛而追說前事,言下四國民命。王肅以“四國”為四方之國,言“從今以後,四方之國苟有此罪,則必誅之”。謂戒其將來之事,與孔不同。 \par}

{\noindent\zhuan\zihao{6}\fzbyks 傳“夾近”至“安乎”。正義曰:夾其旁,旁是近義,故為近也。諸國疏遠周室,不肯以治為功,故責之。顧氏云:“汝眾方諸侯,何不常和協,相親近,大顯見治道於我周王,以享上天之命?而今何以不自安乎?” \par}

{\noindent\zhuan\zihao{6}\fzbyks 傳“今汝”至“疑乎”。正義曰:主遷於上,臣易於下,計汝諸侯之國,應隨殷降黜。今汝殷之諸侯皆尚得居汝常居,臣民畋汝故田。田宅不易,安樂如此,汝何不順從我周王之政,以廣上天之命,使天多佑?汝何故畏我周家,自懷疑乎?諸侯有國,故云“居汝常居”。臣民重田,故云“畋汝故田”。治田謂之“畋”,猶捕魚謂之“漁”,今人以營田求食謂之“畋食”,即此“畋亦田”之義也。 \par}

{\noindent\zhuan\zihao{6}\fzbyks 傳“汝未”至“正道”。正義曰:事君無二臣之道,為人臣者,常宜信之。汝未愛我周家,播棄天命,汝數為叛逆,是汝乃自為此不常謀信於正道。 \par}

我惟時其教告之,我惟時其戰要囚之,\footnote{我惟汝如是不謀信於正道,故其教告之,謂訊以文誥;其戰要囚之,謂討其倡亂,執其朋黨。○要,一遙反。訊音信。倡音唱。}至於再,至於三。\footnote{再,謂三監淮夷叛時。三,謂\CJKunderline{成王}即政又叛。言迪屢不靜之事。}乃有不用我降爾命,我乃其大罰殛之。\footnote{我教告戰要囚汝已至再三,汝其不用我命,我乃大下誅汝君,乃其大罰誅之。○殛,紀力反,本又作極。}非我有周秉德不康寧,乃惟爾自速辜。”\footnote{非我有周執德不安寧,自誅汝,乃惟汝自召罪以取誅。}

{\noindent\zhuan\zihao{6}\fzbyks 傳“我惟”至“朋黨”。正義曰:“教告”與“戰要囚”連文,則告以文辭,是將戰之時,“教告”謂伐紂之事。昭十三年說戰法云:“告之以文辭,董之以武師。”是將戰之時,於法當有文辭告前敵也。我惟汝如是不謀信於正道,故其教告之,謂訊以文辭。“訊”,告也,告以文辭,數其罪也。其“戰要囚之”,謂戰敗其師,執取其人,受其要辭而囚之。謂討其倡亂之人,囚執其朋黨也。此雖總言戰事,但下有至於再三,明此指伐紂也。 \par}

{\noindent\zhuan\zihao{6}\fzbyks 傳“再謂”至“之事”。正義曰:以伐紂為一,故“再”謂攝政之初,三監與淮夷叛時也,“三”謂\CJKunderline{成王}即政又叛也,言上“迪屢不靜”之事。 \par}

{\noindent\shu\zihao{5}\fzkt “今我”至“速辜”。正義曰:今我何敢多以言誥告於汝眾而已,我惟大下黜汝管、蔡、商、奄四國之君也。“民命”,謂民以君為命,謂誅殺四國之君也。我已殺汝四國君矣,汝何不以誠信之心,行寬裕之道於汝眾方諸侯?欲令懲創四國,務崇和協。言汝眾方諸侯何不崇和協,相親近,大顯見治道於我周王,以享愛上天之命,而執心不安乎?今爾殷之諸侯尚得居汝常居,臣民尚得畋汝故田,其安樂如此,汝何得不順從王政,以廣上天之命,而自懷疑乎?汝乃復所蹈行者,數為不安,時或叛逆,是汝心未愛我周家故也。汝乃不大居安天命,是汝乃欲盡播棄天命。汝不愛我周家,播棄天命,是汝乃自為此不常謀信於正道。言其心不常謀正道,故為背違之心。我惟汝如是不謀信於正道之故,其以言辭教告之。我惟汝如是不誠信於正道之故,其用戰伐要察囚繫之。由汝數為不信,故我教告汝,戰伐要囚汝,至於再,至於三。我教告汝,戰伐要囚汝,已至再三,如今而後乃復有不用我命者,我乃其大罰誅之。言我更將殺汝也。非我有周執德不安,數設誅罰,乃惟汝自召罪也。此章反覆殷勤者,恐其更有叛逆,故丁寧戒之。 \par}

王曰:“嗚呼!猷告爾有方多士暨殷多士,\footnote{王嘆而以道告汝眾方與殷多士。}今爾奔走臣我監五祀,\footnote{監謂成周之監,此指謂所遷頑民殷眾士。今汝奔走來徙臣我我監,五年無過,則得還本土。}越惟有胥伯小大多正,爾罔不克臬。\footnote{於惟有相長事小大眾正官之人,汝無不能用法。欲其皆用法。○臬,魚列反,馬作臬刂。長,丁丈反。}自作不和,爾惟和哉!爾室不睦,爾惟和哉!爾邑克明,爾惟克勤乃事。\footnote{小大多正自為不和,汝有方多士,當和之哉!汝親近室家不睦,汝亦當和之哉!汝邑中能明,是汝惟能勤汝職事。}爾尚不忌於凶德,亦則以穆穆在乃位,\footnote{汝庶幾不自忌,入於凶德,亦則用敬敬常在汝位。}克閱於乃邑謀介,爾乃自時洛邑,尚永力畋爾田。\footnote{汝能使我閱具於汝邑,而以汝所謀為大,則汝乃用是洛邑,庶幾長力畋汝田矣。言雖遷徙,而以修善,得反邑里。○閱音悅。}天惟畀矜爾,我有周惟其大介賚爾。\footnote{汝能修善,天惟與汝憐汝,我有周惟其大夫賜汝。言受多福之胙。}迪簡在王庭,尚爾事,有服在大僚。”\footnote{非但受憐賜,又乃蹈大道在王庭,庶幾修汝事,有所服行在大官。}


{\noindent\zhuan\zihao{6}\fzbyks 傳“王嘆”至“多士”。正義曰:言“有方多士與殷多士”,則此二者非一人也。“有方多士”當謂於時所有四方之諸侯也。“與殷多士”當謂遷於成周頑民之眾士也。下雲以“臣我監”者,謂成周之監,明此殷多士也。 \par}

{\noindent\zhuan\zihao{6}\fzbyks 傳“監謂”至“本土”。正義曰:下雲“自時洛邑”,此所戒成周之人,故知“監謂成周之監,此指謂所遷頑民殷家眾士”也。五年再閏,天道有成,故期以五年無過,則得還本土。以民性重遷,設期以誘之。 \par}

{\noindent\zhuan\zihao{6}\fzbyks 傳“於惟”至“用法”。正義曰:“胥”,相也。“伯”,長也。顏氏以“相長事”即“小大眾正官之人”也。 \par}

{\noindent\zhuan\zihao{6}\fzbyks 傳“汝庶”至“汝位”。正義曰:和順為善德,怨惡為凶德。“忌”謂自怨忌,上言“自作不和”,是怨忌也。\CJKunderwave{釋訓}云:“穆穆,敬也。”此戒小大正官之人,故云“敬敬常在汝位”。 \par}

{\noindent\zhuan\zihao{6}\fzbyks 傳“汝能”至“邑里”。正義曰:“閱”謂簡閱其事,觀其具足以否,故言“閱具於汝邑”。“介”,大也。以汝所謀為大,善其治理,聽還本國也。是由在洛邑修善,得反其邑里。王肅云:“其無成,雖五年亦不得反也。” \par}

{\noindent\shu\zihao{5}\fzkt “王曰嗚呼猷”至“大僚”。正義曰:王言而嘆曰:“嗚呼!我以道告汝在此所有四方之多士。”謂四方之諸侯及與殷之眾士,謂頑民遷成周者。因告四方諸侯,遂告成周之人,遍使諸侯知之。此章皆告成周之人辭也。“今汝成周之人,奔走勤事,臣我周之監成周者,五年無罪過,則聽汝還本土。於惟有相長事,謂小大眾正官之人,汝無有不能用法。”欲其皆用法也。“小大眾正官之人自為不和,汝眾官等自當和之哉!汝等親近室家不相和親,汝亦當和之哉!汝邑內之內若能明於和睦之道,汝惟能勤於汝之職事”。言是其教之使然。“汝能庶幾不自相怨忌,入於凶德,若能不入於凶德,亦則用敬敬之道,常在汝之職位,不黜退也。汝若能善相教誨,使我簡閱於汝邑,善汝之事,以汝所謀為大,則汝乃用是洛邑,庶幾得反本土,長得勤畋汝故田。汝能修善,天惟與汝憐汝,我有周惟其大大賞賜汝。汝非但受賞而已,其有蹈大道者,得在王庭被任用。庶幾汝事有所服行,在於大官”。恐其心未服,故丁寧勸誘之。 \par}

王曰:“嗚呼!多士,爾不勸忱我命,爾亦則惟不克享,凡民惟曰不享。\footnote{王嘆而言曰:“眾士,汝不能勸信我命,汝亦則惟不能享天祚矣,凡民亦惟曰不享於汝祚矣。”}爾乃惟逸惟頗,大遠王命,則惟爾多方探天之威,我則致天之罰,離逖爾土。”\footnote{若爾乃為逸豫頗僻,大棄王命,則惟汝眾方取天之威,我則致行天罰,離逖汝土,將遠徙之。○頗,破多反。探,吐南反。闢,四亦反。}


{\noindent\zhuan\zihao{6}\fzbyks 傳“王嘆”至“祚矣”。正義曰:“勸信我命”,勸勉而信順之。“凡民亦惟曰不享於汝祚矣”,言民亦不原汝之子孫長久矣。 \par}

{\noindent\zhuan\zihao{6}\fzbyks 傳“若爾”至“徙之”。正義曰:成周一邑之士,不得謂之多方,此蓋意在成周遷者,兼告四方諸國使知。亦如\CJKunderwave{康誥}王告\CJKunderline{康叔},並使諸侯知之。離遠汝土,更遠徙之。鄭云:“分離奪汝土也。”與孔異也。 \par}

{\noindent\shu\zihao{5}\fzkt “王曰嗚呼”至“爾土”。正義曰:王言而嘆曰:“嗚呼!成周之眾士,汝若不能勸勉信用我之教命,汝則惟不能多受天福祚矣,凡民惟曰不享於汝祚矣。汝乃惟為逸豫,惟為頗僻,大遠棄王命,則惟汝眾方自取天之威刑,我則致天之罰於汝身,將遠徙之,使離遠汝之本土。” \par}

王曰:“我不惟多誥,我惟祗告爾命。”\footnote{我不惟多誥汝而已,我惟敬告汝吉凶之命。}又曰:“時惟爾初,不克敬於和,則無我怨。”\footnote{又誥汝:“是惟汝初不能敬於和道,故誅汝。汝無我怨。”解所以再三加誅之意。}


{\noindent\zhuan\zihao{6}\fzbyks 傳“又誥”至“之意”。正義曰:“又誥”者,更言王意,又謂汝曰也。以上王誥已終,又起別端,故更稱王又復言曰。以序雲\CJKunderline{成王}在豐誥庶邦,則此篇是王親誥之辭,又稱“王曰”者是也。其有\CJKunderline{周公}稱王告者,則上雲“\CJKunderline{周公}曰:王若曰”是也,又曰“嗚呼!王若曰”是也。顧氏云:“又曰者,是王又復言曰也。” \par}

{\noindent\shu\zihao{5}\fzkt “王曰我”至“我怨”。正義曰:王曰:“我今告戒汝者,不惟多為言誥汝而已,惟敬告汝吉凶之命。從我則吉,違我則兇,汝命吉凶在此言也。”王又謂:“汝所以再三被誅者,是惟汝初不能敬於和道,故致此爾。汝自取之,則無於我有怨。” \par}

\section{立政第二十一}


\CJKunderline{周公}作\CJKunderwave{立政}。\footnote{\CJKunderline{周公}既致政\CJKunderline{成王},恐其怠忽,故以君臣立政為戒。}立政\footnote{言用臣當共立政,故以名篇。}\CJKunderline{周公}若曰:“拜手稽首,告嗣天子王矣。”\footnote{順古道盡禮致敬,告\CJKunderline{成王},言:“嗣天子,今已為王矣,不可不慎。”}用咸戒於王曰:“王左右常伯、常任、準人、綴衣、虎賁。”\footnote{\CJKunderline{周公}用王所立政之事皆戒於王曰,常所長事、常所委任,謂三公六卿;準人平法,謂士官;綴衣掌衣服,虎賁以武力事王,皆左右近臣,宜得其人。○任,而鴆反。準,之允反。綴,徐丁衛反,又丁劣反。賁音奔。長,丁丈反,除篇末文注“以長”直良反,餘並同。}\CJKunderline{周公}曰:“嗚呼!休茲,知恤鮮哉!。\footnote{嘆此五者立政之本,知憂得其人者少。○鮮,息淺反。}


{\noindent\zhuan\zihao{6}\fzbyks 傳“順古”至“不慎”。正義曰:\CJKunderline{周公}既拜手稽首,而後發言。還自言“拜手稽首”,示己重其事,欲令受其言,故盡禮致敬以告王也。\CJKunderwave{召誥}云:“拜手稽首,旅王若公。”亦是\CJKunderline{召公}自言已拜手稽首,與此同也。\CJKunderline{成王}嗣世而立,故呼\CJKunderline{成王}為“嗣天子”。\CJKunderline{周公}攝政之時,\CJKunderline{成王}未親王事,此時既已歸政於\CJKunderline{成王},故言“今已為王矣,不可不慎”也。王肅以為於時\CJKunderline{周公}會群臣共戒\CJKunderline{成王},其言曰“拜手稽首”者,是\CJKunderline{周公}贊群臣之辭。 \par}

{\noindent\zhuan\zihao{6}\fzbyks 傳“\CJKunderline{周公}”至“其人”。正義曰:此以“立政”名篇,知“用咸戒”者是“\CJKunderline{周公}用王所立政之事皆戒於王”也。三公,臣之尊者,知常所長事謂三公也。六卿分掌國事,王之所任,知常所委任謂六卿也。“準”訓平也,平法之人謂士官也。士,察也,察獄之官用法必當均平,故謂獄官為準人。\CJKunderwave{周禮}司寇之長在常任之內,此士官當謂士師也。衣服必連綴著之,此歷言官人,知“綴衣”是掌衣服者。此言親近大臣,必非造衣裳者。\CJKunderwave{周禮}:“大僕,下大夫。掌正王之服位,出入王之大命。”此掌衣服者,當是大僕之官也。\CJKunderwave{周禮}:“虎賁氏,下大夫。”言其若虎賁獸,是以武力事王者。此皆左右近臣,宜得其人,言其急於餘官。得其人者,文官得其文人,武官得其武人,違才易務,皆為非其人也。 \par}

{\noindent\zhuan\zihao{6}\fzbyks 傳“嘆此”至“者少”。正義曰:此五官皆親近王,故嘆此五者立政之本也。“休”,美也。王肅雲“此五官美哉”,是“休茲”為美此五官也。嘆其官之美,美官不可不委賢人用之,故嘆之。“知憂得其人者少”,下句惟言\CJKunderline{禹}、湯、文、武官得其人,是知憂得人者少也。 \par}

{\noindent\shu\zihao{5}\fzkt “\CJKunderline{周公}”至“鮮哉”。正義曰:王之大事在於任賢使能,\CJKunderline{成王}初始即政,猶尚幼少,\CJKunderline{周公}恐其怠忽政事,任非其人,故告以用臣之法。\CJKunderline{周公}順古道而告王曰:“我敢拜手稽首,告嗣世天子\CJKunderline{成王},今已為王矣。王者當立善政,其事不可不慎。”\CJKunderline{周公}既為此言,乃用王所立政之事皆戒於王曰,王之親近左右,常所長事,謂三公也;常所委任,謂六卿也;平法之人,謂獄官也;綴衣之人,謂掌衣服者也;虎賁,以武力事王者;此等皆近王左右,最須得人。\CJKunderline{周公}既歷言此官,復言而嘆曰:“嗚呼,美哉!此五等之官,立政之本也。知憂此官宜得賢人者少也。” \par}

古之人迪惟有夏,乃有室大競,籥俊尊上帝,\footnote{古之人道惟有夏禹之時,乃有卿大夫室家大強,猶乃招呼賢俊,與共尊事上天。○籥音預。}迪知忱恂於九德之行。\footnote{禹之臣蹈知誠信於九德之行,謂賢智大臣。九德,\CJKunderline{皋陶}所謀。○忱,市林反。恂音荀。行如字,徐下孟反。}乃敢告教厥後曰:‘拜手稽首,後矣!’曰,宅乃事,宅乃牧,宅乃準,茲惟後矣。\footnote{知九德之臣乃敢告教其君以立政。君矣,亦猶王矣。宅,居也,居汝事,六卿掌事者。牧,牧民,九州之伯。居內外之官及平法者皆得其人,則此惟君矣。}謀面,用丕訓德,則乃宅人,茲乃三宅無義民。\footnote{謀所面見之事,無疑則能用大順德,乃能居賢人於眾官。若此則乃能一居無義民。大罪宥之四裔,次九州之外,次中國之外。}桀德惟乃弗作往任,是惟暴德,罔後。\footnote{桀之為德,惟乃不為其先王之法、往所委任,是惟暴德之人,故絕世無後。}


{\noindent\zhuan\zihao{6}\fzbyks 傳“古之”至“上天”。正義曰:經言“古之人迪”,傳言“古之人道”,當說古之求賢人之道也。王肅云:“古之人道,惟有夏之\CJKunderline{大禹}為天子也。”其意言古人之道說有此事,孔意似不然也。孔以“大夫”稱家室,猶家也。“籥”訓呼也,招呼者乃是臣下之事,故以為夏禹之時,乃有卿大夫室家大強,猶乃招呼在外賢俊,與之共立於朝,尊事上天也。言君既求賢臣之助,言天子事天,臣成君事,故言“共尊事上天”。 \par}

{\noindent\zhuan\zihao{6}\fzbyks 傳“\CJKunderline{禹}之”至“所謀”。正義曰:九德之行,非一人能備,言\CJKunderline{禹}之臣蹈知九德之行,極言其賢智大臣也。\CJKunderline{禹}時伯益之輩,乃可以當此。經典之文,更無九德之事,惟有\CJKunderline{皋陶}謀九德,故言九德。\CJKunderline{皋陶}所謀者,即“寬而慄,柔而立,愿而恭,亂而敬,擾而毅,直而溫,簡而廉,剛而塞,強而義”是也。 \par}

{\noindent\zhuan\zihao{6}\fzbyks 傳“知九”至“君矣”。正義曰:進言戒君非大賢不可,故“知九德之臣乃敢告教其君以立政”也。“君矣”亦猶言“王矣”,言已為君矣,不可不慎也。“君”、“王”一也,變文以相避爾。“宅”訓居也,居汝事,須得賢人,六卿各掌其事者也。居汝牧,九州之伯主養民,亦須得賢人養其民也。居汝準,士官主理刑法,亦須賢人平其獄也。六卿掌內,州牧掌外,內外之官及平法三事皆得其人,則此惟為君矣。言群官失職,則不成為君也。上句\CJKunderline{周公}戒王,歷言五官,其內無州牧。此惟言三官,加州牧者,俱是逐急言之,其有詳略爾。\CJKunderwave{曲禮}云:“九州之長曰牧。”\CJKunderwave{王制}云:“千里之外設方伯,八州八伯。”然則“牧”、“伯”一也。“伯”者言一州之長,“牧”者言牧養下民,“牧”、“伯”俱得言之,故孔以“伯”解“牧”。\CJKunderline{鄭玄}云:“殷之州牧曰伯,虞夏及周曰牧。”與孔不同。 \par}

{\noindent\zhuan\zihao{6}\fzbyks 傳“謀所”至“之外”。正義曰:凡人為主,皆欲臣賢,但大佞以忠,賢不可別。欲知其遠,先驗於近,但\CJKunderline{禹}能謀所面見之事。官賢人,既得其官,分別善惡,無所疑惑。仁賢必用,邪佞必退,然後舉直錯諸枉,則為能用大順德,如是乃能居賢人於眾官。賢人既得居官,則能分別善惡,無義之民必獲大罪。量其輕重,斥之遠地,乃能三處居此無義罪人。三居者,“大罪宥之四裔,次九州之外,次中國之外。”“四裔”者,四海之表最遠者也。“次九州之外者”,四海之內,要服之外。“次中國之外”者,謂罪人所居之國外也,猶若衛人居於晉,去本國千里。故孔注\CJKunderwave{舜典}云:“次千里之外是也。”鄭云:“三處者,自九州之外至於四海,三分其地,遠近若周之夷、鎮、蕃也。”與孔不同。 \par}

{\noindent\shu\zihao{5}\fzkt “古之人”至“罔後”。正義曰:既言知憂得人者少,乃遠述上世之事,此言\CJKunderline{禹}與桀也。古之人能用此求賢之道者,惟有夏禹之時。乃有群臣卿大夫皆是賢人,室家大強,猶尚招呼賢俊之人,與共立於朝,尊事上天。\CJKunderline{禹}之臣蹈知誠信於九德之行者,乃敢告教其君曰:“我敢拜手稽首,君今已為君矣,不可不慎也。”戒其君即告曰,居汝掌事之六卿,居汝牧民之州伯,居汝平法之獄官,使此三者皆得其人,則此惟為君矣。言不得賢人,不成為君也。\CJKunderline{禹}能謀所面見之事,無所疑惑,用大明順之德,則乃能居賢人於官。賢人在官,職事修理,乃能三處居無義之民。善人在朝,惡人黜遠,其國乃為治矣。及其末年,桀乃為天子。桀之為德,惟乃不為其先王之法、往所委任,是暴德之人。以此故,絕世無後。得賢人則興,任小人則滅,是須官賢人以立政也。 \par}

亦越\CJKunderline{成湯}陟,丕釐上帝之耿命,\footnote{桀之昏亂,亦於\CJKunderline{成湯}之道得升,大賜上天之光命,王天下。○釐,力之反。耿,工迥反,徐工穎反,又工永反,下同。王,往況反。}乃用三有宅,克即宅,曰三有俊,克即俊。\footnote{湯乃用三有居惡人之法,能使就其居。言服罪。又曰,能用剛柔正直三德之俊,能就其悛事。言明德。}嚴惟丕式,克用三宅三俊。\footnote{言湯所以能嚴威,惟可大法象者,以能用三居三德之法。}其在商邑,用協於厥邑,其在四方,用丕式見德。\footnote{湯在商邑,用三宅三俊之道和其邑。其在四方,用是大法見其聖德。言遠近化。}


{\noindent\zhuan\zihao{6}\fzbyks 傳“桀之”至“天下”。正義曰:“\CJKunderline{成湯}之道得升”,謂從下而升於天,故天賜之以光命,使之得王天下為天子也。“釐,賜”、“耿,光”皆\CJKunderwave{釋詁}文。 \par}

{\noindent\zhuan\zihao{6}\fzbyks 傳“湯乃”至“明德”。正義曰:\CJKunderwave{皋陶謨}“九德”即\CJKunderwave{洪範}之“三德”,細分以為九爾。以此知“三俊”即是\CJKunderwave{洪範}所言“剛克、柔克、正直”三德之俊也。“能就其俊事。言明德”者,用以俊乂居官,顯明其有德也。上句言“則乃宅人,茲乃三宅無義民”,先言用賢,後言去惡。此經先言“三有宅”,後言“曰三有俊”者,用賢去惡,俱是立政之本。上句先說夏禹,言得賢然後去惡,見其須賢之切。及說\CJKunderline{成湯}、文、武,先言去惡,後言用賢,又見惡宜速去。或先或後,所以互相見爾。 \par}

{\noindent\shu\zihao{5}\fzkt “亦越”至“見德”。正義曰:不有所廢,則無以興。桀之滅亡,夏家乃以開道湯德。此言湯之能用人也。桀之昏亂,亦於\CJKunderline{成湯}之道得升聞於天,大賜受上天之光命,得王有天下。湯既為王,乃用三有居惡人之法,能使各就其居處。言皆服其罪也。又曰用三德之俊人,能使各就其俊事。言皆明其德也。湯所以能嚴威,惟可大法象者,以其能用三居三俊之法故也。\CJKunderline{成湯}其在商邑,用此三居三俊之道,和於其邑。其在四方,用是斷罪任賢之大法,見其聖德於民。言遠近皆從化也。 \par}

嗚呼!其在受德暋,惟羞刑暴德之人,同於厥邦。\footnote{受德,紂字。\CJKunderline{帝乙}愛焉,為作善字,而反大惡自強,惟進用刑,與暴德之人同於其國,併為威虐。○受德,紂字,馬云:“受所為德也。”暋,眉謹反,徐亡巾反,一音閔。為,於為反,下“為之”同。強,其丈反。}乃惟庶習逸德之人,同於厥政。\footnote{乃惟眾習為過德之人,同於其政。言不任賢。}帝欽罰之,乃伻我有夏,式商受命,奄甸萬姓。\footnote{天以紂惡,故敬罰之。乃使我周家王有華夏,得用商所受天命,同治萬姓。言皇天無親,佑有德。○伻,普耕反,徐敷耕反,又甫耕反。}


{\noindent\zhuan\zihao{6}\fzbyks 傳“受德”至“威虐”。正義曰:\CJKunderwave{泰誓}三篇,惟單言“受”,而此雲“受德”者,則“德”本配“受”,共為一人,故知“受德”是紂字也。既“受”之與“德”共為紂字,而經或言“受”,或言“受德”者,呼之有單復爾。其人實為大惡,“德”字乃為善名,非是時人呼有德。知是\CJKunderline{帝乙}愛焉,為作善字,望其為善,而反為大惡,以其行反其字,明非時人呼也。\CJKunderwave{釋詁}云:“暋,強也。”“暋”即昏也,故訓為強,言紂自強為惡,惟進用刑罰。身既進用刑罰,則愛好暴虐之人,故為與之同於其國,言併為威虐。 \par}

{\noindent\zhuan\zihao{6}\fzbyks 傳“乃惟”至“任賢”。正義曰:“暴德”言以暴虐為德,“逸德”言以過惡為德。習效為之眾者,言其所任多也。紂任眾為過德之人,與之同於其政,言其不任賢也。與暴德同於其國,與惡德同於其政,其事一也,異言之爾。\CJKunderwave{牧誓}所云“四方之多罪逋逃,是信是使,是以為大夫卿士。俾暴虐於百姓,以奸究於商邑”,是其事也。 \par}

{\noindent\zhuan\zihao{6}\fzbyks 傳“天以”至“有德”。正義曰:言天知其惡,熟詳審下罰,故言“敬罰”也。商本受天命,周亦受天命,故言“用商所受天命,同治萬姓”。\CJKunderwave{釋言}云:“弇,同也。”同為天於治萬姓,與商同也。此經之意,言周家有德,皇天親有德也。王肅云:“敬罰者,謂須暇五年。” \par}

{\noindent\shu\zihao{5}\fzkt “嗚呼”至“萬姓”。正義曰:既言湯以用賢而興,又說紂之失人而滅。\CJKunderline{周公}又嘆曰:“嗚呼!其在殷王受德,本性大惡自強,惟進用刑罰,與暴德之人同治其國,併為威虐。乃惟眾習為過德之人,與之同共於其政。由其任同惡之人,故上天敬誅罰之,乃使我周家王有華夏,用商所受天命,同治天下萬姓。”言周能用賢,天親有德,故得為天子。 \par}

“亦越\CJKunderline{文王}、\CJKunderline{武王},克知三有宅心,灼見三有俊心,\footnote{紂之不善,亦於文武之道大行,以能知三有居惡人之心,灼然見三有賢俊之心。}以敬事上帝,立民長伯。\footnote{言文武知三宅三俊,故能以敬事上天,立民正長。謂郊祀天,建諸侯。}


{\noindent\zhuan\zihao{6}\fzbyks 傳“紂之”至“之心”。正義曰:桀之昏亂開\CJKunderline{成湯},紂之不善開文武,其事同也。於\CJKunderline{成湯}言能受上天之命,於文武雲能敬事上帝,前聖後聖為行必同,交錯為文,所以互相見爾。\CJKunderline{文王}受命,\CJKunderline{武王}伐紂,二聖共\CJKunderline{成王}道,故文武總言之。猶\CJKunderwave{詩序}雲“文武以\CJKunderwave{天保}已上治內,\CJKunderwave{采薇}已下治外”,文武並言,與此同也。\CJKunderline{文王}之時,未定天下,所立之官,亦未具足。下經所言“立政任人”已下,“三亳阪尹”已上,其所舉官屬,多是文武時事,以見二聖同道,父作之,子述之,言其相成爾。故以“能知三有居惡人之心,灼然見三有賢俊之心”,言\CJKunderline{文王}之聖心能揆度知惡人真惡,須屏黜之;知賢人實賢,須舉用之;故去惡進賢,皆得其所。賢人難識,故特言“灼然”,言其知之審也。 \par}

{\noindent\zhuan\zihao{6}\fzbyks 傳“言文”至“諸侯”。正義曰:上天之道,與善去惡,三宅三俊,行合天心。言文武知三宅三俊,故能敬事上帝。“伯”亦長也,故言“立民正長”。天子祭天,知“敬事上帝”,謂“郊祀天”也。天子建國,知“立民長伯”,謂“建諸侯”也。以下句“立政任人”已下,歷言朝廷之臣與蠻夷眾君,知此“立民長伯”主謂諸侯。\CJKunderwave{詩·周頌·維清}述\CJKunderline{文王}之德言“肇禋”,\CJKunderwave{大雅·皇矣}美\CJKunderline{文王}之伐言“是類”,“類”、“禋”皆是祭天之名,是\CJKunderline{文王}已祀天矣。\CJKunderline{文王}未得封建諸侯,其建諸侯,維\CJKunderline{武王}時爾。 \par}

{\noindent\shu\zihao{5}\fzkt “亦越”至“長伯”。正義曰:既言上天去惡與善,滅殷興周,即說\CJKunderline{文王}、\CJKunderline{武王}能用求賢審官之事。桀惡所以興\CJKunderline{成湯},紂惡所以開文武,言紂之不善,亦於\CJKunderline{文王}、\CJKunderline{武王}使得其道大行。能知居三有惡人之心,居之皆得其所,言服其罪也。灼然見三有賢俊之心,用之皆得其人,言明其德也。文武知此三宅三俊,故能敬事上天,稱天心也。立民正長,合民心也。 \par}

立政,任人、準夫、牧,作三事,\footnote{文武亦法\CJKunderline{禹}湯以立政,常任、準人及牧,治為天地人之三事。}虎賁、綴衣、趣馬小尹,\footnote{趣馬,掌馬之官。言此三者雖小官長,必慎擇其人。○趣,七口反。}左右攜僕、百司庶府,\footnote{雖左右攜持器物之僕,及百官有司主券契藏吏,亦皆擇人。○券音勸。契,苦計反。藏,才浪反。}大都小伯、藝人表臣、百司,\footnote{小臣猶皆慎擇其人,況大都邑之小長,以道藝為表幹之臣及百官有司之職,可以非其任乎?}


{\noindent\zhuan\zihao{6}\fzbyks 傳“文武”至“三事”。正義曰:前聖後聖,其道皆同,未必相放法也。後人法前,自是常事,因其上說\CJKunderline{禹}湯立政,故言“文武亦法\CJKunderline{禹}湯以立政”也。“任人”則前經所云“常任”六卿也,“準夫”則“準人”也,“牧”者前雲“宅乃牧”也。前文有“常伯”、“綴衣”、“虎賁”,不言“牧”,此不言“常伯”、“綴衣”、“虎賁”而言“牧”者,以前文先舉朝臣,故不言“牧”,前已備文,故此不言“常伯”。其“綴衣”、“虎賁”而言“牧”者,以下文自詳,故此惟舉內外要官者言之,故內官舉“任人”、“準夫”,外官舉“牧”。故下云:“繼自今,我立政。立事、準人、牧夫,我其克灼知厥若。”又云:“自古商人,亦越我周\CJKunderline{文王}立政、立事、牧夫、準人,則克宅之,克由繹之,茲乃俾乂。”皆據內外要重官以言之。“夫”即人也,立官所以事天地,治人民,為此三事而已,故以“三事”謂天地人也。王肅云:“\CJKunderline{文王}所以立政,任人,常任也;準夫,準人也;牧者,諸侯之長也。”與孔意同。 \par}

{\noindent\zhuan\zihao{6}\fzbyks 傳“趣馬”至“其人”。正義曰:\CJKunderwave{周禮}趣馬為校人屬官,馬一十二匹,立趣馬一人,“掌贊正良馬,而齊其飲食”,是掌馬之小官也。綴衣是大僕也,虎賁、大僕皆下大夫也。此三公六卿,亦為小尹之官。雖文止三官,亦包通在下之屬官。三官之下小官多矣,趣馬即下士,其馬一匹,有圉師一人,是趣馬之下猶有小官也。 \par}

{\noindent\zhuan\zihao{6}\fzbyks 傳“雖左”至“擇人”。正義曰:諸官有所務從業,從王左右攜持器物之僕,謂寺人、內小臣等也。“百司庶府”,謂百官有司之下,主券契府藏之吏,謂其下賤人,非百官有司之身也。言此等亦皆擇人。 \par}

{\noindent\zhuan\zihao{6}\fzbyks 傳“小臣”至“任乎”。正義曰:“小臣猶皆擇人,況大都邑之小長”,謂公卿,都邑之內大夫士及邑宰之屬,以身有道藝、為民之表的楨幹之臣。其都邑之內屬官,謂之“小長”。\CJKunderwave{周禮·太宰職}云:“乃施則于都鄙,而建其長,立其兩,設其伍,陳其殷。”“兩”謂兩卿,“長”謂公卿,“伍”謂大夫,“殷”謂眾士是也。 \par}

太史、尹伯、庶常吉士,\footnote{太史,下大夫,掌邦六典之貳;尹伯,長官大夫;及旅掌常事之善士,皆得其人。}司徒、司馬、司空、亞旅,\footnote{此有三卿及次卿眾大夫,則是文武未伐紂時。舉文武之初以為法則。}夷微、盧烝、三亳、阪尹。\footnote{蠻夷微、盧之眾帥,及亳人之歸\CJKunderline{文王}者,三所為之立監,及阪地之尹長,皆用賢。○阪音反。}

{\noindent\zhuan\zihao{6}\fzbyks 傳“太史”至“其人”。正義曰:\CJKunderwave{周禮}“太史,下大夫二人”,“掌建邦之六典”。又\CJKunderwave{太宰職}亦云“掌建邦之六典”。太史副貳,太宰掌其正,太史掌其貳。“六典”謂治典、教典、禮典、政典、刑典、事典,六卿所掌者也。“掌邦六典之貳”,其所掌事重,故特言之。“尹伯”,長官大夫。\CJKunderwave{周禮}每官各有長,若太史為史官之長,大司樂為樂官之長,如此類皆是也。“及眾掌常事之善士”,謂士為長官者。其大夫及士不為長官者,則前雲“百司”也。居官必須善人,此是總舉眾官,故特言“吉士”。 \par}

{\noindent\zhuan\zihao{6}\fzbyks 傳“此有”至“法則”。正義曰:\CJKunderline{周公}攝政之時,制禮作樂,其作\CJKunderwave{立政}之篇,必在制禮之後。\CJKunderwave{周禮}六卿,而“此有三卿及次卿眾大夫”,則是副卿之大夫,有若\CJKunderwave{周禮}小宰之類是也。此文武未伐紂之時也,遠舉文武之初以為法則爾。\CJKunderwave{泰誓}下篇雲“王乃大巡六師”,“六師”則六軍也,軍將皆命卿,即伐紂之時已立六卿矣。\CJKunderwave{牧誓}亦云“司徒、司馬、司空”,舉之三卿者,彼傳已解之雲“指誓戰者”也。 \par}

{\noindent\zhuan\zihao{6}\fzbyks 傳“蠻夷”至“用賢”。正義曰:\CJKunderwave{牧誓}所云,有“微、盧、彭、濮人”,此舉“夷微、盧”以見彭、濮之等諸夷也。“烝”訓眾也。此篇所言,皆立官之事,此經惟“阪”下言“尹”,則“夷、微”已下以一“尹”總之,故傳言“蠻夷微、盧之眾帥,及亳民之歸\CJKunderline{文王}者,三所為之立監,及阪地之尹長”。故言“帥”,言“監”,亦是言為之立長,義出經文“尹”也。“亳”是湯之舊都,此言“三亳”,必是亳民分為三處。此篇說立官之意,明是分為三亳,必是三所各為立監也。“亳民之歸\CJKunderline{文王}”,經傳未有其事,\CJKunderline{文王}既未伐紂,亳民不應歸之。鄭、王所說皆與孔同。言亳民歸\CJKunderline{文王}者,蓋以此章雜陳\CJKunderline{文王}、\CJKunderline{武王}時事,其言以\CJKunderline{文王}為主,故先儒因言亳民歸\CJKunderline{文王}爾。即如此意,三亳為已歸周,必是\CJKunderline{武王}時也。“及阪地之尹長”,傳言其山阪之地立長爾,不知其指斥何處也。\CJKunderline{鄭玄}以“三亳阪尹”者共為一事,云:“湯舊都之民服\CJKunderline{文王}者,分為三邑,其長居險,故言阪尹。蓋東成皋,南轘轅,西降谷也。”皇甫謐以為“三亳,三處之地,皆名為亳。蒙為北亳,谷熟為南亳,偃師為西亳”。古書亡滅,既無要證,未知誰得旨矣。 \par}

{\noindent\shu\zihao{5}\fzkt “立政”至“阪尹”。正義曰:言文武亦法\CJKunderline{禹}湯,審官以立美政。“任人”謂六卿。“準夫”者,平法之人,謂理獄官也。“牧”者,九州之牧。治為天地人之三事。自“虎賁”已下,歷舉官名,言此官皆須得其人。不以官之尊卑為次,蓋以從近而至遠。虎賁、綴衣、趣馬,三者官雖小,須慎擇其人。乃至左右攜持器物之僕,及百官有司之下至眾府藏之吏,亦須擇其人。既言近王小官,及遠官大者。小官猶須擇人,況乎大都邑之小長,與有道藝之人為表幹之臣,及百官有司之職,可以非其任乎?以近臣況遠臣,以小官況大官。既以近小況遠大,又舉官之次而掌事要者。若太史下大夫、長官大夫及眾掌常事之善士,皆須得其人。更舉官之大者,司徒、司馬、司空之卿,及次卿之眾大夫,皆須得其人。既略言內外之官,又更遠及夷狄蠻夷微、盧之眾帥,與三處亳民之監,及阪地之尹長,皆須用賢人。言文武於此諸官,皆求賢人為之也。 \par}

\CJKunderline{文王}惟克厥宅心,乃克立茲常事司牧人,以克俊有德。\footnote{文王惟其能居心遠惡舉善,乃能立此常事司牧人,用能俊有德者。○遠,於萬反。}\CJKunderline{文王}罔攸兼於庶言,庶獄庶慎,惟有司之牧夫。\footnote{文王無所兼知於譭譽眾言,及眾刑獄,眾當所慎之事,惟慎擇有司牧夫而已。勞於求才,逸於任賢。○譽音餘,又如字。}是訓用違,庶獄庶慎,\CJKunderline{文王}罔敢知於茲。\footnote{是萬民順法,用違法,眾獄眾慎之事,\CJKunderline{文王}一無敢自知於此,委任賢能而已。}


{\noindent\zhuan\zihao{6}\fzbyks 傳“\CJKunderline{文王}”至“德者”。正義曰:上言\CJKunderline{文王}能知三宅三俊,知此言“能居心”者,以遠惡舉善居其心也。既遠惡舉善,乃能立此常事,用賢養民,是人君之常事也。 \par}

{\noindent\zhuan\zihao{6}\fzbyks 傳“\CJKunderline{文王}”至“任賢”。正義曰:下雲“是訓用違”,即是在上“庶言”也。“是訓”則稱譽之事,“用違”則毀損之事,但分析言之爾。 \par}

{\noindent\shu\zihao{5}\fzkt “\CJKunderline{文王}”至“於茲”。正義曰:上既總言文武,此又分而說之。\CJKunderline{文王}惟能其居心遠惡舉善,乃能立此常事其主養人之官,用能俊有德者。既任用俊人,每事委之,\CJKunderline{文王}無所兼知於眾人之言,或毀或譽,\CJKunderline{文王}皆不知也。眾獄斷罪得失,\CJKunderline{文王}亦不得知也。眾所當慎之事,\CJKunderline{文王}亦不得知也。惟慎擇在朝有司,在外牧養民之夫。是時萬民或順於法,或用違法,眾刑獄,眾所慎之事,\CJKunderline{文王}一皆無敢自知於此,惟委任賢能而已。 \par}

亦越\CJKunderline{武王},率惟敉功,不敢替厥義德,\footnote{亦於\CJKunderline{武王}循惟\CJKunderline{文王}無安天下之功,不敢廢其義德,奉遵父道。○敉,亡婢反。}率惟謀從容德,以並受此丕丕基。\footnote{\CJKunderline{武王}循惟謀從\CJKunderline{文王}寬容之德,故君臣並受此大大之基業,傳之子孫。傳,直專反。}


{\noindent\zhuan\zihao{6}\fzbyks 傳“\CJKunderline{武王}”至“子孫”。正義曰:以言“並受”,則非獨王身,故以為“君臣並受此大大之基業”。謀從寬容之德,是與臣謀,及基業成就,則君臣共有,故言“並受”。且王為天子,臣為諸侯,皆受基業,各傳子孫,是亦為“並受”也。 \par}

{\noindent\shu\zihao{5}\fzkt “亦越”至“丕基”。正義曰:亦於\CJKunderline{武王}遵循父道,所循惟\CJKunderline{文王}撫安天下之功,不敢廢其\CJKunderline{文王}義德。言奉行遵父道也。又言\CJKunderline{武王}遵循者,惟謀從\CJKunderline{文王}寬容之德,故\CJKunderline{武王}君臣能並受此大大之基業。謂受命為天子,傳之子孫。 \par}

“嗚呼!孺子王矣。\footnote{嘆稚子今以為王矣,不可不勤法祖考之德。}繼自今,我其立政、立事、準人、牧夫,我其克灼知厥若,丕乃俾亂。\footnote{繼用今已往,我其立政大臣、立事小臣、及準人、牧夫,我其能灼然知其順者,則大乃使治之。言知臣下之勤勞,然後莫不盡其力。○俾,必爾反,下同。治,直吏反,下同。}相我受民,和我庶獄庶慎,時則勿有間之。\footnote{能治我所受天民,和平我眾獄眾慎之事,如是則勿有以代之。言不可復變。○相如字,馬息亮反,下“勱相”同。間,間廁之間。復,扶又反。}自一話一言,我則末惟成德之彥,以乂我受民。\footnote{言政當用一善,善在一言而已。欲其口無擇言。如此我則終惟有成德之美,以治我所受之民。○話,戶怪反。}


{\noindent\zhuan\zihao{6}\fzbyks 傳“繼用”至“心力”。正義曰:自此已下四言“繼自今”者,凡人靡不有初,鮮克有終,恐王不能終之,戒\CJKunderline{成王}使繼續,從今已往常用賢也。“自”訓為從,亦訓為用,此傳言“用今已往”,下傳言“從今已往”,其意同也。“政”、“事”相對,則“政”大“事”小,故以“立政”為大臣,“立事”為小臣。及“準人”、“牧夫”,略舉四者以總諸臣,戒王任此人也。其能灼然知其能順於事者,則大乃使治。顧氏云:“君能知臣下順於事,則臣感君恩,大乃治理,各盡心力也。” \par}

{\noindent\zhuan\zihao{6}\fzbyks 傳“能治”至“復變”。正義曰:“相”訓助也,助君所以治民事,故“相”為治。天命王者,使之治民,則天與王者此民,故言“能治我所受天民”也。能治下民,理眾獄眾慎之事,使得其所,則為政之大要,能如此,則勿有以代之。言此法盡善,不可復變易也。或據臣身既能如此,不可以餘人代之也。 \par}

{\noindent\zhuan\zihao{6}\fzbyks 傳“言政”至“之民”。正義曰:\CJKunderwave{釋詁}云:“自,用也。話,言也。”舍人曰:“話,政之善言也。”孫炎曰:“話善之言也。”然則“話”之與“言”是一物也。“自一話”者,言人君為政,當用純一善言。又云“一言”者,純一善言,在於一言而已。謂發號施令,當須純一,不得差貳,欲令其口無可擇之言也。顧氏云:“人君為政之道,當須用一善而已,為善之法,惟在一言也。‘末’訓為終,‘彥’訓為美,王能出言皆善,口無可擇,如此我王則終惟有成德之美,以治我所受天民矣。”\CJKunderwave{釋訓}云:“美士為彥。”故“彥”為美。 \par}

{\noindent\shu\zihao{5}\fzkt “嗚呼孺子”至“受民”。正義曰:\CJKunderline{周公}既歷說\CJKunderline{禹}湯文武,乃復指戒\CJKunderline{成王},“嗚呼”而嘆,孺子今已為王矣。既正位為王,事不可不慎。繼續從今已往,我王其與立政,謂大臣也,其與立事,謂小臣也,平法之人及養民之夫,此等諸臣,我王其能察之灼然,知其順於事者,則大乃使之治理。言知其能有勤勞,各盡心力。然後用此賢臣治我所受天民,和平我眾獄訟,及眾當所慎之事,必能如是,則勿復有以代之。言其法不可復變也。政從君出,為人主用是一善之言,善在一言而已,勿以惡言亂之。王能如是,我王則終惟有成德之美,以治我所受天民矣。 \par}

嗚呼!予旦已受人之徽言,咸告孺子王矣。\footnote{嘆所受賢聖說\CJKunderline{禹}湯之美言,皆以告稚子王矣。○稚,直吏反,本亦作稺。}繼自今,文子文孫,其勿誤於庶獄庶慎,惟正是乂之。\footnote{文子文孫,\CJKunderline{文王}之子孫。從今已往,惟以正是之道治眾獄眾慎,其勿誤。}自古商人,亦越我周\CJKunderline{文王}立政、立事、牧夫、準人,則克宅之,克由繹之,茲乃俾乂。\footnote{言用古商湯,亦於我周\CJKunderline{文王}立政立事,用賢人之法,能居之於心,能用陳之,此乃使天下治。○繹音亦。}


{\noindent\zhuan\zihao{6}\fzbyks 傳“言用”至“下治”。正義曰:上陳\CJKunderline{禹}湯文武,此覆上文,惟言湯與\CJKunderline{文王}者,言有詳略,無別意也。“能居之於心”,謂心知其賢也。“能用陳之”,謂陳列於位,用之以為官也。王肅曰:“則能居之在位,能用陳其才力,如此故能使天下治也。” \par}

{\noindent\shu\zihao{5}\fzkt “嗚呼”至“俾乂”。正義曰:“旦”者,\CJKunderline{周公}名也。\CJKunderline{周公}又嘆曰:“嗚呼!我旦已受賢聖人說\CJKunderline{禹}湯之美言,皆以告孺子王矣,王宜依行之。繼續從今以往,\CJKunderline{文王}之子孫,其勿得過誤於眾獄訟眾所慎之事,惟當用是正是之道治之。用古商人\CJKunderline{成湯},亦於我周家\CJKunderline{文王}其立政、立事、牧夫、準人此等諸官,皆用賢人之法,則能居之於心,能用陳之於位,明識賢人,用之為官,此乃使天下大治。”戒\CJKunderline{成王}使法之。 \par}

國則罔有立政用憸人,不訓於德,是罔顯在厥世。\footnote{商周賢聖之國,則無有立政用憸利之人者。憸人不訓於德,是使其君無顯名在其世。○憸,息廉反,徐七漸反,本又作 ,馬云:“憸利,佞人也。”}繼自今立政,其勿以憸人,其惟吉士,用勱相我國家。\footnote{立政之臣,惟其吉士,用勉治我國家。○勱音邁。}

{\noindent\shu\zihao{5}\fzkt “國則”至“國家”。正義曰:既言湯與\CJKunderline{文王}用賢大治,又言其不宜用小人。商周聖賢之國,無有立政用憸利之人者。此憸利之人不順於德,若其用之,是使其君無顯名在其世也。王當繼續從今已往立其善政,其勿用憸利之人,其惟任用善士,使勉力治我國家。教王使用善士,勿使小人也。 \par}

今文子文孫,孺子王矣。\footnote{告\CJKunderline{文王}之子孫,言稚子已即政為王矣,所以厚戒。}其勿誤於庶獄,惟有司之牧夫。\footnote{獨言眾獄、有司,欲其重刑,慎官人。}其克詰爾戎兵,以陟\CJKunderline{禹}之跡,\footnote{其當能治汝戎服兵器,威懷並設,以升\CJKunderline{禹}治水之舊跡。○詰,起一反,馬云:“賓也。”}方行天下,至於海表,罔有不服,\footnote{方,四方。海表,蠻夷戎狄,無不服化者。}以覲\CJKunderline{文王}之耿光,以揚\CJKunderline{武王}之大烈。\footnote{能使四夷賓服,所以見祖之光明,揚父之大業。}嗚呼!繼自今,后王立政,其惟克用常人。”\footnote{其惟能用賢才為常人,不可以天官有所私。}


{\noindent\zhuan\zihao{6}\fzbyks 傳“獨言”至“官人”。正義曰:上有“庶慎”、“立政”、“立事”、“牧夫”、“準人”,此獨言“庶獄”與“有司之牧夫”者,言“庶獄”欲其重刑,言“有司牧夫”欲其慎官人也。 \par}

{\noindent\zhuan\zihao{6}\fzbyks 傳“其當”至“舊跡”。正義曰:立官所以牧養下民,戒備不虞,故以“詰爾戎兵”為言也。“戎”亦“兵”也,以其並言“戎兵”,故傳以為“戎服兵器,威懷並設,以升\CJKunderline{禹}治水之舊跡”。遠行必登山,故以“陟”言之。如舜之“陟方”,意亦然。 \par}

{\noindent\zhuan\zihao{6}\fzbyks 傳“方四”至“化者”。正義曰:“方行天下”,言無所不至,故以“方”為四方。\CJKunderwave{釋地}云:“九夷、八狄、七戎、六蠻謂之四海。”知“海表”謂“夷狄戎蠻,無有不服化者”。即\CJKunderwave{詩·小雅}雲“\CJKunderwave{蓼蕭},澤及四海”是也。 \par}

{\noindent\zhuan\zihao{6}\fzbyks 傳“其惟”至“所私”。正義曰:官須常得賢人,故惟賢是用。用賢是常,常則非賢不可。人主或知其不賢,以私受用之,代天為官,故言“不可以天官有所私”。 \par}

{\noindent\shu\zihao{5}\fzkt “今文”至“常人”。正義曰:今告汝\CJKunderline{文王}之子,\CJKunderline{文王}之孫,孺子今已即政為王矣,我所以須厚戒之。王其勿誤於眾治獄之官,當須慎刑也。惟有司之牧夫,有司主養民者,宜得賢也。治獄之吏,養民之官,若任得其人,使其能治汝戎服兵器,以此升行\CJKunderline{禹}之舊跡,四方而行,至於天下,至於四海之表,無有不服王之化者,以顯見\CJKunderline{文王}之光明,以播揚\CJKunderline{武王}之大業。言任得賢臣,則光揚父祖。\CJKunderline{周公}又嘆曰:“嗚呼!繼續從今已往,後世之王,立行善政,其惟能用常人,必使常得賢人,不可任非其才,此雖有戒\CJKunderline{成王},乃是國之常法,因以戒后王,言此法可常行也。 \par}

\CJKunderline{周公}若曰:“太史,\footnote{順其事並告太史。}司寇蘇公,式敬爾由獄,以長我王國。\footnote{忿生為\CJKunderline{武王}司寇,封蘇國,能用法。敬汝所用之獄,以長施行於我王國。言主獄當求蘇公之比。○比,必二反,又如字。}茲式有慎,以列用中罰。”\footnote{此法有所慎行,必以其列用中罰,不輕不重,蘇公所行。太史掌六典,有廢置官人之制,故告之。○行如字。}


{\noindent\zhuan\zihao{6}\fzbyks 傳“忿生”至“之比”。正義曰:成十一年\CJKunderwave{左傳}云:“昔周克商,使諸侯撫封,蘇忿生以溫為司寇。”是“忿生為\CJKunderline{武王}司寇,封蘇國”也。“蘇”是國名,所都之地其邑名“溫”,故傳言“以溫”也。特舉蘇公治獄官以告太史,知其言主獄之官當求蘇公之比類也。 \par}

{\noindent\zhuan\zihao{6}\fzbyks 傳“此法”至“告之”。正義曰:治獄必有定法,此定法有所慎行。\CJKunderwave{周禮·大司寇}云:“刑新國用輕典,刑平國用中典,刑亂國用重典。”輕重各有體式行列,\CJKunderline{周公}言然之時,是法為平國,故必以其列用中罰,使不輕不重。美蘇公治獄,使列用中罰,明中罰不輕不重,是蘇公所行也。\CJKunderwave{周禮}太宰“以八柄詔王,馭群臣”,有爵、祿、廢、置、生、殺、與、奪之法。太史亦掌邦之六典,以副貳太宰,是太史有廢置官人之制,故特呼而告之也。 \par}

{\noindent\shu\zihao{5}\fzkt “\CJKunderline{周公}”至“中罰”。正義曰:\CJKunderline{周公}順其事而言曰:“太史。”以其太史掌廢置官人,故呼而告之:“昔日司寇蘇公既能用法,汝太史當敬汝所用之獄,以長施行於我王國。”欲使太史選主獄之官,當求蘇公之比也。“此刑獄之法,有所慎行,必以其體式,列用中常之罰,不輕不重,當如蘇公所行也”。 \par}

%%% Local Variables:
%%% mode: latex
%%% TeX-engine: xetex
%%% TeX-master: "../Main"
%%% End:
