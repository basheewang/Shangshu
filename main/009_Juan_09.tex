%% -*- coding: utf-8 -*-
%% Time-stamp: <Chen Wang: 2024-04-02 11:42:41>

% {\noindent \zhu \zihao{5} \fzbyks } -> 注 (△ ○)
% {\noindent \shu \zihao{5} \fzkt } -> 疏

\chapter{卷九}


\section{盤庚上第九(盤庚)}


\CJKunderline{盤庚}五遷,將治亳殷,\footnote{自湯至\CJKunderline{盤庚}凡五遷都,\CJKunderline{盤庚}治亳殷。盤,本又作般,步幹反。治,直吏反。}民諮胥怨,\footnote{胥,相也。民不欲徙,乃諮嗟憂愁,相與怨上。胥,徐思餘反。怨,紆萬反。}作\CJKunderwave{盤庚}三篇。

{\noindent\zhuan\zihao{6}\fzbyks 傳“自湯”至“亳怨”。正義曰:經言“不常厥邑,于今五邦”,故序言“\CJKunderline{盤庚}五遷”。傳嫌一身五遷,故辨之云“自湯至\CJKunderline{盤庚}凡五遷都”也。上文言“自契至于\CJKunderline{成湯}八遷”,並數湯為八,此言“\CJKunderline{盤庚}五遷”,又並數湯為五,湯一人再數,故班固云:“殷人屢遷,前八後五,其實正十二也。”此序雲\CJKunderline{盤庚}“將治亳殷”,下傳云“殷,亳之別名”,則“亳殷”即是一都,湯遷還從先王居也。汲冢古文云:“\CJKunderline{盤庚}自奄遷于殷,殷在鄴南三十里。”束晳云:“\CJKunderwave{尚書序}‘\CJKunderline{盤庚}五遷,將治亳殷’,舊說以為居亳,亳殷在河南。\CJKunderline{孔子}壁中\CJKunderwave{尚書}雲‘將始宅殷’,是與古文不同也。\CJKunderwave{漢書·項羽傳}雲‘洹水南殷墟上’,今安陽西有殷。”束晳以殷在河北,與亳異也。然\CJKunderline{孔子}壁內之書,安國先得其本,此“將治亳殷”不可作“將始宅殷”。“亳”字摩滅,容或為“宅”。壁內之書,安國先得,“始”皆作“亂”,其字與“始”不類,無緣誤作“始”字,知束晳不見壁內之書,妄為說耳。若洹水南有殷墟,或當餘王居之,非\CJKunderline{盤庚}也。\CJKunderline{盤庚}治于亳殷,紂滅在于朝歌,則\CJKunderline{盤庚}以後遷于河北,蓋\CJKunderline{盤庚}后王有從河有亳地遷于洹水之南,後又遷于朝歌。 \par}

{\noindent\zhuan\zihao{6}\fzbyks 傳“胥相”至“怨上”。正義曰:\CJKunderwave{釋詁}云:“胥,皆也。”“相”亦是皆義,故通訓“胥”為相也。民不欲徙,乃諮嗟憂愁,相與怨上,經云“民不適有居”,是怨上之事也。\CJKunderline{仲丁}、\CJKunderline{祖乙}亦是遷都,序無民怨之言,此獨有怨者,\CJKunderline{盤庚},\CJKunderline{祖乙}之曾孫也,\CJKunderline{祖乙}遷都于此,至今多歷年世,民居已久,戀舊情深;前王三徙,誥令則行,曉喻之易,故無此言;此則民怨之深,故序獨有此事。彼各一篇,而此獨三篇者,謂民怨上,故勸誘之難也。民不欲遷,而\CJKunderline{盤庚}必遷者,\CJKunderline{鄭玄}云:“\CJKunderline{祖乙}居耿後,奢侈逾禮,土地迫近山川,嘗圮焉。至陽甲立,\CJKunderline{盤庚}為之臣,乃謀徙居湯舊都。”又序注云:“民居耿久,奢淫成俗,故不樂徙。”\CJKunderline{王肅}云:“自\CJKunderline{祖乙}五世至\CJKunderline{盤庚},元兄陽甲,宮室奢侈,下民邑居墊隘,水泉瀉滷,不可以行政化,故徙都于殷。”皇甫謐云:“耿在河北,迫近山川,自祖辛已來,民皆奢侈,故\CJKunderline{盤庚}遷于殷。”此三者之說皆言奢侈,\CJKunderline{鄭玄}既言君奢,又言民奢,\CJKunderline{王肅}專謂君奢,皇甫謐專謂民奢。言君奢者以天子宮室奢侈,侵奪下民;言民奢者以豪民室宇過度,逼迫貧乏;皆為細民弱劣無所容居,欲遷都改制以寬之。富民戀舊,故違上意,不欲遷也。案檢孔傳無奢侈之語,惟下篇云“今我民用蕩析離居,罔有定極”,傳云:“水泉沈溺,故蕩析離居,無安定之極,徙以為之極。”孔意蓋以地勢洿下,又久居水變,水泉瀉滷,不可行化,故欲遷都,不必為奢侈也。此以君名名篇,必是為君時事,而\CJKunderline{鄭玄}以為上篇是\CJKunderline{盤庚}為臣時事,何得專輒謬妄也! \par}

{\noindent\shu\zihao{5}\fzkt “\CJKunderline{盤庚}”至“三篇”。正義曰:商自\CJKunderline{成湯}以來屢遷都邑,\CJKunderline{仲丁}、\CJKunderline{河亶甲}、\CJKunderline{祖乙}皆有言誥,歷載于篇。\CJKunderline{盤庚}最在其後,故序總之,“自湯至\CJKunderline{盤庚}凡五遷都”。今\CJKunderline{盤庚}將欲遷居,而治于亳之殷治,民皆戀其故居,不欲移徙,諮嗟憂愁,相與怨上,\CJKunderline{盤庚}以言辭誥之。史敘其事,作\CJKunderwave{盤庚}三篇。 \par}

盤庚\footnote{\CJKunderline{盤庚},殷王名。殷質,以名篇。\CJKunderline{盤庚},馬云:“\CJKunderline{祖乙}曾孫,祖丁之子。不言‘\CJKunderline{盤庚}誥’何?非但錄其誥也,取其徙而立功,故以“\CJKunderline{盤庚}”名篇。}


{\noindent\zhuan\zihao{6}\fzbyks 傳“殷質,以名篇”。正義曰:\CJKunderwave{周書}諡法\CJKunderline{成王}時作,故桓六年\CJKunderwave{左傳}云:“周人以諱事神。”殷時質,未諱君名,故以王名名篇也。上\CJKunderwave{仲丁}、\CJKunderwave{祖乙}亦是王名,于此始作傳者,以上篇經亡,此經稱\CJKunderwave{盤庚},故就此解之。\CJKunderwave{史記·殷本紀}云:“\CJKunderline{盤庚}崩,弟小辛立。殷復衰,百姓思\CJKunderline{盤庚},乃作\CJKunderwave{盤庚}三篇。”與此序違,非也。\CJKunderline{鄭玄}云:“\CJKunderline{盤庚},湯十世孫,\CJKunderline{祖乙}之曾孫,以五遷繼湯,篇次\CJKunderwave{祖乙},故繼之。于上累之,\CJKunderline{祖乙}為湯玄孫,七世也,又加\CJKunderline{祖乙},復其祖父,通\CJKunderline{盤庚},故十世。”\CJKunderwave{本紀}云:“\CJKunderline{祖乙}崩,子祖辛立。崩,子開甲立。崩,弟祖丁立。崩,門甲之子南庚立。崩,祖丁子陽甲立。崩,弟\CJKunderline{盤庚}立。”是\CJKunderline{祖乙}生祖辛,祖辛生祖丁,祖丁生\CJKunderline{盤庚},故為曾孫。 \par}

{\noindent\shu\zihao{5}\fzkt “\CJKunderline{盤庚}”。正義曰:此三篇皆以民不樂遷,開解民意,告以不遷之害,遷都之善也。中上二篇,未遷時事,下篇既遷後事。上篇人皆怨上,初啟民心,故其辭尤切。中篇民已少悟,故其辭稍緩。下篇民既從遷,故辭復益緩。哀十一年\CJKunderwave{左傳}引此篇云“\CJKunderline{盤庚}之誥”,則此篇皆誥辭也。題篇不曰“\CJKunderline{盤庚}誥”者,\CJKunderline{王肅}云:“取其徙而立功,故但以‘\CJKunderline{盤庚}’名篇。”然\CJKunderwave{仲丁}、\CJKunderwave{祖乙}、\CJKunderwave{河亶甲}等皆以王名篇,則是史意異耳,未必見他義。 \par}

\CJKunderline{盤庚}遷于殷,\footnote{亳之別名。}民不適有居。\footnote{適,之也,不欲之殷有邑居。}率籲眾戚,出矢言,\footnote{籲,和也。率和眾憂之人,出正直之言。籲音喻。戚,千歷反。}曰:“我王來,既爰宅于茲,\footnote{我王\CJKunderline{祖乙}居耿。爰,于也。言\CJKunderline{祖乙}已居于此。}重我民,無盡劉。\footnote{劉,殺也。所以遷此,重我民,無慾盡殺故。盡,子忍反。}



{\noindent\zhuan\zihao{6}\fzbyks 傳“亳之別名”。正義曰:此序先“亳”後“殷”,“亳”是大名,“殷”是亳內之別名。\CJKunderline{鄭玄}云:“商家自徙此而號曰殷。”鄭以此前未有殷名也。中篇云:“殷降大虐。”將遷于殷,先正其號,明知于此號為殷也。雖兼號為殷,而商名不改,或稱商,或稱殷,又有兼稱殷商。\CJKunderwave{商頌}云“商邑翼翼”、“撻彼殷武”是單稱之也。又\CJKunderwave{大雅}云“殷商之旅”、“諮汝殷商”,是兼稱之也。亳是殷地大名,故殷社謂之亳社,其亳\CJKunderline{鄭玄}以為偃師,皇甫謐以為梁國谷熟縣,或雲濟陰亳縣。說既不同,未知誰是。 \par}

{\noindent\zhuan\zihao{6}\fzbyks 傳“適之”至“邑居”。正義曰:\CJKunderwave{釋詁}云:“適、之,往也。”俱訓為往,故“適”得為之,不欲往彼殷地,別有新邑居也。 \par}

{\noindent\zhuan\zihao{6}\fzbyks 傳“籲和”至“之言”。正義曰:“籲”即裕也,是寬裕,故為和也。憂則不和,“戚”訓憂也,故“率和眾憂之人,出正直之言”。\CJKunderwave{詩}云“其直如矢”,故以“矢言”為“正直之言”。 \par}

{\noindent\zhuan\zihao{6}\fzbyks 傳“我王”至“于此”。正義曰:孔以\CJKunderline{祖乙}圮于相地,遷都于耿,今\CJKunderline{盤庚}自耿遷于殷,以“我王”為\CJKunderline{祖乙},此謂耿也。 \par}

{\noindent\zhuan\zihao{6}\fzbyks 傳“劉殺”至“殺故”。正義曰:“劉,殺”,\CJKunderwave{釋詁}文。水泉咸鹵,不可行化,王化不行,殺民之道。先王所以去彼遷此者,重我民,無慾盡殺故也。 \par}

不能胥匡以生,卜稽曰:‘其如台?’\footnote{言民不能相匡以生,則當卜稽于龜以徙,曰:“其如我所行。”稽,工兮反。臺音怡。}先王有服,恪謹天命,茲猶不常寧,\footnote{先王有所服行,敬謹天命,如此尚不常安,有可遷輒遷。恪,苦各反。}不常厥邑,于今五邦。\footnote{湯遷亳,\CJKunderline{仲丁}遷囂,\CJKunderline{河亶甲}居相,\CJKunderline{祖乙}居耿,我往居亳,凡五徙國都。馬云:“五邦謂商丘、亳、囂、相、耿也。”}今不承于古,罔知天之斷命,\footnote{今不承古而徙,是無知天將斷絕汝命。斷又音短。}矧曰其克從先王之烈?\footnote{天將絕命,尚無知之,況能從先王之業乎?從,才容反。}若顛木之有由櫱,\footnote{言今往遷都,更求昌盛,如顛仆之木,有用生櫱哉。櫱,五達反,本又作枿,馬云:“顛木而肄生曰枿。”僕音赴,又步北反。}天其永我命于茲新邑,\footnote{言天其長我命于此新邑,不可不徙。}紹復先王之大業,厎綏四方。”\footnote{言我徙欲如此。厎之履反。}

{\noindent\zhuan\zihao{6}\fzbyks 傳“言民”至“所行”。正義曰:不徙所以不能相匡以生者,謂水泉沉溺,人民困苦,不能以義相匡正以生。又考卜于龜以徙,\CJKunderwave{周禮·大卜}:“大遷考貞龜。”是遷必卜也。 \par}

{\noindent\zhuan\zihao{6}\fzbyks 傳“先王”至“輒遷”。正義曰:下云“于今五邦”,自湯以來數之,則此言“先王”總謂\CJKunderline{成湯}至\CJKunderline{祖乙}也。“先王有所服行”,謂行有典法,言能敬順天命,即是“有所服行”也。\CJKunderline{盤庚}言先王敬順天命,如此尚不常安,有可遷輒遷;況我不能敬順天命,不遷民必死矣,故不可不遷也。 \par}

{\noindent\zhuan\zihao{6}\fzbyks 傳“湯遷”至“國都”。正義曰:孔以\CJKunderline{盤庚}意在必遷,故通數“我往居亳”為“五邦”。鄭、王皆云,湯自商徙亳,數商、亳、囂、相、耿為五。計湯既遷亳,始建王業,此言先王遷都,不得遠數居亳之前充此數也。 \par}

{\noindent\zhuan\zihao{6}\fzbyks 傳“言今”至“櫱哉”。正義曰:\CJKunderwave{釋詁}云:“枿,餘也。”李巡曰:“枿,槁木之餘也。”郭璞云:“晉衛之間曰枿。”是言木死顛仆,其根更生櫱哉。此都毀壞,若枯死之木,若棄去毀壞之邑,更得昌盛,猶顛仆枯死之木用生櫱哉。 \par}

{\noindent\shu\zihao{5}\fzkt “\CJKunderline{盤庚}”至“四方”。正義曰:\CJKunderline{盤庚}欲遷于亳之殷地,其民不欲適彼殷地別有邑居,莫不憂愁,相與怨上。\CJKunderline{盤庚}率領和諧其眾憂之人,出正直之言以曉告曰:“我先王初居此者,從舊都來,于是宅于此地。所以遷于此者,為重我民,無慾盡殺故。先王以久居墊隘,不遷則死,見下民不能相匡正以生,故謀而來徙。以徙為善,未敢專決,又考卜于龜以徙。既獲吉兆,乃曰:‘其如我所行欲徙之吉。’先王\CJKunderline{成湯}以來,凡有所服行,敬順天命,如此尚不常安,可徙則徙,不常其邑,于今五邦矣。今若不承于古,徙以避害,則是無知天將斷絕汝命矣。天將絕命,尚不能知,況曰其能從先王之基業乎?今我往遷都,更求昌盛,若顛仆之木,有用生櫱哉。人衰更求盛,猶木死生櫱哉。我今遷向新都,上天其必長我殷之王命于此新邑,繼復先王之大業,致行其道,以安四方之人。我徙欲如此耳,汝等何以不原徙乎?”前雲若不徙以避害,則天將絕汝命,謂絕臣民之命,明亦絕我殷王之命。復雲若遷往新都,天其長我殷之王命,明亦長臣民之命,互文也。 \par}

\CJKunderline{盤庚}\xpinyin*{斆}于民,由乃在位,以常舊服,正法度。\footnote{斆,教也。教人使用汝在位之命,用常故事,正其法度。斆,戶教反,下如字。度如字。}曰:“無或敢伏小人之攸箴。”\footnote{言無有故伏絕小人之所欲箴規上者。戒朝臣。箴,之林反,馬云:“諫也。”朝,直遙反。}


{\noindent\zhuan\zihao{6}\fzbyks 傳“斆教”至“朝臣”。正義曰:\CJKunderwave{文王世子}云:“小樂正斆幹,大胥贊之。籥師斆戈,籥師丞贊之。”彼並是教舞干戈,知“斆”為教也。小民等患水泉沉溺,欲箴規上而徙,汝臣下勿抑塞伏絕之。\CJKunderline{鄭玄}云:“奢侈之俗,小民咸苦之,欲言于王。今將屬民而詢焉,故敕以無伏之。” \par}

{\noindent\shu\zihao{5}\fzkt “\CJKunderline{盤庚}”至“攸箴”。正義曰:前既略言遷意,今復並戒臣民。\CJKunderline{盤庚}先教于民云:“汝等當用汝在位之命,用舊常故事,正其法度。”欲令民徙,從其臣言也。民從上命,即是常事法度也。又戒臣曰:“汝等無有敢伏絕小人之所欲箴規上者。” \par}

王命眾悉至于庭。\footnote{眾,群臣以下。}

{\noindent\zhuan\zihao{6}\fzbyks 傳“眾,群臣以下”。正義曰:\CJKunderwave{周禮}:“小司寇掌外朝之政,以致萬民而詢焉,一曰詢國危,二曰詢國遷,三曰詢立君。”是國將大遷,必詢及于萬民。故知眾悉至王庭是“群臣以下”,謂及下民也。民不欲徙,由臣不助王勸民,故以下多是責臣之辭。 \par}

王若曰:“格汝眾,予告汝,\footnote{告汝以法教。}訓汝猷,黜乃心,無傲從康。\footnote{謀退汝違上之心,無傲慢,從心所安。傲,五報反。}古我先王,亦惟圖任舊人共政。\footnote{先王謀任久老成人共治其政。任,而鴆反。}

{\noindent\zhuan\zihao{6}\fzbyks 傳“先王”。正義曰:此篇所言“先王”,其文無指斥者,皆謂\CJKunderline{成湯}已來諸賢王也。下言“神後”、“高后”者,指謂湯耳。下篇言“古我先王,適于山”者,乃謂遷都之王\CJKunderline{仲丁}、\CJKunderline{祖乙}之等也。此言“先王”謂先世賢王。此既言“先王”,下句“王播告之”、“王用丕欽”蒙上之“先”,不言“先”,省文也。 \par}

王播告之,修不匿厥指,\footnote{王佈告人以所修之政,不匿其指。播,波餓反。匿,女力反。}

{\noindent\zhuan\zihao{6}\fzbyks 傳“王布”至“其指”。正義曰:上句言先王用舊人共政,下云“王播告之修”,當謂告臣耳。傳言“佈告人”者,以下云“民用丕變”,是必告臣,亦又告民。 \par}

王用丕欽,罔有逸言,民用丕變。\footnote{王用大敬其政教,無有逸豫之言,民用大變從化。}今汝聒聒,起信險膚,予弗知乃所訟。\footnote{聒聒,無知之貌。起信險偽膚受之言,我不知汝所訟言何謂。聒,古活反,馬雲\CJKunderwave{說文}皆云:“拒善自用之意。”}

{\noindent\zhuan\zihao{6}\fzbyks 傳“聒聒”至“何謂”。正義曰:\CJKunderline{鄭玄}云:“聒讀如‘聒耳’之聒,聒聒,難告之貌。”\CJKunderline{王肅}云:“聒聒,善自用之意也。”此傳以“聒聒”為“無知之貌”,以“聒聒”是多言亂人之意也。“起信險膚”者,言發起所行,專信此險偽膚受淺近之言。信此浮言,妄有爭訟,我不知汝所訟言何謂。言無理也。 \par}

非予自荒茲德,惟汝含德,不惕予一人。予若觀火。\footnote{我之慾徙,非廢此德。汝不從我命,所含惡德,但不畏懼我耳。我視汝情如視火。惕,他歷反。}

{\noindent\shu\zihao{5}\fzkt “非予”至“觀火”。正義曰:言先王敬其教,民用大變。我命教汝,汝不肯徙。非我自廢此丕欽之德,惟汝之所含德甚惡,不畏懼我一人故耳。汝含藏此意,謂我不知。我見汝情若觀火。言見之分明如見火也。 \par}

予亦拙,謀作乃逸。\footnote{逸,過也。我不威脅汝徙,是我拙謀成汝過。拙,之劣反。}

{\noindent\zhuan\zihao{6}\fzbyks 傳“逸過”至“汝過”。正義曰:“逸,過”,\CJKunderwave{釋言}文。我若以威加汝,汝自不敢不遷,則無違上之過也。我不威脅汝徙,乃是我亦拙謀,作成汝過也。恨民以恩導之而不從己也。 \par}

若網在綱,有條而不紊。若農服田力穡,乃亦有秋。\footnote{紊,亂也。穡,耕稼也。下之順上,當如網在綱,各有條理而不亂也。農勤穡則有秋,下承上則有福。紊音問,徐音文。}

{\noindent\zhuan\zihao{6}\fzbyks 傳“紊亂”至“有福”。正義曰:“紊”是絲亂,故為亂也。“稼”、“穡”相對,則種之曰“稼”,斂之曰“穡”。“穡”是秋收之名,得為耕穫總稱,故云“穡,耕稼”。“下承上則有福”,“福”謂祿賞。 \par}

汝克黜乃心,施實德于民,至于婚友,丕乃敢大言,汝有積德。\footnote{汝群臣能退去傲上之心,施實德于民,至于婚姻僚友,則我大乃敢言汝有積德之臣。}乃不畏戎毒于遠邇,惰農自安,不昬作勞,不服田畝,越其罔有黍稷。\footnote{戎,大。昬,強。越,于也。言不欲徙,則是不畏大毒于遠近。如怠惰之農,苟自安逸,不強作勞于田畝,則黍稷無所有。昬,馬同;本或作暋,音敏。\CJKunderwave{爾雅}昬、暋皆訓強,故兩存。越,本又作粵,音曰,于也。強,其丈反。}

{\noindent\zhuan\zihao{6}\fzbyks 傳“戎大”至“所有”。正義曰:“戎,大”、“昬,強”、“越,于”皆\CJKunderwave{釋詁}文。孫炎曰:“昬,夙夜之強也。\CJKunderwave{書}曰:‘不昬作勞。’”引此解彼,是亦讀此為昬也。\CJKunderline{鄭玄}讀“昬”為暋,訓為勉也,與孔不同。傳云“言不欲徙,則是不畏大毒于遠近”,其意言不徙則有毒,“毒”為禍患也;“遠近”謂賒促,言害至有早晚也。不強于作勞,則黍稷無所獲,以喻不遷于新邑,則福祿無所有也。此經惰農弗昬無黍稷,對上“服田力穡,乃亦有秋”,但其文有詳略耳。 \par}

“汝不和吉言于百姓,惟汝自生毒,\footnote{責公卿不能和喻百官,是自生毒害。}

{\noindent\zhuan\zihao{6}\fzbyks 傳“責公”至“毒害”。正義曰:此篇上下皆言“民”,此獨云“百姓”,則知百姓是百官也。百姓既是百官,和吉言者又在百官之上,知此經是責公卿不能和喻善言于百官,使之樂遷也。不和百官,必將遇禍,是公卿自生毒害。 \par}

乃敗禍姦宄,以自災于厥身。\footnote{言汝不相率共徙,是為敗禍姦宄以自災之道。宄音軌。}乃既先惡于民,乃奉其恫,汝悔身何及?\footnote{群臣不欲徙,是先惡于民。恫,痛也。不徙則禍毒在汝身,徙奉持所痛而悔之,則于身無所及。奉,孚勇反,注同。恫,敕動反,又音通,痛也。}

{\noindent\zhuan\zihao{6}\fzbyks 傳“群臣”至“所及”。正義曰:群臣是民之師長,當倡民為善,群臣亦不欲徙,是乃先惡于民也。“恫,痛”,\CJKunderwave{釋言}文。 \par}

相時\xpinyin*{憸}民,猶胥顧于箴言,其發有逸口,矧予制乃短長之命?\footnote{言憸利小民,尚相顧于箴誨,恐其發動有過口之患,況我制汝死生之命,而汝不相教從我,是不若小民。相時,相,息亮反,馬云:“視也。”徐息羊反。憸,息廉反,馬云:“憸利,小小見事之人也。”徐七漸反。}汝曷弗告朕,而胥動以浮言,恐沈于眾?\footnote{曷,何也。責其不以情告上,而相恐欲以浮言,不徙,恐汝沉溺于眾,有禍害。曷,何末反。}若火之燎于原,不可鄉邇,其猶可撲滅?\footnote{火炎不可向近,尚可撲滅。浮言不可信用,尚可得遏絕之。燎,力召反,又力鳥反,又力紹反。向,許亮反。撲,普卜反。近,附近之近。}則惟汝眾自作弗靖,非予有咎。\footnote{我刑戮汝,非我咎也。靖,謀也。是汝自為非謀所致。}

{\noindent\zhuan\zihao{6}\fzbyks 傳“曷何”至“禍害”。正義曰:“曷”、“何”同音,故“曷”為何也。顧氏云:“汝以浮雲恐動不徙,更是無益。我恐汝自取沉溺于眾人,不免禍害也。” \par}

{\noindent\zhuan\zihao{6}\fzbyks 傳“我刑”至“所致”。正義曰:我刑戮汝,汝自招之,非我咎也。“靖,謀”,\CJKunderwave{釋詁}文。告民不徙者,非善謀也。由此而被刑戮,是汝自為非謀所致也。 \par}

{\noindent\shu\zihao{5}\fzkt “相時”至“有咎”。正義曰:又責大臣不相教遷徙,是不如小民。我視彼憸利小民,猶尚相顧于箴規之言,恐其發舉有過口之患,故以言相規。患之小者尚知畏避,況我為天子制汝短長之命?威恩甚大,汝不相教從我,乃是汝不如小民。汝若不欲徙,何不以情告我,而輒相恐動以浮華之言?乃語民云:“國不可徙,我恐汝自取沉溺于眾人,而身被刑戮之禍害。”此浮言流行,若似火之燎于原野,炎熾不可向近,其猶可撲之使滅,以喻浮言不可止息,尚可刑戮使絕也。若以刑戮加汝,則是汝眾自為非謀所致此耳,非我有咎過也。 \par}

“\CJKunderline{遟任}有言曰:‘人惟求舊,器非求舊,惟新。’\footnote{\CJKunderline{遟任},古賢。言人貴舊,器貴新,汝不徙,是不貴舊。遟,直疑反,徐持夷反。任,而今反,馬云:“古老成人。”}古我先王,暨乃祖乃父,胥及逸勤,予敢動用非罰?\footnote{言古之君臣相與同勞逸,子孫所宜法之,我豈敢動用非常之罰脅汝乎?}世選爾勞,予不掩爾善。\footnote{選,數也。言我世世數汝功勤,不掩蔽汝善,是我忠于汝。選,息轉反,又蘇管反。掩,本又作弇。數,色主反。}茲予大享于先王,爾祖其從與享之。\footnote{古者天子錄功臣配食于廟。大享,烝嘗也。所以不掩汝善。與音預。烝,之丞反。}作福作災,予亦不敢動用非德。\footnote{善自作福,惡自作災,我不敢動用非罰加汝,非德賞汝乎?從汝善惡而報之。}

{\noindent\zhuan\zihao{6}\fzbyks 傳“\CJKunderline{遟任}”至“貴舊”。正義曰:其人既沒,其言立于後世,知是古賢人也。\CJKunderline{鄭玄}云:“古之賢史。”\CJKunderline{王肅}云:“古老成人。”皆謂賢也。 \par}

{\noindent\zhuan\zihao{6}\fzbyks 傳“選數”至“于汝”。正義曰:\CJKunderwave{釋詁}云:“算,數也。”舍人曰:“釋數之曰算。”“選”即算也,故訓為數。經言世世數汝功勞,是從先王至己常行此事,故云“是我忠于汝”也。言己之忠,責臣之不忠也。 \par}

{\noindent\zhuan\zihao{6}\fzbyks 傳“古者”至“汝善”。正義曰:\CJKunderwave{周禮·大宗伯}祭祀之名,天神曰“祀”,地祇曰“祭”,人鬼曰“享”。此“大享于先王”,謂天子祭宗廟也。傳解天子祭廟,得有臣祖與享之意,言“古者天子錄功臣配食于廟”,故臣之先祖得與享之也。“古者”\CJKunderline{孔氏}據已而道前世也,此殷時已然矣。“大享,烝嘗”者,烝嘗是秋冬祭名,謂之“大享”者,以事各有對。若烝嘗對禘祫,則禘祫為大,烝嘗為小。若四時自相對,則烝嘗為大,礿祠為小。以秋冬物成,可薦者眾,故烝學為大;春夏物未成,可薦者少,故禘祫為小也。知烝嘗有功臣與祭者,案\CJKunderwave{周禮·司勳}云“凡有功者,銘書于王之太常,祭于大烝,司勳詔之”是也。嘗是烝之類,而傳以嘗配之,\CJKunderwave{魯頌}曰“秋而載嘗”是也。\CJKunderwave{祭統}云“內祭則大嘗禘是也,外祭則郊社是也”。然彼以祫為大嘗,知此不以烝嘗時為禘祫,而直據時祭者,以殷祫于三時,非獨烝嘗也。秋冬之祭,尚及功臣,則禘祫可知。惟春夏不可耳,以物末成故也。近代已來,惟禘祫乃祭功臣配食,時祭不及之也。近代已來,功臣配食各配所事之君,若所事之君其廟已毀,時祭不祭毀廟,其君尚不時祭,其臣固當止矣。禘祫則毀廟之主亦在焉,其時功臣亦當在也。\CJKunderwave{王制}云:“犆礿,祫禘,祫嘗,祫烝,諸侯礿犆,禘,一犆一祫,嘗祫,烝祫。”此\CJKunderwave{王制}之文,夏殷之制,天子春惟時祭,其夏秋冬既為祫,又為時祭。諸侯亦春為時祭,夏惟作祫,不作祭,秋冬先作時祭,而後祫。周則春曰祠,夏曰礿,三年一祫在秋,五年一禘在夏,故\CJKunderwave{公羊傳}云:“五年再殷祭。”\CJKunderwave{禮緯}云:“三年一祫,五年一禘。”此是鄭氏之義,未知孔意如何。 \par}

{\noindent\shu\zihao{5}\fzkt “\CJKunderline{遟任}”至“非德”。正義曰:可遷則遷,是先王舊法。古之賢人\CJKunderline{遟任}有言曰:“人惟求舊,器非求舊,惟新。”言人貴舊,器貴新,汝不欲徙,是不貴舊,反\CJKunderline{遟任}也。古者我之先王及汝祖汝父相與同逸豫,同勤勞,汝為人子孫,宜法汝父祖,當與我同其勞逸。我豈敢動用非常之罰脅汝乎?自我先王以至于我,世世數汝功勞,我不掩蔽汝善,是我忠于汝也。以此故我大享祭于先王,汝祖其從我先王與在宗廟而歆享之,是我不掩汝善也。汝有善自作福,汝有惡自作災,我亦不敢動用非德之賞妄賞汝,各從汝善惡而報之耳。其意告臣言從上必有賞,違命必有罰也。 \par}

予告汝于難,若射之有志。\footnote{告汝行事之難,當如射之有所準志,必中所志乃善。射,食夜反。準音準。中,丁仲反。}

{\noindent\zhuan\zihao{6}\fzbyks 傳“告汝”至“乃善”。正義曰:此傳惟順經文,不言喻意。\CJKunderline{鄭玄}云:“我告汝,于我心至難矣。夫射者,張弓屬矢而志在所射,必中然後發之。為政之道亦如是也,以己心度之,可施于彼,然後出之。” \par}

{\noindent\shu\zihao{5}\fzkt “予告”至“有志”。正義曰:既言作福作災由人行有善惡,故復教臣行善:“我告汝于行事之難,猶如射之有所準志。志之所主,欲得中也,必中所志,乃為善耳。”以喻人將有行,豫思念之,行得其道為善耳。其意言遷都是善道,當念從我言也。 \par}

汝無侮老成人,無弱孤有幼。\footnote{不用老成人之言,是侮老之。不徙則孤幼受害,是弱易之。侮,亡甫反。易,以豉反。}

{\noindent\zhuan\zihao{6}\fzbyks 傳“不用”至“易之”。正義曰:“老”謂見其年老,謂其無所復知。“弱”謂見其幼弱,謂其未有所識。鄭云:“老弱皆輕忽之意也。”老成人之言云可徙,不用其言,是侮老之也。不徙則水泉咸鹵,孤幼受害,不念其害,則是卑弱輕易之也。 \par}

各長于厥居,勉出乃力,聽予一人之作猷。\footnote{\CJKunderline{盤庚}群臣下各思長于其居,勉盡心出力,聽從遷徙之謀。長,丁丈反。}

{\noindent\zhuan\zihao{6}\fzbyks 傳“\CJKunderline{盤庚}”至“之謀”。正義曰:于時群臣難毀其居宅,惟見目前之利,不思長久之計。其臣非一,共為此心。\CJKunderline{盤庚}群臣下各思長久于其居處,勉強盡心出力,聽從我遷徙之謀。自此已下皆是也。 \par}

無有遠邇,用罪伐厥死,用德彰厥善。\footnote{言遠近待之如一,罪以懲之,使勿犯,伐去其死道。德以明之,使勸慕,競為善。去,起呂反。}

{\noindent\shu\zihao{5}\fzkt “無有”至“厥善”。正義曰:此即遷徙之謀也。言我至新都,撫養在下,無有遠之與近,必當待之如一。用刑殺之罪伐去其死道,用照察之德彰明其行善。有過,罪以懲之,使民不犯非法。死刑不用,是“伐去其死道”。“伐”若伐樹然,言止而不復行用也。有善者,人主以照察之德加賞祿以明之,使競慕為善,是彰其善也。此二句相對,上言“用罪伐厥死”,下宜言“用賞彰厥生”,不然者,上言用刑,下言賞善,死是刑之重者,舉重故言“死”;有善乃可賞,故言“彰厥善”;行賞是德,故以“德”言賞;人生是常,無善亦生,不得言“彰厥生”,故文互。 \par}

邦之臧,惟汝眾。\footnote{有善則群臣之功。臧,徐子郎反。}邦之不臧,惟予一人有佚罰。\footnote{佚,失也。是己失政之罰。罪己之義。佚音逸。}凡爾眾,其惟致告:\footnote{致我誠,告汝眾。}自今至于後日,各恭爾事,齊乃位,度乃口。\footnote{奉其職事,正齊其位,以法度居汝口,勿浮言。度,徐如字,亦作渡。}罰及爾身,弗可悔。”\footnote{不從我謀,罰及汝身,雖悔可及乎?}

{\noindent\shu\zihao{5}\fzkt “度乃口”。正義曰:“度”,法度也,故傳言“以法度居汝口”也。 \par}

\section{盤庚中第十(盤庚)}

\CJKunderline{盤庚}作,惟涉河以民遷。\footnote{為此南渡河之法,用民徙。}乃話民之弗率,誕告用亶。\footnote{話,善言。民不循教,發善言大告用誠于眾。話,胡快反,馬云:“告也,言也。”誕,徐音但。亶,丁但反,馬本作單,音同,誠也。}其有眾咸造,勿褻在王庭,\footnote{造,至也。眾皆至王庭,無褻慢。造,七報反,注同;馬在早反,云:“為也。”褻,息列反。}\CJKunderline{盤庚}乃登進厥民。\footnote{升進,命使前。}

{\noindent\zhuan\zihao{6}\fzbyks 傳“為此”至“民徙”。正義曰:\CJKunderline{鄭玄}云“作渡河之具”,\CJKunderline{王肅}云“為此思南渡河之事”,此傳言“南渡河之法”,皆謂造舟舡渡河之具,是濟水先後之次,思其事而為之法也。 \par}

{\noindent\zhuan\zihao{6}\fzbyks 傳“話善”至“于眾”。正義曰:\CJKunderwave{釋詁}云:“話,言也。”孫炎曰:“話,善人之言也。”王苦民不從教,必發善言告之,故以“話”為善言。\CJKunderline{鄭玄}\CJKunderwave{詩箋}亦云:“話,善言也。”曰:“明聽朕言,無荒失朕命。荒,廢。 \par}

{\noindent\shu\zihao{5}\fzkt “\CJKunderline{盤庚}”至“厥民”。正義曰:\CJKunderline{盤庚}于時見都河北,欲遷向河南,作惟南渡河之法,欲用民徙,乃出善言以告曉民之不循教者,大為教告,用誠心于其所有之眾人。于時眾人皆至,無有褻慢之人,盡在于王庭。\CJKunderline{盤庚}乃升進其民,延之使前而教告之。史敘其事,以為\CJKunderline{盤庚}發誥之目。 \par}

曰:“明聽朕言,無荒失朕命。嗚呼!古我先后,罔不惟民之承保。\footnote{言我先世賢君,無不承安民而恤之。}後胥慼鮮,以不浮于天時。\footnote{民亦安君之政,相與憂行君令。浮,行也。少以不行于天時者,言皆行天時。鮮,息淺反。}

{\noindent\zhuan\zihao{6}\fzbyks 傳“民亦”至“天時”。正義曰:以君承安民而憂之,故民亦安君之政,相與憂行君令,使君令必行。責時群臣不憂行君令也。舟舡浮水而行,故以“浮”為行也。行天時也,順時佈政,若\CJKunderwave{月令}之為也。 \par}

殷降大虐,先王不懷厥攸作,\footnote{我殷家于天降大災,則先王不思故居而行徙。}視民利用遷。\footnote{其所為視民有利,則用徙。}汝曷弗念我古后之聞?\footnote{古後先王之聞,謂遷事。曷,何末反,下同。}承汝俾汝,惟喜康共,非汝有咎,比于罰。\footnote{今我法先王惟民之承,故承汝使汝徙,惟與汝共喜安,非謂汝有惡徙汝,令比近于殃罰。俾,必爾反。咎,其九反。比,毗志反,徐扶志反,注及下同。共,群用反。令,力呈反。近,附近之近。}

{\noindent\zhuan\zihao{6}\fzbyks 傳“我殷”至“行徙”。正義曰:遷徙者止為邑居墊隘,水泉咸鹵,非為避天災也。此傳以“虐”為災,“懷”為思,言“殷家于天降大災,則先王不思故居而行徙”者,以天時人事終是相將,邑居不可行化,必將天降之災。上云“不能相匡以生”、“罔知天之斷命”,即是天降災也。 \par}

{\noindent\shu\zihao{5}\fzkt “承汝”至“于罰”。正義曰:先王為政,惟民之承。今我亦法先王,故承安汝使汝徙。惟歡喜安樂皆與汝共之,非謂汝有咎惡而徙汝,令比近于殃罰也。 \par}

予若籲懷茲新邑,亦惟汝故,以丕從厥志。\footnote{言我順和懷此新邑,欲利汝眾,故大從其志而徙之。籲,羊戍反。}

{\noindent\shu\zihao{5}\fzkt “予若”至“厥志”。正義曰:\CJKunderline{盤庚}言,我順于道理,和協汝眾,歸懷此新邑者,非直為我王家,亦惟利汝眾,故為此大從我本志而遷徙,不有疑也。 \par}

“今予將試以汝遷,安定厥邦。\footnote{試,用。}汝不憂朕心之攸困,\footnote{所困,不順上命。}乃咸大不宣乃心,欽念以忱,動予一人。\footnote{汝皆大不布腹心,敬念以誠感我,是汝不盡忠。忱,市林反。}爾惟自鞠自苦,\footnote{鞠,窮也。言汝為臣不忠,自取窮苦。鞠,居六反。}若乘舟,汝弗濟,臭厥載。\footnote{言不徙之害,如舟在水中流不渡,臭敗其所載物。臭,徐尺售反。載如字,又在代反。}

{\noindent\shu\zihao{5}\fzkt “臭厥載”。正義曰:“臭”是氣之別名,古者香氣穢氣皆名為“臭”。\CJKunderwave{易}云“其臭如蘭”,謂香氣為“臭”也。\CJKunderwave{晉語}云“惠公改葬申生,臭徹于外”,謂穢氣為“臭”也。下文覆述此意云“無起穢以自臭”,則此“臭”謂穢氣也。肉敗則臭,故以“臭”為敗。船不渡水,則敗其所載物也。 \par}

爾忱不屬,惟胥以沈。不其或稽,自怒曷\xpinyin*{瘳}?\footnote{汝忠誠不屬逮古,苟不欲徙,相與沉溺,不考之先王,禍至自怒,何瘳差乎?屬音燭,注同,馬云:“獨也。”沈,直林反。瘳,敕留反。}

{\noindent\shu\zihao{5}\fzkt “爾忱”至“曷瘳”。正義曰:\CJKunderline{盤庚}責其臣民,汝等不用徙者,由汝忠誠不能屬逮于古賢。苟不欲徙,惟相與沉溺于眾。不欲徙之言,不其有考驗于先王遷徙之事。汝既不考于古,及其禍至,乃自忿怒,何所瘳差也? \par}

汝不謀長,以思乃災,汝誕勸憂。\footnote{汝不謀長久之計,思汝不徙之災,苟不欲徙,是大勸憂之道。}

{\noindent\shu\zihao{5}\fzkt “汝誕勸憂”。正義曰:凡人以善自勸,則善事多。若以憂自勸,則憂來眾。今不徙則憂來眾,是自勸勵以憂愁之道。 \par}

今其有今罔後,汝何生在上?\footnote{言不徙無後計,汝何得久生在人上,禍將及汝。}

{\noindent\shu\zihao{5}\fzkt “今其”至“在上”。正義曰:顧氏云:“責群臣:汝今日其且有今目前之小利,無後日久長之計,患禍將至,汝何得久生在民上也?” \par}

今予命汝一,無起穢以自臭。\footnote{我一心命汝,汝違我是自臭敗。穢,于廢反。}

{\noindent\shu\zihao{5}\fzkt “今予”至“自臭”。正義曰:今我命汝,是我之一心也。汝當從我,無得起為穢惡,以自臭敗。汝違我命,是起穢以自臭也。 \par}

恐人倚乃身,迂乃心。\footnote{言汝既不欲徙,又為他人所誤。倚,曲。迂,僻。倚,于綺反,徐于奇反。迂音于。僻,匹亦反。}

{\noindent\zhuan\zihao{6}\fzbyks 傳“言汝”至“迂僻”。正義曰:人心不能自決,則好用非理之謀。言汝既不欲遷徙,又為他人所誤。\CJKunderline{盤庚}疑其被誤,故言此也。以物倚物者必曲,故“倚”為曲也。“迂”是回也,回行必僻,故“迂”為僻也。 \par}

{\noindent\shu\zihao{5}\fzkt “恐人”至“乃心”。正義曰:言汝心既不欲徙,旁人或更誤汝。我有恐他人倚曲汝身,迂僻汝心,使汝益不用徙也。 \par}

予迓續乃命于天,予豈汝威?用奉畜汝眾。\footnote{迓,迎也。言我徙,欲迎續汝命于天,豈以威脅汝乎?用奉畜養汝眾。迓,五駕反。畜,許竹反,下同。脅,虛業反。}

{\noindent\zhuan\zihao{6}\fzbyks 傳“迓迎”至“汝眾”。正義曰:“迓,迎”,\CJKunderwave{釋詁}文。不遷必將死矣,天欲遷以延命。天意向汝,我欲迎之。天斷汝命,我欲續之。我今徙者,欲迎續汝命于天,豈以威脅汝乎?遷都惟用奉養汝群臣民耳。 \par}

“予念我先神后之勞爾先,予丕克羞爾,用懷爾然。\footnote{言我亦法湯大能進勞汝,以義懷汝心,而汝違我,是汝反先人。勞,力報反,又如字,注同。}

{\noindent\zhuan\zihao{6}\fzbyks 傳“言我”至“先人”。正義曰:\CJKunderwave{易}稱:“神者,妙萬物而為言也。”殷之先世,神明之君惟有湯耳,故知“神後”謂湯也。下“高后”、“先後”與此“神後”一也。“神”者,言其通聖。“高”者,言其德尊。此“神後”言“先”,于“高后”略而不言“先”,其下直言“先後”,又略而不言“高”,從上省文也。“勞爾先”,謂愛之也。“勞”者,勤也,閔其勤勞而慰勞之,“勞”亦愛之義,故\CJKunderwave{論語}云:“愛之,能勿勞乎?”是“勞”為愛也。此言湯勞汝先,則此所責之臣,其祖于\CJKunderline{成湯}之世已在朝廷。世仕王朝而不用己命,故責之深也。 \par}

{\noindent\shu\zihao{5}\fzkt “予念”至“爾然”。正義曰:我念我先世神後之君\CJKunderline{成湯},愛勞汝之先人,故我大能進用汝,與汝爵位,用以道義懷安汝心耳。然汝乃違我命,是汝反先人也。 \par}

失于政,陳于茲,高后丕乃崇降罪疾,曰:‘曷虐朕民?’\footnote{崇,重也。今既失政,而陳久于此而不徙,湯必大重下罪疾于我,曰:“何為虐我民而不徙乎?”重,直勇反,又直恭反。}汝萬民乃不生生,暨予一人猷同心,\footnote{不進進謀同心徙。}先後丕降與汝罪疾,曰:‘曷不暨朕幼孫有比?’\footnote{言非但罪我,亦將罪汝。幼孫,\CJKunderline{盤庚}自謂。比,同心。}故有爽德,自上其罰汝,汝罔能迪。\footnote{湯有明德在天,見汝情,下罰汝,汝無能道。言無辭。}

{\noindent\zhuan\zihao{6}\fzbyks 傳“崇重”至“徙乎”。正義曰:“崇,重”,\CJKunderwave{釋詁}文。又云:“塵,久也。”孫炎曰:“陳居之久,久則生塵矣。”古者“陳”、“塵”同也,故“陳”為久之義。 \par}

{\noindent\zhuan\zihao{6}\fzbyks 傳“不進”至“心徙”。正義曰:物之生長,則必漸進,故以“生生”為進進。\CJKunderline{王肅}亦然。“進進”是同心原樂之意也。此實責群臣而言“汝萬民”者,民心亦然,因博及之。 \par}

{\noindent\zhuan\zihao{6}\fzbyks 傳“湯有”至“無辭”。正義曰:訓“爽”為明,言其見下,故稱“明德”。\CJKunderwave{詩}稱“三后在天”,死者精神在天,故言下見汝。 \par}

{\noindent\shu\zihao{5}\fzkt “失于”至“能迪”。正義曰:\CJKunderline{盤庚}以民不願遷,言神將罪汝,欲懼之使從己也。我所以必須徙者,我今失于政教,陳久于此,民將有害,高德之君\CJKunderline{成湯}必忿我不徙,大乃重下罪疾于我,曰:“何為殘虐我民而不徙乎?”我既欲徙,而汝與萬民,乃不進進與我一人謀計同心,則我先君\CJKunderline{成湯}大下與汝罪疾,曰:“何故不與我幼孫\CJKunderline{盤庚}有相親比同心徙乎?”汝不與我同心,故湯有明德,從上見汝之情,其下罪罰于汝。汝實有罪,無所能道。言無辭以有解說也。 \par}

古我先后,既勞乃祖乃父,\footnote{勞之共治人。}汝共作我畜民。汝有戕,則在乃心。\footnote{戕,殘也。汝共我治民,有殘人之心而不欲徙,是反父祖之行。戕,在良反,又士良反。行,下孟反。}我先后綏乃祖乃父,乃祖乃父乃斷棄汝,不救乃死。\footnote{言我先王安汝父祖之忠,今汝不忠汝父祖,必斷絕棄汝命,不救汝死。斷,丁緩反。}

{\noindent\zhuan\zihao{6}\fzbyks 傳“勞之共治人”。正義曰:下句責臣之身云“汝共作我畜民”,明先后勞其祖父,是勞之共治民也。 \par}

{\noindent\zhuan\zihao{6}\fzbyks 傳“戕殘”至“之行”。正義曰:\CJKunderwave{春秋}宣十八年“邾人戕鄫子”,\CJKunderwave{左傳}云:“凡自虐其君曰弒,自外曰戕。”“戕”為殘害之義,故為殘也。先后愛勞汝祖汝父,與共治民,汝祖父必有愛人之心。“作”訓為也。汝今共為我養民之官,而有殘民之心,而不用徙以避害,是汝反祖父之行。\CJKunderline{盤庚}距湯,年世多矣,臣父不及湯世而云“父”者,與“祖”連言之耳。 \par}

{\noindent\shu\zihao{5}\fzkt “古我”至“乃死”。正義曰:又責群臣:“古我先君\CJKunderline{成湯},既愛勞汝祖汝父,與之共治民矣。汝今共為我養民之官,是我于汝與先君同也。而汝有殘虐民之心,非我令汝如此,則在汝心自為此惡,是汝反祖父之行。雖汝祖父,亦不祐汝。我先君安汝祖汝父之忠,汝祖汝父忠于先君,必忿汝違我,乃斷絕棄汝命,不救汝死。”言汝違我命,故汝祖父亦忿見湯罪汝,不救汝死也。 \par}

茲予有亂政同位,具乃貝玉。\footnote{亂,治也。此我有治政之臣,同位于父祖,不念盡忠,但念貝玉而已。言其貪。治,直吏反。盡,子忍反。}乃祖先父丕乃告我高后曰:‘作丕刑于朕孫。’\footnote{言汝父祖見汝貪而不忠,必大乃告湯曰:“作大刑于我子孫,求討不忠之罪。”告,工號反。我高后,本又作“乃祖乃父”。}迪高后,丕乃崇降弗祥。\footnote{言汝父祖開道湯,大重下不善以罰汝。陳忠孝之義以督之。}


{\noindent\zhuan\zihao{6}\fzbyks 傳“言汝”至“之罪”。正義曰:上句言\CJKunderline{成湯}罪此諸臣,其祖父不救子孫之死,此句言臣之祖父請\CJKunderline{成湯}討其子孫,以不從已,故責之益深。先祖請討,非\CJKunderline{盤庚}所知,原神之意而為之辭,以懼其子孫耳。 \par}

{\noindent\zhuan\zihao{6}\fzbyks 傳“言汝”至“督之”。正義曰:訓“迪”為道,言汝父祖開道湯也。不從君為不忠,違父祖為不孝,父祖開道湯下罰,欲使從君順祖,陳忠孝之義以督勵之。 \par}

{\noindent\shu\zihao{5}\fzkt “茲予”至“弗祥”。正義曰:又責臣云:“汝祖父非徒不救汝死,乃更請與汝罪。于此我有治政之臣,同位于其父祖。其位與父祖同,心與父祖異。不念忠誠,但念具汝貝玉而已。”言其貪而不忠也。“汝先祖先父以汝如此,大乃告我高后曰:‘為大刑于我子孫。’以此言開道我高后,故我高后大乃下不善之殃以罰汝。\CJKunderline{成湯}與汝祖父皆欲罪汝,汝何以不從我徙乎?”亂治”至“其貪”。正義曰:“亂,治”,\CJKunderwave{釋詁}文。舍人曰:“亂,義之治也。”孫炎曰:“亂,治之理也。”大臣理國之政,此者所責之人,故言于此我有治政大臣。言其同位于父祖,責其位同而心異也。貝者,水蟲。古紉選其甲以為貨,如今之用錢然。\CJKunderwave{漢書·食貨志}具有其事。貝是行用之貨也,貝玉是物之最貴者,責其貪財,故舉二物以言之。當時之臣不念盡忠于君,但念具貝玉而已,言其貪也。 \par}

“嗚呼!今予告汝不易。\footnote{凡所言皆不易之事。易,以豉反,注同。}永敬大恤,無胥絕遠。\footnote{長敬我言,大憂行之,無相與絕遠棄廢之。遠,于萬反,又如字,注同。}汝分猷念以相從,各設中于乃心。\footnote{群臣當分明相與謀念,和以相從,各設中正于汝心。分,扶問反,又如字,注同。}乃有不吉不迪,\footnote{不善不道,謂兇人。}顛越不恭,暫遇姦宄,\footnote{顛,隕。越,墜也。不恭,不奉上命。暫遇人而劫奪之,為奸于外,為宄于內。暫,才淡反。隕,于敏反。}我乃\xpinyin*{劓殄}滅之,無遺育,無俾易種于茲新邑。\footnote{劓,割。育,長也。言不吉之人當割絕滅之,無遺長其類,無使易種于此新邑。劓,魚器反,徐吾氣反。殄,徒典反。易如字,又以豉反,注同。長,丁丈反,下“遺長”同。}往哉生生!今予將試以汝遷,永建乃家。”\footnote{自今以往,進進于善。我乃以汝徙,長立汝家。卿大夫稱家。}

{\noindent\zhuan\zihao{6}\fzbyks 傳“不易之事”。正義曰:此“易”讀為“難易”之“易”,“不易”言其難也。\CJKunderline{王肅}云:“告汝以命之不易。”亦以“不易”為難。\CJKunderline{鄭玄}云:“我所以告汝者不變易,言必行之。”謂\CJKunderline{盤庚}自道己言必不改易,與孔異。 \par}

{\noindent\zhuan\zihao{6}\fzbyks 傳“顛隕”至“于內”。正義曰:\CJKunderwave{釋詁}云:“隕,落也。隕,墜也。”“顛越”是從上倒下之言,故以“顛”為隕,“越”是遺落,為墜也。\CJKunderwave{左傳}僖九年齊桓公云:“恐隕越于下。”文十八年史克云:“弗敢失墜。”“隕”、“越”是遺落廢失之意,故以隕墜不恭為“不奉上命”也。“暫遇人而劫奪之”謂逄人即劫,為之無已。成十七年\CJKunderwave{左傳}曰“亂在外為奸,在內為宄”,是劫奪之事,故以劫奪解其“姦宄”也。 \par}

{\noindent\zhuan\zihao{6}\fzbyks 傳“劓割”至“新邑”。正義曰:五刑截鼻為劓,故“劓”為割也。“育,長”,\CJKunderwave{釋詁}文。“不吉之人當割絕滅之,無遺長其類”,謂早殺其人,不使得子孫,有此惡類也。“易種”者,即今俗語云“相染易”也。惡種在善人之中,則善人亦變易為惡,故絕其類,無使易種于此新邑也。滅去惡種,乃是常法,而言“于此新邑”,言己若至新都,當整齊使絜清。 \par}

{\noindent\zhuan\zihao{6}\fzbyks 傳“自今”至“稱家”。正義曰:“長立汝家”謂賜之以族,使子孫不絕,\CJKunderwave{左傳}所謂“諸侯命氏”是也。王朝大夫,天子亦命之氏,故云“立汝家”也。 \par}

{\noindent\shu\zihao{5}\fzkt “嗚呼”至“乃家”。正義曰:\CJKunderline{盤庚}以言事將畢,欲戒使入之,故“嗚呼”而嘆之。今我告汝皆不易之事,言其難也。事既不易,當長敬我言,大憂行之,無相絕遠棄廢之,必須存心奉行。汝群臣臣分輩相與計謀念,和協以相從,各設中正于汝心,勿為殘害之事。汝群臣若有不善不道,隕墜禮法,不恭上命,暫逄遇人,即為姦宄而劫奪之,我乃割絕滅之,無有遺餘生長。所以然者,欲無使易其種類于此新邑故耳。自今已往哉,汝當進進于善。今我將用以汝遷,長立汝家,使汝在位,傳諸子孫。勿得違我言也。 \par}

\section{盤庚下第十一(盤庚)}

\CJKunderline{盤庚}既遷,奠厥攸居,乃正厥位,\footnote{定其所居,正郊廟朝社之位。奠,田薦反。朝,直遙反。}綏爰有眾,曰:“無戲怠,懋建大命。\footnote{安于有眾,戒無戲怠,勉立大教。}今予其敷心腹腎腸,歷告爾百姓于朕志。\footnote{布心腹,言輸誠于百官以告志。腎,時忍反。腸,徐待良反。}罔罪爾眾,爾無共怒,協比讒言予一人。\footnote{群臣前有此過,故禁其後。今我不罪汝,汝勿共怒我,合比兇人而妄言。比,毗志反。讒,仕咸反。}

{\noindent\zhuan\zihao{6}\fzbyks 傳“定其”至“之位”。正義曰:訓“攸”為所,“定其所居”,總謂都城之內官府萬民之居處也。\CJKunderline{鄭玄}云:“徙主于民,故先定其裡宅所處,次乃正宗廟朝廷之位。”如鄭之意,“奠厥攸居”者,止謂定民之居,豈先令民居使足,待其餘剩之處,然後建王宮乎?若留地以擬王宮,即是先定王居,不得為先定民矣。孔惟言“定其所居”,知是官民之居並定之也。\CJKunderwave{禮}郊在國外,左祖右社,面朝後市,“正厥位”謂正此郊廟朝社之位也。 \par}

{\noindent\zhuan\zihao{6}\fzbyks 傳“安于”至“大教”。正義曰:\CJKunderline{鄭玄}云:“勉立我大命,使心識教令,常行之。”\CJKunderline{王肅}云:“勉立大教,建性命,致之五福。”又案下句爾然共怒予一人,是恐其不從已命,此句宜言我有教命,汝當勉力立之。鄭說如孔旨也。 \par}

{\noindent\zhuan\zihao{6}\fzbyks 傳“布心”至“告志”。正義曰:此論心所欲言,腹內之事耳。以心為五臟之主,腹為六腑之總,腸在腹內,腎在心下,舉“腎腸”以配“腹心”,\CJKunderwave{詩}曰:“公侯腹心。”宣十二年\CJKunderwave{左傳}云:“敢布腹心。”是“腹心”足以表內,“腎腸”配言之也。 \par}

{\noindent\shu\zihao{5}\fzkt “\CJKunderline{盤庚}”至“一人”。正義曰:\CJKunderline{盤庚}既遷至殷地,定其國都處所,乃正其郊廟朝社之位。又屬民而聚之,安慰于其所有之眾,曰:“汝等自今以後,無得遊戲怠惰,勉力立行教命。今我其布心腹腎腸,輸寫誠信,歷遍告汝百姓于我心志者。”欲遷之日,民臣共怒\CJKunderline{盤庚}盤庚,恐其怖懼,故開解之。“今我無復罪汝眾人。我既不罪汝,汝無得如前共為忿怒,協比讒言毀惡我一人”。恕其前愆,與之更始也。 \par}

古我先王,將多于前功,\footnote{言以遷徙多大前人之功美。}適于山,用降我凶德,嘉績于朕邦。\footnote{徙必依山之險,無城郭之勞。下去兇惡之德,立善功于我國。降,工巷反。去,羌呂反。}今我民用蕩析離居,罔有定極。\footnote{水泉沉溺,故蕩析離居,無安定之極,徙以為之極。}

{\noindent\zhuan\zihao{6}\fzbyks 傳“言以”至“功美”。正義曰:“古我先王”,謂遷都者。“前人”謂未遷者。前人久居舊邑,民不能相匡以生,則是居無功矣。\CJKunderline{盤庚}言先王以此遷,徙故多大前人之功美,故我今遷,亦欲多前功矣。 \par}

{\noindent\zhuan\zihao{6}\fzbyks 傳“徙必”至“我國”。正義曰:先王至此五邦,不能盡知其地,所都皆近山,故總稱“適于山”也。\CJKunderwave{易·坎卦}彖云:“王公設險以守其國。”徙必依山之險,欲使下民無城郭之勞。雖則近山,不可全無城郭,言其防守易耳。徙必近山,則舊處新居皆有山矣。而云“適于山”者,言其徙必依山,不適平地,不謂舊處無山,故徙就山也。水泉咸鹵,民居墊隘,時君不為之徙,即是兇惡之德。其徙者是下去兇惡之德,立善功于我新遷之國也。言“下”者,凶德在身,下而墜去之。 \par}

{\noindent\zhuan\zihao{6}\fzbyks 傳“水泉”至“之極”。正義曰:民居積世,穿掘處多,則水泉盈溢,令人沈深而陷溺。其處不可安居,播蕩分析,離其居宅,無安定之極。“極”訓中也。\CJKunderwave{詩}云:“立我烝民,莫匪爾極。”言民賴后稷之功,莫不得其中。今為民失中,故徙以為之中也。 \par}

{\noindent\shu\zihao{5}\fzkt “古我”至“定極”。正義曰:言古者我之先王,將欲多大于前人之功,是故徙都而適于山險之處,用下去我兇惡之德,立善功于我新國。但徙來已久,水泉沉溺,今我在此之民,用播蕩分析,離其居宅,無有安定之極,我今徙而使之得其中也。說其遷都之意,亦欲多大前人之功,定民極也。 \par}

爾謂朕:‘曷震動萬民以遷?’\footnote{言皆不明己本心。}肆上帝將復我高祖之德,亂越我家。\footnote{以徙故,天將復湯德,治理于我家。治,直吏反。}朕及篤敬,恭承民命,用永地于新邑。\footnote{言我當與厚敬之臣,奉承民命,用長居新邑。}肆予沖人,非廢厥謀,吊由靈。\footnote{衝,童。童人,謙也。吊,至。靈,善也。非廢,謂動謀于眾,至用其善。吊音的,或如字。}各非敢違卜,用宏茲賁。\footnote{宏、賁皆大也。君臣用謀,不敢違卜,用大此遷都大業。賁,扶雲反。}

{\noindent\zhuan\zihao{6}\fzbyks 傳“以徙”至“我家”。正義曰:民害不徙,違失湯德。以徙之故,天必祐我,將使復奉湯德,令得治理于我家。言由徙故天福之也。 \par}

{\noindent\zhuan\zihao{6}\fzbyks 傳“衝童”至“其善”。正義曰:“衝”、“童”聲相近,皆是幼小之名。自稱“童人”,言己幼小無知,故為“謙也”。“吊,至”、“靈,善”皆\CJKunderwave{釋詁}文。\CJKunderwave{禮}將有大事,必謀于眾。謀眾乃是常理,故言“非廢,謂動謀于眾”,言己不自專也。眾謀必有異見,故至極用其善者。 \par}

{\noindent\zhuan\zihao{6}\fzbyks 傳“宏賁”至“大業”。正義曰:“宏、賁皆大也”,\CJKunderwave{釋詁}文。樊光曰:“\CJKunderwave{周禮}雲‘其聲大而宏’,\CJKunderwave{詩}雲‘有賁其首’,是宏、賁皆為大之義也。”“各”者非一之辭,故為“君臣用謀,不敢違卜”。\CJKunderwave{洪範}云:“汝則有大疑,謀及卿士,謀及卜筮。”言“非敢違卜”,是既謀及于眾,又決于蓍龜也。“用大此遷都”,“大”謂立嘉績以大之也。 \par}

{\noindent\shu\zihao{5}\fzkt “爾謂”至“茲賁”。正義曰:言我徙以為民立中,汝等不明我心,乃謂我何故震動萬民以為此遷。我以此遷之故,上天將復我高祖\CJKunderline{成湯}之德,治理于我家。我當與厚敬之臣,奉承民命,用是長居于此新邑。以此須遷之故,我童蒙之人,非敢廢其詢謀。謀于眾人,眾謀不同,至用其善者。言善謀者,皆欲遷都也。又決之于龜卜而得吉,我與汝群臣各非敢違卜,用是必遷,光大此遷都之大業。我徙本意如此耳。 \par}

“嗚呼!邦伯師長,百執事之人,尚皆隱哉!\footnote{國伯,二伯及州牧也。眾長,公卿也。言當庶幾相隱括共為善政。長,丁丈反,注同。}予其懋簡相爾,念敬我眾。\footnote{簡,大。相,助也。勉大助汝,念敬我眾民。相,息亮反。}朕不肩好貨,敢恭生生。鞠人、謀人之保居,敘欽。\footnote{肩,任也。我不任貪貨之人,敢奉用進進于善者。人之窮困能謀安其居者,則我式序而敬之。好,呼報反。任,而林反。}

{\noindent\zhuan\zihao{6}\fzbyks 傳“國伯”至“善政”。正義曰:“邦伯”,邦國之伯,諸侯師長,故為東西二伯及九州之牧也。\CJKunderline{鄭玄}注\CJKunderwave{禮記}云:“殷之州長曰伯,虞夏及周皆曰牧。”此殷時而言“牧”者,此乃鄭之所約,孔意不然,故總稱“牧”也。“師”訓為眾,“眾長”,眾官之長,故為三公六卿也。其“百執事”,謂大夫以下,諸有職事之官皆是也。此部敕眾臣,故二伯已下及執事之人皆戒之也。\CJKunderwave{釋言}云:“庶幾,尚也。”反覆相訓,故“尚”為庶幾。“庶”,幸也。“幾”,冀也。“隱”謂隱審也。幸冀相與隱審檢括,共為善政,欲其同心共為善也。“隱括”必是舊語,不知本出何書。何休\CJKunderwave{公羊序}云:“隱括使就繩墨焉。” \par}

{\noindent\zhuan\zihao{6}\fzbyks 傳“簡大”至“眾民”。正義曰:“簡,大”,\CJKunderwave{釋詁}文。又云:“相、助,慮也。”俱訓為“慮”,是“相”得為助也。\CJKunderline{盤庚}欲使群臣同心為善,欲勉力大佐助之,使皆念敬我眾民也。 \par}

{\noindent\zhuan\zihao{6}\fzbyks 傳“肩任”至“敬之”。正義曰:\CJKunderwave{釋詁}云:“肩,勝也。”舍人曰:“肩,強之勝也。”強能勝重,是堪任之義,故為任也。我今不委任貪貨之人。以“恭”為奉。人有向善而心不決志,故美其人能果敢奉用進進于善者,言其人好善不倦也。“鞠”訓為窮,“鞠人”謂窮困之人。“謀人之保居”,謂謀此窮人之安居,若見人之窮困,能謀安其居。愛人而樂安存之者,則我式序而敬之。\CJKunderwave{詩}云:“式序在位。”言其用次序在官位也。鄭、王皆以“鞠”為養,言“能謀養人安其居者,我則次序而敬之”,與孔不同。 \par}

{\noindent\shu\zihao{5}\fzkt “嗚呼”至“敘欽”。正義曰:言遷事已訖,故嘆而敕之:“嗚呼!國之長伯,及眾官之長與百執事之人,庶幾皆相與隱括共為善政哉!我其勉力大助汝等為善,汝當思念愛敬我之眾民。我不任用好貨之人,有人果敢奉用進進于善,見窮困之人能謀此窮困之人安居者,我乃次序而敬用之。” \par}

今我既羞告爾于朕志,若否,罔有弗欽。\footnote{已進告汝之後,順于汝心與否,當以情告我,無敢有不敬。告,故報反。}無總于貨寶,生生自庸。\footnote{無總貨寶以求位,當進進皆自用功德。}式敷民德,永肩一心。”\footnote{用布示民,必以德義,長任一心以事君。}

{\noindent\shu\zihao{5}\fzkt “今我”至“一心”。正義曰:今我既進而告汝于我心志矣,其我所告,順合于汝心以否,當以情告我,無得有不敬者。汝等無得總于貨寶以求官位,當進進自用功德,不當用富也。用此布示于民,必以德義,長任一心以事君,不得懷二意。以遷都既定,故殷勤以戒之。 \par}

%%% Local Variables:
%%% mode: latex
%%% TeX-engine: xetex
%%% TeX-master: "../Main"
%%% End:
