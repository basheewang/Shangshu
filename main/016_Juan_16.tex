%% -*- coding: utf-8 -*-
%% Time-stamp: <Chen Wang: 2024-04-02 11:42:41>

% {\noindent \zhu \zihao{5} \fzbyks } -> 注 (△ ○)
% {\noindent \shu \zihao{5} \fzkt } -> 疏

\chapter{卷十六}


\section{多士第十六}


成周既成,\footnote{洛陽下都。}遷殷頑民,\footnote{殷大夫士心不則德義之經,故徙近王都教誨之。○不則,如字,或作測,非。近,附近之近。}周公以王命誥,\footnote{稱成王命告令之。}作\CJKunderwave{多士}。

多士\footnote{所告者即眾士,故以名篇。}


{\noindent\zhuan\zihao{6}\fzbyks 傳“洛陽下都”。正義曰:周之成周,於漢為洛陽也。洛邑為王都,故謂此為“下都”。遷殷頑民,以成周道,故名此邑為成周。 \par}

{\noindent\zhuan\zihao{6}\fzbyks 傳“殷大”至“誨之”。正義曰:經雲“商王士”、“殷遺多士”,皆非民,事謂之“頑民”,知是殷之大夫士也。經止雲“士”,而知有大夫者,以經雲“迪簡在王庭,有服在百僚”,其意言將任為王官,以為大臣,不惟告士而已,故知有大夫也。“士”者,在官之總號,故言“士”也。“心不則德義之經”,僖二十四年\CJKunderwave{左傳}文,引之以解釋“頑民”之意。經雲“移爾遐逖,比事臣我宗多遜”,是言“徙近王都教誨之”也。\CJKunderwave{漢書·地理志}及賈逵注\CJKunderwave{左傳},皆以為遷邶鄘之民於成周,分衛民為三國。計三國俱是從叛,何以獨遷邶鄘?邶鄘在殷畿三分有二,其民眾矣,非一邑能容,民謂之為“士”,其名不類,故孔意不然。 \par}

{\noindent\shu\zihao{5}\fzkt “成周”至“多士”。正義曰:成周之邑既成,乃遷殷之頑民,令居此邑。“頑民”謂殷之大夫士從武庚叛者,以其無知,謂之“頑民”。民性安土重遷,或有怨恨,周公以成王之命誥此眾士,言其須遷之意。史敘其事,作\CJKunderwave{多士}。 \par}

惟三月,周公初於新邑洛,用告商王士。\footnote{周公致政明年三月,始於新邑洛,用王命告商王之眾士。}


{\noindent\zhuan\zihao{6}\fzbyks 傳“周公”至“眾士”。正義曰:以\CJKunderwave{洛誥}之文,成周與洛邑同時成也。王以周公攝政七年十二月來至新邑,明年即政,此篇繼王居洛之後,故知是“致政明年之三月”也。成周南臨洛水,故云“新邑洛”。周公既以致政在王都,故“新邑”,成周。以成王之命告商王之眾士,鄭雲“成王元年三月,周公自王城初往成周之邑,用成王命告殷之眾士以撫安之”是也。 \par}

{\noindent\shu\zihao{5}\fzkt “惟三月”至“王士”。正義曰:惟成王即政之明年三月,周公初始於所造新邑之洛,用成王之命告商王之眾士。言周公親至成周,告新來者。 \par}

王若曰:“爾殷遺多士,\footnote{順其事稱以告殷遺餘眾士。所順在下。}弗吊,旻天大降喪於殷。\footnote{稱天以愍下,言愍道至者,殷道不至,故旻天下喪亡於殷。○吊音的。旻天,上閔巾反,仁覆愍下謂之旻,馬云:“秋日旻天。秋,氣殺也,方言‘降喪’,故稱旻天也。”愍,眉隕反。喪,息浪反。}我有周佑命,將天明威,\footnote{言我有周受天佑助之命,故得奉天明威。}致王罰,敕殷命終於帝。\footnote{天命周致王者之誅罰,王黜殷命,終周於帝王。}肆爾多士,非我小國敢弋殷命。\footnote{天佑我,故汝眾士臣服我。弋,取也。非我敢取殷王命,乃天命。○弋,徐音翼,馬本作翼,義同。}惟天不畀允罔固亂,弼我,我其敢求位?\footnote{惟天不與信無堅固治者,故輔佑我,我其敢求天位乎?○治,直吏反。畀,必利反,下同。}惟帝不畀,惟我下民秉為,惟天明畏。\footnote{惟天不與紂,惟我周家下民秉心為我,皆是天明德可畏之效。○為,於偽反。畏如字,一音威。}

{\noindent\zhuan\zihao{6}\fzbyks 傳“順其”至“在下”。正義曰:順其殷亡之事,稱王命以告之。從紂之臣,或有身已死者,遺餘在者遷於成周,故“告殷遺餘眾士”。“所順在下”,下文皆是順之辭。 \par}

{\noindent\zhuan\zihao{6}\fzbyks 傳“稱天”至“於殷”。正義曰:此經先言“弗吊”,謂殷道不至也。“不至”者,上不至天,事天不以道;下不至民,撫民不以理也。天有多名,獨言“旻天”者,“旻”,愍也,“稱天以愍下”,言天之所愍,愍道至者也。“殷道不至,故旻天下喪亡於殷”,言將覆滅之。 \par}

{\noindent\zhuan\zihao{6}\fzbyks 傳“天命”至“帝王”。正義曰:“天命周致王者之誅罰”,謂奉上天之命,殺無道之王,此乃王者之事,故為“王者之誅罰”。“敕”訓正也,“正黜殷命”謂殺去虐紂,使周受其終事,是“終周於帝王”。“終”猶舜受堯終,言殷祚終而歸於周。 \par}

{\noindent\zhuan\zihao{6}\fzbyks 傳“天祐”至“天命”。正義曰:“肆”訓故也,直雲故爾多士,辭無所結,此經大意,敘其去殷事周,知其故爾眾士言其臣服我。“弋”,射也,射而取之,故“弋”為取也。\CJKunderline{鄭玄}、王肅本“弋”作翼,王亦云:“翼,取也。”鄭云:“翼猶驅也,非我周敢驅取汝殷之王命。”雖訓為驅,亦為取義。周本殷之諸侯,故周公自稱“小國”。 \par}

{\noindent\shu\zihao{5}\fzkt “王若”至“明畏”。正義曰:周公以王命順其事而呼之曰:“汝殷家遺餘之眾士,汝殷家道教不至,旻天以殷道不至之故,天下喪亡於殷,將欲滅殷。我有周受天祐助之命,奉天明白之威,致王者之誅罰,正黜殷命,終我周家於帝王之事,謂使我周家代殷為天子也。天既助我周王,故汝眾士來為我臣。由天助我,我得為之,非我小國敢取殷之王命以為己有。此乃天與我,惟天不與信無堅固於治者,以是故輔弼我。若其不然,我其敢妄求天子之位乎?”言此位天自與我,非我求而得之。“惟天不與紂,故惟我周家下民秉心為我,故我得之,惟天明德可畏之效也。亦既得喪由天,汝等不得不服”。以殷士未服,故以天命喻之。 \par}

我聞曰:‘上帝引逸。’有夏不適逸,則惟帝降格。\footnote{言上天欲民長逸樂,有夏桀為政不之逸樂,故天下至戒以譴告之。○樂音洛,下同。譴,棄戰反。}向於時夏,弗克庸帝,大淫泆有辭。\footnote{天下至戒,是向於時夏,不背棄。桀不能用天戒,大為過逸之行,有惡辭聞於世。○時夏,絕句,馬以“時”字絕句。向,許亮反。泆音逸,又作佾,注同;馬本作屑,云:“過也。”背音佩。行,下孟反。}惟時天罔念聞,厥惟廢元命,降致罰。\footnote{惟是桀惡有辭,故天無所念聞,言不佑,其惟廢其天命,下致天罰。}乃命爾先祖\CJKunderline{成湯}革夏,俊民甸四方。\footnote{天命湯更代夏,用其賢人治四方。○甸,徒遍反。}


{\noindent\zhuan\zihao{6}\fzbyks 傳“言上”至“告之”。正義曰:襄十四年\CJKunderwave{左傳}稱:“天之愛民甚矣。”又曰:“天生民而立之君,使司牧之。”是言上天欲民長得逸樂,故立君養之,使之長逸樂也。夏桀為政,割剝夏邑,使民不得之適逸樂,故上天下此至戒以譴告之。“降”,下。“格”,至也。直言下至,明是“天下至戒”。天所下戒,惟下災異以譴告人主,使之見災而懼,改修德政耳。古書亡失,桀之災異未得盡聞。 \par}

{\noindent\zhuan\zihao{6}\fzbyks 傳“惟是”至“天罰”。正義曰:桀惡流毒於民,乃有惡辭聞於世。惡既有辭,是惡已成矣。惟是桀惡有辭,故天無所念聞,言天不愛念,不聽聞,是其全棄之,不佑助也。棄而不佑,則當更求賢主。“其惟廢大命”,欲奪其王位也。“下致天罰”,欲殺其凶身也。廢大命,知“降致”是下罰也。 \par}

{\noindent\shu\zihao{5}\fzkt “我聞”至“四方”。正義曰:既言天之效驗,法惡與善,更追說往事,比而喻之:“我聞人有言曰:‘上天之情,欲民長得逸樂。’而有夏王桀逆天害民,不得使民之適逸樂。以此則惟上天下災異至戒以譴告之,欲使夏王桀覺悟,改惡為善,是天歸鄉於是夏家,不背棄之。而夏桀不能用天之明戒,改悔己惡,而反大為過逸之行,致有惡辭以聞於世。惟是桀有惡辭,故天無復愛念,無復聽聞,言天不復助桀。其惟廢其大命,欲絕夏祚也。下致天罰,欲誅桀身也。乃命汝先祖\CJKunderline{成湯},使之改革夏命,用其賢俊之人,以治四方之國。舉桀滅湯興以譬之。 \par}

自\CJKunderline{成湯}至於帝乙,罔不明德恤祀。\footnote{自帝乙以上,無不顯用有德,憂念齊敬,奉其祭祀。言能保宗廟社稷。○上,時掌反。齊,側皆反。}亦惟天丕建保乂有殷,殷王亦罔敢失帝,罔不配天其澤。\footnote{湯既革夏,亦惟天大立安治於殷。殷家諸王皆能憂念祭祀,無敢失天道者,故無不配天布其德澤。}在今後嗣王,誕罔顯於天,矧曰其有聽念於先王勤家?\footnote{後嗣王紂,大無明於天道,行昏虐,天且忽之,況曰其有聽念先祖、勤勞國家之事乎?}誕淫厥泆,罔顧於天顯民祗,\footnote{言紂大過其過,無顧於天,無能明人為敬,暴亂甚。}惟時上帝不保,降若茲大喪。\footnote{惟是紂惡,天不安之,故下若此大喪亡之誅。○喪,息浪反。}惟天不畀不明厥德,凡四方小大邦喪,罔非有辭於罰。”\footnote{惟天不與不明其德者,故凡四方小大國喪滅,無非有辭於天所罰。言皆有闇亂之辭。}


{\noindent\zhuan\zihao{6}\fzbyks 傳“自帝”至“社稷”。正義曰:下篇說中宗、高宗、祖甲三王以外,其後立王,生則逸豫,亦罔或能壽。如彼文,則帝乙以上非無僻王,而此言無不顯用有德,憂念祭祀者,立文之法,辭有抑揚,方說紂之不善,盛言前世皆賢,正以守位不失,故得美而言之。憂念祭祀者,惟有齊肅恭敬,故言“憂念齊敬,奉其祭祀”。言能保宗廟社稷,為天下之主,以見紂不恭敬,故喪亡之。 \par}

{\noindent\zhuan\zihao{6}\fzbyks 傳“湯既”至“德澤”。正義曰:帝乙已上諸王,所以長處天位者,皆由湯之聖德延及後人。“湯既革夏,亦惟天大立安治於殷”者,謂天安治之,故殷家得治理也。殷家諸王自\CJKunderline{成湯}之後,皆能憂念祭祀,無敢失天道者,故得常處王位,無不配天布其德澤於民。為天之子,是“配天”也。號令於民,是“佈德”也。 \par}

{\noindent\zhuan\zihao{6}\fzbyks 傳“言紂”至“亂甚”。正義曰:“淫”、“泆”俱訓為過,“言紂大過其愆過,無顧於天”,言其縱心為惡,不畏天也。“無能明民為敬”,言其多行虐政,不憂民也。不畏於天,不愛於民,言其“暴亂甚”也。此經顧“於天”與“顯民祗”,共蒙上“罔”文,故傳再言“無”也。 \par}

{\noindent\zhuan\zihao{6}\fzbyks 傳“惟天”至“之辭”。正義曰:能明其德,天乃與之,惟天不與不明其德者,紂不明其德,故天喪之。因即廣言天意,凡四方小大邦國,謂諸侯有土之君,其為天所喪滅者,無非皆有惡辭聞於天,乃為上天所罰。言被天罰者,皆有闇亂之辭,上天不罰無辜,紂有闇亂之辭,故天滅之耳。天既滅不明其德,我有明德,為天所立,汝等殷士安得不服我乎?以其心仍不服,故以天道責之。 \par}

{\noindent\shu\zihao{5}\fzkt “自成”至“於罰”。正義曰:既言命湯革夏,又說後世皆賢,至紂始惡,天乃滅之。自\CJKunderline{成湯}至於帝乙,無不顯用有德,憂念祭祀,後世亦賢。非獨\CJKunderline{成湯}以用其行合天意,亦惟天大立安治有殷。殷家諸王皆能明德憂祀,亦無敢失天道者,無不皆配天而布其德澤,以此得天下久為民主。在今後嗣王紂,大無明於天道,敢行昏虐之政於天。天猶且忽之,況曰其有聽念先王父祖、勤勞國家之事乎?乃復大淫過其泆,無所顧於上天,無能明民為敬,以此反於先王,違逆天道。惟是上天不安紂之所為,下若此大喪亡之誅,惟天不與不明其德之人故也。天不與惡,豈獨紂乎?凡四方諸侯小大邦國,其喪滅者,無非皆有惡辭,是以致至於天罰。汝紂以惡而見滅,汝何以不服我也? \par}

王若曰:“爾殷多士,今惟我周王,丕靈承帝事,\footnote{周王,文武也。大神奉天事,言明德恤祀。}有命曰:‘割殷,告敕於帝。’\footnote{天有命,命周割絕殷命,告正於天。謂既克紂,柴於牧野,告天不頓兵傷士。}惟我事不貳適,惟爾王家我適。\footnote{言天下事已之我周矣,不貳之佗,惟汝殷王家已之我,不復有變。○復,扶又反。}予其曰:‘惟爾洪無度,我不爾動,自乃邑。’\footnote{我其曰,惟汝大無法度,謂紂無道。我不先動誅汝,亂從汝邑起。言自召禍。}予亦念天即於殷大戾,肆不正。”\footnote{我亦念天就於殷大罪而加誅者,故以紂不能正身念法。}


{\noindent\zhuan\zihao{6}\fzbyks 傳“周王”至“恤祀”。正義曰:文王受命,武王伐紂,故知“周王”兼文武也。“大神奉天事”,謂以天為神而勤奉事之,勞身敬神,言亦如湯明德恤祀也。 \par}

{\noindent\zhuan\zihao{6}\fzbyks 傳“天有”至“傷士”。正義曰:以周王奉天之故,故天有命,命我周使割絕殷命,告正於天。謂\CJKunderwave{武成}之篇所云,既克紂,柴於牧野,告天不頓兵傷士是也。前敵即服,故無“頓兵傷士”。師以正行,故為“告正”。\CJKunderwave{武成}正告功成,功成無害,即是不頓傷也。“頓兵”者,昭十五年\CJKunderwave{左傳}文“頓折”也。 \par}

{\noindent\zhuan\zihao{6}\fzbyks 傳“我亦”至“念法”。正義曰:言“我亦念天”者,以紂雖無法度,若使天不命我,我亦不往誅紂。以紂既為大惡,上天命我,我亦念天所遣。我就殷加大罪者何故?以紂不能正身念法也。 \par}

{\noindent\shu\zihao{5}\fzkt “王若”至“不正”。正義曰:周公又稱王順而言曰:“汝殷眾士,今惟我周家文武二王,大神能奉天事,故天有命,命我周王曰:‘當割絕殷命,告正於天。’我受天命,已滅殷告天,惟我天下之事,不有二處之適。”言己之適周,不更適他也。“惟汝殷王家事亦於我之適,不復變改”。又追說初伐紂之事:“我其為汝言曰:‘惟汝殷紂大無法度,故當宜誅絕之。伐紂之時,我不先於汝動,自往誅汝,其亂從汝邑先起,汝紂自召禍耳。’我亦念天所以就於殷致大罪者,故以紂不能正身念法故也。” \par}

王曰:“猷告爾多士,予惟時其遷居西爾。\footnote{以道告汝眾士,我惟汝未達德義,是以徙居西汝於洛邑,教誨汝。}非我一人奉德不康寧,時惟天命。\footnote{我徙汝,非我天子奉德,不能使民安之,是惟天命宜然。}無違,朕不敢有後,無我怨。\footnote{汝無違命,我亦不敢有後誅,汝無怨我。}惟爾知,惟殷先人,有冊有典,殷革夏命。\footnote{言汝所親知,殷先世有冊書典籍。說殷改夏王命之意。}今爾又曰:‘夏迪簡在王庭,有服在百僚。’\footnote{簡,大也。今汝又曰:“夏之眾士蹈道者,大在殷王庭,有服職在百官。”言見任用。}予一人惟聽用德,肆予敢求爾於天邑商。\footnote{言我周亦法殷家,惟聽用有德,故我敢求汝於天邑商,將任用之。}予惟率肆矜爾,非予罪,時惟天命。”\footnote{惟我循殷故事,憐愍汝,故徙教汝,非我罪咎,是惟天命。}

{\noindent\zhuan\zihao{6}\fzbyks 傳“以道”至“誨汝”。正義曰:“猷”訓道也,故云“以道告汝眾士”。上言“惟是”,不言其故,故傳辨之,惟是者,未達德義也。遷使居西,正欲教以德義,是以徙居西汝置於洛邑,近於京師教誨汝也。從殷適洛,南行而西回,故為“居西”也。 \par}

{\noindent\zhuan\zihao{6}\fzbyks 傳“汝無”至“怨我”。正義曰:周既伐紂,又誅武庚,殷士懼更有誅,疑其欲違上命,故設此言以戒之。知“無違朕”者,謂戒之使汝無違命也。汝能用命,我亦不敢有後誅,必無後誅,汝無怨我也。 \par}

{\noindent\zhuan\zihao{6}\fzbyks 傳“言我”至“用之”。正義曰:夏人簡在王庭,為其有德見用。言我亦法殷家,惟聽用有德,汝但有德,我必任用。故我往前敢求汝有德之人於天邑商都,將任用之也。\CJKunderline{鄭玄}云:“言天邑商者,亦本天之所建。”王肅云:“言商今為我之天邑。”二者其言雖異,皆以“天邑商”為殷之舊都。言未遷之時,當求往,遷後有德任用之必矣。 \par}

{\noindent\zhuan\zihao{6}\fzbyks 傳“惟我”至“天命”。正義曰:“循殷故事”,此“故”解經中“肆”字,謂殷用夏人,我亦用殷人。“憐愍汝,故徙之教汝”,此“故”解義之言,非經中“肆”。遷汝來西者,非我罪咎,是惟天命也。 \par}

{\noindent\shu\zihao{5}\fzkt “王曰猷”至“大命”。正義曰:又言曰我以道告汝眾士,我惟是以汝未達德義之故,其今徙居西汝置於洛邑,以教誨汝。我之徙汝,非我一人奉行德義,不能使民安而安之,是惟天命宜然。汝無違我,我亦不敢更有後誅罰,汝等無於我見怨。汝既來遷,當為善事。惟汝所親知,惟汝殷先人往世有策書,有典籍,說殷改夏王命之意,汝當案省知之。汝知先人之故事,今往又有言曰:‘夏之諸臣蹈道者,大在殷王之庭,有服行職事,在於百官。’言其見任用,恐我不任汝。我一人惟聽用有德之者,故我敢求汝有德之人於彼天邑商都,欲取賢而任用之。我惟循殷故事,憐愍汝,故徙教汝。此徙非我有罪,是惟天命當然。”聖人動合天心,故每事惟託天命也。 \par}

王曰:“多士,昔朕來自奄,予大降爾四國民命。\footnote{昔我來從奄,謂先誅三監,後伐奄淮夷。民命謂君也。大下汝民命,謂誅四國君。}我乃明致天罰,移爾遐逖,比事臣我宗,多遜。”\footnote{四國君叛逆,我下其命,乃所以明致天罰。今移徙汝於洛邑,使汝遠於惡俗,比近臣我宗周,多為順道。○逖,他力反。比,毗志反,注同。遠,於萬反。}


{\noindent\zhuan\zihao{6}\fzbyks 傳“昔我”至“國君”。正義曰:\CJKunderwave{金縢}之篇說周公東征,言“居東二年,罪人斯得”,則“昔我來從奄”者,謂攝政三年時也。於時王不親行,而王言“我來自奄”者,周公以王命誅四國,周公師還,亦是王來還也。一舉而誅四國,獨言“來自奄”者,謂先誅三監,後伐奄與淮夷,奄誅在後,誅奄即來,故言“來自奄”也。民以君為命,故“民命謂君也”。大下汝民命,謂誅四國君。王肅云:“君為民命,為君不能順民意,故誅之也。” \par}

{\noindent\zhuan\zihao{6}\fzbyks 傳“四國”至“順道”。正義曰:王之所罰,罰有罪也。四國之君,有叛逆之罪。“我下其命,乃所以明致天罰”,言非苟為之也。“遐”、“逖”俱訓為遠。“今移徙汝於洛邑”,令去本鄉遠也。“使汝遠於惡俗”,令去惡俗遠也。比近京師,臣我周家,使汝從我善化,多為順道,所以救汝之性命也。 \par}

{\noindent\shu\zihao{5}\fzkt “王曰多士”至“多遜”。正義曰:王復言曰:“眾士,昔我來從奄國,大黜下汝管蔡商奄四國民命。民之性命,死生在君,誅殺其君,是下民命。由四國叛逆,我乃明白致行天罰。汝等遣餘,當教之為善,故移徙汝居於遠。令汝遠於惡俗,比近服事臣我宗周,多為順道。翼汝相教為善,永不為惡也。” \par}

王曰:“告爾殷多士,今予惟不爾殺,予惟時命有申。\footnote{所以徙汝,是我不欲殺汝,故惟是教命申戒之。}今朕作大邑於茲洛,予惟四方罔攸賓,\footnote{今我作此洛邑,以待四方,無有遠近,無所賓外。○賓如字,徐音殯,馬云:“卻也。”}亦惟爾多士,攸服奔走臣我,多遜。\footnote{非但待四方,亦惟汝眾士,所當服行奔走臣我,多為順事。}爾乃尚有爾土,爾乃尚寧幹止。\footnote{汝多為順事,乃庶幾還有汝本土,乃庶幾安汝故事止居。以反所生誘之。爾克敬,天惟畀矜爾。○汝能敬行順事,則為天所與,為天所憐。}爾不克敬,爾不啻不有爾土,予亦致天之罰於爾躬。\footnote{汝不能敬順,其罰深重,不但不得還本土而已,我亦致天罰於汝身。言刑殺。○啻,始豉反,徐本作翅,音同,下篇放此。}今爾惟時宅爾邑,繼爾居,爾厥有幹有年於茲洛。\footnote{今汝惟是敬順居汝邑,繼汝所當居為,則汝其有安事,有豐年於此洛。邑言由洛修善,得還本土,有幹有年。}爾小子乃興,從爾遷。”\footnote{汝能敬,則子孫乃起從汝化而遷善。}


{\noindent\zhuan\zihao{6}\fzbyks 傳“今汝”至“有年”。正義曰:殷士遠離本鄉,新來此邑,或當居不安,為棄舊業,故戒之。“今汝惟是敬順,居汝新所受邑,繼汝舊日所當居為”,謂繼其本土之事業也。但能如此,得還本土,其有安事,有豐年也。“有幹有年”謂歸本土。有幹年而言於洛者,言“由在洛修善,得還本土,有幹有年”也。王肅云:“汝其有安事,有長久年於此洛邑。”王解於文甚便,但孔上句為雲“爾乃尚有爾本土”,是誘引之辭,故止為得“還本土,有幹有年”也。 \par}

{\noindent\shu\zihao{5}\fzkt “王曰告”至“爾遷”。正義曰:王又言曰:“告汝殷之多士,所以遠徙汝者,今我惟不欲於汝刑殺,我惟是教命有所申戒由此也。今我作大邑於此洛,非但為我,惟以待四方,無所賓外,亦惟為汝眾士所當服行臣事我宗周,多為順事故也。汝若多為順事,汝乃庶幾還有汝本土,乃庶幾安汝故事止居,可不勉之也?汝能敬行順事,天惟與汝憐汝,況於人乎?汝若不能敬行順事,則汝不啻不得還汝本土,我亦致天之罰於汝身。今汝惟是敬順,居汝所受新邑,繼汝舊日所居為,我當聽汝還歸本鄉,有幹事,有豐年,乃由於此洛邑行善也。汝能敬順,則汝之小子與孫等,乃起從汝化而遷善矣。” \par}

王曰:“又曰時予,乃或言,爾攸居。”\footnote{言汝眾士當是我,勿非我也。我乃有教誨之言,則汝所當居行。}


{\noindent\zhuan\zihao{6}\fzbyks 傳“言汝”至“居行”。正義曰:王以誨之已終,故戒之云:“汝當是我,勿非我。既不非我,我乃有教誨汝之言,則汝所當居行。”令其居於心而行用之。\CJKunderline{鄭玄}\CJKunderwave{論語}注云“或之言有”,此亦“或”為有也。凡言“王曰”,皆是史官錄辭,非王語也。今史錄稱王之言曰,以前事未終,故言“又曰”也。 \par}

{\noindent\shu\zihao{5}\fzkt “王曰又”至“攸居”。正義曰:王之所云,又複稱曰:“汝當是我,勿非我也。我乃有教誨之言,則汝所當居行之。” \par}

\section{無逸第十七}


周公作\CJKunderwave{無逸}。\footnote{中人之性好逸豫,故戒以無逸。○好,呼報反。}

無逸\footnote{成王即政,恐其逸豫,本以所戒名篇。}


{\noindent\zhuan\zihao{6}\fzbyks 傳“成王”至“名篇”。正義曰:篇之次第,以先後為序,\CJKunderwave{多士}、\CJKunderwave{君奭}皆是成王即位之初,知此篇是成王始初即政,周公恐其逸豫,故戒之,使無逸,即以所戒名篇也。 \par}

{\noindent\shu\zihao{5}\fzkt 傳“中人”至“無逸”。正義曰:上智不肯為非,下愚戒之無益,故中人之性,可上可下,不能勉強,多好逸豫,故周公作書以戒之,使無逸。此雖指戒成王,以為人之大法,成王以聖賢輔之,當在中人以上,其實本性亦中人耳。 \par}

周公曰:“嗚呼!君子所其無逸。\footnote{嘆美君子之道,所在唸德,其無逸豫。君子且猶然,況王者乎?}先知稼穡之艱難,乃逸,則知小人之依。\footnote{稼穡農夫之艱難,事先知之,乃謀逸豫,則知小人之所依怙。}相小人,厥父母勤勞稼穡,厥子乃不知稼穡之艱難,\footnote{視小人不孝者,其父母躬勤艱難,而子乃不知其勞。○相,息亮反。}乃逸乃諺。既誕,否則侮厥父母曰:‘昔之人無聞知。’”\footnote{小人之子既不知父母之勞,乃為逸豫遊戲,乃叛諺不恭。已欺誕父母,不欺,則輕侮其父母曰:“古老之人無所聞知。”○諺,魚戰反。}


{\noindent\zhuan\zihao{6}\fzbyks 傳“嘆美”至“者乎”。正義曰:周公意重其事,故嘆而為言。鄭云:“嗚呼者,將戒成王,欲求以深感動之。”是欲深感成王,故“嘆美君子之道”。“君子”者,言其可以君正上位,子愛下民,有德則稱之,不限貴賤。君子之人,念德不怠,故“所在唸德,其無逸豫”也。“君子且猶然,而況王者乎”,言王者日有萬幾,彌復不可逸豫。鄭云:“君子止謂在官長者。所,猶處也。君子處位為政,其無自逸豫也。” \par}

{\noindent\zhuan\zihao{6}\fzbyks 傳“稼穡”至“依怙”。正義曰:民之性命,在於穀食,田作雖苦,不得不為。寒耕熱耘,霑體塗足,是稼穡為農夫艱難之事。在上位者,先知稼穡之艱難,乃可謀其逸豫,使家給人足,乃得思慮不勞,是為“謀逸豫”也。能知稼穡之艱難,則知小人之所依怙,言小人依怙此稼穡之事,不可不勤勞也。上句言君子當無逸,此言“乃謀逸豫”者,君子之事,勞心與形。盤於遊畋,形之逸也;無為而治,心之逸也。君子無形逸而有心逸,既知稼穡之艱難,可以謀心逸也。 \par}

{\noindent\zhuan\zihao{6}\fzbyks 傳“視小人”至“其勞”。正義曰:視小人不孝者,其父母勤苦艱難,勞於稼穡,成於生業,致富以遺之。而其子謂己自然得之,乃不知其父母勤勞。 \par}

{\noindent\zhuan\zihao{6}\fzbyks 傳“小人”至“聞知”。正義曰:上言視小人之身,此言“小人之子”者,“小人”謂無知之人,亦是賤者之稱,躬為稼穡,是賤者之事,故言“小人之子”,謂賤者之子,即上所視之小人也。此子既不知父母之勞,謂己自然得富,恃其家富,乃為逸豫遊戲,乃為叛諺不恭,已是欺誕父母矣。若不欺誕,則輕侮其父母曰:“古老之人無所聞知。”言其罪之深也。\CJKunderwave{論語}曰:“由也諺。”諺則叛諺,欺誕不恭之貌。“昔”訓久也,自今而道遠久,故為“古老之人”。\CJKunderwave{詩}云:“召彼故老。” \par}

{\noindent\shu\zihao{5}\fzkt “周公”至“聞知”。正義曰:周公嘆美君子之道以戒王曰:“嗚呼!君子之人,所在其無逸豫。君子必先知農人稼穡之艱難,然後乃謀為逸豫,如是則知小人之所依怙也。視彼小人不孝者,其父母勤勞稼穡,其子乃不知稼穡之艱難,乃為逸豫遊戲,乃叛諺不恭。既為欺誕父母矣,不欺,則又侮慢其父母曰:‘昔之人無所聞知。’小人與君子如此相反,王宜知其事也。” \par}

周公曰:“嗚呼!我聞曰,昔在殷王中宗,\footnote{大戊也,殷家中世尊其德,故稱宗。}嚴恭寅畏天命,自度,\footnote{言太戊嚴恪恭敬,畏天命,用法度。○嚴如字,又魚檢反,注同,馬作儼。}治民祗懼,不敢荒寧。\footnote{為政敬身畏懼,不敢荒怠自安。○治,直吏反。}肆中宗之享國,七十有五年。\footnote{以敬畏之故,得壽考之福。}


{\noindent\zhuan\zihao{6}\fzbyks 傳“太戊”至“稱宗”。正義曰:“中宗”,廟號。“太戊”,王名。商自\CJKunderline{成湯}已後,政教漸衰,至此王而中興之。王者祖有功,宗有德,殷家中世尊其德,其廟不毀,故稱“中宗”。 \par}

{\noindent\zhuan\zihao{6}\fzbyks 傳“言太”至“法度”。正義曰:\CJKunderwave{祭義}雲“嚴威儼恪”,故引“恪”配“嚴”。\CJKunderline{鄭玄}云:“恭在貌,敬在心。”然則“嚴”是威,“恭”是貌,“敬”是心,三者各異,故累言之。 \par}

{\noindent\shu\zihao{5}\fzkt “周公”至“五年”。正義曰:既言君子不逸,小人反之,更舉前代之王以夭壽為戒。周公曰:“嗚呼!我所聞曰,昔在殷王中宗,威儀嚴恪,貌恭心敬,畏天命,用法度,治民敬身畏懼,不敢荒怠自安,故中宗之享有殷國七十有五年。”言不逸之故,而得歷年長也。 \par}

其在高宗,時舊勞於外,爰暨小人。\footnote{武丁,其父小乙使之久居民間,勞是稼穡,與小人出入同事。}作其即位,乃或亮陰,三年不言。\footnote{武丁起其即王位,則小乙死,乃有信默,三年不言。言孝行著。○行,下孟反。}其惟不言,言乃雍,不敢荒寧。\footnote{在喪則其惟不言,喪畢發言,則天下和。亦法中宗,不敢荒怠自安。}嘉靖殷邦,至於小大,無時或怨。\footnote{善謀殷國,至於小大之政,人無是有怨者。言無非。}肆高宗之享國,五十有九年。\footnote{高宗為政,小大無怨,故亦享國永年。}


{\noindent\zhuan\zihao{6}\fzbyks 傳“武丁其”至“同事”。正義曰:“舊”,久也。在即位之前,而言久勞於外,知是其父小乙使之久居民間,勞是稼穡,與小人出入同為農役,小人之艱難事也。太子使與小人同勞,此乃非常之事,不可以非常怪之。於時蓋未為太子也,殷道雖質,不可既為太子,更得與小人雜居也。 \par}

{\noindent\zhuan\zihao{6}\fzbyks 傳“武丁起”至“行著”。正義曰:以上言久勞於外,為父在時事,故言“起其即王位,則小乙死”也。“亮”,信也。“陰”,默也。三年不言,以舊無功,而今有,故言。乃有說此事者,言其孝行著也。\CJKunderwave{禮記·喪服四制}引\CJKunderwave{書}云:“‘高宗諒暗,三年不言。’善之也。王者莫不行此禮,何以獨善之也?曰,高宗者,武丁。武丁者,殷之賢王也。繼世即位,而慈良於喪。當此之時,殷衰而復興,禮廢而復起,故載之於\CJKunderwave{書}中而高之,故謂之高宗。”三年之喪,君不言也,是說此經“不言”之意也。 \par}

{\noindent\zhuan\zihao{6}\fzbyks 傳“在喪”至“自安”。正義曰:\CJKunderline{鄭玄}云:“其不言之時,時有所言,則群臣皆和諧。”\CJKunderline{鄭玄}意謂此“言乃雍”者,在三年之內,時有所言也。孔意則為出言在三年之外,故云“在喪其惟不言,喪畢發言,則天下大和”。知者,\CJKunderwave{說命}云:“王宅憂,亮陰三祀。既免喪,其惟不言。”除喪猶尚不言,在喪必無言矣,故知喪畢乃發言也。高宗不敢荒寧,與中宗正同,故云“亦法中宗,不敢荒怠自安”。殷家之王,皆是明王,所為善事,計應略同,但古文辭有差異,傳因其文同,故言“法中宗”也。 \par}

{\noindent\zhuan\zihao{6}\fzbyks 傳“善謀”至“無非”。正義曰:\CJKunderwave{釋詁}云:“嘉,善也。靖,謀也。”“善謀殷國”,謀為政教,故至於小大之政,皆允人意。人無是有怨高宗者,言其政無非也。鄭云:“小大謂萬人,上及群臣言。”人臣小大皆無怨王也。 \par}

{\noindent\shu\zihao{5}\fzkt “其在”至“九年”。正義曰:其殷王高宗,父在之時,久勞於外,於時與小人同其事。後為太子,起其即王之位,乃有信默,三年不言。在喪其惟不言,喪畢發言,言得其道,乃天下大和。不敢荒怠自安,善謀殷國,至於小大之政,莫不得所。其時之人,無是有怨恨之者。故高宗之享殷國五十有九年。亦言不逸得長壽也。 \par}

其在祖甲,不義惟王,舊為小人。\footnote{湯孫太甲,為王不義,久為小人之行,\CJKunderline{伊尹}放之桐。}作其即位,爰知小人之依,能保惠於庶民,不敢侮鰥寡。\footnote{在桐三年,思集用光,起就王位,於是知小人之所依。依仁政,故能安順於眾民,不敢侮慢惸獨。○惸,求營反,字又作𦬮。}肆祖甲之享國,三十有三年。\footnote{太甲亦以知小人之依,故得久年。此以德優劣、立年多少為先後,故祖甲在下。殷家亦祖其功,故稱祖。}


{\noindent\zhuan\zihao{6}\fzbyks 傳“湯孫”至“之桐”。正義曰:以文在“高宗”之下,世次顛倒,故特辨之,此祖甲是湯孫太甲也。“為王不義”,謂湯初崩。“久為小人之行,故\CJKunderline{伊尹}放之於桐”,言其廢而復興,為下“作其即位”起本也。王肅亦以祖甲為太甲。\CJKunderline{鄭玄}云:“祖甲,武丁子帝甲也。有兄祖庚賢,武丁欲廢兄立弟,祖甲以此為不義,逃於人間,故云久為小人。”案\CJKunderwave{殷本紀}云:“武丁崩,子祖庚立。祖庚崩,弟祖甲立,是為帝甲,淫亂,殷道復衰。”\CJKunderwave{國語}說殷事云:“帝甲亂之,七代而殞。”則帝甲是淫亂之主,起亡殷之源,寧當與二宗齊名,舉之以戒無逸?武丁賢王,祖庚復賢,以武丁之明,無容廢長立少。祖庚之賢,誰所傳說?武丁廢子,事出何書?妄造此語,是負武丁而誣祖甲也。 \par}

{\noindent\zhuan\zihao{6}\fzbyks 傳“在桐”至“惸獨”。正義曰:“在桐三年”,\CJKunderwave{太甲}序文。“思集用光”,\CJKunderwave{詩·大雅}文。彼“集”作“輯”,“輯”,和也。彼鄭言,公劉之遷豳,“思在和其民人,用光大其道”。此傳之意,蓋言太甲之在桐也,思得安集其身,用光顯王政,故起即王位,於是知小人之依。依於仁政,故能施行政教,安順於眾民,不敢侮慢。惸獨鰥寡之類,尤可憐愍,故特言之。 \par}

{\noindent\zhuan\zihao{6}\fzbyks 傳“太甲”至“稱祖”。正義曰:傳於中宗雲“以敬畏之故,得壽考之福”,“高宗之為政,小大無怨,故亦享國永年”,於此雲太甲,亦以知小人之依,故得久年。各順其文而為之說,其言行善而得長壽,經意三王同也。以其世次顛倒,故解之云,此以德優劣、立年多少為先後,故祖甲在太戊、武丁之下。諸書皆言“太甲”,此言“祖甲”者,殷家亦祖其功,故稱之“祖甲”。與二宗為類,惟見此篇,必言祖其功,亦未知其然。殷之先君有祖乙、祖辛、祖丁,稱祖多矣,或可號之為祖,未必祖其功而存其廟也。 \par}

{\noindent\shu\zihao{5}\fzkt “其在”至“三年”。正義曰:其在殷王祖甲,初遭祖喪,所言行不義。惟亦為王,久為小人之行,\CJKunderline{伊尹}廢諸桐。起其即王之位,於是知小人之所依。依於仁政,乃能安順於眾民,不敢侮鰥寡惸獨,故祖甲之享有殷國三十有三年。亦言不逸得長壽也。 \par}

自時厥後立王,生則逸。\footnote{從是三王,各承其後而立者,生則逸豫無度。}生則逸,不知稼穡之艱難,\footnote{言與小人之子同其敝。}不聞小人之勞,惟耽樂之從。\footnote{過樂謂之耽。惟樂之從,言荒淫。○耽,丁南反,注下同。樂音洛,注下同。}自時厥後,亦罔或克壽。\footnote{以耽樂之故,從是其後,亦無有能壽考。}或十年,或七八年,或五六年,或四三年。”\footnote{高者十年,下者三年,言逸樂之損壽。}

{\noindent\shu\zihao{5}\fzkt “自時”至“三年”。正義曰:從是三王其後所立之王,生則逸豫,不知稼穡之艱難,不聞小人之勞苦,惟耽樂之事則從而為之。故從是其後諸王,無有能壽考者。或十年,或七八年,或五六年,或四三年。言逸樂之損壽,故舉以戒成王也。 \par}

周公曰:“嗚呼!厥亦惟我周太王、王季,克自抑畏。\footnote{太王,周公曾祖。王季即祖。言皆能以義自抑,長敬天命。將說文王,故本其父祖。}文王卑服,即康功田功。\footnote{文王節儉,卑其衣服,以就其安人之功,以就田功,以知稼穡之艱難。○卑如字,馬本作俾,使也。}徽柔懿恭,懷保小民,惠鮮鰥寡。\footnote{以美道和民,故民懷之。以美政恭民,故民安之。又加惠鮮乏鰥寡之人。○鮮,息淺反,注同。}自朝至於日中昃,不遑暇食,用咸和萬民。\footnote{從朝至日昳不暇食,思慮政事,用皆和萬民。○昃音側,本亦作仄。昳,田節反。}


{\noindent\zhuan\zihao{6}\fzbyks 傳“大王”至“父祖”。正義曰:“大王,周公曾祖。王季即祖也”,此乃經傳明文,而須詳言之者,此二王之下辭無所結,陳此不為無逸,周公將說文王,故本其父祖,是以傳詳言也。解其言此之意。“以義自抑”者,言其非無此心,以義自抑而不為耳。 \par}

{\noindent\zhuan\zihao{6}\fzbyks 傳“文王”至“艱難”。正義曰:文王卑其衣服,以就安人之功,言儉於身而厚於人也。立君所以牧人,安人之功,諸有美政皆是也。就安人之內,田功最急,故特雲“田功”,以示知稼穡之艱難也。 \par}

{\noindent\zhuan\zihao{6}\fzbyks 傳“以美”至“之人”。正義曰:“徽”、“懿”皆訓為美,“徽柔懿恭”,此是施人之事,以匆厭恭懷安小民,故傳分而配之。“徽柔”配“懷”,“以美道和民,故民懷之”。“懿恭”配“保”,“以美政恭民,故民安之”,“徽懿”言其美而已,不知何所美也。人君施於民,惟有道與政耳,故傳以“美道”、“美政”言之,政與道亦互相通也。少乏鰥寡尢是可憐,故別言“加惠於鮮乏鰥寡之人”也。 \par}

{\noindent\zhuan\zihao{6}\fzbyks 傳“從朝”至“萬民”。正義曰:昭五年\CJKunderwave{左傳}云:“日上其中,食日為二,旦日為三。”則人之常食在日中之前,謂辰時也。\CJKunderwave{易·豐卦}彖曰:“日中則昃。”謂過中而斜昃也。“昃”亦名“昳”,言日蹉跌而下,謂未時也。故日之十位,食時為辰,日昳為未。言文王勤於政事,從朝不食,或至於日中,或至於日昃,猶不暇食。故經“中”、“昃”並言之。傳舉晚時,故惟言“昳”。“遑”亦“暇”也,重言之者,古人自有復語,猶雲“艱難”也。所以不暇食者,為思慮政事,用皆和萬民。政事雖多,皆是為民,故言“咸”。“咸”訓皆也。 \par}

文王不敢盤於遊田,以庶邦惟正之供。\footnote{文王不敢樂於遊逸田獵,以眾國所取法則,當以正道供待之故。○供音恭。}文王受命惟中身,厥享國五十年。”\footnote{文王九十七而終。中身,即位時年四十七。言中身,舉全數。}

{\noindent\zhuan\zihao{6}\fzbyks 傳“文王”至“之故”。正義曰:\CJKunderwave{釋詁}云:“盤,樂也。”“遊”謂遊逸,“田”謂畋獵,二者不同,故並雲“遊逸田獵”。以眾國皆於文王所取其法則,文王當以正義供待之故也。言文王思為政道以待眾國,故不敢樂於遊田。文王世為西伯,故當為眾國所取法則。禮有田獵而不敢者,順時搜狩,不為取樂,故不敢非時畋獵以為樂耳。 \par}

{\noindent\zhuan\zihao{6}\fzbyks 傳“文王”至“全數”。正義曰:“文王年九十七而終”,\CJKunderwave{禮記·文王世子}文也。於九十七內減享國五十年,是未立之前有四十七。在\CJKunderwave{禮}諸侯逾年即位,此據代父之年,故為“即位時年四十七”也。計九十七年半折以為中身,則四十七時於身非中,言“中身”者,舉全數而稱之也。經言“受命”者,\CJKunderline{鄭玄}雲“受殷王嗣位之命”。然殷之末世,政教已衰,諸侯嗣位何必皆待王命?受先君之命亦可也。王肅云:“文王受命,嗣位為君。”不言受王命也。 \par}

{\noindent\shu\zihao{5}\fzkt “周公”至“十年”。正義曰:殷之三王既如此矣,周公又言曰:“嗚呼!其惟我周家大王、王季,能以義自抑而畏敬天命,故王跡從此起也。文王又卑薄衣服,以就其安人之功與治田之功。以美道柔和其民,以美政恭待其民,以此民歸之。以美政恭民之故,故小民安之,又加恩惠於鮮乏鰥寡之人。其行之也,自朝旦至於日中及昃,尚不遑暇食,用善政以諧和萬民故也。文王專心於政,不敢逸樂於遊戲畋獵,以己為眾國所取法,惟當正身行己以供待之。由是文王受命,嗣位為君,惟於中身受之,其享國五十年,亦以不逸得長壽也。” \par}

周公曰:“嗚呼!繼自今嗣王,\footnote{繼從今已往嗣世之王,皆戒之。}則其無淫於觀、於逸、於遊、于田,以萬民惟正之供。\footnote{所以無敢過於觀遊逸豫田獵者,用萬民當惟正身以供待之故。}無皇曰:‘今日耽樂。’乃非民攸訓,非天攸若,時人丕則有愆。\footnote{無敢自暇曰:“惟今日樂,後日止。”夫耽樂者,乃非所以教民,非所以順天,是人則大有過矣。○愆,起虔反。夫音扶。}無若殷王受之迷亂,酗於酒德哉!”\footnote{以酒為兇謂之酗。言紂心迷政亂,以酗酒為德。戒嗣王無如之。○酗,況付反。}


{\noindent\zhuan\zihao{6}\fzbyks 傳“繼從”至“戒之”。正義曰:先言“繼”者,謂繼此後人,即從今以後嗣世之王也。周公思及長遠后王,盡皆戒之,非獨成王也。 \par}

{\noindent\zhuan\zihao{6}\fzbyks 傳“所以”至“之故”。正義曰:傳意訓“淫”為過,\CJKunderline{鄭玄}云:“淫,放恣也。”“淫”者侵淫不止,其言雖殊,皆是過之義也。言“觀”為非時而行,違禮觀物,如\CJKunderwave{春秋}隱公“如棠觀魚”,莊公“如齊觀社”。\CJKunderwave{穀梁傳}曰:“常事曰視,非常曰觀。”此言“無淫於觀”,禁其非常觀也。“逸”謂逸豫,“遊”謂遊蕩,“田”謂田獵,四者皆異,故每事言“於”。“以”訓用也,用萬民皆聽王命,王者惟當正身待之,故不得淫於觀逸遊田也。 \par}

{\noindent\zhuan\zihao{6}\fzbyks 傳“無敢”至“過矣”。正義曰:“無敢自暇”,謂事不寬不暇,而以為原王之意而為辭,故言曰:“耽以為樂,惟今日樂,而後日止。”惟言“今日樂”,明知“後日止”也。夫“耽樂”者,乃非所以教民,教民當恪勤也;非所以順天,順天當肅恭也。是此耽樂之人,則大有愆過矣。戒王不得如此也。 \par}

{\noindent\zhuan\zihao{6}\fzbyks 傳“以酒”至“如之”。正義曰:“酗”從酉,以兇為聲,是“酗”為兇酒之名,故“以酒為兇謂之酗”。“酗”是飲酒而益兇也。言紂心迷亂,以酗酒為德,飲酒為政,心以兇酒為己德,紂以此亡殷。戒嗣王無如之。 \par}

{\noindent\shu\zihao{5}\fzkt “周公”至“德哉”。正義曰:周公又言而嘆曰:“嗚呼!繼此後世自今以後嗣位之王,則其無得過於觀望,過於逸豫,過於遊戲,過於田獵。所以不得然者,以萬民聽王者之教命,王當正己身以供待萬民,必當早夜恪勤,無敢自閒暇。曰:‘今日且樂,後日乃止。’此為耽樂者,非民之所以教訓也,非天之所以敬順也。若是之人,則有大愆過矣。王當自勤政事,莫如殷王受之述亂國政,酗醟於酒德哉!殷紂藉酒為兇,以酒為德,由是喪亡殷國,王當以紂為戒,無得如之。” \par}

周公曰:“嗚呼!我聞曰,古之人,猶胥訓告,胥保惠,胥教誨,\footnote{嘆古之君臣,雖君明臣良,猶相道告,相安順,相教誨以義方。}民無或胥譸張為幻。\footnote{譸張,誑也。君臣以道相正,故下民無有相欺誑幻惑也。○幻音患。誑,九況反。}此厥不聽,人乃訓之,乃變亂先王之正刑,至於小大。\footnote{此其不聽中正之君,人乃教之以非法,乃變亂先王之正法,至於小大,無不變亂。言己有以致之。}民否則厥心違怨,否則厥口詛祝。”\footnote{以君變亂正法,故民否則其心違怨,否則其口詛祝。言皆患其上。○詛,側助反。祝,之又反。}


{\noindent\zhuan\zihao{6}\fzbyks 傳“嘆古”至“義方”。正義曰:此章二事,善惡相反。下句“不聽人”者,是愚闇之君,知此言“古之人”者,是賢明之君。“相”是兩人相與,故知兼有“臣良”,更相教告。隱三年\CJKunderwave{左傳}石碏曰:“臣聞愛子,教之以義方。”故知相教誨者,使“相教誨以義方”也。則知相訓告者,告之以善道也;相保惠者,相安順以美政也。 \par}

{\noindent\zhuan\zihao{6}\fzbyks 傳“譸張”至“惑也”。正義曰:“譸張,誑也”,\CJKunderwave{釋訓}文。孫炎曰:“眩惑誑欺人也。”民之從上,若影之隨形,君臣以道相正,故下民無有相欺誑幻惑者。“幻”即眩也,惑亂之名,\CJKunderwave{漢書}稱西域有幻人是也。 \par}

{\noindent\zhuan\zihao{6}\fzbyks 傳“此其”至“致之”。正義曰:上言善事,此說惡事。如此其不聽者,是不聽中正之君也。既不聽中正,則好聽邪佞,知此“乃訓之”者,是邪佞之人訓之也。邪佞之人必反正道,故言“人乃教之以非法”。暗君即受用之,變亂先王之正法。“至於小大,無不變亂”,言皆變亂正法盡也。暗君所任同己,由已之暗,致此佞人,言此暗君已身有以致之也。上“君明臣良”,由君明而有良臣,亦是己有致。上之言“胥”,此不言者,君在佞臣,國亡滅矣,不待相教為惡,故不言“胥”也。 \par}

{\noindent\zhuan\zihao{6}\fzbyks 傳“以君”至“其土”。正義曰:君既變亂正法,必將困苦下民。民不堪命,忿恨必起,故民忿君乃有二事,否則心違怨,否則口詛祝,言皆患土而為此也。“違怨”,謂違其命而怨其身。“詛祝”,謂告神明令加殃咎也。以言告神謂之“祝”,請神加殃謂之“詛”。襄十七年\CJKunderwave{左傳}曰:“宋國區區,而有詛有祝。”\CJKunderwave{詩}曰:“侯詛侯祝。”是“詛”、“祝”意小異耳。 \par}

{\noindent\shu\zihao{5}\fzkt “周公”至“詛祝”。正義曰:周公言而嘆曰:“我聞人之言曰,古人之雖君明臣良,猶尚相訓告以善道,相安順以美政,相教誨以義方。君臣相正如此,故於時之民順從上教,無有相誑欺為幻惑者。此其不聽中正之君,人乃教訓之以非法之事,乃從其言,變亂先王之正法,至於小大之事,無不皆變亂之。君既變亂如此,其時之民疾苦,否則其心違上怨上,否則其口詛祝之。”言人患之無已,舉此以戒成王,使之君臣相與養下民也。 \par}

周公曰:“嗚呼!自殷王中宗,及高宗,及祖甲,及我周文王,茲四人迪哲。\footnote{言此四人皆蹈智明德以臨下。}厥或告之曰:‘小人怨汝詈汝。’則皇自敬德,\footnote{其有告之,言小人怨詈汝者,則大自敬德,增修善政。○詈,力智反。}厥愆,曰:‘朕之愆。’允若時不啻不敢含怒。\footnote{其人有禍,則曰:“我過,百姓有過,在予一人。”信如是怨詈,則四王不啻不敢含怒以罪之。言常和悅。}


{\noindent\zhuan\zihao{6}\fzbyks 傳“其有”至“善政”。正義曰:\CJKunderwave{釋詁}云:“皇,大也。”故傳言,“大自敬德者,謂增修善政”也。\CJKunderline{鄭玄}以“皇”為暇,言寬暇自敬。王肅本“皇”作“況”,況滋益用敬德也。 \par}

{\noindent\zhuan\zihao{6}\fzbyks 傳“其人”至“和悅”。正義曰:或告之曰“小人怨汝,詈汝”,其言有虛有實。其言若虛,則民之愆也。民有愆過,則曰“我過”,不責彼為虛言,而引過歸己者,湯所云“百姓有過,在予一人”。故若信有如是怨詈,小人聞之,則含怒以罪彼人。此四王即不啻不敢含怒以罪彼人,乃自原聞其愆言,其顏色常和悅也。\CJKunderline{鄭玄}云:“不但不敢含怒,乃欲屢聞之,以知己政得失之源也。” \par}

{\noindent\shu\zihao{5}\fzkt “周公”至“含怒”。正義曰:既言明君暗君,善惡相反,更述二者之行。周公言而嘆曰:“嗚呼!自殷王中宗,及高宗,及祖甲,及我周文王,此四人者,皆蹈明智之道以臨下民。其有告之曰:‘小人怨恨汝,罵詈汝。’既聞此言,則大自敬德,更增修善政。其民有過,則曰:‘是我之過。’民信有如是怨詈,則不啻不敢含怒以罪彼人,乃欲得數聞此言以自改悔。”言寬弘之若是。 \par}

此厥不聽,人乃或譸張為幻,曰:‘小人怨汝詈汝。’則信之。\footnote{此其不聽中正之君,有人誑惑之,言小人怨憾詛詈汝,則信受之。○憾,胡暗反。}則若時,不永念厥闢,不寬綽厥心,\footnote{則如是信讒者,不長念其為君之道,不寬緩其心。言含怒。}亂罰無罪,殺無辜,怨有同,是叢於厥身。”\footnote{信讒含怒,罰殺無罪,則天下同怨讎之,叢聚於其身。○叢,才公反。}


{\noindent\zhuan\zihao{6}\fzbyks 傳“則如”至“含怒”。正義曰:君人者察獄必審其虛實,然後加罪。“不長念其為君之道”,謂不審察虛實也。“不寬緩其心”,言徑即含怒也。王肅讀“闢”為闢,扶亦反,不長念其刑辟,不當加無罪也。 \par}

{\noindent\shu\zihao{5}\fzkt “此厥”至“厥身”。正義曰:此其不聽中正之人,乃有欺誑為幻惑以告之曰:“小人怨汝詈汝。”不原其本情,則信受之。則知是信讒者,不長念其為君之道,不審虛實,不能寬緩其心,而徑即含怒於人。是亂其正法,罰無罪,殺無辜。罰殺欲以止怨,乃令人怨益甚,天下之民有同怨君,令怨惡聚於其身。言褊急使民之怨若是,教成王勿學此也。 \par}

周公曰:“嗚呼!嗣王其監於茲。”\footnote{視此亂罰之禍以為戒。}

\section{君奭第十八}


召公為保,周公為師,相成王為左右。\footnote{保,太保也。師,太師也。馬云:“保氏、師氏皆大夫官。”相音息亮反。左右,馬云:“分陝為二伯,東為左,西為右。”}召公不說,周公作\CJKunderwave{君奭}。

君奭\footnote{尊之曰君。奭,名,同姓也。陳古以告之,故以名篇。○說音悅。奭,始亦反。}


{\noindent\zhuan\zihao{6}\fzbyks 傳“尊之”至“名篇”。正義曰:周公呼為“君奭”,是周公尊之曰君也。“奭”是其名,“君”非名也。僖二十四年\CJKunderwave{左傳},富辰言文王之子一十六國,無名“奭”者,則召公必非文王之子。\CJKunderwave{燕世家}云:“召公奭與周同姓姬氏。”譙周曰:“周之支族。”譙周考校古史,不能知其所出。皇甫謐云:“原公名豐,是其一也,是為文王之子一十六國。”然文王之子本無定數,並原、豐為一,當召公於中以為十六,謬矣。此篇多言先世有大臣輔政,是“陳古道以告之”。呼居奭以告之,故以“君奭”名篇。 \par}

{\noindent\shu\zihao{5}\fzkt “召公”至“君奭”。正義曰:成王即政之初,召公為保,周公為師,輔相成王為左右大臣。召公以周公嘗攝王之政,今覆在臣位,其意不說。周公陳己意以告召公,史敘其事,作\CJKunderwave{君奭}之篇也。\CJKunderwave{周官}篇雲“立太師、太傅、太保,茲惟三公”,則此“為保”、“為師”亦為三公官也。此實太師、太保而不言“太”者,意在師法保安王身,言其實為左右爾,不為舉其官名,故不言“太”也。經傳皆言武王之時,太公為太師,此言“周公為師”,蓋太公薨,命周公代之。於時太傅蓋畢公為之,於此無事,不須見也。三公之次,先師後保,此序先言保者,篇之所作,主為召公不說,故先言召公,不以官位為次也。案經周公之言,皆說己留在王朝之意,則召公不說周公之留也。故鄭、王皆云:“周公既攝王政,不宜複列於臣職,故不說。”然則召公大賢,豈不知周公留意而不說者?以周公留在臣職,當時人皆怪之,故欲開道周公之言,以解世人之惑。“召公疑之,作\CJKunderwave{君奭}。”非不知也。\CJKunderwave{史記·燕世家}云:“成王既幻,周公攝政,當國踐阼,召公疑之,作\CJKunderwave{君奭}。”此篇是致政之後言留輔成王之意,其文甚明,馬遷妄為說爾。\CJKunderline{鄭玄}不見\CJKunderwave{周官}之篇,言此師、保為\CJKunderwave{周禮}師氏、保氏大夫之職,言賢聖兼此官,亦謬矣。 \par}

周公若曰:“君奭,\footnote{順古道呼其名而告之。}弗吊,天降喪於殷,殷既墜厥命,我有周既受。\footnote{言殷道不至,故天下喪亡於殷。殷已墜失其王命,我有周道至已受之。○吊音的。}我不敢知曰,厥基永孚於休,若天棐忱。\footnote{廢興之跡,亦君所知,言殷家其始長信於美道,順天輔誠,所以國也。○棐音匪。忱,市林反。}我亦不敢知曰,其終出於不祥。\footnote{言殷紂其終墜厥命,以出於不善之故,亦君所知。}


{\noindent\zhuan\zihao{6}\fzbyks 傳“廢興”至“以國”。正義曰:孔以\CJKunderwave{召誥}雲“我不敢知”者,其意召公言我不敢獨知,亦王所知,則此言“我不敢知”,亦是周公言我不敢獨知,是君奭所知,故以此及下句為說殷之興亡,言與君奭同知。舉其殷興亡為戒,\CJKunderline{鄭玄}亦然也。 \par}

{\noindent\shu\zihao{5}\fzkt “周公”至“不祥”。正義曰:周公留在王朝,召公不說。周公為師,順古道而呼曰:“君奭,殷道以不至之故,故天下喪亡於殷。殷既墜失其王命,我有周已受之矣。今雖受命,貴在能終,若不能終,與殷無異,故視殷以為監戒。我不敢獨知殷家其初始之時,能長信於美道,能安順於上天之,道輔其誠信,所以有國,此亦君之所知。我亦不敢獨知曰,殷紂其終墜失其王命,由出於不善之故,亦君所知也。” \par}

嗚呼!君已!曰,時我,我亦不敢寧於上帝命。\footnote{嘆而言曰:“君已!當是我之留,我亦不敢安於上天之命,故不敢不留。”○已音以。}弗永遠念天威,越我民罔尤違。\footnote{言君不長遠念天之威,而勤化於我民,使無過違之闕。}惟人在我後嗣子孫,大弗克恭上下,遏佚前人光,在家不知。\footnote{惟眾人共存在我後嗣子孫,若大不能恭承天地,絕失先王光大之道,我老在家,則不得知。○遏,於葛反。}天命不易,天難諶,乃其墜命,弗克經歷。\footnote{天命不易,天難信無德者,乃其墜失王命,不能經久歷遠,不可不慎。○易,以豉反,注同。諶,氏壬反。}嗣前人,恭明德,在今予小子旦。\footnote{繼先王之大業,恭奉其明德,正在今我小子旦。言異於餘臣。}非克有正,迪惟前人光,施於我衝子。”\footnote{我留非能有改正,但欲蹈行先王光大之道,施正於我童子。童子,成王。}


{\noindent\zhuan\zihao{6}\fzbyks 傳“嘆而”至“不留”。正義曰:嘆而言曰:“嗚呼!君已!”“已”是引聲之辭,既呼君奭,嘆而引聲,乃復言曰:“君當是我之留。”以其意不說,故令是我而勿非我。“我不敢安於上天之命”,孔意當謂天既命周,我當成就周道,故不敢不留。 \par}

{\noindent\shu\zihao{5}\fzkt “嗚呼”至“衝子”。正義曰:周公又嘆而呼召公曰:“嗚呼!君已!”“已”,辭也。既嘆乃復言曰:“君當是我之留,勿非我也。我亦不敢安於上天之命,故不敢不留。君何不長遠念天之威罰?禍福難量,當勤教於我下民,使無尤過違法之闕。惟今天下眾人,共誠心存在我後嗣子孫。觀其政之善惡,若此嗣王大不能恭承上天下地,絕失先王光大之道,令使眾人失望,我若退老在家,則不能得知,何得不留輔王也?天命不易,言甚難也。天難信,惡則去之,不常在一家,是難信也。天子若不稱天意,乃墜失其王命,不能經久歷遠,其事可不慎乎?繼嗣前人先王之大業,恭奉其明德也,正在今我小子旦。”周公自言已身當恭奉其先王之明德,留輔佐王。“非能有所改正,但欲蹈行先王光大之道,施政於我童子”。童子謂成王,意欲奉行先王之事,以教成王也。 \par}

又曰:“天不可信,我道惟寧王德延,\footnote{無德去之,是天不可信,故我以道惟安寧王之德,謀欲延久。}天不庸釋於文王受命。”\footnote{言天不用令釋廢於文王所受命,故我留佐成王。}


{\noindent\zhuan\zihao{6}\fzbyks 傳“無德”至“延久”。正義曰:此經言“又曰”,傳不明解。鄭雲“人又云”,則\CJKunderline{鄭玄}以此“又曰”為周公稱人之言也。王肅云:“重言天不可信,明己之留蓋畏其天命。”則肅意以周公重言,故稱“又曰”。孔雖不解,當與王肅意同。言“寧王”者,即文王也,鄭、王亦同。 \par}

{\noindent\shu\zihao{5}\fzkt “又曰”至“受命”。正義曰:周公又言曰:“天不可信,無德則去之,是其不可信也。天難信之,故恐其去我周家,故我以道惟安行寧王之德,謀欲延長之。我原上天之意,不用令廢於文王所受命,若嗣王失德,則還廢之,故我當留佐成王也。” \par}

公曰:“君奭,我聞在昔\CJKunderline{成湯}既受命,\footnote{已放桀,受命為天子。}時則有若\CJKunderline{伊尹},格於皇天。\footnote{尹摯佐湯,功至大天。謂致太平。○摯音至。}在太甲,時則有若保衡。\footnote{太甲繼湯,時則有如此\CJKunderline{伊尹}為保衡,言天下所取安,所取平。}在太戊,\footnote{太甲之孫。}時則有若伊陟、臣扈,格於上帝,巫咸乂王家。\footnote{伊陟、臣扈率\CJKunderline{伊尹}之職,使其君不隕祖業,故至天之功不隕。巫咸治王家,言不及二臣。○隕,于敏反。}


{\noindent\zhuan\zihao{6}\fzbyks 傳“尹摯”至“太平”。正義曰:\CJKunderline{伊尹}名摯,諸子傳記名有其文。“功至大天”猶堯“格於上下”,知其“謂致太平”也。 \par}

{\noindent\zhuan\zihao{6}\fzbyks 傳“太甲”至“取平”。正義曰:據\CJKunderwave{太甲}之篇及諸子傳記,太甲犬臣惟有\CJKunderline{伊尹},知即保衡也。\CJKunderwave{說命}云:“昔先正保衡,作我先王,佑我烈祖,格於皇天。”\CJKunderwave{商頌·那}祀\CJKunderline{成湯}稱為“烈祖”,“烈祖”,湯之號,言保衡佐湯,明保衡即是\CJKunderline{伊尹}也。\CJKunderwave{詩}稱“實維阿衡,實左右商王”,\CJKunderline{鄭玄}云:“阿,倚。衡,平也。\CJKunderline{伊尹}湯所依倚而取平。”至太甲改曰保衡,保,安也,言天下所取安,所取平。此皆三公之官,當時為之號也。孔以\CJKunderwave{太甲}雲“嗣王不惠於阿衡”,則\CJKunderwave{太甲}亦曰阿衡,與鄭異也。 \par}

{\noindent\zhuan\zihao{6}\fzbyks 傳“太甲之孫”。正義曰:\CJKunderwave{史記·殷本紀}云,太甲崩,子沃丁立。崩,弟太庚立。崩,子小甲立。崩,弟雍已立。崩,弟太戊立。是太戊為太甲之孫,太庚之子。\CJKunderwave{三代表}云,小甲,太庚弟;雍己、太戊又是小甲弟,則太戊亦是沃丁弟,太甲子。\CJKunderwave{本紀}、\CJKunderwave{世表}俱出馬遷,必有一誤。孔於\CJKunderwave{咸乂}序傳雲“太戊,沃丁弟之子”,是太戊為太甲之孫也。 \par}

{\noindent\zhuan\zihao{6}\fzbyks 傳“伊陟”至“二臣”。正義曰:\CJKunderline{伊尹}“格於皇天”,此伊陟、臣扈雲“格於上帝”,其事既同,如此二臣能率循\CJKunderline{伊尹}之職,輔佐其君,使其君不隕祖業,故至天之功亦不隕墜也。\CJKunderwave{夏社}序云:“湯既勝夏,欲遷其社,不可。作\CJKunderwave{夏社}、\CJKunderwave{疑至}、\CJKunderwave{臣扈}。”則湯初有臣扈,已為大臣矣,不得至今仍在,與\CJKunderline{伊尹}之子同時立功。蓋二人名同,或兩字一誤也。案\CJKunderwave{春秋}範武子光輔五君,或臣扈事湯而又事太戊也。“格於上帝”之下乃言“巫咸乂王家”,則巫咸亦是賢臣,俱能紹治王家之事而已,其功不得至天,言不及彼二臣。 \par}

在祖乙,時則有若巫賢。\footnote{祖乙,殷家亦祖其功,時賢臣有如此巫賢。賢,咸子。巫,氏。}在武丁,時則有若甘盤。\footnote{高宗即位,甘盤佐之,後有傅說。○說音悅。}

{\noindent\zhuan\zihao{6}\fzbyks 傳“祖乙”至“巫氏”。正義曰:\CJKunderwave{殷本紀}云,中宗崩,子仲丁立。崩,弟外壬立。崩,弟河亶甲立。崩,子祖乙立。則祖乙是太戊之孫也。孔以其人稱“祖”,故云“殷家亦祖其功”。賢是咸子,相傳云然。父子俱稱為“巫”,知“巫”為氏也。 \par}

{\noindent\zhuan\zihao{6}\fzbyks 傳“高宗”至“傅說”。正義曰:\CJKunderwave{孔命}篇高宗云:“臺小子舊學於甘盤,既乃遯於荒野。”高宗未立之前已有甘盤,免喪不言,乃求傅說,明其即位之初,有甘盤佐之,甘盤卒後有傅說。計傅說當有大功,此惟數六人,不言傅說者,周公意所不言,未知其故。 \par}

{\noindent\shu\zihao{5}\fzkt “公曰君奭”至“甘盤”。正義曰:言“時則有若”者,言當其時有如此人也。指謂如此\CJKunderline{伊尹}、甘盤,非謂別有如此人也。以湯是殷之始王,故言“在昔”、“既受命”,見其為天子也。以下“在太甲”、“在武丁”,亦言其為天子之時,有如此臣也。\CJKunderline{成湯}未為天子,已得\CJKunderline{伊尹},言“既受命”者,以功格皇天,在受命之後,故言“既受命”也。“皇天”之與“上帝”,俱是天也,變其文爾。其功至於天帝,謂致太平而天下和之也。保衡、\CJKunderline{伊尹},一人也。異時而別號。“\CJKunderline{伊尹}”之下,已言“格於皇天”,“保衡”之下不言“格於皇天”,從可知也。“伊陟、臣扈”,言“格於上帝”,則其時亦致太平,故與\CJKunderline{伊尹}文異而事同。巫咸、巫賢、甘盤蓋功劣於彼三人,故無格天之言。 \par}

率惟茲有陳,保乂有殷,故殷禮陟配天,多歷年所。\footnote{言\CJKunderline{伊尹}至甘盤六臣佐其君,循惟此道,有陳列之功,以安治有殷,故殷禮能升配天,享國久長,多歷年所。}天惟純佑命,則商實百姓。\footnote{殷禮配天,惟天大佑助其王命,使商家百姓豐實,皆知禮節。}


{\noindent\zhuan\zihao{6}\fzbyks 傳“言伊”至“年所”。正義曰:“率”訓循也。說賢臣佐君雲“循惟此道”,當謂循此為臣之道。盡忠竭力以輔其君,故有陳烈於世,以安治有殷,使殷王得安治民。故殷得此安上治民之禮,能升配上天。天在人上,故謂之“升”。為天之子,是“配天”也。享國久長,多歷年所。 \par}

{\noindent\zhuan\zihao{6}\fzbyks 傳“殷禮”至“禮節”。正義曰:殷能以禮配天,故天降福。天惟大佑助其王命,風雨以時,年穀豐稔,使商家百姓豐實,家給人足。管子曰:“衣食足,知榮辱。倉廩實,知禮節。” \par}

{\noindent\shu\zihao{5}\fzkt “率惟”至“百姓”。正義曰:此\CJKunderline{伊尹}、甘盤六臣等輔佐其君,率循此為臣之道,有陳列之功,以安治有殷,故殷有安上治民之禮,升配上天,享國多歷年之次所。天惟大佑助其為王之命,則使商家富實百姓,為令使商之百姓家給人足,皆知禮節也。 \par}

王人罔不秉德,明恤小臣,屏侯甸。\footnote{自湯至武丁,其王人無不持德立業,明憂其小臣,使得其人,以為蕃屏侯甸之服。小臣且憂得人,則大臣可知。○屏,賓領反。}矧咸奔走,惟茲惟德稱,用又厥闢。\footnote{王猶秉德憂臣,況臣下得不皆奔走?惟王此事,惟有德者舉,用治其君事。○闢,必亦反。}故一人有事於四方,若卜筮,罔不是孚。”\footnote{一人,天子也。君臣務德,故有事於四方,而天下化服。如卜筮,無不是而信之。}


{\noindent\zhuan\zihao{6}\fzbyks 傳“自湯”至“可知”。正義曰:王肅云:“王人猶君人也。”“無不持德立業”,謂持人君之德,立王者之事業。人君之德在官賢人,官得其人,則事業立,故傳以“立業”配“持德”。明憂小臣之不賢,憂欲使得其人,以為蕃屏侯甸之服也。小臣且憂得人,則大臣憂之可知。侯甸尚思得其人,朝廷思之必矣。王肅云:“小臣,巨之微者,舉小以明大也。” \par}

{\noindent\zhuan\zihao{6}\fzbyks 傳“王猶”至“君事”。正義曰:君之所重,莫重於求賢。官之所急,莫急於得人。故此章所陳,惟言君憂得人,臣能舉賢。以王之尊,猶尚秉德憂臣,況其臣下得不皆奔走?惟王此求賢之事,惟有德者必舉之,置於官位用治其君事也。 \par}

{\noindent\zhuan\zihao{6}\fzbyks 傳“一人”至“信之”。正義曰:\CJKunderwave{禮}天子自稱曰予一人,故為天子也。君臣務求有德,眾官得其人,從上至下,遞相師法,職無大小,莫不治理,故天子有事於四方,發號出令而天下化服。譬如卜筮,無不是而信之。事既有驗,言如是則人皆信之。 \par}

{\noindent\shu\zihao{5}\fzkt “王人”至“是孚”。正義曰:“王人”謂與人為王,言此上所說\CJKunderline{成湯}、太甲、太戊、祖乙、武丁,皆王人也。無不持德立業,明憂小臣。雖則小臣,亦憂使得其賢人,以蕃屏侯甸之服。王恐臣之不賢,尚以為憂,況在臣下得不皆勤勞奔走,惟憂王此求賢之事,惟求有德者舉之,用治其君之事乎?君臣共求其有德,所在職事皆治,天子一人有事於四方,天下咸化而服。如有卜筮之驗,無不是而信之。賢臣助君,致使大治,我留不去,亦當如此也。 \par}

公曰:“君奭,天壽平格,保乂有殷,有殷嗣,天滅威。\footnote{言天壽有平至之君,故安治有殷。有殷嗣子紂,不能平至,天滅亡,加之有威。}今汝永念,則有固命,厥亂明我新造邦。”\footnote{今汝長念平至者安治,反是者滅亡。以為法戒,則有堅固王命,其治理足以明我新成國矣”。}


{\noindent\zhuan\zihao{6}\fzbyks 傳“言天”至“以威”。正義曰:“格”訓至也。“平”謂政教均平,“至”謂道有所至也。言“不弔”,謂道有不至者。此言“格”,謂道至者。“天壽有平至之君”,有平至之德,則天與之長壽,則知中宗高宗之屬身是也。由其君有平至之德,故能安治有殷,言有殷國安而民治也。有殷嗣子紂,其德不能平至,國不安,民不治,故天滅亡之而加之以威也。孔傳之意,此經專說君之善惡,其言不及臣也。王肅以為兼言君臣,注云:“殷君臣之有德,故安治有殷。言是者,不可不法殷家有良臣也。”鄭注以為專言臣事,“格”謂至於天也。與孔不同。 \par}

{\noindent\zhuan\zihao{6}\fzbyks 傳“今汝”至“國矣”。正義曰:上句言善者興而惡者亡,此句令其長安治及念明道。念上二者,故言“今汝長念平至者而安治,反是者滅亡”。念此以為法戒,則有堅固王命,王族必不傾壞。若能如此,其治理足以光明我新成國矣。周自武王伐紂,至此年歲末多,對殷而言故為新國。傳意言不及臣,周公說此事者,蓋言興滅由人,我欲輔王,使為平至之君。 \par}

{\noindent\shu\zihao{5}\fzkt “公曰君奭天”至“造邦”。正義曰:周公呼召公曰:“君奭,皇天賦命,壽此有平至之君。”言有德者必壽考也。“殷之先王有平至之德,故能安治有殷”。言故得安治也。“有殷嗣子紂不能平至,故天滅亡而加之以威。今汝奭當長念天道,平至者安治,不平至者滅亡。以此為法戒,則有堅固王命,其治理足以明我新成國矣”。 \par}

公曰:“君奭,在昔上帝,割申勸寧王之德,其集大命於厥躬。\footnote{在昔上天,割制其義,重勸文王之德,故能成其大命於其身。謂勤德以受命。○重,直用反。}惟文王尚克修和我有夏,亦惟有若虢叔,有若閎夭,\footnote{文王庶幾能修政化,以和我所有諸夏,亦惟賢臣之助為治,有如此虢、閎。閎,氏。虢,國;叔,字;文王弟。夭,名。○虢,寡白反,徐公伯反。閎音宏。夭,於表反,徐於驕反。}有若散宜生,有若泰顛,有若南宮括。\footnote{散、泰、南宮皆氏。宜生、顛、括皆名。凡五臣佐文王為胥附、奔走、先後、禦侮之任。}


{\noindent\zhuan\zihao{6}\fzbyks 傳“在昔”至“受命”。正義曰:文王去此未久,但欲遠本天意,故云“在昔上天”,作久遠言之。“割制”謂切割絕斷之意,故云“割制其義”。“重勸文王之德”者,文王既已有德,上天佑助而重勸勉,文王順天之意,故其能成大命於其身。王謂勤行德義,以受天命。 \par}

{\noindent\zhuan\zihao{6}\fzbyks 傳“文王”至“夭名”。正義曰:文王未定天下,庶幾能修政化,以和我所有諸夏,謂三分有二屬己之諸國也。僖五年\CJKunderwave{左傳}雲“虢仲、虢叔,王季之穆也”,是虢叔為文王之弟。虢,國名。叔,字。凡言人之名氏,皆上氏下名,故閎、散、泰、南宮皆氏,夭、宜生、顛、括皆名也。 \par}

{\noindent\zhuan\zihao{6}\fzbyks 傳“散泰”至“之任”。正義曰:\CJKunderwave{詩·綿}之卒章稱文王有疏附、先後、奔奏、禦侮之臣,\CJKunderwave{毛傳}云:“率下親上曰疏附,相通前後曰先後,喻德宣譽曰奔奏,武臣折衝曰禦侮。”鄭箋云:“疏附使疏者親也,奔奏使人歸趨之。”\CJKunderwave{詩}言文王有此四種之臣,經歷言五臣之名,故知五臣佐文王為此任也。此四事者五臣共為此任,非一臣當一事也。鄭云:“不及呂望者,太師致文王以大德,周公謙不可以自比。” \par}

{\noindent\shu\zihao{5}\fzkt “公曰君奭”至“厥躬”。正義曰:公呼召公曰:“君奭,在昔上天斷割其義,重勸文王之德。以文王有德,勸勉使之成功,故文王能成之命於其身。”言文王能順天之意,勤德以受命。 \par}

又曰,無能往來,茲迪彝教文王蔑德,降於國人。\footnote{有五賢臣,猶曰其少,無所能往來。而五人以此道法教文王以精微之德,下政令於國人。言雖聖人,亦須良佐。}亦惟純佑,秉德迪知天威,乃惟時昭文王。\footnote{文王亦如殷家惟天所大佑,文王亦秉德蹈知天威,乃惟是五人明文王之德。}迪見冒聞於上帝,惟時受有殷命哉!\footnote{言能明文王德,蹈行顯見,覆冒下民,彰聞上天,惟是故受有殷之王命。○見,賢遍反,注同。冒,莫報反,下同,馬作勖,勉也。聞音問,或如字。}


{\noindent\zhuan\zihao{6}\fzbyks 傳“有五”至“良佑”。正義曰:“無能往來”一句,周公假為文王之辭。言文王有五賢臣,猶恨其少。又復言曰:“我臣既少,於事無能往來。”謂去還理事,未能周悉,言其好賢之深,不知厭足也。“迪”,道。“彝”,法也。“蔑”,小也,小謂精微也。而五人以此道法教文王以精微之德,用此精微之德下教令於國人。言雖聖人,亦須良佐,以見成王須輔佐之甚也。\CJKunderline{鄭玄}亦云:“蔑,小也。” \par}

{\noindent\shu\zihao{5}\fzkt “又曰”至“命哉”。正義曰:文王既有賢臣五人,又復言曰:“我之賢臣猶少,無所能往來。五人以此道法教文王以微蔑精妙之德,下政令於國人。德政既善,為天所佑。文王亦如殷家,惟為天所大佑。文王亦秉德,蹈知天威。文王德如此者,乃惟是五人明文王之德使然也。五人能明文王德,使蹈行顯見,覆冒下民,聞於上天,惟是之故得受有殷王之命哉!”言文王之聖,猶須良佐,我所以留輔成王。 \par}

武王惟茲四人,尚迪有祿。\footnote{文王沒,武王立,惟此四人,庶幾輔相武王蹈有天祿。虢叔先死,故曰四人。○相,息亮反。}後暨武王,誕將天威,咸劉厥敵。\footnote{言此四人後與武王皆殺其敵。謂誅紂。}惟茲四人,昭武王,惟冒丕單稱德。\footnote{惟此四人,明武王之德,使布冒天下,大盡舉行其德。}


{\noindent\zhuan\zihao{6}\fzbyks 傳“文王”至“四人”。正義曰:文王受命九年而崩,十三年方得殺紂。“文王沒,武王立”,謂武王初立之時,惟此四人而已。“庶幾輔相武王蹈有天祿”,初立則有此志,故下句言後與武王殺紂也。“虢叔先死,故曰四人”,以是文王之弟,其年應長,故言“先死”也。\CJKunderline{鄭玄}疑不知誰死,注云:“至武王時,虢叔等有死者,餘四人也。” \par}

{\noindent\zhuan\zihao{6}\fzbyks 傳“惟此”至“其德”。正義曰:“單”,盡。“稱”,舉也。使武王之德布冒天下,是此四人之力,言此四人大盡舉行武王之德也。 \par}

{\noindent\shu\zihao{5}\fzkt “武王”至“稱德”。正義曰:文王既沒,武王次立,武功初立,惟此四人,庶幾輔相武王蹈有天下之祿。其後四人,與武王大行天之威罰,皆與共殺其強敵,謂其誅紂也。武王之有天下,惟此四人明武王之德,惟武王佈德,覆冒天下,此四人大盡舉行武王之德。言武王亦得良臣之力。 \par}

今在予小子旦,若遊大川,予往暨汝奭其濟小子,同未在位,誕無我責。\footnote{我新還政,今任重在我小子旦,不能同於四方。若遊大川,我往與汝奭其共濟渡成王,同於未在位即政時,汝大無非責我留。}收罔勖不及,耇造德不降,我則鳴鳥不聞,矧曰其有能格?”\footnote{今與汝留輔成王,欲收教無自勉不及道義者,立此化,而老成德不降意為之。我周則鳴鳳不得聞,況曰其有能格於皇天乎?}


{\noindent\zhuan\zihao{6}\fzbyks 傳“我新”至“我留”。正義曰:周公既已還政,則是舍重任矣。而猶言“今任重在我小子旦”者,周公既攝王政,又須傳授得人,若其不能負荷,仍是周公之責,以嗣子劣弱,故言“今任重猶在我小子旦”也。彼四人者能翼贊初基,佑成王業,我不能同於四人,望有大功,惟求救弱而已。\CJKunderwave{詩}雲“泳之遊之”,\CJKunderwave{左傳}稱“閻敖遊湧而逸”,則“遊”者入水浮渡之名。譬若成王在大川,我往與汝奭其同共濟渡成王。若雲從此向川,故言“往”也。 \par}

{\noindent\zhuan\zihao{6}\fzbyks 傳“今與”至“天乎”。正義曰:王朝之臣有不勉力者,今與汝留輔成王者,正欲收斂教誨。無自勉力不及道義者,當教之勉力,使其及道義也。我欲成立此化,而老成德之人不肯降意為之。我周家則鳴鳳尚不聞知,況曰其有能如\CJKunderline{伊尹}之輩,使其功格於皇天乎?言太平不可冀也。經言“耇造德不降”者,周公以己年老應退而留,因即傳言己類,言己若退,則老成德者悉皆退自逸樂,不肯降意為之。政無所成,祥瑞不至,我周家則鳴鳳不得聞。則鳳是難聞之鳥,必為靈瑞之物,故以“鳴鳥”為鳴鳳。\CJKunderline{孔子}稱“鳳鳥不至”,是鳳鳥難聞也。\CJKunderwave{詩·大雅·卷阿}之篇歌成王之德,其九章曰:“鳳皇鳥矣,於彼高岡。”鄭云:“因時鳳皇至,固以喻焉。”則成王之時鳳皇至也。\CJKunderwave{大雅}正經之作,多在周公攝政之後,成王即位之初,則周公言此之時已鳳皇至,見太平矣。而復言此者,恐其不復能然,故戒之。此經之意,言功格上天,難於致鳳,故以鳴鳳況之格天。案\CJKunderwave{禮器}云:“升中於天,而鳳皇降,龜龍假。”“升中”謂功成告天也。如彼\CJKunderwave{記}文,似功至於天,鳳皇乃降,此以鳴鳳易致況格天之難者乎。\CJKunderwave{記}以龍鳳有形,是可見之物,故以鳳降龍至為成功之驗,非言成功告天,然後此物始至也。 \par}

{\noindent\shu\zihao{5}\fzkt “今在”至“能格”。正義曰:周公言:“我新還政成王,今任之重者,其在我小子之身也。我不能同於四人輔文武,使有大功德,但苟求救溺而已。譬如遊於大川,我往與汝奭其共濟渡小子成王,用心輔弼,同於成王未在位之時。恐其未能嗣先人明德,我當與汝輔之,汝大無非責我之留也。我留與汝輔王者,欲收教無自勉力不及道義者。我今欲立此化,而老成德之人不降意為之。我周家則鳴鳳之鳥尚不得聞知,況曰其有能格於皇天者乎?” \par}

公曰:“嗚呼!君,肆其監於茲。我受命無疆惟休,亦大惟艱。\footnote{以朝臣無能立功至天,故其當視於此,我周受命無窮惟美,亦大惟艱難,不可輕忽,謂之易治。}告君乃猷裕,我不以後人迷。”\footnote{告君汝謀寬饒之道,我留與汝輔王,不用後人迷惑,故欲教之。}


{\noindent\zhuan\zihao{6}\fzbyks 傳“告君”至“教之”。正義曰:“猷”訓為謀,告君汝謀寬饒之道,故當以寬饒為法。我留與汝輔王,不用使後人迷惑怪之。無法則迷惑,故欲與汝作法以教之。鄭云:“召公不說似隘急,故令謀於寬裕也。” \par}

{\noindent\shu\zihao{5}\fzkt “公曰嗚呼”至“人迷”。正義曰:周公嘆而呼召公曰:“嗚呼!君,我以朝臣無能立功至天之故,故君其當視於此。”謂視此朝臣無能立功之事。“我周家受天之命,無有境界惟美,亦大惟艱難,不可輕忽,謂之易治。我今告君,汝當謀寬饒之道以治下民,使其事可法,我不用使後世人迷惑,故欲教之也”。 \par}

公曰:“前人敷乃心,乃悉命汝,作汝民極。\footnote{前人文武布其乃心為法度,乃悉以命汝矣,為汝民立中正矣。}曰,汝明勖偶王,在亶乘茲大命。\footnote{汝以前人法度明勉配王,在於成信,行此大命而已。}惟文王德,丕承無疆之恤。”\footnote{惟文王聖德,為之子孫無忝厥祖,大承無窮之憂。}


{\noindent\zhuan\zihao{6}\fzbyks 傳“前人”至“正矣”。正義曰:“乃”,緩辭,不訓為汝。 \par}

{\noindent\zhuan\zihao{6}\fzbyks 傳“汝以”至“而已”。正義曰:“勖”,勉也。“偶”,配也。“亶”,信也。汝當以前人法度明自勉力,配成王,在於誠信行大命而已。言其不復須勞心。傳以“乘”為行,蓋以乘車必行,故訓“乘”為行。 \par}

{\noindent\shu\zihao{5}\fzkt “公曰前”至“之恤”。正義曰:周公又言曰,前人文武布其乃心製法度,乃悉命汝,為民立中正之道矣。治民之法已成就也,戒召公汝當以前人之法度明自勉力,配此成王,在於誠信,行此大命而已。言已有舊法,易可遵行也。惟文王聖德造始周邦,為其子孫欲令無忝厥祖,大承無窮之憂,故我與汝不可不輔。 \par}

公曰:“君,告汝朕允。\footnote{告汝以我之誠信也。}保奭,其汝克敬以予,監於殷喪大否。\footnote{呼其官而名之,敕使能敬以我言,視於殷喪亡大否。言其大不可不戒。}肆念我天威,予不允惟若茲誥,予惟曰:‘襄我二人。’\footnote{以殷喪大故,當念我天德可畏。言命無常,我不信惟若此誥。我惟曰:“當因我文武之道而行之。”}汝有合哉!言曰:‘在時二人,天休滋至,惟時二人弗戡。’\footnote{言汝行事,動當有所合哉!發言常在是文武,則天美周家,日益至矣,惟是文武不勝受。言多福。}其汝克敬德,明我俊民在讓,後人於丕時。\footnote{其汝能敬行德,明我賢人在禮讓,則後代將於此道大且是。}


{\noindent\zhuan\zihao{6}\fzbyks 傳“言汝”至“多福”。正義曰:“動當有所合哉”,舉動皆合文武也。“發言常在是文武”,言非文武道則不言。 \par}

{\noindent\shu\zihao{5}\fzkt “公曰君告”至“丕時”。正義曰:周公呼召公曰:“君,我今告汝以我之誠信。”又呼其官而名之:“太保奭,其汝必須能敬以我之言,視於殷之喪亡。殷之喪亡,其事甚大,不可不戒慎。以殷喪大之故,當念我天德可畏。”言天命無常,無德則去之,甚可畏。“我不信惟若此誥而已。我惟言曰:‘當因我文武二人之道而行之。’汝所行事,舉動必當有所合哉!當與文王武王合也。汝所發言,常在是文王武王二人,則天美我周家,日日滋益至矣。其善既多,惟在是文武二人,不能勝受之矣。其汝能敬行德,明我賢俊之人在於禮讓,則後人於此道大且是也。” \par}

嗚呼!篤棐時二人,我式克至於今日休。\footnote{言我厚輔是文武之道而行之,或用能至於今日其政美。}我咸成文王功於不怠,丕冒海隅出日,罔不率俾。”\footnote{今我周家皆成文王功於不懈怠,則德教大覆冒海隅日所出之地,無不循化而使之。}

{\noindent\shu\zihao{5}\fzkt “嗚呼”至“率俾”。正義曰:周公言而嘆曰:“嗚呼!我厚輔是二人之道而行之,我用能至於今日其政美。”言今日政美,由是文武之道。“我周家若能皆成文王之功,於事常不懈怠,則德教大覆四海之隅,至於日出之處,其民無不循我化,可臣使也”。戒召公與朝臣皆當法文王之功。 \par}

公曰:“君,予不惠若茲多誥,予惟用閔於天越民。”\footnote{我不順若此多誥而已,欲使汝念躬行之閔勉也。我惟用勉於天道加於民。}

{\noindent\shu\zihao{5}\fzkt “公曰君予”至“越民”。正義曰:公呼召公曰:“君,我不徒惟順如此之事多誥而已,欲使汝躬親行之。我惟用勉力自強於天道,行化於民。”顧氏云:“我亦自用勉勸,躬行於天道,加益於民人也。” \par}

公曰:“嗚呼!君,惟乃知民德,亦罔不能厥初,惟其終。\footnote{惟汝所知民德,亦無不能其初,鮮能有終,惟其終則惟君子。戒召公以慎終。○鮮,息淺反。}祗若茲,往敬用治。”\footnote{當敬順我此言,自今以往,敬用治民職事。}


{\noindent\zhuan\zihao{6}\fzbyks 傳“惟汝”至“慎終”。正義曰:\CJKunderwave{詩}云:“靡不有初,鮮克有終。”是凡民之德,無不能其初,少能有終者。凡民皆如是,有終則惟君子。蓋召公至此已說,恐其不能終善,故戒召公汝慎終也。鄭云:“召公是時意說,周公恐其復不說,故依違託言民德以剴切之。” \par}

{\noindent\shu\zihao{5}\fzkt “公曰嗚呼”至“用治”。正義曰:周公嘆而呼召公曰:“嗚呼!君,惟汝知民之德行,亦無有不能其初,惟鮮能其終。”言行之雖易,終之實難,恐召公不能終行善政,故戒之以慎終。“汝當以敬順我此言,自今以往,宜敬用此治民職事”。戒之使行善不懈怠也。 \par}

%%% Local Variables:
%%% mode: latex
%%% TeX-engine: xetex
%%% TeX-master: "../Main"
%%% End:
