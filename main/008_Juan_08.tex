%% -*- coding: utf-8 -*-
%% Time-stamp: <Chen Wang: 2024-04-02 11:42:41>

% {\noindent \zhu \zihao{5} \fzbyks } -> 注 (△ ○)
% {\noindent \shu \zihao{5} \fzkt } -> 疏


\part{商書}


\chapter{卷八}


\section{湯誓第一}


 {\noindent\zhuan\zihao{6}\fzbyks \CJKunderwave{釋文}:凡三十四篇,十七篇存。\par}

\CJKunderline{伊尹}相湯,伐\CJKunderline{桀},升自陑,\footnote{桀都安邑,湯升道從陑,出其不意。陑在河曲之南。相,息亮反。湯如字。馬云:“俗儒以湯為諡,或為號。號者似非其意,言諡近之。然不在\CJKunderwave{諡法},故無聞焉。及\CJKunderline{禹},俗儒以為名,\CJKunderwave{帝系}\CJKunderline{禹}名文命,\CJKunderwave{王侯世本}湯名天乙,推此言之,\CJKunderline{禹}豈復非諡乎?亦不在\CJKunderwave{諡法},故疑焉。”桀,其列反,夏之末天子。升音升。陑音而。}遂與\CJKunderline{桀}戰於鳴條之野,\footnote{地在安邑之西,桀逆拒湯。}作\CJKunderwave{湯誓}。


{\noindent\zhuan\zihao{6}\fzbyks 傳“桀都”至“之南”。正義曰:此序湯自伐桀,必言“\CJKunderline{伊尹}相湯”者,序其篇次,自為首尾,以上雲\CJKunderline{伊尹}醜夏,遂相\CJKunderline{成湯}伐之,故文次言“\CJKunderline{伊尹}”也。計太公之相武王,猶如\CJKunderline{伊尹}之相\CJKunderline{成湯},\CJKunderwave{泰誓}不言“太公相”者,彼文無其次也。且武王之時,有周、召之倫,聖賢多矣。湯稱\CJKunderline{伊尹}云:“聿求元聖,與之戮力。”\CJKunderline{伊尹}稱:“惟尹躬暨湯,咸有一德。”則\CJKunderline{伊尹}相湯,其功多於太公,故特言“\CJKunderline{伊尹}相湯”也。“桀都安邑”,相傳為然,即漢之河東郡安邑縣是也。\CJKunderwave{史記}吳起對魏武侯云:“夏桀之居,左河濟,右太華,伊闕在其南,羊腸在其北,修政不仁,湯放之也。”\CJKunderwave{地理志}云:上黨郡壺關縣有羊腸阪,在安邑之北。是桀都安邑必當然矣。將明陑之所在,故先言“桀都安邑”。桀都在亳西,當從東而往,今乃升道從陑。“升”者,從下向上之名。言陑當是山阜之地,歷險迂路,為出不意故也。陑在河曲之南,蓋今潼關左右。河曲在安邑西南,從陑向北,渡河乃東向安邑。鳴條在安邑之西,桀西出拒湯,故“戰於鳴條之野”。陑在河曲之南,鳴條在安邑之西,皆彼有其跡,相傳云然。湯以至聖伐暴,當顯行用師,而出其不意,掩其不備者,湯承禪代之後,嘗為桀臣,慚而且懼,故出其不意。武王則三分天下有其二,久不事紂,紂有浮桀之罪,地無險要之勢,故顯然致罰,以明天誅。又殷勤誓眾,與湯有異,所以湯惟一誓,武王有三。 \par}

{\noindent\zhuan\zihao{6}\fzbyks 傳“地在”至“拒湯”。正義曰:\CJKunderline{鄭玄}云:“鳴條,南夷地名。\CJKunderwave{孟子}雲舜卒於鳴條,東夷之地,或雲陳留平邱縣今有鳴條亭是也。”皇甫謐云:“\CJKunderwave{伊訓}曰:‘造攻自鳴條,朕哉自亳。’又曰:‘夏師敗績,乃伐三朡。’\CJKunderwave{湯誥}曰:‘王歸自克夏,至於亳。’三朡在定陶,於義不得在陳留與東夷也。今安邑見有鳴條陌、昆吾亭,\CJKunderwave{左氏}以為昆吾與桀同以乙卯日亡,韋顧亦爾。故\CJKunderwave{詩}曰:‘韋顧既伐,昆吾夏桀。’於\CJKunderwave{左氏}昆吾在衛,乃在濮陽,不得與桀異處同日而亡,明昆吾亦來安邑,欲以衛桀,故同日亡,而安邑有其亭也。且吳起言險以指安邑,安邑於此而言,何得在南夷乎?”謐言是也。 \par}

{\noindent\shu\zihao{5}\fzkt “\CJKunderline{伊尹}”至“湯誓”。正義曰:\CJKunderline{伊尹}以夏政醜惡,去而歸湯。輔相\CJKunderline{成湯},與之伐桀,升道從陑,出其不意,遂與桀戰於鳴條之野。將戰而誓戒士眾,史敘其事,作\CJKunderwave{湯誓}。 \par}

湯誓\footnote{戒誓湯士眾。}

{\noindent\shu\zihao{5}\fzkt “湯誓”。正義曰:此經皆誓之辭也。\CJKunderwave{甘誓}、\CJKunderwave{泰誓}、\CJKunderwave{牧誓}發首皆有序引,別言其誓意,記其誓處。此與\CJKunderwave{費誓}惟記誓辭、不言誓處者,史非一人,辭有詳略,序以經文不具,故備言之也。 \par}

王曰:“格爾眾庶,悉聽朕言。\footnote{契始封商,湯遂以為天下號。湯稱王,則比桀於一夫。格,庚白反。}非台小子,敢行稱亂。有夏多罪,天命殛之。\footnote{稱,舉也。舉亂,以諸侯伐天子。非我小子敢行此事,桀有昏德,天命誅之,今順天。臺,以之反,下同。殛,居力反。}今爾有眾,汝曰:‘我後不恤我眾,舍我穡事,而割正夏。’\footnote{汝,汝有眾。我後,桀也。正,政也。言奪民農功而為割剝之政。恤,荀律反。舍音舍,廢也。}予惟聞汝眾言,\footnote{不憂我眾之言。}夏氏有罪,予畏上帝,不敢不正。\footnote{不敢不正桀罪誅之。}今汝其曰:‘夏罪,其如台?’\footnote{今汝其復言桀惡,其亦如我所聞之言。復,扶又反。}夏王率遏眾力,率割夏邑。\footnote{言桀君臣相率為勞役之事以絕眾力,謂廢農功。相率割剝夏之邑居,謂徵賦重。遏,於葛反,徐音謁,馬云:“止也。”}


{\noindent\zhuan\zihao{6}\fzbyks 傳“契始”至“一夫”。正義曰:以湯於此稱“王”,故本其號商之意,“契始封商,湯遂以商為天下之號”。\CJKunderline{鄭玄}之說亦然。惟王肅云:“相士居商丘,湯取商為號。”若取商丘為號,何以不名“商丘”,而單名“商”也?若八遷,國名商不改,則此商猶是契商。非相士之商也。若八遷。遷即改名。則相士至湯改名多矣。相士既非始祖,又非受命,何故用其所居之地以為天下號名?\CJKunderline{成湯}之意,復何取乎?知其必不然也。湯取契封商,以商為天下之號,周不取后稷封邰為天下之號者,契後八遷,商名不改,\CJKunderline{成湯}以商受命,故宜以商為號;后稷之後,隨遷易名,公劉為豳,大王為周,文王以周受命,故當以周為號。二代不同,理則然矣。\CJKunderwave{泰誓}云:“獨夫受。”此湯稱為王,則比桀於一夫;桀既同於一夫,故湯可稱王矣。是言湯於伐桀之時始稱王也。\CJKunderwave{周書·泰誓}稱王,則亦伐紂之時始稱王也。\CJKunderline{鄭玄}以文王生稱王,亦謬也。 \par}

{\noindent\zhuan\zihao{6}\fzbyks 傳“稱舉”至“順天”。正義曰:“稱,舉”,\CJKunderwave{釋言}文。常法以臣伐君,則為亂逆,故舉亂謂“以諸侯伐天子”。“桀有昏德”,宣三年\CJKunderwave{左傳}文。以有昏德,天命誅之,今乃順天行誅,非復臣伐君也。以此解眾人守常之意也。 \par}

{\noindent\zhuan\zihao{6}\fzbyks 傳“今汝”至“之言”。正義曰:如我者,謂湯之自稱我也。湯謂其眾云:“汝言桀之罪,如我誓言所述也。” \par}

{\noindent\zhuan\zihao{6}\fzbyks 傳“言桀”至“賦重”。正義曰:此經與上“舍我穡事,而割正夏”其意一也。上言夏王之身,此言“君臣相率”,再言所以積桀之非也。力施於農,財供上賦,故以止絕眾力謂廢農功,割剝夏邑謂徵賦重。言以農時勞役,又重斂其財,致使民困而怨深,賦斂重則民不安矣。 \par}

有眾率怠弗協,曰:‘時日曷喪?予及汝皆亡!’\footnote{眾下相率為怠惰,不與上和合。比桀於日,曰:“是日何時喪?我與汝俱亡!”欲殺身以喪桀。喪,息浪反。惰,徒臥反。}夏德若茲,今朕必往。\footnote{凶德如此,我必往誅之。}爾尚輔予一人,致天之罰,予其大賚汝。\footnote{賚,與也。汝庶幾輔成我,我大與汝爵賞。罰音伐。賚,力代反,徐音來。}爾無不信,朕不食言。\footnote{食盡其言,偽不實。}爾不從誓言,\footnote{不用命。}予則孥戮汝,罔有攸赦。”\footnote{古之用刑,父子兄弟罪不相及,今雲孥戮汝,無有所赦,權以脅之,使勿犯。}

{\noindent\zhuan\zihao{6}\fzbyks 傳“眾下”至“喪桀”。正義曰:上既馭之非道,下亦不供其命,故“眾下相率為怠惰,不與上和合”,不肯每事順從也。比桀於日,曰:“是日何時喪亡?”欲令早喪桀命也。“我與汝俱亡”者,民相謂之辭,言並欲殺身以喪桀也。所以比桀於日者,以日無喪之理,猶雲桀不可喪,言喪之難也。不避其難,與汝俱亡,欲殺身以喪桀,疾之甚也。鄭云:“桀見民欲叛,乃自比於日,曰:‘是日何嘗喪乎?日若喪亡,我與汝亦皆喪亡。’引不亡之徵以脅恐下民也。” \par}

{\noindent\zhuan\zihao{6}\fzbyks 傳“食盡”至“不實”。正義曰:\CJKunderwave{釋詁}云:“食,偽也。”孫炎曰:“食言之偽也。”哀二十五年\CJKunderwave{左傳}云,孟武伯惡郭重,曰:“何肥也?”公曰:“是食言多矣,能無肥乎?”然則言而不行如食之消盡,後終不行前言為偽,故通謂偽言為“食言”,故\CJKunderwave{爾雅}訓“食”為偽也。 \par}

{\noindent\zhuan\zihao{6}\fzbyks 傳“古之”至“勿犯”。正義曰:昭二十年\CJKunderwave{左傳}引\CJKunderwave{康誥}曰:“父子兄弟,罪不相及。”是古之用刑如是也。既刑不相及,必不殺其子,權時以迫脅之,使勿犯刑法耳。不於\CJKunderwave{甘誓}解之者,以夏啟承舜、\CJKunderline{禹}之後,刑罰尚寬,殷、周以後,其罪或相緣坐,恐其實有孥戮,故於此解之。\CJKunderline{鄭玄}云:“大罪不止其身,又孥戮其子孫。\CJKunderwave{周禮}云:‘其奴男子入於罪隸,女子入於舂稿。’”鄭意以為實戮其子,故\CJKunderwave{周禮}注云:“奴謂從坐而沒入縣官者也。”孔以孥戮為權脅之辭,則\CJKunderwave{周禮}所云非從坐也。\CJKunderline{鄭玄}云:“謂坐為盜賊而為奴者,輸於罪隸。”舂人、稿人之官引此“孥戮汝”。又引\CJKunderwave{論語}云:“箕子為之奴。”或如眾言,別有沒入,非緣坐者也。 \par}

{\noindent\shu\zihao{5}\fzkt “王曰”至“攸赦”。正義曰:商王\CJKunderline{成湯}將與桀戰,呼其將士曰:“來,汝在軍之眾庶,悉聽我之誓言。我伐夏者,非我小子輒敢行此以臣伐君,舉為亂事,乃由有夏君桀多有大罪,上天命我誅之。桀既失君道,我非復桀臣,是以順天誅之,由其多罪故也。桀之罪狀,汝盡知之。今汝桀之所有之眾,即汝輩是也。汝等言曰:‘我君夏桀,不憂念我等眾人,舍廢我稼穡之事,奪我農功之業,而為割剝之政於夏邑,斂我貨財。’我惟聞汝眾言,夏氏既有此罪,上天命我誅,桀我畏上天之命,不敢不正桀罪而誅之。又質而審之,今汝眾人其必言曰:‘夏王之罪其實如我所言。’夏王非徒如此,又與臣下相率遏絕眾力,使不得事農。又相率為割剝之政於此夏邑,使不得安居。上下同惡,民困益甚,由是汝等相率怠,惰不與在上和協。比桀於日,曰:‘是日何時能喪?君其可喪,我與汝皆亡身殺之。’寧殺身以亡桀,是其惡之甚。夏王惡德如此,今我必往誅之。汝庶幾輔成我一人,致行天之威罰,我其大賞賜汝。汝無得不信我語,我終不食盡其言,為虛偽不實。汝若不從我之誓言,我則並殺汝子,以戮汝身,必無有所赦。”勸使勉力,勿犯法也。“庶”亦“眾”也,古人有此重言,猶雲“艱難”也。 \par}

湯既勝夏,欲遷其社,不可。\footnote{湯承堯、舜禪代之後,順天應人,逆取順守而有慚德,故革命創制,改正易服,變置社稷,而後世無及句龍者,故不可而止。社,后土之神。禪,時戰反。應,應對之應。創,初亮反。正音徵,又音政。句音鉤。句龍,\CJKunderline{共工}之子,為后土。}作\CJKunderwave{夏社}、\CJKunderwave{疑至}、\CJKunderwave{臣扈}。\footnote{言夏社不可遷之義,\CJKunderwave{疑至}及\CJKunderwave{臣扈},三篇皆亡。扈音戶。}

{\noindent\zhuan\zihao{6}\fzbyks 傳“湯承”至“而止”。正義曰:傳解湯遷社之意,湯承堯、舜禪代之後,已獨伐而取之,雖復應天順人,乃是逆取順守,而有慚愧之德,自恨不及古人,故革命創制改正易服,因變置社稷也。\CJKunderwave{易·革卦·彖}曰:“湯武革命,順乎天而應乎人。”下篇言湯有慚德。\CJKunderwave{大傳}云:“改正朔,易服色,此其所得與民變革者也。”所以變革此事,欲易人之視聽,與之更新,故於是之時變置社稷。昭二十九年\CJKunderwave{左傳}云:“\CJKunderline{共工氏}有子曰句龍,為后土,后土為社。有烈山氏之子曰柱,為稷,自夏已上祀之。周棄亦為稷,自商已來祀之。”\CJKunderwave{祭法}云:“厲山氏之有天下也,其子曰農,能殖百穀。夏之衰也,周棄繼之,故祀以為稷。\CJKunderline{共工氏}之霸九州也,其子曰后土,能平九州,故祀以為社。”是言變置之事也。\CJKunderwave{魯語}文與\CJKunderwave{祭法}正同,而云“夏之興也,周棄繼之”,“興”當為“衰”字之誤耳。湯於初時,社稷俱欲改之。周棄功多於柱,即令廢柱祀棄。而上世治水土之臣,其功無及句龍者,故不可遷而止。此序之次在\CJKunderwave{湯誓}之下,雲“湯既勝夏”,下雲“夏師敗績,湯遂從之”,是未及逐桀,已為此謀。\CJKunderline{鄭玄}等注此序,乃在\CJKunderwave{湯誓}之上,若在作誓之前,不得雲“既勝夏”也。\CJKunderwave{孟子}曰:“犧牲既成,粢盛既絜,祭祀以時,然而旱乾水益,則變置社稷。”\CJKunderline{鄭玄}因此乃云:“湯伐桀之時,大旱,既置其禮祀,明德以薦而猶旱,至七年,故更致社稷。”乃謂湯即位之後,七年大旱,方始變之。若實七年乃變,何當系之“勝夏”?“勝夏”猶尚不可,況在\CJKunderwave{湯誓}前乎?且\CJKunderwave{禮記}云:“夏之衰也,周棄繼之。”商興七年乃變,安得以“夏衰”為言也?若商革夏命,猶七年祀柱,\CJKunderwave{左傳}亦不得斷為自夏已上祀柱,自商以來祀棄也。由此而言,孔稱改正朔而變置社稷,所言得其旨也。漢世儒者說社稷有二,\CJKunderwave{左傳}說社祭句龍,稷祭柱、棄,惟祭人神而已。\CJKunderwave{孝經}說社為土神,稷為穀神,句龍、柱、棄是配食者也。孔無明說,而此經雲“遷社”,孔傳雲“無及句龍”,即同賈逵、馬融等說,以社為句龍也。 \par}

{\noindent\zhuan\zihao{6}\fzbyks 傳“言夏”至“皆亡”。正義曰:“疑至”與“臣扈”相類,當是二臣名也。蓋亦言其不可遷之意。馬融云:“聖人不可自專,複用二臣自明也。” \par}

{\noindent\shu\zihao{5}\fzkt “湯既”至“臣扈”。正義曰:湯既伐而勝夏,革命創制,變置社稷,欲遷其社,無人可代句龍,故不可而止。於時有言議論其事,故史敘之,為\CJKunderwave{夏社}、\CJKunderwave{疑至}、\CJKunderwave{臣扈}三篇,皆亡。 \par}

夏師敗績,湯遂從之,\footnote{大崩曰敗績。從謂遂討之。績,子寂反。從,才容反。}遂伐三朡,俘厥寶玉。\footnote{三朡,國名,桀走保之,今定陶也。桀自安邑東入山,出太行,東南涉河。湯緩追之,不迫,遂奔南巢。俘,取也。玉以禮神,使無水旱之災,故取而寶之。朡,子公反。俘音孚。行,戶剛反,一音如字。}

{\noindent\shu\zihao{5}\fzkt 傳“三朡”至“寶之”。正義曰:湯伐三朡,知是國名。逐桀而伐其國,知“桀走保之”也。“今定陶”者,相傳為然。安邑在洛陽西北,定陶在洛陽東南,孔跡其所往之路,桀自安邑東入山,出太行,乃東南涉河,往奔三朡;湯緩追之,不迫,遂奔南巢。“俘、取”,\CJKunderwave{釋詁}文。桀必載寶而行,棄於三朡。取其寶玉,取其所棄者也。楚語云:“玉足以庇廕嘉,谷使無水旱之災,則寶之。”韋昭云:“玉,禮神之玉也。”言用玉禮神,神享其德,使風雨調和,可以庇廕嘉穀,故取而寶之。 \par}

\CJKunderline{誼伯}、\CJKunderline{仲伯}作\CJKunderwave{典寶}。\footnote{二臣作\CJKunderwave{典寶}一篇,言國之常寶也,亡。誼,本或作義。}

\section{仲虺之誥第二【偽】}

湯歸自夏,至於大坰,\footnote{自三朡而還。大坰,地名。夏,亥雅反。坰,故螢反,徐,欽螢反,又古螢反。}\CJKunderline{仲虺}作誥。\footnote{為湯左相,奚仲之後。虺,許鬼反。誥,故報反。相,息亮反。奚,弦雞反。}


{\noindent\zhuan\zihao{6}\fzbyks 傳“為湯”至“之後”。正義曰:定元年\CJKunderwave{左傳}云:“薛之皇祖奚仲居薛,以為夏車正。\CJKunderline{仲虺}居薛,以為湯左相。”是其事也。 \par}

{\noindent\shu\zihao{5}\fzkt “湯歸”至“作誥”。正義曰:湯歸自伐夏,至於大坰之地,其臣\CJKunderline{仲虺}作誥以誥湯,使錄其言,作\CJKunderwave{仲虺之誥}。上言“遂伐三朡”,故傳言“自三朡而還”。不言“歸自三朡”,而言“歸自夏”者,伐夏而遂逐桀,於今方始旋歸,以自夏告廟,故序言“自夏”。傳本其來處,故云“自三朡”耳。“大坰,地名”,未知所在,當是定陶向亳之路所經。湯在道而言“予恐來世以臺為口實”,故\CJKunderline{仲虺}至此地而作誥也。序不言“作\CJKunderline{仲虺}之誥”,以理足文便,故略之。 \par}

仲虺之誥\footnote{仲虺,臣名,以諸侯相天子。會同曰誥。}


{\noindent\zhuan\zihao{6}\fzbyks 傳“\CJKunderline{仲虺}”至“曰誥”。正義曰:伯仲叔季,人字之常,“\CJKunderline{仲虺}”必是其名,或字仲而名虺。古人名或不可審知,縱使是字,亦得謂之為名,言是人之名號也。\CJKunderwave{左傳}稱居薛,為湯左相,是“以諸侯相天子”也。\CJKunderwave{周禮·士師}云:“以五戒先後刑罰,一曰誓,用之於軍旅。二曰誥,用之於會同。”是“會同曰誥”。“誥”謂於會之所,設言以誥眾,此惟誥湯一人而言“會同”者,因解諸篇“誥”義,且\CJKunderline{仲虺}必對眾誥湯,亦是“會同曰誥”。 \par}

{\noindent\shu\zihao{5}\fzkt “\CJKunderline{仲虺}之誥”。正義曰:發首二句,史述\CJKunderline{成湯}之心。次二句,湯言己慚之意,\CJKunderline{仲虺}乃作誥。以下皆勸湯之辭。自“曰嗚呼”至“用爽厥師”,言天以桀有罪,命伐夏之事。自“簡賢輔勢”至“言足聽聞”,說湯在桀時怖懼之事。自“惟王弗邇聲色”至“厥惟舊哉”,言湯有德行加民,民歸之事。自“佑賢輔德”以下說天子之法,當擢用賢良,屏黜昏暴,勸湯奉行此事,不須以放桀為惡。\CJKunderwave{康誥}、\CJKunderwave{召誥}之類,二字足以為文,“\CJKunderline{仲虺}誥”三字不得成文,以“之”字足成其句。\CJKunderwave{畢命}、\CJKunderwave{冏命}不言“之”,\CJKunderwave{微子之命}、\CJKunderwave{文侯之命}言“之”,與此同,猶\CJKunderwave{周禮·司服}言“大裘而冕”,亦足句也。 \par}

\CJKunderline{成湯}放\CJKunderline{桀}於南巢,惟有慚德,\footnote{湯伐桀,武功成,故以為號。南巢,地名。有慚德,慚德不及古。湯伐桀,武功成,故號\CJKunderline{成湯}。一云:“成,諡也。”}曰:“予恐來世以台為口實。”\footnote{恐來世論道我放天子,常不去口。}\CJKunderline{仲虺}乃作誥,\footnote{陳義誥湯,可無慚。}曰:“嗚呼!惟天生民有欲,無主乃亂,\footnote{民無君主則恣情慾,必致禍亂。}惟天生聰明時乂。\footnote{言天生聰明,是治民亂。}有夏昏德,民墜塗炭。\footnote{夏桀昏亂,不恤下民,民之危險,若陷泥墜火,無救之者。}天乃錫王勇智,表正萬邦,纘\CJKunderline{禹}舊服,\footnote{言天與王勇智,應為民主,儀表天下,法正萬國,繼\CJKunderline{禹}之功,統其故服。纘,子管反。應,應對之應。}茲率厥典,奉若天命。\footnote{天意如此,但當循其典法,奉順天命而已,無所慚。}

{\noindent\shu\zihao{5}\fzkt “\CJKunderline{成湯}放桀於南巢”。正義曰:桀奔南巢,湯縱而不迫,故稱“放”也。傳言“南巢,地名”,不知地之所在。\CJKunderwave{周書}序有“巢伯來朝”,傳云:“南方遠國。”\CJKunderline{鄭玄}云:“巢,南方之國。世一見者,桀之所奔,蓋彼國也。以其國在南,故稱南耳。”傳並以“南巢”為地名,不能委知其處,故未明言之。 \par}

“夏王有罪,矯誣上天,以布命於下。\footnote{言託天以行虐於民,乃桀之大罪。矯,居表反。誣音無。}帝用不臧,式商受命,用爽厥師。\footnote{天用桀無道,故不善之。式,用。爽,明也。用商受王命,用明其眾,言為主也。臧,作郎反。}簡賢附勢,實繁有徒。\footnote{簡,略也。賢而無勢則略之,不賢有勢則附之。若是者繁多有徒眾,無道之世所常。繁音煩。}肇我邦於有夏,若苗之有莠,若粟之有秕。\footnote{始我商家,國於夏世,欲見翦除,若莠生苗,若秕在粟,恐被鋤治簸颺。莠,羊九反。秕,悲裡反,徐,甫裡反,又必履反。鋤,仕魚反。簸,彼我反。颺音揚。}小大戰戰,罔不懼於非辜。矧予之德,言足聽聞。\footnote{言商家小大憂危,恐其非罪見滅。矧,況也。況我之道德善言足聽聞乎!無道之惡有道,自然理。惡,烏路反。}惟王不邇聲色,不殖貨利。\footnote{邇,近也。不近聲樂,言清簡。不近女色,言貞固。殖,生也。不生資貨財利,言不貪也。既有聖德,兼有此行。近,附近之近。行,下孟反。}


{\noindent\zhuan\zihao{6}\fzbyks 傳“式,用。爽,明也”。正義曰:“式,用”,\CJKunderwave{釋言}文。昭七年\CJKunderwave{左傳}云:“是以有精爽至於神明。”從爽以至於明,則“爽”是明之始,故“爽”為明也。經稱“昧爽”,謂未大明也。 \par}

{\noindent\shu\zihao{5}\fzkt “夏王”至“厥師”。正義曰:“矯”,詐也。“誣”,加也。夏王自有所欲,詐加上天,言天道須然,不可不爾,假此以布苛虐之命於天下,以困苦下民。上天用桀無道之故,故不善之,用使商家受此為王之命,以王天下。用命商王,明其所有之眾,謂湯教之使修德行善以自安樂,是明之也。 \par}

德懋懋官,功懋懋賞。用人惟己,改過不吝。\footnote{勉於德者,則勉之以官。勉於功者,則勉之以賞。用人之言,若自己出;有過則改,無所吝惜,所以能成王業。}克寬克仁,彰信兆民。\footnote{言湯寬仁之德明信於天下。}

{\noindent\shu\zihao{5}\fzkt “德懋”至“不吝”。正義曰:於德能勉力行之者,王則勸勉之以官。於功能勉力為之者,王則勸勉之以賞。用人之言,惟如己之所出;改悔過失,無所吝惜。美湯之行如此。凡庸之主,得人之言,恥非己智,雖知其善,不肯遂從。己有愆失,恥於改過,舉事雖覺其非,不肯更悔,是惜過不改。故以此美湯也。\CJKunderline{成湯}之為此行,尚為\CJKunderline{仲虺}所稱歎,凡人能勉者鮮矣。 \par}

乃\CJKunderline{葛伯}仇餉,初征自葛,東征西夷怨,南征北狄怨,\footnote{\CJKunderline{葛伯}遊行,見農民之餉于田者,殺其人,奪其餉,故謂之仇餉。仇,怨也。湯為是以不祀之罪伐之,從此後遂徵無道。西夷、北狄,舉遠以言,則近者著矣。仇音求。餉,式亮反。}曰:‘奚獨後予?’\footnote{怨者辭也。}攸徂之民,室家相慶,曰:‘徯予後,後來其蘇。’\footnote{湯所往之民,皆喜曰:“待我君來,其可蘇息。”徯,胡啟反。蘇,字亦作穌。}民之戴商,厥惟舊哉!\footnote{舊,謂初征自葛時。}佑賢輔德,顯忠遂良。\footnote{賢則助之,德則輔之,忠則顯之,良則進之。明王之道。}兼弱攻昧,取亂侮亡。\footnote{弱則兼之,暗則攻之,亂則取之,有亡形則侮之。言正義。}推亡固存,邦乃其昌。\footnote{有亡道,則推而亡之;有存道,則輔而固之。王者如此,國乃昌盛。推,土雷反。}

{\noindent\zhuan\zihao{6}\fzbyks 傳“賢則”至“之道”。正義曰:\CJKunderwave{周禮·鄉大夫}云:“三年則大比,考其德行道藝,而興賢者。”\CJKunderline{鄭玄}云:“賢者謂有德行者。”\CJKunderwave{詩序}云:“忠臣良士皆是善也。”然則“賢”是德盛之名,“德”是資賢之實,“忠”是盡心之事,“良”是為善之稱,俱是可用之人,所從言之異耳。“佑之”與“輔、顯之”與“遂”,隨便而言之。 \par}

{\noindent\zhuan\zihao{6}\fzbyks 傳“弱則”至“正義”。正義曰:力少為“弱”,不明為“昧”,政荒為“亂”,國滅為“亡”,“兼”謂包之,“攻”謂擊之,“取”謂取為己有,“侮”謂侮慢其人。“弱”、“昧”、“亂”、“亡”,俱是彼國衰微之狀。“兼”、“攻”、“取”、“侮”,是此欲吞併之意。“弱”、“昧”是始衰之事,來服則製為己屬,不服則以兵攻之。此二者始欲服其人,末是滅其國。“亂”是已亂,“亡”謂將亡,二者衰甚,已將滅其國。亡形已著,無可忌憚,故陵侮其人。既侮其人,必滅其國,故以“侮”言之。此是人君之正義。\CJKunderline{仲虺}陳此者,意亦言桀亂亡,取之不足為愧。下言“推亡”及“覆昏暴”,其意亦在桀也。 \par}

{\noindent\shu\zihao{5}\fzkt “乃\CJKunderline{葛伯}仇餉”。正義曰:此言“乃”者,卻說已過之事。\CJKunderwave{胤徵}雲“乃季秋月朔”,其義亦然。\CJKunderwave{左傳}稱“怨耦曰仇”,謂彼人有負於我,我心怨之,是名為“仇”也。餉田之人不負\CJKunderline{葛伯},\CJKunderline{葛伯}奪其餉而殺之,是\CJKunderline{葛伯}以餉田之人為己之仇。言非所怨而妄殺,故湯為之報也。\CJKunderwave{孟子}稱湯使亳眾往為之耕,有童子以黍肉餉,\CJKunderline{葛伯}奪而殺之。則\CJKunderline{葛伯}所殺,殺亳人也。傳言“\CJKunderline{葛伯}遊行,見農人之餉于田者,殺其人而奪其餉,故謂之仇餉”,乃似\CJKunderline{葛伯}自殺己人,與\CJKunderwave{孟子}違者,湯之徵葛,以人之枉死而為之報耳,不為亳人乃報之,非亳人則赦之,故傳指言殺餉,不辨死者何人。亳人、葛人,義無以異,故不復言“亳”,非是故違\CJKunderwave{孟子}。 \par}

德日新,萬邦惟懷。志自滿,九族乃離。\footnote{日新,不懈怠。自滿,志盈溢。懈,工債反。}

{\noindent\shu\zihao{5}\fzkt “德日”至“乃離”。正義曰:\CJKunderwave{易·繫辭}云:“日新之謂盛德。”修德不怠,日日益新,德加於人,無遠不屆,故萬邦之眾惟盡歸之。志意自滿則陵人,人既被陵,情必不附,雖九族之親,乃亦離之。“萬邦”,舉遠以明近;“九族”,舉親以明疏也。漢代儒者說九族有二,案\CJKunderwave{禮戴}及\CJKunderwave{尚書緯}、歐陽說九族,乃異姓有屬者,父族四,母族三,妻族二。\CJKunderwave{古尚書}說九族,從高祖至玄孫凡九族。\CJKunderwave{堯典}雲“以親九族”,傳雲“以睦高祖玄孫之親”,則此言“九族”,亦謂高祖玄孫之親也。謂“萬邦惟懷”,實歸之。“九族乃離”,實離之。聖賢設言為戒,容辭頗甚,父子之間,便以志滿相棄。此言“九族”,以為外姓九族有屬,文便也。 \par}

王懋昭大德,建中於民,以義制事,以禮制心,垂裕後昆。\footnote{欲王自勉,明大德,立大中之道於民,率義奉禮,垂優足之道示後世。中如字;本或作忠,非。裕,徐以樹反。}予聞曰:‘能自得師者王,\footnote{求賢聖而事之。王,徐於況反,又如字。}謂人莫已若者亡。\footnote{自多足,人莫之益,亡之道。}好問則裕,自用則小。’\footnote{問則有得,所以足,不問專固,所以小。好,呼報反。}嗚呼!慎厥終,惟其始。\footnote{靡不有初,鮮克有終,故戒慎終如其始。鮮,息淺反。}殖有禮,覆昏暴。\footnote{有禮者封殖之,昏暴者覆亡之。覆,芳服反。暴,蒲報反,字或作虣。}欽崇天道,永保天命。”\footnote{王者如此上事,則敬天安命之道。}

\section{湯誥第三【偽】}

\CJKunderline{湯}既黜夏命,\footnote{黜,退也,退其王命。}復歸於亳,作\CJKunderwave{湯誥}。

湯誥\footnote{以伐桀大義告天下。}

{\noindent\shu\zihao{5}\fzkt “湯既”至“湯誥”。正義曰:湯既黜夏王之命,復歸於亳,以伐桀大義誥示天下。史錄其事,作\CJKunderwave{湯誥}。\CJKunderline{仲虺}在路作誥,此至亳乃作,故次\CJKunderwave{仲虺}之下。 \par}

王歸自克夏,至於亳,誕告萬方。\footnote{誕,大也。以天命大義告萬方之眾人。誕音但。告,工毒反。}

{\noindent\shu\zihao{5}\fzkt “王歸自克夏”。正義曰:湯之伐桀,當有諸侯從之,不從行者必應多矣。既已克夏,改正名號,還至於亳,海內盡來,猶如\CJKunderwave{武成}篇所云“庶邦冢君暨百工,受命於周”也。湯於此時大誥諸侯以伐桀之義,故云“誕告萬方”。“誕,大”,\CJKunderwave{釋詁}文。“萬”者舉盈數。下雲“凡我造邦”,是誥諸侯也。 \par}

王曰:“嗟!爾萬方有眾,明聽予一人誥。\footnote{天子自稱曰予一人,古今同義。}惟皇上帝,降衷於下民。\footnote{皇,大。上帝,天也。衷,善也。}

{\noindent\shu\zihao{5}\fzkt “降衷於下民”。正義曰:天生烝民,與之五常之性,使有仁義禮智信,是天降善於下民也。天既與善於民,君當順之,故下傳云,順人有常之性,則是為君之道。 \par}

若有恆性,克綏厥猷惟後。\footnote{順人有常之性,能安立其道教,則惟為君之道。}夏王滅德作威,以敷虐於爾萬方百姓。\footnote{夏桀滅道德,作威刑以布行虐政於天下百官。言殘酷。}爾萬方百姓,罹其兇害,弗忍荼毒,\footnote{罹,被。荼毒,苦也。不能堪忍,虐之甚。罹,力之反;本亦作羅,洛河反。荼音徒。}

{\noindent\shu\zihao{5}\fzkt “弗忍荼毒”。正義曰:\CJKunderwave{釋草}云:“荼,苦菜。”此菜味苦,故假之以言人苦。“毒”謂螫人之蟲,蛇虺之類。實是人之所苦,故並言“荼毒”以喻苦也。 \par}

並告無辜於上下神祇。\footnote{言百姓兆民並告無罪,稱冤訴天地。冤,紆元反。}天道福善禍淫,降災於夏,以彰厥罪。\footnote{政善天福之,淫過天禍之,故下災異以明桀罪惡,譴寤之而桀不改。譴,遣戰反。寤,五故反。}

“肆台小子,將天命明威,不敢赦。\footnote{行天威,謂誅之。臺音怡。}敢用玄牡,敢昭告於上天神後,請罪有夏。\footnote{明告天,問桀百姓有何罪而加虐乎?牡,茂後反。}

{\noindent\shu\zihao{5}\fzkt “敢用玄牡”。正義曰:\CJKunderwave{檀弓}云:“殷人尚白,牲用白。”今雲“玄牡”,夏家尚黑,於時未變夏禮,故不用白也。故安國注\CJKunderwave{論語}“敢用玄牡”之文云,“殷家尚白,未變夏禮,故用玄牡”,是其義也。\CJKunderline{鄭玄}說:“天神有六,周家冬至祭皇天大帝於圜丘,牲用蒼。夏至祭靈威仰於南郊,則牲用騂。”孔注\CJKunderwave{孝經},圜丘與郊共為一事,則孔之所說無六天之事,\CJKunderwave{論語·堯曰}之篇所言“敢用玄牡”即此事是也。孔注\CJKunderwave{論語}以為“堯曰”之章“有二帝三王之事,錄者採合以成章。檢\CJKunderwave{大禹謨}及此篇與\CJKunderwave{泰誓}、\CJKunderwave{武成},則‘堯曰’之章其文略矣。”。\CJKunderline{鄭玄}解\CJKunderwave{論語}云:“‘用玄牡’者,為舜命\CJKunderline{禹}事,於時裛告五方之帝,莫適用,用皇天大帝之牲。”其意與孔異。 \par}

聿求元聖,與之戮力,以與爾有眾請命。\footnote{聿,遂也。大聖陳力,謂\CJKunderline{伊尹}。放桀除民之穢,是請命。聿,允橘反,述也。戮舊音六,又力雕反,\CJKunderwave{說文}力周反,\CJKunderwave{史記}音力消反。穢,於廢反。}

{\noindent\shu\zihao{5}\fzkt 傳“聿遂也”至“請命”。正義曰:“聿”訓述也,述前所以申遂,故“聿”為遂也。“戮力”猶勉力也。\CJKunderwave{論語}云:“陳力就列。”湯臣大賢惟有\CJKunderline{伊尹},故知“大聖陳力,謂\CJKunderline{伊尹}”也。\CJKunderline{伊尹}賢人而謂之“聖”者,相對則“聖”極而“賢”,次散文則“賢”、“聖”相通。舜謂\CJKunderline{禹}曰:“惟汝賢。”是“聖”得謂之“賢”,則“賢”亦可言“聖”。\CJKunderline{鄭玄}\CJKunderwave{周禮注}云:“聖通而先識也。”解先識則為聖名,故\CJKunderline{伊尹}可為聖也。\CJKunderwave{孟子}云:“\CJKunderline{\CJKunderline{伯夷}}聖人之清者也,\CJKunderline{伊尹}聖人之任者也,柳下惠聖人之和者也,\CJKunderline{孔子}聖人之時者也。”是謂\CJKunderline{伊尹}為聖人者也。桀為殘虐,人不自保,故伐桀除人之穢,是為請命。 \par}

上天孚佑下民,罪人黜伏。\footnote{孚,信也。天信佑助下民,桀知其罪,退伏遠屏。}天命弗僣,賁若草木,兆民允殖。\footnote{僣,差。賁,飾也。言福善禍淫之道不差,天下惡除,煥然咸飾,若草木同華,民信樂生。僣,子念反,忒也;劉創林反。賁,彼義反,徐扶雲反。煥,呼亂反。樂音洛。}

{\noindent\shu\zihao{5}\fzkt “天命”至“允殖”。正義曰:桀以大罪,身即黜伏,是天之福善禍淫之命信而不僣差也。既除大惡,天下煥然修飾,若草木同生華,兆民信樂生也。昔日不保性命,今日樂生活矣。僣差,不齊之意,故傳以“僣”為差。“賁,飾”,\CJKunderwave{易·序卦}文也。 \par}

俾予一人,輯寧爾邦家。\footnote{言天使我輯安汝國家。國,諸侯。家,卿大夫。}茲朕未知獲戾於上下,\footnote{此伐桀未知得罪於天地。謙以求眾心。戾,力計反。}

{\noindent\shu\zihao{5}\fzkt 傳“此伐”至“眾心”。正義曰:經言“茲”者,謂此伐桀也。顧氏云:“‘未知得罪於天地’,言伐桀之事,未知得罪於天地以否。”湯之伐桀,上應天心,下符人事,本實無罪,而云未知得罪以否者,謙以求眾心。 \par}

慄慄危懼,若將隕於深淵。\footnote{慄慄危心,若墜深淵。危懼之甚。慄音慄。隕,于敏反。}凡我造邦,無從匪彝,無即慆淫,\footnote{戒諸侯與之更始。彝,常。慆,慢也。無從非常,無就慢過,禁之。彝,徐音夷。慆,他刀反。}各守爾典,以承天休。\footnote{守其常法,承大美道。}爾有善,朕弗敢蔽。罪當朕躬,弗敢自赦,惟簡在上帝之心。\footnote{所以不蔽善人,不赦己罪,以其簡在天心故也。}

{\noindent\shu\zihao{5}\fzkt “惟簡在上帝之心”。正義曰:\CJKunderline{鄭玄}注\CJKunderwave{論語}云:“簡閱在天心,言天簡閱其善惡也。” \par}

其爾萬方有罪,在予一人。\footnote{自責化不至。}予一人有罪,無以爾萬方。\footnote{無用爾萬方,言非所及。}嗚呼!尚克時忱,乃亦有終。”\footnote{忱,誠也。庶幾能是誠道,乃亦有終世之美。忱,市林反。}

\CJKunderline{咎單}作\CJKunderwave{明居}。\footnote{\CJKunderline{咎單},臣名,主土地之官。作明居民法一篇,亡。單音善,卷末同。}

{\noindent\shu\zihao{5}\fzkt “\CJKunderline{咎單}作明居”。正義曰:百篇之序此類有四,\CJKunderline{伊尹}作\CJKunderwave{咸有一德}、周公作\CJKunderwave{無逸}、作\CJKunderwave{立政},與此篇。直言其所作之人,不言其作者之意,蓋以經文分明,故略之。馬融云:“\CJKunderline{咎單}為湯司空。”傳言“主土地之官”,蓋亦為司空也。 \par}

\section{伊訓第四【偽】}


\CJKunderline{成湯}既沒,\CJKunderline{太甲}元年,\footnote{太甲,太丁子,湯孫也。太丁未立而卒,及湯沒而\CJKunderline{太甲}立,稱元年。}\CJKunderline{伊尹}作\CJKunderwave{伊訓}、\CJKunderwave{肆命}、\CJKunderwave{徂後}。\footnote{凡三篇,其二亡。}


{\noindent\zhuan\zihao{6}\fzbyks 傳“\CJKunderline{太甲}”至“元年”。正義曰:“\CJKunderline{太甲},太丁子”,\CJKunderwave{世本}文也。此序以“\CJKunderline{太甲}元年”繼“湯沒”之下,明是太丁未立而卒,\CJKunderline{太甲}以孫繼祖,故湯沒而\CJKunderline{太甲}代立,即以其年稱為元年也。周法以逾年即位,知此即以其年稱元年者,此經雲“元祀十有二月,\CJKunderline{伊尹}祠於先王。奉嗣王祗見厥祖”,\CJKunderwave{太甲}中篇雲“惟三祀十有二月朔,\CJKunderline{伊尹}以冕服奉嗣王歸於亳”,二者皆雲“十有二月”,若是逾年即位,二者皆當以正月行事,何以用十二月也?明此經“十二月”是湯崩之逾月,\CJKunderwave{太甲}中篇“三祀十有二月”是服闋之逾月,以此知湯崩之年,\CJKunderline{太甲}即稱元年也。舜\CJKunderline{禹}以受帝終事,自取歲首,遭喪嗣位,經無其文,夏後之世或亦不逾年也。顧氏云:“殷家猶質,逾月即改元年,以明世異,不待正月以為首也。”商謂年為祀,序稱“年”者,序以周世言之故也。據此經序及\CJKunderwave{太甲}之篇,\CJKunderline{太甲}必繼湯後,而\CJKunderwave{殷本紀}云:“湯崩,太子太丁未立而卒,於是乃立太丁之弟外丙。三年崩,別立外丙之弟仲壬。四年崩,\CJKunderline{伊尹}乃立太丁之子\CJKunderline{太甲}。”與經不同,彼必妄也。劉歆、班固不見古文,謬從\CJKunderwave{史記}。皇甫謐既得此經,作\CJKunderwave{帝王世紀},乃述馬遷之語,是其疏也。顧氏亦云:“止可依經誥大典,不可用傳記小說。” \par}

{\noindent\shu\zihao{5}\fzkt “\CJKunderline{成湯}”至“徂後”。正義曰:\CJKunderline{成湯}既沒,其歲即\CJKunderline{太甲}元年。\CJKunderline{伊尹}以\CJKunderline{太甲}承湯之後,恐其不能纂修祖業,作書以戒之。史敘其事,作\CJKunderwave{伊訓}、\CJKunderwave{肆命}、\CJKunderwave{徂後}三篇。 \par}

伊訓\footnote{作訓以教道\CJKunderline{太甲}。}

惟元祀十有二月乙丑,\CJKunderline{伊尹}祠於先王。\footnote{此湯崩逾月,\CJKunderline{太甲}即位,奠殯而告。祀,年也。夏曰歲,商曰祀,周曰年,唐虞曰載。尹祠音辭,祭也。}


{\noindent\zhuan\zihao{6}\fzbyks 傳“此湯”至“而告”。正義曰:\CJKunderwave{太甲}中篇云:“三祀十有二月,\CJKunderline{伊尹}以冕服奉嗣王。”則是除喪即吉,明十二月服終。\CJKunderwave{禮記}稱:“三年之喪,二十五月而畢。”知此年十一月湯崩,此祠先王是“湯崩逾月,\CJKunderline{太甲}即位,奠殯而告”也。此“奠殯而告”,亦如周康王受顧命屍於天子。春秋之世既有奠殯即位、逾年即位,此逾月即位當奠殯即位也。此言“\CJKunderline{伊尹}祠於先王”,是特設祀也,“嗣王祗見厥祖”是始見祖也。特設祀禮而王始見祖,明是初即王位,告殯為喪主也。 \par}

{\noindent\shu\zihao{5}\fzkt “惟元祀”。正義曰:“\CJKunderline{伊尹}祠於先王”,謂祭湯也。“奉嗣王祗見厥祖”,謂見湯也。故傳解“祠先王”為“奠殯而告”,“見厥祖”為“居位主喪”,“群后咸在”為“在位次”,皆述在喪之事,是言“祠”是奠也。祠喪於殯,斂、祭皆名為奠,虞祔卒哭始名為祭。知“祠”非宗廟者,“元祀”即是初喪之時,未得祠廟,且湯之父祖不追為王,所言“先王”惟有湯耳,故知“祠”實是奠,非祠宗廟也。祠之與奠有大小耳,祠則有主有屍,其禮大;奠則奠器而已,其禮小。奠、祠俱是享神,故可以“祠”言奠,亦由於時猶質,未有節文,周時則祠、奠有異,故傳解“祠”為奠耳。 \par}

奉嗣王祗見厥祖,\footnote{居位主喪。見,賢遍反。}侯甸群后咸在,\footnote{在位次。甸,徒遍反。}百官緫己以聽冢宰。\footnote{\CJKunderline{伊尹}制百官,以三公攝冢宰。緫音揔。}\CJKunderline{伊尹}乃明言烈祖之成德,以訓於王。\footnote{湯有功烈之祖,故稱焉。}

{\noindent\zhuan\zihao{6}\fzbyks 傳“湯有”至“稱焉”。正義曰:“湯有功烈之祖”,\CJKunderwave{毛詩}傳文也。“烈”訓業也,湯有定天下之功業,為商家一代之大祖,故以“烈祖”稱焉。 \par}

曰:“嗚呼!古有夏先後,方懋厥德,罔有天災。\footnote{先君謂\CJKunderline{禹}以下、少康以上賢王。言能以德禳災。少,詩照反。上,時掌反。禳,如羊反。}

{\noindent\zhuan\zihao{6}\fzbyks 傳“先君”至“禳災”。正義曰:有夏先君,總指桀之上世,有德之王皆是也。傳舉聖賢者言“\CJKunderline{禹}已下、少康已上”,惟當\CJKunderline{禹}與啟及少康耳。\CJKunderwave{魯語}云:“杼能師\CJKunderline{禹}者也。”杼少康之子,傳蓋以其德衰薄,故斷自少康已上耳。由勉行其德,故無有天災。言能以德禳災也。 \par}

山川鬼神,亦莫不寧,\footnote{莫,無也。言皆安之。}暨鳥獸魚鱉咸若。\footnote{雖微物皆順之,明其餘無不順。暨,其器反。鱉,必滅反。}

{\noindent\shu\zihao{5}\fzkt “山川”至“咸若”。正義曰:“山川鬼神”,謂山川之鬼神也。“亦莫不寧”者,謂鬼神安人君之政。政善則神安之,神安之則降福人君,無妖孽也。“鳥獸魚鱉咸若”者,謂人君順禽魚,君政善而順彼性,取之有時,不夭殺也。鳥獸在陸,魚鱉在水,水陸所生微細之物,人君為政皆順之,明其餘無不順也。 \par}

於其子孫弗率,皇天降災,假手於我有命,\footnote{言桀不循其祖道,故天下禍災,藉手於我有命商王誅討之。}造攻自鳴條,朕哉自亳。\footnote{造、哉,皆始也。始攻桀伐無道,由我始修德於亳。亳,旁各反,徐扶各反。}

{\noindent\shu\zihao{5}\fzkt “於其”至“自亳”。正義曰:“於其子孫”,於有夏先君之子孫,謂桀也。“不循其祖之道,天下禍災”,謂滅其國而誅其身也。天不能自誅於桀,故藉手於我有命之人,謂\CJKunderline{成湯}也。言湯有天命,將為天子,就湯藉手使誅桀也。既受天命誅桀,始攻從鳴條之地而敗之。天所以命我者,由湯始自修德於亳故也。 \par}

惟我商王,布昭聖武,代虐以寬,兆民允懷。\footnote{言湯布明武德,以寬政代桀虐政,兆民以此皆信懷我商王之德。}今王嗣厥德,罔不在初。\footnote{言善惡之由無不在初,欲其慎始。}立愛惟親,立敬惟長,始於家邦,終於四海。\footnote{言立愛敬之道,始於親長,則家國並化,終洽四海。長,丁丈反。}

{\noindent\shu\zihao{5}\fzkt “立愛”至“四海”。正義曰:王者之馭天下,撫兆人,惟愛敬二事而已。\CJKunderwave{孝經·天子之章}盛論愛敬之事,言天子當用愛敬以接物也。行之所立,自近為始。立愛惟親,先愛其親,推之以及疏。立敬惟長,先敬其長,推之以及幼。即\CJKunderwave{孝經}所云“愛親者不敢惡於人,敬親者不敢慢於人”。是推親以及物,始則行於家國,終乃治於四海,即\CJKunderwave{孝經}所云“德教加於百姓,刑于四海”是也。所異者\CJKunderwave{孝經}論愛敬並始於親,令緣親以及疏,此分敬屬長,言從長以及幼耳。 \par}

嗚呼!先王肇修人紀,從諫弗\xpinyin*{咈},先民時若。\footnote{言湯始修為人綱紀,有過則改,從諫如流,必先民之言是順。咈,扶弗反。}

{\noindent\shu\zihao{5}\fzkt “先民時若”。正義曰:賈逵注\CJKunderwave{周語}云:“先民,古賢人也。”\CJKunderwave{魯語}雲“古曰在昔,昔曰先民”,然則先民在古昔之前,遠言之也。遠古賢人亦是民內之一人,故以“民”言之。先民之言於是順從,言其動皆法古賢也。 \par}

居上克明,\footnote{言理恕。}

{\noindent\shu\zihao{5}\fzkt “居上克明”。正義曰:見下之謂“明”,言其以理恕物,照察下情,是能明也。 \par}

為下克忠,\footnote{事上竭誠。}與人不求備,檢身若不及,\footnote{使人必器之。常如不及,恐有過。}

{\noindent\shu\zihao{5}\fzkt “檢身若不及”。正義曰:“檢”謂自攝斂也,檢敕其身,常如不及,不自大以卑人,不恃長以陵物也。 \par}

以至於有萬邦,茲惟艱哉!\footnote{言湯操心常危懼,動而無過,以至為天子,此自立之難。操,七曹反,又七報反。}敷求哲人,俾輔於爾後嗣,\footnote{布求賢智,使師輔於爾嗣王。言仁及後世。哲,本又作喆。俾,必爾反。}制官刑,儆於有位。\footnote{言湯制治官刑法,以儆戒百官。儆,居領反。}曰:‘敢有恆舞於宮,酣歌於室,時謂巫風。\footnote{常舞則荒淫。樂酒曰酣,酣歌則廢德。事鬼神曰巫。言無政。酣,戶甘反。巫音無。樂音洛。}敢有殉於貨色,恆於遊畋,時謂淫風。\footnote{殉,求也。昧求財貨美色,常遊戲畋獵,是淫過之風俗。殉,辭俊反,徐辭荀反。畋音田。}


{\noindent\zhuan\zihao{6}\fzbyks 傳“常舞”至“無政”。正義曰:酣歌常舞併為耽樂無度,荒淫廢德,俱是敗亂政事,其為愆過不甚異也。恆舞酣歌乃為愆耳,若不恆舞、不酣歌非為過也。“樂酒曰酣”,言耽酒以自樂也。\CJKunderwave{說文}亦云:“酣,樂酒也。”\CJKunderwave{楚語}云:“民之精爽不攜貳者,則明神降之,在男曰覡,在女曰巫。”又\CJKunderwave{周禮}有男巫女巫之官,皆掌接神,故“事鬼神曰巫”也。廢棄德義,專為歌舞,似巫事鬼神然,言其無政也。 \par}

{\noindent\zhuan\zihao{6}\fzbyks 傳“殉求”至“風俗”。正義曰:“殉”者心循其事,是貪求之意,故為求也。志在得之,不顧禮義,“昧求”謂貪昧以求之。\CJKunderwave{無逸}雲“於遊、於畋”,是“遊”與“畋”別,故為遊戲與畋獵為之無度,是淫過之風俗也。 \par}

敢有侮聖言,逆忠直,遠耆德,比頑童,時謂亂風。\footnote{狎侮聖人之言而不行,拒逆忠直之規而不納,耆年有德疏遠之,童稚頑嚚親比之,是荒亂之風俗。遠,於萬反,注同。耆,巨夷反。比,毗志反,徐扶至反。稚,直利反。嚚,魚巾反。}惟茲三風十愆,卿士有一於身,家必喪;\footnote{有一過則德義廢,失位亡家之道。愆,去幹反。喪如字,又息浪。}邦君有一於身,國必亡。\footnote{諸侯犯此,國亡之道。}臣下不匡,其刑墨,具訓於蒙士。’\footnote{邦君卿士則以爭臣自匡正。臣不正君,服墨刑,鑿其頟,涅以墨。蒙士,例謂下士,士以爭友僕隸自匡正。爭,諫爭之爭。鑿,在洛反。頟,魚白反。涅,乃結反。未,郎計反。}

{\noindent\zhuan\zihao{6}\fzbyks 傳“狎侮”至“風俗”。正義曰:“侮”謂輕慢,“狎”謂慣忽,故傳以“狎”配“侮”而言之。\CJKunderwave{旅獒}雲“德盛不狎侮”,是“狎”、“侮”意相類也。 \par}

{\noindent\zhuan\zihao{6}\fzbyks 傳“邦君”至“匡正”。正義曰:言十愆有一,則亡國喪家,邦君卿士慮其喪亡之故,則宜以爭臣自匡正。犯顏而諫,臣之所難,故設不諫之刑以勵臣下,故言“臣不正君,則服墨刑”。墨刑,五刑之輕者。謂“鑿其頟,涅以墨”,\CJKunderwave{司刑}所謂“墨罪五百”者也。“蒙”謂蒙稚,卑小之稱,故“蒙士例謂下士”也。顧氏亦以為“蒙”謂蒙暗之士,“例”字宜從下讀,言此等流例謂下士也。 \par}

{\noindent\shu\zihao{5}\fzkt “曰敢有”至“蒙士”。正義曰:此皆湯所制治官之刑,以儆戒百官之言也。“三風十愆”,謂巫風二,舞也,歌也;淫風四,貨也,色也,遊也,畋也;與亂風四為十愆也。舞及遊、畋,得有時為之,而不可常然,故三事特言“恆”也。歌則可矣,不可樂酒而歌,故以“酣”配之。巫以歌舞事神,故歌舞為巫覡之風俗也。貨色人所貪慾,宜其以義自節,而不可專心殉求,故言“殉於貨色”。心殉貨色,常為遊畋,是謂淫過之風俗也。侮慢聖人之言,拒逆忠直之諫,疏遠耆年有德,親比頑愚幼童,愛惡憎善,國必荒亂,故為“荒亂之風俗”也。此“三風十愆”,雖惡有大小,但有一於身,皆喪國亡家,故各從其類,相配為風俗。“臣下不匡,其刑墨”,言臣無貴賤,皆當匡正君也。“具訓於蒙士”者,謂湯制官刑,非直教訓邦君卿大夫等,使之受諫,亦備具教訓下士,使受諫也。 \par}

“嗚呼!嗣王祇厥身,念哉!\footnote{言當敬身,念祖德。}聖謨洋洋,嘉言孔彰。\footnote{洋洋,美善。言甚明可法。洋音羊,徐音翔。}

{\noindent\shu\zihao{5}\fzkt “聖謨”至“孔彰”。正義曰:此嘆聖人之謨洋洋美善者,謂上湯作官刑,所言三風十愆,令受下之諫,是善言甚明可法也。 \par}

惟上帝不常,作善降之百祥,作不善降之百殃。\footnote{祥,善也。天之禍福,惟善惡所在,不常在一家。}爾惟德罔小,萬邦惟慶。\footnote{修德無小,則天下賚慶。賚,力代反。}爾惟不德罔大,墜厥宗。”\footnote{苟為不德無大,言惡有類,以類相致,必墜失宗廟。此\CJKunderline{伊尹}至忠之訓。}


{\noindent\zhuan\zihao{6}\fzbyks 傳“苟為”至“之訓”。正義曰:“爾惟德”,謂修德以善也。“爾惟不德”,謂不修德為惡也。\CJKunderwave{易·繫辭}曰:“善不積不足以成名,惡不積不足以滅身。”乃謂大善始為福,大惡乃成禍。此訓作勸誘之辭,言為善無小,小善萬邦猶慶,況大善乎?而為惡無,大言小惡猶墜厥宗,況大惡乎?此經二事辭反而意同也。傳“言惡有類”者,解小惡墜宗之意。初為小惡,小惡有族類,以類相致,至於大惡,若致於大惡,必墜失宗廟。言至於大惡乃墜,非小惡即能墜也。\CJKunderwave{晉語}云:“趙文子冠,見韓獻子,曰:‘戒之,此謂成人。成人在始,始與善,善進,不善蔑由至矣。始與不善,不善進,善亦蔑由至矣。’”言惡有類,以類相致也。今\CJKunderline{太甲}初立,恐其親近惡人,以惡類相致禍害,故以言戒之。此是\CJKunderline{伊尹}至忠之訓也。 \par}

{\noindent\shu\zihao{5}\fzkt “爾惟”至“厥宗”。正義曰:又戒王,爾惟修德而為善。德無小,德雖小猶萬邦賴慶,況大善乎?爾惟不德而為惡,惡無大,惡雖小猶墜失其宗廟,況大惡乎?。 \par}

肆命\footnote{陳天命以戒\CJKunderline{太甲},亡。}

徂後\footnote{陳往古明君以戒,亡。}

\section{太甲上第五【偽】}


\CJKunderline{太甲}既立,不明,\footnote{不用\CJKunderline{伊尹}之訓,不明居喪之禮。}\CJKunderline{伊尹}放諸桐。\footnote{湯葬地也。不知朝政,故曰放。朝,直遙反。}三年復歸於亳,思庸,\footnote{念常道。}\CJKunderline{伊尹}作\CJKunderwave{太甲}三篇。


{\noindent\zhuan\zihao{6}\fzbyks 傳“不用”至“之禮”。正義曰:此篇承\CJKunderwave{伊訓}之下,經稱“不惠於阿衡”,知“不明”者,“不用\CJKunderline{伊尹}之訓”也。“王徂桐宮”,始雲“居憂”,是未放已前不明居喪之禮也。 \par}

{\noindent\zhuan\zihao{6}\fzbyks 傳“湯葬”至“曰放”。正義曰:經稱“營於桐宮,密邇先王”,知桐是“湯葬地”也。舜放四凶,徙之遠裔;春秋放其大夫,流之他境;嫌此亦然,故辨之雲“不知朝政,故曰放”。使之遠離國都,往居墓側,與彼放逐事同,故亦稱“放”也。古者天子居喪三年,政事聽於冢宰,法當不知朝政,而云“不知朝政,曰放”者,彼正法三年之內,君雖不親政事,冢宰猶尚諮稟,此則全不知政,故為放也。 \par}

{\noindent\shu\zihao{5}\fzkt “\CJKunderline{太甲}”至“三篇”。正義曰:\CJKunderline{太甲}既立為君,不明居喪之禮,\CJKunderline{伊尹}放諸桐宮,使之思過,三年復歸於亳都,以其能改前過,思念常道故也。自初立至放而復歸,\CJKunderline{伊尹}每進言以戒之,史敘其事作\CJKunderwave{太甲}三篇。案經上篇是放桐宮之事,中下二篇是歸亳之事,此序歷言其事以總三篇也。 \par}

太甲\footnote{戒\CJKunderline{太甲},故以名篇。}

{\noindent\zhuan\zihao{6}\fzbyks 傳“戒\CJKunderline{太甲},故以名篇”。正義曰:\CJKunderwave{盤庚}、\CJKunderwave{仲丁}、\CJKunderwave{祖乙}等皆是發言之人名篇,此\CJKunderwave{太甲}及\CJKunderwave{沃丁}、\CJKunderwave{君奭}以被告之人名篇,史官不同,故以為名有異。且\CJKunderwave{伊訓}、\CJKunderwave{肆命}、\CJKunderwave{徂後}與此三篇及\CJKunderwave{咸有一德}皆是\CJKunderline{伊尹}戒\CJKunderline{太甲},不可同名\CJKunderwave{伊訓},故隨事立稱,以\CJKunderwave{太甲}名篇也。 \par}

惟嗣王不惠於阿衡,\footnote{阿,倚。衡,平。言不順\CJKunderline{伊尹}之訓。倚,於綺反。}


{\noindent\zhuan\zihao{6}\fzbyks 傳“阿倚”至“之訓”。正義曰:古人所讀“阿”、“倚”同音,故“阿”亦倚也。稱上謂之“衡”,故“衡”為平也。\CJKunderwave{詩}毛傳云:“阿衡,\CJKunderline{伊尹}也。”\CJKunderline{鄭玄}亦云:“阿,倚。衡,平也。\CJKunderline{伊尹},湯倚而取平,故以為官名。” \par}

{\noindent\shu\zihao{5}\fzkt “惟嗣”至“阿衡”。正義曰:\CJKunderline{太甲}以元年十二月即位,比至放桐之時,未知凡經幾月。必是\CJKunderline{伊尹}數諫,久而不順,方始放之,蓋以三五月矣,必是二年放之。序言“三年復歸”者,謂即位三年,非在桐宮三年也。史錄其\CJKunderline{伊尹}訓王,有\CJKunderwave{伊訓}、\CJKunderwave{肆命}、\CJKunderwave{徂後},其餘忠規切諫,固應多矣。\CJKunderline{太甲}終不從之,故言“不惠於阿衡”。史為作書發端,故言此為目也。 \par}

\CJKunderline{伊尹}作書曰:“先王顧諟天之明命,以承上下神祇。\footnote{顧謂常目在之。諟,是也。言敬奉天命以承順天地。顧音故。諟音是,\CJKunderwave{說文}:“理也。”祇,巨支反。}

{\noindent\zhuan\zihao{6}\fzbyks 傳“顧謂”至“天地”。正義曰:\CJKunderwave{說文}云:“顧,還視也。”“諟”與“是”,古今之字異,故變文為“是”也。言先王每有所行,必還回視是天之明命,謂常目在之。言其想象如目前,終常敬奉天命,以承上天下地之神祇也。 \par}

社稷宗廟,罔不祇肅。\footnote{肅,嚴也。言能嚴敬鬼神而遠之。遠,於萬反。}天監厥德,用集大命,撫綏萬方。\footnote{監,視也。天視湯德,集王命於其身,撫安天下。監,工暫反。}惟\CJKunderline{尹}躬克左右厥闢宅師,\footnote{\CJKunderline{伊尹}言能助其君居業天下之眾。闢,必亦反,徐甫亦反。}

{\noindent\shu\zihao{5}\fzkt “惟尹躬”。正義曰:\CJKunderwave{孫武兵書}及\CJKunderwave{呂氏春秋}皆雲\CJKunderline{伊尹}名摯,則“尹”非名也。今自稱“尹”者,蓋湯得之,使尹正天下,故號曰“\CJKunderline{伊尹}”;人既呼之為“尹”,故亦以“尹”自稱。禮法君前臣名,不稱名者,古人質直,不可以後代之禮約之。 \par}

肆嗣王丕承基緒。\footnote{肆,故也。言先祖勤德,致有天下,故子孫得大承基業,宜念祖修德。丕,普悲反,徐甫眉反。}惟\CJKunderline{尹}躬先見於西邑夏,自周有終,相亦惟終。\footnote{周,忠信也。言身先見夏君臣用忠信有終。夏都在亳西。先見,並如字,注同。}其後嗣王,罔克有終,相亦罔終。\footnote{言桀君臣滅先人之道德,不能終其業,以取亡。相,悉亮反。}嗣王戒哉!祗爾厥闢,闢不闢,忝厥祖。”\footnote{以不終為戒慎之至,敬其君道,則能終。忝,辱也。為君不君,則辱其祖。}王惟庸,罔念聞。\footnote{言\CJKunderline{太甲}守常不改,無念聞\CJKunderline{伊尹}之戒。}\CJKunderline{伊尹}乃言曰:“先王昧爽丕顯,坐以待旦。\footnote{爽,顯皆明也。言先王昧明思大明其德,坐以待旦而行之。昧音妹。}旁求俊彥,啟迪後人,\footnote{旁非一方。美士曰彥。開道後人。言訓戒。俊,本亦作畯。迪,大曆反。}無越厥命以自覆。\footnote{越,墜失也。無失亡祖命而不勤德,以自顛覆。越,於月反,本又作粵。覆,芳服反,注同。}慎乃儉德,惟懷永圖。\footnote{言當以儉為德,思長世之謀。}若虞機張,往省括於度,則釋。\footnote{機,弩牙也。虞,度也。度機,機有度以準望,言修德夙夜思之,明旦行之,如射先省矢括於度,釋則中。省,息井反。括,故活反。度如字。虞度,待洛反。中,丁仲反。}欽厥止,率乃祖攸行,\footnote{止謂行所安止,君止於仁,子止於孝。}惟朕以懌,萬世有辭。”\footnote{言能循汝祖所行,則我喜悅,王亦見嘆美無窮。懌音亦。}

{\noindent\zhuan\zihao{6}\fzbyks 傳“爽顯”至“行之”。正義曰:昭七年\CJKunderwave{左傳}云:“是以有精爽至於神明。”從“爽”以至於“明”,是“爽”謂未大明也。“昧”是晦冥,“爽”是未明,謂夜向晨也。\CJKunderwave{釋詁}云:“丕,大也。顯,光也。”光亦明也。於夜昧冥之時,思欲大明其德,既思得之,坐以待旦而行之。言先王身之勤也。 \par}

{\noindent\zhuan\zihao{6}\fzbyks 傳“旁非”至“訓戒”。正義曰:“旁”謂四方求之,故言“非一方”也。“美士曰彥”,\CJKunderwave{釋訓}文。舍人曰:“國有美士,為人所言道也。” \par}

{\noindent\zhuan\zihao{6}\fzbyks 傳“機弩”至“則中”。正義曰:“括”謂矢末,“機張”、“省括”,則是以射喻也。“機”是轉關,故為弩牙。“虞”訓度也。度機者,機有法度,以準望所射之物,“準望”則解經“虞”也。如射者弩以張訖機關,先省矢括與所射之物,三者於法度相當,乃後釋弦發矢,則射必中矣。言為政亦如是也。 \par}

{\noindent\shu\zihao{5}\fzkt “\CJKunderline{伊尹}”至“有辭”。正義曰:\CJKunderline{伊尹}作書以告,\CJKunderline{太甲}不念聞之。\CJKunderline{伊尹}乃又言曰:“先王以昧爽之時,思大明其德,既思得其事,則坐以待旦,明則行之。其身既勤於政,又乃旁求俊彥之人,置之於位,令以開導後人。先王之念子孫,其憂勤若是,嗣王今承其後,無得墜失其先祖之命,以自覆敗。王當慎汝儉約之德,令其以儉為德而謹慎守之,惟思其長世之謀。謀為政之事,譬若以弩射也。可準度之機已張之,又當以意往省視矢括,當於所度,則釋而放之。如是而射,則無不中矣。猶若人君所修政教,欲發命也,當以意夙夜思之,使當於民心,明旦行之,則無不當矣。王又當敬其身所安止,循汝祖之所行。若能如此,惟我以此喜悅,王於萬世常有善辭,言有聲譽,亦見嘆美無窮也。” \par}

王未克變。\footnote{未能變,不用訓。\CJKunderline{太甲}性輕脫,\CJKunderline{伊尹}至忠,所以不已。輕,遣政反。}

{\noindent\zhuan\zihao{6}\fzbyks 傳“未能”至“不已”。正義曰:“未能變”者,據在後能變,故當時為未能也。時既未變,是不用\CJKunderline{伊尹}之訓也。\CJKunderline{太甲}終為人主,非是全不可移,但體性輕脫,與物推遷,雖有心向善,而為之不固。\CJKunderline{伊尹}至忠,所以進言不已。是\CJKunderline{伊尹}知其可移,故誨之不止,冀其終從己也。 \par}

\CJKunderline{伊尹}曰:“茲乃不義,習與性成。\footnote{言習行不義,將成其性。義,本亦作誼。}予弗狎於弗順,營於桐宮,密邇先王其訓,無俾世迷。\footnote{狎,近也。經營桐墓立宮,令\CJKunderline{太甲}居之。近先王,則訓於義,無成其過,不使世人迷惑怪之。俾,必爾反,後篇同。近,附近之近。令,力呈反。}


{\noindent\zhuan\zihao{6}\fzbyks 傳“狎近”至“怪之”。正義曰:狎習是相近之義,故訓為近也。不順即是近不順也。習為不義,近於不順,則當日日益惡,必至滅亡,故\CJKunderline{伊尹}言已不得使王近於不順,故經營桐墓,立宮墓旁,令\CJKunderline{太甲}居之,不使復知朝政,身見廢退,必當改悔為善也。 \par}

{\noindent\shu\zihao{5}\fzkt “\CJKunderline{伊尹}”至“世迷”。正義曰:\CJKunderline{伊尹}以王未變,乃告於朝廷群臣曰:“此嗣王所行,乃是不義之事。習行此事,乃與性成。”言為之不已,將以不義為性也。“我不得令王近於不順之事,當營於桐墓立宮,使此近先王,當受人教訓之,無得成其過失,使後世人迷惑怪之”。 \par}

王徂桐宮居憂,\footnote{往入桐宮,居憂位。}克終允德。”\footnote{言能思念其祖,終其信德。}

{\noindent\zhuan\zihao{6}\fzbyks 傳“往入”至“憂位”。正義曰:亦既不知朝政之事,惟行居喪之禮。“居憂位”謂服治喪禮也。\CJKunderline{伊尹}亦使兵士衛之,選賢俊教之,故\CJKunderline{太甲}能終信德也。 \par}

\section{太甲中第六【偽】}


惟三祀十有二月朔,\footnote{湯以元年十一月崩,至此二十六月,三年服闋。闋,苦穴反。}\CJKunderline{伊尹}以冕服奉嗣王歸於亳。\footnote{冕,冠也。逾月即吉服。冕音免。}

{\noindent\shu\zihao{5}\fzkt “惟三”至“於亳”。正義曰:周制,君薨之年屬前君,明年始為新君之元年。此殷法,君薨之年而新君即位,即以其年為新君之元年。“惟三祀”者,\CJKunderline{太甲}即位之三年也。湯以元年十一月崩,至此年十一月為再期,除喪服也。至十二月服闋。闋,息也。如喪服息即吉服。舉事貴初始,故於十二月朔以冕服奉嗣王歸於亳。冕是在首之服,冠內之別名,冠是首服之大名,故傳以“冕”為冠。案\CJKunderwave{王制}云:“殷人冔而祭。”\CJKunderwave{大雅}云:“常服黼冔。”冔是殷之祭冠,今雲“冕”者,蓋“冕”為通名。\CJKunderwave{王制}又云:“有虞氏皇而祭,夏后氏收而祭,殷人冔而祭,周人冕而祭。”並是當代別名。殷禮不知天子幾冕,\CJKunderwave{周禮}天子六冕,大裘之冕,祭天尚質。弁師惟掌五冕,備物盡文,惟袞冕耳。此以“冕服”,蓋以袞冕之服也。顧氏云:“祥禫之制,前儒不同。”案\CJKunderwave{士虞禮}雲“期而小祥”,又“期而大祥”,“中月而禫”。王肅云:“祥月之內又禫祭,服彌寬而變彌數也。”\CJKunderwave{禮記·檀弓}雲“祥而縞,是月禫,徙月樂”。王肅云:“是祥之月而禫,禫之明月可以樂矣。”案此孔傳雲“二十六月,服闋”,則與王肅同。\CJKunderline{鄭玄}以中月為間一月,雲“祥後復更有一月而禫”,則三年之喪凡二十七月,與孔為異。 \par}

作書曰:“民非後,罔克胥匡以生。\footnote{無能相匡,故須君以生。胥,息餘反。}後非民,罔以闢四方。\footnote{須民以君四方。}皇天眷佑有商,俾嗣王克終厥德,實萬世無疆之休。”\footnote{言王能終其德,乃天之顧佑商家,是商家萬世無窮之美。疆,居良反。}王拜手稽首,曰:“予小子不明於德,自底不類。\footnote{君而稽首於臣,謝前過。類,善也。暗於德,故自致不善。底,之履反。}欲敗度,縱敗禮,以速戾於厥躬。\footnote{速,召也。言己放縱情慾,毀敗禮儀法度,以召罪於其身。敗,必邁反,徐甫邁反。縱,子用反。戾,郎計反。}

{\noindent\zhuan\zihao{6}\fzbyks 傳“速召”至“其身”。正義曰:\CJKunderwave{釋言}云:“速,徵也。徵,召也。”轉以相訓,故“速”為召也。“欲”者本之於情,“縱”者放之於外,有欲而縱之,“縱”、“欲”為一也。準法謂之“度”,體見謂之“禮”,“禮”、“度”一也。故傳並釋之,“言己放縱情Q欲Y,毀敗禮儀法度,以召罪於其身”也。 \par}

天作孽,猶可違。自作孽,不可\xpinyin*{逭}。\footnote{孽,災。逭,逃也。言天災可避,自作災不可逃。孽,魚列反。逭,胡亂反。}

{\noindent\zhuan\zihao{6}\fzbyks 傳“孽災”至“可逃”。正義曰:\CJKunderwave{洪範五行傳}有“妖、孽、眚、祥”,\CJKunderwave{漢書·五行志}說云:“凡草物之類謂之妖,妖猶夭胎,言尚微也。蟲豸之類謂之孽,孽則牙孽矣。甚則異物生,謂之眚。自外來謂之祥。”是“孽”為災初生之名,故為災也。“逭,逃也”,\CJKunderwave{釋言}文。樊光云:“行相避逃謂之逭,亦行不相逢也。”天作災者,謂若\CJKunderline{太戊}桑谷生朝,高宗雊雉升鼎耳。可修德以禳之,是“可避”也。“自作災”者,謂若桀放鳴條,紂死宣室,是“不可逃”也。據其將來,修德可去;及其已至,改亦無益。天災自作,逃否亦同。且天災亦由人行而至,非是橫加災也。此\CJKunderline{太甲}自悔之深,故言自作甚於天災耳。 \par}

既往背師保之訓,弗克於厥初,尚賴匡救之德,圖惟厥終。”\footnote{言己已往之前,不能修德於其初,今庶幾賴教訓之德,謀終於善。悔過之辭。背音佩,徐扶代反。}\CJKunderline{伊尹}拜手稽首,\footnote{拜手,首至手。}

{\noindent\zhuan\zihao{6}\fzbyks 傳“拜手,首至手”。正義曰:\CJKunderwave{周禮·太祝}:“辨九拜,一曰稽首,二曰頓首,三曰空首。”\CJKunderline{鄭玄}云:“稽首,拜頭至地也。頓首,拜頭叩地也。空首,拜頭至手,所謂拜手也。”鄭惟解此三者拜之形容,所以為異也。稽首頭至地,頭下至地也。頓首頭下至地,暫一叩之而已。此言“拜手稽首”者,初為拜頭至手,乃復申頭以至於地,至手是為“拜手”,至地乃為“稽首”。然則凡為稽首者,皆先為拜手,乃後為稽首。故“拜手稽首”連言之,諸言“拜手稽首”,義皆同也。\CJKunderwave{太祝}又云:“四曰振動,五曰吉拜,六曰兇拜,七曰奇拜,八曰褒拜,九曰肅拜。”鄭注云,振動者,戰慄變動而拜。吉拜者,拜而後稽顙,謂齊衰不杖以下者之拜。兇拜者,稽顙而後拜,即三年喪拜也。奇拜者,謂君答臣一拜也。褒拜者,謂再拜拜神與屍也。肅拜者,謂揖拜也,禮介者不拜,及婦人之拜也。\CJKunderwave{左傳}云:“天子在,寡君無所稽首。”則諸侯於天子稽首也,諸侯相於則頓首也,君於臣則空首也。 \par}

曰:“修厥身,允德協於下,惟明後。\footnote{言修其身,使信德合於群下,惟乃明君。}先王子惠困窮,民服厥命,罔有不悅。\footnote{言湯子愛睏窮之人,使皆得其所,故民心服其教令,無有不忻喜。}並其有邦厥鄰,乃曰:‘徯我後,後來無罰。’\footnote{湯俱與鄰並有國,鄰國人乃曰:“待我君來。”言忻戴。“君來無罰”,言仁惠。徯,胡啟反。}

{\noindent\shu\zihao{5}\fzkt “並其”至“無罰”。正義曰:言湯昔為諸侯之時,與湯並居其有邦國,謂諸侯之國也。此諸侯國人其與湯鄰近者,皆原以湯為君。乃言曰:“待我後,後來無罰於我。”言羨慕湯德,忻戴之也。 \par}

王懋乃德,視乃厥祖,無時豫怠。\footnote{言當勉修其德,法視其祖而行之,無為是逸豫怠惰。懋音茂。}奉先思孝,接下思恭。\footnote{以念祖德為孝,以不驕慢為恭。}視遠惟明,聽德惟聰。\footnote{言當以明視遠,以聰聽德。}朕承王之休無斁。”\footnote{王所行如此,則我承王之美無厭。斁音亦。厭,於豔反。}

{\noindent\zhuan\zihao{6}\fzbyks 傳“言當”至“聽德”。正義曰:人之心識所知在於聞見,聞見所得在於耳目,故欲言人之聰明,以視聽為主。視若不見,故言“惟明”,“明”謂監察是非也。聽若不聞,故言“惟聰”,“聰”謂識知善惡也。視戒見近迷遠,故言“視遠”。聽戒背正從邪,故言“聽德”。各準其事,相配為文。 \par}

\section{太甲下第七【偽】}


\CJKunderline{伊尹}申誥於王曰:“嗚呼!惟天無親,克敬惟親。\footnote{言天於人無有親疏,惟親能敬身者。}

{\noindent\shu\zihao{5}\fzkt “\CJKunderline{伊尹}申誥於王”。正義曰:\CJKunderline{伊尹}以至忠之心喜王改悔,重告於王,冀王大善,一篇皆誥辭也。天親克敬,民歸有仁,神享克誠,言天民與神皆歸於善也。奉天宜其敬謹,養民宜用仁恩,事神當以誠信,亦準事相配而為文也。 \par}

民罔常懷,懷於有仁。\footnote{民所歸無常,以仁政為常。}鬼神無常享,享於克誠。\footnote{言鬼神不保一人,能誠信者則享其祀。}天位艱哉!\footnote{言居天子之位難,以此三者。}德惟治,否德亂。\footnote{為政以德則治,不以德則亂。治,直吏反,注及下同。}與治同道,罔不興。與亂同事,罔不亡。\footnote{言安危在所任,治亂在所法。}

{\noindent\zhuan\zihao{6}\fzbyks 傳“言安”至“所法”。正義曰:任賢則興,鹽佞則亡,故“安危在所任”。於善則治,於惡則亂,故“治亂在所法”。總言治國則稱“道”,單指所行則言“事”。興難而亡易,道大而事小,故大言“興”而小言“亡”也。此所云“惟言治亂在所法”耳。下句雲“終始慎厥與”,言當與賢不與佞,治亂在於用臣,故傳於此言“安危在所任”也。 \par}

“終始慎厥與,惟明明後。\footnote{明慎其所與治亂之機,則為明王明君。}

{\noindent\shu\zihao{5}\fzkt “惟明明後”。正義曰:重言“明明”,言其為大明耳。傳因文重,故言“明王明君”,君、王猶是一也。 \par}

“先王惟時懋敬厥德,克配上帝。\footnote{言湯推是終始所與之難,勉修其德,能配天而行之。}今王嗣有令緒,尚監茲哉!\footnote{令,善也。繼祖善業,當夙夜庶幾視祖此配天之德而法之。}若升高,必自下。若陟遐,必自邇。\footnote{言善政有漸,如登高升遠必用下近為始,然後終致高遠。}無輕民事,惟難。\footnote{無輕為力役之事,必重難之乃可。}無安厥位,惟危。\footnote{言當常自危懼,以保其位。}慎終於始。\footnote{於始慮終,於終思始。}

{\noindent\shu\zihao{5}\fzkt “慎終於始”。正義曰:欲慎其終,於始即須慎之,故傳雲“於始慮終”。傳以將終戒惰,故又云“於終思始”,言終始皆當慎也。 \par}

有言逆於汝心,必求諸道。\footnote{人以言咈違汝心,必以道義求其意,勿拒逆之。}有言遜於汝志,必求諸非道。\footnote{遜,順也。言順汝心,必以非道察之,勿以自臧。}嗚呼!弗慮胡獲?弗為胡成?一人元良,萬邦以貞。\footnote{胡,何。貞,正也。言常念慮道德,則得道德;念為善政,則成善政。一人,天子。天子有大善,則天下得其正。}

{\noindent\zhuan\zihao{6}\fzbyks 傳“胡何”至“其正”。正義曰:“胡”之與“何”,方言之異耳。\CJKunderwave{易}彖、象皆以“貞”為正也。\CJKunderline{伊尹}此言,勸王為善,“弗慮”、“弗為”,必是善事。人君善事,惟有道德政教。言不慮何獲,是念慮有所得,知心所念慮是道德也。不為何成,則為之有所成,則知心所念是為善政也。謂天子為“一人”者,其義有二。一則天子自稱“一人”,是為謙辭,言己是人中之一耳。一則臣下謂天子為“一人”,是為尊稱,言天下惟一人而已 \par}

君罔以辯言亂舊政,\footnote{利口覆國家,故特慎焉。}臣罔以寵利居成功\footnote{成功不退,其志無限,故為之極以安之。},邦其永孚於休。”\footnote{言君臣各以其道,則國長信保於美。}

{\noindent\zhuan\zihao{6}\fzbyks 傳“成功”至“安之”。正義曰:四時之序,成功者退。臣既成功,不知退謝,其志貪慾無限,其君不堪所求,或有怨恨之心,君懼其謀,必生誅殺之計,自古以來,人臣有功不退者皆喪家滅族者眾矣。經稱臣無以寵利居成功者,為之限極以安之也。\CJKunderline{伊尹}告君而言及臣事者,雖復泛說大理,亦見已有退心也。 \par}

\section{咸有一德第八【偽】}


\CJKunderline{伊尹}作\CJKunderwave{咸有一德}。\footnote{言君臣皆有純一之德,以戒\CJKunderline{太甲}。}

{\noindent\shu\zihao{5}\fzkt “\CJKunderline{伊尹}作\CJKunderwave{咸有一德}”。正義曰:\CJKunderline{太甲}既歸於亳,\CJKunderline{伊尹}致仕而退,恐\CJKunderline{太甲}德不純一,故作此篇以戒之。經稱尹躬及湯咸有一德,言己君臣皆有純一之德,戒\CJKunderline{太甲}使君臣亦然。此主戒\CJKunderline{太甲}而言臣有一德者,欲令\CJKunderline{太甲}亦任一德之臣。經雲“任官惟賢材,左右惟其人”,是戒\CJKunderline{太甲}使善用臣也。\CJKunderline{伊尹}既放\CJKunderline{太甲},又迎而復之,是\CJKunderline{伊尹}有純一之德,己為\CJKunderline{太甲}所信,是己君臣純一,欲令\CJKunderline{太甲}法之。 \par}

咸有一德\footnote{即政之後恐其不一,故以戒之。}

{\noindent\shu\zihao{5}\fzkt “咸有一德”。正義曰:此篇終始皆言一德之事,發首至“陳戒於德”敘其作戒之由,已下皆戒辭也。“德”者,得也,內得於心,行得其理,既得其理,執之必固,不為邪見更致差貳,是之謂“一德”也。而凡庸之主,監不周物,志既少決,性復多疑,與智者謀之,與愚者敗之,則是二三其德,不為一也。經云:“德惟一,動罔不吉。德二三,動罔不兇。”是不二三則為一德也。又曰:“終始惟一,時乃日新。”言守一必須固也。\CJKunderline{太甲}新始即政,\CJKunderline{伊尹}恐其二三,故專以一德為戒。 \par}

\CJKunderline{伊尹}既復政厥闢,\footnote{還政\CJKunderline{太甲}。}將告歸,乃陳戒於德。\footnote{告老歸邑,陳德以戒。}


{\noindent\zhuan\zihao{6}\fzbyks 傳“告老”至“以戒”。正義曰:\CJKunderline{伊尹}湯之上相,位為三公,必封為國君。又受邑於畿內,告老致政事於君,欲歸私邑以自安。將離王朝,故陳戒以德也。\CJKunderwave{無逸}雲“肆祖甲之享國三十三年”,傳稱祖甲即\CJKunderline{太甲}也。\CJKunderwave{殷本紀}云:“\CJKunderline{太甲}崩,子\CJKunderline{沃丁}立。”\CJKunderwave{沃丁}序云:“\CJKunderline{沃丁}既葬\CJKunderline{伊尹}於亳。”則\CJKunderline{伊尹}卒在\CJKunderline{沃丁}之世。湯為諸侯之時已得\CJKunderline{伊尹},此至\CJKunderline{沃丁}始卒,\CJKunderline{伊尹}壽年百有餘歲。此告歸之時,已應七十左右也。\CJKunderwave{殷本紀}云:“\CJKunderline{太甲}既立三年,\CJKunderline{伊尹}放之於桐宮。居桐宮三年,悔過反善,\CJKunderline{伊尹}乃迎而受之政。”謂\CJKunderline{太甲}歸亳之歲已為即位六年,與此經相違,馬遷之說妄也。\CJKunderwave{紀年}云,殷仲壬“即位,居亳,其卿士\CJKunderline{伊尹}”。仲壬崩,\CJKunderline{伊尹}乃放\CJKunderline{太甲}於桐而自立也。\CJKunderline{伊尹}即位於\CJKunderline{太甲}七年。\CJKunderline{太甲}潛出自桐,殺\CJKunderline{伊尹},乃立其子\CJKunderline{伊陟}、伊奮,命復其父之田宅而中分之。案此經序\CJKunderline{伊尹}奉\CJKunderline{太甲}歸於亳,其文甚明。\CJKunderwave{左傳}又稱“\CJKunderline{伊尹}放\CJKunderline{太甲}而相之”,\CJKunderwave{孟子}雲“有\CJKunderline{伊尹}之志則可,無\CJKunderline{伊尹}之志則篡”,\CJKunderline{伊尹}不肯自立,\CJKunderline{太甲}不殺\CJKunderline{伊尹}也。必若\CJKunderline{伊尹}放君自立,\CJKunderline{太甲}起而殺之,則\CJKunderline{伊尹}死有餘罪,義當汙宮滅族,\CJKunderline{太甲}何所感德而復立其子,還其田宅乎?\CJKunderwave{紀年}之書,晉太康八年汲郡民發魏安僖王冢得之,蓋當時流俗有此妄說,故其書因記之耳。 \par}

{\noindent\shu\zihao{5}\fzkt “\CJKunderline{伊尹}”至“於德”。正義曰:自\CJKunderline{太甲}居桐,而\CJKunderline{伊尹}秉政。\CJKunderline{太甲}既歸於亳,\CJKunderline{伊尹}還政其君,將欲告老歸其私邑,乃陳言戒王於德,以一德戒王也。\CJKunderline{太甲}既得復歸,\CJKunderline{伊尹}即應還政,其告歸陳戒,未知在何年也。下雲“今嗣王新服厥命”,則是初始即政,蓋\CJKunderline{太甲}居亳之後即告老也。\CJKunderwave{君奭}云:“在\CJKunderline{太甲},時則有若保衡。”保衡,\CJKunderline{伊尹}也。襄二十一年\CJKunderwave{左傳}云:“\CJKunderline{伊尹}放\CJKunderline{太甲}而相之,卒無怨色。”則\CJKunderline{伊尹}又相\CJKunderline{太甲}。蓋\CJKunderline{伊尹}此時將欲告歸,\CJKunderline{太甲}又留之為相,如成王之留周公,不得歸也。 \par}

曰:“嗚呼!天難\xpinyin*{諶},命靡常。\footnote{以其無常,故難信。諶,徐巿林反。}常厥德,保厥位。厥德匪常,九有以亡。\footnote{人能常其德,則安其位。九有,諸侯。桀不能常其德,湯伐而兼之。}

{\noindent\shu\zihao{5}\fzkt “九有以亡”。正義曰:\CJKunderwave{毛詩}傳云:“九有,九州也。”此傳雲“九有,諸侯”,謂九州所有之諸侯。\CJKunderline{伊尹}此言,泛說大理,未指夏桀,但傳顧下文比桀,為此言之驗,故云“桀不能常其德,湯伐而兼之”。 \par}

夏王弗克庸德,慢神虐民。\footnote{言桀不能常其德,不敬神明,不恤下民。}皇天弗保,監於萬方,啟迪有命,\footnote{言天不安桀所為,廣視萬方,有天命者開道之。}眷求一德,俾作神主。\footnote{天求一德,使伐桀為天地神祇之主。}惟\CJKunderline{尹}躬暨\CJKunderline{湯},咸有一德,克享天心,受天明命,\footnote{享,當也。所徵無敵,謂之受天命。}

{\noindent\zhuan\zihao{6}\fzbyks 傳“享當”至“天命”。正義曰:德當神意,神乃享之,故以“享”為當也。天道遠而人道近,天之命人,非有言辭文誥,正以神明祐之,使之所徵無敵,謂之受天命也。緯候之書乃稱有黃龍白龜白魚赤雀負圖銜書以授聖人,正典無其事也。漢自哀平之間,緯候始起,假託鬼神,妄稱祥瑞,孔時未有其說,縱使時已有之,亦非孔所信也。 \par}

以有九有之師,爰革夏正。\footnote{爰,於也。於得九有之眾,遂伐夏勝之,改其正。}非天私我有商,惟天佑於一德。\footnote{非天私商而王之,佑助一德,所以王。王,於況反,下同;或如字。}非商求於下民,惟民歸於一德。\footnote{非商以力求民,民自歸於一德。}德惟一,動罔不吉。德二三,動罔不兇。\footnote{二三,言不一。}惟吉凶不僣在人,惟天降災祥在德。\footnote{行善則吉,行惡則兇,是不差。德一,天降之善;不一,天降之災;是在德。僣,子念反。}

{\noindent\shu\zihao{5}\fzkt “惟吉”至“在德”。正義曰:指其已然,則為“吉凶”;言其徵兆,則曰“災祥”;其事不甚異也。吉凶已成之事,指人言之,故曰“在人”。災祥未至之徵,行之所招,故言“在德”。“在德”謂為德有一與不一,“在人”謂人行有善與不善也。吉凶已在其身,故不言來處;災祥自外而至,故言“天降”;其實吉凶亦天降也。 \par}

“今嗣王新服厥命,惟新厥德。\footnote{其命,王命。新其德,戒勿怠。}終始惟一,時乃日新。\footnote{言德行終始不衰殺,是乃日新之義。行,下孟反。殺,色界反,衰微也,殺害也,言小小害也。}任官惟賢材,左右惟其人。\footnote{官賢才而任之,非賢材不可任。選左右,必忠良。不忠良,非其人。}


{\noindent\zhuan\zihao{6}\fzbyks 傳“其命”至“勿怠”。正義曰:\CJKunderwave{說命}云:“王言惟作命。”成十八年\CJKunderwave{左傳}云:“人之求君,使出命也。”是言人君職在發命。“新服厥命”,新始服行王命,故云“其命,王命”也。“新其德”者,勤行其事,日日益新,戒王勿懈怠也。 \par}

{\noindent\zhuan\zihao{6}\fzbyks 傳“言德”至“之義”。正義曰:“日新”者,日日益新也。若今日勤而明日惰,昨日是而今日非,自旁觀之,則有新有舊。言王德行終始皆同,不有衰殺,從旁觀之,每日益新,是乃“日新”之義也。 \par}

{\noindent\zhuan\zihao{6}\fzbyks 傳“官賢”至“其人”。正義曰:“任官”謂任人以官,故云“官賢才而任之”,言官用賢才而委任之。\CJKunderwave{詩序}雲“任賢使能”,非賢才不可任也。\CJKunderwave{冏命}云:“小大之臣,咸懷忠良。”故言“選左右,必忠良”,不忠良,即是非其人。“任官”是用人為官,“左右”亦是任而用之,故言“選左右”也。直言其人,“人”字不見,故據\CJKunderwave{冏命}之文,以“忠良”充之。 \par}

臣為上為德,為下為民。\footnote{言臣奉上佈德,順下訓民,不可官所私,任非其人。為上,於偽反,下“為民”同。為德,如字,下“為下”同,徐皆於偽反。}其難其慎,惟和惟一。\footnote{其難無以為易,其慎無以輕之,群臣當和一心以事君,政乃善。易,以豉反。}

{\noindent\zhuan\zihao{6}\fzbyks 傳“言臣”至“其人”。正義曰:“言臣奉上佈德”者,“奉上”謂奉為在上,解經“為上”也;“佈德”者謂布為道德,解經“為德”也。“順下訓民”者,“順下”謂卑順以為臣下,解經“為下”也;“訓民”者,謂以善道訓助下民,解經“為民”也。顧氏亦同此解。 \par}

{\noindent\zhuan\zihao{6}\fzbyks 傳“其難”至“乃善”。正義曰:此經申上臣事既所為如此,其難無以為易,其慎無以輕忽之,戒臣無得輕易臣之職也。既事不可輕,宜和協奉上,群臣當一心以事君,如此政乃善耳。一心即一德,言臣亦當一德也。 \par}

{\noindent\shu\zihao{5}\fzkt “今嗣王”至“惟一”。正義曰:上既言“在德”,此指戒嗣王,今新始服其王命,惟當新其所行之德。所云“新”者,終始所行,惟常如一,無有衰殺之時,是乃“日新”也。王既身行一德,臣亦當然。任人為官,惟用其賢材。輔弼左右,惟當用其忠良之人,乃可為左右耳。此“任官”、“左右”即王之臣也。臣之為用,所施多矣。何者?言臣之助為在上,當施為道德;身為臣下,當須助為於民也。臣之既當為君,又須為民,故不可任非其才,用非其人。此臣之所職,其事甚難,無得以為易。其事須慎,無得輕忽。為臣之難如此,惟當群臣和順,惟當共秉一心,以此事君,然後政乃善耳。言君臣宜皆有一德。 \par}

德無常師,主善為師。\footnote{德非一方,以善為主,乃可師。}善無常主,協於克一。\footnote{言以合於能一為常德。}俾萬姓咸曰:‘大哉!王言。’\footnote{一德之言,故曰大。}又曰:‘一哉!王心。’\footnote{能一德,則一心。}克綏先王之祿,永厎烝民之生。\footnote{言為王而令萬姓如此,則能保安先王之寵祿,長致眾民所以自生之道,是明王之事。烝,之丞反。}嗚呼!七世之廟,可以觀德。\footnote{天子立七廟,有德之王則為祖宗,其廟不毀,故可觀德。}萬夫之長,可以觀政。\footnote{能整齊萬夫,其政可知。長,丁丈反。}


{\noindent\zhuan\zihao{6}\fzbyks 傳“天子”至“觀德”。正義曰:天子立七廟,是其常事,其有德之王則列為祖宗,雖七廟親盡,而其廟不毀,故於七廟之外可以觀德矣。下雲“萬夫之長,可以觀政”,謂觀其萬夫之長。此“七世之廟,可以觀德”,謂觀七世之外。文雖同而義小異耳,所謂辭不害意。漢氏以來,論七廟諸多矣,其文見於記傳,\CJKunderwave{禮器}、\CJKunderwave{家語}、\CJKunderwave{荀卿書}、\CJKunderwave{穀梁傳}皆曰天子立七廟,以為天子常法,不辨其廟之名。\CJKunderwave{王制}云:“天子七廟,三昭三穆,與太祖之廟而七。”\CJKunderwave{祭法}云:“王立七廟,曰考廟,曰王考廟,曰皇考廟,曰顯考廟,曰祖考廟,皆月祭之。遠廟為祧,有二祧,享嘗乃止。”\CJKunderwave{漢書}韋玄成議曰:“周之所以七廟者,后稷始封,文王武王受命而王,是以三廟不毀,與親廟四而七也。”\CJKunderline{鄭玄}用此為說。惟周有七廟,二祧為文王武王廟也,故\CJKunderline{鄭玄}\CJKunderwave{王制}注云:“此周制。七者,太祖及文王武王二祧,與親廟四。太祖,后稷也。殷則六廟,契及湯與二昭二穆。夏則五廟,無太祖,\CJKunderline{禹}與二昭二穆而已。”良由不見古文,故為此謬說。此篇乃是\CJKunderwave{商書},已雲“七世之廟”,則天子立七廟,王者常禮,非獨周人始有七廟也。文武則為祖宗,不在昭穆之數,\CJKunderwave{王制}之文不得雲“三昭三穆”也。劉歆、馬融、王肅雖則不見古文,皆以七廟為天子常禮。所言二祧者,王肅以為高祖之父及祖也,並高祖已下共為三昭三穆耳。\CJKunderwave{喪服小記}云:“王者禘其祖之所自出,以其祖配之而立四廟。庶子王亦如之。”所以不同者,王肅等以為受命之王是初基之王,故立四廟。“庶子王”者,謂庶子之後自外繼立,雖承正統之後,自更別立己之高祖已下之廟,猶若漢宣帝別立戾太子悼皇考廟之類也。或可庶子初基為王,亦得與嫡子同,正立四廟也。 \par}

{\noindent\shu\zihao{5}\fzkt “嗚呼”至“觀政”。正義曰:此又勸王修德以立後世之名。禮王者祖有功、宗有德,雖七世之外其廟不毀。嗚呼!七世之廟其外則猶有不毀者,可以觀知其有明德也。立德在於為政,萬夫之長能使其整齊,可以觀知其善政也。萬夫之長尚爾,況天子乎?觀王使為善政也。 \par}

後非民罔使,民非後罔事。\footnote{君以使民自尊,民以事君自生。}無自廣以狹人,匹夫匹婦,不獲自盡,民主罔與成厥功。”\footnote{上有狹人之心,則下無所自盡矣。言先盡其心,然後乃能盡其力,人君所以成功。狹,戶夾反。盡,徐子忍反,注同。}

{\noindent\shu\zihao{5}\fzkt “無自”至“厥功”。正義曰:既言君民相須,又戒王虛心待物。凡為人主,無得自為廣大,以狹小前人,勿自以所知為大,謂彼所知為小。若謂彼狹小,必待之輕薄。彼知遇薄,則意不自盡。匹夫匹婦不得自盡其意,則在下不肯親上,在上不得下情,如是則人主無與成其功也。 \par}

沃丁既葬\CJKunderline{伊尹}於亳,\footnote{沃丁,\CJKunderline{太甲}子。\CJKunderline{伊尹}既致仕老終,以三公禮葬。沃,烏毒反,徐於毒反。}\CJKunderline{咎單}遂訓\CJKunderline{伊尹}事,\footnote{訓暢其所行功德之事。作\CJKunderwave{沃丁}。\CJKunderline{咎單},忠臣名。作此篇以戒也,亡。}

{\noindent\zhuan\zihao{6}\fzbyks 傳“\CJKunderline{沃丁}”至“禮葬”。正義曰:\CJKunderwave{世本}、\CJKunderwave{本紀}皆雲“\CJKunderline{太甲}崩,子\CJKunderline{沃丁}立”,是為\CJKunderline{太甲}子也。\CJKunderline{伊尹}本是三公,上篇言其告歸,知“致仕老終,以三公禮葬”。皇甫謐云:“\CJKunderline{沃丁}八年,\CJKunderline{伊尹}卒,卒年百有餘歲。大霧三日。\CJKunderline{沃丁}葬之以天子禮,葬祀以太牢,親臨喪,以報大德。”晉文請遂,襄王不許,\CJKunderline{沃丁}不當以天子之禮葬\CJKunderline{伊尹}也。孔言三公禮葬,未必有文,要情事當然也。 \par}

{\noindent\shu\zihao{5}\fzkt “\CJKunderline{沃丁}”至“作\CJKunderline{沃丁}”。正義曰:\CJKunderline{沃丁},殷王名也。“\CJKunderline{沃丁}既葬\CJKunderline{伊尹}”,言重其賢德,備禮而葬之。\CJKunderline{咎單}以\CJKunderline{沃丁}愛慕\CJKunderline{伊尹},遂訓暢\CJKunderline{伊尹}之事以告\CJKunderline{沃丁}。史錄其事,作\CJKunderwave{沃丁}之篇。 \par}

\CJKunderline{伊陟}相\CJKunderline{大戊},\footnote{伊陟,\CJKunderline{伊尹}子。\CJKunderline{太戊},\CJKunderline{沃丁}弟之子。陟,張力反。相,息亮反。\CJKunderline{太戊},馬云:“\CJKunderline{太甲}子。”}亳有祥,桑谷共生於朝。\footnote{祥,妖怪。二木合生,七日大拱,不恭之罰。桑,蘇臧反。谷,工木反,楮也。朝,直遙反。}\CJKunderline{伊陟}贊於巫咸,作\CJKunderwave{咸乂}四篇。\footnote{贊,告也。巫咸,臣名。皆亡。巫咸,馬云:“巫,男巫也,名咸,殷之巫也。”}

{\noindent\zhuan\zihao{6}\fzbyks 傳“\CJKunderline{伊陟}”至“之子”。正義曰:“\CJKunderline{伊陟},\CJKunderline{伊尹}子”,相傳為然。\CJKunderwave{殷本紀}云:“\CJKunderline{沃丁}崩,弟太庚立。崩,子小甲立。崩,弟雍己立。崩,弟\CJKunderline{太戊}立。”是\CJKunderline{太戊}為小甲弟,太庚之子。 \par}

{\noindent\zhuan\zihao{6}\fzbyks 傳“祥妖”至“之罰”。正義曰:\CJKunderwave{漢書·五行志}云:“凡草物之類謂之妖,自外來謂之祥。”“祥”是惡事先見之徵,故為“妖怪”也。“二木合生”謂共處生也。“七日大拱”,伏生\CJKunderwave{書傳}有其文,或當別出餘書,則孔用之也。\CJKunderline{鄭玄}注\CJKunderwave{書傳}云:“兩手扼之曰拱。”生七日而見其大滿兩手也。\CJKunderwave{殷本紀}雲“一暮大拱”,言一夜即滿拱,所聞不同,故說異也。\CJKunderwave{五行傳}曰:“貌之不恭,是謂不肅。時則有青眚之祥。”\CJKunderwave{漢書·五行志}夏侯始昌、劉向等說云:“肅,敬也。內曰恭,外曰敬。人君行己,體貌不恭,怠慢驕蹇,則不能敬。木色青,故有青眚之祥。”是言木之變怪,是貌不恭之罰。人君貌不恭,天將罰之,木怪見其徵也。皇甫謐云:“\CJKunderline{太戊}問於\CJKunderline{伊陟},\CJKunderline{伊陟}曰:‘臣聞妖不勝德,帝之政事有闕。’白帝修德。\CJKunderline{太戊}退而佔之曰:‘桑谷野木而不合生於朝,意者朝亡乎?’\CJKunderline{太戊}懼,修先王之政,明養老之禮,三年而遠方重譯而至七十六國。”是言妖不勝德也。 \par}

{\noindent\zhuan\zihao{6}\fzbyks 傳“贊告”至“臣名”。正義曰:禮有贊者,皆以言告人,故“贊”為告也。\CJKunderwave{君奭}傳曰:“巫,氏也。”當以“巫”為氏,名“咸”。此言“臣名”者,言是臣之名號也。\CJKunderline{鄭玄}雲“巫咸謂之巫官”者,案\CJKunderwave{君奭}咸子又稱,賢父子併為大臣,必不世作巫官,故孔言“巫,氏”是也。 \par}

{\noindent\shu\zihao{5}\fzkt “\CJKunderline{伊陟}”至“四篇”。正義曰:\CJKunderline{伊陟}輔相\CJKunderline{太戊},於亳都之內,有不善之祥,桑谷二木共生於朝。朝非生木之處,是為不善之徵,\CJKunderline{伊陟}以此桑谷之事告於巫咸。史錄其事,作\CJKunderwave{咸乂}四篇。“乂”訓治也,言所以致妖,須治理之,故名篇為\CJKunderwave{咸乂}也。\CJKunderline{伊陟}不先告\CJKunderline{太戊}而告巫咸者,\CJKunderwave{君奭}云:“在\CJKunderline{太戊},時則有若巫咸乂王家。”則咸是賢臣,能治王事,大臣見怪而懼,先共議論,而後以告君。下篇序云:“\CJKunderline{太戊}贊於\CJKunderline{伊陟}。”明先告於巫咸,而後告\CJKunderline{太戊}。 \par}

\CJKunderline{太戊}贊於\CJKunderline{伊陟},\footnote{告以改過自新。}作\CJKunderwave{伊陟}、\CJKunderwave{原命}。\footnote{原,臣名。\CJKunderwave{原命}、\CJKunderwave{伊陟}二篇皆亡。}

{\noindent\shu\zihao{5}\fzkt “\CJKunderline{太戊}”至“原命”。正義曰:言\CJKunderline{太戊}贊於\CJKunderline{伊陟},惟告\CJKunderline{伊陟},不告原也。史錄其事,而作\CJKunderwave{伊陟}、\CJKunderwave{原命}二篇,則\CJKunderline{太戊}告\CJKunderline{伊陟},亦告原,俱以桑谷事告,故序裛以為文也。“原”是臣名而云“原命”,謂以言命原,故以“原命”名篇,猶如\CJKunderwave{冏命}、\CJKunderwave{畢命}也。 \par}

\CJKunderline{仲丁}遷於囂,\footnote{\CJKunderline{太戊}子。去亳。囂,地名。囂,五羔反。}

{\noindent\zhuan\zihao{6}\fzbyks 傳“\CJKunderline{太戊}”至“地名”。正義曰:此及下傳言\CJKunderline{仲丁}是\CJKunderline{太戊}之子,\CJKunderline{河亶甲},\CJKunderline{仲丁}弟也。\CJKunderline{祖乙},\CJKunderline{河亶甲}子,皆\CJKunderwave{世本}文也。\CJKunderline{仲丁}是\CJKunderline{太戊}之子,\CJKunderline{太戊}之時仍雲亳有祥,知\CJKunderline{仲丁}遷於囂去亳也。 \par}

{\noindent\shu\zihao{5}\fzkt “\CJKunderline{仲丁}遷於囂”。正義曰:此三篇皆是遷都之事,俱以君名名篇,並陳遷都之義,如\CJKunderwave{盤庚}之誥民也。發其舊都謂之“遷”,到彼新邑謂之“居”,“遷於囂”與“居相”亦事同也。以“\CJKunderline{河亶甲}”三字句長,不言“於”,其實亦是俱於相也。“圮於耿”者,孔意以為毀於相地,乃遷於耿地,其篇蓋言毀意,故序特言“圮”也。李顒云:“囂在陳留浚儀縣。”皇甫謐云:“\CJKunderline{仲丁}自亳徙囂,在河北也。或曰今河南敖倉,二說未知孰是也。”相地孔雲“在河北”,蓋有文而知也。謐又以耿在河東,皮氏縣耿鄉是也。 \par}

作\CJKunderwave{仲丁}。\footnote{陳遷都之義,亡。}\CJKunderline{河亶甲}居相,\footnote{\CJKunderline{仲丁}弟。相,地名,在河北。亶,丁但反。相,息亮反,在河北,今魏郡有相縣。}作\CJKunderwave{河亶甲}。\footnote{亡。}\CJKunderline{祖乙}圮於耿,\footnote{\CJKunderline{亶甲}子。圮於相,遷於耿。河水所毀曰圮。圮,備美反,徐扶鄙反。馬云:“毀也。”}作\CJKunderwave{祖乙}。\footnote{亡。}

{\noindent\zhuan\zihao{6}\fzbyks 傳“亶甲”至“曰圮”。正義曰:孔以\CJKunderline{河亶甲}居相,\CJKunderline{祖乙}即亶甲之子,故以為圮於相地,乃遷都於耿。\CJKunderwave{釋詁}云:“圮,毀也。”故云“河水所毀曰圮”。據文“圮於耿”也,知非圮毀於耿,更遷餘處,必雲圮於相地,遷於耿者,明與其上文連。上雲“遷於囂”,謂遷來向囂;“居於相”,謂居於相地,故知“圮於耿”謂遷來於耿,以文相類,故孔為此解。謂古人之言,雖尚要約,皆使言足其文,令人曉解,若圮於相,遷居於耿,經言“圮於耿”,大不辭乎?且亶甲居於相,\CJKunderline{祖乙}居耿,今為水所毀,更遷他處,故言毀於耿耳,非既毀乃遷耿也。\CJKunderwave{盤庚}云:“不常厥邑,於今五邦。”及其數之,惟有亳、囂、相、耿四處而已。知此既毀於耿,更遷一處,盤庚又自彼處而遷於殷耳。\CJKunderwave{殷本紀}云:“\CJKunderline{祖乙}遷於邢。”馬遷所為說耳。\CJKunderline{鄭玄}云:“\CJKunderline{祖乙}又去相居耿,而國為水所毀,於是修德以御之,不復徙也。錄此篇者善其國圮毀,改政而不徙。”如鄭所言,稍為文便。但上有\CJKunderwave{仲丁}、\CJKunderwave{亶甲},下有\CJKunderwave{盤庚},皆為遷事,作書述其遷意,匆陽毀而不遷,序當改文見義,不應文類遷居,更以不遷為義。汲冢古文雲“盤庚自奄遷於殷”者,蓋\CJKunderline{祖乙}圮於耿,遷於奄,盤庚自奄遷於殷,亳、囂、相、耿與此奄五邦者。此蓋不經之書,未可依信也。 \par}

%%% Local Variables:
%%% mode: latex
%%% TeX-engine: xetex
%%% TeX-master: "../Main"
%%% End:
