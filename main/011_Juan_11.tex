%% -*- coding: utf-8 -*-
%% Time-stamp: <Chen Wang: 2024-04-02 11:42:41>

% {\noindent \zhu \zihao{5} \fzbyks } -> 注 (△ ○)
% {\noindent \shu \zihao{5} \fzkt } -> 疏


\part{周書}


\chapter{卷十一}


\section{泰誓上第一【偽】}

 {\noindent\zhuan\zihao{6}\fzbyks 案:\CJKunderline{孫星衍}之\CJKunderwave{尚書今古文註疏}中\CJKunderwave{泰誓}與此文不同。 \par}

惟十有一年,\CJKunderline{武王}伐殷。\footnote{周自虞芮質厥成,諸侯並附,以為受命之年。至九年而\CJKunderline{文王}卒,\CJKunderline{武王}三年服畢,觀兵孟津,以卜諸侯伐紂之心。諸侯僉同,乃退以示弱。芮,如銳反。虞、芮,二國名。僉,七廉反。}一月戊午,師渡孟津,\footnote{十三年正月二十八日,更與諸侯期而共伐紂。孟津,地名也。}作\CJKunderwave{泰誓}三篇。\footnote{渡津乃作。}


{\noindent\zhuan\zihao{6}\fzbyks 傳“周自”至“示弱”。正義曰:\CJKunderwave{武成}篇云:“我文考\CJKunderline{文王},誕膺天命,以撫方夏。惟九年,大統未集。”則\CJKunderline{文王}以九年而卒也。\CJKunderwave{無逸}稱\CJKunderline{文王}“享國五十年”至嗣位至卒非徒九年而已。知此十一年者,\CJKunderline{文王}改稱元年,至九年而卒,至此年為十一年也。\CJKunderwave{詩}云:“虞芮質厥成。”\CJKunderwave{毛傳}稱“天下聞虞芮之訟息,歸周者四十餘國”,故知“周自虞芮質厥成,諸侯並附,以為受命之年。至九年而\CJKunderline{文王}卒”,至此十一年,\CJKunderline{武王}居父之喪“三年服畢”也。案\CJKunderwave{周書}云:“\CJKunderline{文王}受命九年,惟暮春在鎬,召太子發作文傳。”其時猶在,但未知崩月。就如暮春即崩,\CJKunderline{武王}服喪至十一年三月大祥,至四月觀兵,故今文\CJKunderwave{泰誓}亦云“四月觀兵”也。知此十一年非\CJKunderline{武王}即位之年者,\CJKunderwave{大戴禮}云“\CJKunderline{文王}十五而生\CJKunderline{武王}”,則\CJKunderline{武王}少\CJKunderline{文王}十四歲也。\CJKunderwave{禮記·文王世子}云:“\CJKunderline{文王}九十七而終,\CJKunderline{武王}九十三而終。”計其終年,\CJKunderline{文王}崩時\CJKunderline{武王}已八十三矣。八十四即位,至九十三而崩,適滿十年,不得以十三年伐紂。知此十一年者,據\CJKunderline{文王}受命而數之。必繼\CJKunderline{文王}年者,為其卒父業故也。緯候之書言受命者,謂有黃龍玄龜白魚赤雀負圖銜書以命人主,其言起于漢哀平之世,經典無文焉,孔時未有此說。\CJKunderwave{咸有一德}傳云:“所徵無敵謂之受天命。”此傳云:“諸侯並附,以為受命之年。”是孔解受命皆以人事為言,無瑞應也。\CJKunderwave{史記}亦以斷虞芮之訟為受命元年,但彼以\CJKunderline{文王}受命七年而崩,不得與孔同耳。三年之喪,二十五月而畢,故九年\CJKunderline{文王}卒,至此一年服畢。此經\CJKunderline{武王}追陳前事,云:“肆予小子發,以爾友邦冢君,觀政于商。”是十一年伐殷者,止為觀兵孟津,以卜諸侯伐紂之心,言“于商”,知亦至孟津也。 \par}

{\noindent\zhuan\zihao{6}\fzbyks 傳“十三年正月”至“伐紂”。正義曰:以“一月戊午”,乃是作誓月日。經言“十三年春,大會于孟津”,又云“戊午,次于河朔”,知此“一月戊午”是十三年正月戊午日,非是十一年正月也。序不別言“十三年”,而以“一月”接“十一年”下者,序以觀兵至而即還,略而不言月日,誓則經有“年”有“春”,故略而不言“年春”,止言“一月”,使其互相足也。戊午是二十八日,以歷推而知之,據經亦有其驗。\CJKunderwave{漢書·律歷志}載舊說云:“死魄,朔也。生魄,望也。”\CJKunderwave{武成}篇說此伐紂之事云:“惟一月壬辰,旁死魄。”則壬辰近朔而非朔,是為月二日也。二日壬辰,則此月辛卯朔矣。以次數之,知戊午是二十八日也。不言“正月”而言“一月”者,以\CJKunderwave{武成}經言“一月”,故此序同之。\CJKunderwave{武成}所以稱“一月”者,\CJKunderwave{易·革卦}彖曰:“湯武革命,順乎天而應乎人。”象曰:“君子以治歷明時。”然則改正治歷,必自\CJKunderline{武王}始矣。\CJKunderline{武王}以殷之十二月發行,正月四日殺紂,既入商郊,始改正朔,以殷之正月為周之二月。其初發時猶是殷之十二月,未為周之正月,改正在後,不可追名為“正月”,以其實是周之一月,故史以“一月”名之。顧氏以為“古史質,或雲正月,或雲一月,不與\CJKunderwave{春秋}正月同”,義或然也。\CJKunderwave{易緯}稱“\CJKunderline{文王}受命,改正朔,布王號于天下”。\CJKunderline{鄭玄}依而用之,言\CJKunderline{文王}生稱王,已改正。然天無二日,民無二王,豈得殷紂尚在而稱周王哉?若\CJKunderline{文王}身自稱王,已改正朔,則是功業成矣,\CJKunderline{武王}何得云“大勳未集”,欲卒父業也?\CJKunderwave{禮記大傳}云:“牧之野,\CJKunderline{武王}之大事也。既事而退,追王\CJKunderline{大王}亶父、\CJKunderline{王季}歷、\CJKunderline{文王}昌。”是追為王,何以得為\CJKunderline{文王}身稱王,已改正朔也?\CJKunderwave{春秋}“王正月”謂周正月也,\CJKunderwave{公羊傳}曰:“王者孰謂?謂\CJKunderline{文王}。”其意以正為\CJKunderline{文王}所改。\CJKunderwave{公羊傳}漢初俗儒之言,不足以取正也。\CJKunderwave{春秋}之“王”,自是當時之王,非改正之王。晉世有王愆期者,知其不可,注\CJKunderwave{公羊}以為春秋制,\CJKunderline{文王}指\CJKunderline{孔子}耳,非周昌也。\CJKunderwave{文王世子}稱\CJKunderline{武王}對\CJKunderline{文王}云:“西方有九國焉,群王其終撫諸。”呼\CJKunderline{文王}為“王”,是後人追為之辭,其言未必可信,亦非實也。 \par}

{\noindent\zhuan\zihao{6}\fzbyks 傳“渡津乃作”。正義曰:“孟”者,河北地名,\CJKunderwave{春秋}所謂“向盟”是也。于孟地置津,謂之“孟津”,言師渡孟津,乃作\CJKunderwave{泰誓},知三篇皆“渡津乃作”也。然則中篇獨言“戊午,次于河朔”者,三篇皆河北乃作,分為三篇耳。上篇未次時作,故言“十三年春”。中篇既次乃作,故言“戊午”之日。下篇則明日乃作,言“時厥明”。各為首引,故文不同耳。\CJKunderwave{尚書}遭秦而亡,漢初不知篇數,武帝時有太常蓼侯孔臧者,安國之從兄也,與安國書云:“時人惟聞\CJKunderwave{尚書}二十八篇,取象二十八宿,謂為信然,不知其有百篇也。”然則漢初惟有二十八篇,無\CJKunderwave{泰誓}矣。後得偽\CJKunderwave{泰誓}三篇,諸儒多疑之。\CJKunderline{馬融}\CJKunderwave{書序}曰:“\CJKunderwave{泰誓}後得,案其文似若淺露。又云:‘八百諸侯,不召自來,不期同時,不謀同辭。’及‘火復于上,至于王屋,流為雕,至五,以谷俱來。’舉火神怪,得無在子所不語中乎?又\CJKunderwave{春秋}引\CJKunderwave{泰誓}曰:‘民之所欲,天必從之。’\CJKunderwave{國語}引\CJKunderwave{泰誓}曰:‘朕夢協朕卜,襲于休祥,戎商必克。’\CJKunderwave{孟子}引\CJKunderwave{泰誓}曰:‘我武惟揚,侵于之疆,取彼兇殘,我伐用張,于湯有光。’\CJKunderwave{孫卿}引\CJKunderwave{泰誓}曰:‘獨夫受。’\CJKunderwave{禮記}引\CJKunderwave{泰誓}曰:‘予克受,非予武,惟朕文考無罪。受克予,非朕文考有罪,惟予小子無良。’今文\CJKunderwave{泰誓},皆無此語。吾見書傳多矣,所引\CJKunderwave{泰誓}而不在\CJKunderwave{泰誓}者甚多,弗復悉記,略舉五事以明之亦可知矣。”\CJKunderline{王肅}亦云:“\CJKunderwave{泰誓}近得,非其本經。”\CJKunderline{馬融}惟言後得,不知何時得之。\CJKunderwave{漢書}婁敬說高祖云:“\CJKunderline{武王}伐紂,不期而會盟津之上者八百諸侯。”偽\CJKunderwave{泰誓}有此文,不知其本出何書也。武帝時董仲舒對策云:“\CJKunderwave{書}曰:‘白魚入于王舟,有火入于王屋,流為烏。\CJKunderline{周公}曰:“復哉!復哉!”’”今引其文,是武帝之時已得之矣。李顒集註\CJKunderwave{尚書},于偽\CJKunderwave{泰誓}篇每引“孔安國曰”,計安國必不為彼偽書作傳,不知顒何由為此言。梁王兼而存之,言“本有兩\CJKunderwave{泰誓},古文\CJKunderwave{泰誓}伐紂事,聖人選為\CJKunderwave{尚書}。今文\CJKunderwave{泰誓}觀兵時事,別錄之以為\CJKunderwave{周書}”,此非辭也。彼偽書三篇,上篇觀兵時事,中下二篇亦伐紂時事,非盡觀兵時事也。且觀兵示弱即退,復何誓之有?設有其誓,不得同以\CJKunderwave{泰誓}為篇名也。 \par}

{\noindent\shu\zihao{5}\fzkt “惟十”至“三篇”。正義曰:惟\CJKunderline{文王}受命十有一年,\CJKunderline{武王}服喪既畢,舉兵伐殷,以卜諸侯伐紂之心。雖諸侯僉同,乃退以示弱。至十三年紂惡既盈,乃復往伐之。其年一月戊午之日,師渡孟津,王誓以戒眾。史敘其事,作\CJKunderwave{泰誓}三篇。 \par}

泰誓\footnote{大會以誓眾。}

{\noindent\shu\zihao{5}\fzkt 傳“大會以示眾”。正義曰:經云:“大會于孟津”,知名曰\CJKunderwave{泰誓}者,其“大會以示眾”也。\CJKunderline{王肅}云:“\CJKunderline{武王}以大道誓眾。”肅解彼偽文,故說謬耳。\CJKunderwave{湯誓}指湯為名,此不言“武誓”而別立名者,以武誓非一,故史推義作名\CJKunderwave{泰誓},見大會也。\CJKunderwave{牧誓}舉戰地,時史意也。顧氏以為:“泰者,大之極也。猶如天子諸侯之子曰太子,天子之卿曰太宰,此會中之大,故稱\CJKunderwave{泰誓}也。” \par}

惟十有三年春,大會于孟津。\footnote{三分二諸侯,及諸戎狄。此周之孟春。“惟十有三年春”或作“十有一年”,後人妄看序文輒改之。}

{\noindent\zhuan\zihao{6}\fzbyks 傳“三分”至“孟春”。正義曰:\CJKunderwave{論語}稱“三分天下有其二”,中篇言“群后以師畢會”,則周之所有諸國皆集。\CJKunderwave{牧誓}所呼有“庸、蜀、羌、髳、微、盧、彭、濮人”,知此大會,謂三分有二之諸侯及諸戎狄皆會也。序言“一月”,知此春是“周之孟春”,謂建子之月也。知者案\CJKunderwave{三統歷}以殷之十二月\CJKunderline{武王}發師,至二月甲子咸劉商王紂,彼十二月即周之正月建子之月也。 \par}

{\noindent\shu\zihao{5}\fzkt “惟十”至“孟津”。正義曰:此三篇俱是孟津之上大告諸國之君,而發首異者,此見大會誓眾,故言“大會于孟津”;中篇徇師而誓,故言“以師畢會”;下篇王更徇師,故言“大巡六師”,皆史官觀事而為作端緒耳。 \par}

王曰:“嗟!我友邦冢君,越我御事庶士,明聽誓。\footnote{冢,大。御,治也。友諸侯,親之。稱大君,尊之。下及我治事眾士,大小無不皆明聽誓。}

{\noindent\zhuan\zihao{6}\fzbyks 傳“冢大”至“聽誓”。正義曰:“冢,大”,\CJKunderwave{釋詁}文。侍御是治理之事,故通訓“御”為治也。同志為“友”,天子友諸侯,親之也。\CJKunderwave{牧誓}傳曰:“言志同滅紂。”令總呼國君皆為大君,尊之也。“下及治事眾士”,謂國君以外卿大夫及士諸掌事者。“大小無不皆明聽誓”,自士以上皆總戒之也。 \par}

惟天地萬物父母,惟人萬物之靈。\footnote{生之謂父母。靈,神也。天地所生,惟人為貴。}

{\noindent\zhuan\zihao{6}\fzbyks 傳“生之”至“為貴”。正義曰:萬物皆天地生之,故謂天地為父母也。\CJKunderwave{老子}云:“神得一以靈。”“靈”、“神”是一,故“靈”為神也。\CJKunderwave{禮運}云:“人者天地之心,五行之端也,食味別聲被色而生者也。”言人能兼此氣性,餘物則不能然。故\CJKunderwave{孝經}云:“天地之性人為貴。”此經之意,天地是萬物之父母,言天地之意,欲養萬物也。人是萬物之最靈,言其尤宜長養也。紂違天地之心而殘害人物,故言此以數之,與下句為首引也。 \par}

亶聰明,作元后,元后作民父母。\footnote{人誠聰明,則為大君,而為眾民父母。亶,丁但反。}今商王\CJKunderline{受},弗敬上天,降災下民。沈湎冒色,敢行暴虐,\footnote{沈湎嗜酒,冒亂女色,敢行酷暴,虐殺無辜。湎,面善反。冒,莫報反,注下同。嗜,市志反,\CJKunderwave{切韻}常利反。酷,苦毒反。}

{\noindent\shu\zihao{5}\fzkt 傳“沈湎”至“無辜”。正義曰:人被酒困,若沈于水,酒變其色,湎然齊同,故“沈湎”為嗜酒之狀。“冒”訓貪也,亂女色,荒也。“酷”解經之“暴”,“殺”辭經之“虐”,皆果敢為之。案\CJKunderwave{說文}云:“酷,酒厚味也。”酒味之厚必嚴烈,人之暴虐與酒嚴烈同,故謂之“酷”。 \par}

罪人以族,官人以世,\footnote{一人有罪,刑及父母兄弟妻子,言淫濫。官人不以賢才,而以父兄,所以政亂。}

{\noindent\zhuan\zihao{6}\fzbyks 傳“一人”至“政亂”。正義曰:秦政酷虐,有三族之刑,謂非止犯者之身,乃更上及其父,下及其子。經言“罪人以族”,故以三族解之。父母,前世也;兄弟及妻,當世也;子孫,後世也。一人有罪,刑及三族,言淫濫也。古者臣有大功,乃得繼世在位。而紂之官人,不以賢才,而以父兄,已濫受寵,子弟頑愚亦用,不堪其職,所以政亂。“官人以世”,惟當用其子耳,而傳兼言“兄”者,以紂為惡,或當因兄用弟,故以“兄”協句耳。 \par}

惟宮室、臺榭、陂池、侈服,以殘害于爾萬姓。\footnote{土高曰臺,有木曰榭,澤障曰陂,停水曰池,侈謂服飾過制。言匱民財力為奢麗。榭,\CJKunderwave{爾雅}云:“有木曰榭。”本又作謝。陂,彼皮反。障,之亮反。匱,其愧反。}

{\noindent\zhuan\zihao{6}\fzbyks 傳“土高”至“奢麗”。正義曰:\CJKunderwave{釋宮}云:“宮謂之室,室謂之宮。”李巡曰:“所以古今通語,明實同而兩名。”此傳不解“宮室”,義當然也。\CJKunderwave{釋宮}又云:“闍謂之臺。有木者謂之榭。”李巡曰:“臺積土為之,所以觀望也。臺上有屋謂之榭。”又云:“無室曰榭,四方而高曰臺。”孫炎曰:“榭但有堂也。”郭璞曰:“榭即今之堂堭也。”然則榭是臺上之屋,歇前無室,今之廳是也。\CJKunderwave{詩}云:“彼澤之陂。”\CJKunderwave{毛傳}云:“陂,澤障也。”障澤之水,使不流溢謂之“陂”,停水不流謂之“池”。“侈”亦奢也,謂依服採飾過于制度,言匱竭民之財力為奢麗也。顧氏亦云:“華侈服飾。”二劉以為宮室之上而加侈服。據孔傳云“服飾過制”,即謂人之服飾,二劉之說非也。\CJKunderwave{殷本紀}云:“紂厚賦稅以實鹿臺之錢,而盈巨橋之粟。益收狗馬奇物,充牣宮室。益廣沙丘苑臺,多聚野獸飛鳥置其中。大聚樂戲于沙丘,以酒為池,懸肉為林,使男女倮相逐其間。”說紂奢侈之事,書傳多矣。 \par}

焚炙忠良,刳剔孕婦。\footnote{忠良無罪焚炙之,懷子之婦刳剔視之。言暴虐。刳,口胡反。剔,他歷反。孕,以證反,徐養證反。}

{\noindent\zhuan\zihao{6}\fzbyks 傳“忠良”至“暴虐”。正義曰:“焚炙”俱燒也,“刳剔”謂割剝也。\CJKunderwave{說文}云:“刳,刲也。”今人去肉至骨謂之“剔去”,是則亦\CJKunderwave{剔}之義也。\CJKunderline{武王}以此數紂之惡,必有忠良被炙,孕婦被刳,不知其姓名為誰也。\CJKunderwave{殷本紀}云,紂為長夜之飲。時諸侯或叛,妲己以為罰輕,紂欲重刑,乃為熨斗,以火燒之然,使人舉輒爛其手,不能勝。紂怒,乃更為銅柱,以膏塗之,亦加于炭火之上,使有罪者緣之,足滑跌墜入中。紂與妲己以為大樂,名曰炮烙之刑。是紂焚炙之事也。後\CJKunderline{文王}獻洛西之地,赤壤之田方千里,請紂除炮烙之刑,紂許之。皇甫謐作\CJKunderwave{帝王世紀}亦云然。謐又云:“紂剖\CJKunderline{比干}妻,以視其胎。”即引此為“刳剔孕婦”也。 \par}

皇天震怒,命我文考,肅將天威,大勳未集。\footnote{言天怒紂之惡,命\CJKunderline{文王}敬行天罰,功業未成而崩。}肆予小子\CJKunderline{發},以爾友邦冢君,觀政于商。\footnote{父業未就之故,故我與諸侯觀紂政之善惡。謂十一年自孟津還時。}惟\CJKunderline{受}罔有悛心,乃夷居弗事上帝神祗,遺厥先宗廟弗祀。\footnote{悛,改也,言紂縱惡無改心,平居無故廢天地百神宗廟之祀。慢之甚。悛,七全反。}

{\noindent\zhuan\zihao{6}\fzbyks 傳“悛改”至“之甚”。正義曰:\CJKunderwave{左傳}稱“長惡不悛”,“悛”是退前創改之義,故為改也。觀政于商,計當恐怖,言紂縱惡無改悔之心,平居無故不事神祗,是紂之大惡。“上帝”,舉其尊者,謂諸神悉皆不事,故傳言“百神”以該之。“不事”亦是“不祀”,別言“遺厥先宗廟弗祀”,遺棄祖父,言其慢之甚也。 \par}

犧牲粢盛,既于兇盜。\footnote{兇人盡盜食之,而紂不罪。粢音諮,黍稷曰粢。盛音成,在器曰盛。}乃曰:‘吾有民有命。’罔懲其侮。\footnote{紂言:“吾所以有兆民,有天命。”故群臣畏罪不爭,無能止其慢心。懲,直承反。爭,爭鬥之爭。}天佑下民,作之君,作之師。\footnote{言天佑助下民,為立君以政之,為立師以教之。為,于偽反。}惟其克相上帝,寵綏四方。\footnote{當能助天寵安天下。相,息亮反。}有罪無罪,予曷敢有越厥志?\footnote{越,遠也。言己志欲為民除惡,是與否,不敢遠其志。否,方有反。}

{\noindent\zhuan\zihao{6}\fzbyks 傳“言天”至“教之”。正義曰:眾民不能自治,立君以治之。立君治民,乃是天意,言天佑助下民為立君也。治民之謂“君”,教民之謂“師”,君既治之,師又教之,故言“作之君,作之師”,“師”謂君與民為師,非謂別置師也。 \par}

{\noindent\zhuan\zihao{6}\fzbyks 傳“當能”至“天下”。正義曰:天愛下民,為立君立師者,當能佑助天意,寵安天下,不奪民之財力,不妄非理刑殺,是助天寵愛民也。 \par}

{\noindent\zhuan\zihao{6}\fzbyks 傳“越遠”至“其志”。正義曰:“越”者,逾越超遠之義,故為遠也。\CJKunderline{武王}伐紂,內實為民除害,外則以臣伐君,故疑其有罪與無罪。“言己志欲為民除害,無問是之與否,不敢遠其志”,言己本志欲伐,何敢遠本志,舍而不伐也? \par}

{\noindent\shu\zihao{5}\fzkt “天佑”至“厥志”。正義曰:已上數紂之罪,此言伐紂之意。上天佑助下民,不欲使之遭害,故命我為之君上,使臨政之;為之師保,使教誨之。為人君為人師者,天意如此,不可違天。我今惟其當能佑助上,天寵安四方之民,使民免于患難。今紂暴虐,無君師之道,故今我往伐之。不知伐罪之事,為有罪也?為無罪也?不問有罪無罪,志在必伐,我何敢有遠其本志而不伐之? \par}

“同力度德,同德度義。\footnote{力鈞則有德者勝,德鈞則秉義者強。揆度優劣,勝負可見。度,徒洛反,下注同。}

{\noindent\zhuan\zihao{6}\fzbyks 傳“力鈞”至“可見”。正義曰:“德”者得也,自得于心。“義”者宜也,動合自宜。但德在于身,故言“有德”;義施于行,故言秉執。\CJKunderline{武王}志在養民,動為除害,有君人之明德,執利民之大義,與紂無者為敵,雖未交兵,揆度優劣,勝負可見。示以必勝之道,令士眾勉力而戰也。 \par}

受有臣億萬,惟億萬心。\footnote{人執異心,不和諧。億,十萬曰億。}予有臣三千,惟一心。\footnote{三千一心,言欲同。}商罪貫盈,天命誅之。予弗順天,厥罪惟鈞。\footnote{紂之為惡,一以貫之,惡貫已滿,天畢其命。今不誅紂,則為逆天,與紂同罪。貫,古亂反。}

{\noindent\zhuan\zihao{6}\fzbyks 傳“紂之”至“同罪”。正義曰:紂之為惡,如物在繩索之貫,一以貫之,其惡貫已滿矣。物極則反,天下欲畢其命,故上天命我誅之。今我不誅紂,則是逆天之命,無恤民之心,是我與紂同罪矣。猶如\CJKunderwave{律}“故縱者與同罪”也。 \par}

予小子夙夜祗懼,受命文考,類于上帝,宜于冢土,以爾有眾,厎天之罰。\footnote{祭社曰宜。冢土,社也。言我畏天之威,告\CJKunderline{文王}廟,以事類告天祭社,用汝眾致天罰于紂。類,師祭名。冢,中勇反。厎,之履反。}

{\noindent\zhuan\zihao{6}\fzbyks 傳“祭社”至“于紂”。正義曰:\CJKunderwave{釋天}引\CJKunderwave{詩}云:“乃立冢土,戎醜攸行。”即云:“起大事,動大眾,必先有事乎社而後出,謂之宜。”孫炎曰:“宜,求見福祐也。”是“祭社曰宜”。“冢”訓大也,社是土神,故“冢土,社也”。\CJKunderwave{毛詩傳}云:“冢土,大社也。”“受命文考”是告廟以行,故為“告\CJKunderline{文王}廟”也。\CJKunderwave{毛詩}云:“天子將出,類乎上帝,宜乎社,造乎禰。”此“受命文考”即是“造乎禰”也。\CJKunderwave{王制}以神尊卑為次,故先言“帝”、“社”,後言“禰”,此以廟是己親,若言家內私義,然後告天,故先言“受命文考”,而後言“類于上帝”。\CJKunderwave{舜典}“類于上帝”傳云:“告天及五帝。”此“以事類告天”,亦當如彼也。罰紂是天之意,故“用汝眾致天罰于紂”也。 \par}

天矜于民,民之所欲,天必從之。\footnote{矜,憐也。言天除惡樹善與民同。從,才容反。}爾尚弼予一人,永清四海。\footnote{穢惡除,則四海長清。}時哉弗可失!”\footnote{言今我伐紂,正是天人合同之時,不可違失。}

\section{泰誓中第二【偽】}


惟戊午,王次于河朔。\footnote{次,止也。戊午渡河而誓,既誓而止于河之北。}

{\noindent\zhuan\zihao{6}\fzbyks 傳“次止”至“之北”。正義曰:“次”是止舍之名,\CJKunderwave{穀梁傳}亦云:“次,止也。”序云“一月戊午,師渡孟津”,則師以戊午日渡也。此戊午日次于河朔,則是師渡之日次止也。上篇是渡河而誓,未及止舍而先誓之,此“次于河朔”者,是“既誓而止于河之北”也。莊三年\CJKunderwave{左傳}例云:“凡師一宿為舍,再宿為信,過宿為次。”此“次”直取止舍之義,非\CJKunderwave{春秋}三日之例也。何則?商郊去河四百餘里,戊午渡河,甲子殺紂,相去才六日耳。是今日次訖又誓,明日誓訖即行,不容三日止于河旁也。 \par}

群后以師畢會,\footnote{諸侯盡會次也。}王乃徇師而誓。曰:“嗚呼!西土有眾,咸聽朕言。\footnote{徇,循也。\CJKunderline{武王}在西,故稱西土。徇,似俊反,\CJKunderwave{字詁}云:“徇,巡也。”}

{\noindent\zhuan\zihao{6}\fzbyks 傳“徇循”至“西土”。正義曰:\CJKunderwave{說文}云:“徇,疾也。循,行也。”“徇”是疾行之意,故以“徇”為循也。下篇“大巡六師”,義亦然也。此誓總戒眾軍,\CJKunderline{武王}國在西偏,此師皆從西而來,故稱“西土”。 \par}

我聞吉人為善,惟日不足。兇人為不善,亦惟日不足。\footnote{言吉人竭日以為善,兇人亦竭日以行惡。竭,苦曷反,又苦蓋反。}今商王\CJKunderline{受},力行無度,\footnote{行無法度,竭日不足,故曰力行。}播棄黎老,暱比罪人。\footnote{鮐背之耇稱黎老,布棄不禮敬。暱近罪人,謂天下逋逃之小人。黎,力私反,又力兮反。暱,女乙反。比,毗志反。鮐,他來反,又音怡,魚名。逋,布吳反。}

{\noindent\zhuan\zihao{6}\fzbyks 傳“鮐背”至“小人”。正義曰:\CJKunderwave{釋詁}云:“鮐背、耇、老,壽也。”舍人曰:“鮐背,老人氣衰,皮膚消瘠,背若鮐魚也。”孫炎曰:“耇,面凍梨色似浮垢也。”然則老人背皮似鮐,面色似梨,故“鮐背之耇”稱“梨老”。傳以“播”為布。布者,遍也,言遍棄之,不禮敬也。“暱,近”,\CJKunderwave{釋詁}文。孫炎曰:“暱,親近也。”\CJKunderwave{牧誓}數紂之罪云:“四方之多罪逋逃,是崇是長,是信是使。”知紂所親近罪人,“謂天下逋逃之小人”也。 \par}

淫酗肆虐,臣下化之,\footnote{過酗縱虐,以酒成惡,臣下化之。言罪同。酗,況付反。}

{\noindent\zhuan\zihao{6}\fzbyks 傳“過酗”至“罪同”。正義曰:“酗”是酒怒,“淫酗”共文,則“淫”非女色,故以“淫”為過,言飲酒過多也。“肆”是放縱之意,酒過則酗,縱情為虐。以酒成此暴虐之惡,臣下化而為之,由紂惡而臣亦惡,言君臣之罪同也。 \par}

朋家作仇,脅權相滅。無辜籲天,穢德彰聞。\footnote{臣下朋黨,自為仇怨,脅上權命,以相誅滅。籲,呼也。民皆呼天告冤無辜,紂之穢德彰聞天地。言罪惡深。脅,虛業反。籲音喻。穢,于廢反。}

{\noindent\zhuan\zihao{6}\fzbyks 傳“臣下”至“惡深”。正義曰:“脅上”謂紂既昏迷,朝無綱紀,姦宄之臣,脅于在下,假用在上之權命,脅之更相誅滅也。 \par}

{\noindent\shu\zihao{5}\fzkt “朋家”至“彰聞”。正義曰:小人好忿,天性之常,化紂淫酗,怨怒無已。臣下朋黨,共為一家,與前人並作仇敵,脅上權命,以相滅亡。無罪之人,怨嗟呼天,紂之穢惡之德,彰聞天地。言其罪惡深也。 \par}

“惟天惠民,惟辟奉天。\footnote{言君天下者當奉天以愛民。闢,必亦反。}有\CJKunderline{夏桀},弗克若天,流毒下國。\footnote{桀不能順天,流毒虐于下國萬民。言兇害。}天乃佑命\CJKunderline{成湯},降黜夏命。\footnote{言天助湯命,使下退桀命。}惟受罪浮于\CJKunderline{桀}。\footnote{浮,過。}

{\noindent\zhuan\zihao{6}\fzbyks 傳“浮過”。正義曰:物在水上謂之浮,“浮”者高之意,故為過也。桀罪已大,紂又過之,言紂惡之甚,故下句說其過桀之狀。案\CJKunderwave{夏本紀}及\CJKunderwave{帝王世紀}云:“諸侯叛桀,關龍逢引皇圖而諫,桀殺之。\CJKunderline{伊尹}諫桀,桀曰:‘天之有日,如吾之有民,日亡吾乃亡矣。’”是桀亦“賊虐諫輔,謂己有天命”。而云過于桀者,\CJKunderwave{殷本紀}云:“紂剖\CJKunderline{比干}觀其心”,桀殺龍逢,無剖心之事;又桀惟比之于日,紂乃詐命于天;又紂有炮烙之刑,又有刳胎斮脛之事,而桀皆無之,是紂罪過于桀也。 \par}

剝喪元良,賊虐諫輔。\footnote{剝,傷害也。賊,殺也。元,善之長。良,善。以諫輔紂,紂反殺之。喪,息浪反。長,丁丈反。}

{\noindent\zhuan\zihao{6}\fzbyks 傳“剝傷”至“殺之”。正義曰:\CJKunderwave{說文}云:“剝,裂也,一曰剝,割也。”裂與割俱是傷害之義也。殺人謂之“賊”,故“賊”為殺也。“元者,善之長”,\CJKunderwave{易}文言文。“良”之為善,書傳通訓也。“元良”俱善而雙舉之者,言其剝喪善中之善,為害大也。“以諫輔紂,紂反殺之”,即\CJKunderline{比干}是也。上篇言“焚炙忠良”,與此經相類而復言此者,以殺害人為惡之大,故重陳之也。 \par}

謂己有天命,謂敬不足行,謂祭無益,謂暴無傷。\footnote{言紂所以罪過于桀。己音紀。}厥監惟不遠,在彼夏王。\footnote{其視紂罪,與桀同辜。言必誅之。}

{\noindent\zhuan\zihao{6}\fzbyks 傳“其視”至“誅之”。正義曰:紂罪過于桀,而言“與桀同辜”者,罪不過死,合死之罪同,言必誅也。 \par}

天其以予乂民,\footnote{用我治民。當除惡。}朕夢協朕卜,襲于休祥,戎商必克。\footnote{言我夢與卜俱合于美善,以兵誅紂必克之佔。}

{\noindent\zhuan\zihao{6}\fzbyks 傳“言我”至“之佔”。正義曰:夢者事之祥,人之精爽先見者也。吉凶或有其驗,聖王採而用之。我卜伐紂得吉,夢又戰勝。\CJKunderwave{禮記}稱“卜筮不相襲”,“襲”者,重合之義。訓“戎”為兵。夢卜俱合于美,是“以兵誅紂必克之佔”也。聖人逆知來物,不假夢卜,言此以強軍人之意耳。\CJKunderwave{史記·周本紀}云:“\CJKunderline{武王}伐紂,卜,龜兆不吉,群公皆懼,惟太公強之。”太公\CJKunderwave{六韜}云:“卜戰,龜兆焦,筮又不吉,太公曰:‘枯骨朽蓍,不逾人矣。’”彼言“不吉”者,\CJKunderwave{六韜}之書,後人所作,\CJKunderwave{史記}又採用\CJKunderwave{六韜},好事者妄矜太公,非實事也。 \par}

受有億兆夷人,離心離德。\footnote{平人,凡人也。雖多而執心用德不同。}

{\noindent\zhuan\zihao{6}\fzbyks 傳“平人”至“不同”。正義曰:昭二十四年\CJKunderwave{左傳}此文,服虔、杜預以“夷人”為夷狄之人。即如彼言,惟云“億兆夷人”,則受率其旅若林,即曾無華夏人矣?故傳訓“夷”為平,平人為凡人,言其智慮齊,識見同。人數雖多,執心用德不同。“心”謂謀慮,“德”謂用行,智識既齊,各欲申意,故“心德不同”也。 \par}

予有亂臣十人,同心同德。\footnote{我治理之臣雖少而心德同。十人,\CJKunderline{周公}旦、\CJKunderline{召公奭}、太公望、\CJKunderline{畢公}、榮公、太顛、\CJKunderline{閎夭}、\CJKunderline{散宜生}生、南宮适及文母。治,直吏反。}

{\noindent\zhuan\zihao{6}\fzbyks 傳“我治”至“德同”。正義曰:\CJKunderwave{釋詁}云:“亂,治也。”故謂我治理之臣有十人也。十人皆是上智,咸識周是殷非,故人數雖少而心能同。同佐\CJKunderline{武王},欲共滅紂也。\CJKunderwave{論語}引此云:“予有亂臣十人。”而\CJKunderline{孔子}論之有一婦人焉,則十人之內其一是婦人,故先儒\CJKunderline{鄭玄}等皆以十人為文母、\CJKunderline{周公}、太公、\CJKunderline{召公}、\CJKunderline{畢公}、榮公、太顛、宏夭、\CJKunderline{散宜生}、\CJKunderline{南宮括}也。 \par}

雖有周親,不如仁人。\footnote{周,至也。言紂至親雖多,不如周家之少仁人。}

{\noindent\zhuan\zihao{6}\fzbyks 傳“周至”至“仁人”。正義曰:\CJKunderwave{詩毛傳}亦以“周”為至,相傳為此訓也。\CJKunderline{武王}三分天下有其二,則紂黨不多于周。但辭有激發,旨有抑揚,欲明多惡不如少善,故言“紂至親雖多,不如周家之少仁人”也。 \par}

“天視自我民視,天聽自我民聽。\footnote{言天因民以視聽,民所惡者天誅之。惡,烏路反,一音如字。}百姓有過,在予一人。\footnote{己能無惡于民,民之有過,在我教不至。}

{\noindent\shu\zihao{5}\fzkt “百姓有過,在予一人”。正義曰:言此者,以上雲民之所惡,天必誅之,己今有善,不為民之所惡,天必佑我。令教化百姓,若不教百姓,使有罪過,實在我一人之身。此“百姓”與下“百姓懍懍”,皆謂天下眾民也。 \par}

今朕必往,我武惟揚,侵于之疆,\footnote{揚,舉也。言我舉武事,侵入紂郊疆伐之。疆,居良反。}取彼兇殘,我伐用張,于湯有光。\footnote{桀流毒天下,湯黜其命。紂行兇殘之德,我以兵取之。伐惡之道張設,比于湯又有光明。}

{\noindent\zhuan\zihao{6}\fzbyks 傳“揚舉”至“伐之”。正義曰:\CJKunderwave{文王世子}論舉賢之法云:“或以事舉,或以言揚。”是“揚”、“舉”義同,故“揚”為舉也。于時猶在河朔,將欲行適商都,言我舉武事,侵入紂之郊疆,往伐之也。\CJKunderwave{春秋}之例有:“鐘鼓曰伐,無曰侵。”此實伐也,言“往侵”者,“侵”是入之意,非如\CJKunderwave{春秋}之例無鐘鼓也。 \par}

{\noindent\shu\zihao{5}\fzkt “今朕”至“有光”。正義曰:既與天下為任,則當為之除害,今我必往伐紂。我之武事惟于此舉之,侵紂之疆境,取彼為兇殘之惡者。若得取而殺之,是我伐兇惡之事用張設矣。湯惟放逐,我能擒取,是比于湯又益有光明。 \par}

勖哉,夫子!罔或無畏,寧執非敵。\footnote{勖,勉也。夫子謂將士。無敢有無畏之心,寧執非敵之志,伐之則克矣。將,子匠反,下篇注同。}

{\noindent\zhuan\zihao{6}\fzbyks 傳“勖勉”至“克矣”。正義曰:“勖,勉”,\CJKunderwave{釋詁}文。呼將士而誓之,知“夫子”是將士也。\CJKunderwave{老子}云:“禍莫大于輕敵。”故今將士“無敢有無畏之心”,令其必以前敵為可畏也。\CJKunderwave{論語}稱:“子路曰:‘子行三軍則誰與?’\CJKunderline{孔子}曰:‘必也臨事而懼。’”令軍士等不欲發意輕前人,寧執非敵之志,恐彼強多,非我能敵,執此志以伐之,則當克矣。 \par}

{\noindent\shu\zihao{5}\fzkt “勖哉”至“非敵”。正義曰:取得紂則功多于湯,宜勉力哉!“夫子”,將士等。呼將士令勉力也。以兵伐人,當臨事而懼,汝將土等無敢有無畏輕敵之心,寧執守似前人之強,非己能敵之志以伐之,如是乃可克矣。 \par}

百姓懍懍,若崩厥角。\footnote{言民畏紂之虐,危懼不安,若崩摧其角,無所容頭。○懍,力甚反。}

{\noindent\zhuan\zihao{6}\fzbyks 傳“言民”至“容頭”。正義曰:“懍懍”是怖懼之意,言民畏紂之虐,危懼不安,其志懍懍然。以畜獸為喻,民之怖懼若似畜獸崩摧其頭角然,無所容頭。顧氏云:“常如人之慾崩其角也,言容頭無地。”隱三年\CJKunderwave{穀梁傳}曰:“高曰崩,頭角之稱崩,體之高也。” \par}

嗚呼!乃一德一心,立定厥功,惟克永世。”\footnote{汝同心立功,則能長世以安民。}

\section{泰誓下第三【偽】}


時厥明,王乃大巡六師,明誓眾士。\footnote{是其戊午明日,師出以律,三申令之,重難之義。眾士,百夫長已上。○令,力政反。重,直用反。長,丁丈反。已音以。上,時掌反。}

{\noindent\zhuan\zihao{6}\fzbyks 傳“是其”至“已上”。正義曰:上篇未次而誓,故略言“大會”。中篇既次乃誓,為文稍詳,故言“以師畢會”。此篇最在其後,為文亦詳,故言“大巡六師”。巡繞周遍大其事,故稱“大”也。“師”者,眾也。天子之行,通以六師為言。于時諸侯盡會,其師不啻六也。“師出以律”,\CJKunderwave{易·師卦}初六爻辭也。“律”,法也。行師以法,即誓敕賞勸是也。禮成于三,故為三篇之誓。三度申重號令,為重慎艱難之義也。\CJKunderwave{孫子兵法}“三令五申之”,此誓三篇,亦為三令之事也。\CJKunderwave{牧誓}王所呼者,從上而下,至“百夫長”而止,知此“眾士”是“百夫長已上”也。 \par}

王曰:“嗚呼!我西土君子,天有顯道,厥類惟彰。\footnote{言天有明道,其義類惟明,言王所宜法則。}

{\noindent\zhuan\zihao{6}\fzbyks 傳“言天”至“法則”。正義曰:\CJKunderwave{孝經}云:“則天之明。”昭二十五年\CJKunderwave{左傳}云:“以象天明。”是治民之事,皆法天之道。天有尊卑之序,人有上下之節,三正五常,皆在于天,有其明道,此天之明道。“其義類惟明”,言明白可效,王者所宜法則之。將言商王不法天道,故先標二句于前。其下乃述商王違天之事,言其罪宜誅也。 \par}

今商王\CJKunderline{受},狎侮五常,荒怠弗敬。\footnote{輕狎五常之教,侮慢不行,大為怠惰,不敬天地神明。○惰,徒臥反。}

{\noindent\zhuan\zihao{6}\fzbyks 傳“輕狎”至“神明”。正義曰:\CJKunderline{鄭玄}\CJKunderwave{論語注}云:“狎,慣忽之。”言慣見而忽也,意與“侮”同,傳因文重而分之。“五常”即五典,謂父義、母慈、兄友、弟恭、子孝,五者人之常行,法天明道為之。輕狎五常之教,侮慢而不遵行之,是違天顯也。訓“荒”為大,大為怠惰。“不敬”謂“不敬天地神明”也。上篇云“不事上帝神祇”,知此“不敬天地神明”也。\CJKunderwave{禮}云:“毋不敬。”傳舉“天地”以言,明每事皆不敬也。 \par}

自絕于天,結怨于民。\footnote{不敬天,自絕之。酷虐民,結怨之。}\xpinyin*{斮}朝涉之脛,剖賢人之心,\footnote{冬月見朝涉水者,謂其脛耐寒,斬而視之。\CJKunderline{比干}忠諫,謂其心異于人,剖而觀之。酷虐之甚。○斮,側略反,又士略反。朝,陟遙反。脛,戶定反。剖,普口反。耐,乃代反。}

{\noindent\zhuan\zihao{6}\fzbyks 傳“冬月”至“之甚”。正義曰:\CJKunderwave{釋器}云:“魚曰斮之。”樊光云:“斮,斫也。”\CJKunderwave{說文}云:“斮,斬也。”斬朝涉水之脛,必有所由,知冬月見朝涉水者,謂其脛耐寒,疑其骨髓有異,斬而視之。其事或當有所出也。\CJKunderwave{殷本紀}云:“\CJKunderline{微子}既去,\CJKunderline{比干}曰:‘為人臣者,不得不以死爭。’乃強諫。紂怒曰:‘吾聞聖人心有七竅。’遂剖\CJKunderline{比干},觀其心。”是紂謂\CJKunderline{比干}心異于人,剖而觀之。言酷虐之甚。 \par}

作威殺戮,毒痡四海。\footnote{痡,病也。言害所及遠。○痡,徐音敷,又普吳反。}

{\noindent\zhuan\zihao{6}\fzbyks 傳“痡病”至“及遠”。正義曰:“痡,病”,\CJKunderwave{釋詁}文。紂之毒害,未必遍及夷狄,而云病四海者,言害所及者遠也。 \par}

崇信奸回,放黜師保,\footnote{回,邪也。奸邪之人,反尊信之。可法以安者,反放退之。○邪,似嗟反。}屏棄典刑,囚奴正士,\footnote{屏棄常法而不顧,\CJKunderline{箕子}正諫而以為囚奴。}郊社不修,宗廟不享,作奇技淫巧以悅婦人。\footnote{言紂廢至尊之敬,營卑褻惡事,作過制技巧,以恣耳目之欲。○技,其綺反。褻,息列反。}

{\noindent\shu\zihao{5}\fzkt “郊社”至“婦人”。正義曰:“不修”謂不掃治也。“不享”謂不祭祀也。與上篇“不祀上帝神祇,遺厥先宗廟不祀”,其事一也,重言之耳。“奇技”謂奇異技能,“淫巧”謂過度工巧,二者本同,但“技”據人身,“巧”指器物為異耳。 \par}

上帝弗順,祝降時喪。\footnote{祝,斷也。天惡紂逆道,斷絕其命,故下是喪亡之誅。○喪,蘇浪反。斷,丁管反。惡,烏路反。}

{\noindent\zhuan\zihao{6}\fzbyks 傳“祝斷”。正義曰:哀十四年\CJKunderwave{公羊傳}云:“子路死,子曰:‘天祝予!’”何休云:“祝,斷也。”是相傳訓也。 \par}

爾其孜孜,奉予一人,恭行天罰。\footnote{孜孜,勸勉不怠。○孜音茲。}古人有言曰:‘撫我則後,虐我則讎。’\footnote{\CJKunderline{武王}述古言以明義,言非惟今紂惡。}獨夫受,洪惟作威,乃汝世讎。\footnote{言獨夫,失君道也。大作威殺無辜,乃是汝累世之讎。明不可不誅。}樹德務滋,除惡務本,\footnote{立德務滋長,去惡務除本。言紂為天下惡本。}肆予小子,誕以爾眾士殄殲乃讎。\footnote{言欲行除惡之義,絕盡紂。○殄,徒典反。纖,子廉反。}爾眾士其尚迪果毅,以登乃闢。\footnote{迪,進也。殺敵為果,致果為毅。登,成也,成汝君之功。○毅,牛既反。}

{\noindent\zhuan\zihao{6}\fzbyks 傳“迪進”至“之功”。正義曰:“迪,進”、“登,成”皆\CJKunderwave{釋詁}文。“殺敵為果,致果為毅”,宣二年\CJKunderwave{左傳}文。“果”謂果敢,“毅”謂強決。能殺敵人謂之為“果”,言能果敢以除賊。致此果敢是各為“毅”,言能強決以立功。皆言其心不猶豫也。軍法以殺敵為上,故勸令果毅成功也。 \par}

功多有厚賞,不迪有顯戮。\footnote{賞以勸之,戮以威之。}嗚呼!惟我文考,若日月之照臨,光于四方,顯于西土。\footnote{稱父以感眾也。言其明德充塞四方,明著岐周。}惟我有周,誕受多方。\footnote{言\CJKunderline{文王}德大,故受眾方之國,三分天下而有其二。}予克受,非予武,惟朕文考無罪。\footnote{推功于父,言\CJKunderline{文王}無罪于天下,故天佑之,人盡其用。}受克予,非朕文考有罪,惟予小子無良。”\footnote{若紂克我,非我父罪,我之無善之致。}

{\noindent\zhuan\zihao{6}\fzbyks 傳“若紂”至“之致”。正義曰:言克受乃是\CJKunderline{文王}之功,若受克予非是\CJKunderline{文王}之罪。而言“非我父罪,我之無善之致”者,其意言勝非我功,敗非父咎,崇孝罪己,以求眾心耳。 \par}

\section{牧誓第四}


\CJKunderline{武王}戎車三百兩,\footnote{兵車,百夫長所載。車稱兩。一車步卒七十二人,凡二萬一千人,舉全數。○車音居。\CJKunderwave{釋名}云:“古者聲如居,所以居人也。今曰車,聲近舍,車舍也。”韋昭\CJKunderwave{辯釋名}云:“古皆尺遮反,從漢始有音居。”長,丁丈反。卒,子忽反。}虎賁三百人,\footnote{勇士稱也,若虎賁獸,言其猛也。皆百夫長。○賁音奔。稱,尺證反。}與\CJKunderline{受}戰于牧野,作\CJKunderwave{牧誓}。

{\noindent\zhuan\zihao{6}\fzbyks 傳“兵車”至“全數”。正義曰:孔以“虎賁三百人”與戎車數同,王于誓時所呼有“百夫長”,因謂“虎賁”即是百夫之長。一人而乘一車,故云“兵車,百夫長所載”也。數車之法,一車謂之一兩。\CJKunderwave{詩}云“百兩迓之”,是車稱兩也。\CJKunderwave{風俗通}說車有兩輪,故稱為兩。猶屨有兩隻,亦稱為兩。\CJKunderwave{詩}云“葛屨五兩”即其類也。“一車步卒七十二人”,\CJKunderwave{司馬法}文也。車有七十二人,三百乘凡二萬一千人。計車有七十二人,三百乘當有二萬一千六百人,孔略六百而不言,故云“舉全數”。顧氏亦同此解。孔既用\CJKunderwave{司馬法}一車七十二人,又云“兵車,百夫長所載”,又下傳以百夫長為“卒帥”,是實領百人,非惟七十二人。依\CJKunderwave{周禮·大司馬法},天子六軍,出自六鄉,凡起徒役,無過家一人,故一鄉出一軍,鄉為正,遂為副。若鄉遂不足,則徵兵于邦國。則司馬法六十四井為甸,計有五百七十六夫,共出長轂一乘,甲士三人,步卒七十二人。至于臨敵對戰布陳之時,則依六鄉軍法,五人為伍,五伍為兩,四兩為卒,五卒為旅,五旅為師,五師為軍。故\CJKunderwave{左傳}云:“先偏後伍。”又云:“廣有一卒,卒偏之兩。”非直人數如此,車數亦然。故\CJKunderwave{周禮}云:“乃會車之卒伍。”鄭云:“車亦有卒伍。”\CJKunderwave{左傳}“戰于繻葛”,杜注云:“車二十五乘為偏。”是車亦為卒伍之數也。則一車七十二人者,自計元科兵之數。科兵既至,臨時配割,其車雖在,其人分散,前配車之人,臨戰不得還屬本車,當更以虎賁甲士配車而戰。孔舉七十二人元科兵數者,欲總明三百兩人之大數。云“兵車,百夫長所載者”,欲見臨敵實一車有百人,既“虎賁”與車數相當,又經稱“百夫長”,故孔為此說。 \par}

{\noindent\zhuan\zihao{6}\fzbyks 傳“勇士”至“夫長”。正義曰:\CJKunderwave{周禮}虎賁氏之官,其屬有虎士八百人,是“虎賁”為“勇士稱”也。若虎之賁走逐獸,言其猛也。此“虎賁”必是軍內驍勇選而為之,當時謂之“虎賁”。\CJKunderwave{樂記}云“虎賁之士說劍”,謂此也。孔意“虎賁”即是經之“百夫長”,故云:“皆百夫長”也。 \par}

{\noindent\shu\zihao{5}\fzkt “\CJKunderline{武王}”至“牧誓”。正義曰:\CJKunderline{武王}以兵戎之車三百兩、虎賁之士三百人與受戰于商郊牧地之野,將戰之時,王設言以誓眾。史敘其事,作\CJKunderwave{牧誓}。 \par}

牧誓\footnote{至牧地而誓眾。○牧如字,徐一音茂,\CJKunderwave{說文}作坶,云:“地名,在朝歌南七十里。”\CJKunderwave{字林}音母。}

時甲子昧爽,\footnote{是克紂之月甲子之日,二月四日。昧,冥;爽,明;早旦。○昧音妹。爽,明也。“昧爽”謂早旦也。馬云:“昧,未旦也。”}

{\noindent\zhuan\zihao{6}\fzbyks 傳“是克”至“早旦”。正義曰:\CJKunderwave{春秋}主書動事,編次為文,于法日月時年皆具,其有不具,史闕耳。\CJKunderwave{尚書}惟記言語,直指設言之日。上篇“戊午,次于河朔”,\CJKunderwave{洛誥}“戊辰,王在新邑”,與此“甲子”,皆言有日無月,史意不為編次,故不具也。是“克紂之月甲子之日,是周之二月四日”,以歷推而知之也。\CJKunderwave{釋言}云:“晦,冥也。”“昧”亦晦義,故為冥也。“冥”是夜,“爽”是明,夜而未明謂早旦之時,蓋雞鳴後也。為下“朝至”發端,“朝”即“昧爽”時也。 \par}

王朝至于商郊牧野,乃誓。\footnote{紂近郊三十里地名牧。癸亥夜陳,甲子朝誓,將與紂戰。○陳,直刃反。}

{\noindent\zhuan\zihao{6}\fzbyks 傳“紂近”至“紂戰”。正義曰:傳言在“紂近郊三十里”,或當有所據也。皇甫謐云:“在朝歌南七十里。”不知出何書也。言“至于商郊牧野”,知“牧”是郊上之地。戰在平野,故言“野”耳。\CJKunderwave{詩}云:“于牧之野。”\CJKunderwave{禮記大傳}云:“牧之野,\CJKunderline{武王}之大事,繼牧言野,明是牧地。”而\CJKunderline{鄭玄}云:“郊外曰野,將戰于郊,故至牧野而誓。”案經“至于商郊牧野乃誓”,豈王行已至于郊,乃後到退適野,誓訖而更進兵乎?何不然之甚也!\CJKunderwave{武成}云:“癸亥夜陳,未畢而雨。”是癸亥夜已布陳,故甲子朝而誓眾,將與紂戰,故戒敕之。 \par}

王左杖黃鉞,右秉白旄以麾,曰:“逖矣,西土之人!”\footnote{鉞,以黃金飾斧。左手杖鉞,示無事于誅。右手把旄,示有事于教。逖,遠也。遠矣,西土之人。勞苦之。○杖,徐直亮反。鉞音越,本又作戌。旄音毛,馬云:“白旄,旄牛尾。”麾,許危反。逖,他歷反。}

{\noindent\zhuan\zihao{6}\fzbyks 傳“鉞以”至“苦之”。正義曰:太公\CJKunderwave{六韜}云:“大柯斧重八斤,一名天鉞。”\CJKunderwave{廣雅}云:“鉞,斧也。”斧稱“黃鉞”,故知“以黃金飾斧”也。鉞以殺戮,殺戮用右手,用左手杖鉞,示無事于誅。右手把旄,示有事于教。其意言惟教軍人,不誅殺也。把旄何以白?旄用白者,取其易見也。“逖,遠”,\CJKunderwave{釋詁}文。 \par}

王曰:“嗟!我友邦冢君,\footnote{同志為友,言志同滅紂。}御事司徒、司馬、司空,\footnote{治事三卿,司徒主民,司馬主兵,司空主土,指誓戰者。}

{\noindent\zhuan\zihao{6}\fzbyks 傳“治事”至“戰者”。正義曰:孔以于時已稱王而有六師,亦應已置六卿。今呼治事惟三卿者,司徒主民,治徒庶之政令;司馬主兵,治軍旅之誓戒;司空主土,治壘壁以營軍;是“指誓戰者”,故不及太宰、大宗、司寇也。其時六卿具否,不可得知,但據此三卿為說耳。此“御事”之文,指三卿而說,是不通于“亞旅”已下。 \par}

亞旅、師氏,\footnote{亞,次。旅,眾也。眾大夫,其位次卿。師氏,大夫,官以兵守門者。}

{\noindent\zhuan\zihao{6}\fzbyks 傳“亞次”至“門者”。正義曰:“亞,次”,\CJKunderwave{釋言}文。“旅,眾”,\CJKunderwave{釋詁}文。此及\CJKunderwave{左傳}皆卿下言“亞旅”,知是“大夫,其位次卿”,而數眾,故以亞次名之,謂諸是四命之大夫,在軍有職事者也。“師氏”亦大夫,其官掌以兵守門,所掌尤重,故別言之。\CJKunderwave{周禮}師氏中大夫,“使其屬帥四夷之隸,各以其兵服守王之門外。朝在野外,則守內列”。\CJKunderline{鄭玄}云:“內列,蕃營之在內者也,守之如守王宮。” \par}

千夫長、百夫長,\footnote{師帥,卒帥。○帥,色類反,下同。}

{\noindent\zhuan\zihao{6}\fzbyks 傳“師帥,卒帥”。正義曰:\CJKunderwave{周禮}二千五百人為師,師帥皆中大夫。百人為卒,卒長皆上士。孔以師雖二千五百人,舉全數亦得為幹夫長,“長”與“帥”其義同,是千夫長亦可以稱“帥”,故以“千夫長”為師帥,“百夫長”為卒帥。\CJKunderline{王肅}云“師長、卒長”,意與孔同,順經文而稱“長”耳。\CJKunderline{鄭玄}以為“師帥,旅帥也”,與孔不同。 \par}

及庸、蜀、羌、髳、微、盧、彭、濮人。\footnote{八國皆蠻夷戎狄屬\CJKunderline{文王}者國名。羌在西蜀叟,髳、微在巴蜀,盧、彭在西北,庸、濮在江漢之南。}

{\noindent\zhuan\zihao{6}\fzbyks 傳“八國”至“之南”。正義曰:九州之外,四夷大名,則東夷、西戎、南蠻、北狄,其在當方,或南有戎而西有夷。此八國並非華夏,故大判言之,“皆蠻夷戎狄屬\CJKunderline{文王}者國名”也。此八國皆西南夷也,\CJKunderline{文王}國在于西,故西南夷先屬焉。大劉以“蜀”是蜀郡,顯然可知,孔不說。又退“庸”就“濮”解之,故以次先解“羌”。云“羌在西蜀叟”者,漢世西南之夷,“蜀”名為大,故傳據“蜀”而說。左思\CJKunderwave{蜀都賦}云:“三蜀之豪,時來時往。”是蜀都分為三,羌在其西,故云“西蜀叟”。“叟”者蜀夷之別名,故\CJKunderwave{後漢書}“興平元年,馬騰、劉範謀誅李𠐶,益州牧劉焉遣叟兵五千人助之”,是蜀夷有名“叟”者也。“髳、微在巴蜀”者,巴在蜀之東偏,漢之巴郡所治江州縣也。“盧、彭在西北”者,在東蜀之西北也。文十八年\CJKunderwave{左傳}稱,庸與百濮伐楚,楚遂滅庸。是“庸、濮在江漢之南”。 \par}

稱爾戈,比爾幹,立爾矛,予其誓。”\footnote{稱,舉也。戈,戟。幹,楯也。○比,徐扶志、毗志二反。楯,食準反,又音允。}

{\noindent\zhuan\zihao{6}\fzbyks 傳“稱舉”至“幹楯”。正義曰:“稱,舉”,\CJKunderwave{釋言}文。方言云:“戟,楚謂之孑,吳揚之間謂之戈。”是“戈”即戟也。\CJKunderwave{考工記}云:“戈柲六尺有六寸,車戟常。”鄭云:“八尺曰尋,倍尋曰常。”然則戈戟長短異名,而云“戈”者即戟,戈戟長短雖異,其形制則同,此雲舉戈,宜舉其長者,故以“戈”為戟也。\CJKunderwave{方言}又云:“楯,自關而東或謂之楯,或謂之幹,關西謂之楯。”是“幹”、“楯”為一也。戈短,人執以舉之,故言“稱”。楯則並以捍敵,故言“比”。矛長立之于地,故言“立”也。 \par}

王曰:“古人有言曰:‘牝雞無晨。\footnote{言無晨鳴之道。○牝,類引反,徐扶忍反。}牝雞之晨,惟家之索。’\footnote{索,盡也。喻婦人知外事,雌代雄鳴則家盡,婦奪夫政則國亡。○索,西各反。}

{\noindent\zhuan\zihao{6}\fzbyks 傳“索盡”至“國亡”。正義曰:\CJKunderwave{禮記·檀弓}曰:“吾離群而索居。”則“索居”為散義。\CJKunderline{鄭玄}云:“索,散也。”物散則盡,故“索”為盡也。“牝雞”,雌也。\CJKunderwave{爾雅}飛曰“雌雄”,走曰“牝牡”,而此言“牝雞”者,\CJKunderwave{毛詩}、\CJKunderwave{左傳}稱“雄狐”,是亦飛、走通也。此以牝雞之鳴喻婦人知外事,故重申喻意云:“雌代雄鳴則家盡,婦奪夫政則國亡。”“家”總貴賤為文,言“家”以對“國”耳。將陳紂用婦言,故舉此古人之語。紂直用婦言耳,非能奪其政,舉此言者,專用其言,賞罰由婦,即是奪其政矣。婦人不當知政,是別外內之分,若使賢如文母,可以興助國家,則非牝雞之喻矣。 \par}

今商王\CJKunderline{受}惟婦言是用,\footnote{妲己惑紂,紂信用之。○妲,丹達反;己音紀;紂妻也。}

{\noindent\zhuan\zihao{6}\fzbyks 傳“妲己”至“用之”。正義曰:\CJKunderwave{晉語}云:“殷辛伐有蘇氏,蘇氏以妲己女焉。妲己有寵而亡殷。”\CJKunderwave{殷本紀}云:“紂嬖于婦人,愛妲己,惟妲己之言是從。”\CJKunderwave{列女傳}云:“紂好酒淫樂,不離妲己,妲己所與言者貴之,妲己所憎者誅之。為長夜飲,妲己好之,百姓怨望,而諸侯有叛者。妲已曰:‘罰輕誅薄,威不立耳。’紂乃重刑辟,為炮烙之法,妲已乃笑。\CJKunderline{武王}伐紂,斬妲已頭懸之于小白旗上,以為亡紂者此女也。” \par}

昏棄厥肆祀弗答,\footnote{昏,亂。肆,陳。答,當也。亂棄其所陳祭祀,不復當享鬼神。○復,扶又反。}

{\noindent\zhuan\zihao{6}\fzbyks 傳“昏亂”至“鬼神”。正義曰:昏暗者于事必亂,故“昏”為亂也。\CJKunderwave{詩}云:“肆筵設席。”“肆”者陳設之意,\CJKunderwave{毛傳}亦以“肆”為陳也。對合,相當之事,故“答”為當也。紂身昏亂,棄其宜所陳設祭祀,不復當享鬼神,與上“郊社不修,宗廟不享”亦一也。不事神祗,惡之大者,故\CJKunderwave{泰誓}及此三言之。 \par}

昏棄厥遺王父母弟不迪,\footnote{王父,祖之昆弟。母弟,同母弟。言棄其骨肉,不接之以道。}

{\noindent\zhuan\zihao{6}\fzbyks 傳“王父”至“以道”。正義曰:\CJKunderwave{釋親}云“父之考為王父”,則“王父”是祖也。紂無親祖可棄,故為“祖之昆弟”。棄其祖之昆弟,則父之昆弟亦棄之矣。\CJKunderwave{春秋}之例,母弟稱“弟”,凡\CJKunderwave{春秋}稱“弟”皆是母弟也。“母弟”謂同母之弟,同母尚棄,別生者必棄矣,舉尊親以見卑疏也。“遺”亦“棄”也,言紂之昏亂,棄其所遺骨肉之親,不接之以道。經先言棄祀、棄親者,\CJKunderline{鄭玄}云:“\CJKunderwave{誓}首言此者,神怒民怨,紂所以亡也。” \par}

乃惟四方之多罪逋逃,是崇是長,\footnote{言紂棄其賢臣,而尊長逃亡罪人,信用之。}是信是使,是以為大夫卿士。\footnote{士,事也。用為卿大夫,典政事。}俾暴虐于百姓,以姦宄于商邑。\footnote{使四方罪人暴虐姦宄于都邑。○俾,必爾反,使也。}

{\noindent\zhuan\zihao{6}\fzbyks 傳“使四”至“都邑”。正義曰:“暴虐”謂殺害,殺害加于人,故言“于百姓”。“姦宄”謂劫奪,劫奪有處,故言“于商邑”。百姓亦是商邑之人,故傳總言“于都邑”也。 \par}

今予\CJKunderline{發}惟恭行天之罰。今日之事,不愆于六步、七步,乃止齊焉。\footnote{今日戰事,就敵不過六步、七步,乃止相齊。言當旅進一心。}

{\noindent\zhuan\zihao{6}\fzbyks 傳“今日”至“一心”。正義曰:戰法布陳然後相向,故設其就敵之限,不過六步、七步,乃止相齊焉。欲其相得力也。\CJKunderwave{樂記}稱“進旅退旅”,是“旅”為眾也,言當眾進一心也。 \par}

夫子勖哉!不愆于四伐、五伐、六伐、七伐,乃止齊焉。\footnote{夫子謂將士,勉勵之。伐謂擊剌,少則四五,多則六七以為例。○勖,許六反。剌,七亦反。}

{\noindent\zhuan\zihao{6}\fzbyks 傳“夫子”至“為例”。正義曰:此及下文三云“夫子”,此“勖哉”在下,下“勖哉”在上。此先呼其人,然後勉之;此既言然,下先令勉勵,乃呼其人,各與下句為目也。上有“戈”、“矛”,戈謂擊兵,矛謂剌兵,故云“伐謂擊剌”,此“伐”猶伐樹然也。 \par}

勖哉夫子!尚桓桓,\footnote{桓桓,武貌。}

{\noindent\zhuan\zihao{6}\fzbyks 傳“桓桓,武貌”。正義曰:\CJKunderwave{釋訓}云:“桓桓,威也。”\CJKunderwave{詩序}云:“桓,武志也。” \par}

如虎如貔,如熊如羆,于商郊。\footnote{貔,執夷,虎屬也。四獸皆猛健,欲使士眾法之,奮擊于牧野。○貔,彼皮反,\CJKunderwave{爾雅}云:“羆如熊,黃白文。”}

{\noindent\zhuan\zihao{6}\fzbyks 傳“貔,執夷”。正義曰:\CJKunderwave{釋獸}云:“貔,白狐,其子豰。”舍人曰:“貔名白狐,其子名豰。”郭璞曰:“一名執夷,虎豹屬。” \par}

弗迓克奔,以役西土,\footnote{商眾能奔來降者,不迎擊之,如此則所以役我西土之義。○迓,五嫁反,馬作御,禁也。役,馬云:“為也。”為,于偽反。}

{\noindent\zhuan\zihao{6}\fzbyks 傳“商眾”至“之義”。正義曰:“迓”訓迎也,不迎擊商眾能奔來降者,兵法不誅降也。“役”謂使用也,如此不殺降人,則所以使用我西土之義。用義于彼,令彼知我有義也。\CJKunderline{王肅}讀“御”為禦,言“不御能奔走者,如殷民欲奔走來降者,無逆之;奔走去者,可不御止。役,為也,盡力以為我西土”。與孔不同。 \par}

勖哉夫子!爾所弗勖,其于爾躬有戮。”\footnote{臨敵所安,汝不勉,則于汝身有戮矣。}

\section{武成第五【偽】}


\CJKunderline{武王}伐殷,往伐歸獸,\footnote{往誅紂克定,偃武修文,歸馬牛于華山桃林之牧地。○獸,徐始售反;本或作獸,許救反。}識其政事,\footnote{記識殷家政教善事以為法。}作\CJKunderwave{武成}。\footnote{武功成,文事修。}


{\noindent\zhuan\zihao{6}\fzbyks 傳“往誅”至“牧地”。正義曰:此序于經“于徵伐商”,是“往伐”也。“歸馬”、“放牛”是“歸獸”也。故傳引經以解之,\CJKunderwave{爾雅}有\CJKunderwave{釋獸}、\CJKunderwave{釋畜},畜、獸形相類也,在野自生為獸,人家養之為畜。歸馬放牛,不復乘用,使之自生自死,若野獸然,故謂之“獸”。獸以野澤為家,故言“歸”也。 \par}

{\noindent\zhuan\zihao{6}\fzbyks 傳“記識”至“為法”。正義曰:紂以昏亂而滅,前世政有善者,故訪問殷家政教,記識善事以為治國之法,經云“列爵惟五,分士惟三”是也。 \par}

{\noindent\shu\zihao{5}\fzkt “\CJKunderline{武王}”至“武成”。正義曰:\CJKunderline{武王}之伐殷也,往則陳兵伐紂,歸放牛馬為獸,記識殷家美政善事而行用之。史敘其事,作\CJKunderwave{武成}。 \par}

武成\footnote{文王受命,有此武功,成于克商。}

{\noindent\zhuan\zihao{6}\fzbyks 傳“\CJKunderline{文王}”至“克商”。正義曰:“\CJKunderline{文王}受命,有此武功”,\CJKunderwave{詩}之文也。彼言“武功”,謂始伐崇耳。殷紂尚在,其功未成,成功在于克商,今武始成矣,故以“武成”名篇,以\CJKunderwave{泰誓}繼\CJKunderline{文王}之年,故本之于\CJKunderline{文王}。鄭云:“著武道至此而成。” \par}

{\noindent\shu\zihao{5}\fzkt “武成”。正義曰:此篇敘事多而王言少,惟辭又首尾不結,體裁異于餘篇。自“惟一月”至“受命于周”,史敘伐殷往反及諸侯大集,為王言發端也。自“王若曰”至“大統未集”,述祖父以來開建王業之事也。自“予小子”至“名山大川”,言己承父祖之意,告神陳紂之罪也。自“曰惟有道”至“無作神羞”,王自陳告神之辭也。“既戊午”已下,又是史敘往伐殺紂,入殷都佈政之事。“無作神羞”以下,惟告神,其辭不結,文義不成,非述作之體。案\CJKunderwave{左傳}荀偃禱河云:“無作神羞,具官臣偃,無敢復濟,惟爾有神裁之。”蒯聵禱祖云:“無作三祖羞,大命不敢請,佩玉不敢愛。”彼二者于“神羞”之下皆更申己意,此經“無作神羞”下更無語,直是與神之言猶尚未訖。且冢君百工初受周命,王當有以戒之,如\CJKunderwave{湯誥}之類。宜應說其除害與民更始,創以為惡之禍,勸以行道之福,不得大聚百官,惟誦禱辭而已。欲徵則殷勤誓眾,既克則空話禱神,聖人有作,理必不爾。竊謂“神羞”之下,更合有言,簡編斷絕,經失其本,所以辭不次耳。或初藏之日,已失其本;或壞壁得之,始有脫漏;故孔稱五十八篇以外,錯亂磨滅,不可復知。明是見在諸篇亦容脫錯,但孔此篇首尾具足,既取其文為之作傳,恥雲有所失落,不復言其事耳。 \par}

惟一月壬辰,旁死魄。\footnote{此本說始伐紂時。一月,周之正月。旁,近也。月二日,近死魄。○旁,步光反。魄,普白反,\CJKunderwave{說文}作霸,匹革反,云:“月始生魄然貌。”近,附近之近。}越翼日癸巳,王朝步自周,于徵伐商。\footnote{翼,明。步,行也。\CJKunderline{武王}以正月三日行自周,往征伐商,二十八日渡孟津。}厥四月,哉生明,王來自商,至于豐。\footnote{其四月。哉,始也。始生明,月三日,與死魄互言。○哉,徐音載。豐,芳弓反,\CJKunderline{文王}所都也。}


{\noindent\zhuan\zihao{6}\fzbyks 傳“此本”至“死魄”。正義曰:將言武成,遠本其始。”此本說始伐紂時。一月,周之正月”,是建子之月,殷十二月也。此月辛卯朔,朔是死魄,故“月二日,近死魄”。“魄”者,形也,謂月之輪郭無光之處名“魄”也。朔後明生而魄死,望後明死而魄生。\CJKunderwave{律歷志}云:“死魄,朔也。生魄,望也。”\CJKunderwave{顧命}云:“惟四月哉生魄。”傳云:“始生魄,月十六日也。”月十六日為始生魄,是一日為始死魄,二日近死魄也。顧氏解“死魄”與小劉同。大劉以三日為始死魄,二日為旁死魄。旁死魄無事而記之者,與下日為發端,猶今之將言日,必先言朔也。 \par}

{\noindent\zhuan\zihao{6}\fzbyks 傳“翼明”至“孟津”。正義曰:“翼,明”,\CJKunderwave{釋言}文。\CJKunderwave{釋宮}云:“堂上謂之行,堂下謂之步。”彼相對為名耳。散則可以通,故“步”為行也。周去孟津千里,以正月三日行自周,二十八日渡孟津,凡二十五日,每日四十許裡,時之宜也。\CJKunderwave{詩}云:“于三十里。”\CJKunderwave{毛傳}云:“師行三十里。”蓋言其大法耳。 \par}

{\noindent\zhuan\zihao{6}\fzbyks 傳“其四”至“互言”。正義曰:“其四月”,此伐商之四月也。“哉,始”,\CJKunderwave{釋詁}文。\CJKunderwave{顧命}傳以“哉生魄”為十六日,則“哉生明”為月初矣。以三日月光見,故傳言“始生明,月三日”也。此經無日,未必非二日也。“生明”、“死魄”俱是月初,上云“死魄”,此云“生明”,而魄死明生互言耳。 \par}

乃偃武修文,\footnote{倒載干戈,包以虎皮,示不用。行禮射,設庠序,修文教。}歸馬于華山之陽,放牛于桃林之野,示天下弗服。\footnote{山南曰陽。桃林在華山東。皆非長養牛馬之地,欲使自生自死,示天下不復乘用。○華,胡化、胡瓜二反;華山在恆農。長,丁丈反。復,扶又反。}丁未,祀于周廟,邦甸、侯、衛,駿奔走,執豆籩。\footnote{四月丁未,祭告后稷以下、文考\CJKunderline{文王}以上七世之祖。駿,大也。邦國甸侯、衛服諸侯皆大奔走于廟執事。○駿,荀俊反。豆,本又作梪。籩,音邊。上,時掌反。}越三日庚戌,柴望,大告武成。\footnote{燔柴郊天,望祀山川,先祖後郊,自近始。}

{\noindent\zhuan\zihao{6}\fzbyks 傳“倒載”至“文教”。正義曰:\CJKunderwave{樂記}云,\CJKunderline{武王}克殷,“濟河而西。車甲釁而藏之府庫,倒載干戈,包之以虎皮,天下知\CJKunderline{武王}之不復用兵也。散軍而郊射。左射,\CJKunderwave{貍首}。右射,\CJKunderwave{騶虞}。而貫革之射息也”。是“偃武修文”之事,故傳引之。郊射是禮射也。\CJKunderwave{王制}論四代學名云:“虞謂之庠,夏謂之序。”故言“設庠序,修文教”也。 \par}

{\noindent\zhuan\zihao{6}\fzbyks 傳“山南”至“乘用”。正義曰:\CJKunderwave{釋山}云:“山西曰夕陽,山東曰朝陽。”李巡曰:“山西暮乃見日,故曰夕陽。山東朝乃見日,故云朝陽。”“陽”以見日為名,故知“山南曰陽”。杜預云:“桃林之塞,今宏農華陰縣潼關是也。”是在“華山東”也。指其所往謂之“歸”,據我釋之則云“放”,“放牛”、“歸馬”互言之耳。華山之旁尤乏水草,非長養牛馬之地,欲使自生自死。此是戰時牛馬,故放之,示天下不復乘用。\CJKunderwave{易·繫辭}云:“服牛乘馬。”“服”、“乘”俱是用義,故以“服”總牛馬。 \par}

{\noindent\zhuan\zihao{6}\fzbyks 傳“四月”至“執事”。正義曰:以“四月”之字,隔文已多,故言“四月丁未”。此以成功設祭,明其遍告群祖,知告“后稷以下”。后稷則始祖以下,容毀廟也。天子七廟,故云“文考\CJKunderline{文王}以上七世之祖”。見是周廟皆祭之,故經總云“周廟”也。“駿,大”,\CJKunderwave{釋詁}文。\CJKunderwave{周禮}六服侯、甸、男、採、衛、要,此略舉邦國在諸侯服,故云“甸、侯、衛”,其言不次。\CJKunderwave{詩·頌}云“駿奔走在廟。”故云:“皆大奔走于廟執事”也。“越三日庚戌”。正義曰:\CJKunderwave{召誥}云“越三日”者,皆從前至今為三日,此從丁未數之,則為四日,蓋史官不同,立文自異。或此“三”當為“四”,由字積與誤。 \par}

{\noindent\shu\zihao{5}\fzkt “惟一”至“武成”。正義曰:此歷敘伐紂往反祖廟告天時日,說武功成之事也。“一月壬辰,旁死魄”,謂伐紂之年周正月辛卯朔,其二日是壬辰也。“翼日癸巳,王朝步自周,于徵伐商”,謂正月三日發鎬京始東行也。其月二十八日戊午渡河。\CJKunderwave{泰誓}序云“一月戊午,師渡孟津”,\CJKunderwave{泰誓}中篇云“惟戊午,王次于河朔”是也。二月辛酉朔,甲子殺紂,\CJKunderwave{牧誓}云“時甲子昧爽,乃誓”是也。其年閏二月庚寅朔,三月庚申朔,四月己丑朔。“厥四月,哉生明,王來自商,至于豐”,謂四月三日,月始生明,其日當是辛卯也。“丁未,祀于周廟”,四月十九日也。“越三日庚戌,柴望”,二十二日也。正月始往伐,四月告成功,史敘其事,見其功成之次也。\CJKunderwave{漢書·律歷志}引\CJKunderwave{武成}篇云:“惟一月壬辰,旁死魄。若翼日癸巳,\CJKunderline{武王}乃朝步自周,于徵伐紂。越若來二月既死魄,越五日甲子,咸劉商王紂。惟四月既旁生魄,越六日庚戌,\CJKunderline{武王}燎于周廟。翼日辛亥,祀于天位。越五日乙卯,乃以庶國祀于周廟。”與此經不同。彼是焚書之後,有人偽為之。漢世謂之“逸書”,其後又亡其篇。\CJKunderline{鄭玄}云:“\CJKunderwave{武成}逸書,建武之際亡。”謂彼偽\CJKunderwave{武成}也。 \par}

既生魄,庶邦冢君暨百工,受命于周。\footnote{魄生明死,十五日之後,諸侯與百官受政命于周。明一統。○暨,其器反。}

{\noindent\zhuan\zihao{6}\fzbyks 傳“魄生”至“一統”。正義曰:月以望虧,望是月半,望在十六日為多,通率在十六日者,四分居三,其一在十五日耳。此言“既生魄”,故言“魄生明死,十五日之後”也。“丁未,祀于周廟”,已是此月十九日矣,此“受命于周”,繼生魄言之,則受命在祀廟之前,故祀廟之時諸侯已奔走執事,豈得未受周命,已助周祭?明其受命在祀廟前矣。史官探其時日,先言告武成既訖,然後卻說受命,故文在下耳。諸侯與百官,舊有未屬周者,今皆受政命于周,于此時始天下一統也。顧氏以既生魄謂庚戌已後,雖十六日始生魄,從十六日至晦皆為生魄,但不知庚戌之後幾日耳。 \par}

王若曰:“嗚呼!群后,\footnote{順其祖業嘆美之,以告諸侯。}惟先王建邦啟土,\footnote{謂后稷也。尊祖,故稱先王。}

{\noindent\zhuan\zihao{6}\fzbyks 傳“謂後”至“先王”。正義曰:此“先王”文在“\CJKunderline{公劉}”之前,知“謂后稷也”。后稷非王,尊其祖,故稱先王。\CJKunderwave{周語}云“昔我先王后稷”,又曰“我先王不窋”,韋昭云:“王之先祖,故稱王。”\CJKunderwave{商頌}亦以契為“玄王”。文武之功,起于後稷,后稷始封于邰,故言“建邦啟土”。 \par}

\CJKunderline{公劉}克篤前烈,\footnote{后稷曾孫。公,爵。劉,名。能厚先人之業。}

{\noindent\zhuan\zihao{6}\fzbyks 傳“后稷”至“之業”。正義曰:\CJKunderwave{周本紀}云:“后稷卒,子不窋立。卒,子鞠陶立。卒,子\CJKunderline{公劉}立。”是\CJKunderline{公劉}為后稷曾孫也。\CJKunderwave{本紀}云,\CJKunderline{公劉}之後有公非、公祖之類,知“公”是爵。殷時未諱,故稱劉名。先公多矣,獨三人稱“公”,當時之意耳。\CJKunderwave{本紀}云:“\CJKunderline{公劉}復修后稷之業,百姓懷之,多徙而歸保焉。”周道之興,自此之後,是“能厚先人之業”也。 \par}

至于\CJKunderline{大王},肇基王跡,\CJKunderline{王季}其勤王家。\footnote{\CJKunderline{大王}修德以翦齊商人,始王業之肇跡\CJKunderline{王季}纘統其業,乃勤立王業。○大音太。肇音兆。王跡,于況反,又如字,注“王業”、“王功”同。}

{\noindent\shu\zihao{5}\fzkt 傳“\CJKunderline{大王}”至“王家”。正義曰:\CJKunderwave{詩}云:“后稷之孫,實惟\CJKunderline{大王}。居岐之陽,實始翦商。”是\CJKunderline{大王}翦齊商人,始王業之兆跡也。\CJKunderwave{周本紀}云:“\CJKunderline{王季}修古公之道,諸侯順之。”是能纘統\CJKunderline{大王}之業,勤立王家之基本也。 \par}

我文考\CJKunderline{文王},克成厥勳,誕膺天命,以撫方夏。\footnote{言我文德之父,能成其王功,大當天命,以撫綏四方中夏。}大邦畏其力,小邦懷其德。\footnote{言天下諸侯,大者畏威,小者懷德,是\CJKunderline{文王}威德之大。}

{\noindent\shu\zihao{5}\fzkt “大邦”至“其德”。正義曰:大邦力足拒敵,故言“畏其力”,小邦必畏矣。小邦或被棄遺,故言“懷其德”,大邦亦懷德矣。量事為文也。 \par}

惟九年,大統未集。\footnote{言諸侯歸之,九年而卒,故大業未就。}

{\noindent\zhuan\zihao{6}\fzbyks 傳“言諸”至“未就”。正義曰:\CJKunderline{文王}斷虞芮之訟,諸侯歸之,改稱元年。至九年而卒,故云“大業未就”也。\CJKunderline{文王}既未稱王,而得輒改元年者,諸侯自于其國各稱元年,是己之所稱,容或中年得改矣。\CJKunderwave{汲冢竹書}魏惠王有後元年,漢初文帝二元,景帝三元,此必有因于古也。\CJKunderline{伏生}、司馬遷、韓嬰之徒不見此書,以為\CJKunderline{文王}受命七年而崩,故\CJKunderline{鄭玄}等皆依用之。 \par}

予小子其承厥志,\footnote{言承\CJKunderline{文王}本意。}厎商之罪,告于皇天后土、所過名山大川,\footnote{致商之罪,謂伐紂之時。后土,社也。名山,華嶽。大川,河。○厎,之履反。}

{\noindent\zhuan\zihao{6}\fzbyks 傳“致商”至“川河”。正義曰:“致商之罪,謂伐紂之時”,欲將伐紂,告天乃發,故文在“所過”之上。\CJKunderwave{禮}天子出征,必類帝宜社。此告皇天后土,即\CJKunderwave{泰誓}上篇“類于上帝,宜于冢土”,故云“后土,社也。”昭二十九年\CJKunderwave{左傳}稱“句龍為后土”,后土為社是也。僖十五年\CJKunderwave{左傳}云,戴皇天而履後上。彼晉大夫要秦伯,故以地神后土而言之,與此異也。自周適商,路過河華,故知所過名山華嶽、大川河也。山川大乃有名,“名”、“大”互言之耳。\CJKunderwave{周禮·大祝}云:“王過大山川,則用事焉。”鄭云:“用事,用祭事告行也。” \par}

曰:‘惟有道曾孫周王\CJKunderline{發},將有大正于商。\footnote{告天地山川之辭。大正,以兵徵之也。}

{\noindent\shu\zihao{5}\fzkt “曰惟有道曾孫周王發”。正義曰:自稱“有道”者,聖人至公,為民除害,以紂無道,言己有道,所以告神求助,不得飾以謙辭也。稱“曾孫”者,\CJKunderwave{曲禮}說諸侯自稱之辭云:“臨祭祀,內事曰孝子某侯某,外事曰曾孫某侯某。”哀二年\CJKunderwave{左傳}蒯聵禱祖亦自稱曾孫,皆是言己承藉上祖奠享之意。 \par}

今商王\CJKunderline{受}無道,\footnote{無道德。}暴殄天物,害虐烝民,\footnote{暴絕天物,言逆天也。逆天害民,所以為無道。○烝,之承反。}

{\noindent\shu\zihao{5}\fzkt “暴殄”至“烝民”。正義曰:“天物”語闊,人在其間,以人為貴,故別言害民。則“天物”之言,除人外,普謂天下百物、鳥獸草木皆暴絕之。 \par}

為天下逋逃主,萃淵藪。\footnote{逋,亡也。天下罪人逃亡者,而紂為魁主,窟聚淵府藪澤。言大奸。○萃,在醉反。藪,素口反。魁,苦回反。窟,口忽反。}

{\noindent\zhuan\zihao{6}\fzbyks 傳“逋亡”至“大奸”。正義曰:“逋”亦逃也,故以為亡。罪人逃亡,而紂為魁主。“魁”,首也,言受用逃亡者,與之為魁首,為主人。“萃”訓聚也,言若蟲獸入窟,故云“窟聚”。水深謂之“淵”,藏物謂之“府”。史游\CJKunderwave{急就篇}云:“司農少府國之淵。”“淵”、“府”類,故言“淵府”。水鍾謂之“澤”,無水則名“藪”。“藪”、“澤”大同,故言“藪澤”。“萃淵藪”三者各為物室,言紂與亡人為主,亡人歸之若蟲之窟聚,魚歸淵府,獸集藪澤,言紂為大奸也。據傳意,“主”字下讀為便。昭七年\CJKunderwave{左傳}引此文,杜預云:“萃,集也。天下逋逃悉以紂為淵藪,集而歸之。”與孔異也。 \par}

予小子既獲仁人,敢祇承上帝,以遏亂略。\footnote{仁人,謂大公、周、召之徒。略,路也。言誅紂敬承天意以絕亂路。○遏,烏末反。召,上照反,本又作邵。}華夏蠻貊,罔不率俾。恭天成命,\footnote{冕服採章曰華,大國曰夏,及四夷皆相率而使奉天成命。○貊,亡白反。俾,必爾反。}

{\noindent\zhuan\zihao{6}\fzbyks 傳“冕服”至“成命”。正義曰:“冕服採章”對被髮左衽,則為有光華也。\CJKunderwave{釋詁}云:“夏,大也。”故大國曰“夏華”。“夏”謂中國也。言“蠻貊”則戎夷可知也。言華夏及四夷皆相率而充己,使奉天成命,欲其共伐紂也。 \par}

肆予東征,綏厥士女。\footnote{此謂十一年會孟津還時。}惟其士女,篚厥玄黃,昭我周王。\footnote{言東國士女筐篚盛其絲帛,奉迎道次。明我周王為之除害。○篚音匪。為,于偽反。}天休震動,用附我大邑周。\footnote{天之美應,震動民心,故用依附我。○應,應對之應。}惟爾有神,尚克相予,以濟兆民,無作神羞。’”\footnote{神庶幾助我渡民危害,無為神羞辱。○相,息亮反。}既戊午,師逾孟津。癸亥,陳于商郊,俟天休命。\footnote{自河至朝歌,出四百里,五日而至。赴敵宜速,待天休命,謂夜雨止畢陳。○逾,亦作逾。陳,直刃反,計同,徐音塵。}甲子昧爽,\CJKunderline{受}率其旅若林,會于牧野。\footnote{旅,眾也。如林,言盛多。會,逆距戰。}



{\noindent\zhuan\zihao{6}\fzbyks 傳“自河”至“畢陳”。正義曰:“出四百里”,驗地為然。戊午明日猶誓于河朔,癸亥已陳于商郊,凡經五日,日行八十里,所以疾者,“赴敵宜速”也。\CJKunderwave{帝王世紀}云:“王軍至鮪水,紂使膠鬲候周師,見王問曰:‘西伯將焉之?’王曰:‘將攻薛也。’膠鬲曰:‘然,原西伯無我欺。’王曰:‘不子欺也,將之殷。’膠鬲曰:‘何日至?’王曰:‘以甲子日,以是報矣。’膠鬲去而報命于紂。而雨甚,軍卒皆諫王曰:‘卒病,請休之。’王曰:‘吾已令膠鬲以甲子報其主矣。吾雨而行,所以救膠鬲之死也。’遂行,甲子至于商郊。”然則本期甲子,故遠行也。\CJKunderwave{周語}云:“王以二月癸亥夜陳,未畢而雨。”是“雨止畢陳”也。“待天休命”,雨是天之美命也。韋昭云:“雨者,天地神人和同之應也。”天地氣和乃有雨降,是雨為和同之應也。 \par}

{\noindent\zhuan\zihao{6}\fzbyks 傳“旅眾”至“距戰”。正義曰:“旅,眾”,\CJKunderwave{釋詁}文。\CJKunderwave{詩}亦云:“其會如林。”言盛多也。\CJKunderwave{本紀}云:“紂發兵七十萬人以距\CJKunderline{武王}。”紂兵雖則眾多,不得有七十萬人,是史官美其能破強敵,虛言之耳。 \par}

罔有敵于我師,前徒倒戈,攻于後以北,血流漂杵。\footnote{紂眾服周仁政,無有戰心,前徒倒戈,自攻于後以北走,血流漂舂杵。甚之言。○倒,丁老反。漂,四妙反,徐敷妙反,又四消反。杵,昌呂反。}

{\noindent\zhuan\zihao{6}\fzbyks 傳“紂眾”至“之言”。正義曰:“罔有敵于我師”,言紂眾雖多,皆無有敵我之心,故“自攻于後以北走”。自攻其後,必殺人不多,“血流漂舂杵,甚之言”也。\CJKunderwave{孟子}云:“信\CJKunderwave{書}不如無\CJKunderwave{書},吾于\CJKunderwave{武成}取二三策而已。仁者無敵于天下,以至仁伐不仁,如何其血流漂杵也?”是言不實也。\CJKunderwave{易·繫辭}云:“斷木為杵,掘地為臼。”是“杵”為臼器也。 \par}

{\noindent\shu\zihao{5}\fzkt “既戊午”至“我師”。正義曰:自此以下皆史辭也,其上闕絕,失其本絕,故文無次第。必是王言既終,史乃更敘戰事。于文次當承“自周,于徵伐商”之下,此句次之,故云“既戊午”也。史官敘事,得言“罔有敵于我師”,稱“我”者,猶如自漢至今,文章之士,雖民論國事,莫不稱“我”,皆云“我大隨”,以心體國,故稱“我”耳,非要王言乃稱“我”也。 \par}

一戎衣,天下大定。\footnote{衣,服也。一著戎服而滅紂,言與眾同心,動有成功。}乃反商政,政由舊。\footnote{反紂惡政,用商先王善政。}釋\CJKunderline{箕子}囚,封\CJKunderline{比干}墓,式\CJKunderline{商容}閭。\footnote{皆\CJKunderline{武王}反紂政。囚,奴,徒隸。封,益其土。\CJKunderline{商容},賢人,紂所貶退,式其閭巷以禮賢。}

{\noindent\zhuan\zihao{6}\fzbyks 傳“皆武”至“禮賢”。正義曰:紂囚其人而放釋之,紂殺其身而增封其墓,紂退其人而式其門閭,皆是“\CJKunderline{武王}反紂政”也。下句散其財粟,亦是反紂,于此須有所解,因言之耳。上篇云“囚奴正士”,\CJKunderwave{論語}云:“\CJKunderline{箕子}為之奴”,是紂囚之,又為奴役之。\CJKunderwave{周禮·司厲職}云:“其奴男子入于罪隸。”\CJKunderline{鄭玄}云:“為之奴者,繫于罪隸之官。”是“囚”為奴,以徒隸役之也。“\CJKunderline{商容}”,賢人之姓名,紂所貶退,處于私室。“式”者,車上之橫木,男子立乘,有所敬則俯而憑式,遂以“式”為敬名。\CJKunderwave{說文}云:“閭,族居里門也。”\CJKunderline{武王}過其閭而式之,言此內有賢人,式之禮賢也。\CJKunderwave{帝王世紀}云:“\CJKunderline{商容}及殷民觀周軍之入,見\CJKunderline{畢公}至,殷民曰:‘是吾新君也。’容曰:‘非也,視其為人嚴乎將有急色,故君子臨事而懼。’見太公至,民曰:‘是吾新君也。’容曰:‘非也,視其為人虎據而鷹趾,當敵將眾,威怒自倍,見利即前,不顧其後,故君子臨眾,果于進退。’見\CJKunderline{周公}至,民曰:‘是吾新君也。’容曰:‘非也,視其為人忻忻休休,志在除賊,是非天子,則周之相國也,故聖人臨眾知之。’見\CJKunderline{武王}至,民曰:‘是吾新君也。’容曰:‘然,聖人為海內討惡,見惡不怒,見善不喜,顏色相副,是以知之。’”是說\CJKunderline{商容}之事也。 \par}

散鹿臺之財,發巨橋之粟,\footnote{紂所積之府倉,皆散發以賑貧民。○散,西旦反。}

{\noindent\zhuan\zihao{6}\fzbyks 傳“紂所”至“貧民”。正義曰:藏財為府,藏粟為倉,故言“紂所積之府倉”也。名曰“鹿臺”,“巨橋”則其義未聞。“散”者言其分佈,“發”者言其開出,互相見也。\CJKunderwave{周本紀}云:“命\CJKunderline{召公}釋\CJKunderline{箕子}之囚,命\CJKunderline{畢公}釋百姓之囚,表\CJKunderline{商容}之閭,命\CJKunderline{閎夭}封\CJKunderline{比干}之墓,命\CJKunderline{南宮括}散鹿臺之錢,發巨橋之粟以賑貧弱也。”然則\CJKunderline{武王}親式\CJKunderline{商容}之閭又表之也。\CJKunderwave{新序}云“鹿臺其大三里,其高千尺”,則容物多矣。此言“鹿臺之財”,則非一物也。\CJKunderwave{史記}作“錢”,後世追論,以錢為主耳。\CJKunderwave{周禮}有泉府之官,\CJKunderwave{周語}稱景王鑄大錢,是周時已名泉為錢也。 \par}

大賚于四海,而萬姓悅服。\footnote{施捨已債,救乏周無,所謂周有大賚,天下皆悅仁服德。○賚,力代反,徐音來。已音以。債,側界反。周音周,本亦作周。}

{\noindent\zhuan\zihao{6}\fzbyks 傳“施捨”至“服德”。正義曰:\CJKunderwave{左傳}成十八年,晉悼公初立,“施捨,已責”。成二年楚將起師,“已責,救乏”。定五年“歸粟于蔡,以周急,矜無資也”。杜預以為“施恩惠,舍勞役”也,已責,“止逋責”也。皆是恤民之事,故傳引之以證“大賚”。“所謂周有大賚”,\CJKunderwave{論語}文。孔安國解\CJKunderwave{堯曰}之篇,有二帝三王之事,“周有大賚”正指此事,故言“所謂”也。“悅”是歡喜,“服”謂聽從,感恩則悅,見義則服,故“天下皆悅仁服德”也。\CJKunderwave{帝王世紀}云:“王命封墓釋囚,又歸施鹿臺之珠玉及傾宮之女于諸侯,殷民咸喜曰:‘王之于仁人也,死者猶封其墓,況生者乎?王之于賢人也,亡者猶表其閭,況存者乎?王之于財也,聚者猶散之,況其復籍之乎?王之于色也,見在者猶歸其父母,況其復徵之乎?’”是悅服之事也。 \par}

列爵惟五,\footnote{即所識政事而法之。爵五等,公侯伯子男。}分土惟三。\footnote{列地封國,公侯方百里,伯七十里,子男五十里,為三品。}

{\noindent\zhuan\zihao{6}\fzbyks 傳“列地”至“三品”。正義曰:爵五等,地三品,\CJKunderline{武王}于此既從殷法,未知\CJKunderline{周公}制禮亦然以否。\CJKunderwave{孟子}曰:“北宮錡問于孟子曰:‘周之班爵祿如何?’孟子曰:‘其詳不可得聞矣,嘗聞其略。天子之制,地方千里,公侯方百里,伯七十里,子男五十里。’”\CJKunderwave{漢書·地理志}亦云:“周爵五等,其土三等也,公侯百里,伯七十里,子男五十里。”漢世儒者多以為然,包咸注\CJKunderwave{論語}云:“千乘之國,百里之國也,謂大國惟百里耳。”\CJKunderwave{周禮·大司徒}云:“諸公之地,封疆方五百里。侯四百里,伯三百里,子二百里,男一百里。”蓋是周室既衰,諸侯相併,自以國土寬大,皆違禮文,乃除去本經,妄為說耳。\CJKunderline{鄭玄}之徒以為\CJKunderline{武王}時大國百里,\CJKunderline{周公}制禮大國五百里,\CJKunderwave{王制}之注具矣。 \par}

建官惟賢,\footnote{立官以官賢才。}位事惟能。\footnote{居位理事,必任能事。}重民五教,\footnote{所重在民及五常之教。}

{\noindent\shu\zihao{5}\fzkt “重民五教”。正義曰:以“重”總下五事,民與五教,“食喪祭”也。“五教”所以教民,故與“民”同句。下句“食”與“喪”、“祭”三者各為一事,相類而別,故以“惟”目之,言此皆聖王所重也。\CJKunderwave{論語}云:“所重民食喪祭。”以\CJKunderwave{論語}即是此事,而彼無五教,錄\CJKunderwave{論語}者自略之耳。 \par}

惟食、喪、祭。\footnote{民以食為命,喪禮篤親愛,祭祀崇孝養,皆聖王所重。○養,羊亮反。}惇信明義,\footnote{使天下厚行信,顯忠義。}崇德報功,\footnote{有德尊以爵,有功報以祿。}垂拱而天下治。\footnote{言\CJKunderline{武王}所修皆是,所任得人,故垂拱而天下治。○任,而鴆反。治,直吏反。}

{\noindent\shu\zihao{5}\fzkt “垂拱而天下治”。正義曰:\CJKunderwave{說文}云:“拱,斂手也。”“垂拱而天下治”,謂所任得人,人皆稱職,手無所營,下垂其拱,故美其“垂拱而天下治”也。 \par}

%%% Local Variables:
%%% mode: latex
%%% TeX-engine: xetex
%%% TeX-master: "../Main"
%%% End:
