%% -*- coding: utf-8 -*-
%% Time-stamp: <Chen Wang: 2024-04-02 11:42:41>

% {\noindent \zhu \zihao{5} \fzbyks } -> 注 (△ ○)
% {\noindent \shu \zihao{5} \fzkt } -> 疏

\chapter{卷四}


\section{大禹謨第三【偽】}


 {\noindent\zihao{6}\fzbyks \CJKunderwave{釋文}:“徐云:‘本\CJKunderwave{虞書}裛為一卷,凡十二卷,今依\CJKunderwave{七志}、\CJKunderwave{七錄}為十三卷。’” \par}

\textcolor{red}{\CJKunderline{皋陶}}矢厥謨,\footnote{矢,陳也。皋音高。陶音遙。}\CJKunderline{禹}成厥功,\footnote{陳其成功。}\CJKunderline{帝舜}申之。\footnote{申,重也,重美二子之言。○重,直用反。}作\CJKunderwave{大禹}、\CJKunderwave{皋陶謨}、\footnote{大禹謀九功,\CJKunderline{皋陶}謀九德。謨亦作謩。}\CJKunderwave{\textcolor{red}{益稷}}。\footnote{凡三篇。}

{\noindent\zhuan\zihao{6}\fzbyks 傳“矢,陳也”。正義曰:“矢,陳”,\CJKunderwave{釋詁}文。 \par}

{\noindent\zhuan\zihao{6}\fzbyks 傳“陳其成功”。正義曰:此是謨篇,\CJKunderline{禹}成其功,陳其言耳。蒙上“矢”文,故傳明之,言“陳其成功”也。序“成”在“厥”上,傳“成”在下者,序順上句,傳從便文,故倒也。 \par}

{\noindent\zhuan\zihao{6}\fzbyks 傳“申重”至“之言”。正義曰:“申,重”,\CJKunderwave{釋詁}文。\CJKunderwave{大禹謨}云:“帝曰:‘俞!地平天成,時乃功。’”又“帝曰:‘\CJKunderline{皋陶},惟茲臣庶,罔或於予正。時乃功,懋哉!”\CJKunderwave{益稷}云:“迪朕德,時乃功。”皆是重美二子之言也。 \par}

{\noindent\zhuan\zihao{6}\fzbyks 傳“\CJKunderline{大禹}”至“九德”。正義曰:二篇皆是謨也,序以一“謨”總二篇,故傳明之。\CJKunderline{大禹}治水能致九功而言“謨”,以其序有“謨”文,故云“謨”也。 \par}

{\noindent\zhuan\zihao{6}\fzbyks 傳“凡三篇”。正義曰:\CJKunderwave{益稷}亦\CJKunderline{大禹}所謀,不言“謨”者,\CJKunderline{禹}謀言及益稷,非是益稷為謀,不得言\CJKunderwave{益稷謨}也。其篇雖有“\CJKunderline{夔}曰”,夔言樂和,本非謀慮,不得謂之“夔謨”。 \par}

{\noindent\shu\zihao{5}\fzkt “\CJKunderline{皋陶}”至“益稷”。正義曰:\CJKunderline{皋陶}為\CJKunderline{帝舜}陳其謀,\CJKunderline{禹}為\CJKunderline{帝舜}陳已成所治水之功,\CJKunderline{帝舜}因其所陳從而重美之。史錄其辭,作\CJKunderwave{大禹}、\CJKunderwave{皋陶}二篇之謨,又作\CJKunderwave{益稷}之篇,凡三篇也。篇先\CJKunderwave{大禹},序先言\CJKunderline{皋陶}者,\CJKunderwave{皋陶}之篇,\CJKunderline{皋陶}自先發端,\CJKunderline{禹}乃然而問之,\CJKunderline{皋陶}言在\CJKunderline{禹}先,故序先言\CJKunderline{皋陶}。其此篇以功大為先,故先\CJKunderwave{禹}也。\CJKunderwave{益稷}之篇亦是\CJKunderline{禹}之所陳,因\CJKunderline{皋陶}之言。而\CJKunderline{禹}論益稷,在\CJKunderwave{皋陶謨}後,故後其篇。 \par}

大禹謨\footnote{禹稱大,大其功。謨,謀也。}

{\noindent\zhuan\zihao{6}\fzbyks 傳“\CJKunderline{禹}稱”至“謀也”。正義曰:余文單稱“\CJKunderline{禹}”,而此獨加“大”者,故解之:\CJKunderline{禹}與\CJKunderline{皋陶}同為舜謀,而\CJKunderline{禹}功實大,\CJKunderline{禹}與\CJKunderline{皋陶}不等,史加大其功,使異於\CJKunderline{皋陶},於此獨加“大”字與\CJKunderline{皋陶}總言故也。“謨,謀”,\CJKunderwave{釋詁}文。此三篇皆是舜史所錄,上取堯事,下錄\CJKunderline{禹}功,善於堯之知己,又美所禪得人,故包括上下以為\CJKunderwave{虞書}。其事以類相從,非由事之先後。若其不然,上篇已言舜死,於此豈死後言乎?此篇已言禪\CJKunderline{禹},下篇豈受禪後乎?聰明史以類聚為文。計此三篇,\CJKunderline{禹}謨最在後,以\CJKunderline{禹}功大,故進之於先。\CJKunderwave{孟子}稱“舜薦\CJKunderline{禹}於天,十有七年”,則\CJKunderline{禹}攝一十七年,舜陟方乃死。不知\CJKunderline{禹}徵有苗,在攝幾年。史述\CJKunderline{禹}之行事,不必以攝位之年即徵苗民也。 \par}

曰若稽古,\CJKunderline{大禹}\footnote{順考古道而言之。}曰:“文命敷於四海,祗承於帝。\footnote{言其外布文德教命,內則敬承堯舜。文命,孔云:“文德教命也。”先儒云:“文命,\CJKunderline{禹}名。”}

{\noindent\zhuan\zihao{6}\fzbyks 傳“順考”至“言之”。正義曰:典是常行,謨是言語,故傳於典云“行之”,於謨云“言之”,皆是順考古道也。 \par}

{\noindent\zhuan\zihao{6}\fzbyks 傳“言其”至“堯舜”。正義曰:“敷於四海”即敷此文命,故言“外布文德教命”也。“四海”舉其遠地,故傳以“外”、“內”言之。“祗”訓敬也,\CJKunderline{禹}承堯舜二帝,故云“敬承堯舜”。傳不訓“祗”而直言“敬”,以易知而略之。 \par}

{\noindent\shu\zihao{5}\fzkt “曰若”至“於帝”。正義曰:史將錄\CJKunderline{禹}之事,故為題目之辭曰,能順而考案古道而言之者,是大功之\CJKunderline{禹}也。此\CJKunderline{禹}能以文德教命布陳於四海,又能敬承堯舜。外布四海,內承二帝,言其道周備。 \par}

\textcolor{red}{曰}:“後克艱厥后,臣克艱厥臣,政乃乂,黎民敏德。”\footnote{敏,疾也。能知為君難,為臣不易,則其政治,而眾民皆疾修德。○易,以豉反。治,直吏反。}帝曰:“俞!允若茲,嘉言罔攸伏,野無遺賢,萬邦咸寧。\footnote{攸,所也。善言無所伏,言必用。如此則賢才在位,天下安寧。俞,羊朱反。攸音由,徐以帚反。}稽於眾,捨己從人,不虐無告,不廢困窮,惟帝\textcolor{red}{時克}。”\footnote{帝謂堯也,舜因嘉言無所伏,遂稱堯德以成其義。考眾從人,矜孤愍窮,凡人所輕,聖人所重。舍音舍。告,故毒反。矜,居陵反。}

{\noindent\zhuan\zihao{6}\fzbyks 傳“敏疾”至“修德”。正義曰:許慎\CJKunderwave{說文}云:“敏,疾也。”是相傳為訓。“為君難,為臣不易”,\CJKunderwave{論語}文。能知為君難,為臣不易,則當謹慎恪勤,求賢自輔,故其政自然治矣。見善則用,知賢必進,眾民各自舉,則皆疾修德矣。此經上不言\CJKunderline{禹}者,承上\CJKunderline{禹}事,以可知而略之。 \par}

{\noindent\zhuan\zihao{6}\fzbyks 傳“攸所”至“下安”。正義曰:“攸,所”,\CJKunderwave{釋言}文。“善言無所伏”者,言其必用之也。言之善者必出賢人之口,但言之易,行之難,或有人不賢而言可用也,故“嘉言”與“賢”異其文也。如此用善言,任賢才在位,則天下安。 \par}

{\noindent\zhuan\zihao{6}\fzbyks 傳“帝謂”至“所重”。正義曰:舜稱為“帝”,故知“帝謂堯”也。舜因嘉言無所伏,以為堯乃能然,故遂稱堯德以成其義。此\CJKunderline{禹}言之義,以堯之聖智,無所不能,惟言其“考眾從人,矜孤愍窮”,以為堯之美者,此是“凡人所輕,聖人所重”。“不虐”、“不廢”,皆謂矜撫愍念之,互相通也。\CJKunderwave{王制}云:“少而無父謂之孤,老而無子謂之獨,老而無妻謂之鰥,老而無夫謂之寡。此四者天民之窮而無告者。”故此“無告”是彼四者。彼四者而此惟言孤者,四者皆孤也,言“孤”足以總之。言“困窮”,謂貧無資財也。 \par}

{\noindent\shu\zihao{5}\fzkt “曰後”至“時克”。正義曰:\CJKunderline{禹}為\CJKunderline{帝舜}謀曰:“君能重難其為君之事,臣能重難其為臣之職,則上之政教乃治,則下之眾民皆化而疾修其德。而帝曰:“然。信能如此,君臣皆能自難,並願善以輔己,則下之善言無所隱伏,在野無遺逸之賢,賢人盡用,則萬國皆安寧也。為人上者考於眾言,觀其是非,舍已之非,從人之是。不苛虐鰥寡孤獨無所告者,必哀矜之;不廢棄困苦貧窮無所依者,必愍念之。惟\CJKunderline{帝堯}於是能為此行,餘人所不能。”言“克艱”之不易也。 \par}

\CJKunderline{益}曰:“都!帝德廣運,乃聖乃神,乃武乃文。\footnote{益因舜言又美堯也。“廣”謂所覆者大,“運”謂所及者遠。聖無所不通,神妙無方,文經天地,武定禍亂。}皇天眷命,奄有四海,為天下君。”\footnote{眷,視。奄,同也。言堯有此德,故為天所命,所以勉舜也。眷,居倦反。奄,於檢反。}

{\noindent\zhuan\zihao{6}\fzbyks 傳“益因”至“禍亂”。正義曰:“廣”者闊之義,故為“所覆者大”。“運”者動之言,故為“所及者遠”。\CJKunderwave{洪範}云“睿作聖”,言通知眾事,故為“無所不通”。案\CJKunderwave{易}曰“神者妙萬物而為言也”,又曰“神妙無方”,此言神道微妙,無可比方,不知其所以然。\CJKunderwave{易}又云:“陰陽不測之謂神。”\CJKunderwave{諡法}云:“經緯天地曰文,克定禍亂曰武。”經傳“文”、“武”倒者,經取韻句,傳以文重故也。 \par}

{\noindent\zhuan\zihao{6}\fzbyks 傳“眷視”至“勉舜”。正義曰:\CJKunderwave{詩}云:“乃眷西顧。”謂視而回首。\CJKunderwave{說文}亦以“眷”為視。“奄,同”,\CJKunderwave{釋言}文。益因帝言盛稱堯善者,亦勸勉舜,冀之必及堯也。 \par}

{\noindent\shu\zihao{5}\fzkt “\CJKunderline{益}曰”至“下君”。正義曰:益承帝言,嘆美堯德曰:“嗚呼!\CJKunderline{帝堯}之德,廣夫運行。乃聖而無所不通,乃神而微妙無方,乃武能克定禍亂,乃文能經緯天地。以此為大天顧視而命之,使同有四海之內,為天下之君。” \par}

\CJKunderline{禹}曰:“惠迪吉,從逆兇,惟影響。”\footnote{迪,道也。順道吉從逆兇。吉凶之報,若影之隨形,響之應聲。言不虛。迪,徒力反。響,許丈反。}\CJKunderline{益}曰:“吁!戒哉!儆戒無虞,罔失法度。\footnote{先籲後戒,欲使聽者精其言。虞,度也。無億度,謂無形。戒於無形,備慎深。秉法守度,言有恆。籲,況俱反。度,徒布反。虞度,徙洛反。}罔遊於逸,罔淫於樂。\footnote{淫,過也。遊逸過樂,敗德之原。富貴所忽,故特以為戒。樂音洛。}任賢勿貳,去邪勿疑。疑謀勿成,百志惟熙。\footnote{一意任賢,果於去邪,疑則勿行,道義所存於心,日以廣矣。去,起呂反。熙,火其反。罔違道以幹百姓之譽,幹,求也。失道求名,古人賤之。}罔咈百姓以從己之慾。\footnote{咈,戾也。專欲難成,犯眾興禍,故戒之。咈,扶弗反。戾,連弟反。}無怠無荒,四夷來王。”\footnote{言天子常戒慎,無怠惰荒廢,則四夷歸往之。怠音待。惰,徒臥反。}

{\noindent\zhuan\zihao{6}\fzbyks 傳“迪,道也”。正義曰:\CJKunderwave{釋詁}文。 \par}

{\noindent\zhuan\zihao{6}\fzbyks 傳“先籲”至“有恆”。正義曰:\CJKunderwave{堯典}傳云:“籲,疑怪之辭。”此無可怪,聞善驚而為聲耳。“先籲後戒”者,驚其言之美,然後設戒辭,欲使聽者精審其言。“虞,度”,\CJKunderwave{釋詁}文。“無億度”者,謂不有此事,無心億度之。\CJKunderwave{曲禮}云:“凡為人子者,聽於無聲,視於無形。”戒於無形見之事,言備慎深也。安不忘危,治不忘亂,是其慎無形也。法度當執守之,故以“秉法守度”解不失,言有恆也。 \par}

{\noindent\zhuan\zihao{6}\fzbyks 傳“淫過”至“為戒”。正義曰:“淫”者過度之意,故為過也。“逸”謂縱體,“樂”謂適心,縱體在於逸遊,適心在於淫恣,故以“遊逸過樂”為文。二者敗德之源,富貴所忽,故特以為戒。 \par}

{\noindent\zhuan\zihao{6}\fzbyks 傳“幹求”至“賤之”。正義曰:“幹,求”,\CJKunderwave{釋言}文。“失道求名”謂曲取人情,苟悅眾意,古人賤之。 \par}

{\noindent\zhuan\zihao{6}\fzbyks 傳“咈戾”至“戒之”。正義曰:\CJKunderwave{堯典}已訓“咈”悉戾。彼謂戾朋儕,此謂戾在下,故詳其文耳。“專欲難成,犯眾興禍”,襄十年\CJKunderwave{左傳}文。 \par}

{\noindent\shu\zihao{5}\fzkt “\CJKunderline{禹}曰”至“來王”。正義曰:\CJKunderline{禹}曰:“益言謀及世事,言人順道則吉,從逆則兇。吉凶之報,惟若影之隨形,響之應聲。”言其無不報也。益聞\CJKunderline{禹}語,驚懼而言曰:“吁!誠如此言,宜誡慎之哉!所誡者,當儆誡其心無億度之事。”謂忽然而有,當誡慎之:“無失其守法度,使行必有恆,無違常也。無遊縱於逸豫,無過耽於戲樂,當誡慎之以保己也。任用賢人勿有二心,逐去回邪勿有疑惑。所疑之謀勿成用之,如是則百種志意惟益廣也。無違越正道以求百姓之譽,無反戾百姓以從己心之慾。常行此事,無怠惰荒廢,則四夷之國皆來歸往之。”此亦所以勸勉舜也。 \par}

\CJKunderline{禹}曰:“於!帝念哉!德惟善政,政在養民。\footnote{嘆而言“念”,重其言。為政以德,則民懷之。}水火金木土谷惟修,\footnote{言養民之本在先修六府。}正德、利用、厚生惟和,\footnote{正德以率下,利用以阜財,厚生以養民,三者和,所謂善政。}九功惟叙,九叙惟歌。\footnote{言六府三事之功有次敘,皆可歌樂,乃德政之致。樂音洛。}

{\noindent\zhuan\zihao{6}\fzbyks 傳“嘆而”至“懷之”。正義曰:“於”,嘆辭。嘆而言“念”,自重其言,欲使帝念之。此史以類相從,共為篇耳。非是一時之事,不使念益言也。\CJKunderline{禹}謀以九功為重,知“重其言”者,九功之言也。 \par}

{\noindent\zhuan\zihao{6}\fzbyks 傳“言養”至“六府”。正義曰:下文帝言“六府”即此經六物也。六者民之所資,民非此不生,故言“養民之本在先修六府”也。“府”者藏財之處,六者貨財所聚,故稱“六府”。襄二十七年\CJKunderwave{左傳}云:“天生五材,民並用之。”即是水火金木土,民用此自資也。彼惟五材,此兼以谷為六府者,谷之於民尢急,谷是土之所生,故於土下言之也。此言五行,與\CJKunderwave{洪範}之次不同者,\CJKunderwave{洪範}以生數為次,此以相刻為次,便文耳。六府是民之急,先有六府乃可施教,故先言“六府”,後言“三事”也。 \par}

{\noindent\zhuan\zihao{6}\fzbyks 傳“止德”至“善政”。正義曰:“正德”者,自正其德,居上位者正己以治民,故所以率下人。“利用”者,謂在上節儉,不為縻費,以利而用,使財物殷阜,利民之用,為民興利除害,使不匱乏,故所以阜財。“阜財”謂財豐大也。“厚生”謂薄徵徭,輕賦稅,不奪農時,令民生計溫厚,衣食豐足,故所以養民也。“三者和”謂德行正、財用利、生資厚。立君所以養民,人君若能如此,則為君之道備矣。故謂“善政”,結上“德惟善政”之言。此三者之次,人君自正乃能正下,故以“正德”為先;利用然後厚生,故後言“厚生”。“厚生”謂財用足,禮讓行也。 \par}

{\noindent\zhuan\zihao{6}\fzbyks 傳“言六”至“之致”。正義曰:上六下三,即是“六府三事”,此總云“九功”,知六府三事之功為九功。“惟敘”者,即上“惟修”、“惟和”為次序。事皆有敘,民必歌樂君德,故九敘皆可歌樂,乃人君德政之致也。言下民必有歌樂,乃為善政之驗,所謂和樂興而頌聲作也。 \par}

戒之用休,董之用威,勸之以九歌,俾勿壞。”\footnote{休,美。董,督也。言善政之道,美以戒之,威以督之,歌以勸之。使政勿壞,在此三者而已。○俾,必爾反。壞,乎怪反。}帝曰:“俞!地平天成,六府三事允治,萬世永賴,時乃功。”\footnote{水土治曰“平”,五行敘曰“成”。因\CJKunderline{禹}陳九功而嘆美之,言是汝之功,明眾臣不及。}

{\noindent\zhuan\zihao{6}\fzbyks 傳“休美”至“而已”。正義曰:“休,美”,\CJKunderwave{釋詁}文。又云:“董,督,正也。”是“董”為督也。此“戒之”、“董之”、“勸之”皆謂人君自戒勸,欲使善政勿壞,在此三事而已。文七年\CJKunderwave{左傳}云,晉卻缺言於趙宣子,引此一經,乃言:“九功之德皆可歌也,謂之九歌。若吾子之德莫可歌也,其誰來之?盍使睦者歌吾子乎?”言“九功之德皆可歌”者,若水能灌溉,火能烹飪,金能斷割,木能興作,土能生殖,谷能養育,古之歌詠各述其功,猶如漢魏已來樂府之歌事,歌其功用,是舊有成辭。人君修治六府以自勸勉,使民歌詠之,三事亦然。 \par}

{\noindent\zhuan\zihao{6}\fzbyks 傳“水土”至“不及”。正義曰:\CJKunderwave{釋詁}云:“平,成也。”是“平”、“成”義同,天、地文異而分之耳。天之不成,由地之不平,故先言“地平”,本之於地以及天也。\CJKunderline{禹}平水土,故“水土治曰平”。五行之神,佐天治物,系之於天,故“五行敘曰成”。\CJKunderwave{洪範}云“\CJKunderline{鯀}堙洪水,汩陳其五行,彝倫攸斁”,\CJKunderline{禹}治洪水,“彝倫攸敘”,是\CJKunderline{禹}命五行敘也。帝因\CJKunderline{禹}陳九功而嘆美之,指言是汝之功,明眾臣不及。 \par}

{\noindent\shu\zihao{5}\fzkt “\CJKunderline{禹}曰”至“乃功”。正義曰:\CJKunderline{禹}因益言,又獻謀於帝曰“嗚呼!帝當念之哉!”言:“所謂德者惟是善於政也。政之所為,在於養民。養民者,使水火金木土谷此六事惟當修治之。正身之德,利民之用,厚民之生,此三事惟當諧和之。修和六府三事,九者皆就有功,九功惟使皆有次敘,九事次敘惟使皆可歌樂,此乃德之所致。是德能為善政之道,終當不得怠惰。但人雖為善,或寡令終,故當戒敕之念用美道,使民慕美道行善。又督察之用威罰。”言其不善當獲罪。“勸勉之以九歌之辭。但人君善政,先致九歌成辭自勸勉也。用此事,使此善政勿有敗壞之時”。勸帝使長為善也。帝答\CJKunderline{禹}曰:“汝之所言為然。汝治水土,使地平天成,六府三事信皆治理,萬代長所恃賴,是汝之功也。”歸功於\CJKunderline{禹},明群臣不及。 \par}

帝曰:“格,汝\CJKunderline{禹}。朕宅帝位三十有三載,耄期倦於勤。汝惟不怠,總朕師。”\footnote{八十、九十曰耄,百年曰期頤。言己年老,厭倦萬機,汝不懈怠於位,稱總我眾,欲使攝。格,庚白反。朕,直錦反。耄,莫報反。倦,其卷反。頤,以之反。厭,於豔反。解,於賣反。}\CJKunderline{禹}曰:“朕德罔克,民不依。\CJKunderline{皋陶}邁種德,德乃降,黎民懷之。\footnote{邁,行。種,布。降,下。懷,歸也。言己無德,民所不能依。\CJKunderline{皋陶}布行其德,下治於民,民歸服之。種,章用反。降,江巷反。}帝念哉!念茲在茲,釋茲在茲,\footnote{茲,此。釋,廢也。念此人在此功,廢此人在此罪。言不可誣。}名言茲在茲,允出茲在茲,惟帝念功。”\footnote{名言此事,必在此義;信出此心,亦在此義。言\CJKunderline{皋陶}之德以義為主,所宜念之。}

{\noindent\zhuan\zihao{6}\fzbyks 傳“八十”至“使攝”。正義曰:“八十、九十曰耄,百年曰期頤”,\CJKunderwave{曲禮}文也。如\CJKunderwave{舜典}之傳,計舜年六十三即政,至今九十五矣。年在耄、期之間,故並言之。鄭云:“期,要也。頤,養也。不知衣服食味,孝子要盡養之道而已。”孔意當然。 \par}

{\noindent\zhuan\zihao{6}\fzbyks 傳“邁行”至“服之”。正義曰:“邁,行”、“降,下”,\CJKunderwave{釋言}文。又云:“懷,來也。”來亦歸也。種物必佈於地,故為布也。 \par}

{\noindent\zhuan\zihao{6}\fzbyks 傳“茲此”至“可誣”。正義曰:“茲,此”,\CJKunderwave{釋詁}文。“釋”為舍義,故為廢也。\CJKunderline{禹}之此意,欲令帝念\CJKunderline{皋陶}。下云“惟帝念功”,“念”是念功,知“廢”是廢罪,言念、廢心依其實,不可誣罔也。 \par}

{\noindent\zhuan\zihao{6}\fzbyks 傳“名言”至“念之”。正義曰:“名言”謂己發於口,“信出”謂始發於心,皆據欲舉\CJKunderline{皋陶},必先念慮於心,而後宣之於口。先言“名言”者,己對帝讓\CJKunderline{皋陶},即是名言之事,故先言其意。然後本其心,故後言“信出”。“以義為主”者,言己讓\CJKunderline{皋陶},事非虛妄,以義為尚。 \par}

{\noindent\shu\zihao{5}\fzkt “帝曰格”至“念功”。正義曰:此舜言。將禪\CJKunderline{禹}帝,呼\CJKunderline{禹}曰:“來,汝\CJKunderline{禹}。我居帝位已三十有三載,在耄、期之間,厭倦於勤勞。汝惟在官不懈怠,可代我居帝位,總領我眾。”\CJKunderline{禹}讓之曰:“我德實無所能,民必不依就我也。”言己不堪總眾也。“\CJKunderline{皋陶}行佈於德,德乃下洽於民,眾皆歸服之,可令\CJKunderline{皋陶}攝也。我所言者,帝當念之哉!凡念愛此人,在此功勞,知有功乃用之。釋廢此人,在此罪釁,知有罪乃廢之”。言進人退人不可誣也。“名目言談此事,必在此事之義而名言之。若信實出見此心,必在此心之義而出見之”。言己名言其口,出見其心,以舉\CJKunderline{皋陶},皆在此義,不有虛妄。“帝當念錄其功以禪之”。言\CJKunderline{皋陶}堪攝位也。 \par}

帝曰:“\CJKunderline{皋陶},惟茲臣庶,罔或干予正。\footnote{或,有也。無有干我正。言順命。}汝作士,明於五刑,以弼五教,期於予治。\footnote{弼,輔。期,當也。嘆其能以刑輔教,當於治體。治音稚。當,丁浪反,又如字。}刑期於無刑,民協於中,時乃功,懋哉!”\footnote{雖或行刑,以殺止殺,終無犯者。刑期於無所刑,民皆命於大中之道,是汝之功,勉之。懋音茂。}\CJKunderline{皋陶}曰:“帝德罔愆,臨下以簡,御眾以寬。\footnote{愆,過也。善則歸君,人臣之義。愆音牽。}罰弗及嗣,賞延於世。\footnote{嗣亦世,俱謂子。延,及也。父子罪不相及,而及其賞。道德之政。}宥過無大,刑故無小。\footnote{過誤所犯,雖大必宥。不忌故犯,雖小必刑。宥音又。}罪疑惟輕,功疑惟重。\footnote{刑疑附輕,賞疑從重,忠厚之至。}與其殺不辜,寧失不經。好生之德,洽於民心,茲用不犯於有司。”\footnote{辜,罪。經,常。司,主也。\CJKunderline{皋陶}因帝勉己,遂稱帝之德,所以明民不犯上也。寧失不常之罪,不枉不辜之善,仁愛之道。辜音孤。好,呼報反。}帝曰:“俾予從欲以治,四方風動,惟乃之休。”\footnote{使我從心所欲而政以治,民動順上命,若草應風,是汝能明刑之美。}

{\noindent\zhuan\zihao{6}\fzbyks 傳“弼輔”至“治體”。正義曰:\CJKunderwave{書傳}稱“左輔右弼”,是“弼”亦輔也。期要是相當之言,故為當也。傳言“當於治體”,言\CJKunderline{皋陶}用刑,輕重得中,於治體與正相當也。 \par}

{\noindent\zhuan\zihao{6}\fzbyks 傳“雖或”至“勉之”。正義曰:言\CJKunderline{皋陶}或行刑,乃是以殺止殺。為罪必將被刑,民終無犯者。要使人無犯法,是期於無所用刑,刑無所用。此“期”為限,與前經“期”義別,而\CJKunderwave{論語}所謂“勝殘去殺”矣。“民皆合於大中”,言舉動每事得中,不犯法憲,是“合大中”即\CJKunderwave{洪範}所謂“皇極”是也。 \par}

{\noindent\zhuan\zihao{6}\fzbyks 傳“愆過”至“之義”。正義曰:“愆,過”,\CJKunderwave{釋言}文。\CJKunderwave{坊記}云:“善則稱君,過則稱己,則民作忠。”是善則稱君,人君之義也。“臨下”據其在上,“御眾”斥其治民,簡易、寬大,亦不異也。\CJKunderwave{論語}云:“居敬而行簡,以臨其民,不亦可乎?”是臨下宜以簡也。又曰:“寬則得眾。”“居上不寬,吾何以觀之哉?”是御眾宜以寬也。 \par}

{\noindent\zhuan\zihao{6}\fzbyks 傳“嗣亦”至“及也”。正義曰:“嗣”謂繼父,“世”謂後胤,故“俱謂子”也。“延”訓長,以長及物,故“延”為及也。 \par}

{\noindent\zhuan\zihao{6}\fzbyks 傳“辜罪”至“之道”。正義曰:“辜,罪”,\CJKunderwave{釋詁}文。“經,常”,“司,主”,常訓也。\CJKunderline{皋陶}因帝勉己,遂稱帝之德。所以明民不犯上者,自由帝化使然,非己力也。“不常之罪”者,謂罪大,非尋常小罪也。枉殺無罪,妄免有罪,二者皆失,必不得民心。寧妄免大罪,不枉殺無罪,以好生之心故也。大罪尚赦,小罪可知。欲極言不可枉殺不辜,寧放有罪故也,故言非常大罪以對之耳。“寧失不經”與“殺不辜”相對,故為放赦罪人,原帝之意,等殺無罪,寧放有罪。傳言帝德之善,寧失有罪,不枉殺無罪,是仁愛之道。各為文勢,故經傳倒也。“治”謂沾漬優渥,洽於民心,言潤澤多也。 \par}

{\noindent\shu\zihao{5}\fzkt “帝曰\CJKunderline{皋陶}”至“之休”。正義曰:帝以\CJKunderline{禹}讓\CJKunderline{皋陶},故述而美之。帝呼之曰:“\CJKunderline{皋陶},惟此群臣眾庶,皆無敢有干犯我正道者。由汝作士官,明曉於五刑,以輔成五教,當於我之治體。用刑期於無刑,以殺止殺,使民合於中正之道,令人每事得中,是汝之功,當勉之哉!”\CJKunderline{皋陶}以帝美己,歸美於君曰:“民合於中者,由帝德純善,無有過失,臨臣下以簡易,御眾庶以優寬。罰人不及後嗣,賞人延於來世。宥過失者無大,雖大亦有之。刑其故犯者無小,雖小必刑之。罪有疑者,雖重,從輕罪之。功有疑者,雖輕,從重賞之。與其殺不辜非罪之人,寧失不經不常之罪。以等枉殺無罪,寧妄免有罪也。由是故帝之好生之德,下洽於民心,民服帝德如此,故用是不犯於有司。”言民之無刑非已力也。帝又述之曰:“使我從心所欲而為政,以大治四方之民,從我化如風之動草,惟汝用刑之美。”言己知其有功也。 \par}

帝曰:“來,\CJKunderline{禹}。降水儆予,成允成功,惟汝賢。\footnote{水性流下,故曰下水。儆,戒也。能成聲教之信,成治水之功,言\CJKunderline{禹}最賢,重美之。儆,居領反。重,直用反。}克勤於邦,克儉於家,不自滿假,惟汝賢。\footnote{滿謂盈實。假,大也。言\CJKunderline{禹}惡衣薄食,卑其宮室,而盡力為民,執心謙沖,不自盈大。假,工雅反。盡,津忍反。為,於偽反。}汝惟不矜,天下莫與汝爭能。汝惟不伐,天下莫與汝爭功。\footnote{自賢曰矜,自功曰伐。言\CJKunderline{禹}推善讓人而不失其能,不有其勞而不失其功,所以能絕眾人。}

{\noindent\zhuan\zihao{6}\fzbyks 傳“水性”至“美之”。正義曰:“降水”,洪水也。水性下流,故曰下水。\CJKunderline{禹}以治水之事儆戒於予。\CJKunderwave{益稷}云:“予創若時,娶於塗山,辛壬癸甲,啟呱呱而泣,予弗子,惟荒度土功之事。”雖文在下篇,實是欲禪前事,故帝述而言之。\CJKunderwave{禹貢}言治水功成云:“朔南暨聲教。”故知“成允”是“成聲教之信”,“成功”是“成治水之功”也。前已言地平天成是汝功,今復說治水之事,“言\CJKunderline{禹}最賢,重美之”也。\CJKunderline{禹}實聖人,美其賢者,其性為聖,其功為賢,猶\CJKunderwave{易·繫辭}云“可久則賢人之德,可大則賢人之業”,亦是聖人之事。 \par}

{\noindent\zhuan\zihao{6}\fzbyks 傳“滿謂”至“盈大”。正義曰:“滿”以器喻,故為盈實也。“假,大”,\CJKunderwave{釋詁}文。言己無所不知,是為自滿。言己無所不能,是為自大。\CJKunderline{禹}實不自滿大,故為賢也。\CJKunderwave{論語}美\CJKunderline{禹}之功德云:“惡衣服,菲飲食,卑宮室,而盡力乎溝洫。”故傳引彼。惡衣、薄食、卑其宮室是“儉於家”,盡力為民是“勤於邦”。上言其功,此言其德,故再云“惟汝賢”。 \par}

{\noindent\zhuan\zihao{6}\fzbyks 傳“自賢”至“眾人”。正義曰:自言己賢曰矜,自言己功曰伐。\CJKunderwave{論語}云:“願無伐善。”\CJKunderwave{詩}云:“矜其車甲。”“矜”與“伐”俱是誇義,以經有“爭能”、“爭功”,故別解之耳。弗矜莫與汝爭能,即矜者矜其能也。賢、能大同小異,故“自賢”解“矜”。\CJKunderwave{老子}云:“夫惟不爭,故天下莫能與之爭。”是故不矜伐而不失其功能,此所以能絕異於眾人也。 \par}

予懋乃德,嘉乃丕績,天之歷數在汝躬,汝終陟元后。\footnote{丕,大也。歷數謂天道。元,大也;大君,天子。舜善\CJKunderline{禹}有治水之大功,言天道在汝身,汝終當升為天子。丕,普悲反。}人心惟危,道心惟微,惟精惟一,允執厥中。\footnote{危則難安,微則難明,故戒以精一,信執其中。}無稽之言勿聽,弗詢之謀勿庸。\footnote{無考無信驗,不詢專獨,終必無成,故戒勿聽用。聽,徐天定反。}

{\noindent\zhuan\zihao{6}\fzbyks 傳“丕大”至“天子”。正義曰:“丕,大”,\CJKunderwave{釋詁}文。“歷數”謂天歷運之數,帝王易姓而興,故言“歷數謂天道”。\CJKunderline{鄭玄}以“歷數在汝身謂有圖籙之名”,孔無讖緯之說,義必不然。當以大功既立,眾望歸之,即是天道在身。\CJKunderwave{釋詁}“元”訓為首,首是體之大也。\CJKunderwave{易}曰“大君有命”,是“大君”謂天子也。 \par}

{\noindent\zhuan\zihao{6}\fzbyks 傳“危則”至“其中”。正義曰:居位則治民,治民必須明道,故戒之以“人心惟危,道心惟微”。道者經也,物所從之路也。因言“人心”,遂云“道心”。人心惟萬慮之主,道心為眾道之本。立君所以安人,人心危則難安。安民必須明道,道心微則難明。將欲明道,必須精心。將欲安民,必須一意。故以戒精心一意。又當信執其中,然後可得明道以安民耳。 \par}

{\noindent\zhuan\zihao{6}\fzbyks 傳“無考”至“聽用”。正義曰:為人之君不當妄用人言,故又戒之:“無可考校之言謂無信驗,不詢於眾人之謀謂專獨用意。”言無信驗是虛妄之言,獨為謀慮是偏見之說,二者終必無成,故戒令勿聽用也。“言”謂率意為語,“謀”謂豫計前事,故互文也。 \par}

可愛非君?可畏非民?非元后何戴?後非眾罔與守邦?\footnote{民以君為命,故可愛。君失道,民叛之,故可畏。言眾戴君以自存,君恃眾以守國,相須而立。}欽哉!慎乃有位,敬修其可願,四海困窮,天祿永終。\footnote{有位,天子位。可原謂道德之美。困窮謂天民之無告者。言為天子勤此三者,則天之祿籍長終汝身。}惟口出好興戎,朕言不再。”\footnote{好謂賞善,戎謂伐惡。言口榮辱之主,慮而宣之,成於一也。出如字,徐尺遂反。好如字,徐許到反。}

{\noindent\zhuan\zihao{6}\fzbyks 傳“民以”至“而立”。正義曰:百人無主,不散則亂,故“民以君為命”。君尊,民畏之,嫌其不愛,故言“愛”也。民賤,君忽之,嫌其不畏,故言“畏”也。 \par}

{\noindent\zhuan\zihao{6}\fzbyks 傳“有位”至“汝身”。正義曰:上云“汝終陟元后”,命昇天位,知其慎汝有位,慎天子位也。道德人之可願,知“可願”者,是道德之美也。惟言“四海困窮”,不結言民之意,必謂四海之內困窮之民,令天子撫育之。故知如\CJKunderwave{王制}所云,孤獨鰥寡“此四者,天民之窮而無告者”,此是困窮者也。言為天子,當慎天位,修道德,養窮民,勤此三者,則天之祿籍長終汝身。祿謂福祿,籍謂名籍,言享大福,保大名也。 \par}

{\noindent\zhuan\zihao{6}\fzbyks 傳“好謂”至“於一”。正義曰:昭二十八年\CJKunderwave{左傳}云:“慶賞刑威曰君。”君出言有賞有刑,“出好”謂愛人而出好言,故為賞善。“興戎”謂疾人而動甲兵,故謂伐惡。\CJKunderwave{易·繫辭}曰:“言語者,君子之樞機。樞機之發,榮辱之主。”必當慮之於心,然後宣之於口,故成之於一而不可再。帝言“我命汝昇天位”者,是慮而宣之,此言故不可再。 \par}

{\noindent\shu\zihao{5}\fzkt “帝曰來”至“不再”。正義曰:帝不許\CJKunderline{禹}讓,呼之曰:“來,\CJKunderline{禹}。下流之水儆戒於我,我恐不能治之。汝能成聲教之信,能成治水之功,惟汝之賢。汝能勤勞於國。”謂盡力於溝洫。“能節儉於家”。謂薄飲食,卑宮室。“常執謙沖,不自滿溢誇大,惟汝之賢也”。又申美之:“汝惟不自矜誇,故天下莫敢與汝爭能。汝惟不自稱伐,故天下莫敢與汝爭功。”美功之大也。“我今勉汝之德,善汝大功,天之歷運之數帝位當在汝身,汝終當升此大君之位,宜代我為天子”。因戒以為君之法:“民心惟甚危險,道心惟甚幽微。危則難安,微則難明,汝當精心,惟當一意,信執其中正之道,乃得人安而道明耳。又為人君,不當妄受用人語。無可考驗之言,勿聽受之。不是詢眾之謀,勿信用之。”言“民所愛者,豈非人君乎?民以君為命,故愛君也”。言“君可畏者,豈非民乎?君失道則民叛之,故畏民也。眾非大君而何所奉戴?無君則民亂,故愛君也。君非眾人無以守國,無人則國亡,故畏民也。君民相須如此,當宜敬之哉!謹慎汝所有之位,守天子之位,勿使失也。敬修其可原之事”。謂道德之美,人所原也。“養彼四海困窮之民,使皆得存立,則天之祿籍長終汝身矣”。又告\CJKunderline{禹}:“惟口之所言,出好事,興戎兵,非善思慮無以出口,我言不可再發。”令\CJKunderline{禹}受其言也。 \par}

\CJKunderline{禹}曰:“枚卜功臣,惟吉之從。”\footnote{枚謂歷卜之而從其吉。此\CJKunderline{禹}讓之志。枚音梅。}帝曰:“\CJKunderline{禹},官佔,惟先蔽志,昆命於元龜。\footnote{帝王立卜佔之官,故曰官佔。蔽,斷。昆,後也。官佔之法,先斷人志,後命於元龜,言志定然後卜。蔽,必世反,徐甫世反。斷,丁亂反。}朕志先定,詢謀僉同,鬼神其依,龜筮協從,卜不習吉。”\footnote{習,因也。然已謀之於心,謀及卜筮,四者合從,卜不因吉,無所枚卜。僉,七潛反。}\CJKunderline{禹}拜稽首,固辭。\footnote{再辭曰固。}帝曰:“毋!惟汝諧。”\footnote{言毋,所以禁其辭。\CJKunderline{禹}有大功德,故能諧和元后之任。禁,今鴆反,又音金。}

{\noindent\zhuan\zihao{6}\fzbyks 傳“枚謂”至“之志”。正義曰:\CJKunderwave{周禮}有銜枚氏,所銜之物狀如箸。今人數物雲一枚、兩枚,則“枚”是籌之名也。“枚卜”謂人人以次歷申卜之,似老枚數然。然請卜不請筮者,舉動也。 \par}

{\noindent\zhuan\zihao{6}\fzbyks 傳“帝王”至“後卜”。正義曰:佔是卜人之佔,而云“官佔”者,帝王立卜筮之官,故曰“官佔”。\CJKunderwave{洪範}“稽疑”云:“擇建立卜筮人。”是帝王立卜筮之官。\CJKunderwave{周禮}司寇斷獄為蔽獄,是“蔽”為斷也。“昆,後”,\CJKunderwave{釋言}文。官佔之法,先斷人志,後命元龜,言志定然後卜也。\CJKunderwave{洪範}云“汝則有大疑,謀及乃心,謀乃卿士,謀及庶人”,是先斷人志;乃云“謀及卜筮”,是後命元龜。“元龜”謂大龜也。 \par}

{\noindent\zhuan\zihao{6}\fzbyks 傳“習因”至“枚上”。正義曰:\CJKunderwave{表記}云“卜筮不相襲”。鄭云:“襲,因也。”然則“習”與“襲”同。重衣謂之襲,習是後因前,故為因也。“朕志先定”言已謀之於心。“龜筮協從”是謀及卜筮。經言“詢謀僉同”,謀及卿士、庶人,謀皆同心。“鬼神其依”,即是龜筮之事,卜筮通鬼神之意,故言“鬼神其依”。“龜筮協從”,謂卜得吉,是依從也。志先定也,謀僉同也,鬼神依也,龜筮從也,四者合從,然後命汝。卜法不得因吉,無所復枚卜也。如帝此言,既謀既卜,方始命\CJKunderline{禹},仍請枚卜者,帝與朝臣私謀私卜,將欲命\CJKunderline{禹},\CJKunderline{禹}不預謀,故不在,更請卜也。 \par}

{\noindent\zhuan\zihao{6}\fzbyks 傳“言毋”至“之任”。正義曰:\CJKunderwave{說文}云:“毋,止之也。”其字從女,內有一畫,象有奸之者,禁止令勿奸也。古人言毋,猶今人言莫,是“言毋者,所以禁其辭”,令勿辭。 \par}

{\noindent\shu\zihao{5}\fzkt “\CJKunderline{禹}曰”至“汝諧”。正義曰:\CJKunderline{禹}以讓而不許,更請帝曰:“每以一枚歷卜功臣,惟吉之人,從而受之。”帝曰:“\CJKunderline{禹},卜官之佔,惟能先斷人志,後乃命其大龜。我授汝之志先以定矣,又詢於眾人,其謀又皆同美矣。我后謀及鬼神,加之卜筮,鬼神其依我矣,龜筮複合從矣。卜法不得因前之吉更復卜之,不須復卜也。”\CJKunderline{禹}猶拜而後稽首,固辭。帝曰:“毋!”毋者,禁止其辭也。“惟汝能諧和此元后之任,汝宜受之”。 \par}


正月朔旦,受命於神宗,\footnote{受舜終事之命。神宗,文祖之宗廟,言“神”尊之。○正音政,徐音徵。}率百官,若帝之初。\footnote{順舜初攝帝位故事奉行之。}

{\noindent\zhuan\zihao{6}\fzbyks 傳“受舜”至“尊之”。正義曰:\CJKunderwave{舜典}說舜之初“受終於文祖”,此言“若帝之初”,知“受命”即是“舜終事之命”也。神宗猶彼文祖,故云“文祖之宗廟”。“文祖”言祖有文德,“神宗”言神而尊之,名異而實同。神宗當舜之始祖。案\CJKunderwave{帝嚳}云,黃帝生昌意,昌意生顓頊,顓頊生窮蟬,窮蟬生敬康,敬康生句芒,句芒生蟜牛,蟜牛生瞽瞍,瞽瞍生舜。即是舜有七廟,黃帝為始祖,其顓頊與窮蟬為二祧,敬康、句芒、蟜牛、瞽瞍為親廟,則文祖為黃帝顓頊之等也。 \par}

{\noindent\zhuan\zihao{6}\fzbyks 傳“順舜”至“行之”。正義曰:“若”不得為“如”也。\CJKunderwave{舜典}巡守之事,言如初者皆言“如”,不言“若”,知此“若”為順也。順舜初攝帝位故事而盡行之,其奉行者當如\CJKunderwave{舜典}在“璿璣”以下,“班瑞群后”以上也。其巡守非率百官之事,舜尚自為陟方,\CJKunderline{禹}攝帝位,未得巡守,此是舜史所錄,以為\CJKunderwave{虞書},故言順帝之初,奉行帝之事故,自美禪之得人也。 \par}

{\noindent\shu\zihao{5}\fzkt “正月”至“之初”。正義曰:舜即政三十三年,命\CJKunderline{禹}代己,\CJKunderline{禹}辭不獲免。乃以明年正月朔旦,受終事之命於舜神靈之宗廟,總率百官。順帝之初攝故事,言與舜受禪之初,其事悉皆同也。此年舜即政三十四年,九十六也。 \par}

帝曰:“諮!\CJKunderline{禹},惟時有苗弗率,汝徂徵。”\footnote{三苗之民數幹王法。率,循。徂。往也。不循常道,言亂逆。命\CJKunderline{禹}討之。數音朔。}\CJKunderline{禹}乃會群后,誓於師曰:“濟濟有眾,咸聽朕命。\footnote{會諸侯共伐有苗。軍旅曰誓。濟濟,眾盛之貌。濟,子禮反。}蠢茲有苗,昏迷不恭,\footnote{蠢,動。昏,暗也。言其所以宜討之。蠢,春允反。}

{\noindent\zhuan\zihao{6}\fzbyks 傳“三苗”至“討之”。正義曰:\CJKunderwave{呂刑}稱苗民“作五虐之刑”,皇帝“遏絕苗民,無世在下”,謂堯初誅三苗。\CJKunderwave{舜典}云“竄三苗於三危”,謂舜居攝之時,投竄之也。\CJKunderwave{舜典}又云“庶績咸熙,分北三苗”,謂舜即位之後,往徙三苗也。今復不率命,命\CJKunderline{禹}徂徵,是三苗之民數千王誅之事,\CJKunderline{禹}率眾徵之,猶尚逆命。即三苗是諸侯之君,而謂之“民”者,以其頑愚,號之為“民”。\CJKunderwave{呂刑}云“苗民弗用靈”,是謂為“民”也。\CJKunderwave{呂刑}稱堯誅三苗云“無世在下”,而得有苗國歷代常存者,“無世在下”謂誅叛者,絕後世耳,蓋不滅其國,又立其近親紹其先祖。\CJKunderline{鯀}既殛死於羽山,\CJKunderline{禹}乃代為崇伯,三苗亦竄其身而存其國。故舜時有被宥者,復不從化,更分北流之。下傳云“三苗之國,左洞庭,右彭蠡”,其國在南方。蓋分北之時,使為南國君,今復不率帝道。“率,循”、“徂,往”皆\CJKunderwave{釋詁}文。不循帝道,言其亂逆,以其亂逆,故命\CJKunderline{禹}討之。案\CJKunderwave{舜典}皆言舜受終之後,萬事皆舜主之。舜自巡守,不稟堯命。此言“若帝之初”,其事亦應同矣。而此言命\CJKunderline{禹}徵苗,舜復陟方乃死,與舜受堯禪事不同者,以題曰\CJKunderwave{虞書},即舜史所錄,明其詳於舜,事略於堯、\CJKunderline{禹}也。 \par}

{\noindent\zhuan\zihao{6}\fzbyks 傳“會諸”至“之貌”。正義曰:軍眾曰誓”,\CJKunderwave{曲禮}文也。隱八年\CJKunderwave{穀梁傳}曰:“誥誓不及五帝,盟詛不及三土,交質不及二伯。”“二伯”謂齊桓公、晉文公也。“不及者”,言於時未有也。據此文,五帝之世有誓。\CJKunderwave{周禮}立司盟之官,三王之世有盟也。\CJKunderwave{左傳}雲平王與鄭交質,二伯之前有質也。\CJKunderwave{穀梁傳}漢初始作,不見經文,妄言之耳。美軍眾而言“濟濟”,知是“眾盛之貌”。 \par}

{\noindent\zhuan\zihao{6}\fzbyks 傳“蠢動”至“討之”。正義曰:蠢,動”,\CJKunderwave{釋詁}文。\CJKunderwave{釋訓}云:“蠢,不遜也。”郭璞云:“蠢動為惡,不謙遜也。”日入為“昏”,是為暗也。動為惡而暗於事,言其所以宜討之。 \par}

侮慢自賢,反道敗德,\footnote{狎侮先王,輕慢典教,反正道,敗德義。侮,亡甫反。慢,亡諫反。}君子在野,小人在位,\footnote{廢仁賢,任奸佞。}民棄不保,天降之咎,\footnote{言民叛,天災之。咎,其九反。}肆予以爾眾士,奉辭罰罪。\footnote{肆,故也。辭謂“不恭”,罪謂“侮慢”以下事。}爾尚一乃心力,其克有勳。”\footnote{尚,庶幾。一汝心力,以從我命。}

{\noindent\zhuan\zihao{6}\fzbyks 傳“狎侮”至“德義”。正義曰:“侮”謂輕人身,“慢”謂忽言語,故為“狎侮先王,輕慢典教”。“侮”、“慢”義同,因有二字而分釋之。\CJKunderwave{論語}云“狎大人,侮聖人之言”,則“狎”、“侮”為異。\CJKunderwave{旅獒}云“狎侮君子”,則“狎”、“侮”意亦同。\CJKunderline{鄭玄}云:“狎,慣忽也。”慣見而忽之,是“侮”之義。傳取“狎”、“侮”連言之。慢先王典,教自謂己賢,不知先王訓教。“道”者,物所由之路,“德”謂自得於心。反正道,從邪徑,敗德義,毀正行也。 \par}

{\noindent\zhuan\zihao{6}\fzbyks 傳“廢仁賢,任奸佞”。正義曰:雖則下愚之君,皆雲好賢疾佞,非知賢而廢之,知佞而任之。但愚人所好,必同於民,賢求其心,佞從其欲,以賢為惡,謂佞為善,故仁賢見廢,奸佞被任,此則“昏迷”之狀也。 \par}

{\noindent\zhuan\zihao{6}\fzbyks 傳“肆故”至“下事”。正義曰:“肆,故”,\CJKunderwave{釋詁}文。所奉之辭即所伐之罪,但天子責其不恭,數其身罪,因其文異而分之。 \par}

{\noindent\zhuan\zihao{6}\fzbyks 傳“尚庶”至“我命”。正義曰:\CJKunderwave{釋言}云:“庶幾,尚也。”反以相解,故“尚”為庶幾。 \par}

{\noindent\shu\zihao{5}\fzkt “帝曰諮”至“有勳”。正義曰:史言\CJKunderline{禹}雖攝位,帝尊如故,時有苗國不順,帝曰:“諮嗟!汝\CJKunderline{禹},惟時有苗之國不循帝道,汝往徵之。”\CJKunderline{禹}得帝命,乃會群臣諸侯,告誓於眾曰:“濟濟美盛之有眾,皆聽從我命。今蠢蠢然動而不遜者,是此有苗之君。昏暗迷惑,不恭敬王命。侮慢眾常,自以為賢。反戾正道,敗壞德義。君子在野,小人在位。由此民棄叛之,不保其有眾,上天降之殃咎。故我以爾眾士,奉此譴責之辭,伐彼有罪之國。汝等庶幾同心盡力,以從我命,其必能有大功勳,不可懈惰。” \par}

三旬,苗民逆命。\footnote{旬,十日也。以師臨之,一月不服,責舜不先有文誥之命、威讓之辭,而便憚之以威,脅之以兵,所以生辭。誥,古報反。憚,徒旦反,一音丹末反。脅,許業反。}\CJKunderline{益}贊於\CJKunderline{禹}曰:“惟德動天,無遠弗\xpinyin*{屆}。\footnote{贊,佐。屆,至也。益以此義佐\CJKunderline{禹},欲其修德致遠。屆音戒。}滿招損,謙受益,時乃天道。\footnote{自滿者人損之,自謙者人益之,是天之常道。}


{\noindent\zhuan\zihao{6}\fzbyks 傳“旬十”至“生辭”。正義曰:\CJKunderwave{堯典}云“三百有六旬”,是知“旬,十日”也。以師臨之,一月不服者,責舜不先有文告之命、威讓之辭,而便憚之以威,脅之以兵,所以有苗得生辭也。傳知然者,昭十三年\CJKunderwave{左傳}論征伐之事云:“告之以文辭,董之以武師。”是用兵者先告,不服然後伐之。今經無先告之文,而有逆命之事,故知責舜不先有文告之命,而即脅之以兵。其“文告之命、威讓之辭”,\CJKunderwave{國語}亦有其事。夫以大舜足達用兵之道,而不為文告之命,使之得生辭者,有苗數幹王誅,逆者難以言服,故憚之以威武,任其生辭。待其有辭,為之振旅,彼若師退而服,我復更有何求?為退而又不降,復往必無辭說。不恭而徵之,有辭而舍之,正是柔服之道也。若先告以辭,未必即得從命。不從而後行師,必將大加殺戮。不以文誥,感德自來,固是大聖之遠謀也。 \par}

{\noindent\zhuan\zihao{6}\fzbyks 傳“贊佐”至“致遠”。正義曰:\CJKunderwave{禮}有贊佐,是助祭之人,故“贊”為佐也。“屆,至也”,\CJKunderwave{釋詁}文。經云:“惟德動天。”天遠而難動,德能動遠。又言:“無遠不屆。”乃據人言德動遠人,無不至也。益以此義佐\CJKunderline{禹},欲修德致遠,使有苗自來也。德之動天,經傳多矣。\CJKunderwave{禮運}云,聖人順民,“天不愛其道”,地不愛其寶,故天降膏露,地出醴泉”。如此之類,皆德動之也。 \par}

{\noindent\zhuan\zihao{6}\fzbyks 傳“自滿”至“常道”。正義曰:自以為滿,人必損之;自謙受物,人必益之。\CJKunderwave{易寶·謙卦}彖曰:“天道虧盈而益謙,地道變盈而流謙,鬼神害盈而福謙,人道惡盈而好謙。”是滿招損,謙受益為天道之常也。益言此者,欲令\CJKunderline{禹}修德息師,持謙以待有苗。 \par}

帝初於歷山,往于田,日號泣於旻天、於父母,\footnote{仁覆愍下謂之旻天。言舜初耕於歷山之時,為父母所疾,日號泣於旻天及父母,克己自責,不責於人。田,本或作畋。號,戶高反。旻,武巾反。}負罪引慝,祗載見瞽瞍,夔夔齋慄,瞽亦允若。\footnote{慝,惡。載,事也。夔夔,悚懼之貌。言舜負罪引惡,敬以事見於父,悚懼齋莊,父亦信順之。言能以至誠感頑父。慝,他則反。見,賢遍反。瞽音古。瞍,素後反。夔,求龜反。齋音側皆反。}至諴感神,矧茲有苗。”\footnote{諴,和。矧,況也。至和感神,況有苗乎!言易感。諴音咸。矧,失忍反。易,以豉反。}

{\noindent\zhuan\zihao{6}\fzbyks 傳“仁覆”至“責於人”。正義曰:“仁覆愍下謂之旻天”,\CJKunderwave{詩}毛傳文也。旻,愍也。求天愍己,故呼曰“旻天”。\CJKunderwave{書傳}言:“舜耕於歷山。”\CJKunderline{鄭玄}云:“歷山在河東。”是耕於歷山之時,為父母所疾,故往于田,日號泣於旻天。何為其然也!\CJKunderwave{孟子}曰:“怨慕也。長息問於公明高曰:‘舜往于田,則予既聞命矣,號泣於旻天及父母,即吾不知矣。’公明高曰:‘非爾所知也。’我竭力耕田,供為子職而已,父母不愛我,何哉?大孝終身慕父母,五十而慕者,予於大舜見之矣。”言舜之號泣怨慕者,克己自責,不責於人也。 \par}

{\noindent\zhuan\zihao{6}\fzbyks 傳“慝惡”至“頑父”。正義曰:“慝”之為惡,常訓耳。\CJKunderwave{舜典}已訓“載”為事,以非常訓,故詳其文。“夔夔”與“齋慄”共文,故為“悚懼之貌”。自負其罪,引惡歸己,事瞽同耳,丁寧深言之。“敬以事見於父”者,謂恭敬,自因事務須見父,恭敬以見。夔夔然悚懼齊慄,是見時之貌。“父亦信順之”者,謂當以事見之時,順帝意不悖怒也。言“能以至誠感頑父”者,言感使當時暫以順耳,不能使每事信順,變為善人。故\CJKunderwave{孟子}說舜既被堯徵用,堯妻之以二女,瞽瞍猶與象欲謀殺舜而分其財物,是下愚之性,終不可改。但舜善養之,使不至於奸惡而已。 \par}

{\noindent\zhuan\zihao{6}\fzbyks 傳“諴和”至“易感”。正義曰:“諴”亦咸也,咸訓為皆,皆能相從,亦和之義也。“矧,況”,\CJKunderwave{釋言}文。上言德能動天,次言帝能感瞽。天以玄遠難感,瞽以頑愚難感,言苗民近於天而智於瞽,故言感天感瞽以況之。天是神也,覆動上天,言至和尚能感天神,而況於有苗乎!言有苗易感。神覆動天而不覆言瞽者,以瞽雖愚,猶是人類,天神事與人隔,感天難於感瞽,故舉難者以見之。其實天與瞽俱言難感,以況有苗易於彼二者。 \par}

禹拜昌言曰:“俞!”班師振旅。\footnote{昌,當也。以益言為當,故拜受而然之,遂還師。兵入曰振旅。言整眾。當,丁浪反,下同。還,經典皆音旋。}帝乃誕敷文德,\footnote{遠人不服,大布文德以來之。誕音但。}舞幹羽於兩階,\footnote{幹,楯。羽,翳也。皆舞者所執。修闡文教,舞文舞於賓主階間,抑武事。階徐音皆。楯,食允反。翳,於計反。闡,尺善反。}七旬,有苗格。\footnote{討而不服,不討自來,明御之者必有道。三苗之國,左洞庭,右彭蠡,在荒服之例,去京師二千五百里也。洞,徒弄反。蠡音禮。}

{\noindent\zhuan\zihao{6}\fzbyks 傳“昌當”至“整眾”。正義曰:“昌,當也”,\CJKunderwave{釋詁}文。\CJKunderline{禹}以益言為當,拜舜而已即還。還不請者,\CJKunderwave{春秋}襄十九年“晉士匄帥師侵齊,聞齊侯卒,乃還”,\CJKunderwave{公羊傳}曰“大夫以君命出,進退在大夫”,是言進退由將,不須請也。或可當時請帝乃還,文不具耳。“兵入曰振旅”,\CJKunderwave{釋天}文。與\CJKunderwave{春秋}二傳皆有此文。振,整也。言整眾而還。 \par}

{\noindent\zhuan\zihao{6}\fzbyks 傳“遠人”至“來之”。正義曰:“遠人不服,文德以來之”,\CJKunderwave{論語}文也。益贊於\CJKunderline{禹},使修德,而帝自誕敷者,言君臣同心.“大布”者,多設文德之教,君臣共行之也。 \par}

{\noindent\zhuan\zihao{6}\fzbyks 傳“幹楯”至“武事”。正義曰:\CJKunderwave{釋言}云:“幹,扞也。”孫炎曰:“幹楯,自蔽扞也。”以楯為人扞。通以“幹”為楯名,故“幹”為楯。\CJKunderwave{釋言}又云:“纛,翳也。“郭璞云:“舞者持以自蔽翳也。”故\CJKunderwave{明堂位}云,朱干玉鏚以舞大武。戚,斧也,是武舞執斧執楯。\CJKunderwave{詩}云:“左手執籥,右手秉翟。”是文舞執籥,故幹羽皆舞者所執。修闡文教,不復征伐,故舞文德之舞於賓主階間,言帝抑武事也。經云“舞幹羽”,即亦舞武也。傳惟言舞文者,以據器言之,則有武有文,俱用以為舞,而不用於敵,故教為文也。 \par}

{\noindent\zhuan\zihao{6}\fzbyks 傳“討而”至“百里”。正義曰:“御之必有道”者,不恭而往徵,得辭而振旅,而御之以道。\CJKunderwave{史記}吳起對魏武侯云:“昔三苗氏左洞庭,右彭蠡,德義不修,而\CJKunderline{禹}滅之。”此言來服,則是不滅。吳起言滅者,以武侯恃險,言滅以懼之。辯士之說,不必皆依實也。知“在荒服之例”者,以其地驗之為然。\CJKunderwave{禹貢}五服,甸、侯、綏、要、荒。荒最在外,王畿面五百里,其外四服又每服五百里,是去京師為二千五百里。 \par}

{\noindent\shu\zihao{5}\fzkt “三旬”至“苗格”。正義曰:\CJKunderline{禹}既誓於眾,而以師臨苗。經三旬,苗民逆帝命,不肯服罪。益乃進謀以佐於\CJKunderline{禹}曰:“惟是有德,能動上天。苟能修德,無有遠而不至。”因言行德之事:“自滿者招其損,謙虛者受其益,是乃天之常道。”欲\CJKunderline{禹}修德,謙虛以來苗。既說其理,又言其驗:“帝乃初耕於歷山之時,為父母所疾。往至於田,日號泣於旻天。於父母乃自負其罪,自引其惡,恭敬以事見父瞽瞍,夔夔然悚懼,齋莊戰慄,不敢言己無罪。舜謙如此,雖瞽瞍之頑愚,亦能信順。帝至和之德尚能感於冥神,況此有苗乎!”言其苗易感於瞽瞍。\CJKunderline{禹}拜益受之當言,曰:“然。”然益語也。遂還師整眾而歸。\CJKunderline{帝舜}乃大布文德,舞幹羽於兩階之間,七旬而有苗自服來至。言主聖臣賢,御之有道也。 \par}

\section{皋陶謨第四(皋陶謨上)}


皋陶謨\footnote{謨,謀也。\CJKunderline{皋陶}為\CJKunderline{帝舜}謀。為,於偽反。}

{\noindent\zhuan\zihao{6}\fzbyks 傳“謨謀”至“舜謀”。正義曰:孔以此篇惟與\CJKunderline{禹}言,嫌其不對\CJKunderline{帝舜},故言“為\CJKunderline{帝舜}謀”。將言“為\CJKunderline{帝舜}謀”故又訓“謨”為謀,以詳其文。 \par}

曰若稽古,\CJKunderline{皋陶}\footnote{亦順考古道以言之。夫典、謨,聖帝所以立治之本,皆師法古道以成不易之則。夫音扶。治,直吏反,下同。}曰:“允迪厥德,謨明弼諧。”\footnote{迪,蹈。厥,其也。其,古人也。言人君當信蹈行古人之德,謀廣聰明以輔諧其政。蹈,徒報反。}\CJKunderline{禹}曰:“俞,如何?”\footnote{然其言,問所以行。}

{\noindent\zhuan\zihao{6}\fzbyks 傳“亦順”至“之則”。正義曰:二謨其目正同,故云“亦順考古道以言”也。堯舜考古以行,謂之為“典”,\CJKunderline{大禹}\CJKunderline{皋陶}考古以言謂之為“謨”。“典”、“謨”之文不同,其目皆雲考古,故傳明言其意:“夫典、謨,聖帝所以立治之本。”雖言行有異,皆是考法古道以成不易之則,故史皆以“稽古”為端目。但君則行之,臣則言之,以尊卑不同,故“典”、“謨”名異。\CJKunderline{禹}亦為君而云“謨者,\CJKunderline{禹}在舜時未為君也。顧氏亦同此解。\CJKunderline{皋陶}德劣於\CJKunderline{禹},皆是考古以言,故得同其題目。但\CJKunderline{禹}能“敷於四海,祇承於帝”,\CJKunderline{皋陶}不能然,故此下更無別辭耳。 \par}

{\noindent\zhuan\zihao{6}\fzbyks 傳“迪蹈”至“其政”。正義曰:\CJKunderwave{釋詁}云:“迪,道也。”聲借為導,導音與蹈同,故“迪”又為蹈也。其德即其上“稽古”,故曰“其,古人也”。而臣為君謀,故云“言人君當信蹈行古人之德”,謂蹈履依行之也。“謀廣聰明”,“聰明”者自是己性,又當受納人言,使多所聞見,以博大此聰明,以輔弼和諧其政。經惟言“明”,傳亦有“聰”者,以耳目同是所用,故以“聰明”言之。此“曰”上不言“\CJKunderline{皋陶}”,猶\CJKunderline{大禹}為謀,“曰”上不言“\CJKunderline{禹}”。\CJKunderline{鄭玄}云:“以\CJKunderline{皋陶}下屬為句。”則“稽古”之下無人名,與上三篇不類甚矣。 \par}

\CJKunderline{皋陶}曰:“都!慎厥身,修思永。\footnote{嘆美之重也。慎修其身,思為長久之道。“身修”絕句。}惇敘九族,庶明勵翼,邇可遠在茲。”\footnote{言慎修其身,厚次敘九族,則眾庶皆明其教,而自勉勵翼戴上命,近可推而遠者,在此道。惇,\CJKunderwave{切韻}都昆反。}\CJKunderline{禹}拜昌言曰:“俞。”\footnote{以\CJKunderline{皋陶}言為當,故拜受而然之。當,丁浪反,下同。}

{\noindent\zhuan\zihao{6}\fzbyks 傳“嘆美”至“之道”。正義曰:案傳之言以“修”為上讀,顧氏亦同也。 \par}

{\noindent\zhuan\zihao{6}\fzbyks 傳“言慎”至“此道”。正義曰:自身以外,九族為近,故慎修其身,又厚次敘九族,猶堯之為政,先以親九族也。人君既能如此,則眾庶皆明其教,而各自勉勵翼戴上命。昭九年\CJKunderwave{左傳}說晉叔向言“翼戴天子”,故以為“翼戴上命”,言如鳥之羽翼而奉戴之。王者率己以化物,親親以及遠,故從近可推而至於遠者,在修己身、親九族之道。\CJKunderline{王肅}云:“以眾賢明為砥礪,為羽翼。”鄭云:“厲,作也,以眾賢明作輔翼之臣。”與孔不同。 \par}

{\noindent\shu\zihao{5}\fzkt “曰若”至“曰俞”。正義曰:史將言\CJKunderline{皋陶}之能謀,故為題目之辭曰,能順而考案古道而言之者,是\CJKunderline{皋陶}也。其為帝謀曰:“為人君者當信實蹈行古人之德,而謀廣其聰明之性,以輔諧己之政事,則善矣。”\CJKunderline{禹}曰:“然。”然其謀是也。“此當如何行之?”\CJKunderline{皋陶}曰:“嗚呼!重其事而嘆美之。“行上謀者,當謹慎其己身,而修治人之事,思為久長之道。又厚次敘九族之親而不遺棄,則眾人皆明曉上意,而各自勉勵翼戴上命,行之於近,而可推而至遠者,在此道也。”\CJKunderline{禹}乃拜受其當理之言,曰:“然。”美其言而拜受之。 \par}

\CJKunderline{皋陶}曰:“都!在知人,在安民。”\footnote{嘆修身親親之道在知人所信任,在能安民。}\CJKunderline{禹}曰:“吁!咸若時,惟帝其難之。\footnote{言\CJKunderline{帝堯}亦以知人安民為難,故曰:“吁!”}知人則哲,能官人。安民則惠,黎民懷之。\footnote{哲,智也。無所不知,故能官人。惠,愛也。愛則民歸之。}能哲而惠,何憂乎\CJKunderline{驩兜}?\footnote{佞人亂德,堯憂其敗政,故流放之。}何遷乎\CJKunderline{有苗}?何畏乎巧言令色孔壬?”\footnote{孔,甚也。巧言,靜言庸違。令色,象恭滔天。\CJKunderline{禹}言有苗、\CJKunderline{驩兜}之徒甚佞如此,堯畏其亂政,故遷放之。}


{\noindent\zhuan\zihao{6}\fzbyks 傳“哲智”至“歸之”。正義曰:“哲,智”,\CJKunderwave{釋言}文。舍人曰:“哲,大智也。”無所不知,知人之善惡,是能官人。“惠,愛”,\CJKunderwave{釋詁}文。君愛民則民歸之。 \par}

{\noindent\zhuan\zihao{6}\fzbyks 傳“孔甚”至“放之”。正義曰:“孔,甚”,\CJKunderwave{釋詁}文。上句既言“\CJKunderline{驩兜}”、“有苗”,則此“巧言令色”\CJKunderline{共工}之行也,故以\CJKunderwave{堯典}\CJKunderline{共工}之事解之:“巧言,靜言庸違”也,“令色,象恭滔天”也。“孔壬”之文在三人之下,總上三人皆甚佞也。“苗”言其名,“巧言令色”言其行,令其文首尾互相見,故傳通言之:“\CJKunderline{禹}言有苗、\CJKunderline{驩兜}之徒甚佞如此,堯畏其亂政,故遷放之。”傳不言\CJKunderline{共工},故云“之徒”以包之。“遷”與“憂”、“畏”亦互相承言,畏之而憂,乃遷之也。四凶惟言三者,\CJKunderline{馬融}云:“\CJKunderline{禹}為父隱,故不言\CJKunderline{鯀}也。” \par}

{\noindent\shu\zihao{5}\fzkt “\CJKunderline{皋陶}曰都在”至“孔壬”。正義曰:\CJKunderline{皋陶}以\CJKunderline{禹}然其言,更述修身親親之道,嘆而言曰:“人君行此道者,在於知人善惡,擇善而信任之;在於能安下民,為政以安定之也。”\CJKunderline{禹}聞此言,乃驚而言曰:“吁!人君皆如是。能知人,能安民,惟\CJKunderline{帝堯}猶其難之,況餘人乎?知人善惡,則為大智。能用官,得其人矣。能安下民,則為惠政,眾民皆歸之矣。此甚不易也,若\CJKunderline{帝堯}能智而惠,則當朝無奸佞,何憂懼於\CJKunderline{驩兜}之佞而流放之?何須遷徙於有苗之君?何所畏懼於彼巧言令色為甚佞之人?”三兇見惡,\CJKunderline{帝堯}方始去之,是知人之難。 \par}

\CJKunderline{皋陶}曰:“都!亦行有九德。\footnote{言人性行有九德,以考察真偽則可知。○行,下孟反,注“性行”、“行正直”之“行”同。}亦言其人有德,乃言曰:‘載采采。’”\footnote{載,行。採,事也。稱其人有德,必言其所行某事某事以為驗。}


{\noindent\zhuan\zihao{6}\fzbyks 傳“言人”至“可知”。正義曰:“言人性行有九德”,下文所云是也。如此九者考察其真偽,則人之善惡皆可知矣。然則\CJKunderline{皋陶}之賢不及\CJKunderline{帝堯}遠矣,\CJKunderline{皋陶}知有此術,\CJKunderline{帝堯}無容不知;而有四凶在朝,\CJKunderline{禹}言帝難之者,堯朝之有四凶,晦跡以顯舜爾。\CJKunderline{禹}言惟帝難之,說彼甚佞,因其成敗以示教法,欲開\CJKunderline{皋陶}之志,故舉大事以為戒;非是此實甚佞,堯不能知也。顧氏亦云:“堯實不以此為難。今雲難者,俯同流俗之稱也。” \par}

{\noindent\zhuan\zihao{6}\fzbyks 傳“載行”至“為驗”。正義曰:“載”者,運行之義,故為行也。此謂薦舉人者稱其人有德,欲使在上用之,必須言其所行之事,雲見此人常行其某事某事,由此所行之事以為有德之驗。\CJKunderwave{論語}云:“如有所譽者,其有所試矣。”是言試之於事,乃可知其德。 \par}

{\noindent\shu\zihao{5}\fzkt “\CJKunderline{皋陶}”至“采采”。正義曰:\CJKunderline{禹}既言知人為難,\CJKunderline{皋陶}又言行之有術,故言曰:“嗚呼!人性雖則難知,亦當考察其所行有九種之德。人慾稱薦人者,不直言可用而已,亦當言其人有德。問其德之狀,乃言曰其德之所行某事某事。以所行之事為九德之驗,如此則可知也。” \par}

\CJKunderline{禹}曰:“何?”\footnote{問九德品例。}\CJKunderline{皋陶}曰:“寬而栗,\footnote{性寬弘而能莊栗。}柔而立,\footnote{和柔而能立事。}愿而恭,\footnote{愨愿而恭恪。願音願。愨,\CJKunderwave{切韻}苦角反。恪,苦各反。}亂而敬,\footnote{亂,治也。有治而能謹敬。}擾而毅,\footnote{擾,順也。致果為毅。擾,而小反,徐音饒。毅,五既反。}直而溫,\footnote{行正直而氣溫和。}簡而廉,\footnote{性簡大而有廉隅。}剛而塞,\footnote{剛斷而實塞。斷,丁亂反。}彊而義。\footnote{無所屈撓,動必合義。撓,女孝反。}彰厥有常,吉哉!\footnote{彰,明。吉,善也。明九德之常,以擇人而官之,則政之善。}

{\noindent\zhuan\zihao{6}\fzbyks 傳“性寬”至“莊栗”。正義曰:此九德之文,\CJKunderwave{舜典}云“寬而栗,直而溫”,與此正同。彼云“剛而無虐,簡而無傲”,與此小異。彼言“剛失入虐”,此言“剛斷而能實塞”,“實塞”亦是不為虐。彼言“簡失入傲”,此言“簡大而有廉隅”,“廉隅”亦是不為傲也。九德皆人性也。\CJKunderline{鄭玄}云:“凡人之性有異,有其上者,不必有下;有其下者,不必有上。上下相協,乃成其德。”是言上下以相對,各令以相對兼而有之,乃為一德。此二者雖是本性,亦可以長短自矯。寬弘者失於緩慢,故性寬弘而能矜莊嚴栗,乃成一德。九者皆然也。 \par}

{\noindent\zhuan\zihao{6}\fzbyks 傳“愨愿而恭恪”。正義曰:“願”者愨謹良善之名。謹願者失於遲鈍,貌或不恭,故愨願而能恭恪乃為德。 \par}

{\noindent\zhuan\zihao{6}\fzbyks 傳“亂治”至“謹敬”。正義曰:“亂,治”,\CJKunderwave{釋詁}文。有能治者,謂才高於人也,堪撥煩理劇者也。負才輕物,人之常性,故有治而能謹敬乃為德也。“願”言“恭”,“治”云“敬”者,恭在貌,敬在心;願者遲鈍,外失於儀,故言“恭”以表貌;治者輕物,內失於心,故稱“敬”以顯情。“恭”與“敬”其事亦通,“願”其貌恭而心敬也。 \par}

{\noindent\zhuan\zihao{6}\fzbyks 傳“擾順”至“為毅”。正義曰:\CJKunderwave{周禮·大宰}云:“以擾萬民。”\CJKunderline{鄭玄}云:“擾猶馴也。”\CJKunderwave{司徒}云:“安擾邦國。”鄭云:“擾亦安也。”“擾”是安馴之義,故為順也。“致果為毅”,宣二年\CJKunderwave{左傳}文。彼文以“殺敵為果,致果為毅”,謂能致果敢殺敵之心,是為強毅也。和順者失於不斷,故順而能決乃為德也。 \par}

{\noindent\zhuan\zihao{6}\fzbyks 傳“性簡”至“廉隅”。正義曰:“簡”者,寬大率略之名。志遠者遺近,務大者輕細,弘大者失於不謹,細行者不修廉隅,故簡大而有廉隅乃為德也。 \par}

{\noindent\zhuan\zihao{6}\fzbyks 傳“剛斷”至“實塞”。正義曰:“塞”訓實也。剛而能斷失於空疏,必性剛正而內充實乃為德也。 \par}

{\noindent\zhuan\zihao{6}\fzbyks 傳“無所”至“合義”。正義曰:強直自立,無所屈撓,或任情違理,失於事宜,動合道義乃為德也。鄭注\CJKunderwave{論語}云:“剛謂強,志不屈撓。”即“剛”、“強”義同。此“剛”、“強”異者,“剛”是性也,“強”是志也。當官而行,無所避忌,剛也。執己所是,不為眾撓,強也。“剛”、“強”相近,鄭連言之。“寬”謂度量寬弘,“柔”謂性行和柔,“擾”謂事理擾順,三者相類,即\CJKunderwave{洪範}云“柔克”也。“願”謂容貌恭正,“亂”謂剛柔治理,“直”謂身行正直,三者相類,即\CJKunderwave{洪範}云“正直”也。“簡”謂器量凝簡,“剛”謂事理剛斷,“強”謂性行堅強,三者相類,即\CJKunderwave{洪範}云“剛克”也。而九德之次,從“柔”而至“剛”也,惟“擾而毅”在“願”、“亂”之下耳。其\CJKunderwave{洪範}三德,先人事而後天地,與此不同。 \par}

{\noindent\zhuan\zihao{6}\fzbyks 傳“彰明”至“之善”。正義曰:“彰,明”、“吉,善”,常訓也。此句言用人之義。所言九德,謂彼人常能然者。若暫能為之,未成為德。故人君取士,必明其九德之常,知其人常能行之,然後以此九者之法擇人而官之,則為政之善也。“明”謂人君明知之。\CJKunderline{王肅}云:“明其有常則善也,言有德當有恆也。”其意亦言彼能有常,人君能明之也。鄭云:“人能明其德,所行使有常,則成善人矣。”其意謂彼人自明之,與孔異也。 \par}

{\noindent\shu\zihao{5}\fzkt “\CJKunderline{禹}曰”至“吉哉”。正義曰:\CJKunderline{皋陶}既言其九德,\CJKunderline{禹}乃問其品例曰:“何謂也?”\CJKunderline{皋陶}曰:“人性有寬弘而能莊栗也,和柔而能立事也,愨願而能恭恪也,治理而能謹敬也,和順而能果毅也,正直而能溫和也,簡大而有廉隅也,剛斷而能實塞也,強勁而合道義也。人性不同,有此九德。人君明其九德所有之常,以此擇人而官之,則為政之善哉!” \par}

日宣三德,夙夜浚明有家。\footnote{三德,九德之中有其三。宣,布。夙,早。浚,須也。卿大夫稱家。言能日日布行三德,早夜思之,須明行之,可以為卿大夫。浚,息俊反,馬云:“大也。”}日嚴祗敬六德,亮采有邦。\footnote{有國,諸侯。日日嚴敬其身,敬行六德,以信治政事,則可以為諸侯。嚴如字,馬、徐魚檢反。}


{\noindent\zhuan\zihao{6}\fzbyks 傳“三德”至“大夫”。正義曰:此文承“九德”之下,故知“三德”是九德之內課有其三也。\CJKunderwave{周語}云:“宣佈哲人之令德。”“宣”以布義,故為布也。“夙,早”,\CJKunderwave{釋詁}文。又云:“須,待也。”此經之意,謂夜思之,明旦行之,“須”為待之意,故“浚”為須也。大夫受采邑,賜氏族,立宗廟,世不絕祀,故稱“家”。位不虛受,非賢臣不可,言能日日布行三德,早夜思之,待明行之,如此念德不懈怠者,乃可以為大夫也。以士卑,故言不及也。計有一德二德,即可以為士也。鄭以“‘三德’、‘六德’,皆‘亂而敬’以下之文”,經無此意也。 \par}

{\noindent\zhuan\zihao{6}\fzbyks 傳“有國”至“諸侯”。正義曰:天子分地建國,諸侯專為己有,故“有國”謂諸侯也。“祗”亦為敬,“敬”有二文,上謂敬身,下謂敬德,“嚴”則敬之狀也。故言“日日嚴敬其身,敬行六德,以信治政事,則可以為諸侯”也。諸侯大夫皆言“日日”者,言人之行德,不可暫時舍也。臣當行君之令,故早夜思之。君是出令者,故言敬身行德。此文以小至大,總以天子之事,故先大夫而後諸侯。 \par}

翕受敷施,九德咸事,俊乂在官。\footnote{翕,和也。能合受三六之德而用之,以佈施政教,使九德之人皆用事。謂天子如此,則俊德治能之士並在官。翕,許及反。俊乂,馬曰:“千人曰俊,百人曰乂。”}百僚師師,百工惟時,\footnote{僚、工皆官也。師師,相師法。百官皆是,言政無非。僚本又作寮。}撫於五辰,庶績其凝。”\footnote{凝,成也。言百官皆撫順五行之時,眾功皆成。撫,方武反。凝,魚陵反,馬云:“定也。”}

{\noindent\zhuan\zihao{6}\fzbyks 傳“翕合”至“在官”。正義曰:“翕,合”,\CJKunderwave{釋詁}文。以文承“三德”、“六德”之下,故言合受三六之德而用之。以此人為官,令其佈施政教,使此九德之人皆居官用事。謂天子各任其所能。大夫所行三德,或在諸侯六德之內,但並此三六之德,既充九數,故言九德皆用事,謂用為大夫,用為諸侯,使之治民事也。大夫諸侯當身自行之,故言“日宣”、“日嚴”。天子當任人使行之,故言“合受而用之”。其實天子亦備九德,故能任用三德、六德也,則俊德治能之士並在官矣。“乂”訓為治,故云“治能”。馬、王、鄭皆云:“才德過千人為俊,百人為乂。” \par}

{\noindent\zhuan\zihao{6}\fzbyks 傳“僚工”至“無非”。正義曰:“僚,官”,\CJKunderwave{釋詁}文。“工,官”,常訓也。“師師”謂相師法也。 \par}

{\noindent\zhuan\zihao{6}\fzbyks 傳“凝成”至“皆成”。正義曰:\CJKunderline{鄭玄}亦云:“凝,成也。”\CJKunderline{王肅}云:“凝猶定也。”皆以意訓耳。文承“百工”之下,“撫於五辰”還是百工撫之,故云“百官皆撫順五行之時,則眾功皆成”也。“五行之時”即四時也。\CJKunderwave{禮運}曰“播五行於四時”,土寄王四季,故為“五行之時”也。所撫順者,即\CJKunderwave{堯典}“敬授民時”,“平秩東作”之類是也。 \par}

{\noindent\shu\zihao{5}\fzkt “日宣”至“其凝”。正義曰:\CJKunderline{皋陶}既陳人有九德,宜擇而官之,此又言官之所宜:“若人能日日宣佈三德,早夜思念而須明行之,此人可以為卿大夫,使有家也。若日日嚴敬其身,又能敬行六德,信能治理其事,此人可以為諸侯,使有國也。然後撫以天子之任,合受有家有國三六之德而用之,佈施政教,使九德之人皆得用事,事各盡其能,無所遺棄,則天下俊德治能之士並在官矣。皆隨賢才任職,百官各師其師,轉相教誨,則百官惟皆是矣,無有非者。以此撫順五行之時,以化隘天下之民,則眾功其皆成矣。”結上知人安民之意。 \par}

無教逸欲,有邦\footnote{不為逸豫貪慾之教,是有國者之常。}兢兢業業,一日二日萬幾。\footnote{兢兢,戒慎。業業,危懼。幾,微也。言當戒懼萬事之微。兢,居凌反。業如字,徐五荅反。幾,徐音機。}無曠庶官,天工人其代之。\footnote{曠,空也。位非其人為空官。言人代天理官,不可以天官私非其才。}


{\noindent\zhuan\zihao{6}\fzbyks 傳“不為”至“之常”。正義曰:“毋”者禁戒之辭,人君身為逸欲,下則效之,是以禁人君使不自為耳。不為逸豫貪慾之教,是有國者之常也。此文主於天子,天子謂天下為國。\CJKunderwave{詩}云“生此王國”之類是也。 \par}

{\noindent\zhuan\zihao{6}\fzbyks 傳“兢兢”至“之微”。正義曰:\CJKunderwave{釋訓}云:“兢兢,戒也。業業,危也。”戒必慎,危必懼,傳言“慎”、“懼”以足之。\CJKunderwave{易·繫辭}云:“幾者動之微。”故“幾”為微也。一日二日之間,微者乃有萬事,言當戒慎萬事之微。微者尚有萬,則大事必多矣。且微者難察,察則勞神,以言不可逸耳。馬、王皆云:“一日、二日,猶日日也。” \par}

{\noindent\zhuan\zihao{6}\fzbyks 傳“曠空”至“其才”。正義曰:“曠”之為空,常訓也。位非其人,所職不治,是為空官。天不自治,立君乃治之。君不獨治,為臣以佐之。下典、禮、德、刑,無非天意者。天意既然,人君當順天,是言人當代天治官。官則天之官,居天之官,代天為治,苟非其人,不堪此任,人不可以天之官而私非其才。\CJKunderline{王肅}云:“天不自下治之,故人代天居之,不可不得其人也。” \par}

天敘有典,敕我五典五惇哉!\footnote{天次敘人之常性,各有分義,當敕正我五常之敘,使合於五厚,厚天下。有典,馬本作\CJKunderwave{五典}。分,扶問反。}天秩有禮,自我五禮有庸哉!\footnote{庸,常。自,用也。天次秩有禮,當用我公、侯、伯、子、男五等之禮以接之,使有常。有庸,馬本作“五庸”。}同寅協恭,和衷哉!\footnote{衷,善也。以五禮正諸侯,使同敬合恭而和善。衷音中。}


{\noindent\zhuan\zihao{6}\fzbyks 傳“天次”至“天下”。正義曰:天敘有典,有此五典,即父義、母慈、兄友、弟恭、子孝是也。五者人之常性,自然而有,但人性有多少耳。天次敘人之常性,使之各有分義。義,宜也。今此義、慈、友、恭、孝各有定分,合於事宜。此皆出天然,是為天次敘之。天意既然,人君當順天之意,敕正我五常之教,使合於五者皆厚,以教天下之民也。五常之教,人君為之,故言“我”也。五教遍於海內,故以“天下”言之。 \par}

{\noindent\zhuan\zihao{6}\fzbyks 傳“庸常”至“有常”。正義曰:“庸,常”,\CJKunderwave{釋詁}文。又云:“由,自也。”“由”是用,故“自”為用也。“天次敘有禮”,謂使賤事貴,卑承尊,是天道使之然也。天意既然,人君當順天意,用我公、侯、伯、子、男五等之禮以接之,使之貴賤有常也。此文主於天子,天子至於諸侯,車旗衣服、國家禮儀、饗食燕好、饔餼飧牢,禮各有次秩以接之。上言“天敘”,此云“天秩”者,“敘”謂定其倫次,“秩”謂制其差等,義亦相通。上云“敕我”,此言“自我”者,五等以教下民,須敕戒之;五禮以接諸侯,當用我意;故文不同也。上言“五惇”,此言“五庸”者,五典施於近親,欲其恩厚;五禮施於臣下,欲其有常;故文異也。\CJKunderline{王肅}云:“五禮謂王、公、卿、大夫、士。”\CJKunderline{鄭玄}云:“五禮,天子也,諸侯也,卿大夫也,士也,庶民也。”此無文可據,各以意說耳。 \par}

{\noindent\zhuan\zihao{6}\fzbyks 傳“衷善”至“和善”。正義曰:“衷”之為善,常訓也。故\CJKunderwave{左傳}云“天誘其衷”,說者皆以衷為善。此文承“五禮”之下,禮尚恭敬,故“以五禮正諸侯,使同敬合恭而和善”也。\CJKunderline{鄭玄}以為“並上之禮共有此事”。五典室家之內,務在相親,非復言以恭敬,恭敬惟為五禮而已,孔言是也。 \par}

天命有德,五服五章哉!\footnote{五服,天子、諸侯、卿、大夫、士之服也。尊卑彩章各異,所以命有德。}天討有罪,五刑五用哉!\footnote{言天以五刑討五罪,用五刑宜必當。}政事懋哉!懋哉!\footnote{言敘典秩禮,命德討罰無非天意者,故人君居天官,聽政治事,不可以不自勉。}

{\noindent\zhuan\zihao{6}\fzbyks 傳“五服”至“有德”。正義曰:\CJKunderwave{益稷}云:“以五采彰施於五色,作服,汝明。”是天子、諸侯、卿、大夫、士之服也。其“尊卑採章各異”,於彼傳具之。天命有德,使之居位,命有貴賤之倫,位有上下之異,不得不立名,以此等之,象物以彰之。先王制為五服,所以表貴賤也。服有等差,所以別尊卑也。 \par}

{\noindent\shu\zihao{5}\fzkt “無教”至“懋哉”。正義曰:\CJKunderline{皋陶}既言用人之法,又戒以居官之事:“上之所為,下必效之。無教在下為逸豫貪慾之事,是有國之常道也。為人君當兢兢然戒慎,業業然危懼。”言當戒慎。“一日二日之間而有萬種幾微之事,皆須親自知之,不得自為逸豫也。萬幾事多,不可獨治,當立官以佐己,無得空廢眾官,使才非其任。此官乃是天官,人其代天治之,不可以天之官而用非其人”。又言:“典禮德刑皆從天出,天次敘人倫,使有常性,故人君為政,當敕正我父、母、兄、弟、子五常之教教之,使五者皆惇厚哉!天又次敘爵命,使有禮法,故人君為政,當奉用我公、侯、伯、子、男五等之禮接之,使五者皆有常哉!接以常禮,當使同敬合恭而和善哉!天又命用有九德,使之居官,當承天意為五等之服,使五者尊卑彰明哉!天又討治有罪,使之絕惡,當承天意為五等之刑,使五者輕重用法哉!典禮德刑,無非天意,人君居天官,聽治政事,當須勉之哉!” \par}

天聰明,自我民聰明;\footnote{言天因民而降之福,民所歸者天命之。天視聽人君之行,用民為聰明。}天明畏,自我民明畏。\footnote{天明可畏,亦用民成其威。民所叛者天討之,是天明可畏之效。畏如字,徐音威,馬本作威。}達於上下,敬哉有土!”\footnote{言天所賞罰,惟善惡所在,不避貴賤。有土之君,不可不敬懼。}\CJKunderline{皋陶}曰:“朕言惠,可厎行?”\footnote{其所陳“九德”以下之言,順於古道,可致行。}\CJKunderline{禹}曰:“俞,乃言厎可績。”\footnote{然其所陳,從而美之曰:“用汝言,致可以立功。”}\CJKunderline{皋陶}曰:“予未有知,思曰贊贊襄哉!”\footnote{言我未有所知,未能思致於善,徒亦贊奏上古行事而言之。因\CJKunderline{禹}美之,承以謙辭,言之序。知如字,徐音智。思如字,徐音息吏反。襄,息羊反,上也。馬云:“因也。”案\CJKunderwave{爾雅}作儴,因也,如羊反。}

{\noindent\zhuan\zihao{6}\fzbyks 傳“言天”至“聰明”。正義曰:皇天無心,以百姓之心為心。此經大意言民之所欲,天必從之。“聰明”謂聞見也,天之所聞見,用民之所聞見也。然則“聰明”直是見聞之義,其言未有善惡;以下言“明威”,是天降之禍,知此“聰明”是天降之福。此即\CJKunderwave{泰誓}所云“天聽自我民聽,天視自我民視”。故“民所歸者,天命之。”大而言之,民所歸就,天命之為天子也。小而言之,雖公卿大夫之任,亦為民所歸向,乃得居之。此文主於天子,故言“天視聽人君之行,用民為聰明”,戒天子使順民心,受天之福也。 \par}

{\noindent\zhuan\zihao{6}\fzbyks 傳“言天”至“敬懼”。正義曰:上句有賞罰,故言“天所賞罰,不避貴賤”。此之“達於上下”,言天子亦不免也。\CJKunderwave{喪服}\CJKunderline{鄭玄}注云:“天子諸侯及卿大夫有地者皆曰君。”即此“有土”可兼大夫以上。但此文本意實主於天子,戒天子不可不敬懼也。 \par}

{\noindent\zhuan\zihao{6}\fzbyks 傳“言我”至“之序”。正義曰:\CJKunderline{皋陶}目言“可致行”,\CJKunderline{禹}言“致可績”,此承而為謙,知其目言未有所知,未能思致於善也。“思”字屬上句,\CJKunderline{王肅}云:“贊贊猶贊奏也。”顯氏云:“襄,上也。謂贊奏上古行事而言之也。”經云“曰”者,謂我上之所言也。傳不訓“襄”為上,已從“襄陵”而釋之。故二劉並以“襄”為因,若必為因,孔傳無容不訓其意。言進習上古行事,因贊成其辭而言之也。傳雖不訓“襄”字,其義當如王說。\CJKunderline{皋陶}慮忽之,自云“言順可行”。因\CJKunderline{禹}美之,即承謙辭。一揚一抑,言之次序也。\CJKunderline{鄭玄}云:“贊,明也。襄之言暢,言我未有所知,所思徒贊明帝德,暢我忠言而已。謙也。” \par}

{\noindent\shu\zihao{5}\fzkt “天聰”至“襄哉。正義曰:此承上“懋哉”以下言所勉之者。以天之聰明視聽,觀人有德。用我民以為耳目之聰明,察人言善者,天意歸賞之。又天之明德可畏,天威者,用我民言惡而叛之,因討而伐之,成其明威。天所賞罰,達於上下,不避貴賤,故須敬哉,有土之君!\CJKunderline{皋陶}既陳此戒,欲其言入之,故曰:“我之此言,順於古道,可致行,不可忽也。”\CJKunderline{禹}即受之曰:“然,汝言用而致可以立功。”重其言以深戒帝。\CJKunderline{皋陶}乃承之以謙曰:“我未有所知,未能思致於善,我所言曰,徒贊奏上古所行而言之哉!非己知思而所自能。”是其謙也。 \par}

%%% Local Variables:
%%% mode: latex
%%% TeX-engine: xetex
%%% TeX-master: "../Main"
%%% End:
