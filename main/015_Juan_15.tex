%% -*- coding: utf-8 -*-
%% Time-stamp: <Chen Wang: 2024-04-02 11:42:41>

% {\noindent \zhu \zihao{5} \fzbyks } -> 注 (△ ○)
% {\noindent \shu \zihao{5} \fzkt } -> 疏

\chapter{卷十五}


\section{召誥第十四}


成王在豐,欲宅洛邑,\footnote{武王克商,遷九鼎於洛邑,欲以為都,故成王居焉。}使召公先相宅,\footnote{相所居而卜之,遂以陳戒。○召,詩照反。相,息亮反,下注同。}作\CJKunderwave{召誥}。

召誥\footnote{召公以成王新即政,因相宅以作誥。}


{\noindent\zhuan\zihao{6}\fzbyks 傳“武王”至“居焉”。正義曰:桓二年\CJKunderwave{左傳}云:“昔武王克商,遷九鼎於洛邑。”服虔注云:“今河南有鼎中觀。”雲“九鼎”者,案宣三年\CJKunderwave{左傳}王孫滿云:“昔夏之方有德也,貢金九牧,鑄鼎象物。”然則九牧貢金為鼎,故稱“九鼎”,其實一鼎。案\CJKunderwave{戰國策}顏率說齊王云,昔武王克商,遷九鼎,鼎用九萬人,則以為其鼎有九。但遊說之辭,事多虛誕,不可信用。然鼎之上備載九州山河異物,亦又可疑。未知孰是,故兩解之。 \par}

{\noindent\zhuan\zihao{6}\fzbyks 傳“相所”至“陳戒”。正義曰:孔以序言“相宅”,於經意不盡,故為傳以助成之。召公相所居而卜之,及其經營大作,遂以陳戒,史錄陳戒為篇。其意不在相宅,序以經具,故略之耳。言“先相宅”者,明於時周公攝政,居洛邑是周公之意,周公使召公先行,故言“先”,以見周公自後往也。 \par}

{\noindent\zhuan\zihao{6}\fzbyks 傳“召公”至“作誥”。正義曰:武王既崩,周公即攝王政,至此已積七年,將歸政成王,故經營洛邑,待此邑成,使王即政。召公以成王將新即政,恐王不順周公之意,或將惰於政事,故因相宅以作誥也。作誥之時,王未即政,周公作\CJKunderwave{洛誥},為反政於成王,召公陳戒,為即政後事,故傳言“新即政”也。 \par}

{\noindent\shu\zihao{5}\fzkt “成王”至“召誥”。正義曰:成王於時在豐,欲居洛邑以為王都,使召公先往相其所居之地,因卜而營之。王與周公從後而往,召公於庶殷大作之時,乃以王命取幣以賜周公,因告王宜以夏殷興亡為戒。史敘其事,作\CJKunderwave{召誥}。 \par}

惟二月既望,\footnote{周公攝政七年二月十五日,日月相望,因紀之。}越六日乙未,王朝步自周,則至於豐。\footnote{於已望後六日,二十一日,成王朝行從鎬京,則至於豐,以遷都之事告文王廟。告文王,則告武王可知,以祖見考。○鎬,胡老反。見,賢遍反,下“不見”同。}惟太保先周公相宅。\footnote{太保,三公官名,召公也。召公於周公前相視洛居,周公後往。○先,息薦反,又如字。}越若來三月,惟丙午。越三日戊申,太保朝至於洛,卜宅。\footnote{朏,明也,月三日明生之名。於順來三月丙午朏。於朏三日,三月五日,召公早朝至於洛邑,相卜所居。○朏,芳尾反,又普沒反,徐又芳憒反。}


{\noindent\zhuan\zihao{6}\fzbyks 傳“周公”至“紀之”。正義曰:\CJKunderwave{洛誥}云:“周公誕保文武受命,惟七年。”\CJKunderwave{洛誥}是攝政七年事也。\CJKunderwave{洛誥}周公云:“予惟乙卯,朝至於洛師。”此篇雲“乙卯,周公朝至於洛”,正是一事,知此“二月”是周公攝政七年之二月也。“望”者,於月之半月,當日衝,日光照月光圓滿,面鄉相當,猶人之相望,故稱“望”也。治歷者必先正朔望,故史官因紀之。將言望後之事,必以望紀之。將言朏後之事,則以朏紀之。猶今人將言日,必先言朔也。望之在月十六日為多,太率十六日者四分之三,十五日者四分之一耳。此年入戊午蔀五十六歲,二月小,乙亥朔。孔雲十五日即為望,是己丑為望,言“已望”者,謂庚寅十六日也。且孔雲“望”與“生魄”、“死魄”皆舉大略而言之,不必恰依歷數。又算術前月大者,後月二日月見,可十五日望也。顧氏亦云:“十五日望,日月正相望也。” \par}

{\noindent\zhuan\zihao{6}\fzbyks 傳“於已”至“見考”。正義曰:“於已望後六日”,是為二十一日也。“步”,行也。此雲“王朝行”,下太保與周公言“朝至”者,君子舉事貴早朝,故皆言“朝”也。宗周者,為天下所宗,止謂王都也。武王已都於鎬,故知宗周是鎬京也。文王居豐,武王未遷之時,於豐立文王之廟,遷都而廟不毀,故成王居鎬京,“則至於豐,以遷都之事告文王廟”也。大事告祖,必告於考,此經不言告武王,以告文王則告武王可知,以告祖見考也。告廟當先祖後考,此必於豐告文王,於鎬京告武王也。 \par}

{\noindent\zhuan\zihao{6}\fzbyks 傳“朏明”至“所居”。正義曰:\CJKunderwave{說文}雲“朏,月未盛之明”,故為“明”也。\CJKunderwave{周書·月令}云:“三日粵朏。”“朏”字從月出,是入月三日明生之名也。“於順來”者,於二月之後依順而來,次三月也。二月乙未而發豐,歷三月丙午朏,又於朏三日,是三月五日,凡發豐至洛為十四日也。“召公早朝至於洛邑,相卜所居”,當以至洛之日即卜也。 \par}

厥既得卜,則經營。\footnote{其已得吉卜,則經營規度城郭郊廟朝市之位處。○度,待洛反。朝,直遙反。處,昌慮反。}越三日庚戌,太保乃以庶殷攻位於洛汭。越五日甲寅,位成。\footnote{於戊申三日庚戌,以眾殷之民治都邑之位於洛水北,今河南城也。於庚戌五日,所治之位皆成。言眾殷,本其所由來。○汭,如銳反。}

{\noindent\zhuan\zihao{6}\fzbyks 傳“其已”至“位處”。正義曰:“經營”,\CJKunderwave{考工記}所云“匠人營國,方九里,左祖右社,面朝後市”是也。下有“丁已郊”,故知“規度城郭郊廟朝市之位處”也。\CJKunderwave{匠人}不言“郊”,以不在國內也。\CJKunderwave{匠人}王城方九里,如\CJKunderwave{典命}文,又以公城方九里,天子城十二里。\CJKunderline{鄭玄}兩說,孔無明解,未知從何文也。“郊”者,\CJKunderwave{司馬法}“百里為郊”,鄭注\CJKunderwave{周禮}雲“近郊五十里”,\CJKunderwave{禮記}祭天於南郊,祭地於北郊,皆謂近郊也。其“廟”,案\CJKunderwave{小宗伯}云:“建國之神位,右社稷,左宗廟。”鄭注\CJKunderwave{朝士職}云,庫門內之左右。其朝者,鄭云,外朝一,在庫門之外,皋門之內,是詢眾庶之朝。內朝二者,其一在路門外,王每日所視,謂之治朝。其一在路門內,路寢之朝,王每日視訖退路寢,謂之燕朝,或與宗人圖私事。顧氏云:“市處王城之北。朝為陽,故在南。市為陰,故處北。”今案\CJKunderwave{周禮·內宰職}“佐後立市”,然則後既主陰,故立市也。 \par}

{\noindent\zhuan\zihao{6}\fzbyks 傳“於戊”至“由來”。正義曰:於戊申後三日庚戌,為三月七日也。水內曰“汭”,蓋以人南面望水,則北為內,故“洛汭”為洛水之北。鄭云:“隈曲中也。”\CJKunderwave{漢書·地理志}河南郡治在洛陽縣,河南城別為河南縣。治都邑之位於洛北,今於漢河南城是也。“所治之位皆成”,佈置處所定也。治位乃是周人,而言“眾殷”者,本其所由來,言本是殷民,今來為我周家役也。莊二十九年\CJKunderwave{左傳}發例云:“凡土功,水昏正而栽,日至而畢。”此以周之三月農時役眾者,彼言尋常土功,此則遷都事大,不可拘以常制也。 \par}

{\noindent\shu\zihao{5}\fzkt “惟二月”至“位成”。正義曰:惟周公攝政七年二月十六日,其日為庚寅,既日月相望矣。於已望後六日乙未,為二月二十一日,王以此日之朝行自周之鎬京,則至於豐,以遷都之事告文王之廟。此日王惟命太保召公先周公往洛水之旁相視所居之處,太保即行。其月小,二十九日癸卯晦。於二月之後順來三月,惟三日丙午朏,而月生明於朏。三日戊申,即三月五日,太保乃以此朝旦至於洛,即卜宅。其已得吉卜,則經營之,規度其城郭郊廟朝市之位處。於戊申三日庚戌,為三月七日,太保乃以眾所受於殷之民,治都邑之位於洛水之汭,謂洛水北也。於庚戌五日,為三月十一日甲寅,而所治之位皆成矣。 \par}

若翼日乙卯,周公朝至於洛,\footnote{周公順位成之明日而朝至於洛汭。}則達觀於新邑營。\footnote{周公通達觀新邑所營。言周遍。}越三日丁巳,用牲於郊,牛二。\footnote{於乙卯三日,用牲告立郊位於天,以後稷配,故二牛。后稷貶於天,有羊豕。羊豕不見,可知。}越翼日戊午,乃社於新邑,牛一,羊一,豕一。\footnote{告立社稷之位,用太牢也。\CJKunderline{共工氏}子曰句龍,能平水土,祀以為社。周祖后稷能殖百穀,祀以為稷。社稷共牢。○共音恭。句,故侯反。}


{\noindent\zhuan\zihao{6}\fzbyks 傳“周公”至“洛汭”。正義曰:周公以順成之明日而朝至,則是三月十二日也。其到洛汭。在召公之後七日。不知初發鎬京以何日也。成王蓋與周公俱來。鄭云:“史不書王往者,王於相宅無事也。” \par}

{\noindent\zhuan\zihao{6}\fzbyks 傳“於乙”至“可知”。正義曰:知此用牲是“告立郊位於天”者,此郊與社,於攻位之時已經營之,今非常祭之月,而特用牲祭天,知是郊位既定,告天使知,而今後常以此處祭天也。\CJKunderwave{禮}郊用特牲,不應用二牛。“以後稷配,故二牛”也。\CJKunderwave{郊特牲}及\CJKunderwave{公羊傳}皆雲養牲必養二,“帝牛不吉,以為稷牛”,言用彼為稷牛者,以之祭帝,其稷牛隨時取用,不在滌養,是帝稷各用一牛,故二牛也。先儒皆雲天神尊,祭天明用犢,貴誠之義。稷是人神,祭用太牢,貶於天神,法有羊豕,因天用牛,遂雲“牛二”,舉其大者,從天言之,羊豕不見,可知也。\CJKunderwave{詩·頌·我將}祀文王於明堂雲“惟羊惟牛”,又\CJKunderwave{月令}雲“以太牢祠於高禖”,皆據配者有羊豕也。 \par}

{\noindent\zhuan\zihao{6}\fzbyks 傳“告立”至“共牢”。正義曰:經有社無稷,稷是社類,知其同告之。告立社稷之位,其祭用太牢,故牛羊豕各一也。句龍能平水土,祀之以為社。后稷能殖百穀,祀以為稷。\CJKunderwave{左傳}、\CJKunderwave{魯語}、\CJKunderwave{祭法}皆有此文。漢世儒者說社稷有二,左氏說社稷惟祭句龍,后稷人神而已,是孔之所用。\CJKunderwave{孝經}說社為土神,稷為穀神,句龍后稷配食者,是鄭之所從。而\CJKunderwave{武成}篇雲“告於皇天后土”,孔以後土為地,言“后土,社也”者,以\CJKunderwave{泰誓}雲“類於上帝,宜於冢土”,故以後土為社也。小劉雲“后土與皇天相對”,以後土為地。若然,\CJKunderwave{左傳}雲“句龍為后土”,豈句龍為地乎?社亦名“后土”,地名“后土”,名同而義異也。“社稷共牢”,經無明說,\CJKunderwave{郊特牲}雲“社稷太牢”,二神共言“太牢”,故傳言“社稷共牢”也。此經上句言“於郊”,此不言“於社”;此言“社於新邑”,上句不言“郊於新邑”;上句言“用牲”,此言牛羊豕,不言“用”;告天不言告地,告社不言告稷;皆互相足,從省文也。\CJKunderwave{洛誥}雲“王在新邑烝祭,王入太室祼”,則洛邑亦立宗廟,此不雲“告廟”,亦從省文也。 \par}

越七日甲子,周公乃朝用書,命庶殷侯、甸、男邦伯。\footnote{於戊午七日甲子,是時諸侯皆會,故周公乃昧爽以賦功屬役書,命眾殷侯、甸、男服之邦伯,使就功。邦伯,方伯,即州牧也。}厥既命殷庶,庶殷丕作。\footnote{其已命殷眾,眾殷之民大作。言勸事。}大保乃以庶邦冢君出取幣,乃復入,\footnote{諸侯公卿並覲於王,王與周公俱至,文不見王,無事。召公與諸侯出取幣,欲因大會顯周公。○復,扶又云。}錫周公,曰:“拜手稽首,旅王若公。\footnote{召公以幣入,稱成王命錫周公,曰:“敢拜手稽首,陳王所宜順周公之事。”}

{\noindent\zhuan\zihao{6}\fzbyks 傳“於戊”至“牧也”。正義曰:\CJKunderwave{康誥}云:“周公初基,作新大邑於東國洛,四方民大和會。侯、甸、男邦、採、衛,百工播民和,見士於周。”與此一事也,故知“是時諸侯皆會,故周公乃昧爽以賦功屬役書,命眾殷在侯、甸、男服之邦伯,使就築作功”也。\CJKunderwave{康誥}五服,此惟三服者,立文有詳略耳。昭三十二年,晉合諸侯城成周,\CJKunderwave{左傳}稱命役書於諸侯,“屬役賦文”,此傳言“賦功屬役”,其意出於彼也。“賦功”謂賦功諸侯之功,科其人夫多少。“屬役”謂付屬役之處,使知得地之尺丈也。“邦伯”,諸國之長,故為方伯州牧。\CJKunderwave{王制}云:“千里之外設方伯。”方伯即州牧也。周公命州牧,使州牧各命其所部。 \par}

{\noindent\zhuan\zihao{6}\fzbyks 傳“諸侯”至“周公”。正義曰:上雲“周公朝用書命庶殷”者,周公自命之,其事不由王也。庶殷既已大作,諸侯公卿乃並覲君王,其時蓋有行宮,王在位而諸侯公卿並覲之。既入見王,乃出取幣。初不言“入”,而經言“出”者,下雲“乃復入”,則上以入可知,從省文也。下賜周公言“旅王若公”,明此出入是覲王之事,而經文不見王至,故傳辯之,王與周公俱至,自此已上於王無事,故不見也。正以經文不見王至,知與周公俱至也。周公居攝功成,將歸政於成王,召公與諸侯出取幣,欲因大會顯周公之功既成。將令王自知政,因賜周公,遂以戒王,故出取幣,復入以待王命。其幣蓋玄纁束帛也。\CJKunderline{鄭玄}云:“所賜之幣,蓋璋以皮,及寶玉大弓,此時所賜。”案鄭注\CJKunderwave{周禮}雲“璋以皮,二王之後享後所用”,寧當以賜臣也?寶玉大弓,魯公之分,伯禽封魯,乃可賜之,不得以此時賜周公也。 \par}

{\noindent\zhuan\zihao{6}\fzbyks 傳“召公”至“之事”。正義曰:太保以庶邦冢君出取幣者,以上太保之意,非王命。幣既入,即雲“賜周公”者,下言召公,不得賜周公,知召公既以幣入,乃稱成王命以賜周公。於時政在周公,成王未得賜周公也。但召公見周公功成作邑,將反王政,欲尊王而顯周公,故稱成王之命以賜周公。\CJKunderline{鄭玄}云:“召公見眾殷之民大作,周公德隆功成,有反政之期,而欲顯之。因大戒天下,故與諸侯出取幣,使戒成王立於位,以其命賜周公。”王肅云:“為戒成王錫周公是也。”曰“拜手稽首”者,召公自言己與冢君等,敢拜手稽首,陳王所宜順周公之事。“宜順之事”,自此以下皆是也。 \par}

{\noindent\shu\zihao{5}\fzkt “若翼”至“若公”。正義曰:順位成之明日乙卯,三月十二日也,周公以此朝旦至於洛,則通達而遍觀於新邑所經營。其位處皆無所改易。於乙卯三日丁巳,三月十四日也,用牲於郊,告立祭天之位,牛二,天與后稷所配各用一牛。於丁巳明日戊午,乃祭社於新邑,用太牢牛一、羊一、豕一。於戊午七日甲子,二十一日也,周公乃以此朝旦用策書,命眾殷在侯、甸、男服之內諸國之長,謂命州牧,使告諸國就功作。其已命殷眾,眾殷皆歡樂歡事而大作矣。太保召公乃以眾國大君諸侯出取幣,乃復入,稱成王命以錫周公,曰:“我敢拜手稽首,以戒王,陳說王所宜順周公之事。” \par}

“誥告庶殷,越自乃御事。\footnote{召公指戒成王,而以眾殷諸侯於自乃御治事為辭,謙也。諸侯在,故託焉。}嗚呼!皇天上帝,改厥元子,茲大國殷之命。\footnote{嘆皇天改其大子,此大國殷之命。言紂雖為天所大子,無道猶改之,言不可不慎。}惟王受命,無疆惟休,亦無疆惟恤。\footnote{所以戒成王,天改殷命,惟王受之,乃無窮惟美,亦無窮惟當憂之。}嗚呼!曷其奈何弗敬?\footnote{何其奈何不憂敬之?欲其行敬。}


{\noindent\zhuan\zihao{6}\fzbyks 傳“嘆皇”至“不慎”。正義曰:\CJKunderwave{釋詁}云:“皇,君也。”天地尊之大,故皇天后土皆以君言之也。“改其大子”,謂改天子之位與他姓,即此大國殷之命,謂紂也。言紂雖為天所大子,無道,尢改之,不可不慎也。以託戒諸侯,故言天子雖大猶改之,況已下乎?\CJKunderwave{釋詁}云:“元,首也。”“首”是體之大,故傳言“大子”。鄭云:“言首子者,凡人皆雲天之子,天子為之首耳。” \par}

{\noindent\shu\zihao{5}\fzkt “誥告”至“弗敬”。正義曰:召公所陳戒王宜順周公之事云:“我為言誥,以告汝庶殷之諸侯,下自汝御事。”欲令君臣皆聽之,其實指以戒王。諸侯皆在,託以為言也。乃曰:“嗚呼!有皇天上帝,改去其大子所受者,即此大國殷之王命也。以其無道,故改命。有德惟王,受得此命,乃無窮惟美,亦無窮惟當憂之。既憂之無窮,嗚呼!何其柰何不敬乎?”欲其長行敬也。“告庶殷”者,告諸侯也。“庶殷”,通尊卑之辭,故民與諸侯同雲“庶殷”,皆謂所受於殷之眾也。 \par}

天既遐終大邦殷之命,茲殷多先哲王在天,\footnote{言天己遠終殷命,此殷多先智王,精神在天不能救者,以紂不行敬故。}越厥後王后民,茲服厥命。\footnote{於其後王后民,謂先智王之後繼世君臣。此服其命,言不忝。}厥終智藏瘝在。\footnote{其終,后王之終,謂紂也。賢智隱藏,瘝病者在位,言無良臣。○瘝,工頑反。}夫知保抱攜持厥婦子,以哀籲天,徂厥亡出執。\footnote{言困於虐政,夫知保抱其子,攜持其妻,以哀號呼天,告冤無辜,往其逃亡,出見執殺,無地自容,所以窮。○夫知,並如字,注同。籲音喻,呼也。號,戶高反。}嗚呼!天亦哀於四方民,其眷命用懋。\footnote{民哀呼天,天亦哀之,其顧視天下有德者,命用勉敬者為民主。}


{\noindent\zhuan\zihao{6}\fzbyks 傳“言天”至“敬故”。正義曰:天既遠終殷命,言其去而不復反也。說天終殷之命,而言智王在天者,言先智王雖精神在天,而不能救紂者,以紂不行敬故也。戒王使行敬。 \par}

{\noindent\zhuan\zihao{6}\fzbyks 傳“於其”至“不忝”。正義曰:“先智王之後繼世君臣”,謂智王之後,紂已前能守位不失者。經言“后王後民”,傳言“君臣”者,見民內有臣。民於此皆服行君之命,言不忝辱父祖也。 \par}

{\noindent\zhuan\zihao{6}\fzbyks 傳“其終”至“良臣”。正義曰:既言“后王”,又復言“其終”,知是“后王之終,謂紂也”。以“瘝”從病類,故言“瘝病”也。鄭、王皆以“瘝”為病,小人在位,殘暴在下,故以病言之。 \par}

{\noindent\zhuan\zihao{6}\fzbyks 傳“言困”至“以窮”。正義曰:言困於虐政,抱子攜妻欲去之。“夫”尢人人,言天下盡然也。“保”訓安也。王肅云:“匹夫知欲安其室,抱其子,攜其妻以悲呼天也。” \par}

{\noindent\shu\zihao{5}\fzkt “天既”至“用懋”。正義曰:更述改殷之事。天既遠終大國殷之王命矣,此殷多有先智之王,精神在天,不能救紂,以紂不行敬故也。於其智王之後人,謂繼世之君及其時之人,皆服行其君之命,由其亦能行敬,故得不忝其先祖。其此後王之終,謂紂之時賢智者隱藏,瘝病者在位,言其時無良臣。多行無禮暴虐,於時之民困於虐政,夫知保抱攜持其婦子,以哀號呼天,告冤枉無辜,往其逃亡,出見執殺,言無地自容以困窮也。天亦哀矜於四方之民,其眷顧天下,選擇賢聖,命用勉力行敬者以為民主,故王今得之也。 \par}

王其疾敬德,相古先民有夏。\footnote{言王當疾行敬德,視古先民有夏之王,以為法戒之。}天迪從子保,面稽天若,今時既墜厥命。\footnote{夏禹能敬德,天道從而子安之。\CJKunderline{禹}亦面考天心而順之,今是桀棄\CJKunderline{禹}之道,天已墜其王命。}今相有殷,\footnote{次復觀有殷。}天迪格保,面稽天若,\footnote{言天道所以至於保安湯者,亦如\CJKunderline{禹}。}今時既墜厥命。\footnote{墜其王命。}今衝子嗣,則無遺壽耇。\footnote{童子,言成王少嗣位治政。無遺棄老成人之言,欲其法之。}曰其稽我古人之德,矧曰其有能稽謀自天?\footnote{衝子成王其考行古人之德則善矣,況曰其有能考謀從天道乎?言至善。}


{\noindent\zhuan\zihao{6}\fzbyks 傳“夏禹”至“王命”。正義曰:勸王疾行敬德,乃言天道安夏,知夏禹能行敬德,天道從而子安之。天既子愛\CJKunderline{禹},\CJKunderline{禹}亦順天心。鄭雲“面猶迴向也”,則“面”為向義。\CJKunderline{禹}亦志意向天,考天心而順安之,言能同於天心也。\CJKunderline{禹}興夏而桀滅之,知天道子保者是\CJKunderline{禹}也,既墜厥命者是桀也。今桀廢\CJKunderline{禹}之道,已墜失其王命矣。 \par}

{\noindent\zhuan\zihao{6}\fzbyks 傳“言天”至“如\CJKunderline{禹}”。正義曰:此說二代興亡,其意同也。於\CJKunderline{禹}言“從而子安之”,則天於湯亦子安之,故於湯因上略文,直言“格保”。“格”,至也,言至於保安湯者,亦如\CJKunderline{禹}也。 \par}

{\noindent\zhuan\zihao{6}\fzbyks 傳“童子”至“法之”。正義曰:“嗣位治政”,謂周公歸政之後,此時王末蒞政,而言“今衝子嗣”者,召公此戒,戒其即政之後故也。“壽”謂長命,“耇”是老稱,無遺棄長命之老人,欲其取老人之言而法效之,老人之言即下雲“古人之德”也。 \par}

{\noindent\shu\zihao{5}\fzkt “王其”至“自天”。正義曰:既言皇天眷,顧命用勉敬者為人主,故戒王,言其疾行敬德,視古先民有夏之君,取\CJKunderline{大禹}以為法戒。\CJKunderline{禹}以能敬之故,天道從而子安之,\CJKunderline{禹}能面考天心而順以行敬。今是桀棄\CJKunderline{禹}之道,已墜失其王命矣。更復視有殷之君,取\CJKunderline{成湯}以為法戒,湯以能敬之故,天亦從而子。安之天道所以至於保安湯者,亦以湯麵考天心而順以行敬也。今是紂棄湯之道,已墜失其王命矣。夏殷二代,能敬則得之,不敬則失之。今童子為王嗣位治政,則無遺棄壽考成人,宜用老成人之言,法古人為治。曰王其考行古人之德,則已善矣,況曰其有能考行所謀以順從天道乎?若能從順天道,則與\CJKunderline{禹}湯同功,言其善不可加也。 \par}

“嗚呼!有王雖小,元子哉!其丕能諴於小民,今休。\footnote{召公嘆曰:“有成王雖少,而大為天所子,其大能和於小民,成今之美。”勉之。○諴音咸。}王不敢後用,顧畏於民碞。\footnote{王為政當不敢後能用之士,必任之為先。碞,僣也。又當顧畏於下民僣差禮義,能此二者,則德化立,美道成也。○碞,五咸反,徐音吟。}


{\noindent\zhuan\zihao{6}\fzbyks 傳“王為”至“道成”。正義曰:王者為政,任賢使能,有能有用,宜先任之,故“王者為政當不敢後其能用之士,必任之為先”也。“碞”即巖也,參差不齊之意,故為僣也。既任能人,復憂下民,故“又當顧畏於下民僣差禮義”。畏其僣差,當治之使合禮義也。能此二者,則德化立,美道成。“美道成”即“今休”是也。 \par}

{\noindent\shu\zihao{5}\fzkt “嗚呼”至“民碞”。正義曰:召公嘆以戒王:“嗚呼!今所有之王,惟今雖復少小,而大為天所子愛哉!”言任大也。“若其大能和同於天下小民,則成今之美”。以勉之。“故王當不敢後其能用之士,必任以為先。又當顧念畏於下民僣差禮義,能此二者,則德化立,美道成矣”。 \par}

王來紹上帝,自服於土中。\footnote{言王今來居洛邑,繼天為治,躬自服行教化於地勢正中。○治,直吏反,下“為治”、“致治”皆同。}旦曰:‘其作大邑,其自時配皇天。\footnote{稱周公言,其為大邑於土中,其用是大邑,配上天而為治。}毖祀於上下,其自時中乂。\footnote{為治當慎祀於天地,則其用是土中大致治。}王厥有成命,治民今休。’\footnote{用是土中致治,則王其有天之成命,治民今獲太平之美。}


{\noindent\zhuan\zihao{6}\fzbyks 傳“言王”至“正中”。正義曰:傳言“躬自服行”,則不訓自也,鄭、王皆以“自”為用。 \par}

{\noindent\zhuan\zihao{6}\fzbyks 傳“稱周”至“為治”。正義曰:王肅云:“旦,周公名也。禮,君前臣名,故稱周公之言為‘旦曰’。”王者為天所子,代天治民,天有其意,天子繼天使成,謂之“紹上帝”也。天子設法,其理合於天道,是為“配皇天”也。天子將欲配天,必宜治居土中,故稱周公之言,其為大邑於土之中,其當令此成王,用是大邑行化,配上天而為治也。說周公之意然,戒成王使順公也。\CJKunderwave{周禮·大司徒}云:“以土圭之法測土深,正日影,以求地中。日南則影短多暑,日北則影長多寒,日東則影夕多風,日西則影朝多陰。日至之影尺有五寸,謂之地中,天地之所合也,四時之所交也,風雨之所會也,陰陽之所和也。然則百物阜安,乃建王國焉。”馬融云:“王國,東都王城,今河南縣是也。” \par}

{\noindent\zhuan\zihao{6}\fzbyks 傳“為治”至“致治”。正義曰:\CJKunderwave{祭法}云:“有天下者祭百神。”天地為大,“上下”即天地也,故“為治當慎祀於天地”。舉天地則百神之祀皆慎之也。能事神訓民,則其用是土中大致治也。 \par}

{\noindent\zhuan\zihao{6}\fzbyks 傳“用是”至“之美”。正義曰:用是土中致治,當於天心,則王其有天之成命,降福與之,使多歷年歲治民,今獲太平之美。自“旦曰”至此,述周公之意也。 \par}

{\noindent\shu\zihao{5}\fzkt “王來”至“今休”。正義曰:周公之作洛邑,將以反政於王,故召公述其遷洛之意。今王來居洛邑,繼上天為治,躬自服行教化於土地王中之處,故周公旦言曰:“其作大邑於土中,其令成王用是大邑,配大天而為治。為治之道,當事神訓民,謹慎祭祀上下神祇,其用是土中大致治也。既能治,則王其有天之成命,治理下民,今獲太平之美矣。” \par}

王先服殷御事,比介於我有周御事,\footnote{召公既述周公所言,又自陳己意,以終其戒。言當先服治殷家御事之臣,使比近於我有周治事之臣,必和協,乃可一。○比,毗志反,徐扶志反。近,附近之近。}節性,惟日其邁。\footnote{和比殷周之臣,時節其性,令不失中,則道化惟曰其行。○令,力呈反。}王敬作所不可不敬德。\footnote{敬為所不可不敬之德,則下敬奉其命矣。}


{\noindent\zhuan\zihao{6}\fzbyks 傳“召公”至“可一”。正義曰:自“今休”已上,文義相連,知皆是稱周公言也。此一句意異於上,知是“召公自陳己意,以終其戒”。“殷家治事之臣”,謂殷朝舊人,常被殷家任使者也。“周家治事之臣”,謂西土新來翼贊周家初基者也。周臣恃功,或加陵殷士;殷人失勢,或疏忌周臣;新舊不和,政必乖戾。故召公戒王當先治殷臣,使比近周臣,必和協,政乃可一也。不使周臣比殷,而令殷臣比周臣者,周臣奉周之法,當使殷臣從之,故治殷臣使比周臣也。 \par}

{\noindent\zhuan\zihao{6}\fzbyks 傳“和比”至“其行”。正義曰:文承殷周之下,故知“和比殷周之臣”。人各有性,嗜好不同,各恣所欲,必或反道。故以禮義時節其性命,示之限分,令不失中。皆得中道,則各奉王化,故王之道化惟日其行。言日日當行之,日益遠也。顧氏云:“和協殷周新舊之臣,制其性命,勿使怠慢也。” \par}

{\noindent\zhuan\zihao{6}\fzbyks 傳“敬為”至“命矣。”。正義曰:聖王為政,當使易從而難犯,故令行如流水,民從如順風。若使設難從之教,為易犯之令,雖迫以嚴刑,而終不用命。故為其德不可不敬也。王必敬為此不可不敬之德,則下民無不敬奉其命矣民奉其王命,是化行也。 \par}

{\noindent\shu\zihao{5}\fzkt “王先”至“敬德”。正義曰:召公既述周公所言,又自陳己意,戒王今為政,先服治殷家御事之臣,使之比近於我有周治事之臣,令新舊和協,政乃可一。和比殷周之臣,時節其性命,令不失其中,則王之道化惟日其行矣。王當敬為所不可不敬之德,其德為下所敬,則下敬奉其上命,則化必行矣。化在下者,常苦命之不行,故以此為戒。 \par}

“我不可不監於有夏,亦不可不監於有殷。\footnote{言王當視夏殷,法其歷年,戒其不長。}我不敢知曰,有夏服天命,惟有歷年。\footnote{以能敬德,故多歷年數。我不敢獨知,亦王所知。}我不敢知曰,不其延,惟不敬厥德,乃早墜厥命。\footnote{言桀不謀長久,惟以不敬其德,故乃早墜失其王命,亦王所知。}我不敢知曰,有殷受天命,惟有歷年。\footnote{夏言服,殷言受,明受而服行之,互相兼也。殷之賢王,猶夏之賢王,所以歷年,亦王所知。}我不敢知曰,不其延,惟不敬厥德,乃早墜厥命。\footnote{紂早墜其命,猶桀不敬其德,亦王所知。}今王嗣受厥命,我亦惟茲二國命,嗣若功。\footnote{其夏殷也,繼受其王命,亦惟當以此夏殷長短之命為監戒,繼順其功德者而法則之。}


{\noindent\zhuan\zihao{6}\fzbyks 傳“言王”至“不長”。正義曰:“相”、“監”俱訓為視,上言“相有夏”、“相有殷”,今復重言“監有夏”、“監有殷”者,上言順天則興,棄命則滅,此言敬則歷年,不敬則短,故重言視夏殷,欲令王法其歷年,戒其不長故也。 \par}

{\noindent\zhuan\zihao{6}\fzbyks 傳“以能”至“所知”。正義曰:下云:“不敬厥德,乃早墜厥命”,知其“以能敬德者,故多歷年數”也。上言“相夏”、“相殷”皆雲“天迪從子保,面稽天若”,言上天以道安人,人主考天順之,非創業之君不能如是,故傳以\CJKunderline{禹}湯當之。此言“敬德”、“歷年”,則繼體賢君亦能如此,所言“歷年”非獨\CJKunderline{禹}湯而已。下傳雲“殷之賢王,猶夏之賢王”,則此多歷年數者,夏則桀前之賢王,殷則紂前之賢王,不失位者皆是也。召公此誥,指以告王,故知言“我不敢獨知”者,其意言亦是王所知也。王說亦然。 \par}

{\noindent\shu\zihao{5}\fzkt “我不”至“若功”。正義曰:言王所以須慎敬所為不可不敬之德者,以我不可不監視於有夏,亦不可不監視於有殷,皆有歷年,長不與長,由敬與不敬故也,王當法其歷年,戒其不長。更說宜監之意,我不敢獨知,亦王所知,曰有夏之君,服行天命,以敬德之故,惟有多歷年數。謂桀父已前也。其末亦我不敢獨知,亦王所知,曰有夏桀不其長久,惟不敬其德,乃早墜失其王命。是為敬者長,不敬者短,所以我不可不監夏也。我不敢獨知,亦王所知,曰有殷之君受天命,以敬德之故,惟有多歷年數。謂紂父已前也。其末亦我不敢獨知,亦王所知,曰殷紂不其長久,惟不敬其德,乃早墜失其王命。亦是所敬者長,不敬者短,所以我不可不監殷也。夏殷短長既如此矣,今王繼受其命,我亦惟當用此二國夏殷長短之命以為監戒,繼順其功德者而法則之。勸王為敬也。 \par}

王乃初服。嗚呼!若生子,罔不在厥初生,自貽哲命。\footnote{言王新即政,始服行教化,當如子之初生,習為善,則善矣。自遺智命,無不在其初生,為政之道,亦猶是也。○遺,唯季反。}今天其命哲,命吉凶,命歷年。\footnote{今天制此三命,惟人所修。脩敬德則有智,則常吉,則歷年,為不敬德則愚兇不長。雖說之,其實在人。}知今我初服,宅新邑,肆惟王其疾敬德。\footnote{天已知我王今初服政,居新邑洛都,故惟王其當疾行敬德。}王其德之用,祈天永命。\footnote{言王當其德之用,求天長命以歷年。}


{\noindent\zhuan\zihao{6}\fzbyks 傳“言王”至“猶是”。正義曰:以此新即政,始行教化,比子之初生,始欲學習為善,則善矣。若能為善,天必授之以賢智之命,是此賢智之命由己行善而來,是自遺智命矣。初習為惡則惡矣,若其為惡,天必授之以頑愚之命,亦是自遺愚命也。方欲勸王慕善,故惟舉智命而不言愚命者,愚智由學習而至,是無不在其初生。此初生謂年長,以解習學,非初始生也。為政之道亦猶是。為善政得福,為惡政得禍,亦如初生之子習善惡也。 \par}

{\noindent\zhuan\zihao{6}\fzbyks 傳“今天”至“在人”。正義曰:命由天授,遠舉天心,故言“今天制此三命”。有“哲”當有“愚”,有“歷年”當有“不長”,文不備者,以“吉凶”相反,言“命吉凶”,則“哲”對“愚”,“歷年”對“不長”可知矣。天制此三命,善惡由人,惟人所修習也。此篇所云,惟勸脩敬德。故云“脩敬德則有智,則常吉,則多年,惟不敬德則愚兇不長也”。愚智夭壽之外而別言吉凶,於凡人則康強為吉,病患為兇,於王者則太平為吉,禍亂為兇,三者雖以託天說之,其實行之在人。人行之有善惡,天隨以善惡授之耳。此是立教誘人之辭,不可以賢智天枉為難也。 \par}

{\noindent\zhuan\zihao{6}\fzbyks 傳“言王”至“歷年”。正義曰:“其德之用”,言為行當用德,用德與“疾敬德”為一事也。故上傳雲“王其當疾行敬德”,則此文是也。 \par}

其惟王勿以小民淫用非彝,\footnote{勿用小民過用非常。欲其重民秉常。}亦敢殄戮用乂民,\footnote{亦當果敢絕刑戮之道,用治民。戒以慎罰。}若有功,其惟王位在德元。\footnote{順行\CJKunderline{禹}湯所以成功,則其惟王居位在德之首。}小民乃惟刑用於天下,越王顯。\footnote{王在德元,則小民乃惟用法於天下。言治政於王亦有光明。}

{\noindent\zhuan\zihao{6}\fzbyks 傳“勿用”至“秉常”。正義曰:勿用小民非常役,用為非常之義,戒王當使民以時,莫為非常勞役,欲其重民秉常也。 \par}

{\noindent\zhuan\zihao{6}\fzbyks 傳“亦當”至“慎罰”。正義曰:聖人作法,以刑止刑,以殺止殺,若直犯罪之人,亦當果敢致罪之,以此絕刑戮之道,用治民。謂獄事無疑,決斷得理,則果敢為絕刑戮之道。若其獄情疑惑,枉濫者多,是為不能果敢絕刑戮之道也。上戒王以明德,此戒王以慎罰,故言“亦”也。 \par}

{\noindent\zhuan\zihao{6}\fzbyks 傳“順行”至“之首”。正義曰:若有功,必順前世有功者也。上文所云“相夏”、“相殷”,謂\CJKunderline{禹}湯之功,故知此“順行\CJKunderline{禹}湯所有成功”。能順\CJKunderline{禹}湯之功,則惟王居位在德之首。\CJKunderline{禹}湯為有德之首,故王亦為首。 \par}

{\noindent\zhuan\zihao{6}\fzbyks 傳“王在”至“光明”。正義曰:\CJKunderwave{詩}稱“民之秉彝,好是懿德”,故王在德元,則小民乃惟法則於王,行王政於天下。王之為政,民盡行之,是言治政於王道有光明也。 \par}

{\noindent\shu\zihao{5}\fzkt “王乃”至“王顯”。正義曰:既言當法則賢王,又戒王為政之要。王乃初始即政,服行教化。嗚呼!王行教化當如初生之子。子之善惡無不在其初生,若習行善道,此乃自遺智命。“智命”謂身有賢智,命由己來,是自遺也。為政之道亦猶是矣。為政初則能善,天必遺王多福,使王有智則常吉,歷年長久也。今天觀人所為以授之命,其命有智與愚也,其命吉與兇也,其命歷年與不長也。若能敬德,則有智常吉,歷年長久也。若不敬德,則愚兇不長也。天已知我王今初始服政,居此新邑,觀王善惡,欲授之命,故惟王其當疾行敬德。“王其德之用”,言為行當用德,則能求天長命以歷年也。其惟王勿妄役小人過用非常之事,亦當果敢絕刑戮之道,以治下民順行\CJKunderline{禹}湯所有成功,則惟王居天子之位,在德行之首矣。王能如是,小民乃惟法則於王,行用王德於天下,如是則於王道亦有光明也。 \par}

上下勤恤,其曰,我受天命,丕若有夏曆年,式勿替有殷歷年。\footnote{言當君臣勤憂敬德,曰,我受天命,大順有夏之多歷年,勿用廢有殷歷年,庶幾兼之。}欲王以小民受天永命。”\footnote{我欲王用小民受天長命。言常有民。}拜手稽首曰:“予小臣,敢以王之讎民百君子,\footnote{拜手,首至手。稽首,首至地。盡禮致敬,以入其言。言我小臣,謙辭。敢以王之匹民百君子,治民者非一人,言民在下,自上匹之。○讎,字或作酬。}越友民,保受王威命明德。\footnote{言與匹民百君子,於友愛民者共安受王之威命,明德奉行之。}王末有成命,王亦顯。\footnote{臣下安受王命,則王終有天成命,於王亦昭著。}我非敢勤,惟恭奉幣,用供王能祈天永命。”\footnote{言我非敢獨勤而已,惟恭敬奉其幣帛,用供待王,能求天長命。將以慶王多福,必上下勤恤,乃與小民受天永命。○奉如字,又芳孔反。供音恭,徐紀用反,注“供待”同。}


{\noindent\zhuan\zihao{6}\fzbyks 傳“言當”至“兼之”。正義曰:王者不獨治,必當以臣助之。上句惟指勸王,故此又言臣助君。“上下”謂君臣,故言當君臣共勤憂敬德,不獨使王勤也。我周家承夏殷之後受天明命,欲其年過二代,既言大順有夏曆年,又言勿廢有殷歷年,庶幾兼彼二代,歷年長久。勤行敬德,即是大順勿廢也。 \par}

{\noindent\zhuan\zihao{6}\fzbyks 傳“拜手”至“匹之”。正義曰:“拜手”,頭至手。“稽首”,頭至地。謂既為拜,當頭至手,又申頭以至地,故“拜手稽首”重言之。諸言“拜手稽首”者,義皆然也。就此文詳而解之。\CJKunderwave{周禮·太祝}“辨九拜,一曰稽首”,施之於極尊。召公為此拜者,恐王忽而不聽,盡禮致敬以入其言於王。此“拜手稽首”一句,史錄其事,非召公語也。召公設言未盡,為此拜乃更言。鄭云:“拜手稽首者,召公既拜,興曰‘我小臣’以下,言召公拜訖而復言也。”王肅云:“我小臣,召公自謂是小臣,為召公之謙辭。讎訓為匹,敢以王之匹民百君子。百者舉其成數,言治民者非一人。”\CJKunderline{鄭玄}云:“王之諸侯與群吏,是非一人也。”嫌“匹”為齊等,故云“民在下,自上匹之”。 \par}

{\noindent\zhuan\zihao{6}\fzbyks 傳“言我”至“永命”。正義曰:“我非敢勤”,召公自道,言我非敢獨勤而已。“必上下勤恤”,言與眾百君子皆勤也。禮執贄必用幣帛,惟恭敬奉其幣帛,用供待王,能求天長命。將以執贄慶王多福,毛能愛養小民,即是求天長命,待王能愛小民,即欲慶之。 \par}

{\noindent\shu\zihao{5}\fzkt “上下”至“永命”。正義曰:上既勸王敬德,又言臣當助君。言君臣上下勤憂敬德,所以勤者,其言曰:“我周家既受天命,當大順有夏之多歷年歲,用勿廢有殷之多歷年歲。夏殷勤行敬德,故多歷年長久。我君臣亦當行敬德,庶幾兼之。如此者,我欲令王用小民受天長命。”言愛下民,則歷年多也。召公既言此,乃拜手稽首,盡禮致敬,欲王納用其言。既拜而又曰:“我小臣,敢以王之匹配於民眾百君子於友愛民者,共安受王之威命明德,敬奉行之,是上勤恤也。臣下安受王命,則王終有天之成命,於王亦為昭著也。我非敢獨勤而已,眾百君子皆然,言我與眾百君子惟恭敬奉其幣帛,用供待王,能求天長命,將以此慶王受天多福也。” \par}

\section{洛誥第十五}


召公既相宅,周公往營成周,使來告卜,\footnote{召公先相宅卜之,周公自後至,經營作之,遣使以所卜吉兆逆告成王。○相,息亮反,注及下同。使,所吏反,注“遣使”同。}作\CJKunderwave{洛誥}。

洛誥\footnote{既成洛邑,將致政成王,告以居洛之義。}


{\noindent\zhuan\zihao{6}\fzbyks 傳“召公”至“成王”。正義曰:上篇云:“三月戊申,太保朝至於洛,卜宅。厥既得卜,則經營。”是召公先相宅則卜之。又云:“乙卯,周公朝至於洛,則達觀於新邑營。”是周公自後至,經營作之。召公相洛邑,亦相成周,周公營成周,亦營洛邑,各舉其一,互以相明。“卜”者,召公卜也。周公既至洛邑,案行所營之處,遣使以所卜吉兆逆告成王也。案上篇傳雲“王與周公俱至”,何得周公至洛逆告王者?王與周公雖相與俱行,欲至洛之時,必周公先到行處所,故得逆告也。顧氏雲“周公既至洛邑,乃遣以所卜吉兆來告於王”是也。經稱成王言:“公既定宅,伻來,來視予卜休恆吉。”是以得吉兆告成王也。上篇召公以戊申至,周公乙卯至,周公在召公後七日也。至洛較七日,其發鎬京或亦較七日。 \par}

{\noindent\zhuan\zihao{6}\fzbyks 傳“既成”至“之義”。正義曰:周公攝政七年三月經營洛邑,既成洛邑,又歸向西都,其年冬將致政成王,告以居洛之義,故名之曰\CJKunderwave{洛誥},言以居洛之事告王也。篇末乃雲“戊辰,王在新邑”,明戊辰已上皆是西都時所誥也。 \par}

{\noindent\shu\zihao{5}\fzkt “召公”至“洛誥”。正義曰:序自上下相顧為文,上篇序雲“召公先相宅”,此承其下,故云“召公既相宅”。召公以三月戊申相宅而卜,周公自後而往,以乙卯日至,經營成周之邑。周公即遣使人來告成王以召公所卜之吉兆。及周公將欲歸於成王,乃陳本營洛邑之事,以告成王。王因請教誨之言,周公與王更相報答。史敘其事,作\CJKunderwave{洛誥}。史錄此篇,錄周公與王相對之言,以為後法,非獨相宅告卜而已。但周公因致政本說往前告卜,經文既具,故序略其事,直舉其發言之端耳。 \par}

周公拜手稽首曰:“朕復子明辟。”\footnote{周公盡禮致敬,言我復還明君之政於子。子,成王。年二十成人,故必歸政而退老。○闢,必亦反。}王如弗敢及天基命定命,\footnote{如,往也。言王往日幼少,不敢及知天始命周家安定天下之命,故己攝。}予乃胤保,大相東土,其基作民明闢。\footnote{我乃繼文武安天下之道,大相洛邑,其始為民明君之治。}


{\noindent\zhuan\zihao{6}\fzbyks 傳“周公”至“退老”。正義曰:周公還政而已,明暗在於人君,而云“復還明君之政”者,其意欲令王明,故稱“復子明辟”也。正以此年還政者,以成王年已二十成人,故必歸政而退老也。傳說成王之年,惟此而已。王肅於\CJKunderwave{金縢}篇末云:“武王年九十三而已,冬十一月崩。其明年稱元年,周公攝政,遭流言,作\CJKunderwave{大誥}而東征。二年克殷,殺管叔。三年歸,制禮作樂,出入四年,六年而成。七年營洛邑,作\CJKunderwave{康誥}、\CJKunderwave{召誥}、\CJKunderwave{洛誥},致政成王。然則武王崩時,成王年已十三矣。周公攝政七年,成王適滿二十。”孔於此言成王年二十,則其義如王肅也。又\CJKunderwave{家語}云:“武王崩時,成王年十三。”是孔之所據也。 \par}

{\noindent\zhuan\zihao{6}\fzbyks 傳“如往”至“己攝”。正義曰:“如,往”,\CJKunderwave{釋詁}文。“及”訓與也,言王往日幼少,志意未成,不敢與知上天始命我周家安定天下之命,故己攝也。天命周家安定天下者,必令天下太平,乃為安定。成王幼少,未能使之安定,故不敢與知之,周公所以攝也。 \par}

{\noindent\zhuan\zihao{6}\fzbyks 傳“我乃”至“之治”。正義曰:“胤”訓繼也,文王受命,武王伐紂,意在安定天下。天下未得安定,故周公言我乃繼續文武安定天下之道,大相洛邑之地,其處可行教化,始營此都,為民明君之政治。言欲為民明君,其意當在此。 \par}

{\noindent\shu\zihao{5}\fzkt “周公”至“民明闢”。正義曰:周公將反歸政,陳成王將居其位,周公拜手稽首,盡禮致敬於王,既拜乃興而言曰,我今復還子明君之政。言王往日幼少,其志意未成,不敢及知天之始命我周家安定天下之命,故我攝王之位,代王為治。我乃繼文王、武王安定天下之道,以此故大視東土洛邑之居,其始欲王居之,為民明君之治。言欲為民明君,必當治土中,故為王營洛邑也。 \par}

予惟乙卯,朝至於洛師。\footnote{致政在冬,本其春來至洛眾,說始卜定都之意。}我卜河朔黎水,我乃卜澗水東、瀍水西,惟洛食。\footnote{我使人卜河北黎水上,不吉。又卜澗瀍之間,南近洛,吉。今河南城也。卜必先墨畫龜,然後灼之,兆順食墨。○河朔,朔北也。瀍,直連反。近,附近之近。}我又卜瀍水東,亦惟洛食。伻來以圖及獻卜。”\footnote{今洛陽也。將定下都,遷殷頑民,故並卜之。遣使以所卜地圖及獻所卜吉兆,來告成王。○伻,普耕反,徐敷耕反,又甫耕反,下同。}


{\noindent\zhuan\zihao{6}\fzbyks 傳“致政”至“之意”。正義曰:下文總結周公攝政之事雲“在十有二月”,是“致政在冬”也。“在冬”,發言嫌此事是冬,故辨之雲“本其春來至洛眾”,追說始卜定都之意也。周公至洛之時,庶殷已集於洛邑,故云“至於洛師”。 \par}

{\noindent\zhuan\zihao{6}\fzbyks 傳“我使”至“食墨”。正義曰:嫌周公自卜,故云“我使人”,謂使召公也。案上篇召公至洛,其日即卜,而得“卜河朔黎水”者,以地合龜,非就地內,此言所卜三處皆一時事也。“黎水”之下不言吉凶者,“我乃”是改卜之辭,明其不吉乃改,故知“卜河北黎水之上,不吉”也。武王定鼎於郟鄏,已有遷都之意,而先卜黎水上者,以帝王所都,不常厥邑,夏殷皆在河北,所以博求吉地,故令先卜河北,不吉乃卜河南也。其“卜澗瀍之間,南近洛,吉。今河南城也”,基趾仍在,可驗而知。所卜黎水之上,其處不可知矣。凡卜之者,必先以墨畫龜,要坼依此墨,然後灼之求其兆,順食此墨畫之處,故云“惟洛食。”顏氏云:“先卜河北黎水者,近於紂都,為其懷土重遷,故先卜近以悅之。”用鄭康成之說,義或然也。 \par}

{\noindent\zhuan\zihao{6}\fzbyks 傳“今洛”至“成王”。正義曰:洛陽即成周,敬王自王城遷而都之。\CJKunderwave{春秋}昭三十二年“城成周”是也。周公慮此頑民未從周化,故既營洛邑,將定下都,以遷殷之頑民,故命召公即並卜之。周公既至,即遣使以所卜地圖及獻所卜吉兆,來告於成王。言己重其事,並獻卜兆者,使王觀兆知其審吉也。 \par}

{\noindent\shu\zihao{5}\fzkt “予惟”至“獻卜”。正義曰:周公追述立東都之事,我惟以七年三月乙卯之日,朝至於洛邑眾作之處,經營此都。其未往之前,我使人卜河北黎水之上,不得吉兆。乃卜澗水東、瀍水西,惟近洛,而其兆得吉,依規食墨。我亦使人卜瀍水東,亦惟近洛,其兆亦吉,依規食墨。我以乙卯至洛,我即使人來以所卜地圖及獻所卜吉兆於王。言卜吉立此都,王宜居之為治也。 \par}

王拜手稽首曰:“公不敢不敬天之休,來相宅,其作周匹休。\footnote{成王尊敬周公,答其拜手稽首而受其言。述而美之,言公不敢不敬天之美,來相宅,其作周以配天之美。}公既定宅,伻來,來視予卜休恆吉。我二人共貞。\footnote{言公前已定宅,遣使來,來視我以所卜之美、常吉之居,我與公共正其美。○貞,正也,馬云:“當也。”}公其以予萬億年敬天之休。”\footnote{公其當用我萬億年敬天之美。十千為萬,十萬為億,言久遠。}拜手稽首誨言。\footnote{成王盡禮致敬於周公,求教誨之言。○盡,子忍反。}


{\noindent\zhuan\zihao{6}\fzbyks 傳“成王”至“之美”。正義曰:拜手稽首,施於極敬。哀十七年\CJKunderwave{左傳}云:“非天子,寡君無所稽首。”諸侯小事大尚不稽首,況於臣乎?成王尊敬周公,答言其拜手稽首而受其言。又述而美之,天命文武使王天下,是天之美事,言公不敢不敬天之美,來相洛邑之宅。 \par}

{\noindent\zhuan\zihao{6}\fzbyks 傳“言公”至“其美”。正義曰:周公追述往前遣使獻卜,故成王複述公言。言公前已定宅,遣使來,來視我所卜之吉兆、常吉之居。自言前已知其卜,既有此美,我當與公二人共正其美意。欲留公輔己,共公正此美事。“來來”重文者,上“來”言使來,下“來”為視我卜也。鄭云:“伻來來者,使二人也。”與孔意異。 \par}

{\noindent\zhuan\zihao{6}\fzbyks 傳“公其”至“久遠”。正義曰:言居洛為治,可以永久,公意其當用我使萬億年敬天之美,言公欲令己祚胤久遠,美公意之深也。\CJKunderwave{王制}云:“方百里者,為方十里者百,為田九十億畝。”方里者萬,則是為田九百萬畝。今\CJKunderwave{記}乃雲“九十億畝”,是名十萬為億也。\CJKunderwave{楚語}雲“百姓、千品、萬官、億醜”,每數相十,是古十萬曰億。今之算術乃萬萬為億也。 \par}

{\noindent\zhuan\zihao{6}\fzbyks 傳“成王”至“之言”。正義曰:此一段史官所錄,非王言也。王求教誨之言,必有求教誨之辭,史略取其意,故直雲“誨言”。為求誨言而拜,故言“成王盡禮致敬於周公,求教誨之言”也。 \par}

{\noindent\shu\zihao{5}\fzkt “王拜手”至“誨言”。正義曰:成王尊敬周公,故亦盡禮致敬,拜手稽首,乃受公之語,述公之美曰:“不敢不敬天之美,來至洛相宅,其意欲作周家配天之美故也。公既定洛邑,即使人來告,亦來視我以所卜之美、常吉之居,我當與公二人共正其美。公定其宅,其當用我萬億年敬天之美故也。”王既言此,又拜手稽首於周公,求教誨之言。 \par}

周公曰:“王肇稱殷禮,祀於新邑,咸秩無文。\footnote{言王當始舉殷家祭祀,以禮典祀於新邑,皆次秩不在禮文者而祀之。}予齊百工,伻從王於周。予惟曰:‘庶有事。’\footnote{我整齊百官,使從王於周,行其禮典。我惟曰:‘庶幾有善政事。’}今王即命曰:‘記功,宗以功,作元祀。’\footnote{今王就行王命於洛邑,曰:“當記人之功,尊人亦當用功大小為序,有大功則列大祀。”謂功施於民者。○曰記,上音越,一音人實反。}


{\noindent\zhuan\zihao{6}\fzbyks 傳“言王”至“祀之”。正義曰:於時制禮已訖,而云“殷禮”者,此“殷禮”即周公所制禮也。雖有損益,以其從殷而來,故稱“殷禮”。猶上篇雲“庶殷”,“本其所由來”,孔於上傳已具,故於此不言。必知殷禮即周禮者,以此雲“祀於新邑”,即下文“烝祭歲”也,既用“騂牛”,明用周禮。雲“始”者,謂於新邑始為此祭。顧氏云:“舉行殷家舊祭祀,用周之常法。”言周禮即殷家之舊禮也。\CJKunderline{鄭玄}云:“王者未制禮樂,恆用先王之禮樂。”是言伐紂以來,皆用殷之禮樂,非始成王用之也。周公制禮樂既成,不使成王即用周禮,仍令用殷禮者,欲待明年即取,告神受職,然後班行周禮。班訖始得用周禮,故告神且印用殷禮也。孔義或然,故復存之。神數多而禮文少,應祭之神名有不在禮文者,故令皆次秩不在禮文而應祀者,皆舉而祀之。 \par}

{\noindent\zhuan\zihao{6}\fzbyks 傳“我整”至“政事”。正義曰:時成王未有留公之意,公以成王初始即政,自慮百官不齊,故雖即致政,猶欲整齊百官,使從王於周,謂從至新邑,行其典禮。周公以成王賢君,今覆成長,故言“我惟曰:‘庶幾有善政事。’”言己私為此言,冀王為政善也。 \par}

{\noindent\zhuan\zihao{6}\fzbyks 傳“今王”至“民者”。正義曰:記臣功者是人主之事,故言“今王就行王命於洛邑”,謂正位為王,臨察臣下,知其有功以否。恐王輕忽此事,故曰“當記人之功”。更言“曰”者,所以致殷勤也。尊人必當用功大小為次序,令功大者居上位,功小者處下位也。“有大功則列為大祀”,謂有殊功,堪載祀典者。\CJKunderwave{祭法}云:“聖王之制祭祀也,法施於民則祀之,以死勤事則祀之,以勞定國則祀之,能御大災則祀之,能捍大患則祀之。”是為大祀“謂功施於民”者也。或時立其祀配享廟廷,亦是也。 \par}

惟命曰:‘汝受命篤,弼丕視功載,乃汝其悉自教工。’\footnote{惟天命我周邦,汝受天命厚矣,當輔大天命,視群臣有功者記載之,乃汝新即政,其當儘自教眾官,躬化之。}孺子其朋,孺子其朋,其往。\footnote{少子慎其朋黨,少子慎朋黨,戒其自今已往。}無若火始焰焰,厥攸灼敘,弗其絕。\footnote{言朋黨敗俗,所宜禁絕。無令若火始然,焰焰尚微,其所及,灼然有次序,不其絕。事從微至著,防之宜以初。○焰音豔。“敘”絕句,馬讀“敘”句字屬下。令,力呈反。}


{\noindent\zhuan\zihao{6}\fzbyks 傳“惟天”至“化之”。正義曰:“惟天命我周邦”,謂天命我文武,故及汝成王復受天命為天子,是天之恩德深厚矣。天以厚德被汝,汝當輔大天命,任賢使能,行合天意,是輔大天也。汝當輔大天命,故宜視群臣有功者記載之,覆上“記功,宗以功”言之也。欲令群臣有功,必須躬自教化之在於初始,故言“乃汝新即政,其當儘自教眾官”。欲令王“躬化之”者,正己之身,使群臣法之,非謂以辭化之也。言“儘自教”者,政有大小,恐王輕大略小,令王儘自親化之。言“惟命曰”,亦是致殷勤。“乃”者,緩辭也。義異上句,故言“乃”耳。王肅云:“此其儘自教百官,謂正身以先之。” \par}

{\noindent\zhuan\zihao{6}\fzbyks 傳“少子”至“已往”。正義曰:鄭云:“孺子,幼少之稱,謂成王也。”此上皆雲成王,此句特言少子者,以明朋黨敗俗,為害尢大,恐年少所忽,故特言“孺子”也。“朋黨”謂臣相朋黨。“慎其朋黨”,令禁絕之。“戒其自今已往”,謂從即政以後,常以此事為戒也。 \par}

{\noindent\zhuan\zihao{6}\fzbyks 傳“言朋”至“以初”。正義曰:“無令若火始然”,以喻無令朋黨始發。若火既然,初雖焰焰尚微,其火所及,灼然有次序,不其復可絕也。以喻朋黨若起,漸漸益大,群黨既成,不可復禁止也。“事從微至著,防之宜以初”,謂朋黨未發之前,防之使不發。 \par}

厥若彝,及撫事如予,惟以在周工。\footnote{其順常道,及撫國事,如我所為,惟用在周之百官。}往新邑,伻鄉即有僚,明作有功,惇大成裕,汝永有辭。”\footnote{往行政化於新邑,當使臣下各鄉就有官,明為有功,厚大成寬裕之德,則汝長有嘆譽之辭於後世。○向,許亮反,注同。惇,都混反。}

{\noindent\zhuan\zihao{6}\fzbyks 傳“其順”至“百官”。正義曰:考古依法,為“順常道”。號令治民,為“撫國事”。周公大聖,動成軌則,“如我所為”,謂如攝政之時事所施為也。惟當用我所為在周之百官,令其行周公之道法於百官也。 \par}

{\noindent\zhuan\zihao{6}\fzbyks 傳“往行”至“後世”。正義曰:此時在西都戒王,故云“往行政化於新邑”。當使臣下各鄉就所有之官,令其各守其職,思不出其位,自當陳力就列,明為有功。在官者當以褊小急躁為累,故令臣下厚大成寬裕之德。臣下既賢,君必明聖,則汝長有嘆譽之辭於後世矣。今\CJKunderwave{周頌}所歌即嘆譽成王之辭也。 \par}

{\noindent\shu\zihao{5}\fzkt “周公”至“有辭”。正義曰:王求教誨之言,公乃誨之。周公曰:“王居此洛邑,當始舉殷家祭祀以為禮典,祀於洛之新邑,皆次秩在禮無文法應祀者,亦次秩而祀之。我雖致政,為王整齊百官,使從王於周,行其禮典。若能如此,我惟曰:‘庶幾有善政事。’今王就行王命於洛邑,曰:‘王當記人之功,尊人亦當用功大小為次序,有大功者則列為大祀。’”又申述所以祀神記臣功者。“政事由臣而立,惟天命我周邦之故,曰:‘汝受天命厚矣,當輔大天命,故須視群臣有功者記載之。君知臣功,則臣皆盡力。欲令群臣盡力,宜於初即教之。乃汝新始即政,其當儘自教誨眾官。’”令王躬自化之,使之立功。又以朋黨害政,尤宜禁絕,故丁寧戒之:“少子慎其朋黨,少子慎其朋黨,戒其自今已往。”令常慎此朋黨之事。“若欲絕止,禁其未犯,無令若火始然。焰焰尚微,火既然焰,其火所及,將灼然有次序矣,不其復可絕也。汝成王其當順此常道,及撫循國事,如我攝政所為。惟當用我此事,在周之百官則當畏服,各立功矣。汝當以此往行政化於新邑,當使臣下百官各向就有官,明為有功,厚大成寬裕之德,則汝長有嘆譽之辭於後世”。此周公誨王之言也。 \par}

公曰:“已!汝惟衝子惟終。\footnote{已乎!汝惟童子,嗣父祖之位,惟當終其美業。}汝其敬識百辟享,亦識其有不享。享多儀,儀不及物,惟曰不享。\footnote{奉上謂之享。言汝為王,其當敬識百君諸侯之奉上者,亦識其有違上者。奉上之道多威儀,威儀不及禮物,惟曰不奉上。}惟不役志於享,凡民惟曰不享,惟事其爽侮。\footnote{言人君惟不役志於奉上,則凡人化之,惟曰不奉上矣。如此則惟政事其差錯侮慢不可治理。}


{\noindent\zhuan\zihao{6}\fzbyks 傳“已乎”至“美業”。正義曰:周公止而復言,故更言“公曰”。“已乎”者,道前言已如是矣,為後言發端也。“童子”者,言其年幼而任重。“嗣父祖之位,當終其美業”,能致太平,是終之也。 \par}

{\noindent\zhuan\zihao{6}\fzbyks 傳“奉上”至“奉上”。正義曰:“享”訓獻也,獻是奉上之辭,故“奉上謂之享”。百官諸侯上事天子,凡所恭承皆是奉上,非獨朝覲貢獻乃為奉上。\CJKunderline{鄭玄}專以朝聘說之,理未盡也。言汝為王當敬識百官諸侯之奉上者,亦識其有違上者,察其恭承王命如法以否,奉上違上皆須記之。奉上者當以禮接之,違上者當以刑威之,所謂賞慶刑威。為君之道,奉上之道,其事非一,故云“多威儀”。威儀既多,皆須合禮,其威儀不及禮物,惟曰不奉上矣。謂旁人觀之,亦言其不奉上也。鄭云:“朝聘之禮至大,其禮之儀不及物,所謂貢篚多而威儀簡也。威儀既簡,亦是不享也。” \par}

{\noindent\shu\zihao{5}\fzkt “公曰”至“爽侮”。正義曰:周公復誨王曰:“嗚呼!前言已如是。”更復教誨:“汝惟童子,嗣父祖之位,惟當終其美業。天子居百官諸侯之上,須知臣下恭之與慢。奉上謂之享,汝為天子,其當恭敬記識百官諸侯奉上者,亦當記識其有不奉上者。奉上之道多威儀,威儀不及禮物,則人惟曰不奉上之道矣。所以須記之者,百官諸侯為下民之君,惟為政教不肯役用其志於此奉上之事,則凡民化之,亦惟曰不奉上矣。百官不承天子,民復不奉百官,上下不相畏敬,惟政事其皆差錯侮慢不可治理矣。故天子須知百官奉上與否也。” \par}

乃惟孺子,頒朕不暇,聽朕教汝於棐民彝。\footnote{我為政常若不暇,汝為小子當分取我之不暇而行之,聽我教汝於輔民之常而用之。○頒音班,徐甫雲反,馬云:“猶也。”棐音匪,又芳鬼反。}汝乃是不蘉,乃時惟不永哉!\footnote{汝乃是不勉為政,汝是惟不可長哉!欲其必勉為可長。○蘉,徐莫剛反,又武剛反,馬云:“勉也。”}

{\noindent\zhuan\zihao{6}\fzbyks 傳“我為”至“用之”。正義曰:“為政常若不暇”,謂居攝時也。聖人為政,務在化人,雖復治致太平,猶恨意之不盡,故謙言己所不暇,若言猶有美事未得施者然。故戒之成王:“汝惟小子,當分取我之不暇而行之。”言己所不暇行者,欲令成王勉行之。\CJKunderline{鄭玄}云:“成王之才,周公倍之,猶未而言分者,誘掖之言也。”生民之為業,雖復志有經營,不能獨自成就,須王者設教以輔助之。“聽我教汝輔民之常法而用之”,謂用善政以安民。\CJKunderwave{說文}云:“頒,分也。” \par}

{\noindent\zhuan\zihao{6}\fzbyks 傳“汝乃”至“可長”。正義曰:成王言“公其以予萬億年”,言欲己長久也,故周公於此戒之:“汝乃於是不勉力為政,汝惟不可長哉!”欲其必勉力勤行政教,為可長久之道,然後可至萬億年耳。“醟”之為勉,相傳訓也。鄭、王皆以為勉。 \par}

篤敘乃正父,罔不若予,不敢廢乃命。\footnote{厚次序汝正父之道而行之,無不順我所為,則天下不敢棄汝命,常奉之。}汝往敬哉!茲予其明農哉!彼裕我民,無遠用戾。”\footnote{汝往居新邑,敬行教化哉!如此我其退老,明教農人以義哉!彼天下被寬裕之政,則我民無遠用來。言皆來。○被,皮寄反,又彼美反。}

{\noindent\zhuan\zihao{6}\fzbyks 傳“厚次”至“奉之”。正義曰:“正父”謂武王,言其德正,故稱“正父”。“厚次序汝正父之道而行之”,令其為武王之政也。武王、周公俱是大聖,無不順我所為。又令法周公之道,既言法武王,又法周公,則天下不敢棄汝命,常奉行之。 \par}

{\noindent\zhuan\zihao{6}\fzbyks 傳“汝往”至“皆來”。正義曰:歸其王政令,汝往居新邑,敬行教化哉!公既歸政,則身當無事,如此我其退老於州里,明教農人以義哉!又令成王行寬裕之政,以治下民。民被寬裕之政,則天下之民無問遠近者,用來歸王,言遠處皆來也。上文使之“惇大成裕”,故此言裕政來民結上事也。伏生\CJKunderwave{書傳}稱,禮致仕之臣,教於州里,大夫為父師,士為少師,朝夕坐於門塾,而教出入之子弟。是“教農人以義”也。 \par}

{\noindent\shu\zihao{5}\fzkt “乃惟”至“用戾”。正義曰:又曰:“己居攝之時,為政常若不暇,汝惟小子,當分取我之不暇而施行之。又聽我教汝於輔民之常而用之。汝乃於是事不勉力為政,則汝是惟不可長久哉!必須勉力為之,乃可長久。此所言皆是汝父所行,汝欲勉之,但厚次序汝正父之道而行之,無不順我所為,則天下不敢棄廢汝命,必常奉而行之。汝往居新邑,敬行教化哉!如此我其退老,明教農人以義哉!汝若能使彼天下之民被寬裕之政,則我天下之民無問遠近者,悉皆用來歸汝矣。” \par}

王若曰:“公,明保予衝子。\footnote{成王順周公意,請留之自輔。言公當明安我童子,不可去之。}公稱丕顯德,以予小子揚文武烈,\footnote{言公當留,舉大明德,用我小子襃揚文武之業而奉順天。○襃,薄謀反,\CJKunderwave{切韻}博毛反。}奉答天命,和恆四方民,居師。\footnote{又當奉當天命,以和常四方之民,居處其眾。}惇宗將禮,稱秩元祀,咸秩無文。\footnote{厚尊大禮,舉秩大祀,皆次秩無禮文而宜在祀典者,凡此待公而行。}


{\noindent\zhuan\zihao{6}\fzbyks 傳“成王”至“去之”。正義曰:成王以周公誨己為善,順周公之意,示己欲行善政,而請留之自輔。王以公若舍我而去,則已政暗而治危,故云“公當明安我童子,不可去”也。 \par}

{\noindent\zhuan\zihao{6}\fzbyks 傳“言公”至“順天”。正義曰:文武受命,功德盛隆,成王自量己身不能繼業,言公當留,舉大明德,以佑助我。“用我小子褒揚文武之業而奉順天”者,下句“奉答天命”是也。孔分經為傳,故探取下句以申之。 \par}

{\noindent\zhuan\zihao{6}\fzbyks 傳“又當”至“其眾”。正義曰:天命周家,欲令民治,故“又當奉當天命,以和常四方之民,居處其眾”也。“奉當”者,尊天意,使允當天心,和協民心,使常行善也。“居處其眾”,使之安土樂業也。 \par}

{\noindent\zhuan\zihao{6}\fzbyks 傳“厚尊”至“而行”。正義曰:\CJKunderwave{釋詁}云:“將,大也。”“厚尊大禮”,謂祭祀之禮。\CJKunderwave{祭統}云:“禮有五經,莫重於祭。”是祭禮最尊大。公誨成王,令“肇稱殷禮,祀於新邑,咸秩無文”,欲答公誨己之事,還述公辭:“舉秩大祀,皆次秩無禮文而宜在祀典者。”其祀事非我所為,凡此皆待公而行者也。言公不可舍我以去也。 \par}

惟公德明,光於上下,勤施於四方。\footnote{言公明德光於天地,勤政施於四海,萬邦四夷服仰公德而化之。}旁作穆穆迓衡,不迷文武勤教,\footnote{四方旁來為敬敬之道,以迎太平之政,不迷惑於文武所勤之教。言化治。○旁,步光反。迓,五嫁反,馬、鄭、王皆音魚據反。}予衝子夙夜毖祀。”\footnote{言政化由公而立,我童子徒早起夜寐,慎其祭祀而已。無所能。}

{\noindent\zhuan\zihao{6}\fzbyks 傳“言公”至“化之”。正義曰:此與下經皆追述居攝時事。\CJKunderwave{堯典}訓“光”為充,此“光”亦為充也。言公之明德充滿天地,即\CJKunderwave{堯典}“格於上下”,勤政施於四方,即\CJKunderwave{堯典}“光被四表”也,意言萬邦四夷皆服仰公德而化之。上言待公乃行之,此言公有是德,言其將來,說其已然,所以深美公也。 \par}

{\noindent\zhuan\zihao{6}\fzbyks 傳“四方”至“化洽”。正義曰:上言施化在公,此言民化公德,四方旁來為敬敬之道,民皆敬鄉公以迎太平之政。言“迎”者,公政從上而下,民皆自下迎之,言其慕化速也。文武勤行教化,欲以教訓利民,民蒙公化,識文武之心,不復迷惑文武所勤之教,言公居攝之時,政化已洽於民也。 \par}

{\noindent\zhuan\zihao{6}\fzbyks 傳“言政”至“所能”。正義曰:此述留公之意,陳自今已後之事。言公若留住,政化由公而立。我童子徒早起夜寐,慎其祭祀而已。於政事無所能,欲惟典祭祀,以政事委公。襄二十六年\CJKunderwave{左傳}云,衛獻公使與寧喜言曰:“苟得反國,政由寧氏,祭則寡人。”亦猶是也。 \par}

{\noindent\shu\zihao{5}\fzkt “王若”至“毖祀”。正義曰:王以周公將退,因誨之而請留公,王順周公之意而言曰:“公當留住而明安我童子,不可去也。所以不可去者,當舉行大明之德,用使我小子褒揚文武之業,而奉當天命,以和常四方之民,居處其眾故也。其厚尊大禮,謂舉秩大祀,皆次秩禮所無文者而皆祀之。凡此皆待公而行,非我能也。”更述居攝時事:“惟公明德光於天地,勤政施於四方,使四方旁來為敬敬之道,以迎太平之政,下民皆不復迷惑於文武所勤之教。”言公化洽使如此也。“今若留輔我童子,惟當早起夜寐,慎其祭祀而已”。言政化由公而立,我無所能也。 \par}

王曰:“公功棐迪篤,罔不若時。”\footnote{公之功輔道我已厚矣,天下無不順而是公之功。}

{\noindent\zhuan\zihao{6}\fzbyks 傳“公之”至“之功”。正義曰:王意言公之居攝,天下若為非,則可舍我而去。公之居攝,天下無不順而是公之功,不可舍我去。 \par}

{\noindent\shu\zihao{5}\fzkt “王曰公功”至“若時”。正義曰:王又重述前言,還說居攝時事也。曰:“公之功輔道我已厚矣,天下無有不順而是公之功者,公所以須留也。” \par}

王曰:“公,予小子其退即闢於周,命公後。\footnote{我小子退坐之後,便就君於周,命正公後,公當留佑我。}四方迪亂,未定於宗禮,亦未克敉公功。\footnote{言四方雖道治,猶未定於尊禮。禮未彰,是亦未能撫順公之大功。明不可以去。○敉,亡婢反。治,直吏反,下同。}迪將其後,監我士師工,\footnote{公留教道,將助我其今已後之政,監篤我政事眾官。委任之言。○監,工銜反,注同。}誕保文武受民亂,為四輔。”\footnote{太安文武所受之民治之,為我四維之輔。明當依倚公。}


{\noindent\zhuan\zihao{6}\fzbyks 傳“我小”至“佑我”。正義曰:“退”者,退朝也。周公於時令成王坐王位而以政歸之,成王順周公言受其政也。言我小子退坐之後,便就君位於周。“周”謂洛邑,許其從公言,適洛邑而行新政也。古者臣有大功,必封為國君,今周公將欲退老,故命立公後,使公子伯禽為國君,公當留佑我。王肅云:“成王前春亦俱至洛邑,是顧無事,既會而還宗周。周公往營成周,還來致政成王也。” \par}

{\noindent\zhuan\zihao{6}\fzbyks 傳“言四”至“以去”。正義曰:王意恐公意以四方既定,不須更留,故謂公云,四方雖已道治,而猶未能定於尊大之禮。言其禮樂未能彰明也。禮既未彰,是天下之民亦未能撫安順行公之大功,公當待其禮法明,公功順乃可去耳。明今不可以去。 \par}

{\noindent\zhuan\zihao{6}\fzbyks 傳“大安”至“倚公”。正義曰:文武受民之於天下,今大安文武所受之民,助我治之,為我四維之輔,明己當依倚公也。“維”者,為之綱紀,猶如用繩維持之。\CJKunderwave{文王世子}雲“設四輔”,謂設眾官為四方輔助。周公一人,事無不統,故一人為四輔。\CJKunderwave{管子}云:“四維不張,國乃滅亡。”傳取\CJKunderwave{管子}之意,故言“四維之輔”也。 \par}

{\noindent\shu\zihao{5}\fzkt “王曰公予”至“四輔”。正義曰:王呼周公曰:“我小子其退此坐,就為君於周。”謂順公之言,行天子之政於洛邑也。“至洛邑當命公後,立公之世子為國君,公當留輔我也。公之攝政,四方雖已道治理,猶自未能定於尊禮,是亦未能撫順公之大功。公當待其定大禮,順公之大功,此時未可去也。公當留教道,將助我其今已後之政,監篤我政事眾官,以此大安文武所受之民而治之,為我四維之輔助”。明己當依倚公也。 \par}

王曰:“公定,予往已。公功肅將祗歡,\footnote{公留以安定我,我從公言,往至洛邑已矣。公功以進大,天下咸敬樂公功。○樂音洛。}公無困哉!我惟無斁其康事,公勿替刑,四方其世享。”\footnote{公必留,無去以困我哉!我惟無厭其安天下事。公勿去以廢法,則四方其世世享公之德。○斁音亦。厭,於豔反。}


{\noindent\zhuan\zihao{6}\fzbyks 傳“公留”至“公功”。正義曰:讀文以“公定”為句,王稱“定”者,言定己也,故傳言“公留以安定我”,“我”字傳加之。“我從公言”,是經之“予”也。“往至洛邑已矣”,言已順從公命,受歸政也。“公功已進大,天下咸敬樂公之功”,亦謂居攝時也。\CJKunderwave{釋詁}云:“肅,進也。” \par}

{\noindent\zhuan\zihao{6}\fzbyks 傳“公必”至“之德”。正義曰:王言己才智淺短,公去則困,故請公“無去以困我哉”。我意欲致太平,惟無厭倦其安天下之事,是以留公,公勿去以廢治國之法,則天下四方之民蒙公之恩,其世世享公之德。“享”謂荷負之。 \par}

{\noindent\shu\zihao{5}\fzkt “王曰公定”至“世享”。正義曰:王又呼公:“公留以安定我,我從公言,往至洛邑已矣。公功已進且大矣,天下皆樂公之功,敬而歡樂。公必留,無去以困我哉!公留助我,我惟無厭其安天下之事。公勿去以廢法,則四方之民其世世享公之德矣。” \par}

周公拜手稽首曰:“王命予來,承保乃文祖受命民,\footnote{拜而後言,許成王留。言王命我來,承安汝文德之祖文王所受命之民,是所以不得去。}越乃光烈考武王,弘朕恭。\footnote{於汝大業之父武王,大使我恭奉其道。敘成王留己意。}孺子來相宅,其大惇典殷獻民,\footnote{少子今所以來相宅於洛邑,其大厚行典常於殷賢人。}亂為四方新闢,作周恭先。\footnote{言當治理天下,新其政化,為四方之新君,為周家見恭敬之王,後世所推先也。}


{\noindent\zhuan\zihao{6}\fzbyks 傳“拜而”至“得去”。正義曰:“拜”是從命之事,故云“拜而後言,許成王留”也。以“退”為去,以“留”為來,故言“王令我來”,來居臣位,為太師也。“承安汝文德之祖文王所受命之民”,天命文王,使為民主,天以民命文王,故民是“文王所受命之民”。“承安”者,承文王之意,安定此民。言王之留己,乃為此事,其事既大,是所以不得去也。 \par}

{\noindent\zhuan\zihao{6}\fzbyks 傳“於汝”至“己意”。正義曰:於汝成王大功業之父武王,王意大使我恭奉其道,敘成王留已之意也。王於文王、武王,皆欲令周公奉其道,安其民,其意一也,周公分言之耳。承安其文王之民,恭奉其武王之道,互相通也。 \par}

{\noindent\zhuan\zihao{6}\fzbyks 傳“少子”至“賢人”。正義曰:“少子”者,呼成王之辭。言我“今所以來相宅於洛邑”者,欲令王居洛,其大厚行典常於殷賢人。而據洛為政,故言“來”。訓“典”為常,故連言“典常”,言其行常道也。周受於殷,故繼之於殷,人有賢性,故稱“賢人”。 \par}

{\noindent\zhuan\zihao{6}\fzbyks 傳“言當”至“推先”。正義曰:\CJKunderwave{易}稱“日新之謂盛德”,雖舊有美政,令王更復新之。言當治理天下,新其政化,為四方之新君,與後人為軌訓,為周家見恭敬之王,後世所推先也。謂周家後世子孫,有德之王被人恭敬推先王。戒成王使為善政,令后王崇重之。 \par}

曰,其自時中乂,萬邦咸休,惟王有成績。\footnote{曰,其當用是土中為治,使萬國皆被美德,如此惟王乃有成功。}予旦以多子越御事,篤前人成烈,答其師,作周孚先。\footnote{我旦以眾卿大夫於御治事之臣,厚率行先王成業,當其眾心,為周家立信者之所推先。}

{\noindent\zhuan\zihao{6}\fzbyks 傳“曰其”至“成功”。正義曰:重以誨王,成其上事,故言“曰”以起之。 \par}

{\noindent\zhuan\zihao{6}\fzbyks 傳“我旦”至“推先”。正義曰:“旦”是周公之名,故自稱“我旦”也。“子”者,有德之稱,大夫皆稱“子”,故以“多子”為眾卿大夫。同欲令成王行善政,為後世賢王所推先。公與群臣盡誠節,為後世賢臣所推先。故欲以眾卿大夫及於御治事之臣,深厚率行先王之業,使當其人眾之心,為周家後世賢臣立信者之所推先也。傳於此不言“後世”,從上省文也。於君言“見恭敬”,於臣言“立信”者,以君尊言人敬,臣卑言自立信,因其所宜以設文也。 \par}

{\noindent\shu\zihao{5}\fzkt “周公”至“孚先”。正義曰:周公拜手稽首,盡禮致敬,許王之留,乃興而為言曰:“王今命我來居臣位,承安汝文德之祖文王所受命之民,令我繼文祖大業,我所以不得去也。又於汝大業父武王,大使我恭奉其道,王意以禮留我,其事甚大,我所以為王留也。”公呼成王云:“小子今所以來相宅於洛邑者,欲其大厚行常道於殷賢人。王當治理天下,新其政化,為四方之新君,為周家後世見恭敬之王所推先也。”重誨王曰:“其當用是土中為治,使萬國皆被美德,如此惟王乃有成功也。”公自稱名曰:“若王居洛邑,則我旦以多眾君子卿大夫等及於御治事之臣,厚率行前人先王成業,使當其眾心,為周家後世人臣立信者之所推先。”言我留輔王,使君臣皆為後世所推先,期於上下俱顯也。 \par}

考朕昭子刑,乃單文祖德,伻來毖殷,乃命寧。\footnote{我所成明子法,乃盡文祖之德,謂典禮也。所以居土中,是文武使己來慎教殷民,乃見命而安之。○單音丹,馬丁但反,信也。}予以秬鬯二卣,曰明禋,拜手稽首,休享。\footnote{周公攝政七年致太平,以黑黍酒二器,明絜致敬,告文武以美享。既告而致政,成王留之。本說之。○秬音巨。鬯,敕亮反,香酒也。卣,由手反,又音由,中樽也。禋音因。}予不敢宿,則禋於文王武王。\footnote{言我見天下太平,則絜告文武,不經宿。}


{\noindent\zhuan\zihao{6}\fzbyks 傳“我所”至“安之”。正義曰:典禮治國,事資聖人,前聖後聖,其終一揆,故言所欲成明子之法,乃盡是汝祖文王之德也。“子”斥成王,下句並告文武,兼用武王可知。又述居洛邑之意,所以居土中者,是文武使己來居此地,周公自言非己意也。文武令我營此洛邑,欲使居土中,慎教殷民,乃是見命於文武而安殷民也。顧氏云:“文武使我來慎教殷民,我今受文武之命以安民也。” \par}

{\noindent\zhuan\zihao{6}\fzbyks 傳“周公”至“說之”。正義曰:\CJKunderwave{康誥}之作,事在七年,雲“四方民大和會”。“和會”即太平之驗,是“周公攝政七年致太平”也。\CJKunderwave{釋草}云:“秬,黑黍。”\CJKunderwave{釋器}云:“卣,中樽也。”以黑黍為酒,煮鬱金之草,築而和之,使芬香調暢,謂之“秬鬯”。鬯酒二器,明絜致敬,告文王武王以美享,謂以太平之美事享祭也。\CJKunderwave{國語}稱:“精意以享謂之禋。”\CJKunderwave{釋詁}云:“禋,敬也。”是明“禋”為“明絜致敬”也。太平是王之美事,故太平告廟是以美享祭也。公既告太平而致政成王,成王留之,故本而說之此事者,欲令成王重其事,厚行之。\CJKunderwave{周禮}鬱鬯之酒實之於彝,此言在卣者,\CJKunderwave{詩·大雅·江漢}及\CJKunderwave{文侯之命}皆言“秬鬯一卣,告於文人”,則未祭實之於卣,祭時實之於彝。彼“一卣”,此“二卣”者,此一告文王,一告武王;彼王賜臣使告其太祖,故惟一卣耳。此經“卣”下言“曰”者,說本盛酒於樽,乃為此辭,故言“曰”也。 \par}

{\noindent\zhuan\zihao{6}\fzbyks 傳“言我”至“經宿”。正義曰:此申述上“明禋”之事,言我見天下太平,則絜告文武,不敢經宿,示虔恭之意也。此三月營洛邑,民已和會,則三月之時已太平矣。既告而致政,則告在歲末,而云“不經宿”者,蓋周公營洛邑,至冬始成,得還鎬京,即告文武,是為“不經宿”也。且太平非一日之事,公雲“不經宿”者,示虔恭之意耳,未必旦見太平,即此日告也。\CJKunderline{鄭玄}以“文祖”為明堂,“曰明禋者,六典成祭於明堂,告五帝太皞之屬也”。既告明堂,則復禋於文武之廟,告成洛邑。 \par}

惠篤敘,無有遘自疾。萬年厭乃德,殷乃引考。\footnote{汝為政當順典常,厚行之使有次序,無有遇用患疾之道者,則天下萬年厭於汝德,殷乃長成為周。○遘,工豆反。厭,於豔反,注同,馬云:“厭,飲也。”徐於廉反。}王伻殷乃承敘萬年,其永觀朕子懷德。”\footnote{王使殷民上下相承有次序,則萬年之道,民其長觀我子孫而歸其德矣勉使終之。}

{\noindent\zhuan\zihao{6}\fzbyks 傳“汝為”至“為周”。正義曰:\CJKunderwave{釋言}云:“惠,順也。”此經述上“惇典”,故言“汝為政當順典常,厚行之使有次序”。\CJKunderwave{釋詁}云:“遘,遇也。”患疾之道謂虐政,使人患疾之。厚行典常,使有次序,則百官諸侯凡為政者皆無有遇用患疾之政以害下民,則經歷萬年厭飽於汝德,則殷國乃長成為周。 \par}

{\noindent\zhuan\zihao{6}\fzbyks 傳“王使”至“終之”。正義曰:上言天下民萬年厭飽王德,此教為王德,使萬年令民厭飽王德也。能使殷民上下有次序,則王德堪至萬年之道。王之子孫當行不怠,則民其長觀我子孫,知其有德,而歸其德矣,此則長成為周。勸勉王使終之。 \par}

{\noindent\shu\zihao{5}\fzkt “考朕”至“懷德”。正義曰:周公又說制禮授王,使王奉之。我所成明子之法,乃盡是汝文祖之德,言用文王之道制禮,其事大不可輕也。又言所以須善治殷獻民者,文武使己來居土中,慎教殷民,乃是見命於文武而安之故也。制典當待太平,我以時既太平,即以秬黍鬯酒,盛於二卣樽內,我言曰:“當以此酒須明絜致敬於文武,我則拜手稽首,告文武以美享。”告云:“今太平,即速告廟,我不敢經宿,則禋告文王武王以致太平之事。”汝王為政,當順典常厚行之,使有次序,則諸為政者無雲有遇用患疾之道苦毒下民,則天下萬年厭飽於汝王之德,殷乃長成為周。王使殷民上下相承有次序,則萬年之道,下民其長觀我子孫而歸其德矣。勸王使終之,皆是誨王之言也。 \par}

戊辰,王在新邑,\footnote{成王既受周公誥,遂就居洛邑,以十二月戊辰晦到。○王在新邑,馬孔絕句,鄭讀“王在新邑烝”。}烝祭歲,文王騂牛一,武王騂牛一。王命作冊,逸祝冊,惟告周公其後。\footnote{明月,夏之仲冬,始於新邑烝祭,故曰“烝祭歲”。古者褒德賞功,必於祭日,示不專也。特加文武各一牛,告曰尊周公,立其後為魯侯。}王賓,殺禋,咸格,王入太室祼。\footnote{王賓異周公,殺牲精意以享文武,皆至其廟親告也。太室,清廟。祼鬯告神。○王賓,絕句。殺禋,絕句,一讀連“咸格”絕句。太室,馬云:“廟中之夾室。”祼,官喚反。}


{\noindent\zhuan\zihao{6}\fzbyks 傳“成王”至“晦到”。正義曰:周公告成王令居洛邑為治,王既受周公之誥,遂東行就居洛邑,以十二月戊辰晦日到洛。指言“戊辰,王在新邑”,知其晦日始到者,此歲入戊午蔀五十六年,三月雲丙午朏,以算術計之,三月甲辰朔大,四月甲戌朔小,五月癸卯朔大,六月癸酉朔小,七月壬寅朔大,八月壬申朔小,九月辛丑朔大,又有閏九月辛未朔小,十月庚子朔大,十一月庚午朔小,十二月己亥朔大,計十二月三十日戊辰晦到洛也。 \par}

{\noindent\zhuan\zihao{6}\fzbyks 傳“明月”至“魯侯”。正義曰:下雲“在十有二月”者,周之十二月,建亥之月也。戊辰是其晦日,故明日即是夏之仲冬建子之月也。言“明月”者,此烝祭非朔日,故言“月”也。自作新邑已來,未嘗於此祭祀,此歲始於新邑烝祭,故曰“烝祭歲”也。\CJKunderwave{周禮·大司馬}仲冬教大閱,遂以享烝是也。王者冬祭,必用仲月,此是周之歲首,故言“歲”耳。王既戊辰晦到,又須戒日致齊,不得以朔日即祭之。\CJKunderwave{祭統}云:“古者明君爵有德而祿有功,必賜爵祿於太廟,示不敢專也。”故云“古者褒德賞功,必於祭日,示不專也”。因封之,特設祭烝之禮。宗廟用大牢,此文武皆言“牛一”,知於太牢之外特加一牛。告白文武之神,言為尊周公,立其後為魯侯。\CJKunderwave{魯頌}所云“王曰叔父,建爾元子,俾侯於魯”,是此時也。“王命作策”者,命有司作策書也。讀策告神謂之“祝”,“逸祝策”者,使史逸讀策書也。\CJKunderline{鄭玄}以“烝祭”上屬。“歲文王騂牛一”者,“歲”是成王元年,正月朔日,特告文武封周公也。案\CJKunderwave{周頌·烈文序}云:“成王即政,諸侯助祭。”鄭箋云:“新王即政,必以朝享之禮祭於祖考,告嗣位也。”則鄭意以朝享之後,特以二牛告文武,封周公之後,與孔義不同。 \par}

{\noindent\zhuan\zihao{6}\fzbyks 傳“王賓”至“告神”。正義曰:“王賓異周公”者,王尊周公為賓,異於其臣。王肅雲“成王尊周公,不敢臣之,以為賓,故封其子”是也。\CJKunderwave{周語}云:“精意以享謂之禋。”既殺二牲,精誠其意以享祭文武。“咸”,皆也。“格”,至也。“皆至其廟”,言王重其事,親告之也。“太室”,室之大者,故為清廟。廟有五室,中央曰“太室”。王肅云:“太室,清廟中央之室。”“清廟”,神之所在,故王入太室祼獻鬯酒以告神也。“祼”者,灌也。王以圭瓚酌鬱鬯之酒以獻屍,屍受祭而灌於地,因奠不飲謂之“祼”。\CJKunderwave{郊特牲}雲“既灌,然後迎牲”,則殺在祼後。此經先言“殺”,後言“祼”者,“殺”者、“咸格”錶王敬公之意,非行事之次也。其“王入太室祼”,乃是祭時行事耳。周人尚臭,祭禮以祼為重,故言王祼。其封伯禽,乃是祭之將末,非祼時也。\CJKunderwave{祭統}賜臣爵祿之法云:“祭之日,一獻,君降立於阼階之南,南向,所命者北面,史由君右,執策命之。”鄭云:“一獻,一酳屍也。”\CJKunderwave{禮}酳屍,屍獻而祭畢,是祭末乃命之,以祼為重,故特言之。 \par}

王命周公後,作冊逸誥。\footnote{王為冊書,使史逸誥伯禽封命之書,皆同在烝祭日,周公拜前,魯公拜後。}在十有二月,惟周公誕保文武受命,惟七年。\footnote{言周公攝政盡此十二月,大安文武受命之事,惟七年,天下太平。自“戊辰”已下,史所終述。○受命,絕句,馬同。惟七年,周公攝政七年,天下太平,馬同;鄭云:“文王武王受命及周公居攝皆七年。”}

{\noindent\zhuan\zihao{6}\fzbyks 傳“王為”至“拜後”。正義曰:“王為策書”,亦命有司為之也。上雲“作策”,作告神之策。此言“作策”,誥伯禽之策。祭於神謂之“祝”,於人謂之“誥”,故云“使史逸誥伯禽封命之書”。封康叔謂之\CJKunderwave{康誥},此命伯禽,當雲\CJKunderwave{伯禽之誥}。定四年\CJKunderwave{左傳}雲“命以伯禽”,即史逸所讀之策也。上言“逸祝策”,此“誥”下不言“策”者,“祝”是讀書之名,故上雲“祝策”;此“誥”是誥伯禽使知,雖復讀書以誥之,不得言“誥策”也。上告周公,其後已言告神封周公,嫌此“逸誥”以他日告之,故云“皆同在烝祭日”。以\CJKunderwave{祭統}言一獻命之,知此亦祭日也。文十三年\CJKunderwave{公羊傳}曰:“封魯公以為周公也,周公拜乎前,魯公拜乎後。曰生以養周公,死以為周公主。” \par}

{\noindent\zhuan\zihao{6}\fzbyks 傳“言周”至“終述”。正義曰:自“戊辰”已上,周公與成王相對語,未有致政年月,故史於此總結之。自“戊辰”已下非是王與周公之辭,故辨之雲“史所終述”也。 \par}

{\noindent\shu\zihao{5}\fzkt “戊辰”至“七年”。正義曰:自此以下,史終述之。周公歸政,成王既受言誥之,王即東行赴洛邑。其年十二月晦戊辰日,王在新邑。後月是夏之仲冬,為冬節烝祭,其月節是周之歲首,特異常祭,加文王騂牛一,武王騂牛一。王命有司作策書,乃使史官名逸者祝讀此策,惟告文武之神,言周公有功,宜立其後為國君也。其時王尊異周公,以為賓,殺牲享祭文王武王,皆親至其廟,王入廟之大室,行祼鬯之禮。言其尊異周公而禮敬深也。於此祭時,王命周公後,令作策書,使逸讀此策辭以告伯禽,言封之於魯,命為周公後也。又總述之,在十有二月,惟周公大安文武受命之事,於此時惟攝攻七年矣。 \par}

%%% Local Variables:
%%% mode: latex
%%% TeX-engine: xetex
%%% TeX-master: "../Main"
%%% End:
