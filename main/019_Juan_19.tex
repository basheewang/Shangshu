%% -*- coding: utf-8 -*-
%% Time-stamp: <Chen Wang: 2024-04-02 11:42:41>

% {\noindent \zhu \zihao{5} \fzbyks } -> 注 (△ ○)
% {\noindent \shu \zihao{5} \fzkt } -> 疏

\chapter{卷十九}


\section{康王之誥第二十五(顧命下)}


\CJKunderline{康王}既屍天子,\footnote{屍,主也,主天子之正號。○馬本此句上更有“\CJKunderline{成王}崩”三字。}遂誥諸侯,作\CJKunderwave{康王之誥}。\footnote{既受顧命,群臣陳戒,遂報誥之。因事曰遂。}

康王之誥\footnote{求諸侯之見匡弼。}

{\noindent\shu\zihao{5}\fzkt “\CJKunderline{康王}既”至“之誥”。正義曰:\CJKunderline{康王}既受顧命,主天子之位,群臣進戒於王,王遂報誥諸侯。史敘其事,作\CJKunderwave{康王之誥}。\CJKunderline{伏生}以此篇合於\CJKunderwave{顧命},共為一篇,後人知其不可,分而為二。馬、鄭、王本此篇自“高祖寡命”已上內於\CJKunderwave{顧命}之篇,“王若曰”已下始為\CJKunderwave{康王之誥},諸侯告王,王報誥諸侯,而使告報異篇,失其義也。 \par}

王出,在應門之內,\footnote{出畢門,立應門內之中庭,南面。}太保率西方諸侯,入應門左,\CJKunderline{畢公}率東方諸侯,入應門右,\footnote{二公為二伯,各率其所掌諸侯,隨其方為位,皆北面。}皆布乘黃朱。\footnote{諸侯皆陳四黃馬朱鬛以為庭實。○乘音繩證反。鬛,力輒反。}


{\noindent\zhuan\zihao{6}\fzbyks 傳“出畢”至“南面”。正義曰:出在門內,不言“王坐”,諸侯既拜,王即答拜,復不言“興”,知立庭中南面也。 \par}

{\noindent\zhuan\zihao{6}\fzbyks 傳“二公”至“北面”。正義曰:二公率領諸侯,知其“為二伯,各率其所掌諸侯”。\CJKunderwave{曲禮}所謂“職方”者,此之義也。\CJKunderline{王肅}云:“\CJKunderline{畢公}代\CJKunderline{周公}為東伯,故率東方諸侯。”然則\CJKunderline{畢公}是太師也。當太師之名,在太保之上,此先言太保者,於時太保領冢宰,相王室,任重,故先言西方。若使東伯任重,亦當先言東方。北面,以東為右,西為左,入左入右隨其方為位。嫌東西相向,故云“皆北面”,將拜王,明北面也。 \par}

{\noindent\zhuan\zihao{6}\fzbyks 傳“諸侯”至“庭實”。正義曰:諸侯朝見天子,必獻國之所有,以表忠敬之心,故“諸侯皆陳四黃馬朱鬛以為庭實”,言實之於王庭也。四馬曰“乘”,言“乘黃”正是馬色黃矣。“黃”下言“朱”,“朱”非馬色。定十年\CJKunderwave{左傳}云:“宋公子地有白馬四,公嬖向魋,魋欲之。公取而朱其尾、鬛以與之。”是古人貴朱鬛。知“朱”者,朱其尾、鬛也。於時諸侯其數必眾,眾國皆陳四馬,則非王庭所容。諸侯各有所獻,必當少陳之也。案\CJKunderwave{周禮·小行人}云:“合六幣,圭以馬,璋以皮,璧以帛,琮以錦,琥以繡,璜以黼。此六物者,以和諸侯之好。”\CJKunderline{鄭玄}云:“六幣所以享也,五等諸侯享天子用壁,享後用琮。用圭璋者,二王之後也。”如鄭彼言,則諸侯之享天子,惟二王之後用馬。此雲皆陳馬者,下云“奉圭兼幣”,幣即馬是也,圭是文馬之物。鄭云:“此幣圭以馬,蓋舉王者之後以言耳。諸侯當璧以帛,亦有庭實。”然則此陳馬者是二王之後享王物也。獨取此物以總表諸侯之意,故云諸侯皆陳馬也。圭亦享王之物,下言“奉圭”,此不陳圭者,圭奉以文命,不陳之也。案\CJKunderwave{覲禮}諸侯享天子“馬卓上,九馬隨之”。此用乘黃者,因喪禮而行朝,故略之。 \par}

賓稱奉圭兼幣,曰:“一二臣衛,敢執壤奠。”\footnote{賓,諸侯也。舉奉圭兼幣之辭,言“一二”,見非一也。為蕃衛,故曰“臣衛”。來朝而遇國喪,遂因見新王,敢執壤地所出而奠贄也。○壤,如丈反。見,賢遍反,下同。蕃,方袁反。朝,直遙反。喪,息浪反。贄音至。}皆再拜稽首。王義嗣德,答拜。\footnote{諸侯拜送幣而首至地,盡禮也。\CJKunderline{康王}以義繼先人明德,答其拜,受其幣。○盡,子忍反。}

{\noindent\zhuan\zihao{6}\fzbyks 傳“賓諸”至“奠贄”。正義曰:天子於諸侯有不純臣之義,故以諸侯為賓。“稱”訓舉也。“舉奉圭兼幣之辭”,以圭幣奉王而為之作辭。辭出一人之口而言“一二”者,見諸侯同為此意,意非一人也。\CJKunderline{鄭玄}云“釋辭者一人,其餘奠幣,拜者稽首而已”是也。言“衛”者,諸侯之在四方,皆為天子蕃衛,故曰“臣衛”。此時\CJKunderline{成王}始崩,即得有諸侯在京師者,來朝而遇國喪,遂因見新王也。諸侯享天子,其物甚眾,非徒圭馬而已。皆是土地所有,故云“敢執壤地所出而奠贄”也。然“舉奉圭兼幣”,乃是享禮。凡享禮,則每一國事畢,乃更餘國復入,其朝則侯氏總入。故\CJKunderline{鄭玄}注\CJKunderwave{曲禮}云:“春“受贄於朝,受享於廟”,是朝與享別。此既諸侯總入而得有庭實享禮者,以新朝嗣王,因行享禮,故鄭注云:“朝兼享禮也,與常禮不同。” \par}

{\noindent\zhuan\zihao{6}\fzbyks 傳“諸侯”至“其幣”。正義曰:\CJKunderwave{周禮·太祝}“辨九拜,一曰稽首”,施之於極尊,故為“盡禮”也。“義嗣德”三字,史言王答拜之意也。\CJKunderline{康王}先是太子,以義繼先人明德,今為天子,無所嫌,故答其拜,受其幣,自許與諸侯為主也。 \par}

{\noindent\shu\zihao{5}\fzkt “王出”至“答拜”。正義曰:此敘諸侯見新王之事。王出畢門,在應門之內,立於中庭。太保\CJKunderline{召公}為西伯,率西方諸侯,入應門左,立於門內之西廂也。太師\CJKunderline{畢公}為東伯,率東方諸侯,入應門右,立於門內之東廂也。諸侯皆布陳一乘四匹之黃馬朱鬛,以為見新王之庭實。諸侯為王之賓,實共使一人少前進,舉奉圭兼幣之辭,言曰:“一二天子之臣,在外為蕃衛者,敢執土壤所有奠之於庭。”既為此言,乃皆再拜稽首,用盡禮致敬,以正王為天子也。\CJKunderline{康王}先為太子,以義嗣先人明德,不以在喪為嫌,答諸侯之拜,以示受其圭幣,與之為主也。 \par}

太保暨\CJKunderline{芮伯}咸進,相揖,皆再拜稽首,\footnote{冢宰與司徒皆共群臣諸侯並進陳戒。不言諸侯,以內見外。}曰:“敢敬告天子,皇天改大邦殷之命,\footnote{大天改大國殷之王命,謂誅紂也。}惟周\CJKunderline{文}、\CJKunderline{武},誕受羑若,克恤西土。\footnote{言文武大受天道而順之,能憂我西土之民。本其所起。○羑,羊久反,馬云:“道也。”}惟新陟王,畢協賞罰,戡定厥功,用敷遺後人休。\footnote{惟周家新升王位,當盡和天下賞罰,能定其功,用布遺後人之美。言施及子孫無窮。○戡音堪。遺,唯季反,注及下同。施,以豉反。}今王敬之哉!\footnote{敬天道,務崇先人之美。}張皇六師,無壞我高祖寡命。”\footnote{言當張大六師之眾,無壞我高德之祖寡有之教命。○壞音怪。}


{\noindent\zhuan\zihao{6}\fzbyks 傳“冢宰”至“見外”。正義曰:\CJKunderline{召公}為冢宰,\CJKunderline{芮伯}為司徒,司徒位次冢宰,故言“太保與\CJKunderline{芮伯}咸進”。\CJKunderline{芮伯}已下,共告群臣諸侯並皆進也。“相揖”者,揖之使俱進也。大保揖群臣,群臣又報揖太保,故言“相揖”。動足然後相揖,故“相揖”之文在“咸進”之下。 \par}

{\noindent\zhuan\zihao{6}\fzbyks 傳“言文”至“所起”。正義曰:“羑”聲近猷,故訓之為道。\CJKunderline{王肅}云:“羑,道也。”文武所憂,非憂西土而已,特言“能憂西土之民”,本其初起於西土故也。 \par}

{\noindent\zhuan\zihao{6}\fzbyks 傳“言當”至“教命”。正義曰:“皇”訓大也。國之大事,在於強兵,故令張大六師之眾。“高德之祖”,謂\CJKunderline{文王}也。\CJKunderline{王肅}云:“美\CJKunderline{文王}少有及之,故曰‘寡有’也。” \par}

{\noindent\shu\zihao{5}\fzkt “太保”至“寡命”。正義曰:太保\CJKunderline{召公}與司徒\CJKunderline{芮伯}皆共諸侯並進,相顧而揖,乃並再拜稽首,起而言曰:“敢告天子,大天改大國殷之王命,誅殺殷紂。惟周家\CJKunderline{文王}、\CJKunderline{武王}大受天道而順之,能憂我西土之民,以此王有天下。惟我周家新升王位,當盡和天下賞罰,戡定其為王之功用,布遺後人之美,將使施及子孫,無有窮盡之期。今王新即王位,其敬之哉!當張大我之六師,令國常強盛,無令傾壞我高祖寡有之命。”戒王使繼先王之業也。 \par}

王若曰:“庶邦侯、甸、男、衛,\footnote{順其戒而告之,不言群臣,以外見內。○馬本從此以下為\CJKunderwave{康王之誥},又云:“與\CJKunderwave{顧命}差異。敘歐陽、大小夏侯同為\CJKunderwave{顧命}。”}惟予一人\CJKunderline{釗}報誥。\footnote{報其戒。}昔君\CJKunderline{文}、\CJKunderline{武}丕平富,不務咎,\footnote{言先君文武道大,政化平美,不務咎惡。}厎至齊\footnote{馬讀絕句。}信,用昭明於天下。\footnote{致行至中信之道,用顯明於天下。言聖德治。○厎,之履反。}則亦有熊羆之士,不二心之臣,保乂王家。\footnote{言文武既聖,則亦有勇猛如熊羆之士,忠一不二心之臣,共安治王家。○熊音雄。羆,彼皮反。}用端命於上帝,皇天用訓厥道,付畀四方。\footnote{君聖臣良,用受端直之命於上天。大天用順其道,付與四方之國,王天下。○畀,必利反,徐甫至反。王,於況反。}乃命建侯樹屏,在我後之人。\footnote{言文武乃施政令,立諸侯,樹以為蕃屏,傳王業在我後之人。謂子孫。}今予一二伯父,尚胥暨顧,綏爾先公之臣,服於先王。\footnote{天子稱同姓諸侯曰伯父。言今我一二伯父,庶幾相與顧念文武之道,安汝先公之臣,服於先王而法循之。}雖爾身在外,乃心罔不在王室,\footnote{言雖汝身在外士之為諸侯,汝心常當忠篤,無不在王室。“熊羆之士”,勵朝臣,此督諸侯。○督,丁木反。}用奉恤厥若,無遺鞠子羞。”\footnote{當各用心奉憂其所行順道,無自荒怠,遺我稚子之羞辱。稚子,\CJKunderline{康王}自謂也。○鞠,居六反。}


{\noindent\zhuan\zihao{6}\fzbyks 傳“順其”至“見內”。正義曰:群臣戒王使勤,王又戒之使輔己,是順其事而告之也。上文太保、\CJKunderline{芮伯}進言,不言諸侯,以內見外。此王告庶邦,不言朝臣,以外見內,欲令互相備也。周制六服,此惟四服,不言採、要者,略舉其事。猶\CJKunderwave{武成}云“甸、侯、衛,駿奔走”,亦略舉之矣。“予一人釗”。正義曰:\CJKunderwave{禮}天子自稱予一人,不言名。此王自稱名者,新即王位,謙也。 \par}

{\noindent\zhuan\zihao{6}\fzbyks 傳“言先”至“咎惡”。正義曰:孔以“富”為美,故云“政化平美”。不務咎惡於人,言哀矜下民,不用刑罰之。\CJKunderline{王肅}云“文武道大,天下以平,萬民以富”是也。 \par}

{\noindent\zhuan\zihao{6}\fzbyks 傳“致行”至“德洽”。正義曰:孔以“齊”為中,致行中正誠信之道。\CJKunderline{王肅}云:“立大中之道也。” \par}

{\noindent\zhuan\zihao{6}\fzbyks 傳“天子”至“循之”。正義曰:\CJKunderwave{覲禮}言天子呼諸侯之禮云:“同姓大國則曰伯父,其異姓則曰伯舅,同姓小邦則曰叔父,其異姓則曰叔舅。”計此時諸侯多矣,獨云“伯父”,舉同姓大國言之也。諸侯先公以臣道服於先王,其事有法,故令安汝先公之用臣,服於先王,以臣之道而法循之。 \par}

{\noindent\zhuan\zihao{6}\fzbyks 傳“言雖”至“諸侯”。正義曰:王之此誥,並誥群臣諸侯,但互相發見,其言不備。言先王有熊羆之士,勵朝臣使用力如先世之臣也。此言汝身在外土,心念王室,督諸侯使然。 \par}

{\noindent\shu\zihao{5}\fzkt “王若”至“子羞”。正義曰:群臣諸侯既進戒王,王順其戒呼而告之曰:“眾邦在侯、甸、男、衛諸服內之國君,惟我一人釗報誥卿士群公。昔先君\CJKunderline{文王}、\CJKunderline{武王}其道甚大,政化平美,專以美道教化,不務咎惡於人,致行至美中正誠信之道,用是顯明於天下”。言聖道博洽也。“文武既聖,時臣亦賢,則亦有如熊如羆之勇士,不二心之忠臣,共安治王家。以君聖臣良之故,用能受端直之命於上天。大天用順其道,付與四方之國,使文武受此諸國,王有天下。”言文武得賢臣之力也。“文武以得臣力之故,乃施政令,封立賢臣為諸侯者,樹之以為蕃屏,令屏衛在我後之人”。先王所立諸侯,即今諸侯之祖,故舉先世之事以告今之諸侯。“今我一二伯父,庶幾相與顧念文武之道,安汝先公之用臣,服於先王之道而法循之,亦當以忠誠輔我天子。雖汝身在外土為國君,汝心常當無有不在王室,當各用心奉憂其所行順道,無自荒怠,以遺我稚子之羞辱”。“稚子”,\CJKunderline{康王}自謂。戒令匡弼己也。 \par}

群公既皆聽命,相揖趨出。\footnote{已聽誥命,趨出罷退,諸侯歸國,朝臣就次。}王釋冕,反喪服。\footnote{脫去黼冕,反服喪服,居倚盧。○去,羌呂反。}

{\noindent\shu\zihao{5}\fzkt “群公”至“喪服”。正義曰:“群公”總謂朝臣與諸侯也。\CJKunderline{鄭玄}云:“群公主為諸侯與王之三公,諸臣亦在焉。王釋冕,反喪服,朝臣諸侯亦反喪服。\CJKunderwave{禮·喪服}篇臣為君,諸侯為天子,皆斬衰。” \par}

\section{畢命第二十六【偽】}


\CJKunderline{康王}命作冊\CJKunderline{畢},\footnote{命為冊書,以命\CJKunderline{畢公}。}分居里,成周郊,\footnote{分別民之居里,異其善惡。成定東周郊境,使有保護。○別,彼列反。}作\CJKunderwave{畢命}。

畢命\footnote{言\CJKunderline{畢公}見命之書。}


{\noindent\zhuan\zihao{6}\fzbyks 傳“命為”至“\CJKunderline{畢公}”。正義曰:\CJKunderwave{周禮·內史}云:“凡命諸侯及孤卿大夫,則策命之。”此云“命作冊”者,命內史為冊書以命\CJKunderline{畢公},故云以冊命\CJKunderline{畢公}。 \par}

{\noindent\zhuan\zihao{6}\fzbyks 傳“分別”至“保護”。正義曰:殷之頑民,遷居此邑,歷世化之,已得純善,恐其變改,故更命\CJKunderline{畢公}分別民之居里,異其善惡。即經所云“旌別淑慝,表厥宅裡,彰善癉惡,樹之風聲”,“殊厥井疆,俾克畏慕”皆是也。“分”者令其善惡分別,使惡者慕善,非分別其處,使之異居也。此邑本名成周,欲以成就周道。民不純善,則是未成,故命\CJKunderline{畢公}教之。“成定東周郊境”,即經“申畫郊圻,慎固封守”,是其使有保護。 \par}

{\noindent\shu\zihao{5}\fzkt “\CJKunderline{康王}”至“畢命”。正義曰:\CJKunderline{康王}命史官作冊書命\CJKunderline{畢公},使\CJKunderline{畢公}分別民之居里,令善惡有異。於成周之邑,成定東周之郊境。史敘其事,作\CJKunderwave{畢命}。 \par}

惟十有二年,六月庚午\xpinyin*{朏},\footnote{康王即位十二年六月三日庚午。○朏,普忽反,徐芳尾反,又芳憒反。}越三日壬申,王朝步自宗周,至於豐。\footnote{於朏三日壬申,王朝行自宗周,至於豐。宗周,鎬京。豐,\CJKunderline{文王}所都。○朝,直遙反。鎬,戶老反。}以成周之眾,命\CJKunderline{畢公}保釐東郊。\footnote{用成周之民眾,命\CJKunderline{畢公}使安理治正成周東郊,令得所。○釐,力之反。治,直吏反,一本作“治政”,則依字讀。令,力呈反。}


{\noindent\zhuan\zihao{6}\fzbyks 傳“\CJKunderline{康王}”至“庚午”。正義曰:漢初不得此篇,有偽作其書以代之者。\CJKunderwave{漢書·律歷志}云:“\CJKunderline{康王}十二年六月戊辰朔,三日庚午,故\CJKunderwave{畢命豐刑}曰‘惟十有二年六月庚午朏,王命作策書\CJKunderwave{豐刑}’。”此偽作者傳聞舊語,得其年月,不得以下之辭,妄言作\CJKunderwave{豐刑}耳,亦不知\CJKunderwave{豐刑}之言何所道也。\CJKunderline{鄭玄}云:“今其逸篇有冊命霍侯之事,不同與此序相應,非也。”\CJKunderline{鄭玄}所見又似異於\CJKunderwave{豐刑},皆妄作也。\CJKunderwave{說文}云:“朏,月未盛之明也。”此日未有事而記此“庚午朏”者,為下言壬申張本,猶如記朔望與生魄死魄然也。 \par}

{\noindent\shu\zihao{5}\fzkt “惟十”至“東郊”。正義曰:惟\CJKunderline{康王}即位十有二年六月三日庚午,月光朏然而明也。於朏後三日壬申,王早朝行從宗周鎬京,至於豐邑,就\CJKunderline{文王}之廟。以成周之民眾命太師\CJKunderline{畢公},使安理東郊之民,令得其所。 \par}

王若曰:“嗚呼!父師,惟\CJKunderline{文王}、\CJKunderline{武王},敷大德於天下,用克受殷命。\footnote{王順其事嘆告\CJKunderline{畢公}代,\CJKunderline{周公}為大師,為東伯,命之代\CJKunderline{君陳}。言文武布大德於天下,故天佑之,用能受殷之王命。○大音泰。}惟\CJKunderline{周公}左右先王,綏定厥家。\footnote{言\CJKunderline{周公}助先王安定其家。}毖殷頑民,遷於洛邑,密邇王室,式化厥訓。\footnote{慎殷頑民,恐其叛亂,故徙於洛邑,密近王室,用化其教。○毖音秘。近如字,又附近之近。}既歷三紀,世變風移,四方無虞,予一人以寧。\footnote{言殷民遷周已經三紀,世代民易,頑者漸化,四方無可度之事,我天子用安矣。十二年曰紀。父子曰世。○度,待洛反,舊作待路反。}道有升降,政由俗革,不臧厥臧,民罔攸勸。\footnote{天道有上下交接之義,政教有用俗改更之理。民之俗善,以善養之。俗有不善,以法御之。若乃不善其善,則民無所勸慕。○上,時掌反。更,古衡反。}


{\noindent\zhuan\zihao{6}\fzbyks 傳“王順”至“王命”。正義曰:\CJKunderline{畢公}代\CJKunderline{周公}為太師,故王呼為“父師”,率東方諸侯,是為“東伯”也。蓋\CJKunderline{君陳}卒,命之使代\CJKunderline{君陳}也。 \par}

{\noindent\zhuan\zihao{6}\fzbyks 傳“言周”至“其家”。正義曰:\CJKunderwave{釋詁}云:“左、右,助也。”言\CJKunderline{周公}助先王安定其家。伐殷之時,\CJKunderline{周公}已有其功,復能遷殷頑民,言其功之多也。 \par}

{\noindent\zhuan\zihao{6}\fzbyks 傳“言殷”至“曰世”。正義曰:\CJKunderline{周公}以攝政七年營成周,\CJKunderline{成王}元年遷殷頑民,\CJKunderline{成王}在位之年雖未知,其實當在三十左右,至今應三十六年,是殷民遷周已歷三紀。十二年者,天之大數。歲星、太歲皆十二年而一周天,故“十二年曰紀”。父子易人為世。\CJKunderwave{大禹謨}云:“賞延於世。”謂緣父及子也。 \par}

{\noindent\zhuan\zihao{6}\fzbyks 傳“天道”至“勸慕”。正義曰:天氣下降,地氣上騰,而有寒暑生焉。刑新國用輕典,刑亂國用重典,輕重隨俗而有寬猛異焉。天道有上下交接之義,故寒暑易節。政教有用俗改更之理,故寬猛相濟。天道有寒暑遞來,政教以寬猛相濟。民之風俗,善惡無常,或善變為惡,或惡變為善,不可以其既善,謂善必不變。民之俗善,須以善養之,令善遂不變。人之俗有不善,當以善法御之,使變而為善。若乃不善其善,則下民無所勸慕。民無所慕,則變為惡矣。殷民今雖已善,更當以善教之。欲以屈\CJKunderline{畢公}之意。 \par}

惟公懋德,克勤小物,弼亮四世,正色率下,罔不祗師言。\footnote{言公勉行德,能勤小物,輔佐文、武、成、康,四世為公卿,正色率下,下人無不敬仰師法。○懋音茂。}嘉績多於先王,予小子垂拱仰成。”\footnote{公之善功多大先人之美。我小子為王,垂拱仰公成理。言其上顯父兄,下施子孫。○拱,九勇反。仰如字,徐五亮反。}

{\noindent\zhuan\zihao{6}\fzbyks 傳“言公”至“師法”。正義曰:“小物”猶小事也,能勤小事,則大事必能勤矣,故舉能勤小事以為\CJKunderline{畢公}之善。\CJKunderwave{釋詁}云:“亮,佐也。”\CJKunderwave{晉語}說\CJKunderline{文王}之事云,“詢於八虞,訪於辛、尹,重之以周、召、畢、榮”,則\CJKunderline{畢公}於\CJKunderline{文王}之世已為大臣,是“輔佐文、武、成、康四世為公卿”也。“正色”謂嚴其顏色,不惰慢,不阿諂。以此率下,下民無不敬仰師法之。 \par}

{\noindent\zhuan\zihao{6}\fzbyks 傳“公之”至“子孫”。正義曰:先王之功,無由可及。言公之善功多大先人之美,方欲委之以事,盛言之,重其功美矣。 \par}

{\noindent\shu\zihao{5}\fzkt “王若”至“仰成”。正義曰:\CJKunderline{康王}順其事嘆而呼\CJKunderline{畢公}曰:“嗚呼!父師,惟\CJKunderline{文王}、\CJKunderline{武王}布大德於天下,用此能受殷之王命,代殷為天子。惟\CJKunderline{周公}佐助先王,安定其家。慎彼殷之頑民,恐其或有叛逆,故遷於洛邑,令之北近王室,用使化其教訓。自爾已來,既歷三紀,人世既變,風俗亦移,四方無可度之事,我天子一人用是而得安寧。但天道有上下交接之義,政教有用俗改更之理。今日雖善,或變為惡,若不善其善,則民無所勸慕。更須選賢教之,舉善勸之,宜此任者,莫先於公。惟公勉力行德,能勤小事,輔佐四世,正色率下,無有不敬仰師法公言者。公之善功多於先王,我小子垂衣拱手,仰公成理。”將欲任之,故盛稱其德也。 \par}

王曰:“嗚呼!父師,今予祗命公以\CJKunderline{周公}之事,往哉!\footnote{今我敬命公以\CJKunderline{周公}所為之事,往為之哉!言非\CJKunderline{周公}所為,不敢枉公往治。○治,直吏反。}旌別淑慝,表厥宅裡,彰善\xpinyin*{癉}惡,樹之風聲。\footnote{言當識別頑民之善惡,表異其居里,明其為善,病其為惡,立其善風,揚其善聲。○別音彼列反。癉音丁但反。}弗率訓典,殊厥井疆,俾克畏慕。\footnote{其不循教道之常,則殊其井居田界,使能畏為惡之禍,慕為善之福,所以沮勸。○俾,必爾反。沮,辭汝反,又慈呂反。}申畫郊圻,慎固封守,以康四海。\footnote{郊圻雖舊所規畫,當重分明之。又當謹慎堅固封疆之守備,以安四海。京圻安,則四海安矣。○守,徐始救反。重,直用反。}政貴有恆,辭尚體要,不惟好異。\footnote{政以仁義為常,辭以理實為要,故貴尚之。若異於先王,君子所不好。○好,呼報反。}商俗靡靡,利口惟賢,餘風未殄,公其念哉!\footnote{紂以靡靡利口為賢,覆亡國家。今殷民利口餘風未絕,公其念絕之。○覆,芳服反。}


{\noindent\zhuan\zihao{6}\fzbyks 傳“言當”至“善聲”。正義曰:旌旗所以表識貴賤,故傳以“旌”為識。“淑”,善也。“慝”,惡也。言當識別頑民之善,惡知其善者,表異其所居之裡,若今孝子、順孫、義夫、節婦,表其門閭者也。表其善者,則惡者自見。明其為善,當褒賞之。病其為惡,當罪罰之。其有善人,立其善風,令邑里使放效之;揚其善聲,告之疏遠,使聞知之。 \par}

{\noindent\zhuan\zihao{6}\fzbyks 傳“其不”至“沮勸”。正義曰:\CJKunderwave{孟子}云,“方里為井,井九百畝”,使民“死徙無出鄉,鄉田同井,出入相友,守望相助,疾病相扶持,則百姓親睦”。然則先王制之為井田也,欲使民相親愛,生相佐助,死相殯葬。不循道教之常者,其人不可親近,與善民雜居,或染善為惡,故殊其井田居界,令民不與來往。猶今下民有大罪過不肯服者,則擯出族黨之外,吉凶不與交通,此之義也。亦既殊其井田,必當思自改悔,使其能畏為惡之禍,慕為善之福,所以沮止為惡者,勸勉為善者。 \par}

{\noindent\zhuan\zihao{6}\fzbyks 傳“郊圻”至“安矣”。正義曰:“郊圻”謂邑之境界。境界雖舊有規畫,而年世久遠,或相侵奪,當重分明畫之,以防後相侵犯。雖舉邑之郊境為言,其民田疆畔亦令更重畫之,不然何以得“殊其井疆”也?王城之立,四郊以為京師屏障,預備不虞,又當謹慎牢固封疆之守備,以安四海之內。此是王之近郊,牢設守備,惟可以安京師耳。而云“安四海”者,京師安,則四海安矣。 \par}

{\noindent\zhuan\zihao{6}\fzbyks 傳“紂以”至“絕之”。正義曰:韓宣子稱,紂使師延作靡靡之樂。“靡靡”者,相隨順之意。紂之為人,拒諫飾非,惡聞其短,惟以靡靡相隨順、利口捷給、能隨從上意者以之為賢。商人效之,遂成風俗,由此所以覆亡國家。殷民利口餘風,至今不絕,公其念絕之。欲令其變惡俗也。 \par}

{\noindent\shu\zihao{5}\fzkt “王曰”至“念哉”。正義曰:王更嘆而呼\CJKunderline{畢公}曰:“嗚呼!父師,今日我敬命公以\CJKunderline{周公}所為之事,公其往為之哉!公往至彼,當識別善之與惡,表異其善者所居之裡,彰明其為善,病其為惡。其為善之人,當立其善風,揚其善聲。其有不循道教之常者,則殊其井田疆界,使之能畏為惡之禍,慕為善之福。更重畫郊圻境界,謹慎牢固其封疆守備,以安彼四海之內。為政貴在有常,言辭尚其禮實要約,當不惟好其奇異。商之舊俗,靡靡然好相隨順,利口辯捷、阿諛順旨者惟以為賢。餘風至今未絕,公其念絕之哉!”戒\CJKunderline{畢公}以治殷民之法。 \par}

我聞曰:‘世祿之家,鮮克由禮,以蕩陵德,實悖天道。\footnote{特言我聞自古有之,世有祿位而無禮教,少不以放蕩陵邈有德者,如此實亂天道。○鮮,息淺反。悖,布內反。}敝化奢麗,萬世同流。’\footnote{言敝俗相化,車服奢麗,雖相去萬世,若同一流。○敝,步寐反。}茲殷庶士,席寵惟舊,怙侈滅義,服美於人。\footnote{此殷眾士,居寵日久,怙恃奢侈,以滅德義。服飾過制,美於其民。言僣上。○怙音戶。}驕淫矜侉,將由惡終。雖收放心,閒之惟艱。\footnote{言殷眾士驕恣過制,矜其所能,以自侉大,如此不變,將用惡自終。雖今順從周制,心未厭服,以禮閒御,其心惟難。○侉音苦瓜反。壓,於葉反,又於甲反,又於豔反。}資富能訓,惟以永年。惟德惟義,時乃大訓。不由古訓,於何其訓?”\footnote{以富資而能順義,則惟可以長年命矣。惟有德義,是乃大順。若不用古訓典籍,於何其能順乎?}王曰:“嗚呼!父師,邦之安危,惟茲殷士,不剛不柔,厥德允修。\footnote{言邦國所以安危,惟在和此殷士而已。治之不剛不柔,寬猛相濟,則其德政信修立。}惟\CJKunderline{周公}克慎厥始,惟\CJKunderline{君陳}克和厥中,惟公克成厥終。\footnote{\CJKunderline{周公}遷殷頑民以消亂階,能慎其始。\CJKunderline{君陳}弘\CJKunderline{周公}之訓,能和其中。\CJKunderline{畢公}闡二公之烈,能成其終。}三後協心,同底於道,道洽政治,澤潤生民。\footnote{三君合心為一,終始相成,同致於道。道至普洽,政化治理,其德澤惠施,乃浸潤生民。言三君之功,不可不尚。○治,直吏反。施,始鼓反。浸,子鴆反。}四夷左衽,罔不咸賴,予小子永膺多福。\footnote{言東夷、西戎、南蠻、北狄被髮左衽之人,無不皆恃賴三君之德,我小子亦長受其多福。○衽,而甚反,又而鴆反。}公其惟時成周,建無窮之基,亦有無窮之聞。\footnote{公其惟以是成周之治,為周家立無窮之基業,於公亦有無窮之名,以聞於後世。○為,於偽反。}子孫訓其成式,惟乂。\footnote{言後世子孫順公之成法,惟以治。}嗚呼!罔曰弗克,惟既厥心。\footnote{人之為政,無曰不能,惟在盡其心而已。}罔曰民寡,惟慎厥事。\footnote{無曰人少不足治也,惟在慎其政事,無敢輕之。○少,詩照反。}欽若先王成烈,以休於前政。”\footnote{敬順文武成業,以美於前人之政。所以勉\CJKunderline{畢公}。}


{\noindent\zhuan\zihao{6}\fzbyks 傳“特言”至“天道”。正義曰:凡以善言教化,無非古之訓典,於此特言“我聞”者,言此事自古有之,所以尢須嚴禁故也。世有祿位,財多勢重,縱恣其心而無禮教,如此之人,少能不以放蕩之心陵邈有德者。天道以上臨下,以善率惡,今乃以下慢上,以惡陵善,如此者實亂天道也。 \par}

{\noindent\zhuan\zihao{6}\fzbyks 傳“此殷”至“僣上”。正義曰:“席”者人之所處,故為居之義。“舊”,久也。殷士多是世貴之家,故為“居寵日久”。怙恃己之奢侈,自謂奢侈為賢,德義廢而不行,故為“以滅德義”。又以人輕位卑,美服盛飾,是“服飾過制度”。“美於其人”,言僣上服,服勝人也。 \par}

{\noindent\zhuan\zihao{6}\fzbyks 傳“言殷”至“惟難”。正義曰:“淫”訓過也,故為“過制”。強梁者不得其死,好勝者必遇其敵,故矜侉“不變,將用惡自終”。言“雖收放心”,則已收之矣。雖令順從周制,思威自止,故怨猶在,心未壓服,故“以禮閒御其心,惟難”也。“閒”謂防閒御止也。 \par}

{\noindent\zhuan\zihao{6}\fzbyks 傳“敬順”至“\CJKunderline{畢公}”。正義曰:“美於前人之政”,謂光前人之政。所以勉勵\CJKunderline{畢公}。 \par}

{\noindent\shu\zihao{5}\fzkt “我聞”至“其訓”。正義曰:我聞古人言曰:“世有祿位之家,恃富驕恣,少能用禮,以放蕩之心陵邈有德之士,如此者實悖亂天道。敝俗相化,奢侈華麗,雖相去萬世,而共同一流。”此殷之眾士,皆是富貴之家,居處寵勢,惟已久矣。怙恃奢侈,以滅德義。身卑而僣上,飾其服,美於其人。驕恣過制,矜能自侉,行如此不變,將用惡自終。令以法約之,雖收斂其放佚之心,恆防閒之,惟大艱難。資財富足,能順道義,則惟可以長年命矣。惟能用德,惟能行義,是乃為大順德也。若不用古之訓典,則於何其能順乎?欲令\CJKunderline{畢公}以古之訓典教殷民也。 \par}

\section{君牙第二十七【偽】}


\CJKunderline{穆王}\CJKunderline{君牙}為周大司徒,\footnote{\CJKunderline{穆王},\CJKunderline{康王}孫,昭王子。○\CJKunderline{穆王},名滿\CJKunderline{君牙},或作君雅。}作\CJKunderwave{君牙}。\footnote{\CJKunderline{君牙},臣名。}

君牙\footnote{命以其名,遂以名篇。}

王若曰:“嗚呼!\CJKunderline{君牙},\footnote{順其事而嘆,稱其名而命之。}惟乃祖乃父,世篤忠貞,服勞王家,厥有成績,紀於太常。\footnote{言汝父祖,世厚忠貞,服事勤勞王家,其有成功,見紀錄書於王之太常,以表顯之。王之旌旗畫日月曰太常。○畫,胡卦反。}惟予小子,嗣守\CJKunderline{文}、\CJKunderline{武}、\CJKunderline{成}、\CJKunderline{康}遺緒,亦惟先正之臣,克左右亂四方。\footnote{惟我小子,繼守先王遺業,亦惟父祖之臣,能佐助我治四方。言己無所能。}心之憂危,若蹈虎尾,涉於春冰。\footnote{言祖業之大,己才之弱,故心懷危懼。虎尾畏噬,春冰畏陷,危懼之甚。○蹈,徒報反。噬,市制反。陷,陷沒之陷。}


{\noindent\zhuan\zihao{6}\fzbyks 傳“言汝”至“太常”。正義曰:\CJKunderwave{周禮·司勳}云:“凡有功者,銘書於王之太常,祭於大烝。”\CJKunderline{鄭玄}云:“銘之言名也。生則書於王旌,以識其人與其功也。死則於烝先王祭之。”是有功者書於王之太常,以表顯之也。\CJKunderwave{周禮·司常}云:“日月為常。”王建太常,是王之旌旗晝日月名之曰太常也。 \par}

{\noindent\shu\zihao{5}\fzkt “\CJKunderline{穆王}”至\CJKunderline{君牙}”。正義曰:\CJKunderline{穆王}命其臣\CJKunderline{君牙}者為周大司徒之卿,以策書命之。史錄其策書,作\CJKunderwave{君牙}。 \par}

今命爾予翼,作股肱心膂。\footnote{今命汝為我輔翼股肱心體之臣。言委任。○膂音旅。}纘乃舊服,無忝祖考,弘敷五典,式和民則。\footnote{繼汝先祖故所服,忠勤無辱累祖考之道,大布五常之教,用和民令有法則。○累,劣偽反。令,力呈反。}爾身克正,罔敢弗正,民心罔中,惟爾之中。\footnote{言汝身能正,則下無敢不正。民心無中,從汝取中。必當正身示民以中正。}夏暑雨,小民惟曰怨諮。\footnote{夏月暑雨,天之常道,小人惟曰怨嘆諮嗟。言心無中也。}冬祁寒,小民亦惟曰怨諮。\footnote{冬大寒,亦天之常道,民猶怨諮。}厥惟艱哉!思其艱以圖其易,民乃寧。\footnote{天不可怨,民猶怨嗟,治民其惟難哉!當思慮其難以謀其易,民乃安。○易,以豉反。}


{\noindent\zhuan\zihao{6}\fzbyks 傳“今命”至“委任”。正義曰:“股”,足也。“肱”,臂也。“膂”,背也。汝為我輔翼,當如我之身,故舉四支以喻為股肱心體之臣,言委任如身也。傳以“膂”為體,以見四者皆體,非獨“膂”為體也。\CJKunderwave{禮記·緇衣}云:“民以君為心,君以民為體。”此舉四體,今以臣為心者,君臣合體,則亦同心。\CJKunderwave{詩}云:“赳赳武夫,公侯腹心”,是臣亦為君心也。 \par}

{\noindent\zhuan\zihao{6}\fzbyks 傳“冬大”至“怨嗟”。正義曰:傳以“祁”為大,故云“冬大寒”。寒言大,則夏暑雨是大雨,於此言“祁”以見之。上言“暑雨”,此不言“寒雪”者,於上言“雨”以見之,互相備也。 \par}

{\noindent\shu\zihao{5}\fzkt “今命”至“乃寧”。正義曰:王言我以危懼之故,今命汝為大司徒,汝當作我股肱心膂。言將任之如已身也。繼汝先世舊所服行,亦如父祖忠勤,無為不忠,辱累汝祖考。當須大布五常之教,用和天下兆氏,令有法則。凡欲率下,當先正身,汝身能正,則下無敢不正。民心無能中正,惟取汝之中正,汝當正身心以率之。夏月大暑大雨,天之常也,小民惟曰怨恨而諮嗟。冬月大寒,亦天之常也,小民亦惟曰怨恨而諮嗟。天不可怨,民尚怨之,治民欲使無怨,其惟難哉!思慮其難,以謀其易,為治不違道,不逆民,民乃安矣。 \par}

嗚呼!丕顯哉,\CJKunderline{文王}謨!\footnote{嘆\CJKunderline{文王}所謀大顯明。}丕承哉,\CJKunderline{武王}烈!\footnote{言\CJKunderline{武王}業美,大可承奉。}啟佑我後人,咸以正罔缺。\footnote{文武之謀業,大明可承奉,開助我後嗣,皆以正道無邪缺。○缺,若穴反。}爾惟敬明乃訓,用奉若於先王,\footnote{汝惟當敬明汝五教,用奉順於先王之道。}對揚文武之光命,追配於前人。”\footnote{言當答揚文武光明之命,君臣各追配於前令名之人。}


{\noindent\zhuan\zihao{6}\fzbyks 傳“言武”至“承奉”。正義曰:\CJKunderline{文王}未克殷,始謀造周,故美其謀。\CJKunderline{武王}以殺紂功成業就,故美其業。謀則明白可遵,業則功成可奉,故謀言“顯”,烈言“承”。\CJKunderwave{詩·周頌·武}篇曰,“於皇\CJKunderline{武王},無競維烈”,亦美\CJKunderline{武王}業之大也。 \par}

{\noindent\zhuan\zihao{6}\fzbyks 傳“文武”至“邪缺”。正義曰:文始謀之,武卒成之。文謀大明,武業可奉。言先王以此成功開道佑助我之後人,使我得安其事而奉行之。以正道見其無邪,罔缺失見其周備,故傳言“無邪缺”。 \par}

{\noindent\shu\zihao{5}\fzkt “嗚呼”至“前人”。正義曰:王又嘆言:“嗚呼!大是顯明哉,\CJKunderline{文王}之謀也!大可承奉哉,\CJKunderline{武王}之業也!\CJKunderline{文王}之謀、\CJKunderline{武王}之業,開道佑助我在後之人,皆以正道無邪缺。”言先王之道易可遵也。“汝惟敬明汝之五教,用奉順於先王之道。汝當答揚文武光明之命,追配於前世令名之人”。令其順先王之道,同古之大賢也。 \par}

王若曰:“\CJKunderline{君牙},乃惟由先正舊典時式,民之治亂在茲。\footnote{汝惟當奉用先正之臣所行故事、舊典、文籍是法,民之治亂在此而已,用之則民治,廢之則民亂。○治,直吏反,下注同。}率乃祖考之攸行,昭乃闢之有乂。”\footnote{言當循汝父祖之所行,明汝君之有治功。○闢,必亦反。}

{\noindent\shu\zihao{5}\fzkt “王若”至“有乂”。正義曰:王順而呼之曰:\CJKunderline{君牙},汝為大司徒,惟當奉用先世正官之法,諸臣所行故事舊典,於是法則之,民之治亂在此而已。汝必奉而用之,循汝祖考之所行,明汝君之有治功。”“汝君”,王自謂也。 \par}

\section{冏命第二十八【偽】}


\CJKunderline{穆王}命\CJKunderline{伯冏}為周太僕正,\footnote{\CJKunderline{伯冏},臣名也。太僕長,太御中大夫。○冏,九永反,字亦作煛。長,誅丈反。}作\CJKunderwave{冏命}。

冏命\footnote{以冏見命名篇。}


{\noindent\zhuan\zihao{6}\fzbyks 傳“\CJKunderline{伯冏}”至“大夫”。正義曰:“正”訓長也。\CJKunderwave{周禮}“太御中大夫”,“太僕下大夫”,孔以此言“太僕正”,則官高於太僕,故以為\CJKunderwave{周禮}太御者,知非\CJKunderwave{周禮}太僕。若是\CJKunderwave{周禮}太僕,則此雲太僕足矣,何須云“正”乎?且此經云“命汝作大正,正於群僕”,案\CJKunderwave{周禮}“太馭中大夫”而下,有戎僕、齊僕、道僕、田僕,太御最為長,既稱正於群僕,故以為太御中大夫。且與君同車,最為親近,故\CJKunderwave{春秋}隨侯寵少師以為車右,\CJKunderwave{漢書}文帝愛趙同命之為御。凡御者最為密暱,故此經云“汝無暱於憸人,充耳目之官”。故以為太御中大夫,掌御玉輅之官。戎僕雖中大夫,以戎事為重,敘在太御之下,故以太御為長。太僕雖掌燕朝,非親近之任,又是下大夫,不得為長。 \par}

{\noindent\shu\zihao{5}\fzkt “\CJKunderline{穆王}”至“冏命”。正義曰:\CJKunderline{穆王}命其臣名\CJKunderline{伯冏}者為周太僕正之官,以策書命之。史錄其策書,作\CJKunderwave{冏命}。 \par}

王若曰:“\CJKunderline{伯冏},惟予弗克於德,嗣先人宅丕後,\footnote{順其事以命\CJKunderline{伯冏},言我不能於道德,繼先人居大君之位,人輕任重。}\xpinyin*{怵惕}惟厲,中夜以興,思免厥愆。\footnote{言常悚懼惟危,夜半以起,思所以免其過悔。○怵,敕律反。惕,他歷反。}昔在\CJKunderline{文}、\CJKunderline{武},聰明齊聖,小大之臣,咸懷忠良,\footnote{聰明,視聽遠。齊通,無滯礙。臣雖官有尊卑,無不忠良。○礙,五代反。}其侍御僕從,罔匪正人。\footnote{雖給侍、進御、僕役從官,官雖微,無不用中正之人。○御如字,一音禦。從,才用反,注及下注“侍從”同。}以旦夕承弼厥辟,出入起居,罔有不欽,\footnote{小臣皆良,僕役皆正,以旦夕承輔其君,故君出入起居,無有不敬。}發號施令,罔有不臧。下民祇若,萬邦咸休。\footnote{言文武發號施令,無有不善。下民敬順其命,萬國皆美其化。}


{\noindent\zhuan\zihao{6}\fzbyks 傳“言常”至“過悔”。正義曰:\CJKunderwave{禮記·祭義}云:“春雨露既濡,君子履之,必有怵惕之心。”“怵惕”是心動之名,多憂懼之意也。“厲”訓危也,言常悚懼,惟恐傾危。\CJKunderwave{易}稱“夕惕若厲”,即此義也。 \par}

{\noindent\zhuan\zihao{6}\fzbyks 傳“聰明”至“忠良”。正義曰:聰發於耳,明發於目,故為“視聽遠”也。“齊”訓中也,“聖”訓通也,動必得中,通而先識,是“無滯礙”也。 \par}

{\noindent\shu\zihao{5}\fzkt “王若”至“咸休”。正義曰:王順其事而呼之曰:“\CJKunderline{伯冏},惟我不能於道德,而繼嗣先人居大君之位。人輕任重,終常悚懼。心內怵惕,惟恐傾危,中夜以起,思望免其愆過。昔在\CJKunderline{文王}、\CJKunderline{武王},聰無所不聞,明無所不見。齊,中也,每事得中。聖,通也,通知諸事。其身明聖如此,又小大之臣無不皆思忠良,其左右侍御僕從無非中正之人。以旦夕承輔其君,故其君出入起居無有不敬,文武發號施令無有不善。以此之故,下民敬順其命,萬邦皆美其化。”由臣善故也。 \par}

“惟予一人無良,實賴左右前後有位之士,匡其不及,\footnote{惟我一人無善,實恃左右前後有職位之士,匡正其不及。言此責群臣正己。}繩愆糾謬,格其非心,俾克紹先烈。\footnote{言恃左右之臣彈正過誤,檢其非妄之心,使能繼先王之功業。○繩,市陵反。俾,必爾反。}


{\noindent\zhuan\zihao{6}\fzbyks 傳“言恃”至“功業”。正義曰:木不正者,以繩正之。“繩”謂彈正,“糾”謂發舉,有愆過則彈正之,有錯謬則發舉之。“格”謂檢括,其有非理枉妄之心,檢括使妄心不作。臣當如此匡君,使能繼先王之功業。言己無能,責臣使如此也。 \par}

{\noindent\shu\zihao{5}\fzkt “惟予”至“先烈”。正義曰:王言:“惟我一人無善,亦既無知,實恃賴左右前後有職位之臣,匡正其智所不及者。”責群臣使正己也,即言正己之事。“繩其愆過,糾其錯謬,格其非妄之心,心有妄作,則格正之,使能繼先王之功業”。言得臣匡輔,乃可繼世也。 \par}

今予命汝作大正,正於群僕侍御之臣,\footnote{欲其教正群僕,無敢佞偽。}懋乃後德,交修不逮。\footnote{言侍御之臣,無小大親疏,皆當勉汝君為德,更代修進其所不及。○更,古衡反。}慎簡乃僚,無以巧言令色、便辟側媚,其惟吉士。\footnote{當謹慎簡選汝僚屬侍臣,無得用巧言無實、令色無質、便辟足恭、側媚諂諛之人,其惟皆吉良正士。○便,婢綿反。闢,匹亦反,徐扶亦反。足,將住反。諛,徐以朱反。}


{\noindent\zhuan\zihao{6}\fzbyks 傳“欲其”至“佞偽”。正義曰:“作大正”,“正”,長也,作僕官之長。“正於群僕”,令教正之。二“正”義不同也。群僕雖官有小大,皆近天子。近人主者多以諂佞自容,今大僕教正群僕,明使教之無敢佞偽也。案\CJKunderwave{周禮}太馭中大夫掌御玉輅,戎僕中大夫掌御戎車,齊僕下大夫掌馭金輅,道僕上士掌馭象輅,田僕上士掌馭田輅。“群僕”謂此也。 \par}

{\noindent\zhuan\zihao{6}\fzbyks 傳“當謹”至“正士”。正義曰:府史已下,官長所自闢除命,士以上皆應人主自選。此令太僕正謹慎簡選僚屬者,人主所用皆由臣下,臣下銓擬,可者然後用之,故令太僕正慎簡僚屬也。\CJKunderwave{論語}稱:“巧言、令色、足恭,左聰明恥之。”“便辟”是巧言令色之類,知是彼“足恭”也。“巧言”者,巧為言語以順從上意,無情實也。“令色”者,善為顏色以媚說人主,無本質也。“便僻”者,前卻俯仰,以是為恭。“側媚”者,為僻側之事以求媚於君。此等皆是諂諛之人,不可用為近官也。“媚”,愛也。襄三十一年\CJKunderwave{左傳}云,鄭子產謂子皮曰:“誰敢求愛於子?”知此為“側媚”者,為側行以求愛,非是愛前人也。若能愛在上,則忠臣也,不當禁其無用。 \par}

{\noindent\shu\zihao{5}\fzkt “今予”至“吉士”。正義曰:今我命汝作太僕官大正,汝當教正於群僕侍御之臣,勸勉汝君為德,汝與同僚交更修進汝君智所不及之事。汝為僕官之長,當慎簡汝之僚屬,必使皆得正人,無得用巧言令色、便辟側媚之人,其惟皆當用吉良善士。令選其在下屬官,小臣僕隸之等,皆用善人。 \par}

僕臣正,厥后克正。僕臣諛,厥后自聖。\footnote{言僕臣皆正,則其君乃能正。僕臣諂諛,則其君乃自謂聖。}後德惟臣,不德惟臣。\footnote{君之有德,惟臣成之。君之無德,惟臣誤之。言君所行善惡,專在左右。}爾無暱於憸人,充耳目之官,迪上以非先王之典。\footnote{汝無親近於憸利小子之人,充備侍從在視聽之官,道君上以非先王之法。○暱,女乙反。憸,息廉反,徐七漸反,利口也,本亦作𢘼。近附近之近。道,導也。}非人其吉,惟貨其吉,\footnote{若非人其實吉良,惟以貨財配其吉良,以求入於僕侍之臣,汝當清審。}若時,瘝厥官,\footnote{若用是行貨之人,則病其官職。○瘝,故頑反。}惟爾大弗克祇厥辟,惟予汝辜。”\footnote{用行貨之人,則惟汝大不能敬其君,惟我則亦以此罪汝。言不忠也。}王曰:“嗚呼,欽哉!永弼乃後於彝憲。”\footnote{嘆而敕之,使敬用所言,當長輔汝君於常法。此\CJKunderline{穆王}庶幾欲蹈行常法。}

\section{呂刑第二十九(呂刑上、下)}


\CJKunderline{呂}命,\footnote{\CJKunderline{呂侯}見命為天子司寇。}\CJKunderline{穆王}訓夏贖刑,\footnote{\CJKunderline{呂侯}以\CJKunderline{穆王}命作書,訓暢夏禹贖刑之法,更從輕以佈告天下。○贖音蜀,注及下同。}作\CJKunderwave{呂刑}。

呂刑\footnote{後為\CJKunderline{甫侯},故或稱\CJKunderwave{甫刑}。}


{\noindent\zhuan\zihao{6}\fzbyks 傳“\CJKunderline{呂侯}”至“司寇”。正義曰:\CJKunderline{呂侯}得王命,必命為王官。\CJKunderwave{周禮}司寇掌刑,知\CJKunderline{呂侯}見命為天子司寇。\CJKunderline{鄭玄}云:“\CJKunderline{呂侯}受王命,入為三公。”引\CJKunderwave{書說}云:“周\CJKunderline{穆王}以\CJKunderline{呂侯}為相。”\CJKunderwave{書說}謂\CJKunderwave{書緯·刑得放}之篇有此言也。以其言“相”,知為三公。即如鄭言,當以三公領司寇,不然,何以得專王刑也。 \par}

{\noindent\zhuan\zihao{6}\fzbyks 傳“\CJKunderline{呂侯}”至“天下”。正義曰:名篇謂之\CJKunderwave{呂刑},其經皆言“王曰”,知“\CJKunderline{呂侯}以\CJKunderline{穆王}命作書”也。經言陳罰贖之事,不言何代之禮,故序言“訓夏”,以明經是夏法。王者代相革易,刑罰世輕世重,殷以變夏,周又改殷。夏法行於前代,廢已久矣。今複訓暢夏禹贖刑之法,以周法傷重,更從輕以佈告天下。以其事合於當時,故\CJKunderline{孔子}錄之以為法。經多說治獄之事,是訓釋申暢之也。金作贖刑,\CJKunderline{唐}、\CJKunderline{虞}之法。\CJKunderwave{周禮}職金“掌受士之金罰、貨罰,入於司兵”,則周亦有贖刑。而遠訓夏之贖刑者,\CJKunderwave{周禮}惟言“士之金罰”,人似不得贖罪。縱使亦得贖罪,贖必異於夏法。以夏刑為輕,故祖而用之。罪實則刑之,罪疑則贖之,故當並言贖刑,非是惟訓贖罰也。\CJKunderwave{周禮}“司刑掌五刑之法,以麗萬民之罪。墨罪五百,劓罪五百,宮罪五百,刖罪五百,殺罪五百”。五刑惟有二千五百。此經“五刑之屬三千”,案刑數乃多於\CJKunderwave{周禮},而言變從輕者,\CJKunderwave{周禮}五刑皆有五百,此則輕刑少而重刑多;此經墨、劓皆千,刖刑五百,宮刑三百,大辟二百,輕刑多而重刑少,變周用夏,是改重從輕也。然則\CJKunderline{周公}聖人,相時製法而使刑罰太重,令\CJKunderline{穆王}改易之者,\CJKunderline{穆王}遠取夏法,殷刑必重於夏。夏承堯舜之後,民淳易治,故制刑近輕。輕則民慢,故殷刑稍重。自湯已後,世漸苛酷,紂作炮烙之刑,明知刑罰益重。周承暴虐之後,不可頓使太輕。雖減之輕,猶重於夏法。成康之間,刑措不用,下及\CJKunderline{穆王},民猶易治。故\CJKunderline{呂侯}度時制宜,勸王改從夏法。聖人之法非不善也,而不以經遠。\CJKunderline{呂侯}之智非能高也,而法可以適時。苟適於時,事即可為善,亦不言\CJKunderline{呂侯}才高於\CJKunderline{周公},法勝於前代。所謂觀民設教,遭時制宜,刑罰所以世輕世重,為此故也。 \par}

{\noindent\zhuan\zihao{6}\fzbyks 傳“後為”至“甫刑”。正義曰:\CJKunderwave{禮記}書傳引此篇之言多稱為“\CJKunderwave{甫刑}曰”,故傳解之“後為\CJKunderline{甫侯},故或稱\CJKunderwave{甫刑}”。知“後為\CJKunderline{甫侯}”者,以\CJKunderwave{詩·大雅·崧高}之篇宣王之詩,云“生甫及申”;\CJKunderwave{揚之水}為平王之詩,云“不與我戍甫”,明子孫改封為\CJKunderline{甫侯}。不知因呂國改作甫名?不知別封餘國而為甫號?然子孫封甫,\CJKunderline{穆王}時未有甫名而稱為\CJKunderwave{甫刑}者,後人以子孫之國號名之也。猶若叔虞初封於唐,子孫封晉,而\CJKunderwave{史記}稱\CJKunderwave{晉世家}。然宣王以後,改呂為甫,\CJKunderwave{鄭語}史伯之言幽王之時也,乃云“申呂雖衰,齊許猶在”,仍得有呂者,以彼史伯論四岳治水,其齊、許、申、呂是其後也。因上“申呂”之文而云“申呂雖衰”,呂即甫也。 \par}

{\noindent\shu\zihao{5}\fzkt “呂命”至“呂刑”。正義曰:\CJKunderline{呂侯}得\CJKunderline{穆王}之命為天子司寇之卿,\CJKunderline{穆王}於是用\CJKunderline{呂侯}之言,訓暢夏禹贖刑之法。\CJKunderline{呂侯}稱王之命而佈告天下。史錄其事,作\CJKunderwave{呂刑}。 \par}

惟\CJKunderline{呂}命王:“享國百年,耄荒,\footnote{言\CJKunderline{呂侯}見命為卿,時\CJKunderline{穆王}以享國百年,耄亂荒忽。\CJKunderline{穆王}即位過四十矣,言百年,大其雖老而能用賢以揚名。○耄,本亦作𦿗,毛報反,\CJKunderwave{切韻}莫報反。}度作刑以詰四方。”\footnote{度時世所宜,訓作贖刑,以治天下四方之民。○度,待洛反。注同,馬如字,云:“法度也。”詰,起一反。}


{\noindent\zhuan\zihao{6}\fzbyks 傳“言呂”至“揚名”。正義曰:史述\CJKunderline{呂侯}見命而記王年,知其得命之時王已享國百年也。\CJKunderwave{曲禮}云:“八十九十曰耄。”是“耄荒”為年老精神耄亂荒忽也。\CJKunderline{穆王}即位之時,已年過四十矣,比至命\CJKunderline{呂侯}之年,未必已有百年。言“百年”者,美大其事,雖則年老而能用賢以揚名,故記其百年之耄荒也。\CJKunderwave{周本紀}云:“\CJKunderline{甫侯}言於王,作修刑辟。”是修刑法者皆\CJKunderline{呂侯}之意,美王能用之。\CJKunderline{穆王}即位過四十者,不知出何書也。\CJKunderwave{周本紀}云:“\CJKunderline{穆王}即位,春秋已五十矣”,“立五十五年崩”。司馬遷若在孔後,或當各有所據。\CJKunderwave{無逸}篇言殷之三王及\CJKunderline{文王}享國若干年者,皆謂在位之年。此言“享國百年”,乃從生年而數,意在美王年老能用賢,而言其長壽,故舉從生之年,以“耄荒”接之,美其老之意也。文不害意,不與彼同。 \par}

{\noindent\shu\zihao{5}\fzkt “惟呂”至“四方”。正義曰:惟\CJKunderline{呂侯}見命為卿,於時\CJKunderline{穆王}享有周國已積百年,王精神耄亂而荒忽矣。王雖老耄,猶能用賢,取\CJKunderline{呂侯}之言,度時世所宜,作夏贖刑以治天下四方之民也。 \par}

王曰:“若古有訓,\CJKunderline{蚩尤}惟始作亂,延及於平民,\footnote{順古有遺訓,言\CJKunderline{蚩尤}造始作亂,惡化相易,延及於平善之人。九黎之君號曰蚩尢。○蚩,尺之反;尤,有牛反;馬云:“少昊之末九黎君名。”}罔不寇賊鴟義,姦宄奪攘矯虔。\footnote{平民化之,無不相寇賊,為鴟梟之義。以相奪攘,矯稱上命,若固有之。亂之甚。○鴟,尺之反;鴟梟,惡鳥;馬云:“鴟,輕也。”義,本亦作誼。宄音軌。攘,如羊反。矯,居表反。虔,其然反。}


{\noindent\zhuan\zihao{6}\fzbyks 傳“順古”至“蚩尢”。正義曰:古有遺訓,順而言之,故為“順古有遺訓”也。“蚩尢造始作亂”,其事往前未有,蚩尢今始造之,必是亂民之事,不知造何事也。下說三苗之主習蚩尢之惡,作五虐之刑,此章主說虐刑之事,蚩尢所作,必亦造虐刑也。以峻法治民,民不堪命,故惡化轉相染易,延及於平善之民,亦化為惡也。“九黎之君號曰蚩尢”,當有舊說云然,不知出何書也。\CJKunderwave{史記·五帝本紀}云:“神農氏世衰,諸侯相侵伐,蚩尢最為暴虐,莫能伐之。黃帝乃徵師諸侯,與蚩尢戰於涿鹿之野,遂擒殺蚩尢,而諸侯咸尊軒轅為天子。”如\CJKunderwave{本紀}之言,蚩尢是炎帝之末諸侯名也。應劭云:“蚩尢,古天子。”鄭云:“蚩尢霸天下,黃帝所伐者。”\CJKunderwave{漢書音義}有臣瓚者,引\CJKunderwave{孔子三朝記}云:“蚩尢,庶人之貪者。”諸說不同,未知蚩尢是何人也。\CJKunderwave{楚語}曰:“少昊氏之衰也,九黎亂德,顓頊受之,使復舊常。”則九黎在少昊之末,非蚩尢也。韋昭云:“九黎氏九人,蚩尢之徒也。”韋昭雖以九黎為蚩尢,要\CJKunderwave{史記}蚩尢在炎帝之末,\CJKunderwave{國語}九黎在少昊之末,二者不得同也。“九黎”之文惟出楚語,孔以蚩尢為九黎。下傳又云:“蚩尢黃帝所滅”,言“黃帝所滅”,則與\CJKunderwave{史記}同矣。孔非不見\CJKunderwave{楚語},而為此說,蓋以蚩尢是九黎之君,黃帝雖滅蚩尢,猶有種類尚在,故下至少昊之末,更復作亂。若其不然,孔意不可知也。\CJKunderline{鄭玄}云:“學蚩尢為亂者,九黎之君,在少昊之代也。”其意以蚩尢當炎帝之末,九黎當少昊之末,九黎學蚩尢,九黎非蚩尢也。 \par}

{\noindent\zhuan\zihao{6}\fzbyks 傳“平民”至“之甚”。正義曰:蚩尢作亂,當是作重刑以亂民。以峻法酷刑,民無所措手足,困於苛虐所酷,人皆苛且,故平民化之,無有不相寇賊。群行攻劫曰“寇”,殺人曰“賊”,言攻殺人以求財也。“鴟梟”,貪殘之鳥。\CJKunderwave{詩}云:“為梟為鴟。”梟是鴟類。\CJKunderline{鄭玄}云:“盜賊狀如鴟梟,鈔掠良善,劫奪人物。”傳言“鴟梟之義”,如鄭說也。\CJKunderwave{釋詁}云:“虔,固也”。“若固有之”,言取得人物,如己自有也。 \par}

苗民弗用靈,制以刑,惟作五虐之刑曰法。\footnote{三苗之君習蚩尢之惡,不用善化民,而制以重刑。惟為五虐之刑,自謂得法。蚩尢黃帝所滅,三苗\CJKunderline{帝堯}所誅,言異世而同惡。}殺戮無辜,爰始淫為劓、刵、椓、黥。\footnote{三苗之主,頑兇若民,敢行虐刑,以殺戮無罪,於是始大為截人耳鼻,椓陰,黥面,以加無辜,故曰“五虐”。○劓,魚器反。刵,徐如志反。椓,丁角反。黥,其京反。}越茲麗刑並制,罔差有辭。\footnote{苗民於此施刑,並制無罪,無差有直辭者。言淫濫。○麗,力馳反。並,必政反。}民興胥漸,泯泯棼棼,罔中於信,以覆詛盟。\footnote{三苗之民瀆於亂政,起相漸化,泯泯為亂,棼棼同惡,皆無中於信義,以反背詛盟之約。○泯,面忍反,徐音民。棼,芳雲反,徐扶雲反。覆,芳服反,徐敷目反。詛,側助反。背音佩。約如字,又於妙反。}


{\noindent\zhuan\zihao{6}\fzbyks 傳“三苗”至“同惡”。正義曰:上說蚩尢之惡,即以“苗民”繼之,知經意言“三苗之君習蚩尢之惡”。“靈”,善也。不用善化民,而制以重刑。學蚩尢制之,用五刑而虐為之,故為“五虐之刑”,不必\CJKunderline{皋陶}五刑之外,別有五也。“曰法”者,述苗民之語,自謂所作得法,欲民行而畏之。如\CJKunderwave{史記}之文,蚩尢黃帝所滅,下句所說“三苗\CJKunderline{帝堯}所誅”,\CJKunderwave{楚語}云“三苗復九黎之惡”,是“異世而同惡”也。\CJKunderline{鄭玄}以為“苗民即九黎之後。顓頊誅九黎,至其子孫為三國。高辛之衰,又復九黎之惡。堯興,又誅之。堯末,又在朝,舜臣堯又竄之。後\CJKunderline{禹}攝位,又在洞庭逆命,\CJKunderline{禹}又誅之。\CJKunderline{穆王}深惡此族三生凶德,故著其惡而謂之民”。孔惟言“異世同惡”,不言三苗是蚩尢之子孫。韋昭云:“三苗,炎帝之後諸侯\CJKunderline{共工}也。” \par}

{\noindent\zhuan\zihao{6}\fzbyks 傳“三苗”至“五虐”。正義曰:三苗之主,實國君也。頑兇若民,故謂之“苗民”。不於上經為傳者,就此惡行解之,以其頑兇,敢行虐刑,以殺戮無罪。\CJKunderwave{釋詁}云:“淫,大也。”“於是大為截人耳鼻,椓陰,黥面”,苗民為此刑也。“椓陰”即宮刑也。“黥面”即墨刑也。\CJKunderwave{康誥}\CJKunderline{周公}戒\CJKunderline{康叔}云“無或劓刵人”,即周世有劓刵之刑,非苗民別造此刑也。以加無辜,故曰“五虐”。\CJKunderline{鄭玄}云:“刵,斷耳。劓,截鼻。椓謂椓破陰,黥為羈黥人面。苗民大為此四刑者,言其特深刻,異於\CJKunderline{皋陶}之為。”鄭意蓋謂截耳截鼻多截之,椓陰苦於去勢,黥面甚於墨頟,孔意或亦然也。 \par}

{\noindent\zhuan\zihao{6}\fzbyks 傳“三苗”至“之約”。正義曰:“三苗之民”,謂三苗國內之民也。“瀆”謂慣瀆,苗君久行虐刑,民慣見亂政,習以為常,起相漸化。“泯泯”,相似之意。“棼棼”,擾攘之狀。“泯泯為亂”,習為亂也。“棼棼同惡”,共為惡也。“中”猶當也,“皆無中於信義”,言為行無與信義合者。\CJKunderwave{詩}云:“君子屢盟,亂是用長。”亂世之民,多相盟詛,既無信義,必皆違之,以此無中於信,反背詛盟之約也。 \par}

虐威庶戮,方告無辜於上。上帝監民,罔有馨香,德刑發聞惟腥。\footnote{三苗虐政作威,眾被戮者方方各告無罪於天,天視苗民無有馨香之行,其所以為德刑,發聞惟乃腥臭。○聞音問,又如字,注同。腥音星。行,下孟反。}皇帝哀矜庶戮之不辜,報虐以威,遏絕苗民,無世在下。\footnote{皇帝,\CJKunderline{帝堯}也。哀矜眾被戮者之不辜,乃報為虐者以威,誅遏絕苗民,使無世位在下國也。○君帝,君宜作皇字,\CJKunderline{帝堯}也。遏,於葛反。}

{\noindent\zhuan\zihao{6}\fzbyks 傳“三苗”至“腥臭”。正義曰:“方方各告無罪於上天”,言其處處告也。天矜於下,俯視苗民,無有馨香之行。“馨香”以喻善也。“其所以為德刑”,苗民自謂是德刑者,發聞於外,惟乃皆是腥臭。“腥臭”喻惡也。 \par}

{\noindent\zhuan\zihao{6}\fzbyks 傳“君帝”至“下國”。正義曰:\CJKunderwave{釋詁}云:“皇,君也。”此言“遏絕苗民”,下句即云“乃命\CJKunderline{重}、\CJKunderline{黎}”,\CJKunderline{重}、\CJKunderline{黎}是\CJKunderline{帝堯}之事,知此滅苗民亦\CJKunderline{帝堯}也。此滅苗民在堯之初興,使無世位在於下國,而堯之末年,又有竄三苗者,禮天子不滅國,擇立其次賢者。此為五虐之君,自無世位在下,其改立者復得在朝。但此族數生凶德,故歷代每被誅耳。 \par}

{\noindent\shu\zihao{5}\fzkt “王曰”至“在下”。正義曰:\CJKunderline{呂侯}進言於王,使用輕刑。又稱王之言以告天下,說重刑害民之義。王曰:“順古道有遺餘典訓,記法古人之事。昔炎帝之末,有九黎之國君號蚩尢者,惟造始作亂,惡化遞相染易,延及末平善之民。平民化之,亦變為惡,無有不相寇盜,相賊害,為鴟梟之義。鈔掠良善,外奸內宄,劫奪人物,攘竊人財,矯稱上命,以取人財,若己固自有之。然蚩尢之惡已如此矣,至於高辛氏之末,又有三苗之國君,習蚩尢之惡,不肯用善化民,而更制重法。惟作五虐之刑,乃言曰此得法也。殺戮無罪之人,於是始大為四種之刑。刵,截人耳。劓,截人鼻。劅,椓人陰。黥,割人面。苗民於此施刑之時,並制無罪之人。對獄有罪者無辭,無罪者有辭,苗民斷獄,並皆罪之,無差簡有直辭者。言濫及無罪者也。三苗之民,慣瀆亂政,起相漸染,皆化為惡。泯泯為亂,棼棼同惡,小大為惡。民皆巧詐,無有中於信義。以此無中於信,反背詛盟之約,雖有要約,皆違背之。三苗虐政作威,眾被戮者方方各告無罪於上天。上天下視苗民,無有馨香之行。其所以為德刑者,發聞於外,惟乃皆腥臭,無馨香也。君帝\CJKunderline{帝堯}哀矜眾被殺戮者,不以其罪,乃報為暴虐者以威,止絕苗民,使無世位在於下國。”言以刑虐,故滅之也。 \par}

乃命\CJKunderline{重}、\CJKunderline{黎},絕地天通,罔有降格。\footnote{重即羲,黎即和。堯命\CJKunderline{義和}世掌天地四時之官,使人神不擾,各得其序,是謂絕地天通。言天神無有降地,地民不至於天,明不相干。○重,直龍反。黎,力兮反。}群后之逮在下,明明棐常,鰥寡無蓋。\footnote{群后諸侯之逮在下國,皆以明明大道輔行常法,故使鰥寡得所,無有掩蓋。○棐音匪,又芳鬼反。鰥,居頑反。}皇帝清問下民,鰥寡有辭於苗。\footnote{\CJKunderline{帝堯}詳問民患,皆有辭怨於苗民。○清問,馬云:“清,訊。”}德威惟畏,德明惟明。\footnote{言堯監苗民之見怨,則又增修其德,行威則民畏服,明賢則德明人,所以無能名焉。}


{\noindent\zhuan\zihao{6}\fzbyks 傳“重即”至“相干”。正義曰:\CJKunderwave{楚語}云:“宅王問於觀射父曰:‘\CJKunderwave{周書}所謂\CJKunderline{重}、\CJKunderline{黎}實使天地不通者,何也?若無然,民將能登天乎?’對曰:‘非此之謂也。古者民神不雜。少昊氏之衰也,九黎亂德,家為巫史,民神同位,禍災荐臻。顓頊受之,乃命南正重司天以屬神,命火正黎司地以屬民,使復舊常,無相侵瀆,是謂絕地天通。其後三苗復九黎之德,堯覆育\CJKunderline{重}、\CJKunderline{黎}之後,不忘舊者,使復典之。’”彼言主說此事,而\CJKunderwave{堯典}云“乃命羲和,欽若昊天”,即所謂育\CJKunderline{重}、\CJKunderline{黎}之後,使典之也。以此知“重即羲”也,“黎即和”也。言羲是重之子孫,和是黎之子孫,能不忘祖之舊業,故以“\CJKunderline{重}、\CJKunderline{黎}”言之。傳言“堯乃命羲和掌天地四時之官”,\CJKunderwave{堯典}文也。“民神不擾,是謂絕地天通”,\CJKunderwave{楚語}文也。孔惟加“各得其序”一句耳。\CJKunderwave{楚語}又云,司天屬神,司地屬民。令神與天在上,民與地在下,定上下之分,使民神不雜,則祭享有度,災厲不生。經言民神分別之意,故言“罔有降格”。言天神無有降至於地者,謂神不幹民。孔因互文雲地民不有上至於天者,言民不幹神也。乃總之云“明不相干”,即是民神不雜也。“地民”或作“地祇”。學者多聞神祇,又“民”字似“祇”,因妄改使謬耳。如\CJKunderwave{楚語}云“乃命\CJKunderline{重}、\CJKunderline{黎}”,是顓頊命之。\CJKunderline{鄭玄}以“‘皇帝哀矜庶戮之不辜’至‘罔有降格’,皆說顓頊之事。乃命\CJKunderline{重}、\CJKunderline{黎}即是命\CJKunderline{重}、\CJKunderline{黎}之身,非羲和也。‘皇帝清問’以下乃說堯事。顓頊與堯再誅苗民,故上言‘遏絕苗民’,下雲‘有辭於苗’,異代別時,非一事也”。案\CJKunderwave{楚語}云“少昊氏之衰也,九黎亂德”,又云“其後三苗復九黎之德”,則“九黎”、“三苗”非一物也。顓頊誅九黎謂之“遏絕苗民”,於鄭義為不愜。\CJKunderwave{楚語}言顓頊命\CJKunderline{重}、\CJKunderline{黎},解為\CJKunderline{帝堯}命羲和,於孔說又未允,不知二者誰得經意也。 \par}

{\noindent\zhuan\zihao{6}\fzbyks 傳“言堯”至“名焉”。正義曰:此經二句說\CJKunderline{帝堯}之德事也,而其言不順。文在“苗民”之下,故傳以為“堯監苗民之見怨,則又增修其德”,敦德以臨之,以德行其威罰,則民畏之而不敢為非。“明賢則德明人”者,若凡人雖欲以德明賢者,不能照察。今堯德明賢者,則能以德明識賢人,故皆勸慕為善。明與上句相互,則“德威”者,凡人雖欲以德行威,不能威肅。今堯行威罰,則能以德威罰罪人,故人皆畏威服德也。 \par}

{\noindent\shu\zihao{5}\fzkt “乃命”至“推明”。正義曰:三苗亂德,民神雜擾。\CJKunderline{帝堯}既誅苗民,乃命\CJKunderline{重}、\CJKunderline{黎}二氏,使絕天地相通,令民神不雜。於是天神無有下至地,地民無有上至天,言天神地民不相雜也。群后諸侯相與在下國,群臣皆以明明大道輔行常法,鰥寡皆得其所,無有掩蓋之者。君帝\CJKunderline{帝堯}清審詳問下民所患,鰥寡皆有辭怨於苗民。言誅之合民意。堯視苗民見怨,則又增修其德。以德行威,則民畏之,不敢為非。以德明人,人皆勉力自修,使德明。言堯所行賞罰得其所也。 \par}

乃命三後,恤功於民。\CJKunderline{伯夷}降典,折民惟刑。\CJKunderline{禹}平水土,主名山川。\CJKunderline{稷}降播種,農殖嘉穀。\footnote{\CJKunderline{伯夷}下典禮教民而斷以法。\CJKunderline{禹}治洪水,山川無名者主名之。后稷下教民播種,農畝生善谷。所謂堯命三君,憂功於民。○折,之設反,下同;馬、鄭、王皆音悊,馬云:“智也。”種音章用反。殖,承力反。斷,丁亂反,下同。}三後成功,惟殷於民。\footnote{各成其功,惟所以殷盛於民。言禮教備,衣食足。}士制百姓於刑之中,以教祇德。\footnote{言\CJKunderline{伯夷}道民典禮,斷之以法。\CJKunderline{皋陶}作士,制百官於刑之中,助成道化,以教民為敬德。○祇,止而反。}


{\noindent\zhuan\zihao{6}\fzbyks 傳“\CJKunderline{伯夷}”至“於民”。正義曰:\CJKunderline{伯夷}與稷言“降”,\CJKunderline{禹}不言“降”,“降”可知降下也,從上而下於民也。\CJKunderwave{舜典}\CJKunderline{伯夷}主禮典,“教民而斷以法”,即\CJKunderwave{論語}所謂“齊之以禮”也。山川與天地並生,民應先與作名。但\CJKunderline{禹}治水,萬事改新,古老既死,其名或滅。故當時無名者,\CJKunderline{禹}皆主名之。言此者,以見\CJKunderline{禹}治山川,為民於此耕稼故也。此三事者皆是為民,故傳既解三事,乃結上句,此即“所謂堯命三君,憂功於民”,憂欲與民施功也。此三事之次,當\CJKunderline{禹}功在先。先治水土,乃得種穀。民得穀食,乃能行禮。\CJKunderwave{管子}云:“衣食足,知榮辱。倉稟實,知禮節。”是言足食足衣然後行禮也。此經先言“\CJKunderline{伯夷}”者,以民為國之本,禮是民之所急,將言制刑,先言用禮,刑禮相須,重禮,故先言之也。 \par}

{\noindent\zhuan\zihao{6}\fzbyks 傳“言伯”至“敬德”。正義曰:此經大意,言\CJKunderline{禹}、稷教民,使衣食充足。\CJKunderline{伯夷}道民,使知禮節,有不從教者,乃以刑威之。故先言三君之功,乃說用刑之事。言\CJKunderline{禹}、稷教民稼穡,衣食既已充足。\CJKunderline{伯夷}道民典禮,又能折之以法。禮法既行,乃使\CJKunderline{皋陶}作士,制百官於刑之中。令百官用刑,皆得中正,使不僣不濫,不輕不重,助成道化,以教民為敬德。言從\CJKunderline{伯夷}之法,敬德行禮也。 \par}

{\noindent\shu\zihao{5}\fzkt “乃命”至“祇德”。正義曰:堯既誅苗民,乃命三君伯、夷、\CJKunderline{禹}、稷憂施功於民。使\CJKunderline{伯夷}下禮典教民,折斷下民,惟以典法。\CJKunderline{伯禹}身平治水土,主名天下山川,其無名者皆與作名。后稷下教民布種,在於農畝種殖嘉。谷三君者各成其功,惟以殷盛於民,使民衣食充足。乃使士官制御百官之姓於刑之中正,以教民為敬德。言先以禮法化民,民既富而後教之,非苟欲刑殺也。 \par}

穆穆在上,明明在下,灼於四方,罔不惟德之勤。\footnote{堯躬行敬敬在上,三後之徒秉明德明君道於下,灼然彰著四方,故天下之士無不惟德之勤。}故乃明於刑之中,率乂於民棐彝。\footnote{天下皆勤立德,故乃能明於用刑之中正,循道以治於民,輔成常教。○治,直吏反。}


{\noindent\zhuan\zihao{6}\fzbyks 傳“堯躬”至“之勤”。正義曰:\CJKunderwave{釋訓}云:“穆穆,敬也。”“明明”重明,則“穆穆”重敬,當敬天敬民,在於上位也。“明明在下”,則是臣事,知是“三後之徒秉明德明君道於下”也。彰著於四方,四方皆法效之,故天下之士無不惟德之勤。 \par}

{\noindent\zhuan\zihao{6}\fzbyks 傳“天下”至“常教”。正義曰:刑者所以助教而不可專用,非是身有明德,則不能用刑。以天下之大,萬方之眾,必當盡能用刑,天下乃治。此美堯能使“天下皆勤立德,故乃能明於用刑之中正”,言天下皆能用刑,盡得中正,循治民之道以治於民,輔成常教。\CJKunderline{伯夷}所典之禮,是常行之教也。 \par}

{\noindent\shu\zihao{5}\fzkt “穆穆”至“棐彝”。正義曰:言堯躬行敬敬之道在於上位,三後之徒躬秉明德明君道在於下,君臣敬明與德,灼然著於四方,故天下之事無不惟德之勤,悉皆勤行德矣。天下之士皆勤立德,故乃能明於用刑之中正,循大道以治於民,輔成常教。美堯君臣明德,能用刑得中以輔禮教。 \par}

典獄,非訖於威,惟訖於富。\footnote{言堯時主獄,有威有德有恕,非絕於威,惟絕於富。世治,貨賂不行。○賂,來故反。}敬忌,罔有擇言在身。\footnote{堯時典獄皆能敬其職,忌其過,故無有可擇之言在其身。}惟克天德,自作元命,配享在下。”\footnote{凡明於刑之中,無擇言在身,必是惟能天德,自為大命,配享天意,在於天下。}


{\noindent\zhuan\zihao{6}\fzbyks 傳“言堯”至“不行”。正義曰:堯時主獄之官,有威嚴,有德行,有恕心。有犯罪必罪之,是“有威”也。無罪則赦之,是“有德”也。有威有德有恕心,行之不受貨賂,是恕心也。“訖”是盡也,故傳以“訖”為絕。不可能使民不犯,非絕於威。能使不受貨賂,惟絕於富。言以恕心行之,世治則貨賂不行,故獄官無得富者。得凡明”至“天下”。正義曰:“惟克天德”,言能效天為德,當謂天德平均,獄官效天為平均。凡能明於刑之中正矣,又能使無可擇之言在身者,此人必是惟能為天平均之德,斷獄必平矣。“皇天無親,惟德是輔”,若能斷獄平均者,必壽長久大命。大命由己而來,是“自為大命”。“享”訓當也,是此人能配當天命,在於天之下。鄭云:“大命謂延期長久也。” \par}

{\noindent\shu\zihao{5}\fzkt “典獄”至“在下”。正義曰:堯時典獄之官,非能止絕於威,有犯必當行威,威刑不可止也。惟能止絕於富,受貨然後得富,無貨富自絕矣。言於時世治,貨賂不行。堯時典獄之官皆能敬其職事,忌其過失,無有可釋之言在於其身。天德平均,惟能為天之德。志性平均,自為長久大命。配當天意,在於天下。言堯德化之深,於時典獄之官皆能賢也。 \par}

王曰:“嗟!四方司政典獄,非爾惟作天牧?\footnote{主政典獄,謂諸侯也。非汝惟為天牧民乎?言任重是汝。○為,於偽反。任,而鴆反。重,輕重之重。}今爾何監,非時\CJKunderline{伯夷}播刑之迪?\footnote{言當視是\CJKunderline{伯夷}布刑之道而法之。}其今爾何懲?惟時苗民匪察於獄之麗。\footnote{其今汝何懲戒乎?所懲戒惟是苗民非察於獄之施刑,以取滅亡。○麗,力馳反。}罔擇吉人,觀於五刑之中,惟時庶威奪貨,\footnote{言苗民無肯選擇善人,使觀視五刑之中正,惟是眾為威虐者任之,以奪取人貨,所以為亂。}斷制五刑,以亂無辜。上帝不蠲,降咎於苗。\footnote{苗民任奪貨奸人,斷制五刑,以亂加無罪。天不潔其所為,故下咎罪。謂誅之。○蠲,吉緣反。咎,其九反。}苗民無辭於罰,乃絕厥世。”\footnote{言罪重,無以辭於天罰,故堯絕其世。申言之為至戒。}


{\noindent\zhuan\zihao{6}\fzbyks 傳“言當”至“法之”。正義曰:\CJKunderline{伯夷}典禮,\CJKunderline{皋陶}主刑,刑禮相成以為治。不使視\CJKunderline{皋陶}而令視\CJKunderline{伯夷}者,欲其先禮而後刑。道之以禮,禮不從乃刑之,則刑亦\CJKunderline{伯夷}之所布,故令視\CJKunderline{伯夷}布刑之道而法之。\CJKunderline{王肅}云:“\CJKunderline{伯夷}道之以禮,齊之以刑。” \par}

{\noindent\zhuan\zihao{6}\fzbyks 傳“其今”至“滅亡”。正義曰:上言“非時”,此言“惟時”,文異者,“非時”者言豈非是事也,“惟時”者言惟當是事也,雖文異而意同。“惟是苗民非察於獄之施刑以取滅亡”也,言其正謂察於獄之施刑不當於罪以取滅亡。 \par}

{\noindent\zhuan\zihao{6}\fzbyks 傳“苗民”至“誅之”。正義曰:“以亂加無罪”者,正謂以罪加無罪,是亂也。“蠲”訓絜也。“天不絜其所為”者,\CJKunderline{鄭玄}云:“天以苗民所行腥臊不絜,故下禍誅之。” \par}

{\noindent\shu\zihao{5}\fzkt “王曰”至“厥世”。正義曰:王呼諸侯戒之曰:“諮嗟!汝四方主政事典獄訟者諸侯之君等,非汝惟為天牧養民乎?”言汝等皆為天養民,言任重也。“受任既重,當觀古成敗,今汝何所監視乎?其所視者,非是\CJKunderline{伯夷}布刑之道也”。言當效\CJKunderline{伯夷}善布刑法,受令名也。“其今汝何所懲創乎?其所創者惟是苗民非察於獄之施刑乎?”言當創苗民施刑不當取滅亡也。“彼苗民之為政也,無肯選擇善人,使觀視於五刑之中正,惟是眾為威虐者任之,以奪取人之貨賂,任用此人,使斷制五刑,以亂加無罪之人。上天不絜其所為,故下咎惡於苗民。苗民無以辭於天罰,堯乃絕滅其世。汝等安得不懲創乎!” \par}

王曰:“嗚呼!念之哉!\footnote{念以\CJKunderline{伯夷}為法,苗民為戒。}伯父、伯兄、仲叔、季弟、幼子、童孫,皆聽朕言,庶有格命。\footnote{皆石同姓,有父兄弟子孫。列者伯仲叔季,順少長也。舉同姓包異姓,言不殊也。聽從我言,庶幾有至命。○聽如字,又他經反。少,詩照反。長,丁丈反。}今爾罔不由慰日勤,爾罔或戒不勤。\footnote{今汝無不用安自居,日當勤之。汝無有徒念戒而不勤。○日,人實反,一音曰。}天齊於民,俾我一日,非終惟終在人。\footnote{天整齊於下民,使我為之,一日所行,非為天所終,惟為天所終,在人所行。○天齊於民,絕句。馬云:“齊,中也。”俾我,絕句。俾,必爾反,馬本作矜;矜,哀也。}爾尚敬逆天命,以奉我一人。雖畏勿畏,雖休勿休,\footnote{汝當庶幾敬逆天命,以奉我一人之戒。行事雖見畏,勿自謂可敬畏。雖見美,勿自謂有德美。}惟敬五刑,以成三德。一人有慶,兆民賴之,其寧惟永。”\footnote{先戒以勞謙之德,次教以惟敬五刑,所以成剛柔正直之三德也。天子有善,則兆民賴之,其乃安寧長久之道。}


{\noindent\zhuan\zihao{6}\fzbyks 傳“皆王”至“至命”。正義曰:此總告諸侯,不獨告同姓,知“舉同姓包異姓”也。“格”訓至也,言庶幾有至命。“至命”當謂至善之命,不知是何命也。\CJKunderline{鄭玄}云:“格,登也。登命謂壽考者。”傳云“至命”亦謂壽考。 \par}

{\noindent\zhuan\zihao{6}\fzbyks 傳“今汝”至“不勤”。正義曰:“由”,用也。“慰”,安也。人之行事多有始無終,從而不改。王既殷勤教誨,恐其知而不行,或當曰欲勤行而中道倦怠,故以此言戒之。今汝等諸侯無不用安道以自居,言曰我當勤之。“安道”者,謂勤其職,是安之道。若不勤其職,是危之道也。 \par}

{\noindent\zhuan\zihao{6}\fzbyks 傳“天整”至“所行”。正義曰:“天整齊於下民”者,欲使之順道依理,以性命自終也。以民不能自治,故使我為之,使我為天子。我既受天委付,務欲稱天之心。墜失天命,是不為天所終。保全祿位,是為天所終。我一日所行善之與惡,非為天所終,惟為天所終,皆在人所行。王言已冀欲使為行稱天意也。 \par}

{\noindent\zhuan\zihao{6}\fzbyks 傳“汝當”至“德美”。正義曰:“逆”,迎也。上天授人為主,是下天命也。諸侯上輔天子,是逆天命也,言與天意相迎逆也。“汝當庶幾敬逆天命,以奉我一人之戒”,欲使之順天意而用己命。凡人被人畏,必當自謂己有可畏敬;被人譽,必自謂已實有德美。故戒之,汝等所行事,雖見畏,勿自謂可敬畏;雖見美,勿自謂有德美。教之令謙而不自恃也。 \par}

{\noindent\zhuan\zihao{6}\fzbyks 傳“先戒”至“之道”。正義曰:上句“雖畏勿畏,雖休勿休”,是“先戒以勞謙之德”也。“勞謙”,\CJKunderwave{易·謙卦}九三爻辭。謙則心勞,故云“勞謙”。天子有善,以善事教天下,則兆民蒙賴之。 \par}

{\noindent\shu\zihao{5}\fzkt “王曰”至“惟永”。正義曰:王言而嘆曰:“嗚呼!汝等諸侯其當念之哉!”念以\CJKunderline{伯夷}為法,苗民為戒。既令念此法戒,又呼同姓諸侯曰:“伯父、伯兄、仲叔、季弟、幼子、童孫等,汝皆聽從我言,依行用之,庶幾有至善之命,命必長壽也。今汝等諸侯無不用安道以自居,曰我當勤之哉。汝已許自勤,即當必勤,汝無有徒念我戒,許欲自勤而身竟不勤。”戒使必自勤也。“上天欲整齊於下民,使我為之令,我為天子整齊下民也。我一日所行失其道,非為天所終。一日所行得其理,惟為天所終。此事皆在人所行”。言已當慎行以順天也。“我已冀欲順天,汝等當庶幾敬逆天命,以奉用我一人之戒。汝所行事,雖見畏,勿自謂可敬畏。雖見美,勿自謂有德美”。欲令其謙而勿自取也。“汝等惟當敬慎用此五刑,以成剛柔正直之三德,以輔我天子。我天子一人有善事,則億兆之民蒙賴之。若能如此,其乃安寧,惟久長之道也”。 \par}

王曰:“吁!來,有邦有土,告爾祥刑。\footnote{籲,嘆也。有國土諸侯,告汝以善用刑之道。○籲,況於反,馬作於;於,於也。}在今爾安百姓,何擇非人?何敬非刑?何度非及?\footnote{在今爾安百姓兆民之道,當何所擇?非惟吉人乎?當何所敬?非惟五刑乎?當何所度?非惟及世輕重所宜乎?○度,待洛反,注同,馬云:“造謀也。”}兩造具備,師聽五辭。\footnote{兩謂囚、證。造,至也。兩至具備,則眾獄官共聽其入五刑之辭。○造,七報反,注同。}五辭簡孚,正於五刑。\footnote{五辭簡核,信有罪驗,則正之於五刑。○核,幸革反。}


{\noindent\zhuan\zihao{6}\fzbyks 傳“在今”至“宜乎”。正義曰:“何度非及”,其言不明。以論刑事,而言度所及,知所度者,度及世之用刑輕重所宜。\CJKunderline{王肅}云:“度,謀也。非當與主獄者謀慮刑事,度世輕重所宜也。” \par}

{\noindent\zhuan\zihao{6}\fzbyks 傳“兩謂”至“之辭”。正義曰:“兩”謂兩人,謂囚與證也。凡競獄必有兩人為敵,各言有辭理。或時兩皆須證,則囚之與證非徒兩人而已。兩人謂囚與證,不為兩敵至者,將斷其罪,必須得證,兩敵同時在官,不須待至;且兩人競理,或並皆為囚,各自須證,故以“兩”為囚與證也。兩至具備,謂囚證具足。各得其辭,乃據辭定罪。與眾獄官共聽其辭,觀其犯狀,斟酌入罪,或入墨劓,或入宮剕,故云“聽其入五刑之辭”也。 \par}

{\noindent\zhuan\zihao{6}\fzbyks 傳“五辭”至“五刑”。正義曰:既得囚證將入五刑之辭,更復簡練核實,知其信有罪狀,與刑書正同,則依刑書斷之,應墨者墨之,應殺者殺之。 \par}

五刑不簡,正於五罰。\footnote{不簡核,謂不應五刑。當正五罰,出金贖罪。○應,應對之應,下同。}五罰不服,正於五過。\footnote{不服,不應罰也。正於五過,從赦免。}五過之疵:惟官,惟反,惟內,惟貨,惟來。\footnote{五過之所病,或嘗同官位,或詐反囚辭,或內親用事,或行貨枉法,或舊相往來,皆病所在。○疵,才斯反。來,馬本作求,云:“有求,請賕也。”}


{\noindent\zhuan\zihao{6}\fzbyks 傳“不簡”至“贖罪”。正義曰:“不簡核”者謂覆審囚證之辭,不如簡核之狀。既囚與證辭不相符合,則是犯狀不定,謂“不應五刑”。不與五刑書同,獄官疑不能決,則當正之於五罰,令其出金贖罪。依準五刑,疑則從罰,故為“五罰”,即下文是也。今律:“疑罪各依所犯以贖。”論虛實之證,等是非之理,均或事涉疑似,旁無證見,或雖有證見,事非疑似,如此者皆為疑罪。 \par}

{\noindent\zhuan\zihao{6}\fzbyks 傳“不服”至“赦免”。正義曰:“不服,不應罰”者,欲令贖罪,而其人不服,獄官重加簡核,無復疑似之狀,本情非罪,不可強遣出金,如是者則正之於五過。雖事涉疑似有罪,乃是過失,過則可原,故從赦免。下文惟有“五刑”、“五罰”而無“五過”,亦稱“五”者,緣五罰為過,故謂之“五過”。五者之過,皆可原也。 \par}

{\noindent\zhuan\zihao{6}\fzbyks 傳“五過”至“所在”。正義曰:\CJKunderwave{釋詁}云:“疵,病也。”此五過之所病,皆謂獄吏故出入人罪,應刑不刑,應罰不罰,致之五過而赦免之,故指言“五過之疵”。於五刑五罰,不赦其罪,未有此病,故不言“五刑之疵”、“五罰之疵”。應刑而罰,亦是其病,於赦免言病,則赦刑從罰亦是病可知。損害王道,於政為病,故謂之“病”。“惟官”謂嘗同官位,與吏舊同僚也。“或詐反囚辭”,拒諱實情,不承服也。“或內親用事”,囚有親戒在官吏,或望其意而曲筆也。或行貨於吏,吏受財枉法也。或囚與吏舊相往來。此五事皆是病之所在。五事皆是枉法,但枉法多是為貨,故於“貨”言“枉”,餘皆枉可知。 \par}

其罪惟均,其審克之。\footnote{以病所在,出入人罪,使在五過,罪與犯法者同。其當清察,能使之不行。}五刑之疑有赦,五罰之疑有赦,其審克之。\footnote{刑疑赦從罰,罰疑赦從免。其當清察,能得其理。}簡孚有眾,惟貌有稽。\footnote{簡核誠信,有合眾心。惟察其貌,有所考合,重刑之至。}


{\noindent\zhuan\zihao{6}\fzbyks 傳“以病”至“不行”。正義曰:以五病所在,出入人罪,不罰不刑使得在於五過,妄赦免之,此獄吏之罪與犯法者同。諸侯國君清證審察,能使之不行,乃為善也。此以病所在,惟出人罪耳,而傳並言“入”者,有罪而妄出與無罪而妄入,獄吏之罪等,故以“出入”言之。今律:“故出入者與同罪。”而此是也。 \par}

{\noindent\zhuan\zihao{6}\fzbyks 傳“刑疑”至“其理”。正義曰:刑疑有赦,赦從罰也。罰疑有赦,赦從免也。上云“五罰不服,正於五過”,即是免之也。不言五過之疑有赦者,知過則赦之,不得疑也。“其當清察,能得其理”,不使應刑妄得罰,應罰妄得免也。\CJKunderwave{舜典}云“眚災肆赦”,\CJKunderwave{大禹謨}云“宥過無大”,\CJKunderwave{易·解卦}象云“君子以赦過宥罪”,\CJKunderwave{論語}云“赦小過”,是過失之罪,皆當赦放,故知過即是赦之。\CJKunderline{鄭玄}云:“不言五過之疑有赦者,過不赦也。\CJKunderwave{禮記}云:‘凡執禁以齊眾者,不赦過。’”如鄭此言,五罰不服正於五過者,五過皆當罪之也。五刑之疑赦刑取贖,五罰疑者反使服刑,是刑疑而輸贖,罰疑而受刑,不疑而更輕,可疑而益重,事之顛倒一至此乎?謂之“祥刑”,豈當若是?然則“不赦過”者,復何所謂“執禁以齊眾”非謂之平常之過失也。人君故設禁約,將以齊整大眾,小事易犯,人必輕之,過犯悉皆赦之,眾人不可復禁,是故不赦小過,所以齊整眾人,令其不敢犯也。今律:“和合御藥誤不如本方,御幸舟舡誤不牢固,罪皆死。乏軍興者斬。”故失等皆是不赦過也。 \par}

{\noindent\zhuan\zihao{6}\fzbyks 傳“簡核”至“之”至“”。正義曰:“簡核誠信,有合眾心”,或皆以為可刑,或可以為赦,未得即斷之,惟當察其囚貌,更有所考合,考合復同,乃從眾議斷之,重刑之至也。“察其貌”者,即\CJKunderwave{周禮}五聽,辭聽、色聽、氣聽、耳聽、目聽也。\CJKunderline{鄭玄}以為辭聽“觀其出言,不直則煩”;色聽“觀其顏色,不直則赧然”;氣聽“觀其氣息,不直則喘”;耳聽“觀其聽聆,不直則惑”;目聽“觀其眸子,視不直則眊然”。是“察其貌,有所考合”也。 \par}

無簡不聽,具嚴天威。\footnote{無簡核誠信,不聽理具獄,皆當嚴敬天威,無輕用刑。}墨闢疑赦,其罰百鍰,閱實其罪。\footnote{刻其顙而涅之曰墨刑,疑則赦從罰。六兩曰鍰。鍰,黃鐵也。閱實其罪,使與罰名相當。○闢,婢亦反。鍰,徐戶關反,六兩也。鄭及\CJKunderwave{爾雅}同。\CJKunderwave{說文}云:“六鋝也。”“鋝,十一銖二十五分述之十三也。”馬同,又云:“賈逵說俗儒以鋝重六兩,\CJKunderwave{周官}劍重九鋝,俗儒近是。”閱音悅。顙,素黨反。涅,乃結反。}劓闢疑赦,其罰惟倍,閱實其罪。\footnote{截鼻曰劓。刑倍百為二百鍰。}剕闢疑赦,其罰倍差,閱實其罪。\footnote{刖足曰剕。倍差謂倍之又半,為五百鍰。}


{\noindent\zhuan\zihao{6}\fzbyks 傳“無簡”至“用刑”。正義曰:“無簡核誠信”者,謂簡核之,於罪無誠信效驗可簡核,即是無罪之人,當赦之。 \par}

{\noindent\zhuan\zihao{6}\fzbyks 傳“刻其”至“相當”。正義曰:五刑之名,見於經傳,\CJKunderline{唐}、\CJKunderline{虞}已來皆有之矣,未知上古起在何時也。漢文帝始除肉刑,其刻顙、截鼻、刖足、割勢皆法傳於先代,\CJKunderline{孔君}親見之。\CJKunderwave{說文}云:“顙,頟也。”“墨”一名黥。\CJKunderline{鄭玄}\CJKunderwave{周禮}注云:“墨,黥也。先刻其面,以墨窒之。”言刻頟為瘡,以墨塞瘡孔,令變色也。“六兩曰鍰”,蓋古語,存於當時,未必有明文也。\CJKunderwave{考工記}云,戈矛重三鋝。\CJKunderline{馬融}云:“鋝,量名。當與\CJKunderwave{呂刑}鍰同。俗儒雲鋝六兩為一川,不知所出耳。”\CJKunderline{鄭玄}云:“鍰,稱輕重之名。今代、東萊稱,或以太半兩為鈞,十鈞為鍰,鍰重六兩太半兩。鍰、鋝似同也。或有存行之者,十鈞為鍰,二鍰四鈞而當一斤,然則鍰重六兩三分兩之二。\CJKunderwave{周禮}謂鍰為鋝。”如\CJKunderline{鄭玄}之言,一鍰之重六兩,多於孔、王所說,惟校十六銖爾。\CJKunderwave{舜典}云:“金作贖刑。”傳以金為黃金,此言“黃鐵”者,古者金銀銅鐵總號為“金”,今別之以為四名,此傳言“黃鐵”,\CJKunderwave{舜典}傳言“黃金”,皆是今之銅也。古人贖罪悉皆用銅,而傳或稱“黃金”,或言“黃鐵”,謂銅為金為鐵爾。“閱實其罪”,撿閱核實其所犯之罪,使與罰名相當,然後收取其贖。此既罪疑而取贖,疑罪不定,恐受贖參差,故五罰之下皆言“閱實其罪”,慮其不相當故也。 \par}

{\noindent\zhuan\zihao{6}\fzbyks 傳“刖足”至“百鍰”。正義曰:\CJKunderwave{釋詁}云:“剕,刖也。”李巡云:“斷足曰刖。”\CJKunderwave{說文}云:“刖,絕也。”是“刖”者斷絕之名,故“刖足曰剕”。贖劓倍墨,剕應倍劓,而云“倍差”,倍之又有差,則不啻一倍也。下句贖宮六百鍰,知倍之又半之為五百鍰也。截鼻重於黥頟,相校猶少。刖足重於截鼻,所校則多。刖足之罪,近於宮刑,故使贖剕不啻倍劓,而多少近於贖宮也。 \par}

宮闢疑赦,其罰六百鍰,閱實其罪。\footnote{宮,淫刑也。男子割勢,婦人幽閉,次死之刑。序五刑,先輕轉至重者,事之宜。}大辟疑赦,其罰千鍰,閱實其罪。\footnote{死刑也。五刑疑各入罰,不降相因,古之制也。}墨罰之屬千,劓罰之屬千,剕罰之屬五百,宮罰之屬三百,大辟之罰其屬二百。五刑之屬三千。\footnote{別言罰屬,合言刑屬,明刑罰同屬,互見其義以相備。○見,賢遍反。}

{\noindent\zhuan\zihao{6}\fzbyks 傳“宮淫”至“之宜”。正義曰:\CJKunderline{伏生}\CJKunderwave{書傳}云:“男女不以義交者,其刑宮。”是宮刑為淫刑也。男子之陰名為勢,割去其勢,與椓去其陰,事亦同也。“婦人幽閉”,閉於宮使不得出也。本制宮刑,主為淫者,後人被此罪者,未必盡皆為淫。昭五年\CJKunderwave{左傳}楚子“以羊舌肸為司宮”,非坐淫也。漢除肉刑,除墨、劓、剕耳,宮刑猶在。近代反逆緣坐,男子十五已下不應死者皆宮之。大隋開皇之初,始除男子宮刑,婦人猶閉於宮。宮是次死之刑,宮於四刑為最重也。人犯輕刑者多,犯重刑者少,又以鍰數以倍相加,序五刑先輕後重,取事之宜。 \par}

{\noindent\zhuan\zihao{6}\fzbyks 傳“死刑”至“制也”。正義曰:\CJKunderwave{釋詁}云:“闢,罪也。”死是罪之大者,故謂死刑為“大辟”。經歷陳罰之鍰數,五刑之疑各自入罰。“不降相因”,不合死疑入宮,宮疑入剕者,是古之制也。所以然者,以其所犯疑不能決,故使贖之。次刑非其所犯,故不得降相因。 \par}

{\noindent\zhuan\zihao{6}\fzbyks 傳“別言”至“相備”。正義曰:此經歷言“二百”、“三百”、“五百”者,各是刑之條也。每於其條有犯者,實則刑之,疑則罰之,刑屬罰屬其數同也。別言罰屬,五者各言其數,合言刑屬,但總云“三千”,明刑罰同其屬數,互見其義以相備也。經云“大辟之罰,其屬二百”,文異於上四罰者,以“大辟”二字不可云“大辟罰之屬”,故分為二句,以其二字足使成文。 \par}

{\noindent\shu\zihao{5}\fzkt “王曰”至“天威”。正義曰:凡與人言,必呼使來前。“籲”,嘆聲也。王嘆而呼諸侯曰:“吁!來,有邦國、有土地諸侯國君等,告汝以善用刑之道。在於今日,汝安百姓兆民之道,何所選擇?非惟選擇善人乎?何所敬慎?非惟敬慎五刑乎?何所謀度?非惟度及世之用刑輕重所宜乎”即教諸侯以斷獄之法。“凡斷獄者,必令囚之與證兩皆來至。囚證具備,取其言語,乃與眾獄官共聽其入五刑之辭。其五刑之辭簡核,信實有罪,則正之於五刑,以五刑之罪罪其身也。五刑之辭不如眾所簡核,不合入五刑,則正之於五罰。罰謂其取贖也。於五罰論之,又有辭不服,則正之於五過,過失可宥,則教宥之。從刑入罰,從罰入過。此五過之所病者,惟嘗同官位,惟詐反囚辭,惟內親用事,惟行貨枉法,惟舊相往來。以此五病出入人罪,其罪與犯法者均。其當清證審察,能使五者不行,乃為能耳。五刑之疑有赦,赦從罰也。五罰之疑有赦,赦從過也,過則赦之矣。其當清證審察使能之,勿使妄入人罪,妄得赦免。既得囚辭,簡核誠信,有合眾心。或記可刑,或皆可放,雖雲合罪,惟更審察其貌,有所考合”。謂貌又當罪,乃決斷之。“無簡不聽”者,謂雖似罪狀,無可簡核誠信合罪者,則不聽理其獄,當放赦之。皆當嚴敬天威,勿輕聽用刑也。 \par}

上下比罪,無僣亂辭,勿用不行,\footnote{上下比方其罪,無聽僣亂之辭以自疑,勿用折獄,不可行。○僣,子念反。}惟察惟法,其審克之。\footnote{惟當清察罪人之辭,附以法理,其當詳審能之。}上刑適輕,下服。\footnote{重刑有可以虧減則之輕,服下罪。}下刑適重,上服。輕重諸罰有權。\footnote{一人有二罪,則之重而輕並數。輕重諸刑罰各有權宜。○並,必政反。數,色住反。}刑罰世輕世重,惟齊非齊,有倫有要。\footnote{言刑罰隨世輕重也。刑新國用輕典,刑亂國用重典,刑平國用中典。凡刑所以齊非齊,各有倫理,有要善。}


{\noindent\zhuan\zihao{6}\fzbyks 傳“上下”至“可行”。正義曰:罪條雖有多數,犯者未必當條,當取故事並之,上下比方其罪之輕重。上比重罪,下比輕罪,觀其所犯當與誰同。獄官不可盡賢,其間或有阿曲,宜預防之。“僣”,不信也。獄官與囚等或作不信之辭,以惑亂在上,人君無得聽此僣亂之辭以自疑惑,勿即用此僣亂之辭以之斷獄,此僣亂之言不可行用也。 \par}

{\noindent\zhuan\zihao{6}\fzbyks 傳“一人”至“權宜”。正義曰:“一人有二罪,則之重而輕並數”者,謂若一人有二罪,則應兩罪俱治,今惟斷獄以重條,而輕者不更別數,與重並數為一。劉君以為“上刑適輕、下刑適重皆以為一人有二罪。上刑適輕者,若今律重罪應贖,輕罪應居作官當者,以居作官當為重,是為上刑適輕。下刑適重者,謂若二者俱是贓罪,罪從重科,輕贓亦備,是為而輕並數也”。知不然者,案經既言“下刑適重,上服”,則是重上服而已,何得為輕贓亦備?又今律云“重罪應贖,輕重應居作官當者,以居作官當為重”者,此即是下刑適重之條,而以為上刑適輕之例,實為未允。且孔傳下經始云“一人有二罪”,則上經所云非一人有二罪者也。劉君妄為其說,故今不從。 \par}

{\noindent\zhuan\zihao{6}\fzbyks 傳“言刑”至“要善”。正義曰:“刑罰隨世輕重”,言觀世而制刑也。“刑新國用輕典,刑亂國用重典,刑平國用中典”,\CJKunderwave{周禮·大司寇}文也。\CJKunderline{鄭玄}云:“新國者,新闢地立君之國。用輕法者,為其民未習於教也。平國,承平守成之國。用中典者,常行之法也。亂國,篡弒叛逆之國。用重典者,以其化惡,伐滅之也。” \par}

{\noindent\shu\zihao{5}\fzkt “上下”至“有要”。正義曰:此又述斷獄之法。將斷獄訟,當上下比方其罪之輕重,乃與獄官眾議斷之。其囚有僣亂之虛辭者,無得聽之,勿用此辭斷獄,此僣亂之辭,言不可行也。惟當清察罪人之辭,惟當附以法理,其當詳審使能之,勿使僣失為不能也。“上刑適輕”者,謂一人雖犯一罪,狀當輕重兩條,據重條之上有可以虧減者,則之輕條,服下罪也。“下刑適重”者,謂一人之身輕重二罪俱發,則以重罪而從上服,令之服上罪。或輕或重,諸所罪罰,皆有權宜,當臨時斟酌其狀,不得雷同加罪。刑罰有世輕世重,當視世所宜,權而行之。行罰者所以齊非齊者,有倫理,有要善。戒令審量之。 \par}

罰懲非死,人極於病。\footnote{刑罰所以懲過,非殺人,欲使惡人極於病苦,莫敢犯者。}非佞折獄,惟良折獄,罔非在中。\footnote{非口才可以斷獄,惟平良可以斷獄,無不在中正。}察辭於差,非從惟從。\footnote{察囚辭其難在於差錯,非從其偽辭,惟從其本情。}哀敬折獄,明啟刑書,胥佔,咸庶中正。\footnote{當憐下人之犯法,敬斷獄之害人,明開刑書,相與佔之,使刑當其罪,皆庶幾必得中正之道。○當,丁浪反。}其刑其罰,其審克之。\footnote{其所刑,其所罰,其當詳審能之,無失中正。}



{\noindent\zhuan\zihao{6}\fzbyks 傳“當憐”至“之道”。正義曰:\CJKunderwave{論語}云,陽膚為士師,曾子戒之云:“如得其情,則哀矜而勿喜。”是斷獄者於斷之時,當憐下民之犯法也。死者不可復生,斷者不可復續,當須敬慎斷獄之害人,勿得輕耳即決之。五刑之屬三千,皆著在刑書,使斷獄者依案用之,宜令斷獄諸官明開刑書,相與佔之,使刑書當其罪。令人之所犯,不必當條,須探測刑書之意,比附以斷其罪,若卜筮之佔然,故稱“佔”也。“皆庶幾必得中正之道”,令獄官同心思使中也。此言“明啟刑書”,而\CJKunderwave{左傳}云“昔先王議事以制,不為刑辟”者,彼鑄刑書以宣示百姓,故云臨事制宜,不預明刑辟。人有犯罪,原其情之善惡,斷定其輕重,乃於刑書比附而罪之。故彼此各據其一,義不相違也。 \par}

獄成而孚,輸而孚。\footnote{斷獄成辭而信,當輸汝信於王。謂上其鞫劾文辭。○上,時掌反,下注同。鞫,九六反。劾,亥代反,\CJKunderwave{玉篇}胡得反。}其刑上備,有並兩刑。”\footnote{其斷刑文書上王府皆當備具,有並兩刑,亦具上之。}

{\noindent\zhuan\zihao{6}\fzbyks 傳“斷獄”至“文辭”。正義曰:“孚”,信也。“輸”,寫也。下“而”為汝也。斷獄成辭而得信實,當輸寫汝之信實以告於王,勿藏隱其情不告王也。曲必隱情,直則無隱,令其不隱情者,欲使之無阿曲也。漢世問罪謂之“鞫”,斷獄謂之“劾”,謂上其鞫劾文辭也。 \par}

{\noindent\zhuan\zihao{6}\fzbyks 傳“其斷”至“上之”。正義曰:“其斷刑文書上王府皆當備具”,若今曹司寫案申尚書省也。“有並兩刑”,謂人犯兩事,刑有上下,雖罪從重斷,有兩刑者,亦並具上之,使王知其事。王或時以下刑為重,改下為上,故並亦上之。 \par}

{\noindent\shu\zihao{5}\fzkt “罰懲”至“兩刑”。正義曰:言聖人之制刑罰,所以懲創罪過,非要使人死也,欲使惡人極於病苦,莫敢犯之而已。非口才辯佞之人可以斷獄,惟良善之人乃可以斷獄。言斷獄無非在其中正,佞人即不能然也。察囚之辭其難在於言辭差錯,斷獄者非從其偽辭,惟從其本情。斷獄之時,當哀憐之下民之犯法,敬慎斷獄之害人,勿得輕耳斷之,必令典獄諸官明開刑書,相與佔之,皆無幾得中正之道,其所刑罰,其當詳審能之,勿使失中。其斷獄成辭,得其信實,又當輸汝信實之狀而告於王。其斷刑文書上於王府,皆使備具,勿有疏漏。其囚若犯二事,罪雖從重,有並兩刑上之者,言有兩刑,亦具上之。恐獄官有所隱沒,故戒之。 \par}

王曰:“嗚呼!敬之哉!官伯、族姓,朕言多懼。\footnote{敬之哉,告使敬刑。官長,諸侯。族,同族。姓,異姓也。我言多可戒懼,以儆之。○儆音景。}朕敬於刑,有德惟刑。\footnote{我敬於刑,當使有德者惟典刑。}今天相民,作配在下,明清於單辭。\footnote{今天治民,人君為配天在下,當承天意,聽訟當清審單辭。單辭特難聽,故言之。○相如字,馬息亮反,助也。}


{\noindent\zhuan\zihao{6}\fzbyks 傳“敬之”至“儆之”。正義曰:此篇主多戒諸侯百官之長,故知“官長”即諸侯也。襄十二年\CJKunderwave{左傳}哭諸侯之例云:“異姓臨於外,同族於禰廟。”是相對則“族”為同姓,“姓”為異姓也。告之以“我言多可戒懼”者,以儆戒之也。下言民無善政,則天罰人主,是儆戒諸侯也。 \par}

{\noindent\zhuan\zihao{6}\fzbyks 傳“我敬”至“典刑”。正義曰:“當使有德者惟典刑”,言將選有德之人使為刑官,刑官不用無德之人也。 \par}

{\noindent\zhuan\zihao{6}\fzbyks 傳“今天”至“言之”。正義曰:傳以“相”為治,“今天治民”者,天有意治民,而天不自治,使人治之。人君為配天在下,當承天意治民,治之當使稱天心也。欲稱天心,聽獄當清審單辭。“單辭”謂一人獨言,未有與對之人。訟者多直已以曲彼,構辭以誣人,單辭特難聽,故言之也。\CJKunderline{孔子}美子路云:“片言可以折獄者,其由也與。”“片言”即“單辭”也。子路行直聞於天下,不肯自道已長,妄稱彼短,得其單辭即可以斷獄者,惟子路爾。凡人少能然,故難聽也。 \par}

民之亂,罔不中聽獄之兩辭,\footnote{民之所以治,由典獄之無不以中正聽獄之兩辭,兩辭棄虛從實,刑獄清則民治。○治,直吏反。}無或私家於獄之兩辭。\footnote{典獄無敢有受貨聽詐,成私家於獄之兩辭。}獄貨非寶,惟府辜功,報以庶尤。\footnote{受獄貨非家寶也,惟聚罪之事,其報則以眾人見罪。}


{\noindent\zhuan\zihao{6}\fzbyks 傳“民之”至“民治”。正義曰:“獄之兩辭”,謂兩人競理,一虛一實,實者枉屈,虛者得理,則此民之所以不得治也。民之所以得治者,由典獄之官其無不以有中正之心聽獄之兩辭,棄虛從實,實者得理,虛者受刑,虛者不敢更訟,則刑獄清而民治矣。\CJKunderline{孔子}稱“必也使無訟乎”,謂此也。 \par}

{\noindent\zhuan\zihao{6}\fzbyks 傳“典獄”至“兩辭”。正義曰:典獄知其虛,受其貨,而聽其詐。詐者虛而得理,獄官致富成私家,此民之所以亂也。故戒諸侯無使獄官成私家於獄之兩辭。 \par}

{\noindent\zhuan\zihao{6}\fzbyks 傳“受獄”至“見罪”。正義曰:“府”,聚也。“功”,事也。受獄貨非是家之寶也,惟最聚近罪之事爾。罪多必有惡報,其報則以眾人見罪也。眾人見罪者多,天必報以禍罰,故下句戒令畏天罰也。 \par}

永畏惟罰,非天不中,惟人在命。\footnote{當長畏懼惟為天所罰,非天道不中,惟人在教命使不中,不中則天罰之。}天罰不極,庶民罔有令政在於天下。”\footnote{天道罰不中,令眾民無有善政在於天下,由人主不中,將亦罰之。○令,力呈反。}

{\noindent\zhuan\zihao{6}\fzbyks 傳“當長”至“罰之”。正義曰:眾人見罪者多,天必報以禍罰,汝諸侯等當長畏懼為天所罰。天之罰人,非天道不得其中,惟人在其教命自使不申,教命不中,則天罰之。諸侯一,施教命於民者也,故戒以施教命中否也。 \par}

{\noindent\zhuan\zihao{6}\fzbyks 傳“天道”至“罰之”。正義曰:天道下罰,罰不中者,令使眾民無有善政在於天下,由人主不中。為人主不中,故無善政,天將亦罰人主。“人主”謂諸侯,此言戒諸侯也。 \par}

{\noindent\shu\zihao{5}\fzkt “王曰”至“天下”。正義曰:王嘆而呼諸侯曰:“嗚呼!刑罰事重,汝當敬之哉!謂諸侯官之長,此同族異姓等,我言多可戒懼。我敬於刑,當刑命有德者惟典刑事。今上天治民,命人君為天子,配天在於下,承天之意,為事甚重。其聽獄訟,當明白清審於獄之單辭。民之所以治者,由獄官無有不用中正聽訟之兩辭。由以中正之故,下民得治。汝獄官無有敢受貨賂,成私家於獄之兩辭。勿於獄之兩家受貨致富,治獄受貨非家寶也,惟是聚罪之事。”言汝身多違則不達,虛言戒行急惡,疏非虛論矣。“多聚罪則天報汝,以眾人見被尤怨而罰責之。汝當長畏惟天所罰,天罰汝者非是天道不中,惟人在於自作教命,使不中爾。教命不中,則天罰汝。天道罰不中也,若令眾民無有善政在於天下,則是人主不中,天亦將罰人主”。諸侯為民之主,故以天罰懼之。 \par}

王曰:“嗚呼!嗣孫,今往何監?非德於民之中?尚明聽之哉!\footnote{嗣孫,諸侯嗣世子孫,非一世。自今已往,當何監視?非當立德於民,為之中正乎?庶幾明聽我言而行之哉!}哲人惟刑,無疆之辭,屬於五極,咸中有慶。\footnote{言智人惟用刑,乃有無窮之善,辭名聞於後世。以其折獄屬五常之中正,皆中有善,所以然也。○屬音燭。}受王嘉師,監於茲祥刑。”\footnote{有邦有土受王之善眾而治之者,視於此善刑。欲其勤而法之,為無疆之辭。}


{\noindent\zhuan\zihao{6}\fzbyks 傳“言智”至“然也”。正義曰:“屬”謂屬著也。“極”,中也。“慶”,善也。“五常”謂仁義禮智信,人所常行之道也。言得有善辭,名聞於後世者,以其斷獄能屬著於五常之中正,皆得其理而法之有善,所以得然也。知“五”是五常者,以人所常行惟有五事,知是五常也。 \par}

{\noindent\shu\zihao{5}\fzkt “王曰”至“祥刑”。正義曰:戒之既終,王又言而嘆曰:“嗚呼!汝諸侯嗣世子孫等,從自今已往,當何所監視?非當視立德於民而為之中正乎?”言諸侯並嗣世惟當視此立德於民為之中正之事。“汝必視此,庶幾明聽我言而行之哉!有智之人惟能用刑,乃有無疆境之善辭。得有無疆善辭者,以其折獄能屬於五常之中正,皆中其理而法有善政故也。汝有邦有土之君,受王之善眾而治之,當視於此善刑。”從上已來舉善刑以告之,欲其勤而法之,使有無窮之美譽。 \par}

%%% Local Variables:
%%% mode: latex
%%% TeX-engine: xetex
%%% TeX-master: "../Main"
%%% End:
