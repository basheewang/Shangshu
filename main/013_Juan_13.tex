%% -*- coding: utf-8 -*-
%% Time-stamp: <Chen Wang: 2024-04-02 11:42:41>

% {\noindent \zhu \zihao{5} \fzbyks } -> 注 (△ ○)
% {\noindent \shu \zihao{5} \fzkt } -> 疏

\chapter{卷十三}


\section{旅獒第七【偽】}


西旅獻獒,\footnote{西戎遠國貢大犬。○獒,五羔反。馬云:“作豪,酋豪也。”}太保作\CJKunderwave{旅獒}。\footnote{\CJKunderline{召公}陳戒。○召,時照反,後“\CJKunderline{召公}”皆仿此。}

旅獒\footnote{因獒而陳道義。}


{\noindent\zhuan\zihao{6}\fzbyks 傳“西戎”至“大犬”。正義曰:“西旅”,西方夷名。西方曰“戎”,克商之後乃來,知是“西戎遠國”也。“獒”是犬名,故云“貢大犬”。 \par}

{\noindent\zhuan\zihao{6}\fzbyks 傳“\CJKunderline{召公}陳戒”。正義曰:\CJKunderline{成王}時\CJKunderline{召公}為太保,知此時“太保”亦\CJKunderline{召公}也。\CJKunderwave{釋詁}云:“旅,陳也。”故云“\CJKunderline{召公}陳戒”。上“旅”是國名,此“旅”訓為陳,二“旅”字同而義異。鄭云:“獒讀曰豪,西戎無君名,強大有政者為酋豪。國人遣其遒豪來獻見於周。”良由不見古文,妄為此說。 \par}

{\noindent\shu\zihao{5}\fzkt “西旅”至“旅獒”。正義曰:西方之戎有國名“旅”者,遣獻其大犬,其名曰“獒”,於是太保\CJKunderline{召公}因陳戒。史敘其事,作\CJKunderwave{旅獒}。 \par}

惟克商,遂通道於九夷八蠻。\footnote{四夷慕化,貢其方賄。九、八言非一。皆通道路,無遠不服。○賄,呼罪反。}西旅厎貢厥獒,\footnote{西戎之長,致貢其獒。犬高四尺曰獒,以大為異。○厎,之履反。長,丁丈反。}太保乃作\CJKunderwave{旅獒},用訓於王。\footnote{陳貢獒之義以訓諫王。}


{\noindent\zhuan\zihao{6}\fzbyks 傳“四夷”至“不服”。正義曰:\CJKunderwave{曲禮}云:“其在東夷、西戎、南蠻、北狄。”經舉“夷”、“蠻”則戎狄可知。“四夷慕化,貢其方賄”,言所貢非獨旅也。四夷各自為國,無大小統領,“九、八言非一也”。釋地云:“九夷、八狄、七戎、六蠻謂之四海。”又云:“八蠻在南方,六戎在西方,五狄在北方。”上下二文三方數目不同。\CJKunderwave{明堂位}稱九夷、八蠻、六戎、五狄,與\CJKunderwave{爾雅}上文不同。\CJKunderwave{周禮}職方氏掌四夷、八蠻、七閩、九貉、五戎、六狄之人。\CJKunderline{鄭玄}云:“四、八、七、九、五、六,周之所服國數也。”遍檢經傳,四夷之數,參差不同,先儒舊解,此\CJKunderwave{爾雅}殷制,\CJKunderwave{明堂位}及\CJKunderwave{職方}並\CJKunderwave{爾雅}下文雲八蠻在南,六戎在西,五狄在北,皆為周制,義或當然。\CJKunderwave{明堂位}言六戎、五狄,\CJKunderwave{職方}言五戎、六狄,趙商以此問鄭,鄭答云:“戎狄但有其國數,其名難得而知。”是鄭亦不能定解。言“克商,遂通道”,是王家遣使通道也。\CJKunderwave{魯語}引此事,韋昭云:“通道,譯使懷柔之。”是王家遣使通彼,彼聞命來獻也。言其通夷蠻而有戎貢,是四夷皆通道路,無所不服。 \par}

{\noindent\zhuan\zihao{6}\fzbyks 傳“西戎”至“為異”。正義曰:“西戎之長”謂旅國之君。“致貢其獒”,或遣使貢之,不必自來也。“犬高四尺曰獒”,\CJKunderwave{釋畜}文。\CJKunderwave{左傳}晉靈公有犬謂之獒。旅國以犬為異,故貢之也。 \par}

{\noindent\shu\zihao{5}\fzkt “惟克”至“於王”。正義曰:惟\CJKunderline{武王}既克商,華夏既定,遂開通道路於九夷八蠻,於是有西戎旅國致貢其大犬名獒。太保\CJKunderline{召公}乃作此篇,陳貢獒之義,用訓諫於王。 \par}

曰:“嗚呼!明王慎德,四夷咸賓。\footnote{言明王慎德以懷遠,故四夷皆賓服。}無有遠邇,畢獻方物,惟服食器用。\footnote{天下萬國無有遠近,盡貢其方土所生之物,惟可以供服食器用者。言不為耳目華侈。○供音恭。為,於偽反。侈,昌氏反,又式氏反。}王乃昭德之致於異姓之邦,無替厥服。\footnote{德之所致,謂遠夷之貢,以分賜異姓諸侯,使無廢其職。}分寶玉於伯叔之國,時庸展親。\footnote{以寶玉分同姓之國,是用誠信其親親之道。}


{\noindent\zhuan\zihao{6}\fzbyks 傳“天下”至“華侈”。正義曰:以言“無有遠近”,是華夷總統之辭。\CJKunderwave{釋詁}云:“畢,盡也。”故云天下萬國無有遠之與近,盡貢其方土所生之物。“惟可以供服食器用”者,玄纁絺紵供服也,橘柚菁茅供食也,羽毛齒革瑤琨篠簜供器用也。下言“不役耳目”,故知言“不為耳目華侈”也。\CJKunderwave{周禮·大行人}云:“九州之外謂之蕃國,世壹見,各以其所貴寶為贄。”\CJKunderline{鄭玄}云:“所貴寶見經傳者,犬戎獻白狼、白鹿是也,餘外則\CJKunderwave{周書·王會}備焉。”案\CJKunderwave{王會}篇諸方致貢,無所不有,此言“惟服食器用”者,遠方所貢雖不充於器用,實亦受之,\CJKunderline{召公}深戒\CJKunderline{武王},故言此耳。 \par}

{\noindent\zhuan\zihao{6}\fzbyks 傳“德之”至“其職”。正義曰:明王有德,四夷乃貢,是“德之所致,謂遠夷之貢”也。“昭德之致”,正謂賜異姓諸侯,令其見此遠物,服德畏威,無廢其貢獻常職也。\CJKunderwave{魯語}稱,\CJKunderline{武王}時,“肅慎氏來貢楛矢、石砮、長尺有咫。先王欲昭令德之致遠,以示後人,使永監焉,故銘其楛曰‘肅慎氏貢矢’,以分大姬,配虞胡公而封諸陳。古者分異姓以遠方之貢,使無忘服也。故分陳以肅慎氏之矢”。是分異姓之事,禮有異姓庶姓,異姓,王之甥舅;庶姓與王無親。其分庶姓亦當以遠方之貢矣。 \par}

{\noindent\zhuan\zihao{6}\fzbyks 傳“以寶”至“之道”。正義曰:寶玉亦是萬國所貢,但不必是遠方所貢耳。“以寶玉分同姓之國”,示己不愛惜,共諸侯有之,是“用誠信其親親之道”也。言用寶以表誠心,使彼知王親愛之也。定四年\CJKunderwave{左傳}稱分魯公以夏后氏之璜,是“以寶玉分同姓”也。異姓疏,慮其廢職,故賜以遠方之物攝彼心。同姓親,嫌王無恩,賜以寶玉貴物錶王心。此亦互相見也。 \par}

{\noindent\shu\zihao{5}\fzkt “曰嗚呼”至“展親”。正義曰:“嗚呼!”嘆而言也。自古明聖之王,慎其德教以柔遠人,四夷皆來賓服。無有遠之與近,盡貢其方土所生之物。其所獻者惟可以供其服食器用而已,不為耳目華侈供玩好之用也。明王既得所貢,乃明其德之所致,分賜於彼異姓之國,明己德致遠,賜異姓之國,令使無廢其服職事也。分寶玉於同姓伯叔之國,見已無所愛惜,是用誠信其親親之道也。 \par}

人不易物,惟德其物。\footnote{言物貴由人,有德則物貴,無德則物賤,所貴在於德。○易,羊質反。}德盛不狎侮。\footnote{盛德必自敬,何狎易侮慢之有?○易,以豉反。}狎侮君子,罔以盡人心。\footnote{以虛受人,則人盡其心矣。○盡,津忍反,下同。}狎侮小人,罔以盡其力。\footnote{以悅使民,民忘其勞,則力盡矣。}


{\noindent\zhuan\zihao{6}\fzbyks 傳“言物”至“於德”。正義曰:有德不濫賞,賞必加於賢人,得者則以為榮,故“有德則物貴”也。無德則濫賞,賞或加於小人,賢者得之反以為恥,故“無德則物賤”也。所貴不在於物,乃在於德。 \par}

{\noindent\zhuan\zihao{6}\fzbyks 傳“以虛”至“心矣。”。正義曰:“以虛受人”,\CJKunderwave{易·咸卦}象辭也。人主以己為虛,受用人言,執謙以下人,則人皆盡其心矣。 \par}

{\noindent\zhuan\zihao{6}\fzbyks 傳“以悅”至“盡矣。”。正義曰:\CJKunderwave{詩序}云:“悅以使民,民忘其死。”故云“以悅使民,民忘其勞”。在上撫悅之,則人皆盡其力矣。此“君子”謂臣,“小人”謂民,\CJKunderwave{太甲}曰“接下思恭”,不可狎侮臣也。\CJKunderwave{論語}雲“使民如承大祭”,不可狎侮民也。襄九年\CJKunderwave{左傳}雲“君子勞心,小人勞力”,故別言之。 \par}

{\noindent\shu\zihao{5}\fzkt “人不”至“其力”。正義曰:既言分物賜人,因說貴不在物。言有德無德之王,俱是以物賜人,所賜之物一也,不改易其物。惟有德者賜人,其此賜者是物。若無德者賜人,則此物不是物矣。恐人主恃已賜人,不自修德,言此者,戒人主使修德也。又說修德之事,德盛者常自敬身,不為輕狎侮慢之事。狎侮君子,則無以盡人心,君子被君侮慢,不肯盡心矣。狎侮小人,則無以盡其力,小人被君侮慢,不復肯盡力矣。君子不盡心,小人不盡力,則國家之事敗矣。 \par}

不役耳目,百度惟貞。\footnote{言不以聲色自役,則百度正。}玩人喪德,玩物喪志。\footnote{以人為戲弄則喪其德,以器物為戲弄則喪其志。○玩,五貫反。喪,息浪反。}志以道寧,言以道接。\footnote{在心為志,發氣為言,皆以道為本,故君子勤道。}不作無益害有益,功乃成。不貴異物賤用物,民乃足。\footnote{遊觀為無益,奇巧為異物,言明王之道以德義為益,器用為貴,所以化治生民。○觀,官喚反。}


{\noindent\zhuan\zihao{6}\fzbyks 傳“言不”至“度正”。正義曰:昭元年\CJKunderwave{左傳}子產論晉侯之疾云:“茲心不爽,昏亂百度。”杜預云:“百度,百事之節也。”此言志既不營聲色,百事皆自用心,則皆得正也。 \par}

{\noindent\zhuan\zihao{6}\fzbyks 傳“以人”至“其志”。正義曰:“喪德”、“喪志”其義一也。“玩人”為重,以“德”言之;“玩物”為輕,以“志”言之;終是志荒而德喪耳。 \par}

{\noindent\zhuan\zihao{6}\fzbyks 傳“在心”至“勤道”。正義曰:“在心為志”,\CJKunderwave{詩序}文也。“在心為志”謂心動有所向也,“發氣為言”言於志所趣也。志是未發,言是已發,相接而成,本末之異耳。志、言並皆用道,但志未發,故“以道寧”,志不依道,則不得寧耳。言是已發,故“以道接”,言不以道,則不可接物。志、言皆以道為本,故君子須勤道也。 \par}

{\noindent\zhuan\zihao{6}\fzbyks 傳“遊觀”至“生民”。正義曰:遊觀徒費時日,故為“無益”。無益多矣,非徒遊觀而已。奇巧世所希有,故為“異物”。異物多矣,非徒奇巧而已。諸是妄作,皆為無益。諸是世所希,皆為異物。異物、無益不可遍舉,舉此二者以明此類皆是也。“不作”是初造之辭,為作有所害,故以為“無益”。“不貴”是愛好之語,有貴必有賤,故以“異物”對“用物”。雖經言“用物”,傳言“器用”可矣。經言“有益”,“有益”不知所謂,故傳以德義是人之本,故德義為有益。諸是益身之物,皆是有益,亦舉重為言。經之戒人主,人主如此,所以化世俗,生養下民也。此言“生民”,宣十二年\CJKunderwave{左傳}雲“分謗生民”,皆謂生活民也。下雲“生民保厥居”,與\CJKunderwave{孝經}雲“生民之本盡矣”,言民生於世,謂之“生民”,與此傳異也。俗本雲“弗賤”,衍“弗”字也。 \par}

犬馬非其土性不畜,\footnote{非此土所生不畜,以不習其用。○畜,許竹反。}珍禽奇獸不育於國\footnote{皆非所用,有損害故。}。不寶遠物,則遠人格。\footnote{不侵奪其利,則來服矣。}所寶惟賢,則邇人安。\footnote{寶賢任能,則近人安。近人安,則遠人安矣。}

{\noindent\zhuan\zihao{6}\fzbyks 傳“非此”至“其用”。正義曰:此篇為戒,止為此句,以西旅之獒,非中國之大,不用令王愛好之,故言此也。僖十五年\CJKunderwave{左傳}言晉侯乘鄭馬,及戰陷於濘,是非此土所生不習其用也。犬不習用,傳記無文。 \par}

{\noindent\zhuan\zihao{6}\fzbyks 傳“寶賢”至“安矣”。正義曰:\CJKunderwave{詩序}雲“任賢使能,周室中興”,故傳以“任能”配“寶賢”言之。\CJKunderwave{論語}雲“舉直錯諸枉,則民服”,故“寶賢任能,則近人安”。嫌安近不及遠,故云“近人安,則遠人安矣”。\CJKunderwave{楚語}云:“王孫圉聘於晉,定公饗之。趙簡子鳴玉以相,問於王孫圉曰:‘楚之白珩猶在乎?’對曰:‘然。’簡子曰:‘其為寶也幾何矣?’曰:‘未嘗為寶。楚之所寶者,曰觀射父,及左史倚相,此楚國之寶也。若夫白珩,先王之所玩,何寶之焉?’”是謂“寶賢”也。 \par}

{\noindent\shu\zihao{5}\fzkt “不役”至“道接”。正義曰:既言不可狎侮,又言不可縱恣。不以聲色使役耳目,則百事之度惟皆正矣。以聲色自娛,必玩弄人物。既玩弄人者,喪其德也;玩弄物者,喪其志也。人物既不可玩,則當以道自處。志當以道而寧身,言當以道而接物,依道而行,則志自得而言自當。 \par}

嗚呼!夙夜罔或不勤,\footnote{言當早起夜寐,常勤於德。}不矜細行,終累大德。\footnote{輕忽小物,積害毀大,故君子慎其微。○行,下孟反。累,劣偽反。}為山九仞,功虧一簣。\footnote{八尺曰仞,喻向成也。未成一簣,猶不為山,故曰功虧一簣。是以聖人乾乾日昃慎終如始。○仞音刃,字又作刃,七尺曰仞。虧,曲為反。簣,其貴反。向,許亮反。幹,其連反。昃音側。}允迪茲,生民保厥居,惟乃世王。”\footnote{言其能信蹈行此誡,則生人安其居,天子乃世世王天下。\CJKunderline{武王}雖聖,猶設此誡,況非聖人,以無誡乎?其不免於過,則亦宜矣。○世王如字,又於況反,注同。}


{\noindent\zhuan\zihao{6}\fzbyks 傳“輕忽”至“其微”。正義曰:“矜”是憐惜之意,故以不惜細行為“輕忽小物”,謂上狎侮君子小人、愛玩犬馬禽獸之類是小事也。積小害,毀大德,故君子慎其微。\CJKunderwave{易·繫辭}曰:“小人以小善為無益而不為也,以小惡為無傷而不去也,故惡積而不可掩,罪大而不可解。”是故君子當慎微也。 \par}

{\noindent\zhuan\zihao{6}\fzbyks 傳“八尺”至“如始”。正義曰:\CJKunderwave{周禮·匠人}有畎、遂、溝、洫皆廣深等,而澮雲“廣二尋,深二仞”,則澮亦廣深等,仞與尋同,故知“八尺曰仞”。王肅\CJKunderwave{聖證論}及注\CJKunderwave{家語}皆雲“八尺曰仞”,與孔義同。\CJKunderline{鄭玄}雲“七尺曰仞”,與孔意異。\CJKunderwave{論語}云:“譬如為山,未成一簣。”鄭云:“簣,盛土器。”“為山九仞”,欲成山,以喻為善向成也。未成一簣,猶不為山,故曰為山“功虧一簣”。古語云:“行百里者半於九十。”言末路之艱難也。是以聖人乾乾不息,至於日昃,不敢自暇,恐末路之失,同於一簣,故“慎終如始”也。“乾乾”,\CJKunderwave{易·乾卦}文。“日昃”,無逸篇文。 \par}

{\noindent\zhuan\zihao{6}\fzbyks 傳“言其”至“宜矣”。正義曰:此總結上文,“信蹈行此誡”,行此以上言也。言君主於治民,故先雲“生民安其居,天子乃得世世王天下”也。傳以庸君多自用己,不受人言,敘經意而申之云,\CJKunderline{武王}雖聖,\CJKunderline{召公}猶設此誡,況非聖人,可以無誡乎?身既非聖,又無善誡,其不免於過則,亦宜其然矣。 \par}

{\noindent\shu\zihao{5}\fzkt “嗚呼”至“世王”。正義曰:聽戒以終,故嘆以結之。嗚呼!為人君所當早起夜寐,無有不勤於德,言當勤行德也。若不矜惜細行,作隨宜小過,終必損累大德矣。譬如為山,已高九仞,其功虧損在於一簣。惟少一簣而止,猶尚不成山,以喻樹德行政,小有不終,德政則不成矣。必當慎終如始,以成德政。王者信能蹈行此誡,生民皆安其居處,惟天子乃世世王天下也。 \par}

\CJKunderline{巢伯}來朝,\footnote{殷之諸侯。伯,爵也。南方遠國。\CJKunderline{武王}克商,慕義來朝。○巢,仕交反,徐呂交反。}\CJKunderline{芮伯}作\CJKunderwave{旅巢命}。\footnote{\CJKunderline{芮伯},周同姓,圻內之國,為卿大夫。陳威德以命巢。亡。○芮,如銳反。圻音祁。}


{\noindent\zhuan\zihao{6}\fzbyks 傳“殷之”至“來朝”。正義曰:\CJKunderline{武王}克商,即來受周之王命,知是“殷之諸侯”。“伯”是爵也。\CJKunderwave{仲虺之誥}雲“\CJKunderline{成湯}放桀於南巢”,或此“巢”是也,故先儒相傳皆以為南方之國。今聞\CJKunderline{武王}克商,慕義而來朝也。\CJKunderline{鄭玄}以為“南方世一見者”。孔以夷狄之爵不過子,此君伯爵,夷夏未明,故直言“遠國”也。 \par}

{\noindent\zhuan\zihao{6}\fzbyks 傳“\CJKunderline{芮伯}”至“巢亡”。正義曰:\CJKunderwave{世本}雲“\CJKunderline{芮伯},姬姓”,是“周同姓”也。杜預云:“芮,馮翊臨晉縣芮鄉是也。”知是“圻內之國”者,\CJKunderline{芮伯}在朝作命,必是王臣。不得其官,故“卿”與“大夫”並言之。“旅”訓為陳,陳王威德以命巢。 \par}

{\noindent\shu\zihao{5}\fzkt “\CJKunderline{巢伯}”至“巢命”。正義曰:“\CJKunderline{巢伯}”,國爵之君,南方遠國也。以\CJKunderline{武王}克商,乃慕義來朝。王之卿大夫有\CJKunderline{芮伯}者,陳王威德以命巢君。史敘其事,作\CJKunderwave{旅巢命}之篇。 \par}

\section{金縢第八}


\CJKunderline{武王}有疾,\CJKunderline{周公}作\CJKunderwave{金縢}。\footnote{為請命之書,藏之於匱,緘之以金,不欲人開之。○\CJKunderline{武王}有疾,馬本作“有疾不豫”。縢,徒登反。緘,工咸反。}

金縢\footnote{遂以所藏為篇名。}

{\noindent\zhuan\zihao{6}\fzbyks 傳“為請”至“開之”。正義曰:經雲“金縢之匱”,則“金縢”是匱之名也。\CJKunderwave{詩}述韔弓之事云:“竹閉緄縢。”\CJKunderwave{毛傳}云:“緄,繩。縢,約也。”此傳言“緘之以金”,則訓“縢”為緘。王、鄭皆云:“縢,束也。”又鄭\CJKunderwave{喪大記}注云:“齊人謂棺束為緘。”\CJKunderwave{家語}稱周廟之內有金人,叄緘其口,則“縢”是束縛之義。“藏之於匱,緘之以金”,若今釘鐷之,不欲人開也。鄭云:“凡藏秘書,藏之於匱,必以金緘其表。”是秘密之書,皆藏於匱,非\CJKunderline{周公}始造此匱,獨藏此書也。“金縢”。正義曰:發首至“\CJKunderline{王季}、\CJKunderline{文王}”,史敘將告神之事也。“史乃策祝”至“屏璧與珪”,告神之辭也。自“乃卜”至“乃瘳”,言卜吉告王差之事也。自“\CJKunderline{武王}既喪”已下,敘\CJKunderline{周公}被流言,東征還反之事也。此篇敘事多而言語少,若使\CJKunderline{周公}不遭流言,則請命之事遂無人知。為\CJKunderline{成王}開書,\CJKunderline{周公}得反,史官美大其事,故敘之以為此篇。 \par}

{\noindent\shu\zihao{5}\fzkt “\CJKunderline{武王}”至“金縢”。正義曰:\CJKunderline{武王}有疾,\CJKunderline{周公}作策書告神,請代\CJKunderline{武王}死。事畢,納書於金縢之匱,遂作\CJKunderwave{金縢}。凡序言“作”者,謂作此篇也。案經\CJKunderline{周公}策命之書,自納金縢之匱,及為流言所謗,\CJKunderline{成王}悟而開之。史敘其事,乃作此篇,非\CJKunderline{周公}作也。序以經具,故略言之。 \par}

既克商二年,王有疾,弗豫。\footnote{伐紂明年,\CJKunderline{武王}有疾,不悅豫。○豫本,又作忬。}二公曰:“我其為王穆卜。”\CJKunderline{周公}曰:“未可以戚我先王。”\footnote{穆,敬。戚,近也。\CJKunderline{召公}、太公言王疾當敬卜吉兇,\CJKunderline{周公}言未可以死近我先王。相順之辭。○為,於偽反。戚,千歷反。}公乃自以為功,\footnote{\CJKunderline{周公}乃自以請命為己事。}為三壇同\xpinyin*{墠}。\footnote{因太王、\CJKunderline{王季}、\CJKunderline{文王}請命於天,故為三壇。壇築土,墠除地,大除地,於中為三壇。○壇,徒丹反,築土也,馬云:“土堂。”墠音善。}

{\noindent\zhuan\zihao{6}\fzbyks 傳“伐紂”至“悅豫”。正義曰:\CJKunderline{武王}以\CJKunderline{文王}受命十三年伐紂,既殺紂,即當稱元年。克紂稱元年,知此“二年”是“伐紂之明年”也。王肅亦云:“克殷明年。”\CJKunderwave{顧命}云:“王有疾,不懌。”“懌,悅也,故不豫為“不悅豫”也。何休因此為例云:“天子曰不豫,諸侯曰負茲,大夫曰犬馬,士曰負薪。” \par}

{\noindent\zhuan\zihao{6}\fzbyks 傳“穆敬”至“之辭”。正義曰:\CJKunderwave{釋訓}云:“穆穆,敬也。”“戚”是親近之義,故為近也。\CJKunderline{武王}時三公惟周、召與太公耳,知“二公”是\CJKunderline{召公}、太公也。言王疾恐死,當敬卜吉兇。\CJKunderline{周公}言\CJKunderline{武王}既定天下,當成就周道,未可以死近我先王。死則神與先王相近,故言近先王。若生則人神道隔,是為遠也。二公恐王死,欲為之卜。\CJKunderline{周公}言王未可以死,是“相順之辭”也。鄭云:“戚,憂也。\CJKunderline{周公}既內知\CJKunderline{武王}有九齡之命,又有\CJKunderline{文王}曰‘吾與爾三之期’,今必瘳,不以此終,故止二公之卜。雲未可以憂怖我先王。”如鄭此言,\CJKunderline{周公}知王不死,先王豈不知乎,而慮先王憂也? \par}

{\noindent\zhuan\zihao{6}\fzbyks 傳“\CJKunderline{周公}”至“己事”。正義曰:“功”訓事也。\CJKunderline{周公}雖許二公之卜,仍恐王疾不瘳,不復與二公謀之,乃自以請命為己之事,獨請代\CJKunderline{武王}死也。所以\CJKunderline{周公}自請為己事者,\CJKunderline{周公}位居冢宰,地則近親,脫或卜之不善,不可使外人知悉,亦不可苟讓,故自以為功也。 \par}

{\noindent\zhuan\zihao{6}\fzbyks 傳“因大”至“三壇”。正義曰:“請命”請之於天,而告三王者,以三王精神已在天矣,故“因\CJKunderline{大王}、\CJKunderline{王季}、\CJKunderline{文王}以請命於天”。三王每王一罈,故“為三壇”。壇是築土,墠是除地,大除其地,於中為三壇。\CJKunderline{周公}為壇於南方,亦當在此墠內,但其處小別,故下別言之。\CJKunderline{周公}北面,則三壇南面可知,但不知以何方為上耳。\CJKunderline{鄭玄}云:“時為壇墠於豐,壇墠之處猶存焉。” \par}

為壇於南方,北面,\CJKunderline{周公}立焉。\footnote{立壇上,對三王。}植璧秉珪,乃告\CJKunderline{大王}、\CJKunderline{王季}、\CJKunderline{文王}。\footnote{璧以禮神。植,置也,置於三王之坐。\CJKunderline{周公}秉桓珪以為贄。告謂祝辭。○植,時織反,徐音置。贄音至。祝如字,或之疚反,下同。}

{\noindent\zhuan\zihao{6}\fzbyks 傳“立壇”至“三王”。正義曰:\CJKunderwave{禮}“授坐不立”,“授立不坐”,欲其高下均也。神位在壇,故\CJKunderline{周公}“立壇上,對三王”也。 \par}

{\noindent\zhuan\zihao{6}\fzbyks 傳“璧以”至“祝辭”。正義曰:\CJKunderwave{周禮·大宗伯}雲“以蒼璧禮天”,\CJKunderwave{詩}說禱旱雲“圭璧既卒”,是璧以禮神,不知其何色也。鄭云:“植,古置字。”故為置也,言置璧於三王之坐也。\CJKunderwave{周禮}云:“公執桓圭。”知\CJKunderline{周公}秉桓圭,又置以為贄也。“告謂祝辭”,下文是其辭也。 \par}

{\noindent\shu\zihao{5}\fzkt “既克”至“\CJKunderline{文王}”。正義曰:“既克商二年”即伐紂之明年也。王有疾病,不悅豫。\CJKunderline{召公}與太公二公同辭而言曰:“我其為王敬卜吉兇,問王疾病瘳否。”\CJKunderline{周公}曰:“王今有疾,未可以死近我先王,故當須卜也。”\CJKunderline{周公}既為此言,公乃自以請命之事為己事,除地為墠,墠內築壇,為三壇同墠。又為一罈於南方,北面,\CJKunderline{周公}立壇上焉。置璧於三王之坐,公自執珪,乃告\CJKunderline{大王}、\CJKunderline{王季}、\CJKunderline{文王},告此三王之神也。 \par}

史乃冊,祝曰:“惟爾元孫某,遘厲虐疾。\footnote{史為冊書,祝辭也。元孫,\CJKunderline{武王}。某,名。臣諱君,故曰某。厲,危。虐,暴也。○遘,工豆反,遇也。}若爾三王,是有丕子之責於天,以\CJKunderline{旦}代某之身。\footnote{大子之責,謂疾不可救於天,則當以旦代之。死生有命,不可請代,聖人敘臣子之心,以垂世教。○丕,普悲反,馬同,徐甫眉反,鄭音不。}予仁若考,能多材多藝,能事鬼神。\footnote{我\CJKunderline{周公}仁能順父,又多材多藝,能事鬼神。言可以代\CJKunderline{武王}之意。}乃元孫不若旦多材多藝,不能事鬼神。


{\noindent\zhuan\zihao{6}\fzbyks 傳“史為”至“虐暴”。正義曰:告神之言,書之於策,“祝”是讀書告神之名,故云“史為策書,祝辭”,史讀此策書以祝告神也。\CJKunderline{武王}是\CJKunderline{大王}之曾孫也,尊統於上,繼之於祖,謂“元孫”,是長孫也。“某”者,\CJKunderline{武王}之名,本告神雲“元孫發”,臣諱君,故曰“某”也。\CJKunderwave{易·乾卦}云:“夕惕若厲。”“厲”為危也。“虐”訓為暴。言性命危而疾暴重也。\CJKunderwave{泰誓}、\CJKunderwave{牧誓}皆不諱發而此獨諱之,孔惟言“臣諱君”,不解諱之意。\CJKunderline{鄭玄}云:“諱之者,由\CJKunderline{成王}讀之也。”意雖不明,當謂\CJKunderline{成王}開匱得書,王自讀之,至此字口改為“某”,史官錄為此篇,因遂\CJKunderline{成王}所讀,故諱之。上篇\CJKunderwave{泰誓}、\CJKunderwave{牧誓}王自稱者,令入史製為此典,故不須諱之。 \par}

{\noindent\zhuan\zihao{6}\fzbyks 傳“太子”至“世教”。正義曰:“責”讀如\CJKunderwave{左傳}“施捨已責”之“責”,“責”謂負人物也。“太子之責於天”,言負天一太子,謂必須死,疾不可救於天。必須一子死,則當以旦代之。死生有命,不可請代,今請代者,“聖人敘臣子之心,以垂世教”耳,非謂可代得也。\CJKunderline{鄭玄}弟子趙商問玄曰:“若\CJKunderline{武王}未終,疾固當瘳。信命之終,雖請不得。自古已來,何患不為?”玄答曰:“君父疾病方困,忠臣孝子不忍默爾,視其歔欷,歸其命於天,中心惻然,欲為之請命。\CJKunderline{周公}達於此禮,著在\CJKunderwave{尚書},若君父之病不為請命,豈忠孝之志也?”然則命有定分,非可代死,\CJKunderline{周公}為此者,自申臣子之心,非謂死實可代。自古不廢,亦有其人,但不見爾,未必\CJKunderline{周公}獨為之。\CJKunderline{鄭玄}云:“丕讀曰不。及子孫曰子。元孫遇疾,若汝不救,是將有不愛子孫之過,為天所責,欲使為之請命也。”與孔讀異。 \par}

{\noindent\zhuan\zihao{6}\fzbyks 傳“我周”至“之意”。正義曰:告神稱“予”,知\CJKunderline{周公}自稱“我”也。“考”是父也,故“仁能順父”。上雲“元孫”,對祖生稱,此言“順父”,從親為始。祖為王考,曾祖為皇考,“考”、“父”,可以通之,傳舉親而言“父”耳。既能順父,又多材多藝,能事鬼神,言己可以代\CJKunderline{武王}之意。上言“丕子之責於天”,則是天欲取\CJKunderline{武王},非父祖取之,此言己能順父祖,善事鬼神者,假令天意取之,其神必共父祖同處,言己是父祖所欲,欲令請之於天也。 \par}

乃命於帝庭,敷佑四方。\footnote{汝元孫受命於天庭為天子,布其德教,以佑助四方。言不可以死。}用能定爾子孫于下地,四方之民,罔不祗畏。\footnote{言\CJKunderline{武王}用受命帝庭之故,能定先人子孫於天下,四方之民無不敬畏。}嗚呼!無墜天之降寶命,我先王亦永有依歸。\footnote{嘆惜\CJKunderline{武王},言不救則墜天之寶命,救之則先王長有依歸。}今我即命於元龜,\footnote{就受三王之命於大龜,卜知吉凶。}爾之許我,我其以璧與珪歸俟爾命。\footnote{許謂疾瘳。待命,當以事神。○瘳,敕留反,下同。}爾不許我,我乃屏璧與珪。\footnote{不許謂不愈也。屏,藏也,言不得事神。}

{\noindent\zhuan\zihao{6}\fzbyks 傳“汝元”至“以死”。正義曰:以王者存亡,大運在天,有德於民,天之所與,是“受命天庭”也。以人況天,故言在庭,非王實至天庭受天命也。既受天命以為天子,布其德教以佑助四方之民,當於天心有功,於民言不可以死也。 \par}

{\noindent\shu\zihao{5}\fzkt “史乃”至“與珪”。正義曰:史乃為策書,執以祝之曰,惟爾元孫某,“某”即發也,遇得危暴重疾,今恐其死。若爾三王,是有太子之責於天,謂負天大子責,必須一子死者,請以旦代發之身,令旦死而發生。又告神以代之狀,我仁能順父,又且多材力,多技藝,又能善事鬼神,汝元孫不如旦多材多藝,又不能事鬼神,言取發不如取旦也。然人各有能,發雖不能事鬼,神則有人君之用,乃受命於天帝之庭,能布其德教以佑助四方之民,用能安定汝三王子孫在於下地,四方之民無不敬而畏之。以此之故,不可使死。嗚呼!發之可惜如此,神明當救助之,無得隕墜天之所下寶命。天下寶命謂使為天子,若\CJKunderline{武王}死,是隕墜之也。若不墜命,則我先王亦永有依歸,為宗廟之主,神得歸之。我與三王人神道隔,許我以否不可知,今我就受三王之命於彼大龜,卜其吉凶。吉則許我,兇則為不許我。爾之許我,使卜得吉兆,旦死而發生,我其以璧與珪歸家待汝神命,我死當以珪璧事神。爾不許我,使卜兆不吉,發死而旦生,我乃屏去璧之與珪。言不得事神,當藏珪璧也。 \par}

乃卜三龜,一習吉。\footnote{習,因也。以三王之龜卜,一相因而吉。}啟籥見書,乃並是吉。\footnote{三兆既同吉,開籥見佔兆書,乃亦並是吉。○籥,於若反,徐以略反,馬云:“藏卜兆書管。”並,必政反。}公曰:“體,王其罔害。\footnote{公視兆曰:“如此兆體,王其無害。”言必愈。}予小子新命於三王,惟永終是圖。\footnote{\CJKunderline{周公}言,我小子新受三王之命,\CJKunderline{武王}惟長終是謀周之道。}


{\noindent\zhuan\zihao{6}\fzbyks 傳“習因”至“而吉”。正義曰:“習”則襲也,襲是重衣之名,因前而重之,故以“習”為因也。雖三龜並卜,卜有先後,後者因前,故云“因”也。\CJKunderwave{周禮}:“太卜掌三兆之法,一曰\CJKunderwave{玉兆},二曰\CJKunderwave{瓦兆},三曰\CJKunderwave{原兆}”。三兆各別,必三代法也。\CJKunderwave{洪範}卜筮之法,三人佔則從二人之言,是必三代之法並用之矣。故知“三龜”,“三王之龜”。龜形無異代之別,但卜法既別,各用一龜,謂之“三王之龜”耳。每龜一人佔之,其後君與大夫等,部佔三代之龜,定其吉凶。未見佔書知已吉者,卜有大體,見兆之吉凶,粗觀可識,故知吉也。 \par}

{\noindent\zhuan\zihao{6}\fzbyks 傳“三兆”至“是吉”。正義曰:\CJKunderline{鄭玄}云:“籥,開藏之管也。開兆書藏之室以管,乃復見三龜佔書,亦合於是吉。”王肅亦云:“籥,開藏佔兆書管也。”然則佔兆別在於藏。\CJKunderwave{大卜}“三兆”之下云:“其經兆之體,皆百有二十,其頌皆千有二百。”佔兆之書,則彼“頌”是也。略觀三兆,既已同吉,開藏以籥,見彼佔兆之書,乃亦並是吉。言其兆頌符,同為大吉也。 \par}

{\noindent\zhuan\zihao{6}\fzbyks 傳“公視”至“必愈”。正義曰:“如此兆體”,指卜之所得兆也。\CJKunderwave{周禮·占人}云:“凡卜筮,君佔體,大夫佔色,史佔墨,卜人佔坼。”\CJKunderline{鄭玄}云:“體,兆象也。色,兆氣也。墨,兆廣也。坼,兆璺也。尊者視兆象而已,卑者以次詳其餘也。\CJKunderline{周公}卜\CJKunderline{武王},佔之曰:‘體,王其無害。’”鄭意此言“體”者,即彼“君佔體”也。但\CJKunderline{周公}令卜,汲汲欲王之愈,必當親視灼龜,躬省兆繇,不惟佔體而日。但鄭以“君佔體”與此文同,故引以為證耳。 \par}

茲攸俟,能念予一人。”\footnote{言\CJKunderline{武王}愈,此所以待能念我天子事,成周道。○公歸,乃納冊於金縢之匱中。}王翼日乃瘳。\footnote{從壇歸。翼,明。瘳,差也。○差,初賣反。}

{\noindent\zhuan\zihao{6}\fzbyks 傳“言武”至“周道”。正義曰:此原三王之意也。言\CJKunderline{武王}得愈者,此謂卜吉\CJKunderline{武王}之愈。言天與三王一一須待\CJKunderline{武王},能念我天子事,成周道。若死,則不復得念天子之事,周道必不成也。\CJKunderwave{禮}天子自稱曰“予一人”,故以“一人”言天子也。 \par}

{\noindent\zhuan\zihao{6}\fzbyks 傳“從壇”至“瘳差”。正義曰:壇所即卜,故“從壇歸”也。“翼,明”,\CJKunderwave{釋言}文。“瘳”訓差,亦為愈,病除之名也。藏此書者,此既告神,即是國家舊事,其書不可捐棄,又不可示諸世人,故藏於金縢之匱耳。 \par}

{\noindent\shu\zihao{5}\fzkt “乃卜”至“乃瘳”。正義曰:祝告已畢,即於壇所乃卜其吉凶。用三王之龜卜,一皆相因而吉。觀兆已知其吉,猶尚未見佔書。佔書在於藏內,啟藏以籥,見其佔書,亦與兆體乃並是吉。公視兆曰,觀此兆體,王身其無患害也。我小子新受命於三王,謂卜得吉也。我\CJKunderline{武王}當惟長終是謀周之道。此卜吉之愈者,上天所以須待\CJKunderline{武王}能念我一人天子之事,成其周道故也。公自壇歸,乃納策於金縢之匱中。王明日乃病瘳。 \par}

\CJKunderline{武王}既喪,\CJKunderline{管叔}及其群弟乃流言於國,\footnote{\CJKunderline{武王}死,\CJKunderline{周公}攝政,其弟\CJKunderline{管叔}及\CJKunderline{蔡叔}、\CJKunderline{霍叔}乃放言於國,以誣\CJKunderline{周公},以惑\CJKunderline{成王}。○喪,蘇浪反。}曰:“公將不利於孺子。”\footnote{三叔以\CJKunderline{周公}大聖,有次立之勢,遂生流言。孺,稚也。稚子,\CJKunderline{成王}。○孺,如樹反。}\CJKunderline{周公}乃告二公曰:“我之弗闢,我無以告我先王。”\footnote{闢,法也。告\CJKunderline{召公}、太公,言我不以法法三叔,則我無以成周道告我先王。○闢,扶亦反,治也;\CJKunderwave{說文}作壁,雲必亦反,法也;馬、鄭音避,謂避居東都。}\CJKunderline{周公}居東二年,則罪人斯得。\footnote{\CJKunderline{周公}既告二公,遂東征之,二年之中,罪人此得。}


{\noindent\zhuan\zihao{6}\fzbyks 傳“\CJKunderline{武王}死”至“\CJKunderline{成王}”。正義曰:\CJKunderline{武王}既死,\CJKunderline{成王}幼弱,故\CJKunderline{周公}攝政。攝政者,雖以\CJKunderline{成王}為主,政令自公出,不復關\CJKunderline{成王}也。\CJKunderwave{蔡仲之命}云:“群叔流言,乃致闢\CJKunderline{管叔}於商,囚\CJKunderline{蔡叔}於郭鄰,降\CJKunderline{霍叔}於庶人。”則知“群弟”是\CJKunderline{蔡叔}、\CJKunderline{霍叔}也。\CJKunderwave{周語}雲“獸三為群”,則滿三乃稱群。蔡霍二人而言群者,並管故稱群也。傳既言\CJKunderline{周公}攝政,乃雲“其弟\CJKunderline{管叔}”,蓋以\CJKunderline{管叔}為\CJKunderline{周公}之弟。\CJKunderwave{孟子}曰:“\CJKunderline{周公},弟也。\CJKunderline{管叔},兄也。”\CJKunderwave{史記}亦以\CJKunderline{管叔}為\CJKunderline{周公}之兄。孔似不用\CJKunderwave{孟子}之說,或可孔以“其弟”謂\CJKunderline{武王}之弟,與\CJKunderwave{史記}亦不違也。“流言”者,宣佈其言,使人聞知,若水流然。“流”即放也,乃放言於國,以誣\CJKunderline{周公},以惑\CJKunderline{成王}。“王亦未敢誚公”,是王心惑也。\CJKunderline{鄭玄}云:“流公將不利於孺子之言於京師,於時管蔡在東,蓋遣人流傳此言於民間也。” \par}

{\noindent\zhuan\zihao{6}\fzbyks 傳“三叔”至“\CJKunderline{成王}”。正義曰:殷法多兄亡弟立,三叔以\CJKunderline{周公}大聖,又是\CJKunderline{武王}之弟,有次立之勢,今復秉國之權,恐其因即篡奪,遂生流言。不識大聖之度,謂其實有異心,非是故誣之也。但啟商共叛,為罪重耳。 \par}

{\noindent\zhuan\zihao{6}\fzbyks 傳“闢,法也”。正義曰:\CJKunderwave{釋詁}文。 \par}

{\noindent\zhuan\zihao{6}\fzbyks 傳“\CJKunderline{周公}”至“此得”。正義曰:\CJKunderwave{詩·東山}之篇歌此事也,序雲“東征”,知“居東”者,遂東往徵也。雖徵而不戰,故言“居東”也。\CJKunderwave{東山}詩曰:“自我不見,於今三年。”又云“三年而歸”,此言“二年”者,\CJKunderwave{詩}言初去及來,凡經三年;此直數居東之年,除其去年,故二年也。罪人既多,必前後得之,故云“二年之中,罪人此得”。惟言“居東”,不知居在何處。王肅云:“東,洛邑也。管蔡與商奄共叛,故東征鎮撫之。案驗其事,二年之間,罪人皆得。” \par}

於後,公乃為詩以貽王,名之曰\CJKunderwave{鴟鴞}。王亦未敢\xpinyin*{誚}公。\footnote{\CJKunderline{成王}信流言而疑\CJKunderline{周公},故\CJKunderline{周公}既誅三監,而作詩解所以宜誅之意以遺王,王猶未悟,故欲讓公而未敢。○貽,羊支反。名如字,徐亡政反。鴟,尺夷反。鴞,於嬌反。誚,在笑反。以遺,唯季反。}

{\noindent\zhuan\zihao{6}\fzbyks 傳“\CJKunderline{成王}”至“未敢”。正義曰:\CJKunderline{成王}信流言而疑\CJKunderline{周公},管蔡既誅,王疑益甚,故\CJKunderline{周公}既誅三監,而作詩解所以宜誅之意。其\CJKunderwave{詩}云:“鴟鴞鴟鴞,既取我子,無毀我室。”\CJKunderwave{毛傳}云:“無能毀我室者,攻堅之故也。寧亡二字,不可以毀我周室。”言宜誅之意也。\CJKunderwave{釋言}云:“貽,道也。”以詩遺王,王猶未悟,故欲讓公而未敢。政在\CJKunderline{周公},故畏威未敢也。\CJKunderline{鄭玄}以為\CJKunderline{武王}崩,\CJKunderline{周公}為冢宰,三年服終,將欲攝政,管蔡流言,即避居東都。\CJKunderline{成王}多殺公之屬黨,公作\CJKunderwave{鴟鴞}之詩,救其屬臣,請勿奪其官位土地。及遭風雷之異,啟金縢之書,迎公來反,反乃居攝,後方始東征管蔡。解此一篇乃\CJKunderwave{鴟鴞}之詩,皆與孔異。 \par}

{\noindent\shu\zihao{5}\fzkt “\CJKunderline{武王}”至“誚公”。正義曰:\CJKunderline{周公}於\CJKunderline{成王}之世,為管蔡所誣,王開金縢之書,方始明公本意,卒得成就周道,天下太平。史官美大其事,述為此篇,故追言“請命”於前,乃說“流言”於後,自此以下,說\CJKunderline{周公}身事。\CJKunderline{武王}既喪,\CJKunderline{成王}幼弱,\CJKunderline{周公}攝王之政,專決萬機。\CJKunderline{管叔}及其群弟\CJKunderline{蔡叔}、\CJKunderline{霍叔}乃流放其言於國中曰:“公將不利於孺子。”言欲篡王位為不利。\CJKunderline{周公}乃告二公曰:“我之不以法法此三叔,則我無以成就周道,告我先王。”既言此,遂東征之。\CJKunderline{周公}居東二年,則罪人於此皆得,謂獲三叔及諸叛逆者。罪人既得訖,\CJKunderline{成王}猶尚疑公。公於此既得罪人之後,為詩遺王,名之曰\CJKunderwave{鴟鴞}。\CJKunderwave{鴟鴞}言三叔不可不誅之意。王心雖疑,亦未敢責誚公。言王意欲責而未敢也。 \par}

秋,大熟,未獲,天大雷電以風,\footnote{二年秋也。蒙,恆風若,雷以威之,故有風雷之異。○獲,戶郭反。}禾盡偃,大木斯拔,邦人大恐。\footnote{風災所及,邦人皆大恐。○拔,皮八反。}王與大夫盡弁,以啟金縢之書,\footnote{皮弁質服以應天。○弁,皮彥反。徐,扶變反。應,應對之應。}乃得\CJKunderline{周公}所自以為功代\CJKunderline{武王}之說。\footnote{所藏請命冊書本。○說如字,徐始銳反。}


{\noindent\zhuan\zihao{6}\fzbyks 傳“二年”至“之異”。正義曰:上文“居東二年”,未有別年之事,知即是“二年秋”也。嫌別年,故辨之。\CJKunderwave{洪範}“咎徵”云:“蒙,恆風若。”以\CJKunderline{成王}蒙暗,故常風順之。風是暗徵而有雷者,以威怒之故,以示天之威怒有雷風之異。 \par}

{\noindent\zhuan\zihao{6}\fzbyks 傳“風災”至“大恐”。正義曰:言“邦人”,則風災惟在周邦,不及寬遠,故云“風災所及,邦人皆大恐”,言獨畿內恐也。 \par}

{\noindent\zhuan\zihao{6}\fzbyks 傳“皮弁質服以應天”。正義曰:皮弁象古,故為“質服”。祭天尚質,故服以應天也。\CJKunderwave{周禮·司服}云:“王祀昊天上帝,則服大裘而冕。”無旒,乃是冕之質者,是事天宜質服,故服之以應天變也。\CJKunderwave{周禮}:“視朝,則皮弁服。”皮弁是視朝服,每日常服而言“質”者,皮弁白布衣,素積裳,故為質也。\CJKunderline{鄭玄}以為爵弁,“必爵弁者,承天變降服,亦如國家未道焉”。 \par}

二公及王乃問諸史與百執事,\footnote{二公倡王啟之,故先見書。史、百執事皆從\CJKunderline{周公}請命。○倡,昌亮反。從,才用反,又如字。}對曰:“信。噫!公命我勿敢言。”\footnote{史、百執事言信有此事,\CJKunderline{周公}使我勿道,今言之則負\CJKunderline{周公}。噫,恨辭。○噫,於其反,馬本作懿,猶億也。}王執書以泣,曰:“其勿穆卜。\footnote{本欲敬卜吉兇,今天意可知,故止之。}昔公勤勞王家,惟予沖人弗及知。\footnote{言己童幼,不及知\CJKunderline{周公}昔日忠勤。○衝,直忠反。}今天動威,以彰\CJKunderline{周公}之德,\footnote{發雷風之威以明\CJKunderline{周公}之聖德。}惟朕小子其新逆,我國家禮亦宜之。”\footnote{\CJKunderline{周公}以\CJKunderline{成王}未寤,故留東未還,改過自新,遣使者迎之,亦國家禮有德之宜。○新逆,馬本作“親迎”。遣使,所吏反。}

{\noindent\zhuan\zihao{6}\fzbyks 傳“二公”至“請命”。正義曰:二公與王若同而問,當言“王及二公”,今言“二公及王”,則是二公先問,知“二公倡王啟之,故先見書”。鄭云:“開金縢之書者,省察變異所由故事也。”以金縢匱內有先王故事,疑其遭遇災變,必有消伏之術,故倡王啟之。史為公造策書,而百執事給使令,皆從\CJKunderline{周公}請命者。 \par}

{\noindent\zhuan\zihao{6}\fzbyks 傳“史百”至“恨辭”。正義曰:\CJKunderline{周公}使我勿道此事者,公以臣子之情,忠心欲代王死,非是規求名譽,不用使人知之。且\CJKunderline{武王}瘳而\CJKunderline{周公}不死,恐人以公為詐,故令知者勿言。今被問而言之,是違負\CJKunderline{周公}也。“噫”者,心不平之聲,故為“恨辭”。 \par}

{\noindent\zhuan\zihao{6}\fzbyks 傳“\CJKunderline{周公}”至“之宜”。正義曰:公之東征,止為伐罪,罪人既得,公即當還。以\CJKunderline{成王}未寤,恐與公不和,故留東未還,待王之察己也。新迎者,改過自新,遣使者迎之。\CJKunderwave{詩·九罭}之篇是迎之事也。“亦國家禮有德之宜”,言尊崇有德,宜用厚禮。\CJKunderwave{詩}稱“袞衣”、“籩豆”,是國家禮也。 \par}

王出郊,天乃雨,反風,禾則盡起。\footnote{郊以玉幣謝天,天即反風起禾,明郊之是。}二公命邦人,凡大木所偃,盡起而築之。歲則大熟。\footnote{木有偃拔,起而立之,築有其根。桑果無虧,百穀豐熟,\CJKunderline{周公}之德。此已上\CJKunderwave{大誥}後,因\CJKunderline{武王}喪並見之。○築音竹,本亦作築,謂築其根,馬云:“築,拾也。”見,賢遍反。}

{\noindent\zhuan\zihao{6}\fzbyks 傳“郊以”至“之是”。正義曰:祭天於南郊,故謂之“郊”,郊是祭天之處也。“王出郊”者,出城至郊,為壇告天也。\CJKunderwave{周禮·大宗伯}云:“以蒼璧禮天,牲幣如其器之色。”是祭天有玉有幣,今言郊者,以玉幣祭天,告天以謝過也。王謝天,天即反風起禾,明王郊之是也。\CJKunderline{鄭玄}引\CJKunderwave{易傳}云:“陽感天不旋日。陽謂天子也,天子行善以感天,不迴旋經口。”故郊之是得反風也。 \par}

{\noindent\zhuan\zihao{6}\fzbyks 傳“木有”至“見之”。正義曰:上文禾偃木拔,拔必亦偃,故云“木有偃拔,起而立之,築有其根,桑果無虧,百穀豐熟”。鄭、王皆雲“築,拾也。禾為大木所偃者,起其木,拾下禾,無所亡失”。意太曲碎,當非經旨。案序將東征,作\CJKunderwave{大誥}。此上“居東二年”以來,皆是\CJKunderwave{大誥}後事,而編於\CJKunderwave{大誥}之前者,因\CJKunderline{武王}喪並見之。 \par}

{\noindent\shu\zihao{5}\fzkt “秋大”至“大熟”。正義曰:為詩遺王之後,其秋大熟,未及收穫,天大雷電,又隨之以風,禾盡偃仆,大木於此而拔。風災所及,邦人大恐。王見此變,與大夫盡皮弁以開金縢之書,案省故事,求變異所由,乃得\CJKunderline{周公}所自以為功請代\CJKunderline{武王}之說。二公及王問於本從公之人史與百執事,問審然以否。對曰:“信。”言有此事也。乃為不平之聲:“噫!公命我勿敢言。”王執書以泣,曰:“其勿敬卜吉兇。”言天之意已可知也。“昔公勤勞王家,惟我幼童之人不及見知,今天動雷電之威,以彰明\CJKunderline{周公}之德,惟朕小子其改過自新,遣人往迎之。我國家褒崇有德之禮,亦宜行之”。王於是出郊而祭以謝天,天乃雨,反風,禾則盡起。二公命邦人,凡大木所偃仆者,盡扶起而築之。禾木無虧,歲則大熟。言\CJKunderline{周公}之所感致若此也。 \par}

\section{大誥第九}


\CJKunderline{武王}崩,三監及淮夷叛,\footnote{三監,管、蔡、商。淮夷徐奄之屬皆叛周。○監,古懺反,視也。}\CJKunderline{周公}相\CJKunderline{成王},將黜殷,作\CJKunderwave{大誥}。\footnote{相謂攝政。黜,絕也。將以誅叛者之義大誥天下。○相,息亮反,注同。}

{\noindent\zhuan\zihao{6}\fzbyks 傳“三監”至“叛周”。正義曰:知“三監”是管、蔡、商者,以序上下相顧為文。此言“三監及淮夷叛”,總舉諸叛之人也。下雲“\CJKunderline{成王}既黜殷命,殺\CJKunderline{武庚},命\CJKunderline{微子}啟代殷後”,又言“\CJKunderline{成王}既伐\CJKunderline{管叔}、\CJKunderline{蔡叔},以殷餘民封\CJKunderline{康叔}”。此序言三監叛,將徵之,下篇之序歷言伐得三人,足知下文\CJKunderline{管叔}、\CJKunderline{蔡叔}、\CJKunderline{武庚},即此“三監”之謂,知“三監”是“管、蔡、商”也。\CJKunderwave{漢書·地理志}云:“周既滅殷,分其畿內為三國,\CJKunderwave{詩·風}邶、鄘、衛是也。邶以封紂子\CJKunderline{武庚},鄘\CJKunderline{管叔}尹之,衛\CJKunderline{蔡叔}尹之,以監殷民,謂之三監。”先儒多同此說,惟\CJKunderline{鄭玄}以三監為管、蔡、霍,獨為異耳。謂之“監”者,當以殷之畿內,被紂化日久,未可以建諸侯,且使三人監此殷民,未是封建之也。三人雖有其分,互相監領,不必獨主一方也。\CJKunderwave{史記·衛世家}云:“\CJKunderline{武王}克殷,封紂子\CJKunderline{武庚}為諸侯,奉其先祀。為\CJKunderline{武庚}未集,恐有賊心,乃令其弟\CJKunderline{管叔}、\CJKunderline{蔡叔}傳相之。”是言輔相\CJKunderline{武庚},共監殷人,故稱“監”也。序惟言“淮夷叛”,傳言“淮夷徐奄之屬共叛周”者,以下序文雲“\CJKunderline{成王}東伐淮夷,遂踐奄,作\CJKunderwave{\CJKunderline{成王}政}”,又云“\CJKunderline{成王}既黜殷命,滅淮夷,作\CJKunderwave{周官}”,又云“魯侯\CJKunderline{伯禽}宅曲阜,徐夷並興,作\CJKunderwave{費誓}”,彼三序者,一時之事,皆在\CJKunderline{周公}歸政之後也。\CJKunderwave{多方}篇數此諸國之罪雲“至於再,至於三”,得不以\CJKunderline{武王}初崩已叛,\CJKunderline{成王}即政又叛,謂此為再三也。以此知“淮夷叛”者,徐奄之屬皆叛也。 \par}

{\noindent\zhuan\zihao{6}\fzbyks 傳“相謂”至“天下”。正義曰:\CJKunderwave{君奭}序云:“\CJKunderline{召公}為保,\CJKunderline{周公}為師,相\CJKunderline{成王}為左右。”於時\CJKunderline{成王}為天子,自知政事,二公為臣輔助之,此言“相\CJKunderline{成王}”者,有異於彼,故辨之“相謂攝政”。攝政者,教由公出,不復關白\CJKunderline{成王}耳,仍以\CJKunderline{成王}為王,故稱“\CJKunderline{成王}”。\CJKunderline{鄭玄}云:“黜,貶退也。”“黜”實退名,但此“黜”乃殺其身,絕其爵,故以“黜”為絕也。\CJKunderline{周公}此行普伐諸叛,獨言黜殷命者,定四年\CJKunderwave{左傳}云:“管蔡啟商,惎間王室”,則此叛\CJKunderline{武庚}為主,且顧\CJKunderwave{微子}之序,故特言黜殷命也。“以誅叛者之義大誥天下”,經皆是也。 \par}

{\noindent\shu\zihao{5}\fzkt “\CJKunderline{武王}”至“大誥”。正義曰:\CJKunderline{武王}既崩,\CJKunderline{管叔}、\CJKunderline{蔡叔}與紂子\CJKunderline{武庚}三人監殷民者又及淮夷共叛。\CJKunderline{周公}相\CJKunderline{成王},攝王政,將欲東征,黜退殷君\CJKunderline{武庚}之命,以誅叛之義大誥天下。史敘其事,作\CJKunderwave{大誥}。 \par}

大誥\footnote{陳大道以誥天下,遂以名篇。}

{\noindent\shu\zihao{5}\fzkt “大誥”。正義曰:此陳伐叛之義,以大誥天下,而兵兇戰危,非眾所欲,故言煩重。其自殷勤,多止而更端,故數言“王曰”。大意皆是陳說\CJKunderline{武庚}之罪,自言己之不能,言己當繼父祖之功,須去叛逆之賊,人心既從,卜之又吉,往伐無有不克,勸人勉力用心。此時\CJKunderline{武王}初崩,屬有此亂,\CJKunderline{周公}以臣伐君,天下未察其志,親弟猶尚致惑,何況疏賤者乎?\CJKunderline{周公}慮其有向背之意,故殷勤告之。陳壽云:“\CJKunderline{皋陶}之謨略而雅,\CJKunderline{周公}之誥煩而悉。何則?\CJKunderline{皋陶}與舜\CJKunderline{禹}共談,\CJKunderline{周公}與群下矢誓也。”其意或亦然乎。但\CJKunderwave{君奭}、\CJKunderwave{康誥}乃與\CJKunderline{召公}、\CJKunderline{康叔}語也,其辭亦甚委悉,抑亦當時設言,自好煩複也。管蔡導\CJKunderline{武庚}為亂,此篇略於管蔡者,猶難以伐弟為言,故專說\CJKunderline{武庚}罪耳。 \par}

王若曰:“猷!大誥爾多邦,越爾御事。\footnote{\CJKunderline{周公}稱\CJKunderline{成王}命,順大道以誥天下眾國,及於御治事者盡及之。○猷音由,道也。邦,馬本作“大誥繇爾多邦”。盡,津忍反。}弗弔,天降割於我家,\footnote{言周道不至,故天下兇害於我家不少。謂三監淮夷並作難。○吊音的,又如字。割,馬本作害。不少,馬讀“弗少延”為句。難,乃旦反。}不少延,洪惟我幼沖人,\footnote{兇害延大,惟累我幼童人。\CJKunderline{成王}言其不可不誅之意。○累,劣偽反。}嗣無疆大曆服。弗造哲,迪民康,\footnote{言子孫承繼祖考無窮大數,服行其政,而不能為智道以安人,故使叛。先自責。}矧曰其有能格知天命?\footnote{安人且猶不能,況其有能至知天命者乎?○矧,失忍反。}已!予惟小子,若涉淵水,予惟往求朕攸濟。\footnote{已,發端嘆辭也。我惟小子,承先人之業,若涉淵水,往求我所以濟渡。言祗懼。}


{\noindent\zhuan\zihao{6}\fzbyks 傳“\CJKunderline{周公}”至“及之”。正義曰:序雲“相\CJKunderline{成王}”,則“王若曰”者,稱\CJKunderline{成王}之言,故言“\CJKunderline{周公}稱\CJKunderline{成王}命”。實非王意,\CJKunderline{成王}爾時信流言,疑\CJKunderline{周公},豈命公伐管蔡乎?“猷”訓道也,故云“順大道以告天下眾國”也。鄭、王本“猷”在“誥”下。\CJKunderwave{漢書}王莽攝位,東郡太守翟義叛莽,莽依此作\CJKunderwave{大誥},其書亦“道”在“誥”下。此本“猷”在“大”上,言以道誥眾國,於文為便。但此經雲“猷”,\CJKunderwave{大傳}雲“大道”,古人之語多倒,猶\CJKunderwave{詩}稱“中谷”,谷中也。“多邦”之下雲於爾御事,是於諸國治事者盡及之也。\CJKunderline{鄭玄}云:“王,\CJKunderline{周公}也,\CJKunderline{周公}居攝,命大事,則權稱王。”惟名與器不可假人,\CJKunderline{周公}自稱為王,則是不為臣矣,大聖作則,豈為是乎? \par}

{\noindent\zhuan\zihao{6}\fzbyks 傳“兇害”至“之意”。正義曰:\CJKunderwave{釋詁}云:“延,長也。洪,大也。”此害長大,敗亂國家,經言惟我幼童人,謂損累之,故傳加“累”字,累我童人,言其不可不誅之意。鄭、王皆以“延”上屬為句,言害不少,乃延長之。王肅又以“惟”為念,向下為義,大念我幼童子與繼文武無窮之道。 \par}

{\noindent\zhuan\zihao{6}\fzbyks 傳“言子”至“自責”。正義曰:“嗣”訓繼也。言子孫承繼祖疆,境界則是無窮,大數長遠,“卜世三十,卜年七百”,是長遠也。 \par}

{\noindent\zhuan\zihao{6}\fzbyks 傳“安人”至“者乎”。正義曰:民近而天遠,以易而況難。天子必當至靈,至靈乃知天命,言己猶不能安民,明其不知天命。自責而謙。 \par}

敷賁敷前人受命,茲不忘大功。\footnote{前人,文武也。我求濟渡,在布行大道,在布陳文武受命,在此不忘大功。言任重。○賁,扶雲反,徐音憤。}予不敢閉於天降威用。\footnote{天下威用,謂誅惡也。言我不敢閉絕天所下威用而不行,將欲伐四國。}寧王遺我大寶龜,紹天明,即命。\footnote{安天下之王,謂\CJKunderline{文王}也。遺我大寶龜,疑則卜之,以繼天明,就其命而行之。言卜不可違。○遺,唯季反。}

{\noindent\zhuan\zihao{6}\fzbyks 傳“前人”至“任重”。正義曰:\CJKunderline{成王}前人,故為“文武”也。以涉水為喻,言求濟者,在於布行大道,行天子之政也。文武有大功德,故受天命,又當布陳文武受命所行之事也。陳行天子之政,又陳文武所行之事。在此不忘大功。“大功”,大平之功也。言己所任至重,不得不奉天道行誅伐也。 \par}

{\noindent\zhuan\zihao{6}\fzbyks 傳“天下”至“四國”。正義曰:王者征伐刑獄,象天震曜殺戮,則征伐者,天之所威用,謂誅惡是也。天有此道,王者用之。用之則開,不用則閉,言我不敢閉絕天之所下威用而不行之。既不敢不行,故將伐四國。 \par}

{\noindent\zhuan\zihao{6}\fzbyks 傳“安天”至“可違”。正義曰:紂為昏虐,天下不安,言\CJKunderline{文王}能安之,安天下之王謂\CJKunderline{文王}也。“遺我大寶龜”者,天子寶藏神龜,疑則卜之。繼天明道,就其命而行之,言卜吉則當行,不可違卜也。所以大寶龜皆得繼天明者,以天道玄遠,龜是神靈,能傳天意以示吉凶,故疑則卜之,以繼天明道。\CJKunderline{鄭玄}云:“時既卜,乃後出誥,故先云然。” \par}

{\noindent\shu\zihao{5}\fzkt “王若”至“即命”。正義曰:\CJKunderline{周公}雖攝王政,其號令大事則假\CJKunderline{成王}為辭。言王順大道而為言曰,我今以大道誥汝天下眾國,及於眾治事之臣。以我周道不至,故上天下其兇害於我家不少。言叛逆者多。此害延長寬大,惟累我幼童人。\CJKunderline{成王}自言害及己也。我之致此兇害,以我為子孫,承繼無疆界之大數,服行其政,不能為智道令民安,故使之叛。自責也。安民猶且不能,況曰其能至於知天之大命者乎?言己不能知天意也。復嘆而言,已乎!我惟小子,承先人之業,如涉淵水,惟往求我所以濟渡。言己恐懼之甚。我所求濟者,惟在布行大道,布陳前人\CJKunderline{文王}\CJKunderline{武王}受命之事,在我此身,不忘大功。既不忘大功,當誅叛逆,由此我不敢絕天之所下威用而不行之。言必將伐四國也。寧天下之王,謂\CJKunderline{文王}也。\CJKunderline{文王}遺我大寶龜,疑則就而卜之,以繼天明命,今我就受其命。言己就龜卜其伐之吉凶,已得吉也。 \par}

曰:‘有大艱於西土,西土人亦不靜,越茲蠢。’\footnote{曰,語更端也。四國作大難於京師,西土人亦不安,於此蠢動。○蠢,尺允反。難,乃旦反,下同,又如字。}殷小腆,誕敢紀其敘。\footnote{言殷後小腆腆之祿父,大敢紀其王業,欲復之。○腆,他典反,馬云:“至也。”誕,大旦反。父音甫,後同。}天降威,知我國有疵,\footnote{天下威,謂三叔流言,故祿父知我周國有疵病。○疵,在斯反,馬云:“叚也。”}民不康。


{\noindent\zhuan\zihao{6}\fzbyks 傳“曰語”至“蠢動”。正義曰:\CJKunderline{周公}丁寧其事,止而復言,別加一“曰”,語更端也。下言“王曰”,此不言“王”,史詳略耳。四國作逆於東,京師以為大艱,故言“作大難於京師”。“西土人亦不安”,亦如東方見其亂,不安也。\CJKunderwave{釋詁}云:“蠢,動也。”鄭云:“周民亦不定,其心騷動,言以兵應之。”當時京師無與應者,鄭言妄耳。 \par}

{\noindent\zhuan\zihao{6}\fzbyks 傳“言殷”至“復之”。正義曰:殷本天子之國,\CJKunderline{武庚}比之為小,故言“小腆”,“腆”是小貌也。\CJKunderline{鄭玄}云:“腆謂小國也。”王肅云:“腆,主也,殷小主謂祿父也。”“大敢紀其王業”,經紀王業,望復之也。 \par}

{\noindent\zhuan\zihao{6}\fzbyks 傳“天下”至“疵病”。正義曰:王肅云:“天降威者,謂三叔流言,當誅伐之。”言誅三叔是天下威也。\CJKunderwave{釋詁}云:“疵,病也。”鄭、王皆云:“知我國有疵病之瑕。” \par}

曰:‘予復。’反鄙我周邦。\footnote{祿父言我殷當復,欺惑東國人,令不安,反鄙易我周家。道其罪無狀。○令,力呈反。易,以豉反,下“其易”同。}今蠢,今翼日,民獻有十夫,予翼以於\xpinyin*{敉}\CJKunderline{寧}、\CJKunderline{武}圖功。\footnote{今天下蠢動,今之明日,四國人賢者有十夫來翼佐我周,用撫安武事,謀立其功。言人事先應。○敉,亡婢反。應,應對之應。}我有大事,休?朕卜並吉。\footnote{大事,戎事也。人謀既從,卜又並吉,所以為美。○並,必政反,注及篇末同。}

{\noindent\zhuan\zihao{6}\fzbyks 傳“祿父”至“無狀”。正義曰:祿父以父罪,滅殷身亦當死,幸得繼其先祀,宜荷天恩。反鄙薄輕易我周家,言其不識恩養,道其罪無狀也。漢代止有“無狀”之語,蓋言其罪大無可形狀也。近代已來遭重喪答人書云:“無狀招禍”,是古人之遺語也。 \par}

{\noindent\zhuan\zihao{6}\fzbyks 傳“今天”至“先應”。正義曰:\CJKunderline{武庚}既叛,聞者皆驚,故“今天下蠢動”,謂聞叛之日也。“今之明日”,聞叛之明日。以“獻”為賢,四國民內賢者十夫,來翼佐我周。十人史無姓名,直是在彼逆地,有先見之明,知彼必敗,棄而歸周。\CJKunderline{周公}喜其來降,舉以告眾,謂之為賢,未必是大賢也。“用撫安武事,謀立其功”,用此十夫為之。將欲伐叛,而賢者即來,言人事先應也。 \par}

{\noindent\zhuan\zihao{6}\fzbyks 傳“大事”至“為美”。正義曰:成十三年\CJKunderwave{左傳}云:“國之大事在祀與戎。”今論伐叛,知“大事,戎事也”。十夫來翼,人謀既從,卜又並吉,所以為美,美即經之“休”也。既言其休,乃說我卜,並言以成此休之意。\CJKunderline{鄭玄}云:“卜並吉者,謂三龜皆從也。”王肅云:“何以言美?以三龜一習吉,是言並吉,證其休也。”與孔異矣。 \par}

{\noindent\shu\zihao{5}\fzkt “曰有”至“並吉”。正義曰:上言為害不少,陳欲徵之意,未說\CJKunderline{武庚}之罪。更復發端言之曰,今四國叛逆,有大艱於西土。言作亂於東,與京師為難也。西土之人為此亦不得安靜,於此人情皆蠢蠢然動。殷後小國腆腆然之祿父,大敢紀其王業之次敘,而欲興復之。祿父所以敢然者,上天下威於三叔,以其流言欲下威誅之,祿父知我周國有此疵病,而欺惑東國人,令人不安。祿父謂人曰:“我殷復。”望得更為天子,反鄙易我周國。今天下蠢動,今之明日,四國民之賢者有十夫,不從叛逆,其來為我翼佐我周。於是用撫安武事,謀立其功。明祿父舉事不當,得賢者叛來投我,為我謀用。是人事先應如此,則我有兵戎大事,征伐必休美矣。人謀既從,我卜又並吉,是其休也。言往必克敵安民之意,告眾使知也。 \par}

“肆予告我友邦君,越尹氏、庶士御事,\footnote{以美,故告我友國諸侯,及於正官尹氏卿大夫、眾士御治事者。言謀及之。}曰:‘予得吉卜,予惟以爾庶邦,於伐殷逋播臣。’\footnote{用汝眾國,往伐殷逋亡之臣。謂祿父。○逋,布吾反。}爾庶邦君,越庶士御事,罔不反曰:‘艱大。’\footnote{汝眾國上下無不反曰:“征伐四國為大難。”敘其情以戒之。}


{\noindent\zhuan\zihao{6}\fzbyks 傳“以美”至“及之”。正義曰:“肆”訓故也,承上“休”之下,以其東征必美之故,我告友國君以下共謀之。“尹氏”,即\CJKunderwave{顧命}雲“百尹”氏也。“尹”,正也,諸官之正,謂卿大夫,故傳言“及於正官尹氏卿大夫”。“尹氏”即官也,總呼大夫為官氏也。上文“大誥爾多邦”,越爾御事”,無“尹氏庶士”,下之“爾庶邦君,越庶士御事”亦無“尹氏”,惟此及下文施義二者詳其文,餘略之,從可知也。 \par}

{\noindent\zhuan\zihao{6}\fzbyks 傳“用汝”至“祿父”。正義曰:“逋”,逃也。“播”謂播蕩逃亡之意。祿父殷君,謂之為“殷”。今日叛逆,是背周逃亡,故云用汝眾國,往伐彼殷君於我周家逋逃亡叛之臣。謂祿父也。 \par}

{\noindent\zhuan\zihao{6}\fzbyks 傳“汝眾”至“戒之”。正義曰:王以卜吉之故,將以諸國伐殷,且彼諸國之情,必有不欲伐者,無不反我之意,相與言曰:“征伐四國為大難。”言其情必如此,敘其情以戒之,使勿然也。鄭云:“汝國君及下群臣不與我同志者,無不反我之意,云:‘三監叛,其為難大。’”是言“反”者謂反上意,反是上意,則知“曰”者,相與言也。 \par}

民不靜,亦惟在王宮邦君室。\footnote{言四國不安,亦在天子諸侯教化之過。自責不能綏近以及遠。}越予小子,考翼不可徵,王害不違卜。\footnote{於我小子先卜敬成周道,若謂今四國不可徵,則王室有害,故宜從卜。}

{\noindent\zhuan\zihao{6}\fzbyks 傳“言四”至“及遠”。正義曰:“自責”惟當言天子教化之過,而並言諸侯者,化從天子佈於諸侯,道之不行,亦邦君之咎,見庶邦亦有過,故並言之。教化之過在於君身,而云“王宮邦君室”者,宮室是行化之處,故指以言之。 \par}

{\noindent\zhuan\zihao{6}\fzbyks 傳“於我”至“從卜”。正義曰:“翼”訓敬也,於我小子,先自考卜,欲敬成周道。汝庶邦御事等,若謂今四國不可徵,則周道不成,於王室有害,故宜從卜。“小子先卜”當謂初即位時,卜其欲成周道也。不可違卜,謂上“朕卜並吉”也。言欲徵卜吉,當從卜征之。 \par}

{\noindent\shu\zihao{5}\fzkt “肆予告”至“違卜”。正義曰:以人從卜吉為美之故,故我告汝有邦國之君,及於尹氏卿大夫、眾士治事者曰,我得吉卜,我惟與汝眾國往伐殷逋亡播蕩之臣。謂伐祿父也。汝國君及於眾治事者,無不反我之意,相與言曰:“伐此四國,為難甚大。”言其不欲徵也。汝不欲伐,罪我之由四國之民不安而叛者,亦惟在我天子王宮與邦君之室教化之過使之然。以此令汝難徵,過事在我。雖然,於我小子,先考疑而卜之,欲敬成周道,若謂四國難大不可徵,則於王室有害,不可違卜,宜從卜往徵也。 \par}

肆予沖人永思艱,曰,嗚呼!允蠢鰥寡,哀哉!\footnote{故我童人\CJKunderline{成王}長思此難而嘆曰:“信蠢動天下,使無妻無夫者受其害,可哀哉!”○鰥,故頑反。}予造天役,遺大投艱於朕身。\footnote{我周家為天下役事,遺我甚大,投此艱難於我身。言不得已。○予造,為也。馬云:“遺也。”}越予沖人,不卬自恤。義爾邦君,越爾多士、尹氏御事,\footnote{言徵四國,於我童人不惟自憂而已,乃欲施義於汝胤國君臣上下至御治事者。○卬,五剛反,我也。}綏予曰:‘無毖於恤。不可不成乃\CJKunderline{寧考}圖功。’\footnote{汝眾國君臣,當安勉我曰:“無勞於憂,不可不成汝寧祖聖考文武所謀之功。”責其以善言之助。○毖音秘。}


{\noindent\zhuan\zihao{6}\fzbyks 傳“我周”至“得已”。正義曰:為天子者,當役己以養天下,故“我國家為天下役事”,總言周家當救天下。此事遺我,故為甚大。以大役遺我,以為甚大,而又投擲此艱難之事於我身,謂當已之時有四國叛逆,言已職當靜亂,不得以己也。 \par}

{\noindent\zhuan\zihao{6}\fzbyks 傳“言徵”至“事者”。正義曰:卬,我。恤,憂也。四國叛逆,害及眾國,君得靜亂,則為大美。言徵四國,於我童人,不惟自憂而已。乃欲施義於汝眾國君臣,言難除則義施也。 \par}

{\noindent\zhuan\zihao{6}\fzbyks 傳“汝眾”至“之助”。正義曰:綏,安也。毖,勞也。言我既施義於汝,汝眾國君臣言得我之力,當安慰勉勸我曰:“無勞於憂。”令我無憂四國,眾國自來徵之。經言“寧”即\CJKunderline{文王},“考”即\CJKunderline{武王},故言“寧祖聖考”也。王以眾國反己,乃復設為此言,責其無善言助己。 \par}

{\noindent\shu\zihao{5}\fzkt “肆予衝”至“圖功”。正義曰:以汝等有難徵之意,故我童子\CJKunderline{成王}長思此難而嘆曰:“嗚呼!四國今叛,信蠢動天下,使鰥寡受害,尤可哀哉!我周家為天下役事,而遺我甚大,乃投此艱難於我身。此難須平,不可以已。今徵四國,於我童人不惟自憂而已,乃欲施義於汝眾國君,於汝多士尹氏治事之人。如此為汝計,汝君臣當安勉我曰:‘無勞於徵伐之憂,我諸侯當往共徵四國。汝王不可不成汝寧祖聖考所謀之功。’宜出此善言以助我。何謂違我不欲徵也?” \par}

已!予惟小子,不敢替上帝命。\footnote{不敢廢天命,言卜吉當必徵之。}天休於寧王,興我小邦周,寧王惟卜用,克綏受茲命。\footnote{言天美\CJKunderline{文王}興周者,以\CJKunderline{文王}惟卜之用,故能安受此天命。明卜宜用。}今天其相民,矧亦惟卜用。\footnote{人獻十夫,是天助民,況亦用卜乎?吉可知矣。亦亦\CJKunderline{文王}。○相,息亮反。}嗚呼!天明畏,弼我丕丕基。”\footnote{嘆天之明德可畏,輔成我大大之基業。言卜不可違也。○畏如字,徐音威。}王曰:“爾惟舊人,爾丕克遠省,爾知寧王若勤哉!\footnote{特命久老之人,知\CJKunderline{文王}故事者,大能遠省識古事,汝知\CJKunderline{文王}若彼之勤勞哉!目所親見,法之又明。○省,息井反。}天閟毖我成功所,予不敢不極卒寧王圖事。\footnote{閟,慎也。言天慎勞我周家成功所在,我不敢不極盡\CJKunderline{文王}所謀之事。謂致太平。○閟音秘。}肆予大化誘我友邦君,\footnote{我欲極盡\CJKunderline{文王}所謀,故大化天下,道我友國諸侯。}



{\noindent\zhuan\zihao{6}\fzbyks 傳“人獻”至“\CJKunderline{文王}”。正義曰:天之助民,乃是常道,而云“民獻十夫,是天助民”者,下雲“亦惟十人,迪知上帝命”,故以民獻十夫為天助民也。“王曰爾”至“休畢”。正義曰:既述\CJKunderline{文王}之事,王又命於眾曰,汝惟久老之人,汝大能遠省識古事,汝知寧王若此之勤勞哉!以老人目所親見,必知之也。以\CJKunderline{文王}勤勞如此,故天命慎勞來我周家,當至成功所在。天意既然,我不敢不極盡\CJKunderline{文王}所謀之事。\CJKunderline{文王}謀致太平,我欲盡行之。我欲盡\CJKunderline{文王}所謀,故我大為教化,勸誘我所友國君,共伐叛逆。天既輔助我周家有大化誠辭,其必成就我之眾民。天意既如此矣,我何其不於前\CJKunderline{文王}安民之道、謀立其功之處所而終竟之乎?天亦惟勞慎我民,若人有疾病,而欲已去之。天意於民如此之急,我何敢不於前安人\CJKunderline{文王}所受美命終畢之乎?以須終畢之故,故當誅除逆亂,安養下民,使之致太平。 \par}

{\noindent\zhuan\zihao{6}\fzbyks 傳“閟慎”至“太平”。正義曰:“閟,慎”,\CJKunderwave{釋詁}文。“天慎勞我周家”者,美其德當天心,慎惜又勞來勸勉之,使至成功所在,在於致太平也。天意欲使之然,我為\CJKunderline{文王}子孫,敢不極盡\CJKunderline{文王}所謀之事?\CJKunderline{文王}本謀,謂致太平。 \par}

天棐忱辭,其考我民,\footnote{言我周家有大化誠辭,為天所輔,其成我民矣。○棐,徐音匪,又芳鬼反。忱,市林反。}予曷其不於前寧人圖功攸終?\footnote{我何其不於前\CJKunderline{文王}安人之道、謀立其功所終乎?}天亦惟用勤毖我民,若有疾,\footnote{天亦勞慎我民欲安之,如人有疾,欲已去之。}予曷敢不於前寧人攸受休畢?”\footnote{天欲安民,我何敢不於前\CJKunderline{文王}所受美命終畢之?}

{\noindent\zhuan\zihao{6}\fzbyks 傳“言我”至“民矣”。正義曰:\CJKunderwave{釋詁}云:“棐,輔也。忱,誠也。”文承“大化”之下,知輔誠辭者,“言周家有大化誠辭,為天所輔”。“其成我民”,必為民除害,使得成也。 \par}

{\noindent\zhuan\zihao{6}\fzbyks 傳“天亦”至“去之”。正義曰:“亦”者,亦民之義也。君民共為一體,天慎勞使成功,亦當勤勞民使安寧,故言“亦”也。如疾,欲已去之,言天急於民至甚也。 \par}

{\noindent\zhuan\zihao{6}\fzbyks 傳“天欲”至“畢之”。正義曰:上雲“卒寧王圖事”,又云“圖功攸終”,此雲“攸受休畢”,“畢”,終也,三者文辭略同,義不甚異。天意惟言當終\CJKunderline{文王}之業,須徵逆亂之賊,\CJKunderline{周公}重兵慎戰,丁寧以勸民耳。 \par}

{\noindent\shu\zihao{5}\fzkt “已予”至“丕基”。正義曰:既敘眾國之情,告以必徵之意:“已乎!我惟小子,不敢廢上帝之命。卜吉不徵,是廢天命。從卜而興,乃有故事。天休美於安天下之\CJKunderline{文王}興我小國周者,以安民之王,惟卜是用,以此之故,安受此上天之命。明卜宜用之。今天助民矣,十夫佐周,是天助也。人事既驗,況亦如\CJKunderline{文王}惟卜之用,吉可知矣。嗚呼!而嘆天之明德可畏也,輔成我周家大大之基業。卜既得吉,不可違也。” \par}

王曰:“若昔朕其逝,朕言艱日思。\footnote{順古道,我其往東征矣。我所言國家之難備矣,日思念之。○日,人實反。難,乃旦反,下“為難”同。}若考作室,既厎法,厥子乃弗肯堂,矧肯構?\footnote{以作室喻治政也。父已致法,子乃不肯為堂基,況肯構立屋乎?不為其易,則難者可知。○厎,之履反。構,古候反。治,直吏反。}厥父菑,厥子乃弗肯播,矧肯獲?\footnote{又以農喻。其父已菑耕其田,其子乃不肯播種,況肯收穫乎?○菑,側其反,草也,田一歲曰菑。獲,戶郭反。}厥考翼,其肯曰:‘予有後,弗棄基?’\footnote{其父敬事創業,而子不能繼成其功,其肯言我有後,不棄我基業乎?今不正,是棄之。}肆予曷敢不越卬敉寧王大命?\footnote{作室農人,猶惡棄基,故我何敢不於今日撫循\CJKunderline{文王}大命以徵逆乎?○惡,烏路反。}若兄考,乃有友伐厥子,民養其勸弗救。”\footnote{若兄弟父子之家,乃有朋友來伐其子,民養其勸不救者,以子惡故。以此四國將誅而無救者,罪大故。}


{\noindent\zhuan\zihao{6}\fzbyks 傳“又以”至“獲乎”。正義曰:上言作室,此言治田,其取喻一也。上言“若考作室,既厎法”,此類上文,當雲“若父為農,既耕田”,從上省文耳。“菑”謂殺草,故治田一歲曰菑,言其始殺草也。“播”謂布種,后稷播殖百穀是也。定本雲“矧弗肯構”、“矧弗肯獲”,皆有“弗”字,檢孔傳所解,“弗”為衍字。 \par}

{\noindent\zhuan\zihao{6}\fzbyks 傳“其父”至“棄之”。正義曰:治田作室,為喻既同,故以此經結上二事。鄭、王本於“矧肯構”下亦有此一經,然取喻既同,不應重出。蓋先儒見下有而上無,謂其脫而妄增之。 \par}

{\noindent\zhuan\zihao{6}\fzbyks 傳“若兄”至“大故”。正義曰:此經大意,言兄不救弟,父不救子,發首“兄考”備文,“伐厥子”,不言“弟”,互相發見,傳言“兄弟父子之家”以足之。“民養其勸”,“民”謂父兄,為家長者也,養其心不退止也。 \par}

{\noindent\shu\zihao{5}\fzkt “王曰若”至“弗救”。正義曰:子孫成父祖之業,古道當然。王又言曰:“今順古昔之道,我其往東征矣。我所言國家之難備矣,日日思念之。乃以作室為喻,若父作室,營建基趾,既致法矣,其子乃不肯為之堂,況肯構架成之乎?又以治田為喻,其父菑耕其田,殺其草,已堪下種矣,其子乃不肯布種,況肯收穫乎?其此作室治田之父,乃是敬事之人,見其子如此,其肯言曰:‘我有後,不棄我基業乎?’必不肯為此言也。我若不終文武之謀,則文武之神亦如此耳,其肯道我不棄基業乎?作室農人猶惡棄其基業,故我何敢不於我身今日撫循安人之\CJKunderline{文王}大命,以征討叛逆乎?我今東征,無往不克,凡人兄及父與子弟為家長者,乃有朋友來伐其子,則民皆養其勸伐之心不救之。何則?以子惡故也。以逾伐四國,雖親如父兄,亦無救之者,以君惡故也。”言罪大不可不誅,無救所以必克也。顧氏以上“不卬自恤”傳雲“不惟自憂”,遂皆以“卬”為惟。但“卬”之為惟,非是正訓,觀孔意亦以不“卬”為惟義也。 \par}

王曰:“嗚呼!肆哉!爾庶邦君,越爾御事。\footnote{嘆今伐四國必克之故,以告諸侯及臣下御治事者。}爽邦由哲,亦惟十人,迪知上帝命。\footnote{言其故,有明國事、用智道十人蹈知天命。謂人獻十夫來佐周。}越天棐忱,爾時罔敢易法,矧今天降戾於周邦?\footnote{於天輔誠,汝天下是知無敢易天法,況今天下罪於周,使四國叛乎?}惟大艱人,誕鄰胥伐於厥室,爾亦不知天命不易?\footnote{惟大為難之人,謂三叔也。大近相伐於其室家,謂叛逆也。若不早誅汝,天下亦不知天命之不易也。○易,以豉反。}


{\noindent\zhuan\zihao{6}\fzbyks 傳“言其”至“佐周”。正義曰:此其必克之故也。“爽”,明也。“由”,用也。“有明國事、用智道”,言其有賢德也。蹈天者,識天命而履行之。此言“十人”,謂上文民獻十夫來佐周家者。此是賢人,賢人既來,彼無所與,是必克之效也。王肅云:“我未伐而知民弗救者,以民十夫用知天命故也。” \par}

{\noindent\zhuan\zihao{6}\fzbyks 傳“於天”至“叛乎”。正義曰:“於天輔誠”,言天之所輔,必是誠信。汝天下於是觀之,始知無敢變易天法。若易天法,則天不輔之,況今天下罪於周,使四國叛乎?以小況大,易法猶尚不可,況叛逆乎? \par}

{\noindent\zhuan\zihao{6}\fzbyks 傳“惟大”至“不易”。正義曰:以下句言相伐於其室家,室家自相伐,知“惟大為難之人,謂三叔也”。“大近相伐於其室家”者,三叔為周室至親,而舉兵作亂,是室家自相伐。為叛逆之罪,是變易天法之極,若汝諸國不肯誅之,是汝天下亦不知天命之不可變易也。王肅云:“惟大為難之人,謂管蔡也。大近相伐於其室家,明不可不誅也。管蔡犯天誅而汝不欲伐,則亦不知天命之不易也。” \par}

{\noindent\shu\zihao{5}\fzkt “王曰鳴”至“不易”。正義曰:既言四國無救之者,王曰,又言嘆今伐四國必克之故,告汝眾國君,及於汝治事之臣。所以知必克者,故有明國事、用智道者,亦惟有十人,匆人皆蹈知上天之命。謂民獻十夫來佐周家,此人既來,克之必也。於我天輔誠信之故,汝天下是知無敢變易天法者,若易法無信,則上天不輔,故無敢易法也。況今天下罪於周國,使四國叛逆。惟大為難之人,謂三叔等,大近相伐於其室家,自欲拔本塞源,反害周室,是其為易天法也。彼變易天法,若不早誅之,汝天下亦不知天命之不可變易也。 \par}

予永念曰,天惟喪殷,若穡夫,予曷敢不終朕畝?\footnote{稼穡之夫,除草養苗。我長念天亡殷惡主,亦猶是矣。我何敢不順天,終竟我壟畝乎?言當滅殷。○壟,力勇反。}天亦惟休於前寧人,予曷其極卜,敢弗於從?\footnote{天亦惟美於\CJKunderline{文王}受命,我何其極卜法,敢不於從?言必從也。}率寧人有指疆土,矧今卜並吉?\footnote{循\CJKunderline{文王}所有指意以安疆土則善矣,況今卜並吉乎?言不可不從。}肆朕誕以爾東征。天命不\xpinyin{僣}{jian4},卜陳惟若茲。”\footnote{以卜吉之故,大以汝眾東征四國。天命不僣差,卜兆陳列惟若此吉,必克之,不可不勉。}


{\noindent\zhuan\zihao{6}\fzbyks 傳“天亦”至“必從”。正義曰:“天亦惟美於\CJKunderline{文王}受命”,言\CJKunderline{文王}德當天心,天每事美之,故得受天命,是\CJKunderline{文王}之德大美也。\CJKunderline{文王}用卜,能受天命,今於我何其窮極\CJKunderline{文王}卜法,敢不從乎?言必從\CJKunderline{文王}卜也。 \par}

{\noindent\zhuan\zihao{6}\fzbyks 傳“循文”至“不從”。正義曰:\CJKunderline{文王}之旨意,欲今天下疆土皆得其宜。有叛逆者,自然須平定之。我直循彼\CJKunderline{文王}所有旨意伐叛,則已善矣,不必須卜筮也,況今卜並吉乎?言不可不從也。王肅云:“順\CJKunderline{文王}安人之道,有旨意盡天下疆土使皆得其所,不必須卜筮也,況今卜三龜皆吉,明不可不從也。” \par}

{\noindent\zhuan\zihao{6}\fzbyks 傳“以卜”至“不勉”。正義曰:“天命不僣”,天意去惡與善,其事必不僣差,言我善而彼惡也。“卜兆陳列惟若此吉”,言往必克之,不可不勉力也。 \par}

{\noindent\shu\zihao{5}\fzkt “予永”至“若茲”。正義曰:所以必當誅四國者,我長思念之曰,天惟喪亡殷國者,若稼穡之夫,務去草也,天意既然,我何敢不終我壟畝也?言穢草盡須除去,殷餘皆當殄滅也。天亦惟美於前寧人\CJKunderline{文王},我何其極\CJKunderline{文王}卜法,敢不於是從乎?言必從之也。我循彼寧人所有旨意以安疆上,不待卜筮,便即東征,已自善矣,況今卜東征而龜並吉?以吉之故,我大以爾東征四國。天命必不僣差,卜兆陳列惟若此吉,不可不從卜,不可不勉力也。 \par}

\section{微子之命第十【偽】}


\CJKunderline{成王}既黜殷命,殺\CJKunderline{武庚},\footnote{一名祿父。}命\CJKunderline{微子啟}代殷後,\footnote{啟知紂必亡而奔周,命為宋公,為湯後。}作\CJKunderwave{微子之命}。\footnote{封命之書。}


{\noindent\zhuan\zihao{6}\fzbyks 傳“啟知”至“湯後”。正義曰:啟知紂必亡,告父師少師而遁於荒野,“\CJKunderline{微子}作誥”是其事也。\CJKunderline{武王}既克紂,\CJKunderline{微子}乃歸之,非去紂即奔周也。傳言得封之由,故言其“奔周”耳。僖六年\CJKunderwave{左傳}云,許僖公見楚子,“面縛銜璧,大夫衰絰,士輿櫬。楚子問諸逢伯,對曰:‘昔\CJKunderline{武王}克殷,\CJKunderline{微子}啟如是。\CJKunderline{武王}親釋其縛,受其璧而祓之。焚其櫬,禮而命之,使復其所。’”\CJKunderwave{史記·宋世家}云:“\CJKunderline{武王}克殷,\CJKunderline{微子}啟乃持其祭器造于軍門,肉袒面縛,左牽羊,右把茅,膝行而前以告。\CJKunderline{武王}乃釋\CJKunderline{微子},復其位如故。”是言\CJKunderline{微子}克殷始歸周也。馬遷之書,辭多錯謬,“面縛”縛手於後,故口銜其璧,又安得“左牽羊,右把茅”也?要言歸周之事是其實耳。\CJKunderwave{樂記}云,\CJKunderline{武王}克殷,既下車,投殷之後於宋。則傳言復其位者,以其自縛為囚,釋之使從本爵,復其卿大夫之位。及下車即封於宋,以其終為殷後,故\CJKunderwave{樂記}雲“投殷之後”,爾時未為殷之後也。\CJKunderline{微子}初封於宋,不知何爵,此時因舊宋命之為公,令為湯後,使祀湯耳,不繼紂也。 \par}

{\noindent\shu\zihao{5}\fzkt “\CJKunderline{成王}”至“之命”。正義曰:\CJKunderline{成王}既黜殷君之命,殺\CJKunderline{武庚},乃命\CJKunderline{微子}啟代\CJKunderline{武庚}為殷後,為書命之。史敘其事,作\CJKunderwave{微子之命}。“黜殷命”,謂絕其爵也。“殺\CJKunderline{武庚}”,謂誅其身也。 \par}

微子之命\footnote{稱其本爵以名篇。}

{\noindent\shu\zihao{5}\fzkt “\CJKunderline{微子}之命”。正義曰:令寫命書之辭以為此篇,\CJKunderwave{君陳}、\CJKunderwave{君牙}、\CJKunderwave{冏命}皆此類也。 \par}

王若曰:“猷!殷王元子,\footnote{微子,\CJKunderline{帝乙}元子,故順道本而稱之。}惟稽古,崇德象賢。\footnote{惟考古典,有尊德象賢之義。言今法之。}統承先王,修其禮物,\footnote{言二王之後,各修其典禮,正朔服色,與時王並通三統。○正音政。}作賓於王家,與國咸休,永世無窮。\footnote{為時王賓客,與時皆美,長世無竟。}嗚呼!乃祖\CJKunderline{成湯},克齊聖廣淵,\footnote{言汝祖\CJKunderline{成湯}能齊德聖達廣大深遠,澤流後世。}皇天眷佑,誕受厥命。\footnote{大天眷顧湯,佑助之,大受其命。謂天命。}撫民以寬,除其邪虐,\footnote{撫民以寬政,放桀邪虐。湯之德。}功加於時,德垂後裔。\footnote{言湯立功加於當時,德澤垂及後世。裔,末也。}爾惟踐修厥猷,舊有令聞,\footnote{汝\CJKunderline{微子},言能踐湯德,久有善譽,昭聞遠近。○令聞,如字,又音問。}恪慎克孝,肅恭神人。予嘉乃德,曰篤不忘。\footnote{言\CJKunderline{微子}敬慎能孝,嚴恭神人,故我善汝德,謂厚不可忘。○篤,本又作竺,東谷反。}


{\noindent\zhuan\zihao{6}\fzbyks 傳“\CJKunderline{微子}”至“稱之”。正義曰:\CJKunderwave{呂氏春秋·仲冬紀}云:“紂之母生\CJKunderline{微子}啟與仲衍,尚為妾,已而為妻後生紂。紂父欲立啟為太子,太史據法而爭之曰:‘有妻之子,又不可立妾之子。’故紂為後。”鄭云:“\CJKunderline{微子}啟,紂同母庶兄也。若,順也。猷,道也。以其本是元子,故順道本而稱之。”\CJKunderwave{釋詁}云:“元、首,始也。”\CJKunderwave{易}曰:“元者,善之長也。” \par}

{\noindent\zhuan\zihao{6}\fzbyks 傳“言二”至“三統”。正義曰:\CJKunderwave{郊特牲}云:“天子存二代之後,猶尊賢也。尊賢不過二代。”\CJKunderwave{書傳}云:“王者存二王之後,與己為三,所以通三統,立三正。周人以日至為正,殷人以日至後三十日為正,夏人以日至後六十日為正。天有三統,土有三王,三王者,所以統天下也。”\CJKunderwave{禮運}云:“杞之郊也,\CJKunderline{禹}也。宋之郊也,契也。”是二王后為郊祭天,以其祖配之。鄭云:“所存二王后者,命使郊天,以天子禮祭其始祖受命之王,自行其正朔服色,此謂通天三統,是立二王后之義也。”此命首言“稽古”,則立先代之後,自古而有此法,不知從何代然也。孔意自夏以上不必改正,縱使正朔不改,典禮服色自當異也。“曰篤不忘”。正義曰:僖十二年\CJKunderwave{左傳}王命管仲之辭曰“謂督不忘”,則“曰”亦謂義。孔訓“篤”為厚,故傳雲“謂厚不可忘”。杜預以“督”為正,可謂正而不可忘也。 \par}

{\noindent\shu\zihao{5}\fzkt “王若曰猷殷王元子”。正義曰:王順道而言曰:“今以大道告汝殷王首子。”告之以下辭也。“曰猷”,如\CJKunderwave{大誥},言以道誥之。 \par}

上帝時歆,下民祗協,庸建爾於上公,尹茲東夏。\footnote{孝恭之人,祭祀則神歆享,施令則人敬和,用是封立汝於上公之位,正此東方華夏之國。宋在京師東。○歆,許今反。}欽哉!往敷乃訓,慎乃服命,率由典常,以蕃王室。\footnote{敬哉,敬其為君之德。往臨人布汝教訓,慎汝祖服命數,循用舊典,無失其常,以蕃屏周室。戒之。}

{\noindent\shu\zihao{5}\fzkt “慎乃服命”。正義曰:傳言“慎汝祖服命數”,謂祭湯廟得用天子之禮,服其殷之本服,命則上公九命,當慎之,無使乖禮制也。 \par}

弘乃烈祖,律乃有民,永綏厥位,毗予一人。\footnote{大汝烈祖\CJKunderline{成湯}之道,以法度齊汝所有之人,則長安其位,以輔我一人。言上下同榮慶。○毗,房脂反。}世世享德,萬邦作式,\footnote{言\CJKunderline{微子}累世享德,不忝厥祖,雖同公侯,而特為萬國法式。}俾我有周無斁。\footnote{汝世世享德,則使我有周好汝無厭。○俾,必爾反。斁音亦。好,呼報反。厭,於豔反。}嗚呼!往哉惟休,無替朕命。”\footnote{嘆其德,遣往之國。言當惟為美政,無廢我命。}

\CJKunderline{唐叔}得禾,異畝同穎,\footnote{\CJKunderline{唐叔},\CJKunderline{成王}母弟。食邑內得異禾也。畝,壟。穎,穗也。禾各生一壟而合為一穗。○穎,役領反。穗,似醉反,本亦作遂。}獻諸天子。\footnote{拔而貢之。}王命\CJKunderline{唐叔}歸\CJKunderline{周公}於東,\footnote{異畝同穎,天下和同之象,\CJKunderline{周公}之德所致。\CJKunderline{周公}東征未還,故命\CJKunderline{唐叔}以禾歸\CJKunderline{周公}。\CJKunderline{唐叔}後封晉。}作\CJKunderwave{歸禾}。\footnote{亡。}


{\noindent\zhuan\zihao{6}\fzbyks 傳“\CJKunderline{唐叔}”至“一穗”。正義曰:昭十五年\CJKunderwave{左傳}云:“叔父\CJKunderline{唐叔},\CJKunderline{成王}之母弟。”指言“\CJKunderline{唐叔}得禾”,知其“所食邑內得異禾”也。\CJKunderline{唐叔}食邑,書傳無文。\CJKunderwave{詩}述后稷種禾,於“實秀”之下乃言“實穎”,\CJKunderwave{毛傳}雲“穎垂”,言穗重而垂,是“穎”為穗也。“禾各生一壟而合為一穗”,言其異也。\CJKunderwave{書傳}云:“\CJKunderline{成王}之時,有三苗貫桑葉而生,同為一穗,其大盈車,長几充箱,民得而上諸\CJKunderline{成王}。”下傳雲“拔而貢之”,若是盈車之穗,不可手拔而貢,孔不用\CJKunderwave{書傳}為說也。 \par}

{\noindent\zhuan\zihao{6}\fzbyks 傳“異畝”至“封晉”。正義曰:禾者,和也,異畝同穎,是天下和同之象,\CJKunderline{成王}以為\CJKunderline{周公}德所感致。於時\CJKunderline{周公}東征未還,故命\CJKunderline{唐叔}以禾歸\CJKunderline{周公}於東也。歸禾年月,史傳無文,不知在啟金縢之先後也。王啟金縢,正當禾熟之月。若是前年得之,於時王疑未解,必不肯歸\CJKunderline{周公}。當是啟金縢之後,喜得東土和平而有此應,故以歸\CJKunderline{周公}也。\CJKunderline{唐叔}後封於晉,經史多矣,傳言此者,欲見此時未封,知在邑內得之。昭元年\CJKunderwave{左傳}稱“\CJKunderline{成王}滅唐,而封太叔焉”,所滅之唐即晉國是也。然則得禾之時,未封於唐,從後稱之為“\CJKunderline{唐叔}”耳。 \par}

{\noindent\shu\zihao{5}\fzkt “\CJKunderline{唐叔}”至“歸禾”。正義曰:\CJKunderline{成王}母弟\CJKunderline{唐叔},於其食邑之內得禾,下異畝壟,上同穎穗,以其有異,拔而貢於天子,以為\CJKunderline{周公}德所感致。於時\CJKunderline{周公}東征未反,王命\CJKunderline{唐叔}歸\CJKunderline{周公}於東,命有言辭。史敘其事,作\CJKunderwave{歸禾}之篇。 \par}

\CJKunderline{周公}既得命禾,旅天子之命,\footnote{已得\CJKunderline{唐叔}之禾,遂陳\CJKunderline{成王}歸禾之命,而推美\CJKunderline{成王}。善則稱君。}作\CJKunderwave{嘉禾}。\footnote{天下和同,政之善者,故\CJKunderline{周公}作書以“嘉禾”名篇告天下。亡。}

{\noindent\zhuan\zihao{6}\fzbyks 傳“已得”至“稱君”。正義曰:鄭雲“受王歸已禾之命與其禾”,以為既得命,“禾”謂復得禾,義當然矣。\CJKunderline{成王}歸禾之命必歸美\CJKunderline{周公},\CJKunderline{周公}陳歸禾之命又推美\CJKunderline{成王},是“善則稱君”之義也。“善則稱君”,\CJKunderwave{坊記}文也。 \par}

{\noindent\zhuan\zihao{6}\fzbyks 傳“天下”至“下亡”。正義曰:“嘉”訓善也,言此禾之善,故以善禾名篇。陳天子之命,故當佈告天下,此以善禾為書之篇名,後世同穎之禾遂名為“嘉禾”,由此也。二篇東征未還時事,\CJKunderline{微子}受命應在此篇後。篇在前者,蓋先封\CJKunderline{微子},後布此書故也。 \par}

{\noindent\shu\zihao{5}\fzkt “\CJKunderline{周公}”至“嘉禾”。正義曰:\CJKunderline{周公}既得王所命禾,乃陳天子歸禾之命為文辭,稱此禾之善,推美於\CJKunderline{成王}。史敘其事,作\CJKunderwave{嘉禾}之篇。 \par}

%%% Local Variables:
%%% mode: latex
%%% TeX-engine: xetex
%%% TeX-master: "../Main"
%%% End:
