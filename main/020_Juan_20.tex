%% -*- coding: utf-8 -*-
%% Time-stamp: <Chen Wang: 2024-04-02 11:42:41>

% {\noindent \zhu \zihao{5} \fzbyks } -> 注 (△ ○)
% {\noindent \shu \zihao{5} \fzkt } -> 疏

\chapter{卷二十}


\section{文侯之命第三十}


\CJKunderline{平王}錫\CJKunderline{晉文侯}秬鬯圭瓚,\footnote{以圭為杓柄謂之圭瓚。○\CJKunderline{平王},馬無平字。錫,星歷反,馬本作賜。秬音巨,鬯,敕亮反。瓚,才但反。杓,上灼反。柄,彼病反。}作\CJKunderwave{文侯之命}。\footnote{所以名篇。幽王為犬戎所殺,\CJKunderline{平王}立而東遷洛邑,\CJKunderline{晉文侯}迎送安定之,故錫命焉。}

文侯之命\footnote{\CJKunderline{平王}命為侯伯。}

{\noindent\zhuan\zihao{6}\fzbyks 傳“以圭”至“圭瓚”。正義曰:祭之初,酌鬱鬯之酒以灌屍。“圭瓚”者,酌鬱鬯之杓,杓下有槃,“瓚”即槃之名也;是以圭為杓之柄,故謂之“圭瓚”。\CJKunderwave{周禮·典瑞}云:“祼圭有瓚,以肆先王,以祼賓客。”鄭司農云:“於圭頭為器,可以挹鬯祼祭謂之瓚。以肆先王,灌先王祭也。”\CJKunderline{鄭玄}云:“肆,解牲體以祭,因以為名。”爵行曰“祼”。漢禮瓚槃大五升,口徑八寸,下有槃,口徑一尺。\CJKunderwave{詩}云:“瑟彼玉瓚,黃流在中。”\CJKunderwave{毛傳}云:“玉瓚,圭瓚也,黃金所以飾流鬯也。”鄭云:“黃流,秬鬯也。圭瓚之狀,以圭為柄,黃金為勺,青金為外,朱中央。”是說圭瓚之形狀也。\CJKunderwave{禮}無明文,而知其然者,\CJKunderwave{祭統}云:“君執圭瓚祼屍,大宗執璋瓚亞祼。”鄭云:“圭瓚、璋瓚,祼器也。以圭璋為柄,酌鬱鬯曰祼。”然則圭瓚、璋瓚惟柄以圭、璋為異,其瓚形則同。\CJKunderwave{考工記·玉人}云:“祼圭尺有二寸,有瓚,以祀廟。大璋、中璋九寸,邊璋七寸,厚寸,黃金勺,青金外,朱中,鼻寸。”鄭云:“鼻,勺流也,凡流皆為龍口也。三璋之勺,形如圭瓚。”是鄭以璋形如此,知圭瓚亦然。\CJKunderwave{毛傳}又云“九命然後錫以秬鬯圭瓚”,則\CJKunderline{晉文侯}於時九命為東西大伯,故得受此賜也。“秬鬯”從經為傳,故此惟解“圭瓚”。 \par}

{\noindent\zhuan\zihao{6}\fzbyks 傳“所以”至“命焉”。正義曰:\CJKunderwave{周本紀}云,幽王嬖褒姒,褒姒生子伯服。幽王廢申後,並去太子,用褒姒為後,伯服為太子。申侯怒,乃與西夷犬戎共攻殺幽王。於是諸侯乃與申侯共立太子宜臼,是為\CJKunderline{平王}。東徙於洛邑,避戎寇。隱六年\CJKunderwave{左傳}:“周桓公言於王曰:‘我周之東遷,晉鄭焉依。’”鄭語云:“\CJKunderline{晉文侯}於是乎定天子。”是迎送安定之,故\CJKunderline{平王}錫命焉。 \par}

{\noindent\zhuan\zihao{6}\fzbyks 傳“\CJKunderline{平王}命為侯伯”。正義曰:“伯”,長也,諸侯之長謂之伯也。僖元年\CJKunderwave{左傳}云:“凡侯伯,救患、分災、討罪,禮也。”是與諸侯之長為“侯伯”。\CJKunderline{王肅}云:“幽王既滅,\CJKunderline{平王}東遷,\CJKunderline{晉文侯}、鄭武公夾輔王室,晉為大國,功重,故\CJKunderline{平王}命為侯伯。” \par}

{\noindent\shu\zihao{5}\fzkt “\CJKunderline{平王}”至“之命”。正義曰:幽王嬖褒姒,廢申後,逐太子宜臼。宜臼奔申。申侯與犬戎既殺幽王,\CJKunderline{晉文侯}與鄭武公迎宜臼立之,是為\CJKunderline{平王},遷於東都。\CJKunderline{平王}乃以文侯為方伯,賜其秬鬯之酒,以圭瓚副焉,作策書命之。史錄其策書,作\CJKunderwave{文侯之命}。 \par}

王若曰:“父\CJKunderline{義和},\footnote{順其功而命之。文侯同姓,故稱曰父。\CJKunderline{義和},字也。稱父者非一人,故以字別之。○\CJKunderline{義和},馬云:“能以\CJKunderline{義和}諸侯。”義本作誼。別,彼列反。}丕顯\CJKunderline{文}、\CJKunderline{武},克慎明德,\footnote{大明乎!\CJKunderline{文王}、\CJKunderline{武王}之道,能詳慎顯用有德。}昭升於上,敷聞在下,惟時上帝集厥命於\CJKunderline{文王}。\footnote{更述\CJKunderline{文王}所以王也。言\CJKunderline{文王}聖德明升於天,而布聞在下居。惟以是,故上天集成其王命,德流子孫。○聞音問。王,於況反。}亦惟先正,克左右昭事厥闢,\footnote{言君既聖明,亦惟先正官賢臣能,左右明事其君,所以然。○闢,必亦反。}越小大謀猷,罔不率從,肆先祖懷在位。\footnote{文王君聖臣良,於小大所謀道德,天下無不循從其化,故我後世先祖歸在王位。}


{\noindent\zhuan\zihao{6}\fzbyks 傳“順其”至“別之”。正義曰:\CJKunderwave{覲禮}說天子呼諸侯之義曰:“同姓大國則曰伯父,其異姓則曰伯舅,同姓小國則曰叔父,其異姓則曰叔舅。”\CJKunderline{鄭玄}\CJKunderwave{禮}注云:“稱之以父與舅,親親之辭。”\CJKunderline{晉文侯}\CJKunderline{唐叔}之後,與王同姓,故稱曰“父”。\CJKunderwave{曲禮}天子謂二伯為伯父伯舅。計文侯為侯伯,天子當呼為“伯父”,此不云“伯”而直稱“父”者,尢親之也。\CJKunderwave{左傳}以文侯名仇,今呼曰“\CJKunderline{義和}”,知是字也。天子於同姓諸侯皆呼為“父”,稱“父”者非一人,若不稱其字,無以知是文侯,故以字別之。\CJKunderline{鄭玄}讀“義”為“儀”,儀、仇皆訓匹也,故名仇,字儀。古人名字不可皆令相配,不必然也。 \par}

{\noindent\zhuan\zihao{6}\fzbyks 傳“\CJKunderline{文王}”至“王位”。正義曰:“後世先祖”謂文武之後,在今王之先祖,成康以至宣幽皆是也。“懷”,歸也。“歸在王位”者,王位是其所有也,若歸向家然,故稱“歸”也。 \par}

{\noindent\shu\zihao{5}\fzkt “王若”至“在位”。正義曰:\CJKunderline{平王}順文侯之功,親之,敬而呼其字曰:“父\CJKunderline{義和}。”既呼其字,乃告以上世之事:“大明乎!\CJKunderline{文王}、\CJKunderline{武王}之道,能詳順顯用有德之人以為大臣。\CJKunderline{文王}之為王也,聖德明升於天。”言其道至天也。“又布聞於在下”。言其德被民也。“惟以是,故上天成其大命於\CJKunderline{文王},使之身為天子,澤流後世。文武聖明如此,亦惟先世長官之臣,能左右明事其君,君聖臣賢之故。於小大所謀道德,天下無有不循從其化,故我之先祖文武之後諸王,皆得歸在王位”。言先世聖王得賢臣之力,將說己無賢臣,故言此也。 \par}

嗚呼!閔予小子嗣,造天丕愆。\footnote{嘆而自痛傷也。言我小子而遭天大罪過,父死國敗,祖業隤隕。○予如字,又音與。愆,去虔反。隤,杜回反。隕,于敏反。}殄資澤於下民,侵戎我國家純。\footnote{言周邦喪亂,絕其資用惠澤於下民,侵兵傷我國及卿大夫之家,禍甚大。○殄,大見反。}即我御事,罔或耆壽俊在厥服,予則罔克。\footnote{所以遇禍,即我治事之臣,無有耆宿壽考俊德在其服位,我則材劣無能之致。}


{\noindent\zhuan\zihao{6}\fzbyks 傳“言周”至“甚大”。正義曰:此經所言,追敘幽王滅事。民不自治,立君以養之。民之資用,是王者佑助以得之。言周邦喪亂,不能撫佑下民,絕其資用惠澤於下民也。幽王之滅,由夷狄交侵,兵傷我國及卿大夫之家,其禍甚大。諸言“國家”者,皆謂國為“國家”,傳意欲見君臣俱被其害,故以“家”為卿大夫之家。\CJKunderline{王肅}云:“遭天之大愆,謂幽王為犬戎所殺,殄絕其先祖之澤於下民。侵犯兵寇,傷我國家甚大,謂犬戎也。” \par}

{\noindent\zhuan\zihao{6}\fzbyks 傳“所以”至“之致”。正義曰:此經亦是追敘往事,言幽王所以遇禍者,即我周家治事之臣,無有耆宿壽考俊德之人在其服位,致使有犬戎之禍,亦是我材劣無能之致。幽王之時,\CJKunderline{平王}被逐在外,國之興亡,非\CJKunderline{平王}所知,言我無能之致者,引過歸己,自懼將來複然,故下句思得賢臣。 \par}

{\noindent\shu\zihao{5}\fzkt “嗚呼!”至“罔克”。正義曰:王又嘆而自傷:“嗚呼!疲病者,是我小子繼嗣先王之位,遭天大罪過於我周家,父死國敗,傾覆祖業,致使周邦喪亂,絕其資用惠澤於下民。”言下民資用盡,致使而王澤竭也。“西夷犬戎,侵兵傷我國及卿大夫之家,其禍亦甚大也。所以遇此禍者,即我治事之臣,無有耆宿壽考俊德之人在其服位,我則材弱無能之致”。自恨已弱不能致得賢臣,恐又不能自立也。 \par}

曰惟祖惟父,其伊恤朕躬。嗚呼!有績,予一人永綏在位。\footnote{王曰:“同姓諸侯在我惟祖惟父列者,其惟當憂念我身。嗚呼!能有成功,則我一人長安在王位。”言恃諸侯。}父\CJKunderline{義和},汝克昭乃顯祖,\footnote{重稱字,親之。不稱名,尊之。言汝能明汝顯祖\CJKunderline{唐叔}之道,獎之。}汝肇刑文武,用會紹乃闢,追孝於前文人。\footnote{言汝今始法文武之道矣。當用是道合會繼汝君以善,使追孝於前文德之人。汝君,\CJKunderline{平王}自謂也。繼先祖之志為孝。○闢,扶亦反。}汝多修,扞我於艱,若汝,予嘉。”\footnote{戰功曰多,言汝之功多,甚修矣。乃扞我於艱難,謂救周,誅犬戎,汝功我所善之。○扞,下旦反,注同。}


{\noindent\zhuan\zihao{6}\fzbyks 傳“王曰”至“諸侯”。正義曰:文侯是同姓諸侯,王言已未得文侯之時,常望同姓助己。王私為言曰:“同姓諸侯在我惟祖惟父列者,惟當憂念我身。”“伊”訓惟也。望得同姓之間有憂己者。以思惟未得,更嘆而為言:“嗚呼!同姓諸侯若有能助我有功,則我一人長得安在王位。”言己恃賴諸侯,思得其人,在後果得文侯。告文侯以此言,言己思文侯之功。 \par}

{\noindent\zhuan\zihao{6}\fzbyks 傳“重稱”至“獎之”。正義曰:天子之於諸侯,當稱“父舅”而已。既呼其“父”,又稱其字,所以別他人也。初則別於他人,重則可以已矣。重稱其字者,親之也。\CJKunderwave{禮}君父之前曰名,朋友之交曰字。是名重於字也。輕前人則斥其名,尊前人則避其重。故不稱其名,尊之也。不於上文作傳,於此言“尊之”者,就此“親之”,並解之也。“昭乃顯祖”,不知所斥,以晉之上世有功名者惟有\CJKunderline{唐叔}耳,故知“明汝顯祖\CJKunderline{唐叔}之道”。所以勸獎之,令其繼\CJKunderline{唐叔}之業也。 \par}

{\noindent\zhuan\zihao{6}\fzbyks 傳“言汝”至“為孝”。正義曰:以其初有大功,終當不殞其業,故言“始法文武之道”。“當用是文武之道合會繼汝君以善”,令以功德佐汝君,使汝君繼前世,追行孝道於前世文德之人。“汝君”者,\CJKunderline{平王}自謂也。先祖之志,在於平定天下,故子孫繼父祖之志為孝也。 \par}

{\noindent\zhuan\zihao{6}\fzbyks 傳“戰功”至“所善”。正義曰:“戰功曰多”者,\CJKunderwave{周禮·司勳}文。又云:“王功曰勳,國功曰功,民功曰庸,事功曰勞,治功曰力,戰功曰多。”彼有此六功也。言功多殊於他人,故云“汝之功多,甚修矣”。言其功修整,美其功之善也。文侯之功,在於誅犬戎,立\CJKunderline{平王},言“乃扞蔽我於艱難”,知“謂救周,誅犬戎”也。“若”訓如也,如汝之功,我所嘉也。\CJKunderline{王肅}亦云:“如汝之功,我所善也。” \par}

{\noindent\shu\zihao{5}\fzkt “曰惟”至“予嘉”。正義曰:“王又言我以無能之致,私為言曰,同姓諸侯,惟我祖之列者,惟我父之列者,其惟當憂念我身。”又自傷嘆:“嗚呼!此諸侯等若有能助我有功,則我一人長安在王位。”言己無能,惟將賴諸侯也。又呼文侯字曰:“父\CJKunderline{義和},汝能明汝顯祖\CJKunderline{唐叔}之道,汝始法文武之道,用是道合會繼汝君以善,追孝於前世文德之人。救周之國,汝功為多,甚修矣。乃能扞蔽我於艱難。”謂救周,誅犬戎也。“如汝之功,是我所善”。陳其前功,以勸勉之。 \par}

王曰:“父\CJKunderline{義和},其歸視爾師,寧爾邦。\footnote{遣令還晉國,其歸視汝眾,安汝國內上下。○令,力呈反。}用賚爾秬鬯一卣,\footnote{黑黍曰秬,釀以鬯草。不言圭瓚,可知。卣,中樽也。當以錫命告其始祖,故賜鬯。○賚,力代反。卣音酉,又音由。釀,女亮反。}彤弓一,彤矢百,盧弓一,盧矢百,\footnote{彤,赤。盧,黑也。諸侯有大功,賜弓矢,然後專征伐。彤弓以講德習射,藏示子孫。○彤,徒冬反。}馬四匹。\footnote{馬供武用。四匹曰乘。侯伯之賜無常,以功大小為度。○供音恭。}


{\noindent\zhuan\zihao{6}\fzbyks 傳“黑黍”至“賜鬯”。正義曰:\CJKunderwave{釋草}云:“秬,黑黍。”李巡曰:“黑黍一名秬。”\CJKunderwave{周禮}:“鬱人掌和鬱鬯,以實彝而陳之。”鄭云:“鬱,鬱金香草也。築鬱金煮之以和鬯酒。”鄭眾云:“鬱為草若蘭。”又有“鬯人掌供秬鬯”。\CJKunderline{鄭玄}云:“鬯釀秬為酒,芬香調暢於上下也。”如彼鄭說,釀黑黍之米為酒,築鬱金之草煮以和之。此傳言“釀以鬯草”,似用鬯草合釀。不同者終是以鬯和黍米之酒,或先或後言之耳。\CJKunderwave{詩}美宣王賜召穆公云:“釐爾圭瓚,秬鬯一卣,告於文人。”知賜秬鬯者必以圭瓚副焉。此不言“圭瓚”,明並賜之,可知也。“卣,中尊也”,\CJKunderwave{釋器}文。孫炎云:“樽,彝為上,罍為下,卣居中。”郭璞曰:“在罍彝之間。”即犧象壺著大山,等六尊是也。\CJKunderwave{周禮·司尊彝}云:“春祠夏禴,祼用雞彝鳥彝。秋嘗冬烝,祼用斝彝黃彝。”則祭時實鬯酒於彝。此用卣者,未祭則盛於卣,及祭則實於彝,此初賜未祭,故盛以卣也。\CJKunderwave{詩}稱“告於文人”,\CJKunderwave{毛傳}云:“文人,文德之人也。”\CJKunderline{鄭玄}云:“王賜召虎以鬯酒一尊,使以祭其宗廟,告其先祖諸有德美見記也。”然則得秬鬯之賜,當遍告宗廟,此傳惟言告始祖者,舉祖之尊者言之耳。 \par}

{\noindent\zhuan\zihao{6}\fzbyks 傳“彤赤”至“子孫”。正義曰:“彤”字從丹,“玈”字從玄,故“彤,赤。玈,黑”也。是“諸侯有大弓,賜弓矢,然後專征伐”,\CJKunderwave{禮記·王制}文也。\CJKunderwave{周禮}“司弓矢掌六弓”,其名王、弧、夾、庾、唐、大。\CJKunderline{鄭玄}云:“六者弓異體之名也。往體寡,來體多,曰王、弧。往體多,來體寡,曰夾、庾。往體來體若一,曰唐、大。”經又云:“唐弓大弓,以授學射者、使者、勞者。”鄭云:“學射者弓用中,後習強,弱則易也。使者、勞者,弓亦用中,遠近可也。勞者勤勞王事,若\CJKunderline{晉文侯}受弓矢之賜者。”\CJKunderline{鄭玄}以此“彤弓”、“玈弓”為\CJKunderwave{周禮}“唐弓”、“大弓”。“唐”、“大”是弓強弱之名,“彤”、“玈”是弓赤黑之色,孔意亦當然也。此傳及\CJKunderwave{毛傳}皆云“彤弓以講德習射”,用\CJKunderwave{周禮}為說也。唐弓大弓以授學射者,是習射也;授使者、勞者,是講德也。講論知其有德,乃賜之耳。襄八年\CJKunderwave{左傳}云,晉範宣子來聘,季武子賦\CJKunderwave{彤弓}。宣子曰:“城濮之役,我先君文公受彤弓於襄王,以為子孫藏。”杜預云:“藏之以示子孫。” \par}

{\noindent\zhuan\zihao{6}\fzbyks 傳“馬供”至“為度”。正義曰:六畜特以馬賜之者,為“馬供武用”故也。\CJKunderwave{周禮·校人}云:“乘馬一師四圉。”圉養一馬,是四匹曰乘,乘車必駕四馬故也。\CJKunderwave{司勳}云:“凡賞無常,輕重視功。”是“侯伯之賜無常,以功大小為度”。 \par}

父往哉!柔遠能邇,惠康小民,無荒寧。\footnote{父往歸國哉!懷柔遠人,必以文德。能柔遠者必能柔近,然後國安。安小人之道必以順,無荒廢人事而自安。}簡恤爾都,用成爾顯德。”\footnote{當簡閡汝所任,憂治汝都鄙之人,人和政治,則汝顯用有德之功成矣。不言鄙,由近以及遠。○核,戶革反。治,直吏反。}

{\noindent\zhuan\zihao{6}\fzbyks 傳“父往”至“自安”。正義曰:\CJKunderwave{論語}云:“遠人不服,則修文德以來之。”是“懷柔遠人,必以文德”也。能柔遠者必能柔近,遠近俱安,然後國安。“惠”,順也。“康”,安也。言順安小民者,安小民之道,必以順道安之,故言順安也。“順”者,順小民之心為其政也。\CJKunderwave{論語}云“因民之所利而利之”,是順安也。 \par}

{\noindent\zhuan\zihao{6}\fzbyks 傳“當簡”至“及遠”。正義曰:“簡恤”者,共有“爾都”之文,當簡閡殃都內善人而任之,令以德憂治汝都鄙之人。人和政治,則汝顯用有德之功成矣。言用賢之名既成,國君之治亦成也。鄭云:“都,國都也。”“鄙”,邊邑也。言“都”不言“鄙”,由近以及遠也。 \par}

{\noindent\shu\zihao{5}\fzkt “王曰”至“顯德”。正義曰:王既陳其功,乃賚賜之。王曰:“父\CJKunderline{義和},其當歸汝晉國,視汝眾民,安汝國內上下。用賜汝秬鬯之酒一卣樽,歸以告祭汝之始祖。又賜汝彤弓一,彤矢百,玈弓一,玈矢百,馬四匹。父往歸國哉!必以文德安彼遠人。欲安遠,必能安近,是遠近乃得安耳。當以順道安汝之小民,無得荒廢人事以自安逸。簡閡汝所任之臣,憂治汝都鄙之人民,用成汝顯明之德。”戒使歸國善治民也。 \par}

\section{費誓第三十一}


魯侯\CJKunderline{伯禽}宅曲阜,\footnote{治封之國居曲阜。\CJKunderline{伯禽},魯侯名。}徐、夷並興,東郊不開,\footnote{徐戎、淮夷並起,為寇於魯,故東郊不開。○開,舊讀皆作開,馬本作闢。}作\CJKunderwave{費誓}。\footnote{魯侯徵之於費地而誓眾也。諸侯之事而連帝王,\CJKunderline{孔子}序\CJKunderwave{書}以魯有治戎征討之備,秦有悔過自誓之戒,足為世法,故錄以備王事,猶\CJKunderwave{詩}錄商魯之\CJKunderwave{頌}。○費音秘。}

\xpinyin{費}{bi4}誓\footnote{費,魯東郊之地名。}


{\noindent\zhuan\zihao{6}\fzbyks 傳“徐戎”至“不開”。正義曰:經稱“淮夷、徐戎”,序言“徐、夷”,略之也。此戎夷在魯之東,諸侯之制,於郊有門,恐其侵逼魯境,故東郊之門不開。 \par}

{\noindent\zhuan\zihao{6}\fzbyks 傳“費魯”至“地名”。正義曰:\CJKunderwave{甘誓}、\CJKunderwave{牧誓}皆至戰地而誓,知“費”非戰地者,“東郊不開”,則戎、夷去魯近矣。此誓令其治兵器,具糗糧,則是未出魯境,故知“費”是魯東郊地名,非戰處也。 \par}

{\noindent\shu\zihao{5}\fzkt “魯侯”至“費誓”。正義曰:魯侯\CJKunderline{伯禽}於\CJKunderline{成王}即政元年始就封於魯,居曲阜之地。於時徐州之戎、淮浦之夷並起,為寇於魯,東郊之門不敢開闢。魯侯時為方伯,率諸侯徵之,至費地而誓戒士眾。史錄其誓辭,作\CJKunderwave{費誓}。 \par}

公曰:“嗟!人無譁,聽命。\footnote{\CJKunderline{伯禽}為方伯,監七百里內之諸侯,帥之以徵。嘆而敕之,使無喧譁,欲其靜聽誓命。○譁,戶瓜反。監,工銜反。}徂茲淮夷、徐戎並興。\footnote{今往徵此淮浦之夷、徐州之戎,並起為寇。此戎夷帝王所羈縻統敘,故錯居九州之內,秦始皇逐出之。}


{\noindent\zhuan\zihao{6}\fzbyks 傳“\CJKunderline{伯禽}”至“誓命”。正義曰:\CJKunderwave{禮}諸侯不得專征伐,惟州牧於當州之內有不順者,得專征之。於時\CJKunderline{伯禽}為方伯,監七百里內之諸侯,故得帥之以徵戎夷。\CJKunderwave{王制}云:“千里之外設方伯。”以八州八伯,是州別立一賢侯以為方伯,即\CJKunderwave{周禮·大宗伯}云“八命作牧”是也。\CJKunderwave{禮記·明堂位}云:“封\CJKunderline{周公}於曲阜,地方七百里。”孔意以周之大國不過百里,\CJKunderwave{禮記}云“七百里”者,監此七百里內之諸侯,非以七百里地並封\CJKunderline{伯禽}也。下云“魯人三郊三遂”,指言“魯人”,明於時軍內更有諸侯之人,故知帥七百里內諸侯之人,以之共徵也。鄭云:“人謂軍之士眾及費地之民。”案下句令填塞坑阱,必使軍旁之民塞之,或當如鄭言也。 \par}

{\noindent\zhuan\zihao{6}\fzbyks 傳“今往”至“出之”。正義曰:\CJKunderwave{詩}美宣王命程伯休父,“率彼淮浦,省此徐土”。知“淮夷”是淮浦之夷,“徐戎”是徐州之戎也。四海之名,東方曰夷,西方曰戎,謂在九州之外。此徐州、淮浦,中夏之地而得有戎夷者,此戎夷帝王之所羈縻而統敘之,不以中國之法齊其風俗,故得雜錯居九州之內。此\CJKunderline{伯禽}之時有淮浦者,淮浦之夷並起,\CJKunderwave{詩}美宣王命召穆公平淮夷,則戎夷之處中國久矣。漢時內地無戎夷者,秦始皇逐出之。始皇之崩至孔之初,惟可三四十年,古老尢在,及見其事,故孔得親知之也。\CJKunderline{王肅}云:“皆紂時錯居中國。”經傳不說其事,無以知紂時來也。 \par}

善\xpinyin*{敹}乃甲冑,敿乃幹,無敢不弔。\footnote{言當善簡汝甲鎧冑兜鍪,施汝楯紛,無敢不令至攻堅使可用。○敹,了雕反。敿,居表反。吊音的。鎧,苦代反。兜,丁侯反。鍪音矛。楯,常準反,又音允。紛,芳雲反。令,力呈反。}備乃弓矢,鍛乃戈矛,礪乃鋒刃,無敢不善。\footnote{備汝弓矢,弓調矢利。鍛練戈矛,磨礪鋒刃。皆使無敢不功善。○鍛,丁亂反。礪,力世反。練,來見反。}

{\noindent\zhuan\zihao{6}\fzbyks 傳“言當”至“可用”。正義曰:\CJKunderwave{世本}云:“杼作甲。”宋仲子云:“少康子杼也。”\CJKunderwave{說文}云:“胄兜鍪也。”“兜鍪”,首鎧也,經典皆言“甲冑”。秦世已來始有“鎧”、“兜鍪”之文。古之作甲用皮,秦漢已來用鐵,“鎧”、“鍪”二字皆從金,蓋用鐵為之,而因以作名也。甲冑為有善有惡,故令 簡,取其善者。鄭云:“ 謂穿徹之。”謂甲繩有斷絕,當使 理穿治之。“幹”是楯也,“敿乃幹”,必施功於楯,但楯無施功之處,惟系紛於楯,故以為“施汝楯紛”。紛如綬而小,繫於楯以持之,其以為飾。鄭云“敿尢系也”,\CJKunderline{王肅}云“敿楯當有紛系持之”,是相傳為此說也。“吊”訓至也,無敢不令至極,攻堅使可用。鄭云:“至,尢善也。” \par}

{\noindent\zhuan\zihao{6}\fzbyks 傳“備汝”至“功善”。正義曰:“備”訓具也。每弓百矢,弓十矢千,使其數備足,令弓調矢利。案\CJKunderwave{毛傳}云“五十矢為束”,或臨戰用五十矢為束。凡金為兵器,皆須鍛礪,有刃之兵,非獨戈矛而已。云“鍛練戈矛,磨礪鋒刃,令其文互相通稱。諸侯兵器,皆使無敢不功善,令皆利快也”。 \par}

{\noindent\shu\zihao{5}\fzkt “公曰”至“不善”。正義曰:魯侯將徵徐戎,召集士眾,嘆而敕之。公曰:“嗟!在軍之人,無得喧譁,皆靜而聽我誓命。在往徵此淮浦之夷、徐州之戎,以其並起為寇故也。汝等善簡擇汝之甲冑,施汝楯紛,無敢不令至攻極堅。備汝弓矢,一弓百矢,令弓調矢利。鍛練汝之戈矛,磨礪汝之鋒刃,無敢不使皆善。”戒之使善,言不善將得罪也。 \par}

“今惟淫舍牿牛馬,\footnote{今軍人惟大放舍牿牢之牛馬,言軍所在必放牧也。○牿,工毒反。}杜乃擭,\xpinyin*{敜}乃阱,無敢傷牿。牿之傷,汝則有常刑。\footnote{擭,捕獸機檻,當杜塞之。阱,穿地陷獸,當以土窒敜之。無敢令傷所以牿牢之牛馬。牛馬之傷,汝則有殘人畜之常刑。○杜,本又作度攵。擭,華化反,徐戶覆反。敜,徐乃協反,又乃結反。阱,在性反。檻,戶減反。窒,珍慄反。畜,許六反,又醜六反。}


{\noindent\zhuan\zihao{6}\fzbyks 傳“今軍”至“放牧”。正義曰:“淫”訓大也。\CJKunderwave{周禮}:“充人掌系祭祀之牲牷。祀五帝,則繫於牢,芻之三月。”\CJKunderline{鄭玄}云:“牢,閒也。”“校人掌王馬之政,天子十有二閒,馬六種。”然則養牛馬之處謂之牢閒,牢閒是周衛之名也。此言大舍牿牛馬,則是出之牢閒,牧於野澤,令其逐草而牧之。故謂此牢閒之牛馬為“牿牛馬”,而知“牿”即閒牢之謂也。故言“大放舍牿牢之牛馬”,言軍人所在,必須放牧此告軍旁之民也。既言牛馬在牿,遂以“牿”為牛馬之名,下云“無敢傷牿”,謂傷牛馬,牿之傷謂牛馬傷也。\CJKunderline{鄭玄}以“牿為桎梏之梏,施梏於牛馬之腳,使不得走失”。 \par}

{\noindent\zhuan\zihao{6}\fzbyks 傳“擭捕”至“常刑”。正義曰:\CJKunderwave{周禮}:冥氏掌“為阱擭以攻猛獸”。知“阱”“擭”皆是捕獸之器也。擭以捕虎豹,穿地為深坑,又設機於上,防其躍而出也。阱以捕小獸,穿地為深坑,入必不能出,其上不設機也。阱以穿地為名,擭以得獸為名,擭亦設於阱中,但阱不設機為異耳。“杜”,塞之;“窒”,敜之;皆閉塞之義。使之填坑廢機,無敢令傷所放牿牢之牛馬。牛馬之傷,汝則有殘人畜之常刑。今律文:“施機槍作坑阱者,杖一百。梢閹之畜產者,償所減價。”\CJKunderline{王肅}云:“杜,閉也。擭,所以捕禽獸機檻之屬。敜,塞也。阱,穿地為之,所以陷墮之。恐害牧牛馬,故使閉塞之。”\CJKunderline{鄭玄}云:“山林之田,春始穿地為阱,或設擭其中,以遮獸。擭,作㓵也。 \par}

{\noindent\shu\zihao{5}\fzkt “今惟”至“常刑”。正義曰:此戒軍旁之民也。今軍人惟欲大放舍牿牢之牛馬,令牧於野澤杜。汝捕獸之擭,塞汝陷獸之阱,無敢令傷所放牿牢之牛馬。牛馬之傷,汝則有殘害人畜之常刑。 \par}

馬牛其風,臣妾逋逃,勿敢越逐,\footnote{馬牛其有風佚,臣妾逋亡,勿敢棄越壘伍而求逐之。役人賤者男曰臣,女曰妾。○逋,布吳反。佚音逸。}祗復之,我商賚爾。\footnote{眾人其有得佚馬牛、逃臣妾,皆敬還復之,我則商度汝功,賜與汝。○商如字,徐音章。賚,力代反,徐音來。度,待洛反。}乃越逐,不復,汝則有常刑。\footnote{越逐為失伍,不還為攘盜,汝則有此常刑。○攘,如羊反。}無敢寇攘,逾垣牆,\footnote{軍人無敢暴劫人,逾越人垣牆,物有自來者,無敢取之。○垣音袁。}竊馬牛,誘臣妾,汝則有常刑。\footnote{軍人盜竊馬牛,誘偷奴婢,汝則有犯軍令之常刑。}甲戌,我惟徵徐戎。\footnote{誓後甲戌之日,我惟徵之。}峙乃糗糧,無敢不逮,汝則有大刑。\footnote{皆當儲峙汝糗糒之糧,使足食,無敢不相逮及,汝則有乏軍興之死刑。○峙,直裡反。\CJKunderwave{爾雅}云:“具也。”糗,去九反,一音昌紹反。糧音良。糒音備。}魯人三郊三遂,峙乃楨幹。甲戌,我惟築,\footnote{總諸國之兵,而但稱魯人。峙具楨幹,道近也。題曰楨,旁曰幹。言“三郊三遂”,明東郊距守不峙,甲戌日當築攻敵壘距堙之屬。○楨,徐音貞。幹,工翰反。築,陟六反。守,手又反。堙音因。}


{\noindent\zhuan\zihao{6}\fzbyks 傳“馬牛”至“曰妾”。正義曰:僖四年\CJKunderwave{左傳}云:“唯是風馬牛不相及也。賈逵云:“風,放也。牝牡相誘謂之風。”然則馬牛風佚,因牝牡相逐而遂至放佚遠去也。“逋”亦逃也。軍士在軍,當各守部署,止則有壘壁,行則有隊伍,勿敢棄越壘伍而遠求逐之。\CJKunderwave{周禮}太宰“以九職任萬民”,“八曰臣妾,聚斂疏材”。僖十七年\CJKunderwave{左傳}云,晉惠公之妻“梁嬴孕,過期。卜招父與其子卜之。其子曰:‘將生一男一女。’招曰:‘然。男為人臣,女為人妾。’”是“役人賤者男曰臣,女曰妾”也。古人或以婦女從軍,故云“臣妾逋逃”也。 \par}

{\noindent\zhuan\zihao{6}\fzbyks 傳“皆當”至“死刑”。正義曰:“峙”,具也。預貯米粟謂之“儲峙”。鄭眾云:“糗,熬大豆及米也。”\CJKunderwave{說文}云:“糗,熬米麥也。”\CJKunderline{鄭玄}云:“糗,搗熬谷也。”謂熬米麥使熟,又搗之以為粉也。“糒”,乾飯也。“糗糒”是行軍之糧。皆當儲峙汝糗糒之糧,使在軍足食。“無敢不相逮及”,謂儲糧少,不及眾人,汝則有乏軍興之死刑。興軍征伐而有乏少,謂之“乏軍興”。今律:“乏軍興者斬。” \par}

{\noindent\zhuan\zihao{6}\fzbyks 傳“總諸”至“之屬”。正義曰:指言“魯人”,明更有他國之人。總諸國之兵,而但謂魯人。峙具楨幹,為道近故也。峙具楨幹以擬築之用。“題曰楨”,謂當牆兩端者也。“旁曰幹”,謂在牆兩邊者也。\CJKunderwave{釋詁}云:“楨,幹也。”舍人曰:“楨,正也,築牆所立兩木也。幹所以當牆兩邊障土者。”“三郊三遂”謂魯人三軍。\CJKunderwave{周禮·司徒}萬二千五百家為鄉。\CJKunderwave{司馬法}:“萬二千五百人為軍。”\CJKunderwave{小司徒}云:“凡起徒役,無過家一人。”是家出一人,一鄉為一軍。天子六軍,出自六鄉,則諸侯大國三軍,亦當出自三鄉也。\CJKunderwave{周禮}又云,萬二千五百家為遂。\CJKunderwave{遂人職}云:“以歲時稽其人民,簡其兵器,以起徵役。”則六遂亦當出六軍,鄉為正,遂為副耳。鄭眾云:“六遂之地在王國百里之外。”然則王國百里為郊,鄉在郊內,遂在郊外。\CJKunderwave{釋地}云:“邑外謂之郊。”孫炎曰:“邑,國都也。設百里之國,去國十里為郊。”則諸侯之制,亦當鄉在郊內,遂在郊外。此言“三郊三遂”者,“三郊”謂三鄉也。蓋使三鄉之民,分在四郊之內,三遂之民,分在四郊之外,鄉近於郊,故以郊言之。鄉遂之民,分在國之四面,當有四郊四遂,惟言“三郊三遂”者,明東郊令留守,不令峙楨幹也。上云“甲戌,我惟徵徐戎”,此云“甲戌,我惟築”,期以至日即築,當築攻敵之壘距堙之屬。\CJKunderwave{兵法}:“攻城築土為山,以闚望城內,謂之距堙。”襄六年\CJKunderwave{左傳}云:“晏弱城東陽,而遂圍萊。甲寅,堙之環城,傅於堞。”杜預云:“堞,女牆也。堙,土山也。周城為土山及女牆。”宣十五年\CJKunderwave{公羊傳}楚子圍宋,“使司馬子反乘堙而窺宋城,宋華元亦乘堙而出見之”。何休云:“堙,距堙,上城具也。”是攻敵城壘必有距堙,知“築”者,築距堙之屬也。 \par}

無敢不供,汝則有無餘刑,非殺。\footnote{峙具楨幹,無敢不供。不供,汝則有無餘之刑。刑者非一也,然亦非殺汝。○供音恭。}魯人三郊三遂,峙乃芻茭,無敢不多,汝則有大刑。”\footnote{郊遂多積芻茭,供軍牛馬。不多,汝則亦有乏軍興之大刑。○芻,初俱反,茭音交。}

{\noindent\zhuan\zihao{6}\fzbyks 傳“峙具”至“殺汝”。正義曰:上云“無敢不逮”,此云“無敢不供”,下云“無敢不多”,文異者,糗糧難備,不得偏少,故云“無敢不逮”;楨幹易得,惟恐闕事,故云“無敢不供”;芻茭賤物,惟多為善,故云“無敢不多”,量事而為文也。“不供,汝則有無餘之刑”者,言刑者非一,謂閤家盡刑之。\CJKunderline{王肅}云:“汝則有無餘刑,父母妻子同產皆坐之,無遺免之者,故謂無餘之刑。然入於罪隸,亦不殺之。”\CJKunderline{鄭玄}云:“無餘刑非殺者,謂盡奴其妻子,不遺其種類,在軍使給廝役,反則入於罪隸、舂槁,不殺之。”\CJKunderwave{周禮·司厲}云:“其奴,男子入於罪隸,女子入於舂槁。”\CJKunderline{鄭玄}云:“奴,從坐而沒入縣官者,男女同名。”鄭眾云:“輸於罪隸、舂人、槁人之官也。”然不供楨幹,雖是大罪,未應緣坐盡及家人,蓋亦權以脅之,使勿犯耳。○“芻茭”。正義曰:鄭云:“茭,乾芻也。” \par}

{\noindent\shu\zihao{5}\fzkt “馬牛”至“常刑”。正義曰:馬牛其有放佚,臣妾其有逋逃,汝無敢棄越壘伍而遠求逐之。其有得逸馬牛、逃臣妾,皆敬還復之,歸於本主,我則商度汝功,賞賜汝。汝若棄越壘伍,遠求逐馬牛臣妾,及有得馬牛臣妾不肯敬還復歸本主者,汝則有常刑。” \par}

\section{秦誓第三十二}


\CJKunderline{秦穆公}伐鄭,\footnote{遣三帥帥師往伐之。○事見魯僖公三十三年。三帥,謂孟明視、西乞術、白乙丙。帥,色類反,下注同。}\CJKunderline{晉襄公}帥師敗諸崤,\footnote{崤,晉要塞也。以其不假道,伐而敗之,囚其三帥。○崤,戶交反。塞,悉代反。假,工下反。}還歸,作\CJKunderwave{秦誓}。\footnote{晉舍三帥,還歸秦,穆公悔過作誓。}

秦誓\footnote{貪鄭取敗,悔而自誓。}


{\noindent\zhuan\zihao{6}\fzbyks 傳“遣三”至“伐之”。正義曰:\CJKunderwave{左傳}僖三十年,晉文公與\CJKunderline{秦穆公}圍鄭,鄭使燭之武說秦伯,秦伯竊與鄭人盟,使杞子、逢孫、揚孫戍之,乃還。三十二年,杞子自鄭使告於秦曰:“鄭人使我掌其北門之管,若潛師以,來國可得也。”穆公訪諸蹇叔,蹇叔曰:“不可。”公辭焉。召孟明、西乞、白乙,使出師伐鄭。是“遣三帥帥師往伐之”事也。序言“穆公伐鄭”,嫌似穆公親行,故辨之耳。 \par}

{\noindent\zhuan\zihao{6}\fzbyks 傳“崤晉”至“三帥”。正義曰:杜預云:“殽在弘農澠池縣西。”築城守道謂之“塞”,言其要塞盜賊之路也。崤山險阨,是晉之要道關塞也。從秦向鄭,路經晉之南境,於南河之南崤關而東適鄭。\CJKunderwave{禮}征伐朝聘,過人之國,必遣使假道。晉以秦不假道,故伐之。\CJKunderwave{左傳}僖三十二年,晉文公卒。三十三年,秦師及滑,鄭商人弦高將市於周,遇之,矯鄭伯之命以牛十二犒師。孟明曰:“鄭有備矣,不可冀也。攻之不克,圍之不繼,吾其還也。”滅滑而還。晉先軫請伐秦師。襄公在喪,墨縗絰。夏四月,敗秦師於殽,獲百里孟明視、西乞術、白乙丙以歸。是襄公親自帥師伐而敗之,囚其三帥也。\CJKunderwave{春秋}之例,君將不言“帥師”,舉其重者。此言“襄公帥師”,依實為文,非彼例也。又\CJKunderwave{春秋}經書此事云:“晉人及羑戎敗秦師於殽。”實是晉侯而書“晉人”者,杜預云:“晉侯諱背喪用兵,通以賤者告也。”是言晉人告魯,不言晉侯親行,而云大夫將兵。大夫賤,不合書名氏,故稱“人”也。直言敗秦師於殽,不言秦之將帥之名,亦諱背喪用兵,故言辭略也。 \par}

{\noindent\zhuan\zihao{6}\fzbyks 傳“晉舍”至“作誓”。正義曰:\CJKunderwave{左傳}又稱,晉文公之夫人文嬴,秦女也,請三帥曰:“彼實構吾二君,寡君若得而食之,不厭,君何辱討焉?使歸就戮於秦,以逞寡君之志,若何?”公許之。秦伯素服郊次,向師而哭曰:“孤違蹇叔,以辱二三子,孤之罪也。不替孟明,孤之過也。”是晉舍三帥而得還,\CJKunderline{秦穆公}於是悔過作誓。序言“還歸”,謂三帥還也,嫌穆公身還,故辨之。\CJKunderwave{公羊傳}說此事云:“四馬只輪無反者。”\CJKunderwave{左傳}稱秦伯“向師而哭”,則師亦少有還者, \par}

{\noindent\shu\zihao{5}\fzkt “秦穆”至“秦誓”。正義曰:\CJKunderline{秦穆公}使孟明視、西乞術、白乙丙三帥帥師伐鄭,未至鄭而還。\CJKunderline{晉襄公}帥師敗之於崤山,囚其三帥。後晉舍三帥,得還歸於秦。\CJKunderline{秦穆公}自悔己過,誓戒群臣。史錄其誓辭,作\CJKunderwave{秦誓}。 \par}

公曰:“嗟!我士,聽無譁。\footnote{誓其群臣,通稱士也。}予誓告汝群言之首。\footnote{總言之本要。}古人有言曰:‘民訖自若,是多盤。’\footnote{言民之行己,盡用順道,是多樂。稱古人言,悔前不順忠臣。○樂音洛。}責人斯無難,惟受責俾如流,是惟艱哉!\footnote{人之有非,以義責之,此無難也。若己有非,惟受人責,即改之如水流下,是惟艱哉。○俾,必爾反,下同。}我心之憂,日月逾邁,若弗云來。\footnote{言我心之憂,欲改過自新,如日月並行過,如不復云來,雖欲改悔,恐死及之,無所益。○復,扶又反。}


{\noindent\zhuan\zihao{6}\fzbyks 傳“誓其”至“稱士”。正義曰:“士”者,男子之大號,故群臣通稱之。鄭云:“誓其群臣,下及萬民,獨雲士者,舉中言之。” \par}

{\noindent\zhuan\zihao{6}\fzbyks 傳“言民”至“忠臣”。正義曰:“訖”,盡也。“自”,用。“若”,順。“盤”,樂也。盡用順道則有福,有福則身樂,故云“是多樂”也。“稱古人言”者,悔前不用古人之言,不順忠臣之謀故也。昔漢明帝問東\CJKunderline{平王}劉蒼云:“在家何者為樂?”對曰:“為善最樂。”是其用順道則多樂。 \par}

{\noindent\zhuan\zihao{6}\fzbyks 傳“言我”至“所益”。正義曰:“逾”,益。“邁”,行也。“員”即“雲”也。言日月益為疾行,並皆過去,如似不復雲來。畏其去而不復來,夜而不復明,言己年老,前途稍近,雖欲改悔,恐死及之,不得修改,身無所益也。\CJKunderline{王肅}云:“年已衰老,恐命將終,日月遂往,若不雲來,將不復見日月,雖欲改過,無所及益。自恨改過遲晚,深自咎責之辭。” \par}

{\noindent\shu\zihao{5}\fzkt “公曰”至“雲來”。正義曰:穆公自悔伐鄭,召集群臣而告之。公曰:“諮嗟!我之朝廷之士,聽我告於汝,無得喧譁。我誓告汝眾言之首,誥汝以言中之最要者。古人有言曰:‘民之行己,盡用順道。是多樂。’言順善事,則身大樂也。見他有非理,以義責之,此無難也。惟己有非理,受人之責,即能改之,使如水之流下,此事是惟難哉!”言己已往之前不受人言,故自悔也。“今我心憂,欲自改過自新,但日月益為疾行,如似不復雲來,恐己老死不得改悔也”。 \par}

惟古之謀人,則曰未就予忌。\footnote{惟為我執古義之謀人,謂忠賢蹇叔等也,則曰未成我所欲,反忌之耳。○為,於偽反,下“為我謀”同。}惟今之謀人,姑將以為親。\footnote{惟指今事為我所謀之人,我且將以為親而用之。悔前違古從今,以取破敗。}

{\noindent\shu\zihao{5}\fzkt “惟古”至“為親”。正義曰:此穆公自說己之前過。我欲伐鄭之時,群臣共為謀計,惟為我執古義之謀人,我則曰未成我之所欲,反猜忌之。惟指今事為我所謀之人,我且將以為親己而用之。悔前違古從今,自取破敗也。其“古之謀人”,當謂忠賢之臣若蹇叔之等。“今之謀人”,勸穆公使伐鄭者,蓋謂杞子之類,國內亦當有此人。 \par}

“雖則云然,尚猷詢茲黃髮,則罔所愆。\footnote{言前雖則有云然之過,今我庶幾以道謀此黃髮賢老,則行事無所過矣。}\xpinyin{番}{po2}番良士,旅力既愆,我尚有之。\footnote{勇武番番之良士,雖眾力已過老,我今庶幾欲有此人而用之。○番音波。}仡仡勇夫,射御不違,我尚不欲。\footnote{仡仡壯勇之夫,雖射御不違,我庶幾不欲用。自悔之至。○仡,許乞反。}


{\noindent\shu\zihao{5}\fzkt “雖則”至“不欲”。正義曰:言我前事雖則有云然之過,我今庶幾以道謀此黃髮賢老,受用其言,則行事無所過也。番番然勇武之善士,雖眾力既過老,而謀計深長,我庶幾欲有此人而用之。仡仡然壯勇之夫,雖射御不有違失,而智慮淺近,我庶幾不欲用之。自悔往前用壯勇之計失也。 \par}

惟截截善\xpinyin{諞}{pian2}言,俾君子易辭,我皇多有之。昧昧我思之,\footnote{惟察察便巧善為辨佞之言,使君子迴心易辭,我前多有之,以我昧昧思之不明故也。○截,才節反。馬云:“辭語截削省要也。”諞音辨,徐敷連反,又甫淺反,馬本作偏,云:“少也,辭約損明,大辨佞之人。”易,羊石反。昧音妹。}如有一介臣,斷斷猗,無他伎,其心休休焉,其如有容。\footnote{如有束脩一介臣,斷斷猗然專一之臣,雖無他伎藝,其心休休焉樂善,其如是,則能有所容。言將任之。○介音界,馬本作界,云:“一介,耿介,一心端愨者。”字又作個,音工佐反。斷,丁亂反,又音短。猗,於綺反,又於宜反。技,其綺反,本亦作伎。他,本亦作它,吐何反。樂音洛。}


{\noindent\zhuan\zihao{6}\fzbyks 傳“惟察”至“故也”。正義曰:“截截”猶“察察”,明辯便巧之意。“諞”猶辯也,由其便巧善為辯佞之言,使君子聽之迴心易辭。“皇”訓大也,我前大多有之,謂杞子之等,及在國從己之人。以我昧昧而暗,思之不明,故有此輩在我側也。 \par}

{\noindent\zhuan\zihao{6}\fzbyks 傳“如有”至“任之”。正義曰:孔注\CJKunderwave{論語},以“束脩”為“束帶修飾”,此亦當然。“一介”謂一心耿介。“斷斷”,守善之貌。“休休”,好善之意。如有束帶修飾,一心耿介,斷斷然守善猗然專一之臣,雖復無他技藝,休休焉好樂善道,其心行如是,則能有所含容。言得此人將任用之。“猗”者,足句之辭,不為義也。\CJKunderwave{禮記·太學}引此作“斷斷兮”,“猗”是“兮”之類,\CJKunderwave{詩}云“河水清且漣漪”是也。\CJKunderline{王肅}云:“一介,耿介,一心端愨,斷斷守善之貌。無他技能,徒守善而已。休休,好善之貌。其如是,人能有所容忍小過,寬則得眾。穆公疾技巧多端,故思斷斷無他技者。” \par}

{\noindent\shu\zihao{5}\fzkt “惟截截”至“有容”。正義曰:惟察察然便巧善為辯佞之言,能使君子迴心易辭。我前大多有之,昧昧然我思之不明故也。如有一心耿介之臣,斷斷守善猗然,雖無他技藝,而其心樂善休休焉,其如是,則能有所含容。如此者,我將任用之。悔前用巧佞之人,今將任寬容善士也。 \par}

人之有技,若己有之。人之彥聖,其心好之,不啻若自其口出,是能容之。\footnote{人之有技,若己有之樂,善之至也。人之美聖,其心好之,不啻如自其口出,心好之至也。是人必能容之。○好,呼報反。啻,失豉反。}以保我子孫黎民,亦職有利哉!\footnote{用此好技聖之人,安我子孫眾人,亦主有利哉!言能興國。}

{\noindent\shu\zihao{5}\fzkt “人之”至“利哉”。正義曰:此說大賢之行也。大賢之人,見人之有技,如似己自有之。見人之有美善通聖者,其心愛好之,不啻如自其口出。愛彼美聖,口必稱揚而薦達之,其心愛之,又甚於口,言其愛之至也。是人於民必能含容之。用此愛好技聖之人,安我子孫眾民,則我子孫眾民亦主有利益哉!言其能興邦也。 \par}

人之有技,冒疾以惡之。人之彥聖,而違之,俾不達。\footnote{見人之有技藝,蔽冒疾害以惡之。人之美聖,而違背壅塞之,使不得上通。○冒,莫報反。惡,烏路反。背音佩。壅,於勇反。塞,先得反。}是不能容,以不能保我子孫黎民,亦曰殆哉!\footnote{冒疾之人,是不能容人用之,不能安我子孫眾人,亦曰危殆哉!○殆,唐在反。}


{\noindent\zhuan\zihao{6}\fzbyks 傳“見人”至“上通”。正義曰:傳以“冒”為覆冒之“冒”,謂蔽障掩蓋之也。“疾”謂疾惡之,謂憎疾患害之也。見人之美善通聖而違背之,不從其言;壅塞之,使不得上通,皆是佞人害賢之行也。 \par}

{\noindent\shu\zihao{5}\fzkt “人之”至“殆哉”。正義曰:此說大佞之行也。大佞之人,見人之有技,蔽冒疾害以惡之。見人之有美善通聖者,而違背壅塞之,使不達於在上。是人之不能含容人也。用此疾惡技聖之人,不能安我子孫眾民,則我子孫眾民亦曰危殆哉!言其必亂邦也。 \par}

邦之杌隉,曰由一人。\footnote{杌隉,不安,言危也。一人所任用,國之傾危,曰由所任不用賢。○杌,五骨反。隉,五結反,徐語折反。}邦之榮懷,亦尚一人之慶。”\footnote{國之光榮,為民所歸,亦庶幾其所任用賢之善也。穆公陳戒,背賢則危,用賢則榮,自誓改前過之意。}

{\noindent\shu\zihao{5}\fzkt “邦之”至“之慶”。正義曰:既言賢佞行異,又言用之安否。邦之杌隉,危而不安,曰由所任一人之不賢也;邦之光榮,為民所歸,亦庶幾所任一人之有慶也。言國家用賢則榮,背賢則危,穆公自誓將改前過,用賢人者也。人者也。 \par}

%%% Local Variables:
%%% mode: latex
%%% TeX-engine: xetex
%%% TeX-master: "../Main"
%%% End:
