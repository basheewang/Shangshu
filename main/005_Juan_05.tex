%% -*- coding: utf-8 -*-
%% Time-stamp: <Chen Wang: 2024-04-02 11:42:41>

% {\noindent \zhu \zihao{5} \fzbyks } -> 注 (△ ○)
% {\noindent \shu \zihao{5} \fzkt } -> 疏

\chapter{卷五}


\section{益稷第五(皋陶謨中、下)}


益稷\footnote{禹稱其人,因以名篇。}

{\noindent\zhuan\zihao{6}\fzbyks 傳“\CJKunderline{禹}稱”至“名篇”。正義曰:\CJKunderline{禹}言“暨益”、“暨稷”,是\CJKunderline{禹}稱其二人,二人佐\CJKunderline{禹}有功,因以此二人名篇。既美\CJKunderline{大禹},亦所以彰此二人之功也。\CJKunderline{禹}先言“暨益”,故“益”在“稷”上。馬、鄭、王所據\CJKunderwave{書序}此篇名為\CJKunderwave{棄稷}。“棄”“稷”一人,不宜言名又言官,是彼誤耳。又合此篇於\CJKunderwave{皋陶謀},謂其別有\CJKunderwave{棄稷}之篇,皆由不見古文,妄為說耳。 \par}

帝曰:“來,\CJKunderline{禹},汝亦昌言。”\footnote{因皋陶謨九德,故呼\CJKunderline{禹}使亦陳當言。當,丁浪反,本亦作讜,當湯反。李登\CJKunderwave{聲類}云:“讜言,善言也。”}\CJKunderline{禹}拜曰:“都!帝,予何言?予思日孜孜。”\footnote{拜而嘆,辭不言,欲使帝重\CJKunderline{皋陶}所陳。言己思日孜孜不怠,奉承臣功而已。思,徐如字,又息吏反。孜音茲。}\CJKunderline{皋陶}曰:“吁!如何?”\footnote{問所以孜孜之事。}\CJKunderline{禹}曰:“洪水滔天,浩浩懷山襄陵,下民昏墊。\footnote{言天下民昏瞀墊溺,皆困水災。○浩,戶老反。墊,丁念反。瞀音務,一音茂,本或作務。溺,乃歷反。}


{\noindent\zhuan\zihao{6}\fzbyks 傳“因皋”至“當言”。正義曰:上篇\CJKunderline{皋陶}謀九德,此帝呼\CJKunderline{禹},令亦陳當言。“亦”者,亦\CJKunderline{皋陶}也。明上篇\CJKunderline{皋陶}雖與益相應,其言亦對帝也。上傳雲“\CJKunderline{皋陶}為\CJKunderline{帝舜}謀”者,以此而知也。 \par}

{\noindent\zhuan\zihao{6}\fzbyks 傳“拜而”至“而已”。正義曰:既已拜而嘆,必有所美,復辭而不言,是知欲使帝重\CJKunderline{皋陶}所陳,言己無以加也。王肅雲“帝在上,\CJKunderline{皋陶}陳謀於下,已備矣,我復何所言乎”是也。既無所言,故言已思惟日孜孜不敢怠惰,奉成臣職而已。孜孜者,勉功不怠之意。 \par}

{\noindent\zhuan\zihao{6}\fzbyks 傳“言天下”至“水災”。正義曰:“瞀”者眩惑之意,故言“昏瞀”。“墊”是下溼之名,故為溺也。言天下之人遭此大水,精神昏瞀迷惑,無有所知,又苦沉溺,皆困此水災也。鄭云:“昏,沒也。墊,陷也。\CJKunderline{禹}言洪水之時,人有沒陷之害。” \par}

予乘四載,隨山刊木,\footnote{所載者四,謂水乘舟,陸乘車,泥乘輴,山乘樏。隨行九州之山林,刊槎其木,開通道路以治水也。○乘音繩。刊,苦安反。輴,醜倫反。\CJKunderwave{漢書}作橇,如淳音蕝,以板置泥上。服虔云:“木橇,形如木箕,擿行泥上。”\CJKunderwave{屍子}云:“澤行乘蕝。”蕝音子絕反,樏,力追反。\CJKunderwave{史記}作橋,徐音丘遙反。\CJKunderwave{漢書}作梮,九足反。行,下孟反。槎,士雅反,下同。\CJKunderwave{說文}云:“袤斫。”又莊下反。}暨\CJKunderline{益}奏庶鮮食。\footnote{奏謂進於民。鳥獸新殺曰鮮。與益槎木,獲鳥獸,民以進食。○塈,其器反。鮮,徐音仙,馬云:“鮮,生也。”}予決九川,距四海,濬\xpinyin*{畎}\xpinyin*{澮}距川。\footnote{距,至也。決九州名川,通之至海。一畎之間,廣尺、深尺曰畝。方百里之間,廣二尋、深二仞曰澮。澮畎深之至川,亦入海。○畎,工犬反。澮,故外反。廣,光浪反。深,屍鴆反。}


{\noindent\zhuan\zihao{6}\fzbyks 傳“所載”至“治水”。正義曰:\CJKunderwave{史記·河渠書}云:“\CJKunderwave{夏書}曰:‘\CJKunderline{禹}湮洪水十三年,三過家不入門。陸行載車,水行載舟,泥行蹈橇。橇音蕝,山行即橋。橋,丘遙反。’”徐廣曰:“橋一作輂。輂,幾玉反,直轅車也。”\CJKunderwave{屍子}云:“山行乘樏,泥行乘蕝。蕝,子絕反。”\CJKunderwave{漢書·溝洫志}云:“泥行乘毳,山行則梮。梮,居足反。”毳行如箕,擿行泥上。如淳云:“毳謂以板置泥上,以通行路也。”\CJKunderwave{慎子}云:“為毳者,患塗之泥也。”應劭云:“梮或作樏,為人所牽引也。”如淳云:“梮謂以鐵如錐,頭長半寸,施之履下以上山,不蹉跌也。”韋昭云:“梮,木器也。如今輿床,人輿以行也。”此經惟言“四載”,傳言所載者四,同彼\CJKunderwave{史記}之說。古書屍子、慎子之徒有此言也。輴與毳為一,樏與梮、輿為一。古篆變形,字體改易,說者不同,未知孰是。\CJKunderline{禹}之施功,本為治水,此經乃雲“隨山刊木”,刊木為治水,治水遍於九州,故云“隨行九州之山林”。襄二十五年\CJKunderwave{左傳}云:“井堙木刊。”“刊”是除木之義也。\CJKunderwave{毛傳}云:“除木曰槎。”故曰“刊槎其木,開通道路以治水”。\par}

{\noindent\zhuan\zihao{6}\fzbyks 傳“奏謂”至“進食”。正義曰:黎民阻飢,為人治水,故知“奏謂進食於人”也。\CJKunderwave{禮}有鮮魚臘,以其新殺鮮淨,故名為鮮,是鳥獸新殺曰鮮,魚鱉新殺亦曰鮮也。此承“山”下,故為鳥獸,下承水後,故為魚鱉,其新殺之意同也。既言刊木乃進鮮食,食是除木所得,故言“與益槎木,獲鳥獸,人以進食”。 \par}

{\noindent\zhuan\zihao{6}\fzbyks 傳“距至”至“入海”。正義曰:“距”者相抵之名,故為至也。非是名川不能至海,故“決九州之名川通之至海”也。\CJKunderwave{考工記}云:“匠人為溝洫,耜廣五寸,二耜為耦。一耦之伐,廣尺、深尺謂之畎。田首倍之,廣二尺、深二尺謂之遂。九夫為井,井間廣四尺、深四尺謂之溝。方十里為成,成間廣八尺、深八尺謂之洫。方百里為同,同間廣二尋、深二仞謂之澮。”是畎、遂、溝、洫、澮皆通水之道也。以小注大,故從畎、遂、溝、洫乃以入澮,澮入於川,川入於海,是畎內之水亦入海也。惟言“畎、澮”,舉大小而略其餘也。先言決川至海,後言濬畎至川者,川既入海,然後澮得入川,故先言川也。 \par}

暨\CJKunderline{稷}播,奏庶艱食鮮食。\footnote{艱,難也。眾難得食處,則與稷教民播種之,決川有魚鱉,使民鮮食之。○艱,工間反,馬本作根,云:“根生之食,謂百穀。”處,昌慮反。鱉,必滅反。}懋遷有無化居。\footnote{化,易也。居謂所宜居積者。勉勸天下,徙有之無,魚鹽徙山,林木徙川澤,交易其所居積。懋音茂。鹽,餘廉反。}烝民乃粒,萬邦作乂。”\footnote{米食曰粒。言天下由此為治本。○烝,之承反。粒音立。治,直吏反,下同。}\CJKunderline{皋陶}曰:“俞!師汝昌言。”\footnote{言\CJKunderline{禹}功甚當,可師法。○當,丁浪反。}

{\noindent\zhuan\zihao{6}\fzbyks 傳“艱難”至“鮮食之”。正義曰:“艱,難也”,\CJKunderwave{釋詁}文。\CJKunderline{禹}主治水,稷主教播種,水害漸除,則有可耕之地,難得食處,先須教導以救之,故云“眾難得食處,則與稷教人播種之”。易得食處,人必自能得之,意在救人艱危之厄,故舉難得食處以言之。於時雖漸播種,得谷猶少,人食未足,故決川有魚鱉,使人鮮食之。言食魚以助谷也。\CJKunderline{鄭玄}云:“與稷教人種澤物菜蔬艱厄之食。”稷功在於種穀,不主種菜蔬也。言后稷種菜蔬艱厄之食,傳記未有此言也。 \par}

{\noindent\zhuan\zihao{6}\fzbyks 傳“化易”至“居積”。正義曰:變化是改易之義,故“化”為易也。“居謂所宜居積者”,近水者居魚鹽,近山者居林木也。“勉勸天下徙有之無”者,謂徙我所有,往彼無鄉;取彼所有,以濟我之所無。“魚鹽徙山,林木徙川澤,交易其所宜居積”,言此“遷”者,謂將物去,不得空取彼物也。王肅云:“易居者不得空去,當滿而去,當滿而來也。” \par}

{\noindent\zhuan\zihao{6}\fzbyks 傳“米食”至“治本”。正義曰:\CJKunderwave{說文}云:“粒,糂也。”今人謂飯為米糂,遺餘之飯謂之一粒、兩粒,是米食曰粒,言是用米為食之名也。人非谷不生,政由谷而就,言天下由此谷為治政之本也。君子之道以謙虛為德,\CJKunderline{禹}盛言己功者,為臣之法當孜孜不怠,自言己之勤苦,所以勉勸人臣,非自伐也。 \par}

{\noindent\shu\zihao{5}\fzkt “帝曰來”至“汝昌言”。正義曰:\CJKunderline{皋陶}既為帝謀,帝又呼\CJKunderline{禹}進之,曰:“來,\CJKunderline{禹},汝亦宜陳其當言。”\CJKunderline{禹}拜曰:“嗚乎!帝,\CJKunderline{皋陶}之言既已美矣,我更何所言?我之所思者,每日孜孜勤於臣職而已。”\CJKunderline{皋陶}怪\CJKunderline{禹}不言,故謂之曰:“吁!”問其所以孜孜之事如何。\CJKunderline{禹}曰:“往者洪水漫天,浩浩然盛大,包山上陵,下民昏惑沉溺,皆困水災。我乘舟車輴樏等四種之載,隨其所往之山,槎木通道而治之。與益所進於人者,惟有槎木所獲眾鳥獸鮮肉為食也。我又通決九州名川,通之至於四海。深其畎澮,以至於川,水漸除矣。與稷播種五穀,進於眾人難得食處,乃決水所得魚鱉鮮肉為食也,人既皆得食矣。又勸勉天下徙有之無,交易其所居積。於是天下眾人乃皆得米粒之食,萬國由此為治理之政。我所言孜孜者在此也。”\CJKunderline{皋陶}曰:“然。可以為師法者,是汝之當言。” \par}

\CJKunderline{禹}曰:“都!帝,慎乃在位。”帝曰:“俞。”\footnote{然\CJKunderline{禹}言,受其戒。}\CJKunderline{禹}曰:“安汝止,惟幾惟康,其弼直,\footnote{言慎在位,當先安好惡所止,念慮幾微,以保其安,其輔臣必用直人。好惡,上呼報反,下烏路反,又並如字。}惟動丕應徯志。\footnote{徯,待也。帝先安所止,動則天下大應之,順命以待帝志。徯志当断为下句首,猶待命。應,應對之應。徯,胡啟反。}


{\noindent\zhuan\zihao{6}\fzbyks 傳“言慎”至“直人”。正義曰:此\CJKunderline{禹}重戒帝,覆上“慎乃在位”。“當先安好惡所止”,謂心之所止,當止好不止惡,言惡以刑好也。\CJKunderwave{大學}云:“為人君止於仁,為人臣止於敬。”“好惡所止”謂此類也。傳意以上“惟”為念,下“惟”為辭,故云念慮幾微,然後以保其好惡所安寧耳。 \par}

{\noindent\zhuan\zihao{6}\fzbyks 傳“徯待”至“帝志”。正義曰:“徯,待”,\CJKunderwave{釋詁}文。帝先能自安所止,心之所止,止於好事,其有舉動,發號出令,則天下大應之,順命以待帝志。謂靜以待命,有命則從也。 \par}

徯志以昭受上帝,天其申命用休。”\footnote{昭,明也。非但人應之,又乃明受天之報施,天又重命用美。施,始豉反。重,直用反。}帝曰:“吁!臣哉鄰哉!鄰哉臣哉!”\CJKunderline{禹}曰:“俞。”\footnote{鄰,近也。言君臣道近,相須而成。}

{\noindent\zhuan\zihao{6}\fzbyks 傳“昭明”至“用美”。正義曰:\CJKunderwave{堯典}已訓“昭”為明,此重訓,詳之。皇天無親,惟德是輔,人之所欲,天必從之。帝若能安所止,非但人歸之,又乃明受天之報施。天下太平,祚胤長遠,是天之報施也。“天又重命用美”,謂四時和祥瑞臻之類也。或當前後非一,故傳言“又”也。 \par}

{\noindent\zhuan\zihao{6}\fzbyks 傳“鄰近”至“而成”。正義曰:\CJKunderwave{周禮}“五家為鄰”,取相近之義,故“鄰”為近也。\CJKunderline{禹}言君當好善,帝言須得臣力,再言鄰哉,言君臣之道當相須而成,\CJKunderline{鄭玄}云:“臣哉,汝當為我鄰哉!鄰哉,汝當為我臣哉!反覆言此,欲其志心入\CJKunderline{禹}。” \par}

{\noindent\shu\zihao{5}\fzkt “\CJKunderline{禹}曰都”至“曰俞”。正義曰:\CJKunderline{禹}以\CJKunderline{皋陶}然已,因嘆而戒帝曰:“嗚呼!帝當謹慎汝所在之位。”帝受其戒,曰:“然。”\CJKunderline{禹}又戒帝曰:“若欲慎汝在位,當須先安定汝心好惡所止,念慮事之微細,以保安其身,其輔弼之臣必用正直之人。若能如此,惟帝所動,則天下大應之,以待帝志。以明受天之佈施,於天其重命帝用美道也。”帝以\CJKunderline{禹}言已重,乃驚而言曰:“吁!臣哉近哉,臣當親近君也!近哉臣哉,君當親近臣也!”言君臣當相親近,共與成政道也。\CJKunderline{禹}應帝曰:“然。”言君臣宜相親近也。 \par}

帝曰:“臣作朕股肱耳目。\footnote{言大體若身。股音古。肱,古弘反。}予欲左右有民,汝翼。\footnote{左右,助也。助我所有之民,富而教之,汝翼成我。}予欲宣力四方,汝為。\footnote{布力立治之功,汝群臣當為之。}予欲觀古人之象,\footnote{欲觀示法象之服制。觀,舊音官,又官喚反。}


{\noindent\zhuan\zihao{6}\fzbyks 傳“言大體若身”。正義曰:君為元首,臣為股肱耳目,大體如一身也。足行手取,耳聽目視,身雖百體,四者為大,故舉以為言。\CJKunderline{鄭玄}云:“動作視聽皆由臣也。” \par}

{\noindent\zhuan\zihao{6}\fzbyks 傳“左右”至“成我”。正義曰:\CJKunderwave{釋詁}云:“左、右、助,慮也”,同訓為慮,是“左右”得為助也。立君所以牧人,人之自營生產,人君當助救之。\CJKunderwave{論語}稱\CJKunderline{孔子}適衛,欲先富民而後教之,故云“助我所有之民,欲富而教之”也。君子施教,本為養人,故先雲助人,舉其重者。以其為人事重,當須翼成,故言“汝翼”。次顯君施教化,須臣為之,故言“汝為”。次明衣服上下,標顯尊卑,故云“汝明”。次雲六律、五聲,故云“汝聽”。各隨事立文,其實不異。 \par}

{\noindent\zhuan\zihao{6}\fzbyks 傳“布力”至“為之”。正義曰:\CJKunderwave{詩}雲“四方於宣”,\CJKunderwave{論語}雲“陳力就列”,是佈政用力,故言“布力立治之功,汝群臣當為之”。 \par}

{\noindent\zhuan\zihao{6}\fzbyks 傳“欲觀”至“服制”。正義曰:“觀示法象之服制”者,謂欲申明古人法象之衣服,垂示在下使觀之也。\CJKunderwave{易·繫辭}云:“黃帝、堯、舜垂衣裳而天下治。”象物制服,蓋因黃帝以還,未知何代而具彩章。舜言己欲觀古,知在舜之前耳。 \par}

日、月、星辰、山、龍、華蟲,\footnote{日月星為三辰。華象草華蟲雉也。畫三辰、山、龍、華蟲於衣服旌旗。蟲,直弓反。}作會,宗彝,\footnote{會,五采也,以五采成此畫焉。宗廟彝樽亦以山、龍、華蟲為飾。會,胡對反,馬、鄭作繪。彝音夷,馬同,鄭雲“宗彝,虎也。”}


{\noindent\zhuan\zihao{6}\fzbyks 傳“日月”至“旌旗”。正義曰:桓二年\CJKunderwave{左傳}云:“三辰旗旗,昭其明也。”三辰謂此日月星也,故“日月星為三辰”。辰即時也,三者皆是示人時節,故並稱辰焉。傳言此者,以“辰”在“星”下,總上三事為辰,辰非別為物也。\CJKunderwave{周禮·大宗伯}云:“實柴祀日月星辰。”\CJKunderline{鄭玄}云:“星謂五緯也,辰謂日月所會十二次也。”“星”、“辰”異者,彼鄭以遍祭天之諸神十二次也,次亦當祭之,故令“辰”與“星”別。此雲畫之於衣,日月合宿之辰,非有形容可畫,且\CJKunderwave{左傳}雲三辰即日月星也。\CJKunderwave{周禮}“司常掌九旗之物”,惟日月為常,不言畫星,蓋太常之上又畫星也。\CJKunderwave{穆天子傳}稱天子葬盛姬,畫日月七星,蓋畫北斗也。草木雖皆有華,而草華為美,故云“華象草華蟲雉”也。\CJKunderwave{周禮·司服}有“鷩冕”,鷩則雉焉,雉五色,象草華也。\CJKunderwave{月令}五時皆雲其蟲,“蟲”是鳥獸之總名也。下雲“作服,汝明”,知“畫三辰、山、龍、華蟲於衣服”也。又言“旌旗”者,\CJKunderwave{左傳}言“三辰旗旗”,\CJKunderwave{周禮·司常}雲“日月為常”,王者禮有沿革,後因於前,故知舜時三辰亦畫之於旌旗也。下傳雲“天子服日月而下”,則三辰畫之於衣服,又畫於旌旗也。\CJKunderwave{周禮·司服}云:“享先王則袞冕。”袞者,卷也,言龍首卷然。以袞為名,則所畫自龍已下,無日月星也。\CJKunderwave{郊特牲}云:“祭之日,王被袞冕以象天也。”又曰:“龍章而設日月,以象天也。”\CJKunderline{鄭玄}云:“謂有日月星辰之章”,“設日月畫於衣服旌旗也”。據此記文,袞冕之服亦畫日月。鄭注\CJKunderwave{禮記}言\CJKunderwave{郊特牲}所云“謂魯禮也”。要其文稱王被服袞冕,非魯事也。或當二代天子衣上亦畫三辰,自龍章為首,而使袞統名耳。\CJKunderwave{禮}文殘缺,不可得詳,但如孔解,舜時天子之衣畫日月耳。\CJKunderline{鄭玄}亦以為然。王肅以為“舜時三辰即畫於旌旗,不在衣也,天子山、龍、華蟲耳。 \par}

{\noindent\zhuan\zihao{6}\fzbyks 傳“會五”至“為飾”。正義曰:“會”者合聚之名,下雲“以五彩彰施於五色,作服”,知“會”謂五色也。\CJKunderwave{禮}衣畫而裳繡,“五色備謂之繡”,知畫亦備五色,故云“以五彩成此畫焉”,謂畫之於衣、宗彝。文承“作會”之下,故云“宗廟、彝樽亦以山、龍、華蟲為飾”。知不以日月星為飾者,孔以三辰之尊不宜施於器物也。\CJKunderwave{周禮}有山罍、龍勺、雞彝、鳥彝,以類言之,知彝樽以山、龍、華蟲為飾,亦畫之以為飾也。\CJKunderwave{周禮}彝器所云犧、象、雞、鳥者,\CJKunderline{鄭玄}皆為畫飾,與孔意同也。\CJKunderwave{周禮}彝器無山、龍、華蟲為飾者,帝王革易,所尚不同,故有異也。 \par}

藻、火、粉、米、黼、黻、絺、繡,\footnote{藻,水草有文者。火為火字,粉若粟冰,米若聚米,黼若斧形,黻為兩己相背,葛之精者曰絺,五色備曰繡。藻音早,本又作薻。粉米,\CJKunderwave{說文}作黺𪓋,徐米作{\hanaa 䋛},音米。黼音甫,白與黑謂之黼。黻音弗,黑與青謂之黻。絺,徐敕私反,又敕其反,馬同,鄭陟裡反,刺也。繡音秀。背音佩。}以五采彰施於五色,作服,汝明。\footnote{天子服日月而下,諸侯自龍袞而下至黼黻,士服藻火,大夫加粉米。上得兼下,下不得僣上。以五采明施於五色,作尊卑之服,汝明制之。袞,工本反。僣,子念反。}


{\noindent\zhuan\zihao{6}\fzbyks 傳“藻水”至“曰繡”。正義曰:\CJKunderwave{詩}雲“魚在在藻”,是“藻”為水草。草類多矣,獨取此草者,謂此草有文故也。“火為火字”,謂刺繡為“火”字也。\CJKunderwave{考工記}云:“火以圜。”鄭司農云:“謂圜形似火也。\CJKunderline{鄭玄}云:“形如半環。”然\CJKunderwave{記}是後人所作,何必能得其真?今之服章繡為“火”字者,如孔所說也。“粉若粟冰”者,粉之在粟,其狀如冰。“米若聚米”者,刺繡為文,類聚米形也。“黼若斧形”,\CJKunderwave{考工記}云:“白與黑謂之黼。”\CJKunderwave{釋器}云:“斧謂之黼。”孫炎云:“黼文如斧形”,蓋半白半黑,似斧刃白而身黑。“黻謂兩己相背”,謂刺繡為“己”字,兩“己”字相背也。\CJKunderwave{考工記}云:“黑與青謂之黻。”刺繡為兩\CJKunderwave{己}字,以青黑線繡也。\CJKunderwave{詩·葛覃}雲“為絺為綌”,是絺用葛也。\CJKunderwave{玉藻}云:“浴用二巾,上絺下綌。”\CJKunderwave{曲禮}云:“為天子削瓜者副之,巾以絺。為國君者華之,巾以綌。”皆以絺貴而綌賤,是絺精而綌粗,故“葛之精者曰絺”。“五色備謂之繡”,\CJKunderwave{考工記}文也。計此所陳,皆述祭服。祭服玄纁為之,後代無用絺者,蓋於時仍質,暑月染絺為纁而繡之以為祭服。孔以“華象草華蟲雉”,則合華蟲為一,\CJKunderwave{周禮}\CJKunderline{鄭玄}注亦然,則以日、月、星辰、山、龍、華蟲六章畫於衣也。藻、火、粉、米、黼、黻六章繡於裳也。天之大數不過十二,故王者製作皆以十二象天也。顧氏取先儒等說,以為“日月星取其照臨,山取能興雲雨,龍取變化無方,華取文章,雉取耿介”。顧氏雖以唬骸藹蟲為二,其取象則同。又云:“藻取有文,火取炎上,粉取絜白,米取能養,黼取能斷,黻取善惡相背。”\CJKunderline{鄭玄}云:“會讀為繪。宗彝謂宗廟之鬱鬯樽也。故虞夏以上,蓋取虎彝蜼彝而已。粉米,白米也。絺讀為黹。黹,紩也。自日月至黼黻凡十二章,天子以飾祭服。凡畫者為繪,刺者為繡。此繡與繪各有六,衣用繪,裳用繡。至周而變之,以三辰為旗旗,謂龍為袞,宗彝為毳,或損益上下,更其等差。”鄭意以“華蟲”為一,“粉米”為一,加“宗彝”謂虎蜼也。\CJKunderwave{周禮}宗廟彝器有虎彝、蜼彝,故以“宗彝”為虎蜼也。此經所云凡十二章,日也,月也,星也,山也,龍也,華蟲也,六者畫以作繪,施於衣也;宗彝也,藻也,火也,粉米也,黼也,黻也,此六者紩以為繡,施之於裳也。\CJKunderline{鄭玄}雲“至周而變易之,損益上下,更其等差”,\CJKunderwave{周禮·司服}之注具引此文,乃云:“此古天子冕服十二章也。王者相變,至周而以日月星畫於旌旗。冕服九章,登龍于山,登火於宗彝,尊其神明也。九章,初一曰龍,次二曰山,次三曰華蟲,次四曰火,次五曰宗彝,皆畫以為繢;次六曰藻,次七曰粉米,次八曰黼,次九曰黻,以絺為繡。則袞之衣五章,裳四章,凡九也。鷩畫以雉,謂華蟲也。其衣三章,裳四章,凡七也。毳畫虎蜼,謂宗彝也。其衣三章,裳二章,凡五也。”是鄭以冕服之名皆取章首為義,袞冕九章,以龍為首,龍首卷然,故以袞為名。鷩冕七章,華蟲為首,華蟲即鷩雉也。毳冕五章,虎蜼為首,虎蜼毛淺,毳是亂毛,故以毳為名。如鄭此解,配文甚便,於絺繡之義,總為消帖。但解“宗彝”為虎蜼,取理太回,未知所說誰得經旨。 \par}

{\noindent\zhuan\zihao{6}\fzbyks 傳“天子”至“制之”。正義曰:此言“作服,汝明”,故傳辨其等差。天子服日月而下十二章,諸侯自龍袞而下至黼黻八章,再言“而下”,明天子諸侯皆至黼黻也。士服藻火二章,大夫加粉米四章。孔註上篇“五服”,謂“天子、諸侯、卿、大夫、士”,則卿與大夫不同,當加之以黼黻為六章。孔略而不言,孔意蓋以\CJKunderwave{周禮}制諸侯有三等之服,此諸侯同八章者,上古樸質,諸侯俱南面之尊,故合三為一等。且\CJKunderwave{禮}諸侯多同為一等,故\CJKunderwave{雜記}雲“天子九虞,諸侯七虞”,\CJKunderwave{左傳}雲“天子七月而葬,諸侯五月而葬”,是也。孔以此經上句“日、月、星辰、山、龍、華蟲”尊者在上,下句“藻、火、粉、米、黼、黻”尊者在下,黼黻尊於粉米,粉米尊於藻火,故從上以尊卑差之,士服藻火,大夫加以粉米,並藻火為四章。馬融不見孔傳,其注亦以為然,以古有此言,相傳為說也。蓋以衣在上為陽,陽統於上,故所尊在先。裳在下為陰,陰統於下,故所重在後。\CJKunderwave{詩}稱“玄袞及黼”\CJKunderwave{顧命}雲“麻冕黼裳”,當以黼為裳,故首舉黼以言其事如孔說也。天子諸侯下至黼黻,大夫粉米兼服藻火,是“上得兼下”也。士不得服粉米,大夫不得服黼黻,是“下不得僣上”也。訓“彰”為明,以五種之彩明施於五色,作尊卑之服,汝當分明制之,令其勿使僣濫也。\CJKunderline{鄭玄}云:“性曰採,施曰色。”以本性施於繒帛,故云“以五采施於五色”也。鄭云:“作服者,此十二章為五服,天子備有焉,公自山龍而下,侯伯自華蟲而下,子男自藻火而下,卿大夫自粉米而下。”亦是以意說也。此雲“作服”,推據衣服,所以經有“宗彝”,及孔雲旌旗亦以山、龍、華蟲為飾者,但此雖以服為主,上既雲“古人之象”,則法象分在器物,皆悉明之,非止衣服而已。旌旗器物皆是彩飾,被服以明尊卑,故總雲“作服”以結之。 \par}

予欲聞六律、五聲、八音,在治忽,以出納五言,汝聽。\footnote{言欲以六律和聲音,在察天下治理及忽怠者,又以出納仁義禮智信五德之言,施於民以成化,汝當聽審之。出如字,又尺遂反,注同。納如字,又音內。}予違,汝弼。汝無面從,退有後言。\footnote{我違道,汝當以義輔正我。無得面從我違,而退後有言我不可弼。}欽四鄰。庶頑讒說,若不在時,\footnote{四近前後左右之臣,敕使敬其職。眾頑愚讒說之人,若所行不在於是而為非者,當察之。}


{\noindent\zhuan\zihao{6}\fzbyks 傳“言欲”至“審之”。正義曰:此經大意,令臣審聽樂音,察世之治否以報君也。金、石、絲、竹、匏、土、革、木,八物各出其音,謂之“八音”。八音之聲皆有清濁,聖人差之以為五品,宮、商、角、徵、羽,謂之“五聲”。五聲高下各有所準則,聖人制為六律,與五聲相均,作樂者以律均聲,聲從器出。帝言我欲以六律和彼五聲八音,以此樂之音聲,察世之治否。\CJKunderwave{詩序}云:“治世之音安以樂,其政和;亂世之音怨以怒,其政乖。”此則聽聲知政之道也。言今聽作樂,若其音安樂和平,則時政辨治而修理也;若其音怨怒乖離,則時政忽慢而怠惰也;是用樂之聲音察天下治理及忽怠者也。知其治理,則保以修之;知其忽怠,則改以修之;此治理忽怠,人君所願聞也。又樂之感人,使和易調暢,若樂音合度,則言必得理。以此樂音出納仁義禮智信五德之言,乃君之發言,合彼五德,施之於人,可以成其教化,是出五言也。人之五言,合彼五德,歸之於君,可以成諷諫,是納五言也。君言可以利民,民言可以益君,是言之善惡由樂音而知也。此言之善惡,亦人君之所原聞也。政之理忽,言之善惡,皆是上所願聞,欲令察知以告己,得守善而改惡,故帝令臣,汝當為我聽審之也。六律、六呂,當有十二,惟言“六律”者,\CJKunderline{鄭玄}云:“舉陽,陰從可知也。”傳以“五言”為“五德之言”者,\CJKunderwave{漢書·律曆志}稱五聲播於五常,則角為仁,商為義,徵為禮,羽為智,宮為信,\CJKunderwave{志}之所稱必有舊說也。言五聲與五德相協,此論樂事而云“出納五言”,知是出納五德之言也。樂音和,則五德之言得其理;音不和,則五德之言違其度;故亦以樂音察五言也。帝之此言,自說臣之大法。於舜所聽,使聽韶樂也。襄二十九年\CJKunderwave{左傳}吳季札見舞韶樂而嘆曰:“德至矣哉,大矣!如天之無不幬也,如地之無不載也。”然則韶樂盡善盡美,有理無忽,而並言“忽”者,韶樂自美,取樂採人歌為曲,若其怠忽,則音辭亦有焉,故常使聽察之也。 \par}

{\noindent\zhuan\zihao{6}\fzbyks 傳“四近”至“察之”。正義曰:\CJKunderwave{冏命}云:“惟予一人無良,實賴左右前後有位之士,匡其不及。”知“四近”謂前後左右四者近君之臣,敕使敬其職也。更欲告以此下之辭,故敕之。眾頑愚讒說之人,若有所行不在於是而為非者,當察之。知其非,乃撻之書之。此與以下發端也。“庶頑讒說”謂朝廷之臣,“格則承之”乃謂天下之人。舜之朝廷當無讒說之人,故設為大法,戒慎之耳。四近之臣,普謂近君之臣耳,無常人也。\CJKunderline{鄭玄}以“四近為左輔右弼,前疑後承”,惟伏生\CJKunderwave{書傳}有此言,\CJKunderwave{文王世子}雲“有師保,有疑承”,以外經傳無此官也。 \par}

侯以明之,撻以記之,\footnote{當行射侯之禮,以明善惡之教。笞撻不是者,使記識其過。撻,他末反,又他達反。笞,敕疑反。}書用識哉,欲並生哉!\footnote{書識其非,欲使改悔,與共並生。}工以納言,時而颺之,\footnote{工,樂官,掌誦詩以納諫,當是正其義而颺道之。颺音揚。}格則承之庸之,否則威之。”\footnote{天下人能至於道則承用之,任以官。不從教則以刑威之。否,方有反,徐音鄙。任,汝鴆反。}

{\noindent\zhuan\zihao{6}\fzbyks 傳“當行”至“其過”。正義曰:\CJKunderwave{禮}射皆張侯射之,知“侯以明之”,“當行射侯之禮,以明善惡之教”。射禮有序賓以賢,詢眾擇善之義,是可以明善惡也。“笞撻不是者,使記識其過”,謂過輕者也,大罪刑殺之矣。古之射侯之士,無以言之。案\CJKunderwave{周禮·司裘}云:“王大射則供虎侯、熊侯、豹侯、設其鵠。諸侯則供熊侯、豹侯,卿大夫則供麋侯,皆設其鵠。”\CJKunderline{鄭玄}注云:“虎九十弓,即方一丈八尺。熊七十弓,方一丈四尺。豹、麋五十弓,方一丈。”鄭又引\CJKunderwave{梓人}“為侯,廣與崇方,三分其廣,而鵠居一焉”。則丈八之侯,鵠方六尺。丈四之侯,鵠方四尺六寸大半寸。一丈之侯,鵠方三尺三寸少半寸,此皆大射之侯也。\CJKunderwave{射人}云:“王以六耦射三侯,五正。諸侯以四耦射二侯,三正。孤卿大夫以三耦射一侯,二正。士以三耦射豹侯,二正。”\CJKunderline{鄭玄}注云:“五正者,五采。中朱,次白,次蒼,次黃,玄居外。三正者,去玄、黃。二正者,去白、蒼而畫以朱、綠。”此賓射之侯也。鄭以賓射三侯步數高廣,與大射侯同,正大如鵠。\CJKunderwave{司裘}及\CJKunderwave{射人}所云諸侯者,謂圻內諸侯。若圻外諸侯,則\CJKunderwave{儀禮·大射}云,大侯九十弓,熊侯七十弓,豹侯五十弓,皆以三耦;其賓射則無文。若天子已下之燕射,案\CJKunderwave{鄉射記}云:“天子熊侯,白質。諸侯麋侯,赤質。大夫布侯,畫以虎豹。士布侯,畫以鹿豕。”熊侯已下同五十弓,即侯身高一丈,君臣共射之。 \par}

{\noindent\zhuan\zihao{6}\fzbyks 傳“書識”至“並生”。正義曰:書識其非,亦是小過者也,欲並生哉。總上三者,“侯以明之,撻以記之,書用識哉”,皆是欲其改悔,與無過之人共並生也。 \par}

{\noindent\zhuan\zihao{6}\fzbyks 傳“工樂”至“道之”。正義曰:\CJKunderwave{禮}通謂樂官為工,知\CJKunderwave{工}是樂官,則\CJKunderwave{周禮}大師、瞽蒙之類也。樂官掌頌詩言以納諫,以詩之義理或微,人君聽之,若有不悟,當正其義而揚道之。揚,舉也,舉而道向君也。 \par}

{\noindent\zhuan\zihao{6}\fzbyks 傳“天下”至“威之”。正義曰:言“承之用之”,則此人未在官也,故言謂天下民必也。能至於道即賢者,故承用之而任以官也。“否”謂不從教者,則以刑威之而罪其身也。臣過必小,故撻之書之;人罪或大,故以刑威之。 \par}

{\noindent\shu\zihao{5}\fzkt “帝曰臣”至“威之”。正義曰:帝以\CJKunderline{禹}然己言,又說須臣之事:“作我股肱耳目。”言已動作視聽皆由臣也。“我欲助我所有之人,使之家給人足,汝當翼讚我也。我欲布陳智力於天下四方,為立治之功,汝等當與我為之。我欲觀示君臣上下以古人衣服之法象,其日、月、星辰、山、龍、華蟲作會,合五采而畫之。又畫山、龍、華蟲於宗廟彝樽。其藻、火、粉、米、黼、黻於絺葛而刺繡,以五種之彩明施於五色,製作衣服,汝當為我明其差等而制度之。我欲聞知六律,和五聲,播之於八音,以此音樂察其政治與忽怠者,其樂音又以出納五德之言,汝當為我聽審之。我有違道,汝當以義輔成我。汝無得知我違非而對面從我,退而後更有言,雲我不可輔也”。既言其須臣之力,乃總敕之:“敬其職事哉,汝在我前後左右四旁鄰近之臣也。其眾類頑愚讒說之人,若有所行不在於是而為非者,汝當察之以法,行射侯之禮,知其善惡以明別之。行有不是者,又撻其身以記之。書其過者以以識哉。所以撻之書之者,冀其改悔,欲與並生活哉!工樂之官以納諫言於上,當是正其義而顯揚之,使我自知得失也。”又總言御下之法:“天下之人有能至於道者,則當承受而進用之,當任以官也。不從教者,則以刑罰威之,當罪其身也。此等皆汝臣之所為。” \par}

\CJKunderline{禹}曰:“俞哉!帝光天之下,至於海隅蒼生,\footnote{光天之下,至於海隅,蒼蒼然生草木。言所及廣遠。}萬邦黎獻,共惟帝臣,惟帝時舉。敷納以言,明庶以功,車服以庸。\footnote{獻,賢也。萬國眾賢,共為帝臣。帝舉是而用之,使陳布其言,明之皆以功大小為差,以車服旌其能用之。}誰敢不讓?敢不敬應。\footnote{上惟賢是用,則下皆敬應上命而讓善。應,應對之應。}帝不時敷,同日奏,罔功。\footnote{帝用臣不是,則遠近布同而日進於無功,以賢愚並位,優劣共流故。}


{\noindent\zhuan\zihao{6}\fzbyks 傳“光天”至“廣遠”。正義曰:\CJKunderwave{堯典}之序,訓“光”為充,即此亦為充,言充滿大天之下也。據其方面即四隅,為遠至於海隅,舉極遠之處,言帝境所及廣遠,其內多賢人也。 \par}

{\noindent\zhuan\zihao{6}\fzbyks 傳“獻賢”至“用之”。正義曰:\CJKunderwave{釋言}云:“獻,聖也。賢是聖之次,臣德不宜言聖,故為賢也。“萬國眾賢,共為帝臣”,言求臣之處多也。帝舉是眾賢而用之,使陳布其言,令其自說己之所能,聽其言而納受之,依其言而考試之,顯明眾臣,皆以功大小為差,然後賜車服以旌別其人功能事用,是舉賢用人之法也。\CJKunderwave{舜典}云:“敷奏以言,明試以功。”“奏”、“試”二字與此異者,彼言施於諸侯,其人見為國君,故令奏言試功;此謂方始擢用,故言“納”、“庶”。“納”謂受取之,“庶”謂在群眾。 \par}

{\noindent\zhuan\zihao{6}\fzbyks 傳“帝用”至“流故”。正義曰:帝用臣不是,不以言考功,在下知帝不分別善惡,則無遠近遍佈同心,日日進於無功之人,由其賢愚並位,優劣共流故也。“敷”是布之義,故言“遠近布同”,同心妄舉也。 \par}

無若\CJKunderline{丹朱}傲,惟慢遊是好。\footnote{\CJKunderline{丹朱},堯子。舉以戒之。傲,五報反,字又作奡。好,呼報反。}傲虐是作,罔晝夜頟頟。\footnote{傲戲而為虐,無晝夜,常崑崙肆惡無休息。傲,五羔反,徐五報反,注同。頟,五客反。}罔水行舟,朋淫於家,用殄厥世。\footnote{朋,群也。\CJKunderline{丹朱}習於無水陸地行舟,言無度。群淫於家,妻妾亂。用是絕其世,不得嗣。殄,徒見反。}


{\noindent\zhuan\zihao{6}\fzbyks 傳“\CJKunderline{丹朱},堯子”。正義曰:\CJKunderwave{漢書·律曆志}云:“堯讓舜,使子朱處於丹淵,為諸侯。”則“朱”是名,“丹”是國也。 \par}

{\noindent\zhuan\zihao{6}\fzbyks 傳“傲戲”至“休息”。正義曰:\CJKunderwave{詩}美衛武公云:“善戲謔兮,不為虐兮。”\CJKunderline{丹朱}反之,故“傲戲而為虐”也。“崑崙”是不休息之意,“肆”謂縱恣也,晝夜常崑崙然縱恣為惡,無休息時也。 \par}

{\noindent\zhuan\zihao{6}\fzbyks 傳“朋群”至“得嗣”。正義曰:朋輩與群聚義同,故“朋”為群也。聖人作車以行陸,作舟以行水,\CJKunderline{丹朱}乃習於無水而陸地行舟,言其所為惡事無節度也。此乃稟受惡性,習惡事也。\CJKunderline{鄭玄}云:“\CJKunderline{丹朱}見洪水時人乘舟,今水已治,猶居舟中,崑崙使人推行之。”案下句云,予創若時,乃勤治水,則\CJKunderline{丹朱}行舟之時,水尚未除,非效洪水之時人乘舟也。“群淫於家”,言群聚妻妾,恣意淫之,無男女之別,故言“妻妾亂”也。用是之惡,故絕其世位,不得嗣父也。此用“殄厥世”一句,\CJKunderline{禹}既見世絕,今始言之,以明行惡之驗。此句非\CJKunderline{禹}所創,創之者,創其行之惡耳。 \par}

予創若時,娶於塗山,辛、壬、癸、甲。\footnote{創,懲也。塗山,國名。懲\CJKunderline{丹朱}之惡,辛日娶妻,至於甲日,復往治水,不以私害公。娶,促住反。復,扶又反。}\CJKunderline{啟}呱呱而泣,予弗子,惟荒度土功。\footnote{啟,\CJKunderline{禹}子也。\CJKunderline{禹}治水,過門不入,聞啟泣聲,不暇子名之,以大治度水土之功故。呱音孤。予如字,鄭將吏反。度,徒洛反。}


{\noindent\zhuan\zihao{6}\fzbyks 傳“創懲”至“害公”。正義曰:“創”與“懲”皆是見惡自止之意,故云“創,懲也”。哀七年\CJKunderwave{左傳}云:“\CJKunderline{禹}會諸侯於塗山。”杜預云:“塗山在壽春縣東北。”“塗山,國名”,蓋近彼山也。“娶於塗山”,言其所娶之國耳,非就妻家見妻也。懲\CJKunderline{丹朱}之惡,故不可不勤,故辛日娶妻,至於甲日復往治水。孔雲“復往”,則已嘗治水,而輟事成昏也。\CJKunderline{鄭玄}云:“登用之年,始娶於塗山氏,三宿而為帝所命治水。”鄭意娶後始受帝命,娶前未治水也。然娶後始受帝命,當雲聞命即行,不須計辛之與甲日數多少,當如孔說,輟事成昏也。此時\CJKunderline{禹}父新殛,而得為昏者,\CJKunderline{鯀}放而未死,不妨\CJKunderline{禹}娶。且治水四年,兗州始畢,\CJKunderline{禹}娶不必在殛\CJKunderline{鯀}之年也。 \par}

{\noindent\zhuan\zihao{6}\fzbyks 傳“啟\CJKunderline{禹}”至“功故”。正義曰:“啟,\CJKunderline{禹}子”,\CJKunderwave{世本}文也。\CJKunderwave{孟子}稱\CJKunderline{禹}治水,三過其門而不入,是至門而聞啟泣聲,不暇如人父子,名為己子而愛念之,以其為大治度水土之功故也。訓“荒”為“大治”,謂去其水。“度”謂量其功,故“治度”連言之。 \par}

弼成五服,至於五千,州十有二師。\footnote{五服,侯、甸、綏、要、荒服也。服五百里,四方相距為方五千裡,治洪水輔成之。一州用三萬人功,九州二十七萬庸。至於五千,馬云:“面五千裡,為方萬里。”鄭云:“五服已五千,又弼成為萬里。”州十有二師,二千五百人為師,鄭云:“師,長也。”要,一遙反。}


{\noindent\zhuan\zihao{6}\fzbyks 傳“五服”至“萬庸”。正義曰:據\CJKunderwave{禹貢}所云五服之名數,知五服即甸、侯、綏、要、荒服也。彼五服每服五百里,四面相距為方五千裡也。王肅云:“五千裡者,直方之數。若其迥邪委曲,動有倍加之較。”是直路五千裡也。“治洪水輔成之”者,謂每服之內,為其小數,定其差品,各有所掌,是\CJKunderline{禹}輔成之也。\CJKunderwave{周禮}大司馬法,二千五百人為師。每州十有二師,通計之一州用三萬人功,總計九州用二十七萬庸。“庸”亦功也。州境既有闊狹,用功必有多少,例言“三萬人”者,大都通率為然,惟言用“三萬人”者,不知用功日數多少,治水四年乃畢,用功蓋多矣,不知用幾日也。\CJKunderline{鄭玄}云:“輔五服而成之,至於面方,各五千裡,四面相距為方萬里。九州州立十二人為諸侯師,以佐牧。堯初制五服,服各五百里。要服之內方四千裡,曰九州。其外荒服,曰四海。此\CJKunderline{禹}所受,\CJKunderwave{地記書}曰‘崑崙山東南,地方五千裡,名曰神州’者。\CJKunderline{禹}弼五服之殘數,亦每服者合五百里,故有萬里之界、萬國之封焉。猶用要服之內為九州,州更方七千裡。七七四十九,得方千里者四十九。其一以為圻內,餘四十八,八州分而各有六。\CJKunderwave{春秋}傳曰:‘\CJKunderline{禹}朝群臣於會稽,執玉帛者萬國。’言執玉帛者,則九州之內諸侯也。其制特置牧,以諸侯賢者為之師。蓋百國一師,州十有二師,則州千二百國也。八州凡九千六百國,其餘四百國在圻內。與\CJKunderwave{王制}之法準之,八州通率封公侯百里之國者一,伯七十里之國二,子男五十里之國四,方百里者三,封國七有畸,至於圻內,則子男而已。”鄭云:“\CJKunderline{禹}弼成五服,面各五千裡。”王肅\CJKunderwave{禹貢}之注已難之矣。傳稱“萬”,盈數也,“萬國”舉盈數而言,非謂其數滿萬也。\CJKunderwave{詩}桓曰“綏萬邦烝民”,曰“揉此萬邦”,豈周之建國復有萬乎?天地之勢,平原者甚少,山川所在不啻居半,豈以不食之地,亦封建國乎?王圻千里,封五十里之國四百,則圻內盡以封人,王城宮室無建立之處,言不顧實,何至此也?百國一師,不出典記,自造此語,何以可從?“\CJKunderline{禹}朝群臣於會稽”,\CJKunderwave{魯語}文也。“執玉帛者萬國”,\CJKunderwave{左傳}文也,採合二事,亦為謬矣。 \par}

外薄四海,咸建五長。\footnote{薄,迫也。言至海。諸侯五國立賢者一人為方伯,謂之五長,以相統治,以獎帝室。薄,蒲各反,徐扶各反。長,丁丈反。五長,眾官之長。}各迪有功,苗頑弗即工,帝其念哉!\footnote{九州五長各蹈為有功,惟三苗頑兇,不得就官。善惡分別。別,彼列反。}帝曰:“迪朕德,時乃功惟敘。”\footnote{言天下蹈行我德,是汝治水之功有次序,敢不念乎!}

{\noindent\zhuan\zihao{6}\fzbyks 傳“薄迫”至“帝室”。正義曰:\CJKunderwave{釋言}云:“逼,迫也。”“薄”者逼近之義,故云迫也。外迫四海,言從京師而至於四海也。\CJKunderwave{釋地}云:“九夷、八狄、七戎、六蠻謂之四海。”謂九州之外也。\CJKunderwave{王制}云:“五國以為屬,屬有長。”此“建五長”亦如彼文,故云“諸侯五國立賢者一人為方伯,謂之五長,以相統治,欲以共獎帝室故”也。僖元年\CJKunderwave{公羊傳}曰:“上無天子,下無方伯。”“方伯”謂\CJKunderwave{周禮}“九命作伯”者也。\CJKunderwave{王制}云:“千里之外設方伯。”“方伯”一為之長,謂\CJKunderwave{周禮}“八命作牧”者也。傳言五國立一人為方伯,直是五國之長耳,與彼異也,以其是當方之長,故傳以“方伯”言之。 \par}

{\noindent\zhuan\zihao{6}\fzbyks 傳“九州”至“分別”。正義曰:蹈為有功之長,言蹈履典法,持之有功。惟三苗頑兇,不得就官,謂舜分北三苗之時,苗君有罪,不得就其諸侯國君之官,而被流於遠方也。言“九州五長各蹈為有功”,則海內諸侯皆有功矣。惟有三苗不得就官,以見天下大治,而惡者少耳。頑則不得就官,言善惡分別也。 \par}

{\noindent\shu\zihao{5}\fzkt “\CJKunderline{禹}曰”至“惟敘”。正義曰:\CJKunderline{禹}既得帝言,乃答帝曰:“然。既帝之任臣,又言當擇人,充滿大天之下,旁至四海之隅,蒼蒼然生草木之處,皆是帝德所及。其內有萬國眾賢,皆共為帝臣。”言其可用者甚眾也。“帝當就是眾賢之內,舉而用之。其舉用之法,各使陳布其言,納受之,以其言之所能,從其所能而驗試之。明顯眾人所能,當以功之大小。既知有功,乃賜之以車服,以表其功有能用。帝以此法用人,即在下之人,知官不妄授,必用度才能而使之。如此,誰敢不讓有德?敢不敬應帝命而推先善人也?若帝用臣不是,不宜試驗,不知臧否,則群臣遠近,遍佈同心,而日進無功之人”。既戒帝擇人,又勸帝自勤。“無若\CJKunderline{丹朱}之傲,惟慢褻之遊是其所好。傲戲而為虐,是其所為。為此惡事,不問晝夜,而頟頟然恆為之無休息。又無水而陸地行舟,群朋淫泆於室家之內。用此之故,絕其世嗣,不得居位。我本創\CJKunderline{丹朱}之惡若是也,故娶於塗山之國,歷辛、壬、癸、甲四日而即往治水。其後過門不入,聞啟呱呱而泣,我不暇入而子名之,惟以大治度水土之功故也。水土既平,乃輔成五服,四面相距,至於五千裡。州十有二師,其治水之時,所役人功,每州用十有二師,各用三萬人也。自京師外迫及四海,其間諸侯五國皆立一長,迆相統領。以此諸侯各蹈行所職,併為有功,惟有三苗頑兇,不能就官,我以供勤之故,得使災消沒。帝念此事哉!不可不自勤也”。帝答\CJKunderline{禹}曰:“天下之人皆蹈行我德,是汝治水之功,惟有次敘故也。”受其戒而美其功也。 \par}

\CJKunderline{皋陶}方祗厥敘,方施象刑惟明。\footnote{方,四方。\CJKunderline{禹}五服既成,故\CJKunderline{皋陶}敬行其九德考績之次序於四方,又施其法刑,皆明白。史因\CJKunderline{禹}功重美之。重,直用反。}

{\noindent\zhuan\zihao{6}\fzbyks 傳“方四”至“美之”。正義曰:\CJKunderline{皋陶}為帝所任,遍及天下,故“方”為四方也。天下蹈行帝德,水土既治,亦由刑法彰明,若使水害不息,\CJKunderline{皋陶}法無所施,若無\CJKunderline{皋陶}以刑,人亦未能奉法。天下蹈行帝德,二臣共有其功,故史因帝歸功於\CJKunderline{禹},兼記\CJKunderline{皋陶}之功。\CJKunderwave{舜典}與\CJKunderwave{大禹謨}已美\CJKunderline{皋陶},故言“重美之”也。傳言“考績之次敘”者,\CJKunderline{皋陶}所言九德,依德以考其功績,亦是刑法之事,故兼言也。鄭雲“歸美於二臣”,則以此經為帝語。此文上無所由,下無所結,形勢非語辭也,故傳以為史因記之 \par}

{\noindent\shu\zihao{5}\fzkt “\CJKunderline{皋陶}”至“惟明”。正義曰:此經史述為文,非帝言也。史以\CJKunderline{禹}成五服,帝念\CJKunderline{禹}功,故因美\CJKunderline{皋陶}。言\CJKunderline{禹}既弼成五服,故\CJKunderline{皋陶}於其四方敬行九德考績之法,有次敘也。又於四方施其刑法,惟明白也。由\CJKunderline{禹}有此大功,故史重美之也。 \par}

\CJKunderline{夔}曰:“戛擊鳴球,搏拊、琴、瑟以詠。祖考來格。\footnote{戛擊,柷敔,所以作止樂。搏拊以韋為之,實之以糠,所以節樂。球,玉磬。此舜廟堂之樂,民悅其化,神歆其祀,禮備樂和,故以祖考來至明之。夔,求龜反。戛,居八反,徐古八反,馬云:“櫟也。”球音求。搏音博。拊音撫。柷,尺叔反,所以作樂。敔,魚呂反,所以止樂。糠音康。歆,許金反。}虞賓在位,群后德讓。\footnote{\CJKunderline{丹朱}為王者後,故稱賓。言與諸侯助祭,班爵同,推先有德。}


{\noindent\zhuan\zihao{6}\fzbyks 傳“戛擊”至“明之”。正義曰:“戛擊”是作用之名,非樂器也,故以“戛擊”為柷敔。柷敔之狀,經典無文,漢初已來學者相傳,皆雲柷如漆桶,中有椎柄,動而擊其旁也。敔狀如伏虎,背上有刻,戛之以為聲也。樂之初,擊柷以作之;樂之將末,戛敔以止之,故云“所以作止樂”雙解之。\CJKunderwave{釋樂}云:“所以鼓柷謂之止,所以鼓敔謂之籈。”郭璞云:“柷如漆桶,方二尺四寸,深一尺八寸,中有椎,柄連氐,挏之令左右擊。止者,其椎名也。敔如伏虎,背上有二十七鉏鋙刻,以木長一尺櫟之。籈者,其名也。”是言擊柷之椎名為“止”,戛敔之木名為“籈”,“戛”即櫟也。漢禮器制度及\CJKunderwave{白虎通}、馬融、\CJKunderline{鄭玄}、李巡其說皆為然也。惟郭璞為詳,據見作樂器而言之。搏拊形如鼓,以韋為之,實之以糠,擊之以節樂,漢初相傳為然也。\CJKunderwave{釋器}云:“球,玉也。”“鳴球”謂擊球使鳴,樂器惟磬用玉,故球為玉磬。\CJKunderwave{商頌}雲“依我磬聲”,磬亦玉磬也。\CJKunderline{鄭玄}云:“磬,懸也,而以合堂上之樂。玉磬和,尊之也。”然則鄭以球玉之磬懸於堂下,尊之,故進之使在上耳。此“舜廟堂之樂”,謂廟內堂上之樂,言“祖考來格”,知在廟內,下雲“下管”,知此在堂上也。馬融見其言“祖考”,遂言“此是舜除瞽瞍之喪,祭宗廟之樂”,亦不知舜父之喪在何時也。但此論韶樂,必在即政後耳。此說樂音之和,而云“祖考來格”者,聖王先成於人,然後致力於神。言“人悅其化,神歆其祀,禮備樂和,所以“祖考來至“明矣。以祖考來”至“明樂之和諧也。詩稱神之格思不可度思而云:“祖考來至”者,王肅云:“祖考來至者,見其光輝也。”蓋如\CJKunderwave{漢書·郊祀志}稱武帝郊祭天祠,上有美光也。此經文次,以“柷敔”是樂之始終,故先言“戛擊”。其“球”與“搏拊琴瑟”,皆當彈擊,故使“鳴”冠於“球”上,使下共蒙之也。\CJKunderline{鄭玄}以“戛擊鳴”球三者,皆總下樂,櫟擊此四器也”。樂器惟敔當櫟耳,四器不櫟,鄭言非也。 \par}

{\noindent\zhuan\zihao{6}\fzbyks 傳“\CJKunderline{丹朱}”至“有德”。正義曰:\CJKunderwave{微子之命}雲“作賓於王家”,\CJKunderwave{詩}頌\CJKunderline{微子}之來,謂之“有客”,是王者之後,為時王所賓也。故知“虞賓”謂\CJKunderline{丹朱},為王者後,故稱賓也。王者立二代之後,而獨言\CJKunderline{丹朱}者,蓋高辛氏之後,無文而言,故惟指\CJKunderline{丹朱}也。王者之後,尊於群后,故殊言“在位”。群后亦在位也,後言德讓,\CJKunderline{丹朱}亦以德讓矣,故言“與諸侯助祭,年爵同者,推先有德”也。二王之後併為上公,亦有與\CJKunderline{丹朱}爵同,故\CJKunderline{丹朱}亦讓也。\CJKunderline{丹朱}之性下愚,堯不能化,此言“有德”者,猶上雲“瞽亦允若”,暫能然也。 \par}

下管鼗鼓,合止柷敔,\footnote{堂下樂也。上下合止樂,各有柷敔,明球、弦、鍾、籥,各自互見。鼗音桃。合如字,徐音合。籥,餘若反。互音護。見,賢遍反,下“見細器”同。}笙鏞以間,鳥獸蹌蹌。\footnote{鏞,大鐘。跡葴迭也。吹笙擊鐘,鳥獸化德,相率而舞,蹌蹌然。鏞音庸。間,間側之間。鳥獸,孔以為自舞也。馬云:“鳥獸,筍篪也。”蹌,七羊反,舞貌。\CJKunderwave{說文}作蹌,云:“鳥獸求食聲。”迭,直結反。}


{\noindent\zhuan\zihao{6}\fzbyks 傳“堂下”至“互見”。正義曰:經言“下管”,知是“堂下樂”也。敔當戛之,柷當擊之,上言“戛擊”,此言“柷敔”,其事是一,故云“上下合止樂,各有柷敔”也。言堂下堂上合樂各以柷,止樂各以敔也。上言作用,此言器名,兩相備也。上下皆有“柷敔”,兩見其文,明球、弦、鍾、籥,上下樂器不同,各自更互見也。弦謂琴瑟。鍾,鏞也。籥,管也。琴瑟在堂,鍾籥在庭,上下之器各別,不得兩見其名,各自更互見之。依\CJKunderwave{大射}禮,鐘磬在庭,今鳴球於廟堂之上者,案\CJKunderwave{郊特牲}雲“歌者在上”,貴人聲也。\CJKunderwave{左傳}雲“歌鐘二肆”,則堂上有鍾,明磬亦在堂上,故漢魏已來登歌皆有鐘磬。\CJKunderwave{燕禮}、\CJKunderwave{大射}堂上無鐘磬者,諸侯樂不備也。 \par}

{\noindent\zhuan\zihao{6}\fzbyks 傳“鏞大”至“蹌蹌然”。正義曰:\CJKunderwave{釋樂}云:“大鐘謂之鏞。”李巡曰:“大鐘音聲大。鏞,大也。”孫炎曰:“鏞,深長之聲。”\CJKunderwave{釋詁}云:“間,代也。”孫炎曰:“間廁之代也。”\CJKunderwave{釋言}云:“遞,迭也。”李巡曰:“遞者,更迭間廁,相代之義。”故“間”為迭也。吹笙擊鐘,更迭而作,鳥獸化德,相率而舞,蹌蹌然。下雲“百獸率舞”,知此“蹌蹌然”亦是舞也。\CJKunderwave{禮}雲“凡行容愓愓”,“大夫濟濟,士蹌蹌”,是為行動之貌,故為舞也。 \par}

簫\CJKunderwave{韶}九成,鳳皇來儀。”\footnote{韶,舜樂名。言簫,見細器之備。雄曰鳳,雌曰皇,靈鳥也。儀,有容儀。備樂九奏而致鳳皇,則餘鳥獸不待九而率舞。韶,時昭反。}\CJKunderline{夔}曰:“於!予擊石拊石,百獸率舞,庶尹允諧。”\footnote{尹,正也,眾正官之長。信皆和諧,言神人治。始於任賢,立政以禮,治成以樂,所以太平。於、予並如字。}

{\noindent\zhuan\zihao{6}\fzbyks 傳“韶舜”至“率舞”。正義曰:“韶”是舜樂,經傳多矣,但余文不言“簫”。“簫”乃樂器,非樂名,簫是樂器之小者。“言簫,見細器之備”,謂作樂之時,小大之器皆備也。\CJKunderwave{釋鳥}云:“鶠,鳳。其雌皇。”是此鳥“雄曰鳳,雌曰皇”。\CJKunderwave{禮運}云:“麟、鳳、龜、龍謂之四靈。”是鳳皇為神靈之鳥也。\CJKunderwave{易·漸卦}上九:“鴻漸於陸,其羽可用為儀。”是儀為“有容儀”也。“成”謂樂曲成也。鄭云:“成猶終也。”每曲一終,必變更奏,故經言“九成”,傳言“九奏”,\CJKunderwave{周禮}謂之“九變”,其實一也。言“簫”見細器之備,備樂九奏而致鳳皇,則其餘鳥獸不待九而率舞也。尊者體盤,靈瑞難致,故“九成”之下始言“鳳皇來儀”。“鳥獸蹌蹌”乃在上句,傳據此文言鳥獸易來,鳳皇難致,故云“鳥獸不待九”也。樂之作也,依上下遞奏間合而後曲成,神物之來,上下共致,非堂上堂下別有忻感。以祖考尊神,配堂上之樂;鳥獸賤物,故配堂下之樂。總上下之樂,言九成致鳳。尊異靈瑞,故別言爾,非堂上之樂獨致神來,堂下之樂偏令獸舞也。\CJKunderline{鄭玄}注\CJKunderwave{周禮}具引此文,乃雲“此其在於宗廟九奏效應也”。是言祖考來格、百獸率舞皆是九奏之事也。\CJKunderwave{大司樂}云:“凡六樂者,六變而致象物及天神。”\CJKunderline{鄭玄}云:“象物,有象在天,所謂四靈者。”彼謂大蜡之祭,作樂以致其神。此謂鳳皇身至,故九奏也。 \par}

{\noindent\zhuan\zihao{6}\fzbyks 傳“尹正”至“太平”。正義曰:“尹,正”,\CJKunderwave{釋言}文。“眾正官之長”,謂每職之首,\CJKunderwave{周官}所謂“\CJKunderline{唐}、\CJKunderline{虞}稽古,建官惟百”是也。“信皆和諧”,言職事修理也。上雲“祖考來格”,此言眾正官治,言神人洽,樂音和也。此篇初說用臣之法,末言樂音之和,言其“始用任賢,立政以禮,治成以樂,所以得致太平”,解史錄夔言之意。 \par}

{\noindent\shu\zihao{5}\fzkt “\CJKunderline{夔}曰”至“允諧”。正義曰:\CJKunderline{皋陶}、\CJKunderline{大禹}為帝設謀,大聖納其昌言,天下以之致治,功成道洽,禮備樂和,史述夔言,繼之於後。\CJKunderline{夔}曰:“在舜廟堂之上,戛敔擊柷,鳴球玉之磬,擊搏拊,鼓琴瑟,以歌詠詩章,樂音和協,感致幽冥,祖考之神來至矣。虞之賓客\CJKunderline{丹朱}者在於臣位,與群君諸侯以德相讓。此堂上之樂,所感深矣。又於堂下吹竹管,擊鼗鼓,合樂用柷,止樂用敔,吹笙擊鐘,以次迭作,鳥獸相率而舞,其容蹌蹌然。堂下之樂,感亦深矣。簫韶之樂,作之九成,以致鳳皇來而有容儀也。”夔又曰:“嗚呼!”嘆舜樂之美。“我大擊其石磬,小拊其石磬,百獸相率而舞,鳥獸感德如此,眾正官長信皆和諧矣。”言舜致教平而樂音和,君聖臣賢,謀為成功所致也。 \par}

帝庸作歌曰:“敕天之命,惟時惟幾。”\footnote{用庶尹允諧之政,故作歌以戒,安不忘危。敕,正也。奉正天命以臨民,惟在順時,惟在慎微。}乃歌曰:“股肱喜哉!元首起哉!百工熙哉!”\footnote{元首,君也。股肱之臣喜樂盡忠,君之治功乃起,百官之業乃廣。樂者洛。盡,津忍反。}\CJKunderline{皋陶}拜手稽首,颺言曰:“念哉!\footnote{大言而疾曰颺。承歌以戒帝。颺音揚。}率作興事,慎乃憲,欽哉!\footnote{憲,法也。天子率臣下為起治之事,當慎汝法度,敬其職。}


{\noindent\zhuan\zihao{6}\fzbyks 傳“用庶”至“慎微”。正義曰:此承夔言之下,既得總言而歌,故知“帝庸作歌”者,用“庶尹允諧之政,故作歌以自戒之,安不忘危”也。“敕”是正齊之意,故為正也。言人君奉正天命,以臨下民,惟在順時,不妨農務也,惟在慎微,不忽細事也。\CJKunderline{鄭玄}以為戒臣,孔以為自戒者,以正天之命是人君之事故也。 \par}

{\noindent\zhuan\zihao{6}\fzbyks 傳“元首”至“乃廣”。正義曰:\CJKunderwave{釋詁}云:“元、良,首也。”僖三十三年\CJKunderwave{左傳}稱狄人歸先軫之元,則“元”與“首”各為頭之別名,此以“元首”共為頭也。君臣大體猶如一身,故“元首,君也”。“股肱之臣喜樂盡忠”,謂樂行君之化。“君之治功乃起”言無變事業,事業在於百官,故眾功皆起,百官之業乃廣也。 \par}

{\noindent\zhuan\zihao{6}\fzbyks 傳“憲法”至“其識”。正義曰:“憲,法”,\CJKunderwave{釋詁}文。此言“興事”,對上“起哉”。“天子率臣下為起治之事”,言臣不能獨使起也。 \par}

屢省乃成,欽哉!”\footnote{屢,數也。當數顧省汝成功,敬終以善,無懈怠。屢,力具反。省,悉井反。數,色角反。懈,隹賣反。}乃賡載歌曰:“元首明哉!股肱良哉!庶事康哉!”\footnote{賡,續。載,成也。帝歌歸美股肱,義未足,故續歌。先君後臣,眾事乃安,以成其義。賡,加孟反,劉皆行反,\CJKunderwave{說文}以為古續字。}又歌曰:“元首叢脞哉!股肱惰哉!萬事墮哉!\footnote{叢脞,細碎無大略。君如此,則臣懈惰,萬事墮廢,其功不成。歌以申戒。叢,太公反。脞,倉果反。徐音鎖,馬云:“叢,總也。脞,小也。”惰,徒臥反。墮,許規反。}帝拜曰:“俞,往欽哉!”\footnote{拜受其歌,戒群臣自今以往,敬其職事哉。}

{\noindent\zhuan\zihao{6}\fzbyks 傳“屢數”至“懈怠”。正義曰:\CJKunderwave{釋詁}云:“屢、數,疾也。”俱訓為疾,故“屢”為數也。“顧省汝成功”,謂已有成功,令數顧省之,敬終以善,無懈怠也。恐其惰於已成功,故以此為戒。 \par}

{\noindent\zhuan\zihao{6}\fzbyks 傳“賡續”至“其義”。正義曰:\CJKunderwave{詩}云:“西有長賡。”\CJKunderwave{毛傳}亦以“賡”為續,是相傳有此訓也。\CJKunderline{鄭玄}以“載”為始,孔以“載”為成,各以意訓耳。“帝歌歸美股肱,義未足”者,非君之明,為臣不能盡力,空責臣功是其義未足。以此續成帝歌,必先君後臣,眾事乃安,故以此言成其義也。 \par}

{\noindent\zhuan\zihao{6}\fzbyks 傳“叢脞”至“申戒”。正義曰:孔以“叢脞”為細碎無大略,鄭以“叢脞,總聚小小之事以亂大政”,皆是以意言耳。君無大略,則不能任賢,功不見知,則臣皆懈惰,萬事墮廢,其功不成,故又歌以重戒也。“庶事”、“萬事”,為義同而文變耳。 \par}

{\noindent\shu\zihao{5}\fzkt “帝庸”至“往欽哉”。正義曰:帝既得夔言,用此庶尹允諧之政,故乃作歌自戒。將歌而先為言曰:“人君奉正天命,以臨下民,惟當在於順時,惟當在於慎微。”既為此言,乃歌曰:“股肱之臣喜樂其事哉!元首之君政化乃起哉!百官事業乃得廣大哉!”言君之善政由臣也。\CJKunderline{皋陶}拜手稽首,颺聲大言曰:“帝當念是言哉!率領臣下,為起政治之事,慎汝天子法度而敬其職事哉!又當數自顧省已之成功而敬終之哉!”乃續載帝歌曰:“會是元首之君能明哉!則股肱之臣乃善哉!?事皆得安寧哉!”既言其美,又戒其惡:“元首之君叢脞細碎哉!則股肱之臣懈怠緩慢哉!眾事悉皆墮廢哉!”言政之得失由君也。帝拜而受之曰:“然。”然其所歌顯是也。“汝群臣自今已往,各敬其職事哉!” \par}

%%% Local Variables:
%%% mode: latex
%%% TeX-engine: xetex
%%% TeX-master: "../Main"
%%% End:
