%% -*- coding: utf-8 -*-
%% Time-stamp: <Chen Wang: 2024-04-02 11:42:41>

% {\noindent \zhu \zihao{5} \fzbyks } -> 注 (△ ○)
% {\noindent \shu \zihao{5} \fzkt } -> 疏

\chapter{卷十二}


\section{洪範第六(洪範上、下)}


武王勝殷,殺受,立武庚,\footnote{不放而殺,紂自焚也。武庚,紂子,以為王者後;一名祿父。勝,商證反。父音甫。}以箕子歸,作\CJKunderwave{洪範}。\footnote{歸鎬京,箕子作之。○範音範。鎬,胡老反,本又作鄗,武王所都也。}

{\noindent\zhuan\zihao{6}\fzbyks 傳“不放”至“祿父”。正義曰:放桀也。湯放桀,此不放而殺之者,紂自焚而死也。\CJKunderwave{殷本紀}雲“紂兵敗,紂走入登鹿臺,衣其寶玉衣,赴火而死。武王遂斬紂頭懸之太白旗”是也。\CJKunderwave{泰誓}雲“取彼兇殘”,則志在於殺也。死猶斬之,則生亦不放,傳據實而言之耳。\CJKunderwave{本紀}雲“封紂子武庚祿父以續殷祀”,是以為王者後也。\CJKunderwave{本紀}“武庚祿父”雙言之,伏生\CJKunderwave{尚書}雲“武王勝殷,繼公子祿父”,是一名祿父也。鄭云:“武庚字祿父,春秋之世有齊侯祿父、蔡侯考父、季孫行父,父亦是名,未必為字,故傳言‘一名祿父’。” \par}

{\noindent\zhuan\zihao{6}\fzbyks 傳“歸”至“作之”。正義曰:上篇雲“至於豐”者,文王之廟在豐,至豐先告廟耳。時王都在鎬,知“歸”者,“歸鎬京”也。此經文旨異於餘篇,非直問答而已,不是史官敘述,必是箕子既對武王之問,退而自撰其事,故傳特雲“箕子作之”。\CJKunderwave{書傳}云:“武王釋箕子之囚,箕子不忍周之釋,走之朝鮮。武王聞之,因以朝解封之。箕子既受周之封,不得無臣禮,故於十三祀來朝,武王因其朝而問洪範。”案此序云:“勝殷,以箕子歸。”明既釋其囚,即以歸之,不令其走去而後來朝也。又朝鮮去周,路將萬里,聞其所在,然後封之,受封乃朝,必歷年矣,不得仍在十三祀也。\CJKunderwave{宋世家}云:“既作\CJKunderwave{洪範},武王乃封箕子於朝鮮。”得其實也。 \par}

{\noindent\shu\zihao{5}\fzkt “武王”至“洪範”。正義曰:武王伐殷,既勝,殺受,立其子武庚為殷後,以箕子歸鎬京,訪以天道,箕子為陳天地之大法,敘述其事,作\CJKunderwave{洪範}。此惟當言“箕子歸”耳,乃言“殺受,立武庚”者,序自相顧為文。上\CJKunderwave{武成}序云:“武王伐紂”,故此言勝之,下\CJKunderwave{微子之命}序雲“黜殷命,殺武庚”,故此言立之,敘言此以順上下也。 \par}

洪範\footnote{洪,大。範,法也。言天地之大法。}


{\noindent\zhuan\zihao{6}\fzbyks 傳“洪大”至“大法”。正義曰:“洪,大”、“範,法”皆\CJKunderwave{釋詁}文。 \par}

{\noindent\shu\zihao{5}\fzkt “洪範”。正義曰:此經開源於首,覆更演說,非復一問一答之勢,必是箕子自為之也。發首二句,自記被問之年,自“王乃言”至“彝倫攸敘”,王問之辭。自“箕子乃言”至“彝倫攸敘”,言\CJKunderline{禹}得九疇之由。自“初一曰”至“威用六極”,言\CJKunderline{禹}第敘九疇之次。自“一五行”已下,箕子更條說九疇之義。此條說者,當時亦以對王,更復退而修撰,定其文辭,使成典教耳。 \par}

惟十有三祀,王訪於箕子。\footnote{商曰祀,箕子稱祀,不忘本。此年四月歸宗,周先告武成,次問天道。}王乃言曰:“嗚呼!箕子,惟天陰騭下民,相協厥居,\footnote{騭,定也。天不言而默定下民,是助合其居,使有常生之資。○陰,默也,馬云:“覆也。”騭,之逸反,馬云:“升也。升猶舉也,舉猶生也。”相,息亮反,助也。}我不知其彝倫攸敘。”\footnote{言我不知天所以定民之常道理次敘。問何由。○彝,以之反。}


{\noindent\zhuan\zihao{6}\fzbyks 傳“商曰”至“天道”。正義曰:“商曰祀,周曰年”,\CJKunderwave{釋天}文。案此\CJKunderwave{周書}也,\CJKunderwave{泰誓}稱“年”,此獨稱“祀”,故解之“箕子稱祀,不忘本”也。此篇箕子所作,箕子商人,故記傳引此篇者,皆雲“\CJKunderwave{商書}曰”,是箕子自作明矣。序言“歸,作\CJKunderwave{洪範}”,似歸即作之,嫌在\CJKunderwave{武成}之前,故云“此年四月歸宗周,先告武成,次問天道”,以次在\CJKunderwave{武成}之後,故知“先告武成”也。 \par}

{\noindent\zhuan\zihao{6}\fzbyks 傳“騭定”至“之資”。正義曰:傳以“騭”即質也,質訓為成,成亦定義,故為定也。言民是上天所生,形神天之所授,故“天不言而默定下民”。群生受氣流形,各有性靈心識,下民不知其然,是天默定也。相,助也。協,和也。“助合其居”者,言民有其心,天佑助之,令其諧合其生。出言是非,立行得失,衣食之用,動止之宜,無不稟諸上,天乃得諧合。失道則死,合道則生,言天非徒賦命於人,授以形體心識,乃復佑助諧合其居業,使有常生之資。九疇施之於民,皆是天助之事也。此問答皆言“乃”者,以天道之大,沈吟乃問,思慮乃答。宣八年\CJKunderwave{公羊傳}曰:“乃,緩辭也。”王肅以“陰騭下民”一句為天事,“相協”以下為民事,注云:“陰,深也。言天深定下民,與之五常之性,王者當助天和合其居所行天之性,我不知常道倫理所以次敘,是問承天順民,何所由。”與孔異也。 \par}

{\noindent\shu\zihao{5}\fzkt “惟十”至“攸敘”。正義曰:此箕子陳王問已之年,被問之事。惟文王受命十有三祀,武王訪問於箕子,即陳其問辭,王乃言曰:“嗚呼!箕子,此上天不言而默定下民,佑助諧合其安居,使有常生之資。我不知此天之定民常道所以次敘。”問天意何由也。 \par}

箕子乃言曰:“我聞在昔,\CJKunderline{鯀}陻洪水,汩陳其五行。\footnote{陻,塞。汩,亂也。治水失道,亂陳其五行。○\CJKunderline{鯀},工本反。陻音因。汩,工忽反。行,戶更反。}帝乃震怒,不畀洪範九疇,彝倫攸斁。\footnote{畀,與。斁,敗也。天動怒\CJKunderline{鯀},不與大法九疇。疇,類也。故常道所以敗。○畀,必二反,徐甫至反,注同。斁,多路反,徐同路反,敗也。}


{\noindent\zhuan\zihao{6}\fzbyks 傳“陻塞”至“五行”。正義曰:襄二十五年\CJKunderwave{左傳}說陳之伐鄭雲“井陻木刊”,謂塞其井,斬其木,是“陻”為塞也。“汩”是亂之意,故為亂也。水是五行之一,水性下流,\CJKunderline{鯀}反塞之,失水之性,水失其性,則五行皆失矣。是塞洪水為亂,陳其五行,言五行陳列皆亂也。\CJKunderwave{大禹謨}帝美\CJKunderline{禹}治水之功云:“地平天成。”傳云:“水土治曰平,五行敘曰成。”水既治,五行序,是治水失道,為亂五行也。 \par}

{\noindent\zhuan\zihao{6}\fzbyks 傳“畀與”至“以敗”。正義曰:“畀,與”,\CJKunderwave{釋詁}文。“斁,敗”,相傳訓也。以\CJKunderline{禹}得而\CJKunderline{鯀}不得,故為天動威怒\CJKunderline{鯀},不與大法九疇。“疇”是輩類之名,故為類也。言其每事自相類者有九,九者各有一章,故\CJKunderwave{漢書}謂之為九章。此謂九類,是天之常道,既不得九類,故常道所以敗也。自古以來得九疇者惟有\CJKunderline{禹}耳,未聞餘人有得之者也。若人皆得之,\CJKunderline{鯀}獨不得,可言天帝怒\CJKunderline{鯀}。餘人皆不得,獨言天怒\CJKunderline{鯀}者,以\CJKunderline{禹}由治水有功,故天賜之,\CJKunderline{鯀}亦治水而天不與,以\CJKunderline{鯀}\CJKunderline{禹}俱是治水,父不得而子得之,所以彰\CJKunderline{禹}之聖當於天心,故舉\CJKunderline{鯀}以彰\CJKunderline{禹}也。 \par}

\CJKunderline{鯀}則殛死,\CJKunderline{禹}乃嗣興,\footnote{放\CJKunderline{鯀}至死不赦。嗣,繼也。廢父興子,堯舜之道。○殛,紀力反,本或作極,音同。}天乃錫\CJKunderline{禹}洪範九疇,彝倫攸敘。\footnote{天與\CJKunderline{禹}洛出書,神龜負文而出,列於背,有數至於九。\CJKunderline{禹}遂因而第之,以成九類,常道所以次敘。○錫,星曆反。}

{\noindent\zhuan\zihao{6}\fzbyks 傳“放\CJKunderline{鯀}”至“之道”。正義曰:傳嫌“殛”謂被誅殺,故辨之雲“放\CJKunderline{鯀}至死不赦”也。“嗣,繼”,\CJKunderwave{釋詁}文。三代以還,父罪子廢,故云“廢父興子,堯舜之道”。賞罰各從其實,為天下之至公也。 \par}

{\noindent\zhuan\zihao{6}\fzbyks 傳“天與”至“次敘”。正義曰:\CJKunderwave{易·繫辭}云:“河出圖,洛出書,聖人則之。”九類各有文字,即是書也。而云“天乃錫\CJKunderline{禹}”,知此天與\CJKunderline{禹}者即是\CJKunderwave{洛書}也。\CJKunderwave{漢書·五行志}:“劉歆以為伏羲系天而王,河出圖,則而畫之,八卦是也。\CJKunderline{禹}治洪水,錫\CJKunderwave{洛書},法而陳之,\CJKunderwave{洪範}是也。”先達共為此說,龜負\CJKunderwave{洛書},經無其事,\CJKunderwave{中候}及諸緯多說黃\CJKunderline{帝堯}、舜\CJKunderline{禹}、湯文武受圖書之事,皆雲龍負圖,龜負書。緯候之書,不知誰作,通人討核,謂偽起哀平,雖復前漢之末,始有此書,以前學者必相傳此說,故孔以九類是神龜負文而出,列於背,有數從一而至於九。\CJKunderline{禹}見其文,遂因而第之,以成此九類法也。此九類陳而行之,常道所以得次敘也。言\CJKunderline{禹}第之者,以天神言語,必當簡要,不應曲有次第。丁寧若此,故以為\CJKunderline{禹}次第之。\CJKunderline{禹}既第之,當有成法可傳,應人盡知之,而武王獨問箕子者,\CJKunderwave{五行志}云:“聖人行其道而寶其真,降及於殷,箕子在父師之位而典之。周既克殷,以箕子歸周,武王親虛已而問焉。”言箕子典其事,故武王特問之,其義或當然也。若然,大禹既得九類,常道始有次敘,未有\CJKunderwave{洛書}之前,常道所以不亂者,世有澆淳,教有疏密,三皇已前,無文亦治,何止無\CJKunderwave{洛書}也。但既得九類以後,聖王法而行之,從之則治,違之則亂,故此說常道攸敘攸斁由\CJKunderwave{洛書}耳。 \par}

{\noindent\shu\zihao{5}\fzkt “箕子”至“攸敘”。正義曰:箕子乃言,答王曰:“我聞在昔,\CJKunderline{鯀}障塞洪水,治水失道,是乃亂陳其五行而逆天道也。天帝乃動其威怒,不與\CJKunderline{鯀}大法九類,天之常道所以敗也。\CJKunderline{鯀}則放殛,至死不赦。\CJKunderline{禹}以聖德繼父而興,代治洪水,決道使通,天乃賜\CJKunderline{禹}大法九類,天之常道所以得其次敘。”此說其得九類之由也。 \par}

“初一曰五行,\footnote{九類,類一章,以五行為始。}次二曰敬用五事,\footnote{五事在身,用之必敬乃善。}次三曰農用八政,\footnote{農,厚也,厚用之政乃成。○農,馬云:“食為八政之首,故以農名之。”}次四曰協用五紀,\footnote{協,和也,和天時使得正用五紀。}次五曰建用皇極,\footnote{皇,大。極,中也。凡立事當用大中之道。}次六曰乂用三德,\footnote{治民必用剛柔正直之三德。}次七曰明用稽疑,\footnote{明用卜筮考疑之事。}


{\noindent\zhuan\zihao{6}\fzbyks 傳“農厚”至“乃成”。正義曰:\CJKunderline{鄭玄}云:“農讀為醲。”則“農”是醲意,故為厚也。政施於民,善不厭深,故“厚用之政乃成”也。張晏、王肅皆言“農,食之本也。食為八政之首,故以農言之”。然則農用止為一食,不兼八事,非上下之例,故傳不然。“八政”、“三德”總是治民,但“政”是被物之名,“德”是在己之稱,故分為二疇也。 \par}

{\noindent\zhuan\zihao{6}\fzbyks 傳“協和”至“五紀”。正義曰:“協,和”,\CJKunderwave{釋詁}文。天是積氣,其狀無形,列宿四方,為天之限。天左行,晝夜一周。日月右行,日遲月疾。周天三百六十五度有餘,日則日行一度,月則日行十三度有餘,日月行於星辰,乃為天之歷數。和此天時,令不差錯,使行得正用五紀也。日月逆天道而行,其行又有遲疾,故須調和之。 \par}

{\noindent\zhuan\zihao{6}\fzbyks 傳“皇大”至“之道”。正義曰:“皇,大”,\CJKunderwave{釋詁}文。“極”之為中,常訓也。凡所立事,王者所行皆是,無得過與不及,常用大中之道也。\CJKunderwave{詩}雲“莫匪爾極”,\CJKunderwave{周禮}“以為民極”,\CJKunderwave{論語}“允執其中”,皆謂用大中也。 \par}

次八曰念用庶徵,次九曰鄉用五福,威用六極。\footnote{言天所以鄉勸人用五福,所以威沮人用六極。此已上\CJKunderline{禹}所第敘。○向,許亮反,又許兩反。沮,在汝反。此已上,時掌反。\CJKunderline{禹}所第敘,馬云:“從‘五行’已下至‘六極’,\CJKunderwave{洛書}文也。\CJKunderwave{漢書·五行志}以‘初一’已下皆\CJKunderwave{洛書}文也。”}

{\noindent\zhuan\zihao{6}\fzbyks 傳“言天”至“第敘”。正義曰:“貧”、“弱”等六者,皆謂窮極惡事,故目之“六極”也。“福”者人之所慕,皆鄉望之。“極”者人之所惡,皆畏懼之。“勸”,勉也,勉之為善。“沮”止也,止其為惡。福、極皆上天為之,言天所以鄉望勸勉人用五福,所以畏懼沮止人用六極,自“初一曰”已下至此“六極”已上,皆是\CJKunderline{禹}所次第而敘之。下文更將此九類而演說之,知此九者皆\CJKunderline{禹}所第也。\CJKunderline{禹}為此次者,蓋以五行世所行用,是諸事之本,故“五行”為初也。發見於人則為五事,故“五事”為二也。正身而後及人,施人乃名為政,故“八政”為三也。施人之政,用天之道,故“五紀”為四也。順天佈政,則得大中,故“皇極”為五也。欲求大中,隨德是任,故“三德”為六也。政雖在德,事必有疑,故“稽疑”為七也。行事在於政,得失應於天,故“庶徵”為八也。天監在下,善惡必報,休咎驗於時氣,禍福加於人身,故“五福”、“六極”為九也。“皇極”居中者,總包上下,故“皇極”傳雲“大中之道”。大立其有中,謂行九疇之義是也。“福”、“極”處末者,顧氏云:“前八事俱得,五福歸之。前八事俱失,六極臻之。故福極處末也。”發首言“初一”,其末不言“終九”者,數必以一為始,其九非數之終,故從上言“次”而不言“終”也。五行不言“用”者,五行萬物之本,天地百物莫不用之,不嫌非用也。傳於“五福”、“六極”言天用者,以前並是人君所用,五福六極受之於天,故言天用。傳言此“\CJKunderline{禹}所第敘”,不知\CJKunderwave{洛書}本有幾字。\CJKunderwave{五行志}悉載此一章,乃云:“凡此六十五字,皆\CJKunderwave{洛書}本文。”計天言簡要,必無次第之數。上傳雲“\CJKunderline{禹}因而第之”,則孔以第是\CJKunderline{禹}之所為,“初一曰”等二十七字必是\CJKunderline{禹}加之也。其“敬用”、“農用”等一十八字,大劉及顧氏以為龜背先有總三十八字。小劉以為“敬用”等亦\CJKunderline{禹}所第敘,其龜文惟有二十字。並無明據,未知孰是,故兩存焉。“皇極”不言數者,以總該九疇,理兼萬事,非局數能盡故也。“稽疑”不言數者,以卜五筮二,共成為七,若舉卜不得兼筮,舉筮不得兼卜,且疑事既眾,不可以數總之故也。“庶徵”不言數者,以“庶徵”得為五休,失為五咎,若舉休不兼咎,舉咎不兼休,若休咎並言,便為十事,本是五物,不可言十也。然“五福”、“六極”所以善惡皆言者,以沮勸在下,故丁寧明言善惡也。且“庶徵”雖有休咎,皆以念慮包之。“福”、“極”鄉威相反,不可一言為目,故別為文焉。知“五福”、“六極”非各分為疇,所以其為一者,蓋以龜文“福”、“極”相近一處,故\CJKunderline{禹}第之總為一疇。等行五事,所以福五而極六者,大劉以為“皇極”若得,則分散總為五福,若失則不能為五事之主,與五事並列其咎弱,故為六也。猶\CJKunderwave{詩}平王以後與諸侯並列同為國風焉。咎徵有五而極有六者,\CJKunderwave{五行傳}云:“皇之不極,厥罰常陰。”即與咎徵“常雨”相類,故以“常雨”包之為五也。 \par}

{\noindent\shu\zihao{5}\fzkt “初一”至“六極”。正義曰:天所賜\CJKunderline{禹}大法九類者,初一曰五材氣性流行,次二曰敬用在身五種之行事,次三曰厚用接物八品之政教,次四曰和用天象五物之綱紀,次五曰立治用大為中正之道,次六曰治民用三等之德,次七曰明用小筮以考疑事,次八曰念用天時眾氣之應驗,次九曰鄉勸人用五福,威沮人用六極。此九類之事也。 \par}

“一,五行。一曰水,二曰火,三曰木,四曰金,五曰土。\footnote{皆其生數。}


{\noindent\zhuan\zihao{6}\fzbyks 傳“皆其生數”。正義曰:\CJKunderwave{易·繫辭}曰:“天一,地二,天三,地四,天五,地六,天七,地八,天九,地十。”此即是五行生成之數。天一生水,地二生火,天三生木,地四生金,天五生土,此其生數也。如此則陽無匹,陰無耦,故地六成水,天七成火,地八成木,天九成金,地十成土,於是陰陽各有匹偶,而物得成焉,故謂之成數也。\CJKunderwave{易·繫辭}又曰“天數五,地數五,五位相得而各有合,此所以成變化而行鬼神”,謂此也。又數之所起,起於陰陽。陰陽往來,在於日道。十一月冬至日南極,陽來而陰往。冬,水位也,以一陽生為水數。五月夏至日北極,陰進而陽退。夏,火位也,當以一陰生為火數。但陰不名奇,數必以偶,故以六月二陰生為火數也。是故\CJKunderwave{易說}稱幹貞於十一月子,坤貞於六月未,而皆左行,由此也。冬至以及於夏至,當為陽來。正月為春木位也,三陽已生,故三為木數。夏至以及冬至,當為陰進。八月為秋金位也,四陰已生,故四為金數。三月春之季,四季土位也,五陽已生,故五為土數,此其生數之由也。又萬物之本,有生於無,者生於微,及其成形,亦以微著為漸。五行先後,亦以微著為次。五行之體,水最微,為一。火漸著,為二。木形實,為三。金體固,為四。土質大,為五。亦是次之宜。大劉與顧氏皆以為水火木金,得土數而成,故水成數六,火成數七,木成數八,金成數九,土成數十。義亦然也。 \par}

水曰潤下,火曰炎上,\footnote{言其自然之常性。○炎,榮鉗反。上,時掌反,又如字,下同。}木曰曲直,金曰從革,\footnote{木可以揉曲直,金可以改更。○揉,如酉反。}土爰稼穡。\footnote{種曰稼,斂曰穡。土可以種,可以斂。}


{\noindent\zhuan\zihao{6}\fzbyks 傳“言其自然之常性”。正義曰:\CJKunderwave{易·文言}云:“水流溼,火就燥。”王肅曰:“水之性潤萬物而退下,火之性炎盛而升上。”是“潤下”、“炎上”,言其自然之本性。 \par}

{\noindent\zhuan\zihao{6}\fzbyks 傳“木可”至“改更”。正義曰:此亦言其性也,“揉曲直”者,為器有須曲直也。“可改更”者,可銷鑄以為器也。木可以揉令曲直,金可以從人改更,言其可為人用之意也。由此而觀,水則潤下,可用以灌溉;火則炎上,可用以炊爨,亦可知也。水既純陰,故潤下趣陰。火是純陽,故炎上趣陽。木金陰陽相雜,故可曲直改更也。 \par}

{\noindent\zhuan\zihao{6}\fzbyks 傳“種曰”至“以斂”。正義曰:\CJKunderline{鄭玄}\CJKunderwave{周禮注}云:“種穀曰稼,若嫁女之有所生。”然則“穡”是惜也,言聚畜之可惜也。共為治田之事,分為“種”、“斂”二名耳。土上所為,故為土性。上文“潤下”、“炎上”、“曲直”、“從革”,即是水火木金體有本性。其稼穡以人事為名,非是土之本性,生物是土之本性,其稼穡非土本性也。“爰”亦“曰”也,變“曰”言“爰”,以見此異也。“六府”以“土”、“谷”為二,由其體異故也。 \par}

潤下作咸,\footnote{水滷所生。○咸音咸。滷音魯。}炎上作苦,\footnote{焦氣之味。}曲直作酸,\footnote{木實之性。}從革作辛,\footnote{金之氣味。}稼穡作甘。\footnote{甘味生於百穀。五行以下,箕子所陳。}

{\noindent\zhuan\zihao{6}\fzbyks 傳“水滷所生”。正義曰:水性本甘,久浸其地,變而為滷,滷味乃咸。\CJKunderwave{說文}云:“滷,西方咸地。東方謂之斥,西方謂之滷。”\CJKunderwave{禹貢}云:“海濱廣斥。”是海浸其旁地,使之咸也。\CJKunderwave{月令·冬}雲“其味咸,其臭朽”是也。土言“曰”者,言其本性。此言“作”者,從其發見。指其體則稱“曰”,致其類即言“作”。下“五事”、“庶徵”言“曰”、“作”者,義亦然也。 \par}

{\noindent\zhuan\zihao{6}\fzbyks 傳“焦氣之味”。正義曰:火性炎上,焚物則焦,焦是苦氣。\CJKunderwave{月令·夏}雲“其臭焦,其味苦”,苦為焦味,故云“焦氣之味”也。臭之曰“氣”,在口曰“味”。 \par}

{\noindent\zhuan\zihao{6}\fzbyks 傳“木實之性”。正義曰:木生子實,其味多酸,五果之味雖殊,其為酸一也,是木實之性然也。\CJKunderwave{月令·春}雲“其味酸,其臭羶”是也。 \par}

{\noindent\zhuan\zihao{6}\fzbyks 傳“金之氣味”。正義曰:金之在火,別有腥氣,非苦非酸,其味近辛,故辛為金之氣味。\CJKunderwave{月令·秋}雲“其位辛,其臭腥”是也。 \par}

{\noindent\zhuan\zihao{6}\fzbyks 傳“甘味生於百穀”。正義曰:“甘味生於百穀”,谷是土之所生,故甘為土之味也。\CJKunderwave{月令·中央}雲“其味甘,其臭香”是也。 \par}

{\noindent\shu\zihao{5}\fzkt “一五行”至“作甘”。正義曰:此以下箕子所演陳\CJKunderline{禹}所第疇名於上,條列說以成之。此章所演,文有三重,第一言其名次,第二言其體性,第三言其氣味,言五者性異而味別,各為人之用。\CJKunderwave{書傳}云:“水火者百姓之所飲食也,金木者百姓之所興作也,土者萬物之所資生也。是為人用。”“五行”即五材也,襄二十七年\CJKunderwave{左傳}雲“天生五材,民並用之”,言五者各有材幹也。謂之“行”者,若在天則五氣流行,在地世所行用也。 \par}

“二,五事。一曰貌,\footnote{容儀。○貌,本亦作䫉。}二曰言,\footnote{詞章。}三曰視,\footnote{觀正。○視,常止反,徐市止反。}四曰聽,\footnote{察是非。}


{\noindent\zhuan\zihao{6}\fzbyks 傳“察是非”。正義曰:此五事皆有是非,\CJKunderwave{論語}云:“非禮勿視,非禮勿聽,非禮勿言,非禮勿動。”又引\CJKunderwave{詩}云:“思無邪。”故此五事皆有是非也。此經歷言五名,名非善惡之稱,但為之有善有惡,傳皆以是辭釋之。“貌”者言其動有容儀也,“言”道其語有辭章也,“視”者言其觀正不觀邪也,“聽”者受人言察是非也,“思”者心慮所行使行得中也。傳於“聽”雲“察是非”,明五者皆有是非也,所為者為正不為邪也。於“視”不言“視邪正”,於“聽”言“察是非”,亦所以互相明也。 \par}

五曰思,\footnote{心慮所行。○思如字,徐息吏反,下同。}貌曰恭,\footnote{儼恪。○儼,魚檢反。}言曰從,\footnote{是則可從。}視曰明,\footnote{必清審。}聽曰聰,\footnote{必微諦。諦音帝。}思曰睿。\footnote{必通於微。睿,悅歲反,馬云:“通也。”}


{\noindent\zhuan\zihao{6}\fzbyks 傳“必通於微”。正義曰:此一重言敬用之事。貌戒惰容,故“恭”為儼恪。\CJKunderwave{曲禮}曰:“儼若思。”“儼”是嚴正之貌也。“恪”,敬也,貌當嚴正而莊敬也。言非理則人違之,故言是則可從也。視必明於善惡,故必清徹而審察也。聽當別彼是非,必微妙而審諦也。王肅云:“睿,通也。思慮苦其不深,故必深思使通於微也。”此皆敬用使然,故經以善事明之。\CJKunderline{鄭玄}云:“此恭、明、聰、睿行之於我身,其從則是彼人從我,以與上下違者,我是而彼從,亦我所為不乖倒也。”此據人主為文,皆是人主之事,\CJKunderwave{說命}雲“接下思恭,視遠惟明,聽德惟聰”,即此是也。 \par}

恭作肅,\footnote{心敬。}從作乂,\footnote{可以治。}明作晢,\footnote{照了。○晢,之舌反,徐丁列反,又之世反。}聰作謀,\footnote{所謀必成當。○當,丁浪反。}睿作聖。\footnote{於事無不通謂之聖。}

{\noindent\zhuan\zihao{6}\fzbyks 傳“於事”至“之聖”。正義曰:此一重言所致之事也。恭在貌而敬在心,人有心慢而貌恭,必當緣恭以致敬,故貌恭作心敬也。下從上則國治,故人主言必從,其國可以治也。視能清審,則照了物情,故視明致照晢也。聽聰則知其是非,從其是為謀必當,故聽聰致善謀也。睿、聖俱是通名,聖大而睿小,緣其能通微,事事無不通,因睿以作聖也。\CJKunderline{鄭玄}\CJKunderwave{周禮注}云:“聖通而先識也。”是言識事在於眾物之先,無所不通,以是名之為聖。聖是智之上,通之大也。此言人主行其小而致其大,皆是人主之事也。鄭云:“皆謂其政所致也。君貌恭則臣禮肅,君言從則臣職治,君視明則臣照晢,君聽聰則臣進謀,君思睿則臣賢智。”鄭意謂此所致皆是君致臣也。案“庶徵”之意,休徵、咎徵皆肅、乂所致,若肅、乂、明、聰皆是臣事,則休、咎之所致,悉皆不由君矣。又聖大而睿小,若君睿而致臣聖,則臣皆上於君矣,何不然之甚乎!“晢”字王肅及\CJKunderwave{漢書·五行志}皆云:“晢,智也。”定本作“晢”,則讀為哲。 \par}

{\noindent\shu\zihao{5}\fzkt “二五”至“作聖”。正義曰:此章所演亦為三重,第一言其所名,第二言其所用,第三言其所致。“貌”是容儀,舉身之大名也,“言”是口之所出,“視”是目之所見,“聽”是耳之所聞,“思”是心之所慮,一人之上有此五事也。貌必須恭,言必可從,視必當明,聽必當聰,思必當通於微密也。此一重即是敬用之事。貌能恭,則心肅敬也。言可從,則政必治也。視能明,則所見照晢也。聽能聰,則所謀必當也。思通微,則事無不通,乃成聖也。此一重言其所致之事。\CJKunderwave{洪範}本體與人主作法,皆據人主為說。貌總身也,口言之,目視之,耳聽之,心慮之,人主始於敬身,終通萬事,此五事為天下之本也。五事為此次者,鄭云:“此數本諸陰陽,昭明人相見之次也。”\CJKunderwave{五行傳}曰:“貌屬木,言屬金,視屬火,聽屬水,思屬土。”\CJKunderwave{五行傳}伏生之書也。孔於太戊桑谷之下雲“七日大拱,貌不恭之罰”,高宗雊雉之下雲“耳不聰之異”,皆\CJKunderwave{書傳}之文也。孔取\CJKunderwave{書傳}為說,則此次之意亦當如\CJKunderwave{書傳}也。木有華葉之容,故貌屬木。言之決斷若金之斬割,故言屬金。火外光,故視屬火。水內明,故聽屬水。土安靜而萬物生,心思慮而萬事成,故思屬土。又於\CJKunderwave{易}東方震為足,足所以動容貌也。西方兌為口,口出言也。南方離為目,目視物也。北方坎為耳,耳聽聲也。土在內,猶思在心。亦是五屬之義也。 \par}

“三,八政。一曰食,\footnote{勤農業。}二曰貨,\footnote{寶用物。}三曰祀,\footnote{敬鬼神以成教。}四曰司空,\footnote{主空土以居民。}五曰司徒,\footnote{主徒眾,教以禮義。}六曰司寇,\footnote{主奸盜,使無縱。○縱,子用反,或作從,音同。}七曰賓,\footnote{禮賓客,無不敬。}八曰師。\footnote{簡師所任必良,士卒必練。○卒,子忽反。}

{\noindent\zhuan\zihao{6}\fzbyks 傳“寶用物”。正義曰:“貨”者,金玉布帛之總名,皆為人用,故為“用物”。\CJKunderwave{旅獒}雲“不貴異物賤用物”是也。食則勤農以求之,衣則蠶績以求之,但貨非獨衣,不可指言求處,故云得而寶愛之。\CJKunderwave{孝經}云:“謹身節用。”\CJKunderwave{詩序}云:“儉以足用。”是寶物也。 \par}

{\noindent\zhuan\zihao{6}\fzbyks 傳“主空土以居民”。正義曰:\CJKunderwave{周官}篇云:“司空掌邦土,居四民,時地利。司徒掌邦教,敷五典,擾兆民。司寇掌邦禁,詰奸慝,刑暴亂。”\CJKunderwave{周禮}司徒教以禮義,司寇無縱罪人,其文具矣。 \par}

{\noindent\zhuan\zihao{6}\fzbyks 傳“簡師”至“必練”。正義曰:經言“賓”、“師”,當有賓師之法,故傳以“禮賓客,無不敬”,教民待賓客相往來也。“師”者,眾之通名,必當選人為之,故傳言“簡師”,選人為師也。“所任必良”,任良將也。“士卒必練”,“練”謂教習使知義,若練金使精也。\CJKunderwave{論語}:“以不教民戰,是謂棄之。”是士卒必須練也。 \par}

{\noindent\shu\zihao{5}\fzkt “三八政”至“曰師”。正義曰:“八政”者,人主施政教於民有八事也。一曰食,教民使勤農業也。二曰貨,教民使求資用也。三曰祀,教民使敬鬼神也。四曰司空之官,主空土以居民也。五曰司徒之官,教眾民以禮義也。六曰司寇之官,詰治民之奸盜也。七曰賓,教民以禮待賓客,相往來也。八曰師,立師防寇賊,以安保民也。八政如此次者,人不食則死,食於人最急,故食為先也。有食又須衣貨為人之用,故“貨”為二也。所以得食貨,乃是明靈祐之,人當敬事鬼神,故“祀”為三也。足衣食、祭鬼神,必當有所安居,司空主居民,故“司空”為四也。雖有所安居,非禮義不立,司徒教以禮義,故“司徒”為五也。雖有禮義之教,而無刑殺之法,則彊弱相陵,司寇主奸盜,故“司寇”為六也。民不往來,則無相親之好,故“賓”為七也。寇賊為害,則民不安居,故“師”為八也。此用於民緩急而為次也。“食”、“貨”、“祀”、“賓”、“師”指事為之名,三卿舉官為名者,三官所主事多,若以一事為名,則所掌不盡,故舉官名以見義。\CJKunderline{鄭玄}云:“此數本諸其職先後之宜也。食謂掌民食之官,若后稷者也。貨掌金帛之官,若\CJKunderwave{周禮}司貨賄是也。祀掌祭祀之官,若宗伯者也。司空掌居民之官。司徒掌教民之官也。司寇掌詰盜賊之官。賓掌諸侯朝覲之官,\CJKunderwave{周禮}大行人是也。師掌軍旅之官,若司馬也。”王肅云:“賓掌賓客之官也。”即如鄭、王之說,自可皆舉官名,何獨三事舉官也?八政主以教民,非謂公家之事,司貨賄掌公家貨賄,大行人掌王之賓客,若其事如\CJKunderwave{周禮},皆掌王家之事,非復施民之政,何以謂之“政”乎?且司馬在上,司空在下,今司空在四,司馬在八,非取職之先後也。 \par}

“四,五紀。一曰歲,\footnote{所以紀四時。}二曰月,\footnote{所以紀一月。}三曰日,\footnote{紀一日。}四曰星辰,\footnote{二十八宿迭見以敘氣節,十二辰以紀日月所會。○宿音秀。迭,田節反。見,賢遍反。}五曰歷數。\footnote{歷數節氣之度以為歷,敬授民時。}

{\noindent\zhuan\zihao{6}\fzbyks 傳“二十”至“所會”。正義曰:二十八宿,佈於四方,隨天轉運,昏明迭見。\CJKunderwave{月令}十二月皆紀昏旦所中之星。若\CJKunderwave{月令}孟春昏參中,旦尾中;仲春昏弧中,旦建星中;季春昏七星中,旦牽牛中;孟夏昏翼中,旦婺女中;仲夏昏亢中,旦危中;季夏昏心中,旦奎中;孟秋昏建星中,旦畢中;仲秋昏牽牛中,旦觜觿中;季秋昏虛中,旦柳中;孟冬昏危中,旦七星中;仲冬昏東壁中,旦軫中;季冬昏婁中,旦氐中;皆所以敘氣節也。氣節者,一歲三百六十五日有餘,分為十二月,有二十四氣。一為節氣,謂月初也。一為中氣,謂月半也。以彼迭見之星,敘此月之節氣也。昭七年\CJKunderwave{左傳}晉侯問士文伯曰:“多語寡人辰而莫同,何謂也?”對曰:“日月之會是謂辰。”“會”者,日行遲,月行疾,俱循天度而右行,二十九日過半月行一周天,又前及日而與日會,因謂會處為辰。則\CJKunderwave{月令}孟春日在營室,仲春日在奎,季春日在胃,孟夏日在畢,仲夏日在東井,季夏日在柳,孟秋日在翼,仲秋日在角,季秋日在房,孟冬日在尾,仲冬日在鬥,季冬日在婺女,十二會以為十二辰。“辰”即子醜寅卯之謂也,十二辰所以紀日月之會處也。鄭以為“星,五星也”。然五星所行,下民不以為候,故傳不以“星”為五星也。 \par}

{\noindent\zhuan\zihao{6}\fzbyks 傳“歷數”至“民時”。正義曰:天以積氣無形,二十八宿分之為限,每宿各有度數,合成三百六十五度有餘。日月右行,循此宿度。日行一度,月行十三度有餘,二十九日過半而月一周與日會,每於一會謂之一月,是一歲為十二月,仍有餘十一日。為日行天未周,故置閏以充足。若均分天度以為十二次,則每次三十度有餘。一次之內有節氣、中氣,次之所管,其度多每月之所統,其日入月朔,參差不及,節氣不得在月朔,中氣不得在月半。故聖人歷數此節氣之度,使知氣所在,既得氣在之日,以為一歲之歷,所以敬授民時。王肅院“日月星辰所行,布而數之,所以紀度數”是也。“歲”、“月”、“日”、“星”傳皆言“紀”,“歷數”不言“紀”者,歷數數上四事為紀,所紀非獨一事,故傳不得言“紀”。但成彼四事為紀,故通數以為五耳。 \par}

{\noindent\shu\zihao{5}\fzkt “四五紀”至“歷數”。正義曰:“五紀”者,五事為天時之經紀也。一曰歲,從冬至以及明年冬至為一歲,所以紀四時也。二曰月,從朔至晦,大月三十日,小月二十九日,所以紀一月也。三曰日,從夜半以至明日夜半周十二辰為一日,所以紀一日也。四曰星辰,星謂二十八宿,昏明迭見;辰謂日月別行,會於宿度,從子至於醜為十二辰。星以紀節氣早晚,辰以紀日月所會處也。五曰歷數,算日月行道所歷,計氣朔早晚之數,所以為一歲之歷。凡此五者,皆所以紀天時,故謂之“五紀”也。五紀不言“時”者,以歲月氣節正而四時亦自正,時隨月變,非歷所推,故不言“時”也。五紀為此節者,歲統月,月統日,星辰見於天,其曰“歷數”,總歷四者,故歲為始,歷為終也。 \par}

“五,皇極。皇建其有極,\footnote{大中之道,大立其有中,謂行九疇之義。}斂時五福,用敷錫厥庶民。\footnote{斂是五福之道以為教,用布與眾民使慕之。}


{\noindent\zhuan\zihao{6}\fzbyks 傳“大中”至“之義”。正義曰:此疇以“大中”為名,故演其大中之義。“大中之道,大立其有中”,欲使人主先自立其大中,乃以大中教民也。凡行不迂僻則謂之“中”,\CJKunderwave{中庸}所謂“從容中道”,\CJKunderwave{論語}“允執其中”,皆謂此也。九疇為德,皆求大中,是為善之總,故云“謂行九疇之義”,言九疇之義皆求得中,非獨此疇求大中也。此大中是人君之大行,故特敘以為一疇耳。 \par}

{\noindent\zhuan\zihao{6}\fzbyks 傳“斂是”至“慕之”。正義曰:五福生於五事,五事得中,則福報之。“斂是五福之道”,指其敬用五事也。用五事得中,則各得其福,其福乃散於五處,不相集聚。若能五事皆敬,則五福集來歸之。普敬五事,則是斂聚五福之道。以此敬五事為教,布與眾民,使眾民勸慕為之。福在幽冥,無形可見,敬用五事,則能致之,“斂是五福”,正是敬用五事。不言“敬用五事以教”,而云“斂是五福以為教”者,福是善之見者,故言“福”以勸民,欲其慕而行善也。“汝”者,箕子“汝”王也。 \par}

惟時厥庶民於汝極,錫汝保極。\footnote{君上有五福之教,眾民於君取中,與君以安中之善。言從化。}凡厥庶民,無有淫朋,人無有比德,惟皇作極。\footnote{民有安中之善,則無淫過朋黨之惡、比周之德,為天下皆大為中正。○比,毗志反,注同。}

{\noindent\zhuan\zihao{6}\fzbyks 傳“君上”至“從化”。正義曰:凡人皆有善性,善不能自成,必須人君教之,乃得為善。君上有五福之教,以大中教民,眾民於君取中。“保”訓安也,既學得中,則其心安之。君以大中教民,民以大中向君,是民與君皆以大中之善。君有大中,民亦有大中,言從君化也。 \par}

{\noindent\zhuan\zihao{6}\fzbyks 傳“民有”至“中止”。正義曰:民有安中之善,非中不與為交,安中之人則無淫過朋黨之惡,無有比周之德。“朋黨”、“比周”是不中者。善多惡少,則惡亦化而為善,無復有不中之人,惟天下皆大為中正矣。 \par}

{\noindent\shu\zihao{5}\fzkt “五皇極”至“作極”。正義曰:“皇”,大也。“極”,中也。施政教,治下民,當使大得其中,無有邪僻。故演之云,大中者,人君為民之主,當大自立其有中之道,以施教於民。當先敬用五事,以斂聚五福之道,用此為教,布與眾民,使眾民慕而行之。在上能教如此,惟是其眾民皆效上所為,無不於汝人君取其中道而行。積久漸以成性,乃更與汝人君以安中之道。言皆化也。若能化如是,凡其眾民無有淫過朋黨之行,人無有惡相阿比之德,惟皆大為中正之道。言天下眾民盡得中也。 \par}

凡厥庶民,有猷有為有守,汝則念之。\footnote{民戢有道,有所為,有所執守,汝則念錄敘之。}不協於極,不罹於咎,皇則受之。\footnote{凡民之行,雖不合於中,而不罹於咎惡,皆可進用,大法受之。○罹,馬力馳反,又來多反。行,下孟反。}


{\noindent\zhuan\zihao{6}\fzbyks 傳“民戢”至“敘之”。正義曰:“戢”,斂也,因上“斂是五福”,故傳以“戢”言之。“戢”文兼下三事,民能斂德行智,能使其身有道德,其才能有所施為,用心有所執守。如此人者,汝念錄敘之,宜用之為官也。“有所為”,謂藝能也。“有執守”,謂得善事能守而勿失,言其心正不逆邪也。 \par}

{\noindent\zhuan\zihao{6}\fzbyks 傳“凡民”至“受之”。正義曰:“不合於中,不罹於咎”,謂未為大善,又無惡行,是中人已上,可勸勉有方將者也,故皆可進用,以大法受之。“大法”謂用人之法,取其所長,棄瑕錄用也。上文人君以大中教民,使天下皆為大中,此句印敝痢邦不合於中亦用之者,上文言設教耳。其實天下之大,兆民之眾,不可使皆合大中;且庶官交曠,即須任人,不可待人盡合大中,然後敘用。言各有為,不相妨害。 \par}

而康而色,曰:‘予攸好德。’汝則錫之福,\footnote{汝當安汝顏色,以謙下人。人曰:“我所好者德。”汝則與之爵祿。○好,呼報反。下,遐嫁反。}時人斯其惟皇之極。\footnote{不合於中之人,汝與之福,則是人此其惟大之中。言可勉進。}無虐煢獨而畏高明。\footnote{煢,單,無兄弟也。無子曰獨。單獨者,不侵虐之寵貴者,不枉法畏之。○無虐,馬本作亡侮。煢,岐扃反。畏如字,徐云:“鄭音威。”}

{\noindent\zhuan\zihao{6}\fzbyks 傳“汝當”至“爵祿”。正義曰:安汝顏色,以謙下人,其此不合於中人之,皆人言曰:“我所好者德也。”是有慕善之心,有方將者也。汝則與之爵祿以長進之。上句言“受之”,謂治受以,此言“與爵祿”,謂用為官也。 \par}

{\noindent\zhuan\zihao{6}\fzbyks 傳“不合”至“勉進”。正義曰:“不合於中之人”,初時未閤中也,汝與之爵祿,置之朝廷,見人為善,心必慕之,則是人此其惟大中之道,為大中之人,言可勸勉使進也。\CJKunderwave{荀卿書}曰:“蓬生麻中,不扶自直。白沙在涅,與之俱黑。”斯言信矣。此經或言“時人德”,鄭、王諸本皆無“德”字。此傳不以“德”為義,定本無“德”,疑衍字也。 \par}

{\noindent\zhuan\zihao{6}\fzbyks 傳“煢單”至“畏之”。正義曰:\CJKunderwave{詩}云:“獨行煢煢。”是為單,謂無兄弟也。“無子曰獨”,\CJKunderwave{王制}文。“高明”與“煢獨”相對,非謂才高,知寵貴之人位望高也。“不枉法畏之”,即\CJKunderwave{詩}所謂“不畏強禦”是也。此經皆是據天子,無陵虐煢獨而畏避高明寵貴者。顧氏亦以此經據人君,小劉以為據人臣,謬也。 \par}

{\noindent\shu\zihao{5}\fzkt “凡厥”至“高明”。正義曰:又說用人為官,使之大中。凡其眾民,有道德,有所為,有所執守,汝為人君則當念錄敘之,用之為官。若未能如此,雖不合於中,亦不罹於咎惡,此人可勉進,宜以取人大法則受取之。其受人之大法如何乎?汝當和安汝之顏色,以謙下人。彼欲仕者謂汝曰:“我所好者德也。”汝則與之以福祿,隨其所能,用之為官。是人庶幾必自勉進,此其惟為大中之道。又為君者無侵虐單獨而畏忌高明,高明謂貴寵之人,勿枉法畏之。如是即為大中矣。 \par}

人之有能有為,使羞其行,而邦其昌。\footnote{功能有為之士,使進其所行,汝國其昌盛。○其行,如字,徐下孟反。}凡厥正人,既富方谷,\footnote{凡其正直之人,既當以爵祿富之,又當以善道接之。}汝弗能使有好於而家,時人斯其辜。\footnote{不能使正直之人有好於國家,則是人斯其詐取罪而去。}


{\noindent\zhuan\zihao{6}\fzbyks 傳“功能”至“昌盛”。正義曰:“功能有為之士”,謂其身有才能,所為有成功,此謂已在朝廷任用者也。“使進其行”者,謂人之有善,若上知其有能有為,或以言語勞來之,或以財貨賞賜之,或更任之以大位,如是則其人喜於見知,必當行自進益,人皆漸自修進,汝國其昌盛矣。 \par}

{\noindent\zhuan\zihao{6}\fzbyks 傳“凡其”至“接之”。正義曰:“凡其正直之人”,普謂臣民有正直者。爵祿所設,正直是與。已知彼人正直,必當授之以官。“既當與爵祿富之,又當以善道接之”,言其非徒與官而已,又當數加燕賜,使得其歡心也。 \par}

{\noindent\zhuan\zihao{6}\fzbyks 傳“不能”至“而去”。正義曰:授之以官爵,加之以燕賜,喜於知己,荷君恩德,必進謀樹功,有好善於國家。若雖用為官,心不委任,禮意疏薄,更無恩紀,言不聽,計不用,必將奮衣而去,不肯久留,故言“不能使正直之人有好於國家,則是人斯其詐取罪而去”也。 \par}

於其無好德,汝雖錫之福,其作汝用咎。\footnote{於其無好德之人,汝雖與之爵祿,其為汝用惡道以敗汝善。○其為,於偽反。}

{\noindent\zhuan\zihao{6}\fzbyks 傳“於其”至“汝善”。正義曰:“無好”對“有好”,“有好”謂有善也。“無好德之人”,謂彼性不好德、好惡之人也。\CJKunderwave{論語}曰:“未見好德如好色者。”傳記言好德者多矣,故傳以“好德”言之。定本作“無惡”者,疑誤耳。不好德者性行本惡,君雖與之爵祿,不能感恩行義,其為汝臣,必用惡道以敗汝善也。\CJKunderwave{易·繫辭}云:“無咎者善補過也。”“咎”是過之別名,故為惡耳。 \par}

{\noindent\shu\zihao{5}\fzkt “人之”至“用咎”。正義曰:此又言用臣之法。人之在位者,有才能,有所為,當褒賞之,委任使進其行,汝國其將昌盛也。凡其正直之人,既以爵祿富之,又復以善道接之,使之荷恩盡力。汝若不能使正直之人有好善於汝國家,是人於此其將詐取罪而去矣。於其無好德之人,謂性行惡者,汝雖與之福,賜之爵祿,但本性既惡,必為惡行,其為汝臣,必用惡道以敗汝善。言當任善而去惡。 \par}

無偏無陂,遵王之義。\footnote{偏,不平。陂,不正。言當循先王之正義以治民。○陂音秘,舊本作頗,音普多反。}無有作好,遵王之道。無有作惡,遵王之路。\footnote{言無有亂為私好惡,動必循先王之道路。○好,呼報反。惡,烏路反,注同。}無偏無黨,王道蕩蕩。\footnote{言開闢。○闢,婢必反。}無黨無偏,王道平平。\footnote{言辯治。○平平,婢綿反。治,直吏反。}無反無側,王道正直。\footnote{言所行無反道不正,則王道平直。}會其有極,歸其有極。\footnote{言會其有中而行之,則天下皆歸其有中矣。}


{\noindent\zhuan\zihao{6}\fzbyks 傳“偏下”至“治民”。正義曰:“不平”謂高下,“不正”謂邪僻,與下“好”、“惡”、“反”、“側”其義一也。偏頗阿黨是政之大患,故箕子殷勤言耳。下傳雲“無有亂為私好私惡”者,人有私好惡則亂於正道,故傳以“亂”言之。 \par}

{\noindent\zhuan\zihao{6}\fzbyks 傳“言會”至“中矣”。正義曰:“會”謂集會,言人之將為行也,集會其有中之道而行之,行實得中,則天下皆歸其為有中矣。“天下”者,大言之。\CJKunderwave{論語}云:“一曰克己復禮,天下歸仁焉。”此意與彼同也。 \par}

{\noindent\shu\zihao{5}\fzkt “無偏”至“有極”。正義曰:更言大中之體。為人君者當無偏私,無陂曲,動循先王之正義。無有亂為私好,謬賞惡人,動循先王之正道。無有亂為私惡,濫罰善人,動循先王之正路。無偏私,無阿黨,王家所行之道蕩蕩然開闢矣。無阿黨,無偏私,王者所立之道平平然辯治矣。所行無反道,無偏側,王家之道正直矣。所行得無偏私皆正直者,會集其有中之道而行之。若其行必得中,則天下歸其中矣。言人皆謂此人為大中之人也。 \par}

曰皇極之敷言,是彝是訓,於帝其訓。\footnote{曰者,大其義,言以大中之道布陳言教,不失其常,則人皆是順矣。天且其順,而況於人乎?}凡厥庶民,極之敷言,是訓是行,以近天子之光。\footnote{凡其眾民中心之所陳言,凡順是行之,則可以近益天子之光明。○近,附近之近。}曰天子作民父母,以為天下王。\footnote{言天子佈德惠之教,為兆民之父母,是為天下所歸往,不可不務。}

{\noindent\shu\zihao{5}\fzkt “曰皇”至“下王”。正義曰:既言有中矣,為天下所歸,更美之曰,以大中之道布陳言教,不使失是常道,則民皆於是順矣。天且其順,而況於人乎?以此之故,大中為天下所歸也。又大中之道至矣,何但出於天子為貴?凡其眾民中和之心,所陳之言,謂以善言聞於上者,於是順之,於是行之,悅於民而便於政,則可近益天子之光明矣。又本人君須大中者,更美大之曰,人君於天所子,佈德惠之教,為民之父母,以是之故,為天下所歸往,由大中之道教使然。言人君不可不務大中矣。 \par}

“六,三德。一曰正直,\footnote{能正人之曲直。}二曰剛克,\footnote{剛能立事。○克,馬云:“勝也。”}三曰柔克。\footnote{和柔能治,三者皆德。}平康正直,\footnote{世平安,用正直治之。}彊弗友剛克,\footnote{友,順也。世強御不順,以剛能治之。○御,魚呂反。能治,直吏反。}燮友柔克。\footnote{燮,和也。世和順,以柔能治之。○燮,息協反。}沈潛剛克,\footnote{沈潛謂地,雖柔亦有剛,能出金石。}高明柔克。\footnote{高明謂天,言天為剛德,亦有柔克,不幹四時,喻臣當執剛以正君,君亦當執柔以納臣。}


{\noindent\zhuan\zihao{6}\fzbyks 傳“和柔”至“皆德”。正義曰:剛不恆用,有時施之,故傳言“立事”。柔則常用以治,故傳言“能治”。三德為此次者,正直在剛柔之間,故先言。二者先剛後柔,得其敘矣。王肅意與孔同。\CJKunderline{鄭玄}以為“三德,人各有一德,謂人臣也”。 \par}

{\noindent\zhuan\zihao{6}\fzbyks 傳“友順”至“治之”。正義曰:\CJKunderwave{釋訓}云:“善兄弟為友。”“友”是和順之名,故為順也。傳雲“燮,和也”,\CJKunderwave{釋詁}文。詁此三德是王者一人之德,視世而為之,故傳三者各言“世”。世平安,雖時無逆亂,而民俗未和,其下猶有曲者,須在上以正之,故世平安用正直之德治之。世有強御不順,非剛無以制之,故以剛能治之。世既和順,風俗又安,故以柔能治之。\CJKunderline{鄭玄}以為人臣各有一德,天子擇使之,注云:“安平之國,使中平守一之人治之,使不失舊職而已。國有不順孝敬之行者,則使剛能之人誅治之。其有中和之行者,則使柔能之人治之,差正之。”與孔不同。 \par}

{\noindent\zhuan\zihao{6}\fzbyks 傳“高明”至“納臣”。正義曰:\CJKunderwave{中庸}云:“博厚配地,高明配天。”高而明者惟有天耳,知“高明謂天”也。以此“高明”是天,故上傳“沈潛謂地”也。文五年\CJKunderwave{左傳}云:“天為剛德,猶不幹時。”是言天亦有柔德,不幹四時之序也。地柔而能剛,天剛而能柔,故以“喻臣當執剛以正君,君當執柔以納臣”也。 \par}

惟闢作福,惟闢作威,惟闢玉食。\footnote{言惟君得專威福,為美食。○闢,徐補亦反。玉食,張晏注\CJKunderwave{漢書}云:“玉食,珍食也”。韋昭云:“諸侯備珍異之食。”}臣無有作福作威玉食。臣之有作福作威玉食,其害於而家,兇於而國。人用側頗僻,民用僣忒。\footnote{在位不敦平,則下民僣差。○頗,普多反。僻,匹亦反。僣,子念反。忒,他得反,馬云:“惡也。”}

{\noindent\zhuan\zihao{6}\fzbyks 傳“言惟”至“美食”。正義曰:於“三德”之下說此事者,以德則隨時而用,位則不可假人,故言尊卑之分,君臣之紀,不可使臣專威福,奪君權也。衣亦不得僣君而獨言食者,人之所資,食最為重,故舉言重也。王肅云:“闢,君也。不言王者,關諸侯也,諸侯於國得專賞罰。”其義或當然也。 \par}

{\noindent\zhuan\zihao{6}\fzbyks 傳“在位”至“僣差”。正義曰:此經“福”、“威”與“食”於君每事言“闢”,於臣則並文而略之也。“作福作威”謂秉國之權,勇略震主者也。“人用側頗僻”者,謂在位小臣見彼大臣威福由己,由此之故,小臣皆附下罔上,為此側頗僻也。下民見此在位小臣秉心僻側,用此之故,下民皆不信恆,為此僣差也。言在位由大臣,下民由在位,故皆言“用”也。傳不解“家”,王肅云:“大夫稱家,言秉權之臣必滅家,復害其國也。” \par}

{\noindent\shu\zihao{5}\fzkt “六三得”至“僣忒”。正義曰:此三德者,人君之德,張弛有三也。一曰正直,言能正人之曲使直。二曰剛克,言剛強而能立事。三曰柔克,言和柔而能治。既言人主有三德,又說隨時而用之。平安之世,用正直治之。強御不順之世,用剛能治之。和順之世,用柔能治之。既言三德張弛,隨時而用,又舉天地之德,以喻君臣之交。地之德沉深而柔弱矣,而有剛,能出金石之物也。天之德高明剛強矣,而有柔,能順陰陽之氣也。以喻臣道雖柔,當執剛以正君;君道雖剛,當執柔以納臣也。既言君臣之交,剛柔遞用,更言君臣之分,貴賤有恆。惟君作福,得專賞人也。惟君作威,得專罰人也。惟君玉食,得備珍食也。為臣無得有作福作威玉食,言政當一統,權不可分也。臣之有作福作威玉食者,其必害於汝臣之家,兇於汝君之國,言將得罪喪家,且亂邦也。在位之人,用此大臣專權之故,其行側頗僻。下民用在位頗僻之故,皆言不信,而行差錯。 \par}

“七,稽疑。擇建立卜筮人,\footnote{龜曰卜,蓍曰筮。考正疑事,當選擇知卜筮人而建立之。○蓍音屍。}乃命卜筮。\footnote{建立其人,命以其職。}曰雨,曰霽,\footnote{龜兆形有似雨者,有似雨止者。○霽,子細反。}曰蒙,\footnote{蒙,陰暗。○蒙,武工反,徐亡鉤反。}曰驛,\footnote{氣洛驛不連屬。○驛音亦,注同。屬音燭。}曰克,\footnote{兆相交錯。五者卜兆之常法。}曰貞,曰悔,\footnote{內卦曰貞,外卦曰悔。}凡七。\footnote{卜筮之數。}


{\noindent\zhuan\zihao{6}\fzbyks 傳“龜曰”至“立之”。正義曰:“龜曰卜,蓍曰筮”,\CJKunderwave{曲禮}文也。考正疑事,當選擇知卜筮人而建立之。“建”亦“立”也,復言之耳。鄭、王皆以“建”、“立”為二,言將考疑事,選擇可立者,立為卜人筮人。 \par}

{\noindent\zhuan\zihao{6}\fzbyks 傳“兆相”至“常法”。正義曰:此上五者,灼龜為兆,其璺拆形狀有五種,是“卜兆之常法”也。\CJKunderwave{說文}云:“霽,雨止也。”“霽”似雨止,則“雨”似雨下。\CJKunderline{鄭玄}曰:“霽如雨止者,雲在上也。”“雺”聲近蒙,\CJKunderwave{詩}雲“零雨其蒙”,則蒙是暗之義,故以雺為兆,蒙是陰暗也。“圛”即驛也,故以為兆。“氣落驛不連屬”,“落驛”,希稀之意也。“雨”、“霽”既相對,則“蒙”、“驛”亦相對,故“驛”為落驛氣不連屬,則“雺”為氣連蒙暗也。王肅云:“圛,霍驛消減如雲陰。雺,天氣下地不應,闇冥也。”其意如孔言。\CJKunderline{鄭玄}以“圛”為明,言色澤光明也。“雺”者氣澤鬱郁冥冥也。自以“明”、“暗”相對,異於孔也。“克”謂兆相交錯。王肅云:“兆相侵入,蓋兆為二拆,其拆相交也。”\CJKunderline{鄭玄}云:“克者如雨氣色相侵入。”卜筮之事,體用難明,故先儒各以意說,未知孰得其本。今之用龜,其兆橫者為土,立者為木,斜向徑者為金,背徑者為火,因兆而細曲者曲為水,不知與此五者同異如何。此五兆不言“一曰”、“二曰”者,灼龜所遇,無先後也。 \par}

{\noindent\zhuan\zihao{6}\fzbyks 傳“內卦”至“曰悔”。正義曰:僖十五年\CJKunderwave{左傳}云,秦伯伐晉,卜徒父筮之。其卦遇蠱,蠱卦巽下艮上,說卦云,巽為風,艮為山。其佔云:“蠱之貞,風也;其悔,山也。”是內卦為貞,外卦為悔也。筮法爻從下起,故以下體為內,上體為外。下體為本,因而重之,故以下卦為貞。貞,正也,言下體是其正。\CJKunderline{鄭玄}云:“悔之言晦,晦猶終也。”晦是月之終,故以為終,言上體是其終也。下體言正,以見上體不正;上體言終,以見下體為始;二名互相明也。 \par}

卜五,佔用二,衍忒。立時人作卜筮,三人佔,則從二人之言。\footnote{立是知卜筮人,使為卜筮之事。夏殷周卜筮各異,三法並卜。從二人之言,善鈞從眾。卜筮各三人。○佔用二,馬云:“佔,筮也。”衍,以淺反。}汝則有大疑,謀及乃心,謀及卿士,謀及庶人,謀及卜筮。\footnote{將舉事而汝則有大疑,先盡汝心以謀慮之,次及卿士眾民,然後卜筮以決之。}


{\noindent\zhuan\zihao{6}\fzbyks 傳“立是”至“三人”。正義曰:此經“卜五,佔用二,衍忒”,孔不為傳。\CJKunderline{鄭玄}云:“‘卜五佔用’謂雨、霽、蒙、驛、克也,‘二衍忒’謂貞、悔也。”斷“用”從上句,“二衍忒”者,指謂筮事。王肅云:“‘卜五’者,筮短龜長,故卜多而筮少。‘佔用二’者,以貞、悔佔六爻。‘衍忒’者,當推衍其爻義以極其意。”“卜五,佔二”,其義當如王解,其“衍忒”宜總謂卜筮,皆當衍其義,極其變,非獨筮衍而卜否也。傳言“立是知卜筮人,使為卜筮之事”者,言經之此文覆述上句“立卜筮人”也。言“三人佔”,是佔此卜筮,法當有三人。\CJKunderwave{周禮}:“太卜掌三兆之法,一曰玉兆,二曰瓦兆,三曰原兆。掌三易之法,一曰\CJKunderwave{連山},二曰\CJKunderwave{歸藏},三曰\CJKunderwave{周易}。”杜子春以為“玉兆,帝顓頊之兆。瓦兆,\CJKunderline{帝堯}之兆”。又云“\CJKunderwave{連山},虙犧。\CJKunderwave{歸藏},黃帝。三兆三易皆非夏殷”。而孔意必以三代夏殷周法者,以\CJKunderwave{周禮}指言“一曰”、“二曰”,不辯時代之名。案\CJKunderwave{考工記}云,夏曰世室,殷曰重屋,周曰明堂。又\CJKunderwave{禮記·郊特牲}雲“夏收,殷冔,周冕”。皆以夏殷周三代相因,明三易亦夏殷周相因之法。子春之言,孔所不取。\CJKunderline{鄭玄}\CJKunderwave{易贊}亦云:“夏曰\CJKunderwave{連山},殷曰\CJKunderwave{歸藏}。”與孔同也。所言三兆三易,必是三代異法,故傳以為夏殷周卜筮各以三代異法,三法並卜,法有一人,故三人也。“從二人之言”者,二人為善既鈞,故從眾也。若三人之內賢智不等,雖少從賢,不從眾也。“善鈞從眾”,成六年\CJKunderwave{左傳}文。既言“三法並卜”,嫌筮不然,故又云“卜筮各三人”也。經惟言三佔從二,何知不一法而三佔,而知“三法並用”者?\CJKunderwave{金縢}云:“乃卜三龜,一習吉。”\CJKunderwave{儀禮}士喪卜葬,佔者三人,貴賤俱用三龜,知卜筮並用三代法也。 \par}

{\noindent\zhuan\zihao{6}\fzbyks 傳“將舉”至“決之”。正義曰:非有所舉,則自不卜,故云“將舉事”,事有疑,則當卜筮。人君先盡己心以謀慮之,次及卿士眾民,人謀猶不能定,然後問卜筮以決之。故先言“乃心”,後言“卜筮”也。\CJKunderline{鄭玄}云:“卿士,六卿掌事者。”然則“謀及卿士”,以卿為首耳,其大夫及士亦在焉。以下惟言“庶人”,明大夫及士寄“卿”文以見之矣。\CJKunderwave{周禮}:“小司寇掌外朝之政,以致萬民而詢焉。一曰詢國危,二曰詢國遷,三曰詢立君。”是有大疑而詢眾也。又曰“小司寇以敘進而問焉”,是謀及之也。大疑者不要是彼三詢,其謀及則同也。謀及庶人,必是大事,若小事不必詢於萬民,或謀及庶人在官者耳。\CJKunderwave{小司寇}又曰:“以三剌斷庶民獄訟之中,一曰訊群臣,二曰訊群吏,三曰訊萬民。”彼“群臣”、“群吏”分而為二,此惟言“卿士”者,彼將斷獄,令眾議然後行刑,故臣與民為三,其人主待眾議而決之;此則人主自疑,故一人主為一,又總群臣為一也。 \par}

汝則從,龜從,筮從,卿士從,庶民從,是之謂大同。\footnote{人心和順,龜筮從之,是謂大同于吉。}身其康強,子孫其逢吉。\footnote{動不違眾,故後世遇吉。○逢,馬云:“逢,大也。”}汝則從,龜從,筮從,卿士逆,庶民逆,吉。\footnote{三從二逆,中吉,亦可舉事。}卿士從,龜從,筮從,汝則逆,庶民逆,吉。\footnote{君臣不同,決之卜筮,亦中吉。}


{\noindent\zhuan\zihao{6}\fzbyks 傳“人心”至“于吉”。正義曰:人主與卿士庶民皆從,是“人心和順”也。此必臣民皆從,乃問卜筮,而進龜筮於上者,尊神物,故先言之。不在“汝則”之上者,卜當有主,故以人為先,下三事亦然。改“卜”言“龜”者,“卜”是請問之意,吉凶龜佔兆告以人,故改言“龜”也。“筮”則本是蓍名,故不須改也。 \par}

{\noindent\zhuan\zihao{6}\fzbyks 傳“動不”至“遇吉”。正義曰:物貴和同,故大同之吉,延及於後。宣三年\CJKunderwave{左傳}稱“成王定鼎,卜世三十,卜年七百”,是“後世遇吉”。 \par}

{\noindent\zhuan\zihao{6}\fzbyks 傳“三從”至“舉事”。正義曰:此與下二事,皆是三從二逆,除龜、筮以外,有汝與卿士、庶民,分三者各為一從二逆,嫌其貴賤有異,從逆或殊,故三者各以有一從為主,見其為吉同也。方論得吉,以從者為主,故次言“卿士從”,下言“庶民從”也。以從為主,故退“汝則”於下。傳解其意,卿士從者,“君臣不同”也;庶民從者,“民與上異心”也。解臣民與君異心,得其筮之意也。不言四從一逆者,吉可知,不假言之也。四從之內,雖龜筮相違,亦為吉,以其從者多也。若三從之內,龜筮相違,雖不如龜筮俱從,猶勝下龜筮相違,二從三逆。必知然者,以下傳雲“二從三逆,龜筮相違”,既計從之多少,明從多則吉。故杜預云:“龜筮同卿士之數者,是龜筮雖靈,不至越於人也。”上言“庶人”,又言“庶民”者,嫌“庶人”惟指在官者,變“人”言“民”見其同也。民人之賤,得與卿士敵者,貴者雖貴,未必謀慮長,故通以民為一,令與君臣等也。 \par}

庶民從,龜從,筮從,汝則逆,卿士逆,吉。\footnote{民與上異心,亦卜筮以決之。}汝則從,龜從,筮逆,卿士逆,庶民逆,作內吉,作外兇。\footnote{二從三逆,龜筮相違,故可以祭祀冠婚,不可以出師征伐。○冠,官喚反。}龜筮共違於人,\footnote{皆逆。}用靜吉,用作兇。\footnote{安以守常則吉,動則兇。}

{\noindent\zhuan\zihao{6}\fzbyks 傳“民與”至“決之”。正義曰:天子聖人,庶民愚賤,得為識見同者,但聖人生知,不假卜筮,垂教作訓,晦跡同凡,且庶民既眾,以眾情可否,亦得上敵於聖。故\CJKunderwave{老子}雲“聖人無常心,以百姓心為心”是也。 \par}

{\noindent\zhuan\zihao{6}\fzbyks 傳“二從”至“征伐”。正義曰:此“二從三逆”為小吉,故猶可舉事。“內”謂國內,“故可以祭祀冠婚”。“外”謂境外,“故不可以出師征伐”。征伐事大,此非大吉故也。此經“龜從,筮逆”,其筮從、龜逆為吉亦同。故傳言“龜筮相違”,見龜筮之智等也。若龜筮智等,而僖四年\CJKunderwave{左傳}雲“筮短龜長”者,於時晉獻公欲以驪姬為夫人,卜既不吉,而更令筮之,神靈不以實告,筮而得吉,必欲用之,卜人欲令公舍筮從卜,故曰“筮短龜長”,非是龜實長也。\CJKunderwave{易·繫辭}云:“蓍之德圓而神,卦之德方以智。”神以知來,智以藏往,然則知來藏往,是為極妙,雖龜之長,無以加此。聖人演筮為易,所知豈是短乎?明彼長短之說,乃是有為言耳。“此二從三逆”,以汝與龜為二從耳。卿士庶民課有一從,亦是二從,兇吉亦同,故不復設文,同可知也。若然,汝、卿士、庶民皆逆,龜、筮並從,則亦是二從三逆,而經無文者,若君與臣民皆逆,本自不問卜矣,何有龜從筮從之理也?前三從之內,龜筮既從,君與卿士庶民各有一從以配龜筮,凡有三條。若惟君與卿士從配龜為一條,或君與庶民從配龜又為一條,或卿士庶民從配龜又為一條,凡有三條。若筮從龜逆,其事亦然。二從三逆,君配龜從為一條,於經已具。卿士配龜從為二條,庶民配龜從為三條。若筮從龜逆,以人配筮,其事亦同。案\CJKunderwave{周禮·筮人}:“國之大事,先筮而後卜。”\CJKunderline{鄭玄}云:“於筮之兇則止。”何有筮逆龜從及龜筮俱違者?崔靈恩以為筮用三代之佔,若三佔之俱主兇,則止不卜,即鄭注\CJKunderwave{周禮}“筮兇則止”是也。若三佔二逆一從,兇猶不決,雖有筮逆,猶得更卜,故此有筮逆龜從之事。或筮兇則止而不卜,乃是\CJKunderline{鄭玄}之意,非是\CJKunderwave{周禮}經文,未必孔之所取。\CJKunderwave{曲禮}云:“卜筮不相襲。”鄭云:“卜不吉則又筮,筮不吉則又卜,是謂瀆龜筮。”\CJKunderwave{周禮}太卜小事筮,大事卜,應筮而又用卜,應卜而又用筮,及國之大事先筮後卜,不吉之後更作卜筮,如此之等,是為相襲,皆據吉凶分明,不可重為卜筮,若吉凶未決,於事尚疑者,則得更為卜筮。僖二十五年晉侯卜納王,得阪泉之兆,曰:“吾不堪也。”公曰:“筮之。”遇大有之睽;又哀九年晉趙鞅卜救鄭,遇水適火,又筮之,遇泰之需之類是也。\CJKunderwave{周禮}既先筮後卜,而春秋時先卜後筮者,不能依禮故也。 \par}

{\noindent\shu\zihao{5}\fzkt “七稽”至“之言”。正義曰:“稽疑”者,言王者考正疑事。當選擇知卜筮者而建立之,以為卜筮人,謂立為卜人筮人之官也。既立其官,乃命以卜筮之職。雲卜兆有五。曰雨兆,如雨下也。曰霽兆,如雨止也。曰雺兆,氣蒙暗也。曰圛兆,氣落驛不連屬也。曰克兆,相交也。筮卦有二重,二體乃成一卦。曰貞,謂內卦也。曰悔,謂外卦也。卜筮兆卦其法有七事,其卜兆用五,雨、霽、蒙、驛、克也。其筮佔用二,貞與悔也。卜筮皆就此七者推衍其變,立是知卜筮人,使作卜筮之官。其卜筮必用三代之法,三人佔之,若其所佔不同,而其善鈞者,則從二人之言,言以此法考正疑事也。 \par}

“八,庶徵。曰雨,曰暘,曰燠,曰寒,曰風,曰時。\footnote{雨以潤物,暘以幹物,暖以長物,寒以成物,風以動物,五者各以其時,所以為眾驗。○暘音陽。幹音幹。暖,乃管反。長,之丈反。}五者來備,各以其敘,庶草蕃廡。\footnote{言五者備至,各以次序,則眾草蕃滋廡豐也。○蕃音煩。廡,無甫反,徐莫柱反。}一極備,兇。一極無,兇。\footnote{一者備極,過甚則兇。一者極無,不至亦兇。謂不時失敘。}

{\noindent\shu\zihao{5}\fzkt 正義曰:“庶”,眾也。“徵”,驗也。王者用九疇,為大中,行“稽疑”以上為善政,則眾驗有美惡,以為人主。自“曰雨”至“一極無兇”,總言五氣之驗,有美有惡。“曰休徵”敘美行之驗,“曰咎徵”敘惡行之驗。自“曰王省”至“家用平康”,言政善致美也。“日月歲時”至“家用不寧”,言政惡致咎也。“庶民惟星”以下,言人君當以常度齊正下民。 \par}

{\noindent\zhuan\zihao{6}\fzbyks 傳“雨以”至“眾驗”。正義曰:\CJKunderwave{易·說卦}云:“風以散之,雨以潤之,日以烜之。”日,暘也;烜,幹也;是“雨以潤物,暘以幹物,風以動物”也。\CJKunderwave{易·繫辭}云:“寒往則暑來,暑往則寒來,寒暑相推而歲成焉。”是言天氣有寒有暑,暑長物而寒成物也。\CJKunderwave{釋言}云:“燠,暖也。”舍人曰:“燠,溫暖也。”是“燠”、“暖”為一,故傳以“暖”言之。不言“暑”而言“燠”者,“燠”是熱之始,“暑”是熱之極,“涼”是冷之始,“寒”是冷之極,長物舉其始,成物舉其極,理宜然也。五者各以其時而至,所以為眾事之驗也。所以言“時”者,謂當至則來,當止則去,無常時也。冬寒夏燠,雖有定時,或夏須漸寒,冬當漸熱,雨足則思暘,暘久則思雨,草木春則待風而長,秋則待風而落,皆是無定時也。不言“一曰”、“二曰”者,為其來無先後也。依五事所致為次,下雲“休徵”、“咎徵”、“雨若”、“風若”,是其致之次也。昭元年\CJKunderwave{左傳}云:“天有六氣,陰、陽、風、雨、晦、明也。”以彼六氣校此五氣,“雨”、“暘”、“風”文與彼同,彼言“晦”、“明”,此言“寒”、“燠”則“晦”是“寒”也,“明”是“燠”也,惟彼“陰”於此無所當耳。\CJKunderwave{五行傳}說五事致此五氣云:“貌之不恭,是謂不肅,厥罰恆雨,惟金沴木。言之不從,是謂不又,厥罰恆暘,惟木沴金。視之不明,是謂不晢,厥罰恆燠,惟水沴火。聽之不聰,是謂不謀,厥罰恆寒,惟火沴水。思之不睿,是謂不聖,厥罰恆風,惟木金水火沴土。”如彼\CJKunderwave{五行傳}言,是雨屬木,暘屬金,燠屬火,寒屬水,風屬土。鄭云:“雨,木氣也,春始施生,故木氣為雨。暘,金氣也,秋物成而堅,故金氣為暘。燠,火氣也。寒,水氣也。風,土氣也。凡氣非風不行,猶金木水火非土不處,故土氣為風。”是用\CJKunderwave{五行傳}為說,孔意亦當然也。六氣有“陰”,五事休咎皆不致陰,\CJKunderwave{五行傳}又曰:“皇之不極,厥罰常陰”,是陰氣不由五事,別自屬皇極也。蓋立用大中,則陰順時為休。大之不中,陰恆若為咎也。 \par}

{\noindent\zhuan\zihao{6}\fzbyks 傳“言五”至“廡豐”。正義曰:五氣所以生成萬物,正可時來時去,不可常無常有,故言“五者備至,各以次序”。須至則來,須止則去,則眾草百物蕃滋廡豐也。\CJKunderwave{釋詁}云:“廡,豐茂也。”“草蕃廡”言草滋多而茂盛也。下言“百穀用成”,此言“眾草蕃廡”者,舉草茂盛則谷成必矣,舉輕以明重也。 \par}

{\noindent\zhuan\zihao{6}\fzbyks 傳“一者”至“失敘”。正義曰:此謂不以時來,其至無次序也。“一者備極,過甚則兇”,謂來而不去也。“一者極無,不至亦兇”,謂去而不來也。即卜雲“恆雨若”、“恆風若”之類是也。有無相刑,去來正反,恆雨則無暘,恆寒則無燠,恆雨亦兇,無暘亦兇,恆寒亦兇,無燠亦兇,謂至不待時,失次序也。如此則草不茂,谷不成也。 \par}

{\noindent\shu\zihao{5}\fzkt “曰雨”至“無兇”。正義曰:將說其驗,先立其名。五者行於天地之間,人物所以得生成也。其名曰雨,所以潤萬物也。曰暘,所以幹萬物也。曰燠,所以長萬物也。曰寒,所以成萬物也。曰風,所以動萬物也。此是五氣之名。“曰時”言五者各以時來,所以為眾事之驗也。更述時與不時之事,五者於是來皆備足,須風則風來,須雨則雨來,其來各以次序,則眾草木蕃滋而豐茂矣。謂來以時也。若不以時,五者之內,一者備極,過甚則兇。一者極無,不至亦兇。雨多則澇,雨少則旱,是備極亦兇,極無亦兇。其餘四者亦然。 \par}

曰休徵。\footnote{敘美行之驗。○行,下孟反。}曰肅,時寒若。\footnote{君行敬,則時雨順之。}曰乂,時暘若。\footnote{君行政治,則時暘順之。○治,直吏反,下“政治”、“治其職”皆同。}曰晢,時燠若。\footnote{君能照晢,則時燠順之。○晢,之設反,徐音制,又音哲。}曰謀,時寒若。\footnote{君能謀,則時寒順之。}曰聖,時風若。\footnote{君能通理,則時風順之。}

{\noindent\shu\zihao{5}\fzkt “曰休徵”至“風若”。正義曰:既言五者次序,覆述次序之事,曰美行致以時之驗,何者是也?曰人君行敬,則雨以時而順之。曰人君政治,則暘以時而順之。曰人君照晢,則燠以時而順之。曰人君謀當,則寒以時而順之。曰人通聖,則風以時而順之。此則致上文“各以其次,敘庶草蕃廡”也。 \par}

曰咎徵。\footnote{敘惡行之驗。○咎,其九反。}曰狂,恆雨若。\footnote{君行狂妄,則常雨順之。}曰僣,恆暘若。\footnote{君行僣差,則常暘順之。}曰豫,恆燠若。\footnote{君行逸豫,則常燠順之。○豫,羊庶反,徐又音舒。}曰急,恆寒若。\footnote{君行急,則常寒順之。}曰蒙,恆風若。\footnote{君行蒙暗,則常風順之。}

{\noindent\zhuan\zihao{6}\fzbyks 傳“君行”至“順之”。正義曰:此“休”、“咎”皆言“若”者,其所致者皆順其所行,故言“若”也。\CJKunderwave{易·文言}云:“雲從龍,風從虎,水流溼,火就燥。”是物各以類相應,故知天氣順人所行以示其驗也。其咎反於休者,人君行不敬則狂妄,故“狂”對“肅”也。政不治則僣差,故“僣”對“乂”也。明不照物則行自逸豫,故“豫”對“晢”也。心無謀慮則行必急躁,故“急”對“謀”也。性不通曉則行必蒙暗,故“蒙”對“聖”也。\CJKunderline{鄭玄}以“狂”為倨慢,以對“不敬”,故為慢也。鄭、王本“豫”作“舒”,鄭雲“舉遲也”,王肅雲“舒,惰也”,以對“昭晢”,故為遲惰。鄭云:“急促,自用也。”以“謀”者用人之言,故“急”為自用己也。鄭云:“蒙,見冒亂也。”王肅云:“蒙,瞽蒙。”以“聖”是通達,故“蒙”為瞽蒙。所見冒亂,言其不曉事,與“聖”反也。與孔各小異耳。 \par}

{\noindent\shu\zihao{5}\fzkt “曰咎之”至“風若”。正義曰:上既言失次序,覆述失次序之事。曰惡行致備極之驗,何者是也?曰君行狂妄,則常雨順之。曰君行僣差,則常暘順之。曰君行逸豫,則常暖順之。曰君行急躁,則常寒順之。曰君行蒙暗,則常風順之。此即致上文“一極備兇,一極無兇”也。 \par}

曰王省惟歲,\footnote{王所省職,兼所總群吏,如歲兼四時。○省,息井反。}卿士惟月,\footnote{卿士各有所掌,如月之有別。○別,彼列反。}師尹惟曰。\footnote{眾正官之吏,分治其職,如日之有歲月。}歲月曰時無易,\footnote{各順常。}百穀用成,乂用明,\footnote{歲月日時無易,則百穀成。君臣無易,則政治明。}俊民用章,家用平康。\footnote{賢臣顯用,國家平寧。}日月歲時既易,\footnote{是三者已易,喻君臣易職。}百穀用不成,乂用昏不明,俊民用微,家用不寧。\footnote{君失其柄,權臣擅命,治暗賢隱,國家亂。}

{\noindent\zhuan\zihao{6}\fzbyks 傳“王所”至“四時”。正義曰:下雲“庶民惟星”以“星”喻民,知此“歲月日”者,皆以喻職事也。於王言“省”,則卿士師尹亦為“省”也。王之所省,職無不兼,所總群吏如歲兼四時。下句惟有“月”、“日”,群臣無喻“時”者,但時以統月,故傳以“四時”言之,言其兼下月日也。 \par}

{\noindent\zhuan\zihao{6}\fzbyks 傳“眾正”至“歲月”。正義曰:師,眾也。尹,正也。“眾正官之吏”謂卿士之下。有正官大夫,與其同類之官為長。\CJKunderwave{周禮}大司樂為樂官之長,太卜為卜官之長,此之類也。此等分治其職,屬王屬卿,“如日之有歲月”,言其有系屬也。\CJKunderwave{詩}稱“赫赫師尹”,乃謂三公之官,此以“師尹”為正官之吏,謂大夫者,以此“師尹”之文在“卿士”之下,卑於卿士,知是大夫。與小官為長,亦是眾官之長,故師尹之名同耳。鄭雲“所以承休徵、咎徵言之者,休咎五事得失之應,其所致尚微,故大陳君臣之象,成皇極之事。其道得,則其美應如此。其道失,則敗德如彼。非徒風雨寒燠而已”是也。 \par}

{\noindent\shu\zihao{5}\fzkt “曰王省”至“不寧”。正義曰:既陳五事之休咎,又言皇極之得失,與上異端,更復言曰,王之省職,兼總群吏,惟如歲也。卿士分居列位,惟如月也。眾正官之長各治其職,惟如日也。此王也,卿士也,師尹也,掌事猶歲月日者,言皆無改易,君秉君道,臣行臣事。則百穀用此而成,歲豐稔也。其治用是而明,世安泰也。俊民用此而章,在官位也。國家用此而平安,風俗和也。若王也,卿士也,師尹也,掌事猶如日月歲者,是已變易,君失其柄權,臣各專恣。百穀用此而不成,歲饑饉也。其治用此昏暗而不明,政事亂也。俊民用此而卑微,皆隱遁也。國家用此而不安泰,時世亂也。此是皇極所致,得中則致善,不中則致惡。歲月日無易,是得中也。既易,是不中也。所致善惡乃大於庶徵,故於此敘之也。 \par}

庶民惟星,星有好風,星有好雨。\footnote{星,民象,故眾民惟若星。箕星好風,畢星好雨,亦民所好。○好,呼報反。}日月之行,則有冬有夏。\footnote{日月之行,冬夏各有常度。君臣政治,小大各有常法。}月之從星,則以風雨。\footnote{月經於箕則多風,離於畢則多雨。政教失常以從民欲,亦所以亂。}

{\noindent\zhuan\zihao{6}\fzbyks 傳“星民”至“所好”。正義曰:星之在天,猶民之在地,星為民象,以其象民,故因以星喻,故“眾民惟若星”也。直言“星有好風”,不知何星,故云“箕星好風”也。畢星好雨亦如民有所好也,不言“畢星好雨”,具於下傳。 \par}

{\noindent\zhuan\zihao{6}\fzbyks 傳“日月”至“常法”。正義曰:日月之行,四時皆有常法,變冬夏為南北之極,故舉以言之。“日月之行,冬夏各有常度”,喻人君為政小大,各有常法。張衡、蔡雍、王蕃等說渾天者皆云,周天三百六十五度四分度之一,天體圓如彈丸,北高南下,北極出地上三十六度,南極入地下三十六度。南極去北極直徑一百二十二度弱,其依天體隆曲。南極去北極一百八十二度強,正當天之中央。南北二極中等之處謂之赤道,去南北極各九十一度。春分日行赤道,從此漸北夏至赤道之北二十四度,去北極六十七度,去南極一百一十五度,日行黑道。從夏至日以後日漸南,至秋分還行赤道,與春分同。冬至行赤道之南二十四度,去南極六十七度,去北極一百一十五度,其日之行處謂之黃道。又有月行之道與日道相近,交絡而過,半在日道之裡,半在日道之表,其當交則兩道相合,交去極遠處,兩道相去六度。此其日月行道之大略也。王肅云:“日月行有常度,君臣禮有常法,以齊其民。” \par}

{\noindent\zhuan\zihao{6}\fzbyks 傳“月經”至“以亂”。正義曰:\CJKunderwave{詩}云:“月離於畢,俾滂沱矣。”是離畢則多雨,其文見於經。經箕則多風,傳記無其事。\CJKunderline{鄭玄}引\CJKunderwave{春秋緯}云:“月離於箕,則風揚沙。”作緯在\CJKunderline{孔君}之後,以前必有此說,孔依用之也。月行雖有常度,時或失道從星,經箕多風,離畢多雨,此天象之自然,以箕為簸揚之器,畢亦捕魚之物故耳。鄭以為“箕星好風者,箕東方木宿,風中央土氣,木克土為妻,從妻所好,故好風也。畢星好雨者,畢西方金宿,雨東方木氣,金克木為妻,從妻所好,故好雨也。推此則南宮好暘,北宮好燠,中宮四季好寒,以各尚妻之所好故也”。未知孔意同否。顧氏所解,亦同於鄭,言“從星者,謂不應從而從,以致此風雨,故喻政教失常以從民欲,亦所以亂也”。上雲“日月之行”,此句惟言“月”者,鄭云:“不言日者,日之從星,不可見故也。” \par}

{\noindent\shu\zihao{5}\fzkt “庶民”至“風雨”。正義曰:既言大中治民,不可改易,又言民各有心,須齊正之。言庶民之性惟若星然。“星有好風,星有好雨”,以喻民有好善,亦有好惡。“日月之行,則有冬有夏”,言日月之行,冬夏各有常道,喻君臣為政小大,各有常法。若日月失其常道,則天氣從而改焉。月之行度失道,從星所好,以致風雨,喻人君政教失常,從民所欲,則致國亂。故常立用大中,以齊正之,不得從民欲也。 \par}

“九,五福。一曰壽,\footnote{百二十年。}二曰富,\footnote{財豐備。}三曰康寧,\footnote{無疾病。}四曰攸好德,\footnote{所好者德福之道。}五曰考終命。\footnote{各成其短長之命以自終,不橫夭。○橫,華孟反,又如字。}六極。


{\noindent\zhuan\zihao{6}\fzbyks 傳“百二十年”。正義曰:人之大期,百年為限,世有長壽雲百二十年者,故傳以最長者言之,未必有正文也。 \par}

{\noindent\zhuan\zihao{6}\fzbyks 傳“所好”至“之道”。正義曰:人所嗜好,稟諸上天,性之所好,不能自已。好善者或當知善是善,好惡者不知惡之為惡,是善,故好之無厭。任其所好,從而觀之,所好者德,是福之道也。好德者天使之然,故為福也。鄭云:“民皆好有德也。”王肅云:“言人君所好者道德為福。”\CJKunderwave{洪範}以人君為主,上之所為,下必從之,人君好德,故民亦好德,事相通也。 \par}

{\noindent\zhuan\zihao{6}\fzbyks 傳“各成”至“橫夭”。正義曰:成十三年\CJKunderwave{左傳}云:“民受天地之中以生,所謂命也。能者養之以福,不能者敗以取禍。”是言命之短長雖有定分,未必能遂其性,不致夭枉,故各成其短長之命以自終,不橫夭者亦為福也。 \par}

一曰兇短折,\footnote{動不遇吉,短未六十,折未三十,言辛苦。○兇,馬云:“終也。”折,時設反,又之舌反。}二曰疾,\footnote{常抱疾苦。}三曰憂,\footnote{多所憂。}四曰貧,\footnote{困於財。}五曰惡,\footnote{醜陋。}六曰弱。”\footnote{尫劣。○尫,鳥黃反。}

{\noindent\zhuan\zihao{6}\fzbyks 傳“動不”至“辛苦”。正義曰:“動不遇吉”者,解“兇”也。傳以“壽”為“百二十年”,“短”者半之,為“未六十”,“折”又半,為“未三十”。“辛苦”者,味也,辛苦之味入口,猶困厄之事在身,故謂殃厄勞役之事為辛苦也。\CJKunderline{鄭玄}以為“兇短折皆是夭枉之名。未齔曰兇,未冠曰短,未婚曰折”。\CJKunderwave{漢書·五行志}云:“傷人曰兇,禽獸曰短,草木曰折。一曰兇,夭是也,兄喪弟曰短,父喪子曰折。”並與孔不同。 \par}

{\noindent\zhuan\zihao{6}\fzbyks 傳“尫劣”。正義曰:“尫”、“劣”並是弱事,為筋力弱,亦為志氣弱。\CJKunderline{鄭玄}云:“愚懦不毅曰弱。”言其志氣弱也。\CJKunderwave{五行傳}有“致極”之文,無致福之事。\CJKunderline{鄭玄}依\CJKunderwave{書傳}云:“兇短折,思不睿之罰。疾,視不明之罰。憂,言不從之罰。貧,聽不聰之罰。惡,貌不恭之罰。弱,皇不極之罰。反此而云,王者思睿則致壽,聽聰則致富,視明則致康寧,言從則致攸好德,貌恭則致考終命。所以然者,不但行運氣性相感,以義言之,以思睿則無擁,神安而保命,故壽。若蒙則不通,殤神夭性,所以短所也。聽聰則謀當,所求而會,故致富。違而失計,故貧也。視明照了,性得而安寧。不明,以擾神而疾也。言從由於德,故好者德也。不從而無德,所以憂耳。貌恭則容儼形美而成性,以終其命。容毀,故致惡也。不能為大中,故所以弱也。”此亦孔所不同焉。此福極之文,雖主於君,亦兼於下,故有“貧”、“富”、“惡”、“弱”之等也。 \par}

{\noindent\shu\zihao{5}\fzkt “九五福”至“曰弱”。正義曰:“五福”者,謂人蒙福祐有五事也。一曰壽,年得長也。二曰富,家豐財貨也。三曰康寧,無疾病也。四曰攸好德,性所好者美德也。五曰考終命,成終長短之命,不橫夭也。“六極”謂窮極惡事有六。一曰兇短折,遇兇而橫夭性命也。二曰疾,常抱疾病。三曰憂,常多憂愁。四曰貧,困之於財。五曰惡,貌狀醜陋。六曰弱,志力尫劣也。“五福”、“六極”,天實得為之,而歷言此者,以人生於世,有此福極,為善致福,為惡致極,勸人君使行善也。“五福”、“六極”如此次者,鄭云:“此數本諸其尤者,福是人之所欲,以尤欲者為先。極是人之所惡,以尤所不欲者為先。以下緣人意輕重為次耳。” \par}

武王既勝殷,邦諸侯,班宗彝,\footnote{賦宗廟彝器酒樽賜諸侯。○班,本又作般,音同。}作\CJKunderwave{分器}。\footnote{言諸侯尊卑,各有分也。亡。○分,扶問反,注同。}

{\noindent\zhuan\zihao{6}\fzbyks 傳“賦宗”至“諸侯”。正義曰:序雲“邦諸侯”者,立邦國,封人為諸侯也。\CJKunderwave{樂記}雲“封有功者為諸侯”,\CJKunderwave{詩·賚}序雲“大封於廟”,謂此時也。\CJKunderwave{釋言}云:“班,賦也。”\CJKunderwave{周禮}有司尊彝之官,鄭云:“彝亦尊也。鬱鬯曰彝。彝,法也,言為尊之法正。”然則盛鬯者為彝,盛酒者為尊,皆祭宗廟之酒器也。分宗廟彝器酒尊以賦諸侯,既封乃賜之也。 \par}

{\noindent\zhuan\zihao{6}\fzbyks 傳“言諸”至“也亡”。正義曰:篇名\CJKunderwave{分器},知其篇“言諸侯尊卑,各有分也”。昭十二年\CJKunderwave{左傳}楚靈王云:“昔我先王熊繹與呂伋、王孫牟、谿父、禽父並事康王,四國皆有分,我獨無。”十五年傳曰:“諸侯之封也,皆受明器於王室。”杜預云:“謂明德之分器也。”是諸侯各有分也,亡。 \par}

{\noindent\shu\zihao{5}\fzkt “武王”至“分器”。正義曰:武王既已勝殷,制邦國以封有功者為諸侯。既封為國君,乃班賦宗廟彝器以賜之,於時有言誥戒敕。史敘其事,作\CJKunderwave{分器}之篇。 \par}

%%% Local Variables:
%%% mode: latex
%%% TeX-engine: xetex
%%% TeX-master: "../Main"
%%% End:
