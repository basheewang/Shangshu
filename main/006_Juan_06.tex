%% -*- coding: utf-8 -*-
%% Time-stamp: <Chen Wang: 2024-04-02 11:42:41>

% {\noindent \zhu \zihao{5} \fzbyks } -> 注 (△ ○)
% {\noindent \shu \zihao{5} \fzkt } -> 疏

\part{夏書}


\chapter{卷六}


\section{禹貢第一(禹貢上、中、下)}

禹別九州,\footnote{分其圻界。別,彼列反。九州,\CJKunderwave{周公職錄}云:“黃帝受命,風后受圖,割地布九州。”\CJKunderwave{鄹子}云:“中國為赤縣,內有九州。”\CJKunderwave{春秋說題辭}云:“州之言殊也。”圻,其依反。}隨山濬川,\footnote{刊其木,深其流。濬,思俊反。刊,苦安反。}任土作貢。\footnote{任其土地所有,定其貢賦之差。此堯時事,而在\CJKunderwave{夏書}之首,禹之王以是功。任,而鴆反。貢,字或作贛。王,於況反。}

{\noindent\zhuan\zihao{6}\fzbyks 傳“分其圻界”。正義曰:\CJKunderwave{詩}傳云:“圻,疆也。”分其疆界,使有分限。計九州之境,當應舊定,而云“禹別”者,以堯遭洪水,萬事改新,此為作貢生文,故言“禹別”耳。 \par}

{\noindent\zhuan\zihao{6}\fzbyks 傳“刊其木,深其流”。正義曰:經言“隨山刊木”,序以較略為文,直雲“隨山”,不雲隨山為何事,故傳明之隨山刊其木也。“濬川”,深其流也。“隨山”本為“濬川”,故連言之。 \par}

{\noindent\zhuan\zihao{6}\fzbyks 傳“任其”至“是功”。正義曰:九州之土,物產各異,任其土地所有,以定貢賦之差,既任其所有,亦因其肥瘠多少不同,製為差品。\CJKunderline{鄭玄}云:“任土謂定其肥磽之所生。”是言用肥瘠多少為差也。“賦”者,自上稅下之名,謂治田出谷,故經定其差等,謂之“厥賦”。“貢”者,從下獻上之稱,謂以所出之谷,市其土地所生異物,獻其所有,謂之“厥貢”。雖以所賦之物為貢用,賦物不盡有也,亦有全不用賦物,直隨地所有,採取以為貢者,此之所貢,即與\CJKunderwave{周禮·太宰}“九貢”不殊,但\CJKunderwave{周禮}分之為九耳。其賦與\CJKunderwave{周禮}“九賦”全異,彼賦謂口率出錢。不言“作賦”而言“作貢”者,取下供上之義也。諸序皆言作某篇,此序不言“作\CJKunderwave{禹貢}”者,以發首言“禹”,句末言“貢”,篇名足以顯矣。百篇之序,此類有三。“微子作誥父師、少師”,不言“作\CJKunderwave{微子}”,“仲虺作誥”,不言“作\CJKunderwave{仲虺之誥}”,與此篇皆為理足而略之也。又解篇在此之意,此治水是堯末時事,而在\CJKunderwave{夏書}之首,禹之得王天下,以是治水之功,故以為\CJKunderwave{夏書}之首。此篇史述時事,非是應對言語,當是水土既治,史即錄此篇,其初必在\CJKunderwave{虞書}之內,蓋夏史抽入\CJKunderwave{夏書},或仲尼始退其第,事不可知也。 \par}

{\noindent\shu\zihao{5}\fzkt “禹別”至“作貢”。正義曰:禹分別九州之界,隨其所至之山,刊除其木,深大其川,使得注海。水害既除,地複本性,任其土地所有,定其貢賦之差,史錄其事,以為\CJKunderwave{禹貢}之篇。 \par}

禹貢\footnote{禹制九州貢法。}

{\noindent\zhuan\zihao{6}\fzbyks 傳“禹制九州貢法”。正義曰:禹制貢法,故以“禹貢”名篇。貢賦之法其來久矣,治水之後更復改新,言此篇貢法是禹所制,非禹始為貢也。 \par}

{\noindent\shu\zihao{5}\fzkt “禹貢”。正義曰:此篇史述為文,發首“奠高山大川”,言禹治九州之水,水害既除,定山川次秩,與諸州為引序。自“導岍”至“嶓冢”,條說所治之山,言其首尾相及也。自“導弱水”至“導洛”,條說所治之水,言其發源注海也。自“九州攸同”至“成賦中邦”,總言水土既平,貢賦得常之事也。“錫土姓”三句,論天子於土地布行德教之事也。自“五百里甸服”至“二百里流”,總言四海之內,量其遠近,分為五服之事也。自“東漸於海”以下,總結禹功成受錫之事也。 \par}

禹敷土,隨山刊木,\footnote{洪水泛溢,禹布治九州之土,隨行山林,斬木通道。○敷,芳無反,馬云:“分也。”泛,敷劍反。行,下孟反。}奠高山大川。\footnote{奠,定也。高山,五嶽。大川,四瀆。定其差秩,祀禮所視。奠,田遍反。瀆音獨,下同。}

{\noindent\zhuan\zihao{6}\fzbyks 傳“洪水”至“通道”。正義曰:\CJKunderwave{詩}傳云:“泛,泛流也。”泛是水流之貌,洪水流而泛溢,浸壞民居,故禹分佈治之。知者,文十八年\CJKunderwave{左傳}雲“舉八凱使主后土”,則伯益之輩佐禹多矣,禹必身行九州,規謀設法,乃使佐巳之人分佈治之。於時平地盡為流潦,鮮有陸行之路,故將欲治水,隨行山林,斬木通道。鄭云:“必隨州中之山而登之,除木為道,以望觀所當治者,則規其形而度其功焉。”是言禹登山之意也。\CJKunderwave{孟子}曰,禹三過門不入。其家門猶三過之,則其餘所歷多矣。來而復往,非止一處,故言分佈治之之。 \par}

{\noindent\zhuan\zihao{6}\fzbyks 傳“奠定”至“所視”。正義曰:\CJKunderwave{禮}定器於地,通名為“奠”,是“奠”為定也。山之高者,莫高於嶽;川之大者,莫大於瀆;故言“高山,五嶽”,謂、嵩、岱、衡、華、恆也;“大川,四瀆”,謂江、河、淮、濟也。此舉高大為言,卑小亦定之矣。\CJKunderwave{舜典}云:“望秩于山川。”故言“定其差秩”,定其大小次敘也。定其“祀禮所視”,謂\CJKunderwave{王制}所云“五嶽視三公,四瀆視諸侯,其餘視伯子男”。往者洪水滔天,山則為水所包,川則水皆泛溢,祭祀禮廢,今始定之,以見水土平,復舊制也。經雲“荊岐既旅”,“蔡蒙旅平”,“九山刊旅”,是次秩既定,故旅祭之。 \par}

{\noindent\shu\zihao{5}\fzkt “禹敷”至“大川”。正義曰:言禹分佈治此九州之土,其治之也,隨行所至之山,除木通道,決流其水,水土既平,乃定其高山大川。謂定其次秩尊卑,使知祀禮所視。言禹治其山川,使復常也。 \par}

\textcolor{red}{冀州}既載,\footnote{堯所都也。先施貢賦役,載於書。冀,居器反。州,九州名義見\CJKunderwave{爾雅音}。載如字。載,載於書也;馬同,鄭、韋昭云:“載,事也。”}

{\noindent\zhuan\zihao{6}\fzbyks 傳“堯所”至“於書”。正義曰:史傳皆雲堯都平陽,\CJKunderwave{五子之歌}曰:“惟彼陶唐,有此冀方。”是冀州堯所都也。諸州冀為其先,治水先從冀起,為諸州之首。記其役功之法,“既載”者,言先施貢賦役,載於書也。謂計人多少,賦功配役,載於書籍,然後徵而用之,以治水也。冀州如此,則餘州亦然,故於此特記之也。王肅云:“言巳賦功屬役,載於書籍。”傳意當然,鄭云:“載之言事,事謂作徒役也。禹知所當治水,又知用徒之數,則書於策以告帝,徵役而治之。”惟解“載”字為異,其意亦同孔也。 \par}

{\noindent\shu\zihao{5}\fzkt “冀州”。正義曰:九州之次,以治為先後。以水性下流,當從下而洩,故治水皆從下為始。冀州,帝都,於九州近北,故首從冀起。而東南次兗,而東南次青,而南次徐,而南次揚,從揚而西次荊,從荊而北次豫,從豫而西次梁,從梁而北次雍,雍地最高,故在後也。自兗已下,皆準地之形勢,從下向高,從東向西。青、徐、揚三州併為東偏,雍州高於豫州,豫州高於青、徐,雍、豫之水從青、徐而入海也。梁高於荊,荊高於揚,梁、荊之水從揚而入海也。兗州在冀州東南,冀、兗二州之水,各自東北入海也。冀州之水不經兗州,以冀是帝都,河為大患,故先從冀起,而次治兗。若使冀州之水東入兗州,水無去處,治之無益,雖是帝都,不得先也。此經大體每州之始先言山川,後言平地。青州、梁州先山後川,徐州、雍州先川后山,兗、揚、荊、豫有川無山,揚、豫不言平地,冀州田賦之下始言“恆、衛既從”,史以大略為文,不為例也。每州之下言水路相通,通向帝都之道,言禹每州事了,入朝以白帝也。 \par}

壺口治梁及岐。\footnote{壺口在冀州,梁、岐在雍州,從東循山治水而西。壺音胡,馬云:“壺口,山名。”治如字。岐,其宜反。雍,於用反,後州名同。}

{\noindent\zhuan\zihao{6}\fzbyks 傳“壺口”至“而西”。正義曰:\CJKunderwave{史記}稱高祖入咸陽,蕭何先收圖籍,則秦焚詩書,圖籍皆在。\CJKunderline{孔君}去漢初七八十年耳,身為武帝博士,必當具見圖籍,其山川所在,必是驗實而知。“壺口在冀州,梁、岐在雍州”,當時疆界為然也。此於冀州之分,言及雍州之山者,“從東循山治水而西”故也。鄭云:“於此言‘治梁及岐’者,蓋治水從下起,以襄水害易也。”班固作\CJKunderwave{漢書·地理志},據前漢郡縣言山川所在。\CJKunderwave{志}雲壺口在河東北屈縣東南。應劭云:“已有南屈,故稱北屈。”梁山在左馮翊夏陽縣西北,岐山在右扶風美陽縣西北,然則壺口西至梁山,梁山西至岐山,從東而向西言之也。經於“壺口”之下言“治”者,孔意蓋雲欲見上下皆治也。 \par}

既修太原,至於岳陽。\footnote{高平曰太原,今以為郡名。嶽,太嶽,在太原西南。山南曰陽。嶽,字又作嶽。太嶽,山名。陽,山南曰陽,水北亦曰陽。}

{\noindent\zhuan\zihao{6}\fzbyks 傳“高平”至“曰陽”。正義曰:“太原”,原之大者,\CJKunderwave{漢書}以為郡名,傳欲省文,故云“高平曰太原,今以為郡名”,即晉陽縣是也。\CJKunderwave{釋地}云:“廣平曰原,高平曰陸。”孔以太原地高,故言“高平”,其地高而廣也。下文導山雲“壺口、雷首至於太嶽”,知此“嶽”即太嶽也。屬河東郡,在太原西南也。\CJKunderwave{地理志}河東彘縣東有霍太山,此彘縣周厲王所奔,順帝改為永安縣,\CJKunderwave{周禮·職方氏}冀州其山鎮曰霍山,即此太嶽是也。山南見日,故“山南曰陽”。此說循理平地,言從太原至嶽山之南,故云“岳陽”也。 \par}

覃懷厎績,至於衡漳。\footnote{覃懷,近河地名。漳水橫流入河,從覃懷致功至橫漳。覃,徒南反。厎,之履反。衡如字,橫也;馬云:“水名。”漳音章。近,附近之近。}

{\noindent\zhuan\zihao{6}\fzbyks 傳“覃懷”至“衡漳”。正義曰:\CJKunderwave{地理志}河內郡有懷縣,在河之北,蓋“覃懷”二字共為一地,故云“近河地名”。“衡”即古“橫”字,漳水橫流入河,故云“橫漳”。漳在懷北五百餘里,從覃懷致功而北至橫漳也。\CJKunderwave{地理志}云,清漳水出上黨沾縣大黽谷,東北至渤海阜城縣入河,過郡五,行千六百八十里,此沾縣因水為名。\CJKunderwave{志}又云:“沾水出壺關。\CJKunderwave{志}又云,濁漳水出長子縣,東至鄴縣入清漳。\CJKunderline{鄭玄}亦云:“橫漳,漳水橫流。”王肅云:“衡、漳,二水名。” \par}

厥土惟白壤,\footnote{無塊曰壤,水去土復,其性色白而壤。壤,若丈反。馬云:“天性和美也。”塊,苦對反。}

{\noindent\zhuan\zihao{6}\fzbyks 傳“無塊”至“而壤”。正義曰:\CJKunderwave{九章算術}“穿地四,為壤五。壤為息土”,則“壤”是土和緩之名,故云“無塊曰壤”。此土本色為然,水去土復其性,色白而壤。雍州色黃而壤,豫州直言“壤”,不言其色,蓋州內之土不純一色,故不得言色也。 \par}

厥賦惟上上錯,\footnote{賦謂土地所生,以供天子。上上,第一。錯,雜,雜出第二之賦。上如字,賦第一。錯,倉各反,馬云:“地有上下相錯,通率第一。”供音恭。}

{\noindent\zhuan\zihao{6}\fzbyks 傳“賦謂”至“之賦”。正義曰:以文承“厥土”之下,序雲“任土作貢”,又“賦”者稅斂之名,往者洪水為災,民皆墊溺,九州賦稅蓋亦不行,水災既除,土複本性,以作貢賦之差,故云“賦謂土地所生,以供天子”。謂稅谷以供天子,\CJKunderline{鄭玄}雲“此州入谷不貢”是也。因九州差為九等,“上上”是第一也。交錯是間雜之義,故“錯”為雜也。顧氏雲“上上之下即次上中”,故云“雜出第二之賦”也。\CJKunderwave{孟子}稱稅什一為正;輕之於堯舜,為大貊小貊;重之於堯舜,為大桀小桀;則此時亦什一。稅俱什一,而得為九等差者,人功有強弱,收穫有多少。傳以荊州“田第八,賦第三”,為“人功修”也,雍州“田第一,賦第六”,為“人功少”也,是據人功多少總計以定差。此州以上上為正,而雜為次等,言出上上時多,而上中時少也。多者為正,少者為雜,故云“第一”。此州言“上上錯”者,少在正下,故先言“上上”,而後言“錯”。豫州言“錯上中”者,少在正上,故先言“錯”,而後言“上中”。揚州雲“下上上錯”,不言“錯下上”者,以本設九等,分三品為之上中下,下上本是異品,故變文言“下上上錯”也。梁州雲“下中三錯”者,梁州之賦凡有三等,其出下中時多,故以“下中”為正,上有下上,下有下下,三等雜出,故言“三錯”,是明雜有下上、下下可知也。此九等所較無多,諸州相準為等級耳。此計大率所得,非上科定也。但治水據田責其什一,隨上豐瘠,是上之任土,而下所獻自有差降,即以差等為上之定賦也。然一升一降,不可常同。冀州自出第二,與豫州同,時則無第一之賦。豫州與冀州等一同,時則無第二之賦。或容如此,事不可恆。\CJKunderline{鄭玄}云:“賦之差,一井,上上出九夫稅,下下出一夫稅,通率九州一井稅五夫。”如鄭此言,上上出稅,九倍多於下下。鄭\CJKunderwave{詩箋}云:“井稅一夫,其田百畝。”若上上一井稅一夫,則下下九井乃出一夫,稅太少矣;若下下井稅一夫,則上上全入官矣,豈容輕重頓至是乎? \par}

厥田惟中中。\footnote{田之高下肥瘠,九州之中為第五。中,丁仲反,又如字。中,馬云:“土地有高下。”肥,符非反。瘠,在亦反。}

{\noindent\zhuan\zihao{6}\fzbyks 傳“田之”至“第五”。正義曰:\CJKunderline{鄭玄}雲“田著高下之等者,當為水害備也”,則鄭謂地形高下為九等也。王肅雲“言其土地各有肥瘠”,則肅定其肥瘠以為九等也。如鄭之義,高處地瘠,出物既少,不得為上。如肅之義,肥處地下,水害所傷,出物既少,不得為上。故孔雲“高下肥瘠”,共相參對,以為九等。上言“敷土”,此言“厥田”,“田”、“土”異者,\CJKunderline{鄭玄}云:“地當陰陽之中,能吐生萬物者曰土。據人功作力競得而田之,則謂之田。”“田”、“土”異名,義當然也。 \par}

恆、衛既從,大陸既作。\footnote{二水已治,從其故道,大陸之地巳可耕作。從,才容反。}

{\noindent\zhuan\zihao{6}\fzbyks 傳“二水”至“耕作”。正義曰:二水泛溢漫流已治,從其故道故,今已可耕作也。青州“濰、淄其道”,與此“恆、衛既從”同,是從故道也。荊州“雲土、夢作乂”,與此“大陸既作”同,是水治可耕作也。其文不同,史異辭耳,無義例也。“壺口”與雍州之山連文,故傳言“壺口在冀州”。此無所嫌,故不言在冀州,以下皆如此也。\CJKunderwave{地理志}云,恆水出常山上曲陽縣,東入滱水。衛水出常山靈壽縣,東北入滹。大陸在鉅鹿縣北。\CJKunderwave{釋地}“十藪”云:“晉有大陸。”孫炎等皆云:“今鉅鹿縣北廣阿澤也。”郭璞云:“廣阿,猶大陸,以地名言之。”近為是也。\CJKunderwave{春秋}:“魏獻子畋於大陸,焚焉,還,卒於寧。”杜氏\CJKunderwave{春秋}說云,嫌鉅鹿絕遠,以為汲郡修武縣吳澤也。寧即修武也。然此二澤相去其遠,所以得為“大陸”者,以\CJKunderwave{爾雅}“廣平曰陸”,但廣而平者則名大陸,故異所而同名焉。然此二澤地形卑下,得以廣平為陸者,澤雖卑下,旁帶廣平之地,故統名焉。故大陸,澤名;廣河,以旁近大陸故也。 \par}

島夷皮服,\footnote{海曲謂之島。居島之夷還服其皮,明水害除。○島,當老反,馬云:“島夷,北夷國。”}

{\noindent\zhuan\zihao{6}\fzbyks 傳“海曲”至“害除”。正義曰:孔讀“鳥”為“島”。島是海中之山,\CJKunderwave{九章算術}所云“海島邈絕不可踐量”是也。傳雲“海曲謂之島”,謂其海曲有山。夷居其上,此居島之夷,常衣鳥獸之皮,為遭洪水,衣食不足,今還得衣其皮服,以明水害除也。\CJKunderline{鄭玄}云:“鳥夷,東方之民,搏食鳥獸者也。”王肅云:“鳥夷,東北夷國名也。”與孔不同。 \par}

夾右碣石,入於河。\footnote{碣石,海畔山。禹夾行此山之右,而入河逆上。此州帝都,不說境界,以餘州所至則可知。先賦後田,亦殊於餘州。不言貢篚,亦差於餘州。夾音協,注同帶也。碣,其列反,韋昭其逝反。上,時掌反。篚,方尾反。}

{\noindent\zhuan\zihao{6}\fzbyks 傳“碣石”至“餘州”。正義曰:\CJKunderwave{地理志}碣石山在北平驪城縣西南,是碣石為海畔山也。鄭云:“\CJKunderwave{戰國策}碣石在九門縣,今屬常山郡,蓋別有碣石與此名同。今驗九門無此山也。”下文“導河入於海”,傳云:“入於渤海。”渤海之郡當以此海為名。計渤海北距碣石五百餘里,河入海處遠在碣石之南,禹行碣石不得入於河也。蓋遠行通水之處,北盡冀州之境,然後南迴入河而逆上也。“夾右”者,孔雲“夾行此山之右”,則行碣石山西,南行入河,在碣石之右,故云“夾右”也。顧氏亦云:“山西曰右。”\CJKunderline{鄭玄}云:“禹由碣石山西北行,盡冀州之境,還從山東南行,入河。”鄭以北行則東為右,南行西為右,故夾山兩旁,山常居右,與孔異也。“梁州”傳云:“浮東渡河而還帝都,白所治也。”則入河逆上,為還都白所治也。禹之治水,必每州巡行,度其形勢,計其人功,施設規模,指授方略,令人分佈並作,還都白帝所治。於時帝都近河,故於每州之下皆言浮水達河,記禹還都之道也。冀、兗、徐、荊、豫、梁、雍州各自言河,惟青、揚二州不言河耳。兗州雲“浮於濟、漯,達於河”,故青州直雲“達於濟”;徐州雲“浮於淮、泗,達於河”,故揚州雲“達於淮、泗”;皆記禹入河之道也。王肅云:“凡每州之下說諸治水者,禹功主於治水,故詳記其所治之州往還所乘涉之水名。”肅雖不言“還都白帝”,亦謂為治水,故浮水也。\CJKunderline{鄭玄}以為“治水既畢,更復行之,觀地肥瘠,定貢賦上下”.其意與孔異也。八州皆言境界,而此獨無,故解之,“此州帝都,不說境界,以餘州所至則可知”也。兗州雲“濟、河”,自東河以東也。豫州雲“荊、河”,自南河以南也。雍州雲“西河”,自西河以西也。明東河之西,西河之東,南河之北是冀州之境也。馬、鄭皆雲“冀州不書其界者,時帝都之,使若廣大然”。文既局以州名,復何以見其廣大?是妄說也。又解餘州先田後賦,此州先賦後田,亦如境界,殊於餘州也。言“殊”者,當為田賦以收穫為差,田以肥瘠為等。若田在賦上,則賦宜從田,田美則宜賦重,無以見人功修否,故令賦先於田也。以見賦由人功,此州既見此理,餘州從而可知,皆令賦在田下,欲見賦從田出,為此故殊於餘州也。\CJKunderline{鄭玄}云:“此州入谷不貢。”下雲“五百里甸服”,傳雲“為天子服治田”,是田入谷,故不獻貢篚,差異於餘州也。甸服止方千里,冀之北土境界甚遙,遠都之國,必有貢篚,舉大略而言也。 \par}

濟、河惟兗州。\footnote{東南據濟,西北距河。濟,子禮反,下同。兗,悅轉反。}

{\noindent\zhuan\zihao{6}\fzbyks 傳“東南”至“距河”。正義曰:此下八州,發首言山川者,皆謂境界所及也。“據”謂跨之,“距”,至也。濟、河之間相去路近,兗州之境,跨濟而過,東南越濟水,西北至東河也。李巡注\CJKunderwave{爾雅}解州名云:“兩河間其氣清,性相近,故曰冀。冀,近也。濟、河間其氣專,體性信謙,故云兗。兗,信也。淮海間其氣寬舒,稟性安徐,故曰徐。徐,舒也。江南其氣燥勁,厥性輕揚,故曰揚。揚,輕也。荊州其氣燥剛,稟性彊梁,故曰荊。荊,強也。河南其性安舒,厥性寬豫,故曰豫。豫,舒也。河西其氣蔽壅,受性急兇,故云雍。雍,壅也。”\CJKunderwave{爾雅}“九州”無樑、清,故李巡不釋,所言未必得其本也。 \par}

九河既道,\footnote{河水分為九道,在此州界,平原以北是。九河,徒駭一,太史二,馬頰三,覆釜四,胡蘇五,簡六,絜七,鉤盤八,鬲津九;出\CJKunderwave{爾雅}。}

{\noindent\zhuan\zihao{6}\fzbyks 傳“河水”至“北是”。正義曰:河自大陸之北敷為九河,謂大陸在冀州,嫌九河亦在冀州,故云“在此州界”也。河從大陸東畔北行,而東北入海。冀州之東境,至河之西畔。水分大河,東為九道,故知在兗州界平原以北是也。\CJKunderwave{釋水}載“九河”之名云:“徒駭、太史、馬頰、覆釜、胡蘇、簡、絜、鉤盤、鬲津。”李巡曰:“徒駭,禹疏九河,以徒眾起,故云徒駭。太史,禹大使徒眾通其水道,故曰太史。馬頰,河勢上廣下狹,狀如馬頰也。覆釜,水中多渚,往往而處,形如覆釜。胡蘇,其水下流,故曰胡蘇。胡,下也。蘇,流也。簡,大也,河水深而大也。絜,言河水多山石,治之苦絜。絜,苫也。鉤盤,言河水曲如鉤,屈折如盤也。鬲津,河水狹小,可鬲以為津也。”孫炎曰:“徒駭,禹疏九河,用功雖廣,眾懼不成,故曰徒駭。胡蘇,水流多散胡蘇然。”其餘同李巡。郭璞云:“徒駭今在成平。東光縣今有胡蘇亭。”覆釜之名同李巡,餘名皆雲其義未詳。計禹陳九河,雲復其故道,則名應先有,不宜“徒駭”、“太史”因禹立名,此郭氏所以未詳也。或九河雖舊有名,至禹治水,更別立名,即\CJKunderwave{爾雅}所云是也。\CJKunderwave{漢書·溝洫志}成帝時,河堤都尉許商上書曰:“古記九河之名,有徒駭、胡蘇、鬲津,今見在成平、東光、鬲縣界中。自鬲津以北至徒駭,其間相去二百餘里。”是知九河所在,徒駭最北,鬲津最南。蓋徒駭是河之本道,東出分為八枝也。許商上言三河,下言三縣,則徒駭在成平,胡蘇在東光,鬲津在鬲縣,其餘不復知也。\CJKunderwave{爾雅}“九河”之次,從北而南。既知三河之處,則其餘六者,太史、馬頰、覆釜在東光之北,成平之南;簡、絜、鉤盤在東光之南,鬲縣之北也。其河填塞,時有故道。\CJKunderline{鄭玄}云:“周時齊桓公塞之,同為一河。今河間弓高以東,至平原鬲津,往往有其遺處。”\CJKunderwave{春秋緯·寶乾圖}云:“移河為界在齊呂,填閼八流以自廣。”\CJKunderline{鄭玄}蓋據此文為“齊桓公塞之”也。言閼八流拓境,則塞其東流八枝,並使歸於徒駭也。 \par}

雷夏既澤,灉沮會同。\footnote{雷夏,澤名。灉、沮,二水,會同此澤。灉,徐音邕,王於用反。沮,七餘反。}

{\noindent\zhuan\zihao{6}\fzbyks 傳“雷夏”至“此澤”。正義曰:洪水之時,高原亦水,澤不為澤。“雷夏既澤”,高地水盡,此復為澤也。於“澤”之下言“灉、沮會同”,謂二水會合而同入此澤也。\CJKunderwave{地理志}云,雷澤在濟陰城陽縣西北。 \par}

桑土既蠶,是降丘宅土。\footnote{地高曰丘。大水去,民下丘,居平土,就桑蠶。蠶,在南反。}

{\noindent\zhuan\zihao{6}\fzbyks 傳“地高”至“桑蠶”。正義曰:\CJKunderwave{釋丘}云:“非人為之丘。”孫炎曰:“地性自然也。”是“地高曰丘”也。“降丘宅土”與“既蠶”連文,知“下丘,居平土,就桑蠶”也。計下丘居土,諸處皆然,獨於此州言之者,\CJKunderline{鄭玄}云:“此州寡于山,而夾川兩大流之間,遭洪水,其民尤困。水害既除,於是下丘居土,以其免於厄,尤喜,故記之。 \par}

{\noindent\shu\zihao{5}\fzkt “桑土”至“宅土”。正義曰:宜桑之土既得桑養蠶矣。洪水之時,民居丘上,於是得下丘陵,居平土矣。 \par}

厥土黑墳,\footnote{色黑而墳起。墳,扶粉反,後同;韋昭音勃僨反,起也;馬云:“有膏肥也。”}厥草惟繇,厥木惟條。\footnote{繇,茂。條,長也。繇音遙,馬云:“抽也。”}

{\noindent\zhuan\zihao{6}\fzbyks 傳“繇,茂。條,長也”。正義曰:“繇”是茂之貌,“條”是長之體,言草茂而木長也。九州惟此州與徐揚三州言草木者,三州偏宜之也。宜草木,則地美矣。而田非上者,為土下溼故也。 \par}

厥田惟中下,\footnote{田第六。}厥賦貞,\footnote{貞,正也。州第九,賦正與九相當。}

{\noindent\zhuan\zihao{6}\fzbyks 傳“貞正”至“相當”。正義曰:\CJKunderwave{周易}彖、象皆以“貞”為正也。諸州賦無下下,“貞”即下下,為第九也。此州治水最在後畢,州為第九成功,其賦亦為第九,列賦於九州之差,與第九州相當,故變文為“貞”,見此意也。 \par}

作十有三載,乃同。\footnote{治水十三年,乃有賦法,與他州同。載,馬、鄭本作年。}

{\noindent\zhuan\zihao{6}\fzbyks 傳“治水”至“州同”。正義曰:“作”者,役功作務,謂治水也。治水十三年,乃有賦法,始得貢賦,與他州同也。他州十二年,此州十三年,比於他州最在後也。\CJKunderwave{堯典}言\CJKunderline{鯀}治水九載,績用不成,然後堯命得舜,舜乃舉禹治水,三載功成,堯即禪舜。此言“十三載”者,並\CJKunderline{鯀}九載數之。\CJKunderwave{祭法}雲“禹能修\CJKunderline{鯀}之功”,明\CJKunderline{鯀}已加功,而禹因之也。此言“十三載”者,記其治水之年,言其水害除耳,非言十三年內皆是禹之治水施功也。馬融曰:“禹治水三年,八州平,故堯以為功而禪舜。”是十二年而八州平,十三年而兗州平,兗州平在舜受終之年也。 \par}

厥貢漆絲,厥篚織文。\footnote{地宜漆林,又宜桑蠶。織文,錦綺之屬。盛之篚篚而貢焉。漆音七。盛音成。}

{\noindent\zhuan\zihao{6}\fzbyks 傳“地宜”至“貢焉”。正義曰:任土作貢,此州貢漆,知“地宜漆林”也。\CJKunderwave{周禮·載師}雲“漆林之徵”,故以“漆林”言之。“綺”是織繒之有文者,是綾錦之別名,故云“錦綺之屬”,皆是織而有文者也。“篚”是入貢之時盛在於篚,故云“盛之篚篚而貢焉”。\CJKunderline{鄭玄}云:“貢者百功之府受而藏之。其實於篚者,入於女功,故以貢篚別之。”歷檢篚之所盛,皆供衣服之用,入於女功,如鄭言矣。“檿絲”中琴瑟之弦,亦是女功所為也。“織貝”,\CJKunderline{鄭玄}以為織如貝文,傳謂“織為細紵。貝為水物”,則貝非服飾所須,蓋恐其損缺,故以筐篚盛之也。諸州無“厥篚”者,其諸州無入篚之物,故不貢也。漢世陳留襄邑縣置服官,使製作衣服,是兗州綾錦美也。 \par}

浮於濟、漯,達於河。\footnote{順流曰浮。濟、漯兩水名。因水入水曰達。漯,天答反,\CJKunderwave{篇韻}作他合反。}

{\noindent\zhuan\zihao{6}\fzbyks 傳“順流”至“曰達”。正義曰:\CJKunderwave{地理志}云,漯水出東郡東武陽縣,至樂安千乘縣入海,過郡三,行千二十里。其濟則下文具矣,是“濟、漯”為二水名也。言“因水入水曰達當”,謂從水入水,不須舍舟而陸行也。揚州雲“沿於江海,達於淮泗”,傳雲“沿江入海,自海入淮,自淮入泗”,是言水路相通,得乘舟經達也。案青州雲“浮於汶,達於濟”,經言濟會於汶,浮汶則達濟也。此雲“浮於濟、漯,達於河”,從漯入濟,自濟入河徐州雲“浮於淮泗,達於河”,蓋以徐州北接青州,既浮淮泗,當浮汶入濟,以達於河也。 \par}

海、岱惟\textcolor{red}{青州}。\footnote{東北據海,西南距岱。岱音代,泰山也。}

{\noindent\zhuan\zihao{6}\fzbyks 傳“東北”至“距岱”。正義曰:海非可越而言“據”者,東萊東境之縣,浮海入海曲之間,青州之境,非至海畔而已,故言“據”也。漢末有公孫度者,竊據遼東,自號青州刺史,越海收東萊諸郡。堯時青州當越海而有遼東也。舜為十二州,分青州為營州,營州即遼東也。 \par}

嵎夷既略,濰、淄其道。\footnote{嵎夷,地名。用功少曰略。濰、淄二水復其故道。嵎音隅。濰音惟,本亦作惟,又作維。淄,側其反。}

{\noindent\zhuan\zihao{6}\fzbyks 傳“嵎夷”至“故道”。正義曰:“嵎夷,地名”,即\CJKunderwave{堯典}“宅嵎夷”是也。“嵎夷”、“萊夷”、“和夷”為地名,“淮夷”為水名,“島夷”為狄名,皆觀文為說也。“略”是簡易之義,故“用功少為略”也。\CJKunderwave{地理志}云,濰水出琅邪箕屋山,北至都昌縣入海,過郡三,行五百二十里。淄水出泰山萊蕪縣原山,東北至千乘博昌縣入海。 \par}

厥土白墳,海濱廣斥。\footnote{濱,涯也。言復其斥鹵。濱,必人反。斥,徐音尺。\CJKunderwave{說文}云:“東方謂之斥,西方謂之滷。”鄭云:“斥謂地咸鹵。”涯,魚佳反。}

{\noindent\zhuan\zihao{6}\fzbyks 傳“濱涯”至“斥鹵”。正義曰:“濱,涯”,常訓也。\CJKunderwave{說文}云:“滷,咸地也。東方謂之斥,西方謂之滷。”海畔迥闊,地皆斥鹵,故云“廣斥”,言水害除,復舊性也。 \par}

厥田惟上下,厥賦中上。\footnote{田第三,賦第四。}厥貢鹽、絺,海物惟錯。\footnote{絺,細葛。錯,雜,非一種。鹽,餘佔反。絺,敕其反。種,章勇反。}岱畎絲、枲、鉛、松、怪石。\footnote{畎,谷也。怪,異;好石似玉者。岱山之谷,出此五物,皆貢之。畎,工犬反,徐本作畎谷。枲,思似反。鉛,寅專反,字從㕣;㕣音以選反。怪如字。怪石,碔砆之屬。}

{\noindent\zhuan\zihao{6}\fzbyks 傳“畎谷”至“貢之”。正義曰:\CJKunderwave{釋水}云:“水注川曰谿,注谿曰谷。”谷是兩山之閒流水之道,“畎”言畎去水,故言“谷”也。“怪石”,奇怪之石,故云“好石似玉”也。“枲”,麻也。“鉛”,錫也。岱山之谷有此五物,美於他方所有,故貢之也。 \par}

萊夷作牧,\footnote{萊夷,地名。可以放牧。萊音來。牧,牧養之牧,徐音目,一音茂,注同。}厥篚檿絲。\footnote{檿桑蠶絲,中琴瑟弦。檿,烏簟反,山桑也。}

{\noindent\zhuan\zihao{6}\fzbyks 傳“檿桑”至“瑟弦”。正義曰:\CJKunderwave{釋木}云:“檿桑,山桑。”郭璞曰:“柘屬也。”“檿絲”,是蠶食檿桑,所得絲韌,中琴瑟弦也。 \par}

浮於汶,達於濟。

{\noindent\shu\zihao{5}\fzkt “浮於汶”。正義曰:\CJKunderwave{地理志}云,汶水出泰山萊蕪縣原山,西南入濟也。 \par}

海、岱及淮惟\textcolor{red}{徐州}。\footnote{東至海,北至岱,南及淮。汶音問。}淮、沂其乂,蒙、羽其藝。\footnote{二水已治,二山已可種藝。沂,魚依反,水名。藝,魚世反。}

{\noindent\zhuan\zihao{6}\fzbyks 傳“二水”至“種藝”。正義曰:“乂”訓治也,故云“二水已治”。\CJKunderwave{地理志}云,沂水出泰山,蓋縣臨樂子山南至下邳入泗,過郡五,行六百里。淮出桐柏山,發源遠矣,於此州言之者,淮水至此而大,為害尤甚,喜得其治,故於此記之。\CJKunderwave{地理志}云,蒙山在泰山蒙陰縣西南。羽山在東海祝其縣南。\CJKunderwave{詩}雲“藝之荏菽”,故“藝”為種也。 \par}

大野既豬,東原底平。\footnote{大野,澤名。水所停曰豬。東原致功而平,言可耕。豬,張魚反,馬云:“水所停止深者曰豬。”劉東胡反。}

{\noindent\zhuan\zihao{6}\fzbyks 傳“大野”至“可耕”。正義曰:\CJKunderwave{地理志}云,大野澤在山陽鉅野縣北。“巨”即大也。\CJKunderwave{檀弓}云:“汙其宮而豬焉。”又澤名孟豬,停水處也,故云“水所停曰豬”。往前漫溢,今得豬水為澤也。“東原”即今之東平郡也。致功而地平,言其可耕也。 \par}

厥土赤埴墳,草木漸包。\footnote{土黏曰埴。漸,進長。包,叢生。埴,市力反,鄭作戠,徐、鄭、王皆讀曰熾,韋昭音試。漸如字,本又作蔪,\CJKunderwave{字林}才冉反,草之相包裹也。包,必茅反,字或作苞,非叢生也,馬云:“相包裹也。”黏,女佔反。長,丁丈反。叢,才公反。}

{\noindent\zhuan\zihao{6}\fzbyks 傳“土黏”至“叢生”。正義曰:“戠”“埴”、音義同。\CJKunderwave{考工記}用土為瓦,謂之“摶埴之工”,是“埴”謂黏土,故“土黏曰埴”。\CJKunderwave{易·漸卦}彖云:“漸,進也。”\CJKunderwave{釋言}云:“苞,稹也。”孫炎曰“物叢生曰苞,齊人名曰稹”。郭璞曰:“今人呼叢致者為稹。”“漸苞”謂長進叢生,言其美也。 \par}

厥田惟上中,厥賦中中。\footnote{田第二,賦第五。}厥貢惟土五色,\footnote{王者封五色土為社,建諸侯則各割其方色土與之,使立社。燾以黃土,苴以白茅,茅取其潔,黃取王者覆四方。燾,徒報反,覆也。苴,子餘反,包裹也。}

{\noindent\zhuan\zihao{6}\fzbyks 傳“王者”至“四方”。正義曰:傳解貢土之意,王者封五色土以為社,若封建諸侯,則各割其方色土與之,使歸國立社。其土燾以黃土。燾,覆也。四方各依其方色,皆以黃土覆之。其割土與之時,苴以白茅,用白茅裹土與之。必用白茅者,取其絜清也。\CJKunderwave{易}稱“藉用白茅”,茅色白而絜美。\CJKunderwave{韓詩外傳}云:“天子社廣五丈,東方青,南方赤,西方白,北方黑,上冒以黃土。將封諸侯,各取其方色土,苴以白茅,以為社。明有土謹敬絜清也。”蔡邕\CJKunderwave{獨斷}云:“天子大社,以五色土為壇。皇子封為王者,授之大社之土,以所封之方色苴以白茅,使之歸國以立社,謂之茅社。”是必古書有此說,故先儒之言皆同也。 \par}

羽畎夏翟,嶧陽孤桐,\footnote{夏翟,翟,雉名。羽中旌旄,羽山之谷有之。孤,特也。嶧山之陽特生桐,中琴瑟。夏,行雅反。翟,徒歷反。嶧音亦,一音夕。}

{\noindent\zhuan\zihao{6}\fzbyks 傳“夏翟”至“琴瑟”。正義曰:\CJKunderwave{釋鳥}云:“翟,山雉。”此言“夏翟”,則“夏翟”共為雉名。\CJKunderwave{周禮}立夏採之官,取此名也。\CJKunderwave{周禮·司常}云:“全羽為旞,析羽為旌。”用此羽為之,故云“羽中旌旄”也。\CJKunderwave{地理志}云,東海下邳縣西有葛嶧山,即此山也。 \par}

泗濱浮磬,淮夷蠙珠暨魚。\footnote{泗,水涯。水中見石,可以為磬。蠙珠,珠名。淮、夷二水出蠙珠及美魚。泗音四,水名。淮夷,鄭云:“淮水之夷民也。”馬云:“淮、夷,二水名。”孔傳雲“淮夷之水”,本亦有作“淮夷二水”也。蠙音蒲邊反,徐扶堅反,字又作蚍,韋昭薄迷反,蚌也。暨,其器反。見,賢遍反。}

{\noindent\zhuan\zihao{6}\fzbyks 傳“泗水”至“美魚”。正義曰:泗水旁山而過,石為泗水之涯,石在水旁,水中見石,似若水中浮然,此石可以為磬,故謂之“浮磬”也。貢石而言磬者,此石宜為磬,猶如“砥礪”然也。蠙是蚌之別名,此蚌出珠,遂以蠙為珠名。蠙之與魚皆是水物,而以“淮夷”冠之,知“淮夷”是二水之名。“淮”即四瀆之淮也。“夷”蓋小水,後來竭涸,不復有其處耳。王肅亦以“淮夷”為水名。\CJKunderline{鄭玄}以為“淮水之上,夷民獻此珠與魚”也。\CJKunderwave{地理志}泗水出濟陰乘氏縣,東南至臨淮雎陵縣入淮,行千一百一十里也。 \par}

厥篚玄纖縞。\footnote{玄,黑繒。縞,白繒。纖,細也。纖在中,明二物皆當細。纖,息廉反。縞,古老反,徐古到反。繒,似陵反。}

{\noindent\zhuan\zihao{6}\fzbyks 傳“玄黑”至“當細”。正義曰:篚之所盛,例是衣服之用,此單言“玄”,玄必有質,玄是黑色之別名,故知玄是黑繒也。\CJKunderwave{史記}稱高祖為義帝發喪,諸侯皆縞素,是縞為白繒也。 \par}

浮於淮、泗,達於河。\footnote{河如字,\CJKunderwave{說文}作菏,工可反,云:“水出山陽湖陵南。”}

淮、海惟\textcolor{red}{揚州}。\footnote{北據淮,南距海。}彭蠡既豬,陽鳥攸居。\footnote{彭蠡,澤名。隨陽之鳥,鴻雁之屬,冬月所居於此澤。蠡音禮,張勃\CJKunderwave{吳錄}云:“今名洞庭湖。”案今在九江郡界。}

{\noindent\zhuan\zihao{6}\fzbyks 傳“彭蠡”至“此澤”。正義曰:“彭蠡”是江漢合處,下雲“導漾水,南入於江,東匯為彭蠡”是也。日之行也,夏至漸南,冬至漸北,鴻雁之屬,九月而南,正月而北,左思\CJKunderwave{蜀都賦}所云“木落南翔,冰泮北徂”是也。日,陽也,此鳥南北與日進退,隨陽之鳥,故稱“陽鳥”,冬月所居於此彭蠡之澤也。 \par}

三江既入,震澤厎定。\footnote{震澤,吳南大湖名。言三江已入,致定為震澤。三江,韋昭云:“謂吳松江、錢唐江、浦陽江也。”\CJKunderwave{吳地記}云:“松江東北行七十里,得三江口,東北入海為婁江,東南入海為東江,並松江為三江。震澤,吳都太湖。厎,之履反,致也,\CJKunderwave{史記}音致。大湖,音太湖。}

{\noindent\zhuan\zihao{6}\fzbyks 傳“震澤”至“震澤”。正義曰:\CJKunderwave{地理志}云,會稽吳縣,故周泰伯所封國也。具區在西,古文以為震澤,是“吳南大湖名”。蓋縣治居澤之東北,故孔傳言“南”,\CJKunderwave{志}言“西”。大澤畜水,南方名之曰湖,三江既入此湖也。治水致功,今江入此澤,故“致定為震澤”也。下傳云:“自彭蠡江分為三,入震澤,遂為北江而入海。”是孔意江從彭蠡而分為三,又共入震澤,從震澤復分為三,乃入海。鄭云:“三江分於彭蠡,為三孔東入海。”其意言“三江既入”,入海耳,不入震澤也。又案\CJKunderwave{周禮·職方}揚州藪曰具區,浸曰五湖。五湖即震澤。若如\CJKunderwave{志}雲具區即震澤,則浸、藪為一。案餘州浸、藪皆異而揚州同者,蓋揚州浸、藪同處,論其水謂之浸,指其澤謂之藪。 \par}

篠簜既敷,\footnote{篠,竹箭。簜,大竹。水去巳布生。篠,西了反。簜,徒黨反,或作𥯕,他莽反。}

{\noindent\zhuan\zihao{6}\fzbyks 傳“篠,竹箭。簜,大竹”。正義曰:\CJKunderwave{釋草}云:“篠,竹箭。”郭璞云:“別二名也。”又云“簜,竹”。李巡曰:“竹節相去一丈曰簜。”孫炎曰:“竹闊節者曰簜。”郭璞云:“竹別名。”是篠為小竹,簜為大竹。 \par}

厥草惟夭,厥木惟喬。\footnote{少長曰夭。喬,高也。夭,於嬌反,馬云:“長也。”喬,其嬌反,徐音驕。少,詩照反。長,丁丈反。}

{\noindent\zhuan\zihao{6}\fzbyks 傳“少長曰夭。喬,高也”。正義曰:夭是少長之貌,\CJKunderwave{詩}曰:“桃之夭夭”是也。“喬,高”,\CJKunderwave{釋詁}文。\CJKunderwave{詩}曰“南有喬木”是也。 \par}

厥土惟塗泥,\footnote{地泉溼。}厥田惟下下,厥賦下上錯。\footnote{田第九,賦第七,雜出第六。}厥貢惟金三品,\footnote{金、銀、銅也。}

{\noindent\zhuan\zihao{6}\fzbyks 傳“金、銀、銅也”。正義曰:“金”既總名,而云“三品”,黃金以下惟有白銀與銅耳,故為“金、銀、銅也”。\CJKunderwave{釋器}云:“黃金謂之璗,其美者謂之鏐。白金謂之銀,其美者謂之鐐。”郭璞曰:“此皆道金銀之別名及其美者也。鏐即紫磨金也。”\CJKunderline{鄭玄}以為“金三品者,銅三色也。” \par}

瑤、琨、篠簜,\footnote{瑤、琨皆美玉。瑤音遙。琨音昆,美石也,馬本作瑻,韋昭音貫。}

{\noindent\zhuan\zihao{6}\fzbyks 傳“瑤、琨,皆美玉”。正義曰:美石似玉者也。玉、石其質相類,美惡別名也。王肅云:“瑤、琨,美石次玉者也。” \par}

齒、革、羽、毛、惟木。\footnote{齒,象牙。革,犀皮。羽,鳥羽。毛,旄牛尾。木,楩、軺、豫章。犀,細兮反。旄音毛。楩音緶,又婢善反。}

{\noindent\zhuan\zihao{6}\fzbyks 傳“齒象”至“豫章”。正義曰:\CJKunderwave{詩}雲“元龜象齒”,知“齒”是象牙也。\CJKunderwave{說文}云:“齒,口齗骨也。牙,牡齒也。”隱五年\CJKunderwave{左傳}雲“齒牙骨角”,“牙”、“齒”小別,統而名之,“齒”亦“牙”也。\CJKunderwave{考工記}:“犀甲七屬,兕甲六屬。”宣二年\CJKunderwave{左傳}云:“犀兕尚多,棄甲則那?”是甲之所用,犀革為上;革之所美,莫過於犀;知“革”是犀皮也。\CJKunderwave{說文}云:“獸皮治去其毛為革。”革與皮去毛為異耳。\CJKunderwave{說文}云:“羽,鳥長毛也。”知“羽”是鳥羽。南方之鳥,孔雀、翡翠之屬,其羽可以為飾,故貢之也。\CJKunderwave{說文}云:“犛,西南夷長旄牛也。”此犛牛之尾可為旌旗之飾,經傳通謂之“旄”。\CJKunderwave{牧誓}雲“右秉白旄”,\CJKunderwave{詩}雲“建旐設旄”,皆謂此牛之尾,故知“毛”是旄牛尾也。直雲“惟木”,不言木者,故言“楩、軺、豫章”,此三者是揚州美木,故傳舉以言之,所貢之木不止於此。 \par}

島夷卉服,\footnote{南海島夷草服葛越。卉,徐許貴反。}

{\noindent\zhuan\zihao{6}\fzbyks 傳“南海”至“葛越”。正義曰:上傳“海曲謂之島”,知此“島夷”是南海島上之夷也。\CJKunderwave{釋草}云:“卉,草。”舍人曰:“凡百草一名卉。”知“卉服”是“草服葛越”也。葛越,南方布名,用葛為之。左思\CJKunderwave{吳都賦}雲“蕉葛升越,弱於羅紈”是也。冀州雲“島夷皮服”,是夷自服皮,皮非所貢也。此言“島夷卉服”,亦非所貢也。此與“萊夷作牧”並在貢篚之間,古史立文不次也。\CJKunderline{鄭玄}云:“此州下溼,故衣草服。貢其服者,以給天子之官。”與孔異也。 \par}

厥篚織、貝。\footnote{織,細繒。貝,水物。}

{\noindent\zhuan\zihao{6}\fzbyks 傳“織細”至“水物”。正義曰:傳以“貝”非織物,而云“織貝”,則貝、織異物,織是織而為之,揚州紵之所出,此物又以篚盛之,為衣服之用,知是“細紵”,謂細紵布也。\CJKunderwave{釋魚}之篇貝有居陸居水,此州下溼,故云“水物”。\CJKunderwave{釋魚}有“玄貝,貽貝。餘貾,黃白文。餘泉,白黃文”,當貢此有文之貝以為器物之飾也。\CJKunderline{鄭玄}云:“貝,錦名。\CJKunderwave{詩}云:‘萋兮斐兮,成是貝錦。’凡為織者先染其絲,乃織之則文成矣。\CJKunderwave{禮記}曰:‘土不衣織。’”與孔異也。 \par}

厥包橘柚錫貢。\footnote{小曰橘,大曰柚。其所包裹而致者,錫命乃貢。言不常。橘,均必反。柚,由究反。裹音果。}

{\noindent\zhuan\zihao{6}\fzbyks 傳“小曰”至“不常”。正義曰:橘、柚二果,其種本別,以實相比,則柚大橘小,故云“小曰橘,大曰柚”。猶\CJKunderwave{詩傳}雲“大曰鴻,小曰雁”,亦別種也。此物必須裹送,故云其所包裹而送之。以須之有時,故待錫命乃貢,言不常也。文在“篚”下,以不常故耳。荊州“納錫大龜”,豫州“錫貢磬錯”,皆為非常,並在“篚”下。荊州言“包”,傳雲“橘柚”也,文在“篚”上者,荊州橘柚為善,以其常貢。此州則不常也。王肅云:“橘與柚錫其命而後貢之,不常入,當繼荊州乏無也。”鄭云:“有錫則貢之,此州有錫而貢之,或時無,則不貢。錫,所以柔金也。\CJKunderwave{周禮·考工記}云,攻金之工掌執金錫之齊故也。” \par}

沿於江、海,達於淮、泗。\footnote{順流而下曰沿。沿江入海,自海入淮,自淮入泗。沿,悅專反;鄭本作松,松當為沿;馬本作均,云:“均,平。”}

{\noindent\zhuan\zihao{6}\fzbyks 傳“順流”至“入泗”。正義曰:文十年\CJKunderwave{左傳}云:“沿漢溯江。”溯是逆,沿是順,故“順流而下曰沿”。“沿江入海”,順也。“自海入淮,自淮入泗”,逆也。 \par}

荊及衡陽惟荊州。\footnote{北據荊山,南及衡山之陽。}

{\noindent\zhuan\zihao{6}\fzbyks 傳“北據”至“之陽”。正義曰:此州北界至荊山之北,故言“據”也。“南及衡山之陽”,其境過衡山也。以衡是大山,其南無復有名山大川可以為記,故言“陽”見其南至山南也。 \par}

江、漢朝宗於海,\footnote{二水經此州而入海,有似於朝,百川以海為宗。宗,尊也。朝,直遙反。}

{\noindent\zhuan\zihao{6}\fzbyks 傳“二水”至“宗尊也”。正義曰:\CJKunderwave{周禮·大宗伯}諸侯見天子之禮,“春見曰朝,夏見曰宗”。鄭云:“朝猶朝也,欲其來之早也。宗,尊也,欲其尊王也。”“朝宗”是人事之名,水無性識,非有此義。以海水大而江、漢小,以小就大,似諸侯歸於天子,假人事而言之也。\CJKunderwave{詩}云:“沔彼流水,朝宗於海。”\CJKunderwave{毛傳}云:“水猶有所朝宗。”“朝宗”是假人事而言水也。\CJKunderwave{老子}云:“滄海所以能為百谷王者,以其下之。”是百川以海為宗。鄭云:“江水、漢水其流遄疾,又合為一,共赴海也。猶諸侯之同心尊天子而朝事之。荊楚之域,國有道則後服,國無道則先強,故記其水之義,以著人臣之禮。 \par}

九江孔殷,\footnote{江於此州界分為九道,甚得地勢之中。九江,\CJKunderwave{潯陽地記}云:“一曰烏白江,二曰蚌江,三曰烏江,四曰嘉靡江,五曰畎江,六曰源江,七曰累江,八曰提江,九曰箘江。”張須元\CJKunderwave{緣江圖}云:“一曰三里江,二曰五州江,三曰嘉靡江,四曰烏土江,五曰白蚌江,六曰白烏江,七曰箘江,八曰沙提江,九曰廩江。參差隨水長短,或百里,或五十里。始於鄂陵,終於江口,會於桑落洲。\CJKunderwave{太康地記}曰:“九江,劉歆以為湖漢九水,入彭蠡澤也。”}

{\noindent\zhuan\zihao{6}\fzbyks 傳“江於”至“之中”。正義曰:傳以“江”是此水大名,“九江”謂大江分而為九,猶大河分為九河,故言“江於此州之界分為九道”。訓“孔”為甚,“殷”為中,言“甚得地勢之中”也。鄭云:“殷猶多也。九江從山谿所出,其孔眾多,言治之難也。\CJKunderwave{地理志}九江在今廬江潯陽縣南,皆東合為大江。”如鄭此意,九江各自別源,其源非大江也,下流合於大江耳。然則江以南水無大小,俗人皆呼為江,或從江分出,或從外合來,故孔、鄭各為別解。應劭注\CJKunderwave{地理志}雲“江自潯陽分為九道”,符於孔說,\CJKunderwave{潯陽記}有九江之名:“一曰烏江,二曰蚌江,三曰烏白江,四曰嘉靡江,五曰畎江,六曰源江,七曰廩江,八曰提江,九曰箘江。”雖名起近代,義或當然。 \par}

沱、潛既道,\footnote{沱,江別名。潛,水名。皆復其故道。沱,徒河反。潛,捷廉反,馬云:“沱,湖也。其中泉出而不流者謂之潛。”}

{\noindent\zhuan\zihao{6}\fzbyks 傳“沱江”至“故道”。正義曰:下文“泯山導江,東別為沱”,是“沱”為江之別名也。經無“潛”之本源,直雲“水名”。\CJKunderwave{釋水}云,水自江出為沱,漢為潛。鄭注此,既引\CJKunderwave{爾雅},乃云:“今南郡枝江縣有沱水,其尾入江耳,首不於江出也。華容有夏水,首出江,尾入沔蓋,此所謂沱也。潛則未聞象類。”此解荊州之沱、潛發源此州。若如鄭言,此水南流,不入荊州界,非此潛也。此下樑州注云:“二水亦謂自江漢出者。\CJKunderwave{地理志}在今蜀郡鄆縣江沱及漢中安陽皆有沱水、潛水,其尾入江漢耳,首不於此出。江源有𨞪江,首出江,南至犍為武陽又入江,豈沱之類與?蓋漢西漢,出嶓冢,東南至巴郡江州入江,行二千七百六十里。”此解梁州之沱、潛也。郭璞\CJKunderwave{爾雅音義}云:“沱水自蜀郡都水縣揃山與江別而更流。”璞又云:“有水從漢中沔陽縣南流,至梓潼漢壽入大穴中,通峒山下西南潛出,一名沔水,舊俗雲即\CJKunderwave{禹貢}潛也。”郭璞此言,亦解梁州沱、潛,與鄭又異。然\CJKunderwave{地理志}及鄭皆以荊、梁二州各有沱、潛,又郭氏所解沱、潛惟據梁州,不言荊州之沱、潛,而孔梁州注云“沱、潛發源此州,入荊州”,以二州沱、潛為一者。然彼州山水古今不可移易,孔為武帝博士,\CJKunderwave{地理志}無容不知,蓋以水從江漢出者皆曰“沱潛”,但地勢西高東下,雖於梁州合流,還從荊州分出,猶如濟水入河,還從河出,故孔舉大略為發源梁州耳。 \par}

雲土、夢作乂。\footnote{雲夢之澤在江南,其中有平土丘,水去可為耕作畎畝之治。云,徐本作雲。夢,亡弄反,一音武仲反,徐莫公反。治,直吏反。}

{\noindent\zhuan\zihao{6}\fzbyks 傳“雲夢”至“之治”。正義曰:昭三年\CJKunderwave{左傳}楚子與鄭伯田於江南之夢,是“雲夢之澤在江南”也。\CJKunderwave{地理志}南郡華容縣南有云夢澤,杜預雲“南郡枝江縣西有云夢城”,江夏安陸縣亦有云夢,或曰南郡華容縣東南有巴丘湖。江南之夢,雲夢一澤,而每處有名者,司馬相如\CJKunderwave{子虛賦}雲“雲夢者方八九百里”,則此澤跨江南北,每處名存焉。定四年\CJKunderwave{左傳}稱楚昭王寢於雲中,則此澤亦得單稱“雲”,單稱“夢”。經之“土”字在二字之間,蓋史文兼上下也。此澤既大,其內有平土,有高丘,水去可為耕作畎畝之治。 \par}

厥土惟塗泥,厥田惟下中,厥賦上下。\footnote{田第八,賦第三,人功修。}厥貢羽、毛、齒、革,惟金三品,\footnote{土所出與揚州同。}

{\noindent\zhuan\zihao{6}\fzbyks 傳“土所”至“州同”。正義曰:與揚州同,而揚州先“齒、革”,此州先“羽、毛”者,蓋以善者為先。由此而言之,諸州貢物多種,其次第皆以當州貴者為先也。 \par}

杶、幹、栝、柏,\footnote{幹,柘也。柏葉松身曰栝。杶,敕倫反,徐敕荀反,木名,又作櫄。幹,本又作幹,故旦反。栝,古活反。馬云:“白栝也。”柘,章夜反。}

{\noindent\zhuan\zihao{6}\fzbyks 傳“幹柘”至“曰栝”。正義曰:“幹”為弓幹,\CJKunderwave{考工記}云,弓人取幹之道也,以柘為上,知此“幹”是柘也。\CJKunderwave{釋木}云:“栝,柏葉松身。”陸機\CJKunderwave{毛詩義疏}雲“杶、㯉、栲、漆相似如一”,則杶似㯉漆也。杶、栝、柏皆木名也,以其所施多矣,柘木惟用為弓幹,弓幹莫若柘木,故舉其用也。 \par}

礪、砥、砮、丹,\footnote{砥細於礪,皆磨石也。砮,石,中矢鏃。丹,朱類。礪,力世反。砥音脂,徐之履反,韋昭音旨。砮音奴,韋昭乃固反。磨,末佐反。鏃,子木反,一音七木反。}

{\noindent\zhuan\zihao{6}\fzbyks 傳“砥細”至“朱類”。正義曰:“砥”以細密為名,“礪”以粗糲為稱,故“砥細於礪,皆磨石也”。鄭云:“礪,磨刀刃石也。精者曰砥。”\CJKunderwave{魯語}曰:“肅慎氏貢楛矢石砮。”賈逵云:“砮,矢鏃之石也。”故曰“砮,石,中矢鏃”。“丹”者,丹砂,故云“朱類”。王肅云:“丹可以為採。” \par}

惟菌、簵、楛,三邦厎貢厥名。\footnote{箘、簵,美竹。楛中矢幹。三物皆出雲夢之澤,近澤三國常致貢之,其名天下稱善。箘,求隕反,韋昭一名聆風。簵音路。楛音戶,馬云:“木名,可以為箭。”\CJKunderwave{毛詩草木疏}云:“葉如荊而赤,莖似蓍。”近,附近之近。}

{\noindent\zhuan\zihao{6}\fzbyks 傳“箘簵”至“稱善”。正義曰:“箘、簵,美竹”,當時之名猶然。鄭云:“箘簵,䉁風也。”竹有二名,或大小異也,箘、簵是兩種竹也。“肅慎氏貢楛矢”,知“楛中矢幹”。“三物皆出雲夢之澤”,當時驗之猶然。經言“三邦厎貢”,知近澤三國致此貢也。文續“”,則其物特有美名,故云“其名天下稱善”。\CJKunderline{鄭玄}以“厥名”下屬“包匭菁茅”。 \par}

包\footnote{橘柚。}匭菁茅,\footnote{匭,匣也。菁以為菹,茅以縮酒。匭音軌。菁,子丁反,徐音精,馬同,鄭云:“茅有毛刺曰菁茅。”匣,胡甲反,又音甲。菹,\CJKunderwave{切韻}側魚反。縮,所六反。}

{\noindent\zhuan\zihao{6}\fzbyks 傳“橘柚”。正義曰:“包”下言“匭菁茅”,\CJKunderwave{說文}云:“匚,受物之器。象形也。凡匚之屬皆從匚。”“匱”、“匣”之字皆從匚,“匭”亦從匚,故“匭”是匣也。“菁茅”既以匭盛,非所包之物,明包必有裹也。此州所出與揚州同,揚州“厥包橘柚”,知此“包”是橘柚也。王肅云:“揚州‘厥包橘柚’,從省而可知也。” \par}

{\noindent\zhuan\zihao{6}\fzbyks 傳“匭,匣也。菁以為菹,茅以縮酒”。正義曰:“匭”是匱之別名,匱之小者。菁茅所盛,不須大匱,故用匣也。\CJKunderwave{周禮·醢人}有“菁菹”、“鹿臡”,故知“菁以為菹”。鄭云:“菁,蔓菁也。”蔓菁處處皆有,而令此州貢者,蓋以其味善也。僖四年\CJKunderwave{左傳}齊桓公責楚雲“爾貢包茅不入,王祭不供,無以縮酒”,是“茅以縮酒”也。\CJKunderwave{郊特牲}云:“縮酒用茅,明酌也。”鄭注云:“以茅縮酒也。”,\CJKunderwave{周禮·甸師}云:“祭祀供蕭茅。”鄭興云:“蕭字或為莤,莤讀為縮,束茅立之祭前,酒沃其上,酒滲下,若神飲之,故謂之縮。”杜預解\CJKunderwave{左傳},用鄭興之說,未知誰同孔旨。特令此州貢茅,茅當異於諸處。杜預云:“茅之為異,未審也。”或雲茅有三脊,案\CJKunderwave{史記}齊桓公欲封禪,管仲睹其不可窮以辭,因設以無然之事云:“古之封禪,江淮之間,三脊茅以為藉。”此乃懼桓公耳,非荊州所有也。\CJKunderline{鄭玄}以“菁茅”為一物,“匭猶纏結也。菁茅之有毛刺者重之,故既包裹而又纏結也”。 \par}

厥篚玄纁、璣組,\footnote{此州染玄纁色善,故貢之。璣,珠類,生於水。組,綬類。纁,許雲反。璣,其依反,又音機,馬同,\CJKunderwave{說文}云:“珠不圜也。”\CJKunderwave{字書}云:“小珠也。”\CJKunderwave{玉篇}渠依、居沂二反。組音祖,馬云:“組,文也。”}

{\noindent\zhuan\zihao{6}\fzbyks 傳“此州”至“綬類”。正義曰:\CJKunderwave{釋器}云:“三染謂之纁。”李巡云:“三染其色已成為絳,纁、絳一名也。”\CJKunderwave{考工記}云:“三入為纁,五入為緅,七入為緇。”鄭云:“纁者三入而成,又再染以黑則為緅,又再染以黑則為緇。玄色在緅、緇之間,其六人者是染玄纁之法也。”此州染玄纁色善,故令貢之。\CJKunderwave{說文}雲“璣,珠不圓者”,故為“珠類”。\CJKunderwave{玉藻}說佩玉所懸者皆雲“組綬”,是組、綬相類之物也。 \par}

九江納錫大龜。\footnote{尺二寸曰大龜,出於九江水中。龜不常用,錫命而納之。馬云:“納,入也。”}

{\noindent\zhuan\zihao{6}\fzbyks 傳“尺二”至“納之”。正義曰:\CJKunderwave{史記·龜策傳}云:“龜千歲滿尺二寸”,\CJKunderwave{漢書·食貨志}雲“元龜距髯長尺二寸”,故以“尺二寸為大龜”。冠以“九江”,知“出九江水中”也。文在“篚”下而言“納錫”,是言“龜不常用,故錫命乃納之”,言此大龜錫命乃貢之也。 \par}

浮於江、沱、潛、漢,逾於洛,至於南河。\footnote{逾,越也。河在冀州,南東流,故越洛而至南河。江、沱、潛、漢,四水名。本或作“潛於漢”,非。逾,羊朱反。}

{\noindent\shu\zihao{5}\fzkt “浮於江、沱、潛、漢”。正義曰:浮此四水乃得至洛。本或“潛”下有“於”,誤耳。 \par}

荊、河惟\textcolor{red}{豫州}。\footnote{西南至荊山,北距河水。}伊、洛、瀍、澗既入於河,\footnote{伊出陸渾山,洛出上洛山,澗出沔池山,瀍出河南北山,四水合流而入河。瀍,直然反。澗,故晏反。渾音魂,又胡囷、胡昆二反。沔,亡淺反,又亡忍反,下同。陸渾、沔池二縣屬河南郡。}

{\noindent\zhuan\zihao{6}\fzbyks 傳“伊出”至“入河”。正義曰:\CJKunderwave{地理志}云,伊水出弘農盧氏縣東熊耳山,東北入洛。洛水出弘農上洛縣冢領山,東北至鞏縣入河。瀍水出河南谷城縣潛亭北,東南入洛。澗水出弘農新安縣,東南入洛。\CJKunderwave{志}與傳異者,熊耳山在陸渾縣西,冢領山在上洛縣境之內,沔池在新安縣西、谷城潛亭北,此即是河南境內之北山也。\CJKunderwave{志}詳而傳略,所據小異耳。伊、瀍、澗三水入洛合流而入河,言其不復為害也。 \par}

滎波既豬,\footnote{滎澤波水已成遏豬。zhuan,戶扃反;6澤也。波如字,馬本又播;滎播,澤名。遏,byks反。}

{\noindent\shu\zihao{5}\fzkt 傳“滎澤”至“遏豬”。正義曰:沇水入河而溢為滎,“滎”是澤名。洪水之時,此澤水大,動成波浪。此澤其時波水已成遏豬,言壅遏而為豬,畜水而成澤,不濫溢也。鄭云:“今塞為平地,滎陽民猶謂其處為滎澤,在其縣東。”言在滎澤縣之東也。馬、鄭王本皆作“滎播”,謂此澤名“滎播”。\CJKunderwave{春秋}閔二年衛侯“及狄人戰於滎澤”,不名“播”也。\CJKunderline{鄭玄}謂衛狄戰在此地,杜預雲“此滎澤當在河北,以衛敗方始渡河,戰處必在河北”。蓋此澤跨河南北,多而得名耳。 \par}

導菏澤,被孟豬。\footnote{菏澤在胡陵。孟豬,澤名,在菏東北,水流溢覆被之。導音道,下同。菏,徐音柯,又土可反,注同,韋胡阿反。被,皮寄反,徐扶義反,注同。豬,張魚反,又音諸;\CJKunderwave{左傳}及\CJKunderwave{爾雅}皆作“孟諸”,宋藪澤也。}

{\noindent\zhuan\zihao{6}\fzbyks 傳“菏澤”至“被之”。正義曰:\CJKunderwave{地理志}山陽郡有胡陵縣,不言其縣有菏澤也。又云,菏澤在濟陰定陶縣東。孟豬在梁國雎陽縣東北。以今地驗之,則胡陵在雎陽之東,定陶在雎陽之北,其水皆不流溢東北被孟豬也。然郡縣之名,隨代變易,古之胡陵當在雎陽之西北,故得東出被孟豬也。於此作“孟豬”,\CJKunderwave{左傳}、\CJKunderwave{爾雅}作“孟諸”,\CJKunderwave{周禮}作“望諸”,聲轉字異,正是一地也。 \par}

厥土惟壤,下土墳壚。\footnote{高者壤,下者壚,壚疏。壚音盧,\CJKunderwave{說文}:“黑剛土也。”}厥田惟中上,厥賦錯上中。\footnote{田第四,賦第二,又雜出第一。}厥貢漆、枲、絺、紵,厥篚纖纊,\footnote{纊,細綿。絺,敕其反。紵,直呂反。纊音曠。綿,\CJKunderwave{切韻}武延反。}

{\noindent\zhuan\zihao{6}\fzbyks 傳“纊,細綿”。正義曰:\CJKunderwave{禮·喪大記}候死者“屬纊以俟絕氣”,即“纊”是新綿耳。“纖”是細,故言“細綿”。 \par}

錫貢磬錯。\footnote{治玉石曰錯。治磬錯。}

{\noindent\zhuan\zihao{6}\fzbyks 傳“治玉”至“磬錯”。正義曰:\CJKunderwave{詩}云:“佗山之石,可以攻玉。”又曰:“可以為錯。”磬有以玉為之者,故云“治玉石曰錯”,謂“治磬錯”也。 \par}

浮於洛,達於河。

華陽、黑水惟\textcolor{red}{梁州}。\footnote{東據華山之南,西距黑水。華,胡化反,又胡瓜反。}

{\noindent\zhuan\zihao{6}\fzbyks 傳“東據”至“黑水”。正義曰:\CJKunderwave{周禮·職方氏}豫州其山鎮曰華山,在豫州界內。此梁州之境東據華山之南,不得其山,故言“陽”也。此山之西,雍州之境也。 \par}

岷、嶓既藝,沱、潛既道。\footnote{岷山、嶓冢皆山名。水去已可種藝。沱、潛發源此州,入荊州。岷,武巾反。嶓音波,徐甫河反,韋音播。}

{\noindent\zhuan\zihao{6}\fzbyks 傳“岷山”至“荊州”。正義曰:漢制,縣有羌夷曰道。\CJKunderwave{地理志}云,蜀郡有湔道,岷山在西徼外,江水所出也。隴西郡西縣冢山,西漢水所出。是二者皆山名也。沱出於江,潛出於漢,二水發源此州而入荊州,故荊州亦云“沱、潛既道”。 \par}

蔡、蒙旅平,和夷厎績。\footnote{蔡,蒙二山名。祭山曰旅。平言治功畢。和夷之地,致功可藝。旅如字,韋音盧。和如字,又作龢,鄭云:“和讀曰洹。”治,直吏反,下同。}

{\noindent\zhuan\zihao{6}\fzbyks 傳“蔡蒙”至“可藝”。正義曰:\CJKunderwave{地理志}云,蒙山在蜀郡青衣縣。應劭云:“順帝改曰漢嘉縣。”蔡山不知所在。\CJKunderwave{論語}云:“季氏旅於泰山。”是“祭山曰旅”也。“平”者言其治水畢,猶上“既藝”也。“和夷”平地之名,致功可藝。“藝”與“平”互言耳。 \par}

厥土青黎,\footnote{色青黑而沃壤。黎,鄭力兮反,徐力私反,馬云:“小疏也。”}

{\noindent\zhuan\zihao{6}\fzbyks 傳“色青黑而沃壤”。正義曰:孔以“黎”為黑,故云“色青黑”。其地“沃壤”,言其美也。王肅曰:“青,黑色。黎,小疏也。” \par}

厥田惟下上,厥賦下中三錯。\footnote{田第七,賦第八,雜出第七、第九三等。}

{\noindent\zhuan\zihao{6}\fzbyks 傳“田第”至“三等”。正義曰:傳以既言“下中”,復雲“三錯”,舉下中第八為正,上下取一,故雜出第七、第九與第八為三也。鄭云:“三錯者,此州之地有當出下之賦者少耳,又有當出下上、中下者差復益小。”與孔異也。 \par}

厥貢璆、鐵、銀、鏤、砮、磬,\footnote{璆,玉名。鏤,剛鐵。璆音蚪,徐又居蚪反,又閭幼反,馬同,韋昭、郭璞云:“紫磨金。”案郭注\CJKunderwave{爾雅}璆即紫磨金。鐵,天結反。鏤,婁豆反。}

{\noindent\zhuan\zihao{6}\fzbyks 傳“璆玉”至“剛鐵”。正義曰:\CJKunderwave{釋器}云:“璆、琳,玉也。”郭璞云:“璆、琳,美玉之別名。”“鏤”者,可以刻鏤,故為“剛鐵”也。 \par}

熊、羆、狐、貍織皮。\footnote{貢四獸之皮,織金罽。熊音雄。羆,彼宜反,如熊而黃。貍,力疑反。罽,紀例反。}

{\noindent\zhuan\zihao{6}\fzbyks 傳“貢四”至“金罽”。正義曰:與“織皮”連文,必不貢生獸,故云“貢四獸之皮”。\CJKunderwave{釋言}云:“犛,罽也。”舍人曰:“犛謂毛罽也。胡人續羊毛作衣。”孫炎曰:“毛犛為罽。”織毛而言“皮”者,毛附於皮,故以“皮”表毛耳。 \par}

西傾因桓是來,浮於潛,逾於沔,\footnote{西傾,山名。桓水自西傾山南行,因桓水是來,浮於潛。漢上曰沔。傾,窺井反。}

{\noindent\zhuan\zihao{6}\fzbyks 傳“西傾”至“曰沔”。正義曰:下文導山有“西傾”,知是山名也。\CJKunderwave{地理志}云,西傾在隴西臨洮縣西南。西傾在雍州,自西傾山南行,因桓水是來,浮於潛水也。\CJKunderwave{地理志}云,桓水出蜀郡蜀山,西南行羌中,入南海,則初發西傾未有水也,不知南行幾里得桓水也。下傳雲“泉始出山為漾水,東南流為沔水,至漢中東行為漢水”,是“漢上曰沔”。 \par}

入於渭,亂於河。\footnote{越沔而北入渭,浮東渡河而還帝都,白所治。正絕流曰亂。渭音謂。}

{\noindent\zhuan\zihao{6}\fzbyks 傳“越沔”至“曰亂”。正義曰:計沔在渭南五百餘里,故越沔陸行而北入渭。渭水入河,故浮渭而東。帝都在河之東,故渡河陸行而還帝都也。以每州之下言入河之事,河近帝都,知是還都白所治也。“正絕流曰亂”,\CJKunderwave{釋水}文。孫炎曰:“橫渡也。” \par}

黑水、西河惟\textcolor{red}{雍州}。\footnote{西距黑水,東據河。龍門之河在冀州西。雍,於用反。}

{\noindent\zhuan\zihao{6}\fzbyks 傳“西距”至“州西”。正義曰:禹治豫州,乃次梁州,自東向西,故言梁州之境,先以華陽而後黑水。從梁適雍,自南向北,故先黑水而後西河。計雍州之境,被荒服之外,東不越河,而西逾黑水,王肅雲“西據黑水,東距西河”,所言得其實也。遍檢孔本,皆雲“西距黑水,東據河”,必是誤也。又河在雍州之東,而謂之“西河”者,龍門之河在冀州西界,故謂之“西河”。\CJKunderwave{王制}云:“自東河至於西河,千里而近。”是河相對而為東西也。 \par}

弱水既西,\footnote{導之西流,至於合黎。}

{\noindent\zhuan\zihao{6}\fzbyks 傳“導之”至“合黎”。正義曰:諸水言“既導”,此言“既西”,由地勢不同,導之使西流也。鄭云:“眾水皆東,此水獨西,故記其西下也。” \par}

涇屬渭汭。\footnote{屬,逮也。水北曰汭。言治涇水入於渭。涇音經。屬,之蜀反。汭,本又作內,同;如銳反,馬云:“入也。”逮音代。}

{\noindent\zhuan\zihao{6}\fzbyks 傳“屬逮”至“於渭”。正義曰:“屬”謂相連屬,故訓為逮。逮,及也,言水相及。\CJKunderwave{詩}毛傳云:“汭,水涯也。”鄭云:“汭之言內也。”蓋以人皆南面望水,則北為汭也。且涇水南入渭,而名為“渭汭”,知“水北曰汭”。言治涇水使之入渭,亦是從故道也。\CJKunderwave{地理志}云,涇水出安定涇陽縣西岍頭山,東南至馮翊陽陵縣入渭,行千六百里。 \par}

漆沮既從,灃水攸同。\footnote{漆沮之水,已從入渭。灃水所同,同於渭也。沮,七徐反。灃,芳弓反。}

{\noindent\zhuan\zihao{6}\fzbyks 傳“漆沮”至“於渭”。正義曰:\CJKunderwave{詩}云:“自土沮漆。”\CJKunderwave{毛傳}云:“沮水、漆水也。”則“漆沮”本為二水。\CJKunderwave{地理志}云,漆水出扶風漆縣西。闞駰\CJKunderwave{十三州志}云:“漆水出漆縣西北岐山,東入渭。”沮則不知所出,蓋東入渭時已與漆合。渭發源遠,以渭為主,上雲“涇屬渭”是矣。故此言“漆沮既從”,已從於渭;灃水所同,亦同於渭;以渭為主故也。\CJKunderwave{地理志}灃水出扶風鄠縣東南,北過上林苑入渭也。 \par}

荊、岐既旅,\footnote{已旅祭,言治功畢。此荊在岐東,非荊州之荊。治,直吏反。}

{\noindent\zhuan\zihao{6}\fzbyks 傳“已旅”至“之荊”。正義曰:洪水之時,祭祀禮廢,已旅祭而言治功畢。治水從下,自東而西,先荊後岐,荊在岐東,嫌與上荊為一,故云“非荊州之荊”也。\CJKunderwave{地理志}云,\CJKunderwave{禹貢}北條荊山在馮翊懷德縣南,南條荊山在南郡臨沮縣北。彼是荊州之荊也。 \par}

終南、惇物,至於鳥鼠。\footnote{三山名,言相望。終南,山名。\CJKunderwave{漢書·地理志}一名太一。\CJKunderwave{山秦記}云:“又名地肺。”惇物,山名,\CJKunderwave{漢書}云:“垂山也。”}

{\noindent\zhuan\zihao{6}\fzbyks 傳“三山”至“相望”。正義曰:以“荊”、“岐”單名,此山複名,故辯之雲“三山名”也。至於為首尾之辭。故“言相望”也。三山空舉山名,不言治意,蒙上“既旅”之文也。\CJKunderwave{地理志}云,扶風武功縣有太一山,古文以為終南。垂山,古文以為惇物。皆在縣東。 \par}

原隰厎績,至於豬野。\footnote{下溼曰隰。豬野,地名。言皆致功。}

{\noindent\zhuan\zihao{6}\fzbyks 傳“下溼”至“致功”。正義曰:“下溼曰隰”,\CJKunderwave{釋地}文。\CJKunderwave{地理志}云,豬野澤,在武威縣東北有休屠澤,古文以為豬野澤。\CJKunderline{鄭玄}以為“\CJKunderwave{詩}雲‘度其隰原’,即此‘原隰’是也。原隰,豳地。從此致功,西至豬野之澤也”。 \par}

三危既宅,三苗丕敘。\footnote{西裔之山已可居,三苗之族大有次敘。美禹之功。丕,普悲反。}

{\noindent\zhuan\zihao{6}\fzbyks 傳“西裔”至“之功”。正義曰:\CJKunderwave{左傳}稱舜去四凶,投之四裔,\CJKunderwave{舜典}雲“竄三苗於三危”,是“三危”為西裔之山也。其山必是西裔,未知山之所在。\CJKunderwave{地理志}杜林以為敦煌郡,即古瓜州也。昭九年\CJKunderwave{左傳}云:“先王居檮杌於四裔,故允姓之奸居於瓜州。”杜預云:“允姓之祖與三苗俱放於三危。瓜州,今敦煌也。”\CJKunderline{鄭玄}引\CJKunderwave{地記書}云:“三危之山在鳥鼠之西,南當岷山,則在積石之西南。”\CJKunderwave{地記}乃妄書,其言未必可信。要知三危之山必在河之南也。禹治水未已竄三苗,水災既除,彼得安定,故云三危之山已可居,三苗之族大有次敘,記此事以美禹治之功也。 \par}

厥土惟黃壤,厥田惟上上,厥賦中下。\footnote{田第一,賦第六,人功少。}

{\noindent\zhuan\zihao{6}\fzbyks 傳“田第一”至“功少”。正義曰:此與荊州賦田升降皆較六等,荊州升之極,故云“人功修”;此州降之極,故云“人功少”。其餘相較少者,從此可知也。\CJKunderwave{王制}云:“凡居民,量地以致邑,度地以居民,地、邑、民居必參相得也。”則民當相準,而得有人功修、人功少者,\CJKunderwave{記}言初置邑者,可以量之,而州境闊遠,民居先定,新遭洪水,存亡不同,故地勢有美,惡人功有多少。治水之後即為此差,在後隨人少多必得更立其等,此非永定也。 \par}

厥貢惟球、琳、琅玕。\footnote{球、琳皆玉名。琅玕,石而似珠。球音求。琳,韋音來金反。琅音郎。玕音幹,\CJKunderwave{山海經}云:“崑崙山有琅玕樹。”}

{\noindent\zhuan\zihao{6}\fzbyks 傳“球琳”至“似珠”。正義曰:\CJKunderwave{釋地}云:“西北之美者,有崑崙虛之璆琳琅玕焉。”說者皆云:“球、琳,美玉名。琅玕,石而似珠者。”必相傳驗,實有此言也。 \par}

浮於積石,至於龍門西河,\footnote{積石山在金城西南,河所經也。沿河順流而北,千里而東,千里而南。龍門山在河東之西界。}

{\noindent\zhuan\zihao{6}\fzbyks 傳“積石”至“西界”。正義曰:\CJKunderwave{地理志}云,積石山在金城河關縣西南羌中,河行塞外東,北入塞內。積石非河之源,故云“河所經也”。河從西來,至此北流,故禹“沿河順流而北”。\CJKunderwave{釋水}云:“河千里一曲一直。”故千里而東,千里而南,至於龍門西河也。\CJKunderwave{地理志}云,龍門山在馮翊夏陽縣北。此山當河之道,禹鑿以通河東郡之西界也。禹至此渡河而還都白帝也。“沿”或誤為“治”,此說禹行,不說治水也。 \par}

會於渭汭。\footnote{逆流曰會。自渭北涯逆水西上。上,時掌反。}

{\noindent\zhuan\zihao{6}\fzbyks 傳“逆流”至“西上”。正義曰:“會”,合也。人行逆流而水相向,故“逆流曰會”。從河入渭,自渭北涯逆水西上,言禹白帝訖,從此而西上,更入雍州界也。諸州之末,惟言還都之道,此州事終,言發都更去,明諸州皆然也。 \par}

織皮崑崙、析支、渠、搜、西戎即敘。\footnote{織皮,毛布。有此四國,在荒服之外,流沙之內,羌髳之屬皆就次敘。美禹之功及戎狄也。侖,曾門反,馬云:“崑崙在臨羌西。”析,星曆反,馬云:“析支在河關西。”搜,所由反。\CJKunderwave{漢書·志}朔方郡有渠搜縣,\CJKunderwave{武紀}雲“北發渠搜”是也。髳音謀,又音毛。西戎,國名。}

{\noindent\zhuan\zihao{6}\fzbyks 傳“織皮”至“戎狄也”。正義曰:四國皆衣皮毛,故以“織皮”冠之。傳言“織皮,毛布。有此四國”,崑崙也,析支也,渠也,搜也,四國皆是戎狄也。末以“西戎”總之。此戎在荒服之外,流沙之內。\CJKunderwave{牧誓}云,武王伐紂,有羌髳從之。此是羌髳之屬,禹皆就次敘。美禹之功遠及戎狄,故記之也。\CJKunderline{鄭玄}云:“衣皮之民,居此崑崙、析支、渠搜三山之野者,皆西戎也。”王肅云:“崑崙在臨羌西,析支在河關西。西戎,西域也。”王肅不言“渠搜”,鄭並“渠搜”為一,孔傳不明。或亦以“渠搜”為一,通“西戎”為四也。鄭以“崑崙”為山,謂別有崑崙之山,非河所出者也。所以孔意或是地名國號,不必為山也。 \par}

導岍及岐,至於荊山,\footnote{更理說所治山川首尾所在,治山通水,故以山名之。三山皆在雍州。導音道,從首起也。岍音牽,字又作汧,山名,一名吳嶽,馬本作開。}

{\noindent\zhuan\zihao{6}\fzbyks 傳“更理”至“雍州”。正義曰:“荊”、“岐”上已具矣,而此復言之,以山勢相連而州境隔絕,更從上理說所治山川首尾所在,總解此下導山水之意也。其實通水而文稱導山者,導山本為治水,故以導山名之。\CJKunderwave{地理志}云,吳嶽在扶風岍縣西,古文以為岍山,岐山在美陽縣西北,荊山在懷德縣。三山皆在雍州。 \par}

{\noindent\shu\zihao{5}\fzkt “導岍及岐”。正義曰:上文每州說其治水登山,從下而上,州境隔絕,未得徑通。今更從上而下,條說所治之山,本以通水,舉其山相連屬,言此山之傍所有水害皆治訖也。因冀州在北,故自北為始。從此“導岍”至“敷淺原”,舊說以為三條。\CJKunderwave{地理志}云,\CJKunderwave{禹貢}北條荊山,在馮翊懷德縣南,南條荊山,在南郡臨沮縣東北。是舊有三條之說也。故馬融、王肅皆為三條,“導岍”北條,“西傾”中條,“嶓冢”南條。\CJKunderline{鄭玄}以為四列,“導岍”為陰列,“西傾”為次陰列,“嶓冢”為次陽列,“岷山”為正陽列。\CJKunderline{鄭玄}創為此說,孔亦當為三條也。“岍”與“嶓冢”言“導”,“西傾”不言“導”者,史文有詳略,以可知,故省文也。 \par}

逾於河。\footnote{此謂梁山龍門西河。}

{\noindent\zhuan\zihao{6}\fzbyks 傳“此謂”至“西河”。正義曰:“逾於河”謂山逾之也。此處山勢相望,越河而東,故云此謂龍門西河,言此處山不絕,從此而渡河也。 \par}

壺口、雷首,至於太嶽。\footnote{三山在冀州。太嶽,上黨西。}

{\noindent\zhuan\zihao{6}\fzbyks 傳“三山”至“黨西”。正義曰:\CJKunderwave{地理志}云,壺口在河東北屈縣東南,雷首在河東蒲阪縣南,太嶽在河東彘縣東。是“三山在冀州”。以太嶽東近上黨,故云“在上黨西”也。 \par}

厎柱、析城,至於王屋。\footnote{此三山在冀州南河之北東行。厎,之履反。柱如字,韋知父反,又知女反。厎柱,山名,在河水中。}

{\noindent\zhuan\zihao{6}\fzbyks 傳“此三”至“東行”。正義曰:\CJKunderwave{地理志}云,析城在河東濩澤縣西,王屋在河東垣縣東北。\CJKunderwave{地理志}不載厎柱。厎柱在太陽關東,析城之西。從厎柱至王屋,在冀州南河之北東行也。 \par}

大行、恆山,至於碣石,入於海。\footnote{此二山連延東北,接碣石而入滄海。百川經此眾山,禹皆治之,不可勝名,故以山言之。行,戶剛反,又如字。滄音倉。勝音升。}

{\noindent\zhuan\zihao{6}\fzbyks 傳“此二”至“言之”。正義曰:\CJKunderwave{地理志}云,大行山在河內山陽縣西北,恆山在常山上曲陽縣西北。大行去恆山太遠,恆山去碣石又遠,故云“此二山連延東北,接碣石而入滄海”,言山傍之水皆入海,山不入海也。又解治水言山之意,“百川經此眾山,禹皆治之,川多不可勝名,故以山言之”也。謂漳、潞、汾、涑在壺口、雷首、太行,經厎柱、析城,濟出王屋,淇近大行,恆、衛、滹沲、滱、易近恆山、碣石之等也。 \par}

西傾、朱圉、鳥鼠,\footnote{西傾、朱圉在積石以東。鳥鼠,渭水所出,在隴西之西。三者雍州之南山。傾,窺井反。圉,魚呂反。}

{\noindent\zhuan\zihao{6}\fzbyks 傳“西傾”至“南山”。正義曰:\CJKunderwave{地理志}云,西傾在隴西臨洮縣西南,朱圉在天水冀縣南。言“在積石以東”,見河所經也。\CJKunderwave{地理志}云,鳥鼠同穴山,在隴西首陽縣西南,渭水所出,在隴西郡之西。是三者皆雍州之南山也。 \par}

至於太華。\footnote{相首尾而東。華如字,又戶化反。}

{\noindent\zhuan\zihao{6}\fzbyks 傳“相首尾而東”。正義曰:\CJKunderwave{地理志}云,太華在京兆華陰縣南。鳥鼠東望太華太遠,故云“相首尾而東”也。 \par}

熊耳、外方、桐柏,至於陪尾。\footnote{四山相連,東南在豫州界。洛經熊耳,伊經外方,淮出桐柏,經陪尾。凡此皆先舉所施功之山於上,而後條列所治水於下,互相備。陪音裴;陪尾,山名;\CJKunderwave{漢書}作“橫尾”。列如字,本或作別,彼列反。}

{\noindent\zhuan\zihao{6}\fzbyks 傳“四山”至“相備”。正義曰:\CJKunderwave{地理志}云,熊耳山在弘農盧氏縣東,伊水所出。嵩高山在穎川嵩高縣,古文以為外方山。桐柏山在南陽平氏縣東南。橫尾山在江夏安陸縣東北,古文以為陪尾山。是四山接華山而相連,東南皆在豫州界也。凡舉山名,皆為治水,故言水之所經,“洛出熊耳,伊經外方,淮出桐柏,經陪尾”。導山本為治水,故云“皆先舉所施功之山於上,而後條列所治水於下,互相備”也。 \par}

導嶓冢,至於荊山。\footnote{漾水出嶓冢,在梁州,經荊山。荊山在荊州。漾,羊尚反。}

{\noindent\zhuan\zihao{6}\fzbyks 傳“漾水”至“荊州”。正義曰:下云:“嶓冢導漾”,梁州雲“岷、嶓既藝”,是嶓冢在梁州也。荊州以荊山為名,知“荊山在荊州”也。 \par}

內方,至於大別。\footnote{內方、大別,二山名。在荊州,漢所經。}

{\noindent\zhuan\zihao{6}\fzbyks 傳“內方”至“所經”。正義曰:\CJKunderwave{地理志}云,章山在江夏竟陵縣東北,古文以為內方山。\CJKunderwave{地理志}無大別。\CJKunderline{鄭玄}云:“大別在廬江安豐縣。”杜預解\CJKunderwave{春秋}云:“大別闕,不知何處。”或曰大別在安豐縣西南,\CJKunderwave{左傳}云,吳既與楚夾漢,然後楚“乃濟漢而陳,自小別至於大別”。然則二別近漢之名,無緣得在安豐縣。如預所言,雖不知其處,要與內方相接,漢水所經,必在荊州界也。 \par}

岷山之陽,至於衡山。\footnote{岷山,江所出,在梁州。衡山,江所經,在荊州。}

{\noindent\zhuan\zihao{6}\fzbyks 傳“岷山”至“荊州”。正義曰:其下雲“岷山導江”,梁州“岷、嶓既藝”,是岷山在梁州也。\CJKunderwave{地理志}云,衡山在長沙湘南縣東南,上言“衡陽惟荊州”,是“江所經,在荊州”也。 \par}

過九江,至於敷淺原。\footnote{言衡山連延過九江,接敷淺原。言“導”從首起,言“陽”從南。敷淺原,一名博陽山,在揚州豫章界。}

{\noindent\zhuan\zihao{6}\fzbyks 傳“言衡”至“章界”。正義曰:“衡”即橫也,東西長,今之人謂之為嶺。東行連延過九江之水,而東接於敷淺原之山也。經於岍及嶓冢言“導”,岷山言“陽”,故解之“言‘導’從首起,言‘陽’從南”,言岷山之南至敷淺原,別以岷山為首,不與大別相接。由江所經,別記之耳,以見岷非三條也。\CJKunderwave{地理志}豫章歷陵縣南有博陽山,古文以為敷淺原。 \par}

導弱水,至於合黎,\footnote{合黎,水名,在流沙東。弱,本或作溺。合如字。黎,力兮反,馬云:“地名。”}

{\noindent\zhuan\zihao{6}\fzbyks 傳“合黎”至“沙東”。正義曰:弱水得入合黎,知“合黎”是水名。顧氏云:“\CJKunderwave{地說書}合黎,山名。”但此水出合黎,因山為名。\CJKunderline{鄭玄}亦以為山名。\CJKunderwave{地理志}張掖郡刪丹縣,桑欽以為導弱水自此,西至酒泉、合黎。張掖郡又有居延澤,在縣東北,古文以為流沙。如\CJKunderwave{志}之言,酒泉郡在張掖郡西,居延屬張掖,合黎在酒泉,則流沙在合黎之東,與此傳不合。案經弱水西流,水既至於合黎,餘波入於流沙,當如傳文合黎在流沙之東,不得在其西也。 \par}

{\noindent\shu\zihao{5}\fzkt “導弱水”。正義曰:此下所導,凡有九水,大意亦自北為始。以弱水最在西北,水又西流,故先言之。黑水雖在河南,水從雍、梁西界南入南海,與諸水不相參涉,故又次之。四瀆江、河為大,河在北,故先言河也。漢入於江,故先漢後江。其濟發源河北,越河而南,與淮俱為四瀆,故次濟,次淮。其渭與洛俱入於河,故後言之。計流水多矣,此舉大者言耳。凡此九水,立文不同,弱水、黑水、沇水不出于山,文單,故以“水”配。其餘六水,文與山連,既繫於山,不須言“水”。積石山非河上源,記施功之處,故云“導河積石”,言發首積石起也。漾、江先山後水,淮、渭洛先水後山,皆是史文詳略,無義例也。又淮、渭、洛言“自某山”者,皆是發源此山,欲使異於導河,故加“自”耳。\CJKunderline{鄭玄}云:“凡言‘導’者,發源於上,未成流。凡言‘自’者,亦發源於上,未成流。”必其俱未成流,何須別“導”與“自”?河出崑崙,發源甚遠,豈至積石,猶未成流而云“導河”也? \par}

餘波入於流沙。\footnote{弱水餘波西溢入流沙。溢音逸。}導黑水,至於三危,入於南海。\footnote{黑水自北而南,經三危,過樑州,入南海。}

{\noindent\zhuan\zihao{6}\fzbyks 傳“黑水”至“南海”。正義曰:\CJKunderwave{地理志}益州郡計在蜀郡西南三千餘里,故滇王國也。武帝元封二年始開為郡。郡內有滇池縣,縣有黑水祠,止言有其祠,不知水之所在。鄭云:“今中國無也。”傳之此言,順經文耳。案酈元\CJKunderwave{水經}:“黑水出張掖雞山,南流至敦煌,過三危山,南流入於南海。”然張掖、敦煌並在河北,所以黑水得越河入南海者,河自積石以西皆多伏流,故黑水得越而南也。 \par}

導河積石,至於龍門;\footnote{施功發於積石,至於龍門,或鑿山,或穿地以通流。}

{\noindent\zhuan\zihao{6}\fzbyks 傳“施功”至“通流”。正義曰:河源不始於此,記其施功處耳,故言“施功發於積石”。\CJKunderwave{釋水}雲“河千里一曲一直”,則河從積石北行,又東,乃南行至於龍門,計應三千餘里。龍門厎柱,鑿山也。其餘平地,穿地也。“或鑿山、或穿地以通流”,言自積石至海皆然也。\CJKunderwave{釋水}云:“河出崑崙虛,色白。”李巡曰:“崑崙,山名。虛,山下地也。”郭璞云:“發源高處激湊,故水色白。潛流地中,受渠眾多,渾濁,故水色黃。”\CJKunderwave{漢書·西域傳}云:“河有兩源,一出蔥嶺,一出於闐。于闐在南山下,其河北流,與蔥嶺河合,東注蒲昌海。蒲昌海,一名鹽澤者,去玉門、陽關三百餘里,廣袤三四百里。其水停居,冬夏不增減,皆以為潛行地下,南出於積石,為中國河。”郭璞云:“其去崑崙,裡數遠近未得詳也。” \par}

南至於華陰,\footnote{河自龍門南流至華山,北而東行。}東至於厎柱;\footnote{厎柱,山名。河水分流,包山而過,山見水中若柱然,在西虢之界。見,賢遍反。虢,寡白反。}又東至於孟津,\footnote{孟津,地名。在洛北,都道所湊,古今以為津。孟津,如字。洛北,地名。湊,七豆反。}

{\noindent\zhuan\zihao{6}\fzbyks 傳“孟津”至“為津”。正義曰:“孟”是地名,“津”是渡處,在孟地致津,謂之“孟津”。傳雲“地名”,謂“孟”為地名耳。杜預云:“孟津,河內河陽縣南孟津也。在洛陽城北,都道所湊,古今常以為津。武王渡之,近世以來呼為武濟。” \par}

東過洛汭,至於大伾;\footnote{洛汭,洛入河處。山再成曰伾。至於大伾而北行。伾,本或作岯,音丕,又皮鄙反;徐扶眉反,又敷眉反;韋音嚭;郭撫梅反,字或作 。}

{\noindent\zhuan\zihao{6}\fzbyks 傳“洛汭”至“北行”。正義曰:“洛汭,洛入河處”,河南鞏縣東也。\CJKunderwave{釋山}云:“再成英,一成岯。”李巡曰:“山再重曰英,一重曰岯。”傳雲“再成曰岯”,與\CJKunderwave{爾雅}不同,蓋所見異也。\CJKunderline{鄭玄}云:“大岯在修武武德之界。”張揖云:“成皋縣山也。”\CJKunderwave{漢書音義}有臣瓚者,以為:“修武武德無此山也。成皋縣山,又不一成,今黎陽縣山臨河。豈不是大岯乎?”瓚言當然。 \par}

北過降水,至於大陸;\footnote{降水,水名,入河。大陸,澤名。降如字,鄭戶江反。}

{\noindent\zhuan\zihao{6}\fzbyks 傳“降水”至“澤名”。正義曰:\CJKunderwave{地理志}云,降水在信都縣。案班固\CJKunderwave{漢書}以襄國為信都,在大陸之內。或降水發源在此,下尾至今之信都,故得先過降水,乃至大陸。若其不爾,則降水不可知也。鄭以“降讀為降,下江反,聲轉為共,河內共縣,淇水出焉,東至魏郡黎陽縣入河。北近降水也,周時國於此地者惡言降水,改謂之共”。此鄭胸臆,不可從也。\par}

又北播為九河,\footnote{北分為九河,以殺其溢,在兗州界。殺,所界反。溢,字又作隘,於賣反。}同為逆河,入於海。\footnote{同合為一大河,名逆河,而入於渤海。皆禹所加功,故敘之。渤,蒲兀反。}

{\noindent\zhuan\zihao{6}\fzbyks 傳“同合”至“敘之”。正義曰:傳言九河將欲至海,更同合為一大河,名為逆河,而入於渤海也。\CJKunderline{鄭玄}云:“下尾合,名為逆河,言相向迎受。”王肅云:“同逆一大河,納之於海。”其意與孔同。 \par}

嶓冢導漾,東流為漢;\footnote{泉始出山為漾水,東南流為沔水,至漢中東流為漢水。}

{\noindent\zhuan\zihao{6}\fzbyks 傳“泉始”至“漢水”。正義曰:傳之此言,當據時人之名為說也。\CJKunderwave{地理志}云,漾水出隴西氐道縣,至武都為漢水。不言中為沔水。孔知嶓冢之東、漢水之西而得為沔水者,以禹治梁州,入帝都白所治,雲“逾於沔,入於渭”,是沔近於渭,當梁州向冀州之路也。應劭云:“沔水自江別,至南郡華容縣為夏水,過江夏郡入江。”既雲“江別”,明與此沔別也。依\CJKunderwave{地理志},漢水之尾變為夏水,是應劭所云沔水下尾亦與漢合,乃入於江也。 \par}

又東為滄浪之水;\footnote{別流在荊州。浪音郎。}

{\noindent\zhuan\zihao{6}\fzbyks 傳“別流在荊州”。正義曰:傳言“別流”,似分為異水。案經首尾相連,不是分別,當以名稱別流也。又上在梁州,故此雲“在荊州”。 \par}

過三澨,至於大別,\footnote{三澨,水名,入漢。大別,山名。澨,市制反。}南入於江。\footnote{觸山回南,入江。觸,\CJKunderwave{切韻}尺玉反。}東匯澤為彭蠡,\footnote{匯,回也。水東回為彭蠡大澤。匯,徐胡罪反,韋空為反。}東為北江,入於海。\footnote{自彭蠡江分為三,入震澤,遂為北江而入海。}

{\noindent\zhuan\zihao{6}\fzbyks 傳“自彭”至“入海”。正義曰:揚州云:“三江既入,震澤厎定”,孔為“三江既入”,入震澤也,故言江自彭蠡分而為三江,復共入震澤。出澤又分為三,此水遂為北江而入於海。\CJKunderline{鄭玄}以為:“‘三江既入’,入於海,不入震澤也。”孔必知入震澤者,以震澤屬揚州,彭蠡在揚州之西界,今從彭蠡有三江,則震澤之西三江具矣。今雲“三江既入”,繼以“震澤厎定”,故知三江入震澤矣。今南人以大江不入震澤,震澤之東別有松江等三江。案\CJKunderwave{職方}揚州“其川曰三江”,宜舉州內大川,其松江等雖出震澤,入海既近,\CJKunderwave{周禮}不應舍岷山大江之名,而記松江等小江之說。山水古今變易,故鄭雲既知今,亦當知古,是古今同之驗也。 \par}

岷山導江,東別為沱;\footnote{江東南流,沱東行。沱,唐河反。}

{\noindent\zhuan\zihao{6}\fzbyks 傳“江東”至“東行”。正義曰:以上雲“浮於江、沱、潛、漢”,其次自南而北,江在沱南,知“江東南流,而沱東行”。 \par}

又東至於澧,\footnote{澧,水名。澧音禮。}

{\noindent\zhuan\zihao{6}\fzbyks 傳“澧,水名”。正義曰:\CJKunderline{鄭玄}以此經自“導弱水”已下,言“過”言“會”者,皆是水名,言“至於”者或山或澤,皆非水名,故以“合黎”為山名,“澧”為陵名。\CJKunderline{鄭玄}云:“今長沙郡有澧陵縣,其以陵名為縣乎?”孔以“合黎”與“澧”皆為水名,弱水餘波入於流沙,則本源入合黎矣。合黎得容弱水,知是水名。\CJKunderwave{楚辭}曰:“濯餘佩兮澧浦”是,“澧”亦為水名。 \par}

過九江,至於東陵;\footnote{江分為九道,在荊州。東陵,地名。}

{\noindent\zhuan\zihao{6}\fzbyks 傳“江分”至“地名”。正義曰:九江之水,禹前先有其處。禹今導江,過歷九江之處,非是別有九江之水。 \par}

東迆北會於匯;\footnote{迆,溢也。東溢分流,都共北會為彭蠡。迆,以爾反,馬雲靡也。}

{\noindent\zhuan\zihao{6}\fzbyks 傳“迆溢”至“彭蠡”。正義曰:“迆”言靡迆,邪出之言,故為溢也。東溢分流,又都共聚合,北會彭蠡,言散流而複合也。鄭雲“東迆者為南江”,孔意或然。“至”之與“會”,史異文耳。 \par}

東為中江,入於海。\footnote{有北,有中,南可知。}

{\noindent\zhuan\zihao{6}\fzbyks 傳“有北,有中,南可知”。正義曰:\CJKunderwave{地理志}云,南江從會稽吳縣南東入海,中江從丹陽無湖縣西東至會稽陽羨縣東入海,北江從會稽毗陵縣北入於海。 \par}

導沇水,東流為濟,\footnote{泉源為沇,流去為濟,在溫西北平地。沇音兗,又以轉反。}

{\noindent\zhuan\zihao{6}\fzbyks 傳“泉源”至“平地”。正義曰:\CJKunderwave{地理志}云,濟水出河東垣縣王屋山,東南至河內武德縣入河。傳言“在溫西北平地”者,濟水近在河內,孔必驗而知之。見今濟水所出,在溫之西北七十餘里,溫是古之舊縣,故計溫言之。 \par}

入於河,溢為滎;\footnote{濟水入河,並流十數里,而南截河。又並流數里,溢為滎澤,在敖倉東南。數,色主反,下同,一本作十所。}

{\noindent\zhuan\zihao{6}\fzbyks 傳“濟水”至“東南”。正義曰:此皆目驗為說也。濟水既入於河,與河相亂,而知截河過者,以河濁濟清,南出還清,故可知也。 \par}

東出於陶丘北,\footnote{陶丘,丘再成。陶音桃。}

{\noindent\zhuan\zihao{6}\fzbyks 傳“陶丘,丘再成”。正義曰:\CJKunderwave{釋丘}云:“再成為陶丘。”李巡曰:“再成,其形再重也。”郭璞云:“今濟陰定陶城中有陶丘。”\CJKunderwave{地理志}云,定陶縣西南有陶丘亭。 \par}

又東至於菏;\footnote{菏澤之水。}又東北,會於汶;\footnote{濟與汶合。}又北東入於海。\footnote{北折而東。折,之設反。}導淮自桐柏,\footnote{桐柏山,在南陽之東。}

{\noindent\zhuan\zihao{6}\fzbyks 傳“桐柏”至“之東”。正義曰:\CJKunderwave{地理志}云,桐柏山在南陽平氏縣東南,淮水所出。\CJKunderwave{水經}云:“出胎簪山,東北過桐柏山。”胎簪蓋桐柏之傍小山,傳言南陽郡之東也。 \par}

東會於泗、沂,東入於海。\footnote{與泗、沂二水合,入海。}

{\noindent\zhuan\zihao{6}\fzbyks 傳“與泗”至“入海”。正義曰:\CJKunderwave{地理志}云,沂水出泰山蓋縣,南至下邳,入泗。泗水出濟陰乘氏縣,至臨淮雎陵縣入淮。乃沂水先入泗,泗入淮耳。以沂水入泗處去淮已近,故連言之。 \par}

導渭自鳥鼠同穴,\footnote{鳥鼠共為雌雄,同穴處此山,遂名山曰鳥鼠,渭水出焉。}

{\noindent\zhuan\zihao{6}\fzbyks 傳“鳥鼠”至“出焉”。正義曰:\CJKunderwave{釋鳥}云:“鳥鼠同穴,其鳥為鵭,其鼠為鼵。”李巡曰:“鵭鼵鳥鼠之名,共處一穴,天性然也。”郭璞曰:“鼵如人家鼠而短尾,鵭似鵽而小,黃黑色。穴入地三四尺,鼠在內,鳥在外,今在隴西首陽縣有鳥鼠同穴山。\CJKunderwave{尚書}孔傳雲‘共為雄雌’,張氏\CJKunderwave{地理記}雲‘不為牝牡’。”璞並載此言,未知誰得實也。\CJKunderwave{地理志}云,隴西首陽西南有鳥鼠同穴山,渭水所出,至京兆北船司空縣入河,過郡四,行千八百七十里。 \par}

東會於灃,又東會於涇,\footnote{灃水自南,涇水自北而合。灃音豐。}又東過漆沮,入於河。\footnote{漆沮,一水名,亦曰洛水,出馮翊北。翊,與職反。}

{\noindent\zhuan\zihao{6}\fzbyks 傳“漆沮”至“翊北”。正義曰:\CJKunderwave{地理志}云,漆水出扶風漆縣。依\CJKunderwave{十三州記},漆水在岐山東入渭,則與漆沮不同矣。此雲“會於涇”,又“東過漆沮”,是漆沮在涇水之東,故孔以為洛水一名漆沮。\CJKunderwave{水經}沮水出北池直路縣,東入洛水。又云鄭渠在太上皇陵東南,濯水入焉,俗謂之漆水,又謂之漆沮,其水東流注於洛水。\CJKunderwave{志}云,出馮翊懷德縣,東南入渭。以水土驗之,與\CJKunderwave{毛詩}古公“自土沮漆”者別也。彼漆即扶風漆水也,彼沮則未聞。 \par}

導洛自熊耳,\footnote{在宜陽之西。}東北會於澗瀍,\footnote{會於河南城南。}又東會於伊,\footnote{合於洛陽之南。}又東北入於河。\footnote{合於鞏之東。鞏,恭勇反,縣名,屬河南郡。}九州攸同,\footnote{所同事在下。}四隩既宅,\footnote{四方之宅巳可居。隩,於六反,\CJKunderwave{玉篇}於報反。}九州刊旅,九川滌源,九澤既陂,\footnote{九州名山與槎木通道而旅祭矣,九州之川已滌除泉源無壅塞矣,九州之澤已陂障無決溢矣。滌,待歷反。陂,彼宜反。槎,仕雅反。障,章尚反。}四海會同,六府孔修。\footnote{四海之內會同京師,九州同風,萬國共貫,水、火、金、木、土、谷甚修治。言政化和。貫,工喚反。}庶土交正,厎慎財賦,\footnote{交,俱也。眾土俱得其正,謂壤、墳、壚。致所慎者,財貨貢賦。言取之有節,不過度。}咸則三壤,成賦中邦。\footnote{皆法壤田上中下大較三品,成九州之賦,明水害除。較音角。}

{\noindent\zhuan\zihao{6}\fzbyks 傳“所同事在下”。正義曰:九州所同,與下為目,故言“所同事在下”。“四隩既宅”已下皆是也,其言“九山”、“九川”、“九澤”,最是同之事矣。 \par}

{\noindent\zhuan\zihao{6}\fzbyks 傳“四方”至“可居”。正義曰:室隅為“隩”,“隩”是內也。人之造宅為居,至其隩內,遂以“隩”表宅,故傳以“隩”為宅,以宅內可居,言四方舊可居之處皆可居也。 \par}

{\noindent\zhuan\zihao{6}\fzbyks 傳“九州”至“溢矣”。正義曰:上文諸州有言山川澤者,皆舉大言之。所言不盡,故於此復更總之。“九山”、“九川”、“九澤”,言九州之內所有山川澤,無大無小,皆刊槎決除已訖,其皆旅祭。惟據名山大川言“旅”者,往前大水,旅祭禮廢,已旅見已治也。山非水體,故以“旅”見治。其實水亦旅矣,發首雲“奠高山大川”,但是定位,皆已旅祭也。川言“滌除泉源”,從其所出,至其所入,皆蕩除之,無壅塞也。澤言“既陂”,往前濫溢,今時水定,或作陂以障之,使無決溢。\CJKunderwave{詩}云:“彼澤之陂。”\CJKunderwave{毛傳}云:“陂,澤障也。” \par}

{\noindent\zhuan\zihao{6}\fzbyks 傳“四海”至“化和”。正義曰:\CJKunderwave{禮}諸侯之見天子,“時見曰會,殷見曰同”。此言“四海會同”,乃謂官之與民皆得聚會京師,非據諸侯之身朝天子也。夷狄戎蠻謂之四海,但天子之於夷狄,不與華夏同風,故知“四海”謂“四海之內”,即是九州之中,乃有萬國。萬國同其風化,若物在繩索之貫,故云“九州同風,萬國共貫”。\CJKunderwave{大禹謨}云,水、火、金、木、土、谷謂之六府。皆修治者,言政化和也。由政化和平,民不失業,各得殖其資產,故六府修治也。 \par}

{\noindent\zhuan\zihao{6}\fzbyks 傳“交俱”至“過度”。正義曰:交錯、更互,“俱”之義,故“交”為俱也。洪水之時,高下皆水,土失本性。今水災既除,“眾土俱得其正,謂壤、墳、壚”,還復其壤、墳、壚之性也。諸州之土,“青黎”是色,“塗泥”是溼,土性之異,惟有“壤、墳、壚”耳,故舉三者以言也。致所慎者,財貨貢賦,謹慎其事,不使害人,言取民有節,什一而稅,不過度也。 \par}

{\noindent\zhuan\zihao{6}\fzbyks 傳“皆法”至“害除”。正義曰:土壤各有肥瘠,貢賦從地而出,故分其土壤為上中下,計其肥瘠等級甚多,但舉其大較,定為三品,法則地之善惡,以為貢賦之差。雖細分三品,以為九等,人功修少,當時小異,要民之常稅必準其土,故皆法三壤成九州之賦。言得施賦法,以明水害除也。“九州”即是“中邦”,故傳以“九州”言之。 \par}

{\noindent\shu\zihao{5}\fzkt “九州”至“中邦”。正義曰:昔堯遭洪水,道路阻絕,今水土既治,天下大同,故總敘之,今九州所共同矣。所同者,四方之宅已儘可居矣,九州之山刊槎其木旅祭之矣,九州之川滌除泉源無壅塞矣,九州之澤已皆陂障無決溢矣,四海之內皆得會同京師無乖異矣,六材之府甚修治矣。言海內之人皆豐足矣。水災已除,天下眾土墳壤之屬俱得其正,複本性故也。民既豐足,取之有藝,致所重慎者惟財貨賦稅也。慎之者,皆法則其三品上壤,準其地之肥瘠,為上中下三等,以成其貢賦之法於中國。美禹能治水土,安海內,於此裛結之。 \par}

錫土姓,祗臺德先,不距朕行。\footnote{臺,我也。天子建德,因生以賜姓。謂有德之人生此地,以此地名賜之姓以顯之。王者常自以敬我德為先,則天下無距違我行者。}

{\noindent\zhuan\zihao{6}\fzbyks 傳“臺我”至“行者”。正義曰:“臺,我”,\CJKunderwave{釋詁}文。“天子建德,因生以賜姓”,隱八年\CJKunderwave{左傳}文。既引其文,又解其義:土,地也,謂有德之人生於此地,天子以地名賜之姓以尊顯之。\CJKunderwave{周語}稱帝嘉禹德,賜姓曰姒;祚四嶽,賜姓曰姜;\CJKunderwave{左傳}稱周賜陳胡公之姓為媯,皆是因生賜姓之事也。臣蒙賜姓,其人少矣,此事是用賢大者,故舉以為言。王者既能用賢,又能謹敬,其立意也常自以敬我德為先,則天下無有距違我天子之行者。\CJKunderwave{論語}云:“上好禮,則民莫敢不敬。上好義,則民莫敢不服。上好信,則民莫敢不用情。”王者自敬其德,則民豈敢不敬之?人皆敬之,誰敢距違者?聖人行而天下皆悅,動而天下皆應,用此道也。 \par}

{\noindent\shu\zihao{5}\fzkt “錫土”至“朕行”。正義曰:此一經皆史美禹功,言九州風俗既同,可以施其教化,天子惟當擇任其賢者,相與共治之。選有德之人,賜與所生之土為姓,既能尊賢如是,又天子立意,常自以敬我德為先,則天下之民無有距違我天子所行者。皆禹之使然,故敘而美之。 \par}

五百里甸服。\footnote{規方千里之內謂之甸服。為天子服治田,去王城面五百里。甸,田遍反。為,於偽反。}

{\noindent\zhuan\zihao{6}\fzbyks 傳“規方”至“百里”。正義曰:“先王規方千里,以為甸服”,\CJKunderwave{周語}文。\CJKunderwave{王制}亦云:“千里之內曰甸。”\CJKunderline{鄭玄}云:“服治田,出谷稅也。言甸者,主治田,故服名甸也。” \par}

{\noindent\shu\zihao{5}\fzkt “五百里甸服”。正義曰:既言九州同風,法壤成賦,而四海之內路有遠近,更敘弼成五服之事。甸、侯、綏、要、荒五服之名,堯之舊制。洪水既平之後,禹乃為之節文,使賦役有恆,職掌分定。甸服去京師最近,賦稅尤多,故每於百里即為一節。侯服稍遠,近者供役,故二百里內各為一節,三百里外共為一節。綏、要、荒三服,去京師益遠,每服分而為二,內三百里為一節,外二百里為一節。以遠近有較,故其任不等。甸服入谷,故發首言賦稅也。賦令自送入官,故三百里內每皆言“納”。四百里、五百里不言“納”者,從上省文也。於三百里言“服”者,舉中以明上下,皆是服王事也。侯服以外貢不入谷,侯主為斥候。二百里內徭役差多,故各為一名。三百里外同是斥候,故共為一名。自下皆先言三百里,而後二百里,舉大率為差等也。 \par}

百里賦納緫,\footnote{甸服內之百里近王城者。禾稿曰緫,入之供飼國馬。納如字,本又作內,音同,下如字。緫音揔。近,附近之近。稿,故老反。供音恭。飼音嗣。}

{\noindent\zhuan\zihao{6}\fzbyks 傳“甸服”至“國馬”。正義曰:去王城五百里總名甸服,就其甸服內又細分之。從內而出,此為其首,故云“甸服之內近王城者”,“緫”者,總下“銍”、“秸”,禾穗與稿,緫皆送之,故云“禾稿曰緫,入之供飼國馬”。\CJKunderwave{周禮}掌客待諸侯之禮有芻,有禾,此緫是也。 \par}

二百里納銍,\footnote{銍,刈,謂禾穗。銍,珍慄反。穗亦作穟,音遂。}

{\noindent\zhuan\zihao{6}\fzbyks 傳“銍,刈,謂禾穗”。正義曰:\CJKunderline{劉熙}\CJKunderwave{釋名}云:“銍,獲禾鐵也。”\CJKunderwave{說文}云:“銍,獲禾短鐮也。”\CJKunderwave{詩}雲“奄觀銍刈”,用銍刈者,謂禾穗也。禾穗用銍以刈,故以“銍”表禾穗也。 \par}

三百里納秸服,\footnote{秸,稿也,服瑽役。秸,本或作稭,工八反,馬云:“去其穎,音鞂。”}

{\noindent\zhuan\zihao{6}\fzbyks 傳“秸,稿也,服稿役”。正義曰:\CJKunderwave{郊特牲}云:“莞簟之安,而稿秸之設。”“秸”亦“稿”也,雙言之耳。去穗送稿,易於送穗,故為遠彌輕也。然計什一而得,稿粟皆送,則秸服重於納銍,則乖近重遠輕之義。蓋納粟之外,斟酌納稿。“服稿役”者,解經“服”字,於此言“服”,明上下服皆並有所納之役也。四百里猶尚納粟,此當稿、粟別納,非是徒納稿也。 \par}

四百里粟,五百里米。\footnote{所納精者少,粗者多。}

{\noindent\zhuan\zihao{6}\fzbyks 傳“所納”至“者多”。正義曰:直納粟米為少,禾稿俱送為多。其於稅也。皆當什一,但所納有精粗,遠輕而近重耳。 \par}

五百里侯服。\footnote{甸服外之五百里。侯,候也。斥候而服事。}

{\noindent\zhuan\zihao{6}\fzbyks 傳“甸服”至“服事”。正義曰:“侯”聲近候,故為候也。襄十八年\CJKunderwave{左傳}稱晉人伐齊,使司馬斥山澤之險。“斥”謂檢行之也。“斥候”謂檢行險阻,伺候盜賊。此五百里主為斥候而服事天子,故名“侯服”。因見諸言“服”者,皆是服事也。 \par}

百里採,\footnote{侯服內之百里,供王事而已,不主一。}

{\noindent\zhuan\zihao{6}\fzbyks 傳“侯服”至“主一”。正義曰:“採”訓為事,此百里之內主供王事而已。“事”謂役也,有役則供,不主於一,故但言“採”。 \par}

二百里男邦,\footnote{男,任也,任王者事。任,而針反,又而鴆反,下同。}

{\noindent\zhuan\zihao{6}\fzbyks 傳“男,任也,任王者事”。正義曰:“男”聲近任,故訓為任。“任王者事”,任受其役,此任有常,殊於“不主一”也。言“邦”者,見上下皆是諸侯之國也。 \par}

三百里諸侯。\footnote{三百里同為王者斥候,故合三為一名。為,於偽反。}

{\noindent\zhuan\zihao{6}\fzbyks 傳“三百”至“一名”。正義曰:經言“諸侯”者,三百里內同為王者斥候,在此內所主事同,故合三百、四百、五百共為一名,言“諸侯”以示義耳。 \par}

五百里綏服。\footnote{綏,安也。侯服外之五百里,安服王者之政教。綏,息遺反。}

{\noindent\zhuan\zihao{6}\fzbyks 傳“綏安”至“政教”。正義曰:“綏,安”,\CJKunderwave{釋詁}文。要服去京師已遠,王者以文教要束使服。此綏服路近,言“王者政教”,以示不待要束言安服自服也。\CJKunderwave{周語}云:“先王之制,邦內甸服,邦外侯服,侯衛賓服,夷蠻要服,戎狄荒服。”彼“賓服”當此“綏服”。韋昭云:“以文武侯衛為安,王賓之,因以名服。”然則“綏”者據諸侯安王為名,“賓”者據王敬諸侯為名,故云“先王之制”,則此服舊有二名。 \par}

三百里揆文教,\footnote{揆,度也。度王者文教而行之,三百里皆同。揆,葵癸反。度,待洛反。}

{\noindent\zhuan\zihao{6}\fzbyks 傳“揆度”至“皆同”。正義曰:\CJKunderwave{釋詁}訓“揆”為度,故雙言之。以王者有文教,此服諸侯揆度王者政教而行之,必自揆度,恐其不合上耳。即是安服王者之義。 \par}

二百里奮武衛。\footnote{文教外之二百里奮武衛,天子所以安。奮,方問反。}

{\noindent\zhuan\zihao{6}\fzbyks 傳“文教”至“以安”。正義曰:既言“三百”,又言“二百”,嫌是“三百”之內,以下二服文與此同,故於此解之,此是“文教外之二百里”也。由其心安王化,奮武以衛天子,所以名此服為安也。內文而外武,故先“揆文教”,後言“奮武衛”,所從言之異,與安之義同。奮武衛天子,是其安之驗也。言服內諸侯,心安天子,非言天子賴諸侯以安也。 \par}

五百里要服。\footnote{綏服外之五百里,要束以文教。要,一遙反。束如字,一音來。}

{\noindent\zhuan\zihao{6}\fzbyks 傳“綏服”至“文教”。正義曰:“要”者約束之義。上言“揆文教”,知“要”者,“要束以文教”也。綏服自揆天子文教,恐其不稱上旨。此要服差遠,已慢王化,天子恐其不服,乃以文教要服之。名為“要”,見其疏遠之義也。 \par}

三百里夷,\footnote{守平常之教,事王者而已。馬云:“夷,易也。”}二百里蔡。\footnote{蔡,法也。法三百里而差簡。差,初佳反,又初賣反。}

{\noindent\zhuan\zihao{6}\fzbyks 傳“蔡法”至“差簡”。正義曰:“蔡”之為法,無正訓也。上言“三百里夷”,“夷”訓平也,言守平常教耳。此名為“蔡”,義簡於“夷”,故訓“蔡”為法。法則三百里者,去京師彌遠,差復簡易,言其不能守平常也。 \par}

五百里荒服。\footnote{要服外之五百里。言荒又簡略。}

{\noindent\zhuan\zihao{6}\fzbyks 傳“要服”至“簡略”。正義曰:服名“荒”者,王肅云:“政教荒忽,因其故俗而治之。”傳言“荒又簡略”,亦當以為荒忽,又簡略於要服之蔡也。 \par}

三百里蠻,\footnote{以文德蠻來之,不制以法。}

{\noindent\zhuan\zihao{6}\fzbyks 傳“以文”至“以法”。正義曰:鄭云:“蠻者聽從其俗,羈縻其人耳。故云蠻,蠻之言緡也。”其意言“蠻”是緡也,緡是繩也,言“蠻”者以繩束物之名。揆度文教,\CJKunderwave{論語}稱“遠人不服,則修文德以來之”,故傳言“以文德蠻來之”,不制以國內之法強逼之。王肅云:“蠻,慢也,禮儀簡慢。”與孔異。然甸、侯、綏、要四服,俱有三日之役,什一而稅,但二百里蔡者,稅微差簡,其荒服力役田稅並無,故鄭注云:“蔡之言殺,減殺其賦。”荒服既不役作其人,又不賦其田事也。其侯綏等所出稅賦,各入本國,則亦有納緫、納銍之差,但此據天子立文耳。要服之內,皆有文教,故孔於要服傳雲“要束以文教”,則知已上皆有文教可知。獨於綏服三百里雲“揆文教”者,以去京師既遠,更無別供,又不近外邊,不為武衛。其要服又要束始行文教,無事而能揆度文教而行者,惟有此三百里耳。“奮武衛”者,在國習學兵武,有事則征討夷狄。不於要服內奮武衛者,以要服逼近夷狄,要束始來,不可委以兵武。 \par}

二百里流。\footnote{流,移也。言政教隨其俗。凡五服相距為方五千裡。}

{\noindent\zhuan\zihao{6}\fzbyks 傳“流移”至“千里”。正義曰:流,如水流,故云“移也”。其俗流移無常,故政教隨其俗,任其去來,不服蠻來之也。凡五服之別,各五百里,是王城四面,面別二千五百里,四面相距為方五千裡也。賈逵、馬融以為“甸服之外百里至“五百里米特有此數,去王城千里;其侯、綏、要、荒服各五百里,是面三千里,相距為方六千里”。\CJKunderline{鄭玄}以為“五服服別五百里,是堯之舊制。及禹弼之,每服之間更增五百里,面別至於五千裡,相距為方萬里”。司馬遷與孔意同,王肅亦以為然,故肅注此云:“賈、馬既失其實,\CJKunderline{鄭玄}尤不然矣。禹之功在平治山川,不在拓境廣土。土地之廣三倍於堯,而書傳無稱也,則\CJKunderline{鄭玄}創造,難可據信。漢之孝武,疲弊中國,甘心夷狄,天下戶口至減太半,然後僅開緣邊之郡而已。禹方憂洪水,三過其門不入,未暇以征伐為事,且其所以為服之名,輕重顛倒,遠近失所,難得而通矣。先王規方千里,以為甸服,其餘均分之公、侯、伯、子、男,使各有寰宇,而使甸服之外諸侯入禾稿,非其義也。”史遷之旨蓋得之矣,是同於孔也。若然,\CJKunderwave{周禮}王畿之外別有九服,服別五百里,是為方萬里,復以何故三倍於堯?又\CJKunderwave{地理志}言漢之土境東西九千三百二里,南北萬三千三百六十八里。驗其所言山川,不出\CJKunderwave{禹貢}之域,山川戴地,古今必同,而得裡數異者,堯與周漢其地一也,\CJKunderwave{尚書}所言,據其虛空鳥路方直而計之,\CJKunderwave{漢書}所言,乃謂著地人跡屈曲而量之,所以數不同也。故王肅上篇注云:“方五千裡者,直方之數,若其回邪委曲,動有倍加之較。”是言經指直方之數,漢據回邪之道。有九服、五服,其地雖同,王者革易,自相變改其法,不改其地也。\CJKunderline{鄭玄}不言禹變堯法,乃雲地倍於堯,故王肅所以難之。\CJKunderwave{王制}云:“西不盡流沙,東不盡東海,南不盡衡山,北不盡恆山。”凡四海之內,斷長補短,方三千里者,彼自言“不盡”,明未至遠界,且\CJKunderwave{王制}漢世為之,不可與經合也。 \par}

東漸於海,西被於流沙,朔南暨聲教,\footnote{漸,入也。被,及也。此言五服之外皆與王者聲教而朝見。漸,子廉反。被,皮寄反。朔,朔北也。與音預。朝,直遙反。見,賢遍反。}訖於四海。禹錫玄圭,告厥成功。\footnote{玄,天色。禹功盡加於四海,故堯賜玄圭以彰顯之。言天功成。訖,斤密反。}

{\noindent\zhuan\zihao{6}\fzbyks 傳“漸入”至“朝見”。正義曰:“漸”是沾溼,故為入,謂入海也。覆被是遠及之辭,故為及也。海多邪曲,故言“漸入”;流沙長遠,故言“被及”,皆是過之急也。五服之下乃說此事,故言“此五服之外皆與王者聲教而朝見”,言其聞風感德而來朝也。\CJKunderline{鄭玄}云:“南北不言所至,容逾之。”此言“西被於流沙”,流沙當是西境最遠者也。而\CJKunderwave{地理志}以流沙為張掖居延澤是也,計三危在居延之西,太遠矣,\CJKunderwave{志}言非也。 \par}

{\noindent\zhuan\zihao{6}\fzbyks 傳“玄天”至“功成”。正義曰:\CJKunderwave{考工記}“天謂之玄”,是“玄”為天色。禹之蒙賜,必是堯賜,故史敘其事,“禹功盡加於四海,故堯賜玄圭以玄顯之”。必以天色圭者,“言天功成”也。\CJKunderwave{大禹謨}舜美禹功雲“地平天成”,是天功成也。 \par}

{\noindent\shu\zihao{5}\fzkt “東漸”至“成功”。正義曰:言五服之外,又東漸入於海,西被及於流沙,其北與南雖在服外,皆與聞天子威聲文教,時來朝見,是禹治水之功盡加於四海。以禹功如是,故帝賜以玄色之圭,告其能成天之功也。 \par}

%%% Local Variables:
%%% mode: latex
%%% TeX-engine: xetex
%%% TeX-master: "../Main"
%%% End:
